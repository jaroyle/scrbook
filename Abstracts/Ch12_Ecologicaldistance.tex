Ecological Distance

In this chapter we present an alternative to using
Euclidean distance to measure the distance from traps to
individual activity centers.
As a practical matter, models based on Euclidean
distance imply circular, symmetric, and stationary home ranges of
individuals, and these are not often biologically realistic.  While
these simple encounter probability models will often be sufficient for
practical purposes, especially in small data sets, sometimes
developing more complex models of the detection process as it relates
to space usage of individuals will be useful. After discussing the shortcomings
of Euclidean distance models, we present an alternative approach based
on the least-cost path between an individual's activity center and a
trap location. Examples of how to calculate cost-weighted distances is provided
as well as how to simulate SCR data using ``ecological distance".  In this chapter,
we also provide a likelihood analysis based on the ecological distance models.  
Being able to conduct such analyses requires knowledge of how to do basic 
geographic analysis in R, which we show specifically how to do.  This 
functionality is also useful for calculating distances in an irregular patch, such as
would be applicable to corridor and reserve design analyses.
Parameterizing SCR models in terms of one or more
parameters that relate the {\it resistance} of the landscape to
explicit covariates allows us to explicitly accommodate
landscape structure and account for connectivity of the landscape. 

Key Words:  Euclidean distance, least-cost path, cost-weighted paths, landscape connectivity, irregular patches
geographical analysis