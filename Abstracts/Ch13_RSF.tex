Resource Selection and Space Usage

 In this chapter, we extend SCR models to incorporate models of
resource selection, such as when one or more explicit landscape
covariates are available which the investigator believes might affect
how individual animals use space within their home range.  We present
a method that integrates a standard family of resource selection
models based on auxiliary telemetry data into the capture-recapture
model for encounter probability.  The important distinction between
SCR and resource selection function (RSF) studies is that, in SCR
studies, encounter of individuals is imperfect (i.e., ``$p<1$'')
whereas, with RSF data obtained by telemetry, encounter is perfect.
Thus in this chapter, we argue that SCR and telemetry data can
therefore be combined in the same likelihood by formally recognizing
this distinction in the model.  We demonstrate a model for integrating
capture-recapture data with RSF data under a Poisson model of space
usage.  We present and analyze data from a study of black bears in
southwest NY, USA to which we fit a variety of models that highlight
the differences between models including only SCR data or models that
include both SCR and RSF data.  We also provide a simulation study to
validate the model and for the reader to become more familar with the
models and data.  This integration of SCR and RSF models wil allow
researchers to model how the landscape and habitat influence the
movement and space use of individuals around their home range, using
non-invasively collected capture-recapture data that can be augmented
with telemetry data.

Key Words:  
black bears, 
habitat selection,
landscape structure, 
multiple data sources,
resource selection functions,
RSF,
telemetry,
utilization distribution

