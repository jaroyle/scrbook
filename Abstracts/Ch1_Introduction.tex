
Introduction


Capture-recapture methods represent perhaps the most common technique
for studying animal populations, and their use is growing in
popularity due to recent technological advances that provide mechanisms
to study many taxa which before could not be studied efficiently, if
at all. These advance include,
but are not limited to, camera trapping, DNA sampling, acoustic sampling, and 
search-encounter methods.   In this chapter, we review the basics of traditional
capture recapture modeling and present a case study on black bear data collected at Fort Drum, NY.
These data are analyzed using traditional capture-recapture methods to demonstrate the modeling
approach and to highlight the issues related to calculating the effective trapping area.  In traditional
capture-recapture models, abundance ($N$) is estimated; however, researchers are often interested in
density and thus an area must be associated with the estimate of abudance.  Animals move in space and time, and thus their exposure to trapping is variable, rending the effective trapping area a complicated issue when space is not incorporated directly into the model.  We discuss previous work to deal with this issue including buffering the trapping array, using an approach related to modeling temporary emigration, and considering the average location as an individual covariate.  However, none of the approaches was fully satistifying and this led to the development of spatial capture recapture (SCR) models, which incorporate space explicitly and allow individuals to have variable exposure to trapping based generally on where they live and how much they move.   In this chapter, we also introduce the fundamental concepts behind SCR models including the formulation of animal activity centers as a point process, the state-space, and estimation of abundance and density.

 Key words:  ordinary capture recapture, buffering, temporary emigration, activity centers, sampling methods





