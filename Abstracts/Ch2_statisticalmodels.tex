Statistical Models and SCR

In this chapter, we present a brief overview of the basic statistical
principals that are referenced throughout the remainder of this
book. Emphasis is placed on the definition of a random variable, the
common probability distributions used to model random variables, and
how hierarchical models can be used to describe conditionally related
random variables.  We discuss what a probability distribution is and
describe probability density functions as well as other properties of
probability distributions.  For a set of common probability
distributions, we describe the distribution, its associated
properties, and provide examples of when the distribution could be
used.  The common probability distributions described are: binomial,
Bernoulli, Poisson, multinomial, uniform, and normal (Gaussian).  We
focus in this chapter on frequentist inference for parameter
estimation, describing how to carry out maximum likelihood for a set
of examples.  Moving to more complex examples of more than one random
variable, we also provide background material on the joint, marginal,
and conditional distributions.  Then, after covering some basic
concepts of hierarchical modeling, the chapter ends by describing
spatial capture-recapture models using hierarchical modeling
notation. This makes the concepts outlined in the previous chapter
more precise, and it highlights the fact that SCR models include
explicit models for the ecological processes of interest (e.g. spatial
variation in density) and the observation process, which describes how
individuals are encountered.

Key words: statistical inference, probability distributions, probability density functions, 
hierarchical modeling, notation 