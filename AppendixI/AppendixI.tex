
\chapter*{Appendix I - Useful software and R packages}
\markboth{Appendix I}{}
\label{chapt.app1}

\vspace{.3in}

Throughout this book we have used a suite of software and R packages, all of which are freely available online. To make life a little easier for you, here we provide you with a list of all software and R packages, download links and some (hopefully) helpful tips regarding their installation.  


\section{WinBUGS}
Although {\bf WinBUGS} \citep{gilks_etal:1994} is becoming increasingly obsolete with the faster and more flexible {\bf OpenBUGS} and {\bf JAGS}, there are still situations in which the program comes in handy.  
The .exe file can be downloaded from url{\http://www.mrc-bsu.cam.ac.uk/bugs/winbugs/contents.shtml}. On 32 bit machines you can just go ahead and double-click on the .exe file and follow the installation instructions on the screen.
On 64 bit machines, according to the BUGS project you should download a zip file (from the same page) and unzip it into a folder of your choice.
There are a couple of additional steps to make BUGS run. 
First, you need to obtain a key (which is free and valid for life) here: \url{http://www.mrc-bsu.cam.ac.uk/bugs/winbugs/WinBUGS14_immortality_key.txt}. The key comes with instructions on how to activate it.
Second, you need to update the basic {\bf WinBUGS} version to the most current one (which is from August 2007) following the instructions given here: \url{http://www.mrc-bsu.cam.ac.uk/bugs/winbugs/WinBUGS14_cumulative_patch_No3_06_08_07_RELEASE.txt}.
{\bf WinBUGS} is ready to use after quitting and re-opening it.
Remember that {\bf WinBUGS} only runs on Windows machines. Also, there appears to be a problem installing the program in Vista, although we have no personal experience with this.

\subsection{WinBUGS through R}
While you can run{\bf WinBUGS} as a standalone application, we recommend you access it from within {\bf R} using the package {\tt R2WinBUGS} \citep{sturtz_etal:2005}, so you can conveniently process your output, make graphs etc.   {\tt R2WinBUGS} also allows you to run models in {\bf OpenBUGS} (see below). You can install the package from within {\bf R} directly from a cran mirror. In addition to the usual package help document (http://cran.r-project.org/web/packages/R2WinBUGS/R2WinBUGS.pdf) you can also download a short manual with some examples (\url{http://voteview.com/bayes_beach/R2WinBUGS.pdf}). 



\section{OpenBUGS}
{\bf OpenBUGS} is the up-to-date version of {\bf WinBUGS} and can be downloaded here: \url{http://www.openbugs.info/w/Downloads} (Windows, Mac and Linux versions are available).  The
name {\bf OpenBUGS} refers to the software being open source, so users 
do
not need to download a license key, like they have to for {\bf WinBUGS}
(although the license key for {\bf WinBUGS} is free and valid for life). For Windows, install by double-clicking on the .exe file and following the instructions on the installer screen. Compared to {\bf WinBUGS}, {\bf OpenBUGS} 
has  more built-in functions. The
method of how to determine the right updater for each model parameter
has changed and the user can manually control the MCMC algorithm used
to update model parameters.  Several other changes have been
implemented in {\bf OpenBUGS} and a detailed list of differences between the
two {\bf BUGS} versions, can be found at
\url{http://www.openbugs.info/w/OpenVsWin}. We have encountered convergence problems with simple scr models in this program. 
There is an extensive help archive for both {\bf WinBUGS} and {\bf OpenBUGS}
 and you can subscribe to a mailing list, where people pose and answer 
 questions of how to use these programs at 
 \url{http://www.mrc-bsu.cam.ac.uk/bugs/overview/list.shtml}

\subsection{OpenBUGS through R}
Like {\bf WinBUGS}, {\bf OpenBUGS} can be used as a standalone application or through {\bf R}. There are several packages that allow {\R} to interface with {\bf OpenBUGS}, all of which can be installed directly from a cran mirror:

{\flushleft \tt R2WinBUGS: }
One of the options in the {\tt bugs()} call is {\tt program}, which lets you specify either {\bf WinBUGS} or {\bf OpenBUGS}. This is a convenient option because after having worked through some of this book you will likely be familiar with the format of {\tt bugs()} output and other functions of the {\tt R2WinBUGS} package.

{\flushleft \tt R2OpenBUGS: }
{\tt R2OpenBUGS} \citep{sturtz_etal:2005} is very similar to, and actually based on, {\tt R2WinBUGS} and it is unclear to us what can be gained by using the former over the latter. Arguments of the {\tt bugs()} call differ slightly between the two packages and given that {\tt R2WinBUGS} allows for the use of both {\bf OpenBUGS} and {\bf WinBUGS} it is probably easiest to stick with it. 

{\flushleft \tt BRugs: }
{\tt BRugs} \citep{thomas_etal:2006} can be installed from within {\bf R} directly from a cran mirror. In addition to the help document at \url{http://www.biostat.umn.edu/~brad/software/BRugs/BRugs_9_21_07.pdf}  there is a {\bf WinBUGS} style manual you can access at \url{http://www.rni.helsinki.fi/openbugs/OpenBUGS/Docu/BRugs%20Manual.html}.
{\tt BRugs} has the convenient feature that all pieces of a {\bf BUGS} analysis can be run from within {\bf R}, including checking the model syntax, something that requires opening the {\bf BUGS} GUI with other packages. 



\section{JAGS}
{\bf JAGS} (Just Another Gibbs Sampler) \citep{plummer:2003} runs scr models considerably faster than {\bf WinBUGS}, does not have the convergence problem with simple scr models we have encountered in {\bf OpenBUGS} but similar to the latter program, is flexible and constantly updated. Writing a \mbox{\bf JAGS} model is virtually identical to writing a {\bf WinBUGS}
 model. However, some functions may have slightly different names and you 
 can look up available functions and their use in the {\bf JAGS} 
 manual. One potential downside is that {\bf JAGS} can be very particular 
 when it comes to initial values. These may have to be set as close to 
 truth as possible for the model to start. Although {\bf JAGS} lets 
 you run several parallel Markov chains, this characteristic interferes 
 with the idea of using overdispersed initial values for the different 
 chains. Also, we have found that when running models, sometimes {\bf JAGS} crashes for unclear reasons, taking {\bf R} down with it. Oftentimes, in order to make it run again you'll have to go through downloading and installing it again (remove the non-functioning version first).

{\bf JAGS} has a variety of functions that are 
 not available in {\bf WinBUGS}. For example, {\bf JAGS} allows you to 
 supply observed data for some deterministic functions of unobserved 
 variables. In {\bf BUGS} we cannot supply data to logical nodes. 
 Another useful feature is that the adaptive phase of the model 
 (the burn-in) is run separately from the sampling from the stationary 
 Markov chains. This allows you to easily add more iterations to the 
 adaptive phase if necessary without the need to start from 0. There 
 are other, more subtle differences and there is an entire manual section 
 on differences between {\bf JAGS} and {\bf OpenBUGS}.

{\bf JAGS} is available for download at \url{http://sourceforge.net/projects/mcmc-jags/files/}, together with the R package \mbox{\tt rjags} \citep{plummer:2011}, which allows running {\bf JAGS} through {\bf R}, user and installation manuals and examples. At this site {\bf JAGS} is available for Windows and Mac; Linux binaries are distributed separately and you can find links to various sources here: \url{http://mcmc-jags.sourceforge.net/}. {\bf JAGS} comes with a 32 bit and a 64 bit version and can be installed by double-clicking on the .exe file and following the instructions on the installer screen. For questions and problems concerning {\bf JAGS} there is a forum online at 
\url{http://sourceforge.net/projects/mcmc-jags/forums/forum/610037}.

\subsection{JAGS through R}
Unlike the two {\bf BUGS} programs, {\bf JAGS} does not have a GUI interface but a command line interface that can be used to run the program as a standalone application. {\bf JAGS} will solely perform the MCMC simulation; analyzing and summarizing the output has to be done outside of {\bf JAGS}. To run {\bf JAGS} through {\bf R} you have two options.
 
{\flushleft \tt rjags}:
 As mentioned above, \mbox{\tt rjags} \citep{plummer:2011} can be found together with {\bf JAGS} and was developed/is being maintained by the inventor of {\bf JAGS}, which means it is guaranteed to stay up to date when/as {\bf JAGS} changes. The package can be installed from a cran mirror and the help document can be accessed at \url{http://cran.r-project.org/web/packages/rjags/rjags.pdf}
 
{\flushleft \tt R2jags}:
 Alternatively, the package \mbox{\tt R2jags} \citep{su_yajima:2011} provides a means of accessing {\bf JAGS} through {\bf R}. We prefer \mbox{\tt rjags} for the reason named above, as well as because it stores data in a more memory-efficient way and has better \mbox{\tt plot()} and \mbox{\tt summary()} methods. 



\section{R}
At the time of the preparation of this list, {\bf R} for Windows is at version 2.15.0, which can be downloaded at url{http://cran.r-project.org/bin/windows/base/}. 
This site also contains helpful tips on how to install {\bf R} in Windows Vista, how to update {\bf R} packages etc. 
Installation of {\bf R} in Windows is straightforward: download the .exe file, double-click on it and follow the instructions of the Windows installer. The later versions of {\bf R} come with versions for both 64 bit and 32 bit machines. 
The {\bf R} site (\url{http://mirrors.softliste.de/cran/}) has an extensive FAQ section \citet{hornik:2011}, which includes instructions on how to install R on Unix and Mac computers.  

\subsection{R packages}
This section provides an alphabetical list of useful {\bf R}  packages. There is a large number of {\bf R} packages and by no means is this list intended to be complete in terms of what is useful. Rather, we list packages that we are familiar with and that we employ at one point or the other in this book. Unless explicitly stated otherwise, all packages can be installed directly from within {\bf R} trough a cran mirror. 

 {\flushleft \tt adapt}:
\mbox{\tt adapt} \citep{genz_etal:2007} is a package for multidimensional numerical integration.
The package has been removed from the CRAN repository but can be obtained from \url{http://cran.r-project.org/src/contrib/Archive/adapt/}.

 {\flushleft \tt coda}: 
\mbox{\tt coda} \citep{plummer_etal:2006} lets you summarize and perform diagnostics on mcmc output. For a list and description of functions, see the manual at \url{http://cran.r-project.org/web/packages/coda/coda.pdf}. 

 {\flushleft \tt gdistance}:
\mbox{\tt gdistance} \citep{vanetten:2011} is a package for calculating distances and routes on geographical grids and can be used to calculate least cost path surfaces. Manual at \url{http://cran.r-project.org/web/packages/gdistance/gdistance.pdf}.
 
 {\flushleft \tt igraph}:
  \mbox{\tt igraph} \citep{csardi:2010} provides routines for graphs and network analysis. Manual at \url{http://cran.r-project.org/web/packages/igraph/igraph.pdf}. 

 {\flushleft \tt inline}:
  \mbox{\tt inline} \citep{sklyar_etal:2010} allows the user to define R functions with in-lined {\bf C}, {\bf C++} or {\bf Fortran} code. Manual at \url{http://cran.r-project.org/web/packages/inline/inline.pdf}. 
  
 {\flushleft \tt maps}: 
\mbox{\tt maps} \citep{becker_etal:2012} is a library for the display of maps. Manual at \url{http://cran.r-project.org/web/packages/maps/index.html}. 

 {\flushleft \tt maptools}:
\mbox{\tt maptools} \citep{bivand_levin-koh:2013} provides a set of tools for reading and manipulating spatial data, especially ESRI shapefiles. Manual at \url{http://cran.r-project.org/web/packages/maptools/maptools.pdf}. 

 {\flushleft \tt mvtnorm}:
\mbox{\tt mvtnorm} \citep{genz_etal:2012} computes multivariate normal and t probabilities, quantiles, random deviates and densities. Manual at \url{http://cran.r-project.org/web/packages/mvtnorm/mvtnorm.pdf}. 

 {\flushleft \tt parallel}:
\mbox{\tt parallel} contains a suite of functions for parallel computing on multiple computer cores and comes with {\bf R} versions 2.14.0 or higher. More information about the package can be found at \url{http://stat.ethz.ch/R-manual/R-devel/library/parallel/doc/parallel.pdf}. 

 {\flushleft \tt R2cuba}: 
\mbox{\tt R2cuba} \citep{hahn_etal:2011} is another package for multidimensional integration. Manual at \url{http://cran.r-project.org/web/packages/R2Cuba/R2Cuba.pdf}. 

 {\flushleft \tt raster}: 
\mbox{\tt raster} \citep{hijmans_vanetten:2012} provides functions for geographic analysis and modeling with raster data. Manual at \url{http://cran.r-project.org/web/packages/raster/raster.pdf}. 

 {\flushleft \tt Rcpp}: 
\mbox{\tt Rcpp} \citep{eddelbuettel_francois:2011} provides R functions as well as a {\bf C++} library which facilitate the integration of {\bf R} and {\bf C++}. Manual at \url{http://cran.r-project.org/web/packages/Rcpp/Rcpp.pdf}. 

 {\flushleft \tt RcppArmadillo}:
\mbox{\tt RcppArmadillo} \citep{francois_etal:2011} is a templated {\bf C++} linear algebra library, integrating the {\bf Armadillo} library and {\bf R}. Manual at \url{http://cran.r-project.org/web/packages/RcppArmadillo/RcppArmadillo.pdf}. 

 {\flushleft \tt reshape}:
\mbox{\tt reshape} \citep{wickham_hadley:2007} allows you to easily manipulate, summarize and reshape data. Manual at \url{http://cran.r-project.org/web/packages/reshape/reshape.pdf}. 
  
{\flushleft \tt rgeos}: 
\mbox{\tt rgeos} \citep{bivand_rundel:2011} provides many useful functions for spatial operations such as intersecting or buffering spatial features. Manual at \url{http://cran.r-project.org/web/packages/rgeos/rgeos.pdf}. 

 {\flushleft \tt SCRbayes}:
\citep{russell_etal:2012}. XXXXXXX Manual at \url{XXXXX}.

{\flushleft \tt secr}:
\mbox{\tt secr} \citep{efford_etal:2009euring} is an allround package for fitting a wide array of SCR models in a frequentist framework. Manual at \url{http://cran.r-project.org/web/packages/secr/secr.pdf}.

{\flushleft \tt shapefiles}: 
\mbox{\tt shapefiles} \citep{stabler:2006} allows you to read and write ESRI shapefiles (i.e. shapefiles you would use in {\bf ArcGIS}). Manual at \url{http://cran.r-project.org/web/packages/shapefiles/shapefiles.pdf}. 

 {\flushleft {\tt snow}, {\tt snowfall}: }
\mbox{\tt snow} \citep{tierney_etal:2011} and \mbox{\tt snowfall} \citep{knaus:2010} provide functionality for parallel computing. The latter is a more user-friendly wrapper around the former. Manuals at \url{http://cran.r-project.org/web/packages/snowfall/snowfall.pdf} and \url{http://cran.r-project.org/web/packages/snow/snow.pdf}. 

{\flushleft \tt sp}:
\mbox{\tt sp}  \citep{pebesma_bivand:2011} is a package for plotting, selecting, subsetting etc. spatial data. \mbox{\tt sp}  and \mbox{\tt spatstat} (see below) are complementary in may ways and data formats can be easily converted between the two packages. Manual at \url{http://cran.r-project.org/web/packages/sp/sp.pdf}. 

{\flushleft \tt SPACECAP}:
 \mbox{\tt SPACECAP} \citep{gopalaswamy_etal:2012mee} provides a user friendly GUI interface to fit SCR models with a Binomial observation model in a Bayesian framework. Manual at \url{http://www.icesi.edu.co/CRAN/web/packages/SPACECAP/SPACECAP.pdf}.
 
{\flushleft \tt spatstat}:
\mbox{\tt spatstat} \citep{baddeley_turner:2005} is an extensive package for analyzing spatial data. We use it, for example, to generate random points within a state space that cannot be described as a rectangle but consists of a (or several) arbitrary polygon(s). Manual at \url{http://cran.r-project.org/web/packages/spatstat/spatstat.pdf}. 

{\flushleft \tt unmarked}: 
\mbox{\tt unmarked} \citep{fiske_chandler:2011} fits hierarchical models of animal abundance and occurrence to data collected using a range of predominantly direct observation based methods. Manual at \url{http://cran.r-project.org/web/packages/unmarked/unmarked.pdf}.

% \begin{verbatim}

% ############################################################################################################################################################################################
% References

% \end{verbatim}
