XXXXXXXXXXXXXXXXXXXXXXXXXXXXXXXXXXXXXXXXXXXXXXXXXXXXXXXXXXXXXXXXXXXXXXXXXXXXXXXXXXXXXXXXXXXXXXXXXXXXXXXXXXXXXXXXXXXXXXXXXXXXXXXXXXXXXXXXXXXXXXXXXXXXXXXXXXXXXXXXXXXXXXXXXXXXXXXXXXXXXXXXXXXXXXXXXXXXXX
XXX PLEASE DUMP YOUR STUFF HERE XXXXXXXXXXXXXXXXXXXXXXXXXXXXXXXXXXXXXXXXXXXXXXXXXXXXXXXXXXXXXXXXXXXXXXXXXXXXXXXXXXXXXXXXXXXXXXXXXXXXXXXXXXXXXXXXXXXXXXXXXXXXXXXXXXXXXXXXXXXXXXXXXXXXXXXXXXXXXXXXXXXXXX

\chapter{Fully Spatial Capture-Recapture Models}
\markboth{Chapter 4 }{}
\label{chapt.scr0}

\vspace{.3in}

Throughout this book we have used a suite of software and R packages, all of which are freely available online. To make life a little easier for you, here we provide you with a list of all software and R packages, download links and some (hopefully) helpful tips and tricks regarding their installation.  


\section{WinBUGS}
Although {\bf WinBUGS} \citep{gilks_etal:1994} is becoming increasingly obsolete with the faster and more flexible {\bf OpenBUGS} and {\bf JAGS}, there are still situations in which the program comes in handy.  
The .exe file can be downloaded from http://www.mrc-bsu.cam.ac.uk/bugs/winbugs/contents.shtml. On 32 bit machines you can just go ahead and double-click on the .exe file and follow the installation instructions on the screen.
On 64 bit machines, according to the BUGS project you should download a zip file (from the same page) and unzip it into a folder of your choice.
There are a couple of additional steps to make BUGS run. 
First, you need to obtain a key (which is free and valid for life) here: http://www.mrc-bsu.cam.ac.uk/bugs/winbugs/WinBUGS14_immortality_key.txt. The key comes with instructions on how to activate it.
Second, you need to update the basic {\bf WinBUGS} version to the most current one (which is from August 2007) following the instructions given here: http://www.mrc-bsu.cam.ac.uk/bugs/winbugs/WinBUGS14_cumulative_patch_No3_06_08_07_RELEASE.txt.
{\bf WinBUGS} is ready to use after quitting and re-opening it.
Remember that {\bf WinBUGS} only runs on Windows machines. Also, there appears to be a problem installing the program in Vista, although we have no personal experience with this setup to draw upon.

\subsection{WinBUGS through R}
While you can run{\bf WinBUGS} as a standalone application, we recommend you access it from within {\bf R} using the package {\tt R2WinBUGS} \citep{sturtz_etal:2005}, so you can conveniently process your output, make graphs etc.   {\tt R2WinBUGS} also allows you to run models in {\bf OpenBUGS} (see below). You can install the package from within {\bf R} directly from a cran mirror. In addition to the usual package help document (http://cran.r-project.org/web/packages/R2WinBUGS/R2WinBUGS.pdf) you can also download a short manual with some examples (\url{''http://voteview.com/bayes_beach/R2WinBUGS.pdf''}). 



\section{OpenBUGS}
{\bf OpenBUGS} is the open-source up-to-date version of {\bf WinBUGS} and can be downloaded here: \url{''http://www.openbugs.info/w/Downloads''} (Windows, Mac and Linux versions are available).  For Windows, install by double-clicking on the .exe file and following the instructions on the installer screen. While {\bf OpenBUGS} is constantly being updated and provides more flexible modeling options than {\bf WinBUGS}, we have encountered convergence problems with simple scr models in this program. 

\subsection{OpenBUGS through R}
Like {\bf WinBUGS}, {\bf OpenBUGS} can be used as a standalone application or through {\bf R}. There are several packages that allow {\R} to interface with {\bf OpenBUGS}, all of which can be installed directly from a cran mirror:

\{flushleft R2WinBUGS: }
One of the options in the {\tt bugs()} call is {\tt program}, which lets you specify either {\bf WinBUGS} or {\bf OpenBUGS}. This is a convenient option because after having worked through some of this book you will likely be familiar with the format of {\tt bugs()} output and other functions of the {\tt R2WinBUGS} package.

\{flushleft R2OpenBUGS: }
{\tt R2OpenBUGS} \citep{sturtz_etal:2005} is very similar to, and actually based on, {\tt R2WinBUGS} and it is unclear to us what can be gained by using the former over the latter. Arguments of the {\tt bugs()} call differ slightly between the two packages and given that {\tt R2WinBUGS} allows for the use of both {\bf OpenBUGS} and {\bf WinBUGS} it is probably easiest to stick with it. 

\{flushleft BRugs: }
{\tt BRugs} \citep{thomas_etal:2006} can be installed from within {\bf R} directly from a cran mirror. In addition to the help document at \url{''http://www.biostat.umn.edu/~brad/software/BRugs/BRugs_9_21_07.pdf''}  there is a {\bf WinBUGS} style manual you can access at \url{''http://www.rni.helsinki.fi/openbugs/OpenBUGS/Docu/BRugs%20Manual.html''}.
{\tt BRugs} has the convenient feature that all pieces of a {\bf BUGS} analysis can be run from within {\bf R}, including checking the model syntax, something that requires opening the {\bf BUGS} GUI with other packages. 



\section{JAGS}
{\bf JAGS} (Just Another Gibbs Sampler) \citep{plummer:2003} runs scr models considerably faster than {\bf WinBUGS}, does not have the convergence problem with simple scr models we have encountered in {\bf OpenBUGS} but similar to the latter program, is flexible and constantly updated. It is available for download at \url{''http://sourceforge.net/projects/mcmc-jags/files/''}, together with the R package \mbox {\tt rjags} \citep{plummer:2011}, which allows running {\bf JAGS} through {\bf R}, user and installation manuals and examples. At this site {\bf JAGS} is available for Windows and Mac; Linux binaries are distributed separately and you can find links to various sources here: \url{''http://mcmc-jags.sourceforge.net/''}. {\bf JAGS} comes with a 32 bit and a 64 bit version and can be installed by double-clicking on the .exe file and following the instructions on the installer screen. We have found that sometimes {\bf JAGS} crashes for unclear reasons, taking {\bf R} down with it. Oftentimes, in order to make it run again you'll have to go through downloading and installing it again (remove the non-functioning version first). 

\subsection{JAGS through R}
Unlike the two {\bf BUGS} programs, {\bf JAGS} does not have a GUI interface but a command line interface that can be used to run the program as a standalone application. {\bf JAGS} will solely perform the MCMC simulation; analyzing and summarizing the output has to be done outside of {\Jbf JAGS}. To run {\bf JAGS} through {\bf R} you have two options.
 
 \{flushleft rjags: }
 As mentioned above, \mbox {\tt rjags} \citep{plummer:2011} can be found together with {\bf JAGS} and was developed/is being maintained by the inventor of {\bf JAGS}, which means it is guaranteed to stay up to date when/as {\bf JAGS} changes. The package can be installed from a cran mirror and the help document can be accessed at \url{''http://cran.r-project.org/web/packages/rjags/rjags.pdf''}
 
 \{flushleft R2jags: }
 Alternatively, the package \mbox{\tt R2jags} \citep{su_yajima:2011} provides a means of accessing {\bf JAGS} through {\bf R}. We prefer \mbox {\tt rjags} for the reason named above, as well as because it stores data in a more memory-efficient way and has better \mbox{\tt plot()} and \mbox{\tt summary()} methods. 



\section{R}
At the time of the preparation of this list, {\bf R} for Windows is at version 2.15.0, which can be downloaded at http://cran.r-project.org/bin/windows/base/ 
This site also contains helpful tips on how to install {\bf R} in Windows Vista, how to update {\bf R} packages etc. 
Installation of {\bf R} in Windows is straightforward: download the .exe file, double-click on it and follow the instructions of the Windows installer. The later versions of {\bf R} come with versions for both 64 bit and 32 bit machines. 
The {\bf R} site (\url{''http://mirrors.softliste.de/cran/''}) has an extensive FAQ section \citet{hornik:2011}, which includes instructions on how to install R on Unix and Mac computers.  

\subsection{R packages}
This section provides an alphabetical list of useful {\bf R}  packages. There is a large number of {\bf R} packages and by no means is this list intended to be complete in terms of what is useful. Rather, we list packages that we are familiar with and that we employ at one point or the other in this book. Unless explicitly stated otherwise, all packages can be installed directly from within {\bf R} trough a cran mirror. 

 \{flushleft {{\tt coda}: }
\mbox{{\tt coda} \citep{plummer_etal:2006} lets you summarize and perform diagnostics on mcmc output. For a list and description of functions, see the manual at \url{''http://cran.r-project.org/web/packages/coda/coda.pdf''}. 

 \{flushleft {{\tt maptools}: }
\mbox{\tt maptools} \citep{lewin-koh_etal:2011} provides a set of tools for reading and manipulating spatial data, especially ESRI shapefiles. Manual at \url{''http://cran.r-project.org/web/packages/maptools/maptools.pdf''}. 

 \{flushleft {{\tt raster}: }
\mbox{\tt raster} \citep{hijmans_vanetten:2012} provides functions for geographic analysis and modeling with raster data. Manual at \url{''http://cran.r-project.org/web/packages/raster/raster.pdf''}. 

 \{flushleft {{\tt reshape}: }
\mbox{\tt reshape} \citep{wickham_hadley:2007} allows you to easily manipulate, summarize and reshape data. Manual at \url{''http://cran.r-project.org/web/packages/reshape/reshape.pdf''}. 
  
 \{flushleft {{\tt rgeos}: }
\mbox{\tt rgeos} \citep{bivand_rundel:2011} provides many useful functions for spatial operations such as intersecting or buffering spatial features. Manual at \url{'http://cran.r-project.org/web/packages/rgeos/rgeos.pdf''}. 

 \{flushleft {{\tt shapefiles}: }
\mbox{{\tt shapefiles} \citep{stabler:2006} allows you to read and write ESRI shapefiles (i.e. shapefiles you would use in {\bf ArcGIS}). Manual at \url{''http://cran.r-project.org/web/packages/shapefiles/shapefiles.pdf''}. 

 \{flushleft {{\tt sp}: }
\mbox{{\tt sp}  \citep{pebesma_bivand:2011} is a package for plotting, selecting, subsetting etc. spatial data. \mbox{{\tt sp}  and \mbox{{\tt spatstat} (see below) are complementary in may ways and data formats can be easily converted between the two packages. Manual at \url{''http://cran.r-project.org/web/packages/sp/sp.pdf''}. 

 \{flushleft {{\tt spatstat}: }
\mbox{{\tt spatstat} \citep{baddeley_turner:2005} is an extensive package for analyzing spatial data. We use it, for example, to generate random points within a state space that cannot be described as a rectangle but consists of a (or several) arbitrary polygon(s). Manual at \url{''http://cran.r-project.org/web/packages/spatstat/spatstat.pdf''}. 




############################################################################################################################################################################################
References
@Misc{hornik:2011,
       author        = {Kurt Hornik},
       title         = {The {R} {FAQ}},
       year          = {2011},
       note          = {{ISBN} 3-900051-08-9},
       url           = {http://CRAN.R-project.org/doc/FAQ/R-FAQ.html}
     }
     

  @Article{sturtz_etal:2005,
    title = {R2WinBUGS: A Package for Running WinBUGS from R},
    author = {Sibylle Sturtz and Uwe Ligges and Andrew Gelman},
    journal = {Journal of Statistical Software},
    year = {2005},
    pages = {1--16},
    number = {3},
    volume = {12},
    url = {http://www.jstatsoft.org},
  }

  @Article{lunn_etal:2009,
    title = {The BUGS project: Evolution, critique, and future directions},
    author = {Lunn, D. and  Spiegelhalter, D. and Thomas, A. and Best, N.},
    journal = {Statistics in Medicine},
    year = {2009},
    pages = {3049--3067},
    volume = {28}
  }
  
  @article{yoshizaki_etal:2009,
	title = {Modeling misidentification errors in capture-recapture studies using photographic identification of evolving marks},
	volume = {90},
	number = {1},
	journal = {Ecology},
	author = {Yoshizaki, J. and Pollock, K. H. and Brownie, C. and Webster, R. A.},
	year = {2009},
	pages = {3--9}
}


@article{lukacs_burnham:2005,
	title = {Research Notes: Estimating Population Size from {DNA-based} Closed Capture-recapture Data Incorporating Genotyping Error},
	volume = {69},
	shorttitle = {Research Notes},
	number = {1},
	journal = {Journal of Wildlife Management},
	author = {Lukacs, P. M. and Burnham, K. P.},
	year = {2005},
	pages = {396--403}
}

@article{link_etal:2010,
	title = {Uncovering a latent multinomial: analysis of mark–recapture data with misidentification},
	volume = {66},
	shorttitle = {Uncovering a latent multinomial},
	number = {1},
	journal = {Biometrics},
	author = {Link, W. A. and Yoshizaki, J. and Bailey, L. L. and Pollock, K. H.},
	year = {2010},
	pages = {178--185}
}

@Article{plummer_etal:2006,
    title = {CODA: Convergence Diagnosis and Output Analysis for MCMC},
    author = {Martyn Plummer and Nicky Best and Kate Cowles and Karen Vines},
    journal = {R News},
    year = {2006},
    volume = {6},
    number = {1},
    pages = {7--11},
    url = {http://CRAN.R-project.org/doc/Rnews/},
    pdf = {http://CRAN.R-project.org/doc/Rnews/Rnews_2006-1.pdf},
  }

  @Manual{stabler:2006,
    title = {shapefiles: Read and Write ESRI Shapefiles},
    author = {Ben Stabler},
    year = {2006},
    note = {R package version 0.6},
  }

  @Article{baddeley_turner:2005,
    title = {Spatstat: an {R} package for analyzing spatial point patterns},
    author = {Adrian Baddeley and Rolf Turner},
    journal = {Journal of Statistical Software},
    volume = {12},
    number = {6},
    pages = {1--42},
    year = {2005},
    note = {{ISSN} 1548-7660},
    url = {www.jstatsoft.org},
  }
  
    @Article{wickham_hadley:2007,
    author = {{Wickham} and {Hadley}},
    journal = {Journal of Statistical Software},
    number = {12},
    title = {Reshaping data with the reshape package},
    url = {http://www.jstatsoft.org/v21/i12/paper},
    volume = {21},
    year = {2007},
  }

  @Manual{hijmans_vanetten:2012,
    title = {raster: Geographic analysis and modeling with raster data},
    author = {Robert J. Hijmans & Jacob van Etten},
    year = {2012},
    note = {R package version 1.9-92},
    url = {http://CRAN.R-project.org/package=raster},
  }

