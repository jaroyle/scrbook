\begin{verbatim}
"A preface or foreword deals with the genesis, purpose, limitations,
and scope of the book and may include acknowledgments of indebtedness;

an introduction deals with the subject of the book, supplementing and
introducing the text and indicating a point of view to be adopted by
the reader. The introduction usually forms a part of the text [and the
text numbering system]; the preface does not." (In other words, the
arabic numbering of the book (1,2,3) starts with the introduction, if
there is one. The other front matter takes i, ii, iii, etc.)
\end{verbatim}


Spatial capture-recapture models (aka spatially-explicit
capture-recapture) are a relatively new invention but they stand to
revolutionize how people study animal populations. We feel
that SCR models are really important and will come to dominate the
study of animal populations. 
% Maybe add something why we'd say that; assuming that we decided that we don't want to piss all the traditional CR folks off.
% Maybe: 'SCR models present an important improvement over traditional, i.e. non-spatial CR models because the explicitly deal with animal exposure to trapping and that results from individual distribution in space, as well as the ecologically intuitive linke between abundance and area, both of which are ignored in traditional CR. Includign spatial processes, these models can be expanded/adapted/... to directly address many questions related to animal ecology and conservation. With such advanced tools at hand, we believe that but for some specific situations, ordinary closed population
% models are essentially obsolete in our view, except as a conceptual
% device. '
At this time, ordinary closed population
models are essentially obsolete in our view, except as a conceptual
device. The recent literature on SCR models has exploded -- and people are
keenly interested -- so the purpose of this book is to bring together
all of the recent developments over the last few years and give people
practial options for analyzing their own data using the large 8and growing) class of SCR models. 

In this book we
provide a survey of the models that are out there now, even some new
ones, and  how people use them in practice. Some of this is really new
stuff that we wrote in response to gaps in the literature. Things we
had  said from time to time that SCR models should be useful for,
but had not been established yet. 

CR and SCR have been thought of mostly as ways to ``estimate density''
with not so much of a direct link to understanding ecological
processes. So one of the things that motivated us here was to
elaborate on, and develop, some ideas related to modeling ecological
processes (movement, space usage, landscape connectivity) in the
context of SCR models. 
So we do have a lot of material on density estimation -- this is
problem \# 1 in applied ecology, but we try to cover a lot more on
spatial aspects of populations as they related to SCR.
The 2nd thing that reallly motivated this book was to provide a
broad-based framework for {\it implementation} using R/WinBUGS/JAGS and
using both Bayesian and classical modes of inference. 
Why both? (see below) 

\section*{Statistical Modeling}

SCR models are a type of statistical model. Therefore, 
the book is basically an applied statistical modeling book.

In essence, SCR
are simple hierarchical models. 
Hierarchical models are models that consist of a formal description of a state process -- a process that characterizes what the true state of our study system looks like and usually the process we are utimately interested in -- and a separate observation process, which describes how our data were generated conditional on the state process. This acknowledges that, usually, we cannot directly observe the state process, but rather have a XXX (biased?incomplete?) picture of the state we're interested in. XXX This might be a little confusing... XXXX
SCR models condition on a latent variable ${\bf s}$ which we interpret
as an explicit biological process -- the outcome of animals
distributing themselves over a landscape.  Some conceptual and
methodological aspects of hierarchical modeling can be found in
\citet{royle_dorazio:2008}, we rehash some of those ideas as we need
to but we don't focus too much on the broader framework of
hierarchical modeling.
We do, however, 
 provide some background material on inference strategies and,
 especially, Bayesian analysis using the  {\bf BUGS} variants {\bf
   JAGS} and {\bf WinBUGS}. 
XXXX ANDY: Maybe move the rest of this paragraph down to the overview of the chapters XXX 
We introduce Bayesian inference in some
 detail because people are less likely to have had a class in that and
 we also wanted to produce a stand-alone thing.   We also do
 likelihood analysis of many models and we provide an introduction to
 the relevant elements of likelihood analysis in Chapt. XXX and the
 implementation of SCR models in the package secr \citep{efford_etal:2009euring}.

We develop SCR models using a dual inference paradigm involving both
Bayesian and classical methods based on likelihood inference.
We are not strong advocates of either Bayesian or likelihood-based
inference but rather we think either works in most problems and one or
the other works betters in some cases. 
We show examples throughout the
book for which the specific problem solved using Bayesian methods
would be difficult using likelihood methods and vice versa.


\section*{Computing}

Emphasis on doing things in the 
R programming language which by now is something that very many
ecologists not only knows about, but actually uses. We adopt R because it is free; because it has a large community that constantly develops code for new applications; and because it gives us great flexibility.
  There are some
great books out there including Bolker (XXXX) and XXXXX. We use a
large number of R packages in our analyses, we address them in
Appendix 1, and moreover, we provide an R package with this book (see below).

We do Bayesian analysis almost exclusively in the BUGS language, using
WinBUGS and JAGS. Mostly we are transitioning to use of JAGS but we
still like WinBUGS a lot. Oold habits die hard..... WinBUGS is not in
active development anymore, but JAGS is. So is OpenBUGS but we didn't
want to use all of them (whats the point?) 
XXXX I don't know if that's a good thing to put into a preface but I thought there were some issues with standard SCR0 models in OpenBUGS giving weird/wrong results. I think we could just say it's a new habit and that JAGS has some nice features (keep it vague...) XXXXX
 We love the BUGS language
because, as Marc Kery said, it ``frees the modeler in you''.
If you can express your model algebraically, in the BUGS language,
then JAGS or WinBUGS or OpenBUGS will do the MCMC for you. Thats
pretty handy.

We do a limited amount of developing our own custom MCMC algorithms
(see chapt. XXXX) which we think is really handy for certain
problems. In fact, there are problems that WinBUGS or JAGS can't do,
and so we have had to develop our own custom algorithms (e.g.,
Sollmann et al. 2012; Chandler and Royle 2013). 
XXXX We can do that with BUGS also, to some extent, so maybe just stick to usefulness in situations where JAGS fails?XXXXXX
This is really handy
because you can then exploit large linux or windows clusters to
distribute your computing efficienctly. There are R packages for that
(snowfall XXXX and others). 

We do a fair amount of likelihood analysis in this book. We have a
chapter on how to do this, and several chapters use likelihood methods
exclusively. We rely 
on the R package \mbox{\tt secr} \citep{efford_etal:2009euring} for
many analyses and we think people should use this because it {\it is}
fairly general and has the usual R summary methods and considerably
capability for doing analysis from start to finish. In some chapters
we discuss models that we have to use likelihood methods for, but
which are not implemented in secr  (e.g., Chapt XXXX). This is a good
example of why it
is useful to understand the principles and to be able to implement
these methods yourself. 


\section{The R package scrbook}

As we were developing content for the book it was obvious that it
would be useful if the
R/BUGS scripts and data were availble for readers to reprdocue the
analyses and also to modify so that they can do their own analysis. 
Almost every analysis that we did is in the R package. The R package
should be a very dynamic thing, as we plan to 
 continue to update and expand it.

The purpose of the package is  not meant to be general-purpose and
flexible software for doing SCR models but, rather, a set of examples
and templates to see how specific things are done. Code can be used by the reader to develop methods tailored to his/her situation, or possibly even more general methods.  Because we use so
many different software packages and computing platforms, we think its
impossible to put all of what is convered in this book into a single
integrated package. 

XXXXX Moved that up XXXXX
We give many other examples
that are not meant to be general and flexible, but rather meant so that
 interested readers can better understand the computational
aspects of the problems, and can modify it for their purposes.


\section{Organization}

We didn't
think to write a 550 page book on SCR models because there wasn't even
that much material to work when when the project was started back in
the early part of 2009. But during the project great, and new things were
produced, and we developed new models and concepts taht led to new
material for the book, including models of resource selection,
landscape connectivity, and methods for dealing with unmarked
individuals. There are at least 10 chapters in the book that we
couldn't have thought about 4 years ago. Actually, more ideas kept coming up and we ended up not even being able to cover all of them extensivvely in this book; so we also in places provide ideas of how SCR models could be developed further. 

In the following chapters we develop a comprehensive synthesis and extension of
spatial capture-recapture models.
Roughly the first third of the book is introductory material --
In Chapt. \ref{chapt.glms} we provide the basic analysis tools to understand and
analyze SCR models - namely generalized linear models (GLMs) with random effects, and their
analysis in {\bf R} and {\bf WinBUGS}.  Because SCR models represent extensions of
basic closed population models, we cover ordinary closed population
models in Chapt. \ref{chapt.closed} wherein, along with Chapts. \ref{chapt.scr0} and \ref{chapt.poisson-mn}
\footnote{might ought to put Modeling Encounter Probability
  as chapter 5 instead}, provides the basic introduction
to capture-recapture models and their spatial extension.... this
covers XXXXX.
observation models related to the type of encounter device (Chapt. \ref{chapt.poisson-mn}), models for encounter probability
(Chapt. \ref{chapt.covariates}),
In the middle part of the book we extend capture-recapture to SCR models
and discuss a number of different conceptual and technical topis
including 
tools for likelihood inference (Chapt. \ref{chapt.mle}), sampling design
(Chapt. XXXXX) and analysis of model fit and model selection
(Chapt. \ref{chapt.gof}). 

The last part of the book, which is really about 40\% or more by page
count, is ``advanced'' stuff. This covers a chapter on developing your
own MCMC algorithsm for SCR models. We do this because many advanced
models require you to do this and we think much of the future of SCR
will require MCMC methods. 
We also have a number of chapters on spatial modeling aspects related
to SCR including modeling 
space usage or resource selection (Chapt \ref{chapt.rsf}), modeling
landscape connectivity (Chapt. \ref{chapt.ecoldist}), and modeling
spatial variation in density. 
We cover open population models (Chapt. \ref{chapt.open}),
a bunch of stuff on unmarked individuals either entirely (Chapt. \ref{chapt.scr-unmarked})
or partially marked samples (Chapt. \ref{chapt.partialID}), 
a chapter on search-encounter methods XXXX, a chapter on multi-session
models XXXXXX (probably move this to chapter 10?)


\section*{Point of View}
What is the approach or philosophy -- the overriding set of ideas that
motivates how we developed and organized conent?
[I stole this from chapter 2]:
In our experience, students in ecology and even many established
scientists simply cannot separate what they need to do from how to do
it.  They cannot distinguish clearly (either conceptually or actually)
the difference between the model for their data, and the actual
procedure of how to estimate parameters of that model, or make
predictions - ie., how to do the calculations. Sometimes this issue
raises itself in an email from some hapless grad student wondering
``what is the right statistical test for this type of data?''  In a
sense it is this view that drives our approach to developing elements
of this book.

In contemporary statistical ecology, models and methods are sometimes
obscured by named procedures often that are completely uninformative,
the technical details of which hide in obscurity in some black boxes
such as MARK, PRESENCE, DISTANCE, etc., known only by the few
specialist experts in the field. 
XXXX I think we should say something to the extent that these black boxes are very useful but their user-friendlyness comes with the risk of abuse - the risk that you can run an analysis and get a result without really knowing what you're doing XXXXXX
While it is sometimes convenient to
refer to a type or class of models by a name (logistic regression or
even ``model Mh'') in order to emphasize a broad concept or
methodological area, this is only useful if the fundamental
statistical and mathematical structure underlying that name is
clear. As such, we try to focus on model development and keep the
model development distinct from how to combine our data with the model
to produce estimates and so forth. We talk a lot about hypothetical
data we wish we could observe - complete data sets - data sets as if
$N$ were known, etc.. We talk about the model in precise terms and
then break down various ways for analyzing the model either using
likelihood methods or Bayesian methods or some black-box that does one
or the other.

XXX All of the stuff in the paragraph below you already covered in one place or another XXXXX
To fit models, we rely heavily on the various implementations of the
{\bf BUGS} language including {\bf WinBUGS} \citep{lunn_etal:2000},
{\bf JAGS} \citep{plummer:2003}
 and {\bf OpenBUGS} \citep{thomas_etal:2006}. We really like
the {\bf BUGS} language, not merely  as a computational device for
fitting models but because it emphasizes
understanding of what the model is and fosters understanding how to
build models - as Kery XYZ XYZ says ``it frees the modeler in you.''  (direct
citation for this would be nice).  However, in addition to using the
{\bf BUGS} language and its various implementations, we also develop our own
{\bf R} code both for doing MCMC
and maximum likelihood, for which we also use the R
package \mbox{\tt secr} \citep{efford:2011}. In addition, we have
created an {\bf R} package to go with this book, \mbox{\tt scrbook},
which contains the data sets, {\bf R} and {\bf BUGS} scripts, and {\bf
  R} code for doing summary analyses, and some likelihood and MCMC
functions written solely in {\bf R}.


\section*{Who should read this book}

This book is not a book about Bayesian analysis, not a book about
hierarchical models, not a book about capture-recapture, and not about
programming in R. In a sense though, our book integrates elements of
all of these things into what we hope is a coherent package for
analyzing data from this class of data collection methods
that produce spatially-explicit capture-recapture data, which is really a wide range of methods.   As such, we
expect that people have a basic understanding of statistical models
and classical inference (What is frequentist inference? what is a
likelihood? Generalized linear model? Generalized linear mixed
model?), 
 Bayesian analysis (what is s a prior distribution and a
posterior distribution?),
certainly have used the {\bf R} programming environment,
and maybe even a little bit
of Bayesian
computation (MCMC and perhaps the BUGS language).
The ideal candidate for reading this book has basic knowledge of these
topics. However, we do provide introductory chapters on the necessary
components which we hope can serve as a brief and cursory tutorial for
those who might have only limited technical knowledge, e.g., many
carnivore biologists who implement field sampling programs but do not
have extensive experience analyzing data.
XXX Hey, don't single out the carnivore folks! They're no stupider than other field biologists! XXXX

\section*{Why should you read this book?}

The future of studying animal populations is SCR. There is not doubt
about that. Many questions in ecology and conservation are related to how individuals distribute themselves in space, and SCR models deal with incorporating aspects of spatial ecology into a capture-recapture framework for estimating abundance, density and other demographic parameters. As such, we believe that there is material of interest for most wildlife ecologists in this book. While traditional CR models, and especially those for open populations, have been developed for decades to deal with aspects of populations, including spatial stuff into models explicitly, we believe will almost always lead to a more realistic representation of your study system.  So this book is not only extremely interesting in ecological terms, but also at the forefront of capture-recapture model development.

\section{What sucks about this book?}

Nothing. Right?
Nah, it's awesome. 



\section{Scope of this Book}

{\bf XXX JUST COPIED FROM CHAPT 1 . Should go here. XXXX}


In this book, we try to achieve a broad methodological scope from
basic closed population models %using a number of distinct observation
%models
for inference about population density, movement, space usage and resource
selection, on up to open population models for inference about vital
rates such as survival and recruitment. %---spatial versions of
%conventional Jolly-Seber models. %A number of conceptual and
%methodological themes unify the main topical coverage of this book, and
%those are:
Much of the material is a synthesis of recent research but we also
expand SCR models in a number of useful directions, including to
accommodate unmarked individuals (Chapt. \ref{chapt.scr-unmarked}),
use of telemetry information (Chapt. \ref{chapt.rsf}), and developing
explicit models of landscape connectivity based on ecological or
least-cost distance (Chapt. \ref{chapt.ecoldist}), and many other new
topics that have only recently appeared in the literature.  Our intent
is to provide a comprehensive resource for ecologists interested in
understanding and applying SCR models to solve common problems faced
in the study of populations.  To do so, we make use of hierarchical
models \citep{royle_dorazio:2008}, which allow extraordinary
flexibility in accommodating many types of capture-recapture data. We
present many example analyses, of real and simulated data using
likelihood-based and Bayesian methods---examples that readers can
replicate using the code presented in the text and the resources made
available on-line and in our accompanying {\bf R} package {\tt
  scrbook}.

Although we aim to reach a broad audience, at times we go into details
that may only be of interest to advanced practitioners who need to
extend capture-recapture models to unique situations.  We hope that
these advanced topics will not discourage those new to these methods,
but instead out intent is to allow readers to advance their own
understanding and become less reliant on restrictive tools and
software.
%Before discussing the specifics of SCR models, we begin with
%an overview of the methods used to collect capture-recapture data, and
%provide a brief summary of traditional non-spatial capture-recapture
%models.
%In this book we present a diverse array of modeling approaches for
%making inference about density and population dynamics using spatial
%capture-recapture data. 
A number of conceptual and methodological
themes unify the main topical coverage of this book, and those are:

\begin{itemize}
\item[(1)] Hierarchical modeling. We develop hierarchical models
  consisting of explicit models for both the observation process and
  the underlying ``ecological process'' which describes the
  organization of individuals in space.

\item[(2)] Spatial processes in capture-recapture. We emphasize the
  linkage of capture-recapture data to underlying ecological processes
  including density or distribution of individuals, 
space usage, resource selection and movement.

\item[(3)] Formal inference using both classical (frequentist,
  likelihood-based) and Bayesian methods. We often emphasize
  Bayesian analysis because this allows us to focus the technical
  formulation of models, and spatial capture-recapture is mainly
  concerned with modeling random effects and estimating functions of
  random effects. However, we also explore likelihood methods using existing
  software such as the {\bf R} package \mbox{\tt secr} \citep{efford:2011}, as well as
  development of custom solutions along the way.

\item[(4)] In developing Bayesian analyses of SCR models, we emphasize
  the use of the {\bf BUGS} language for describing models. The {\bf BUGS}
langage emphasizes the syntactic description of the essential
  assumptions of models in a special kind of pseudo-code language,
  which is used in software ({\bf WinBUGS}, {\bf JAGS}, {\bf OpenBUGS}) to devise Markov
  chain Monte Carlo (MCMC) algorithms for Bayesian analysis of
  models. The {\bf BUGS} language focuses your thinking on model development
  and lets you develop an understanding of models at the level of
  their basic assumptions and structure.  Despite our focus on
  describing models in {\bf BUGS}, we also show readers how
  to devise their own MCMC algorithms for Bayesian analysis of SCR
  models, which can be convenient (even necessary) in some practical
  situations.

%\item[(4)] Data augmentation -- dealing with the fact that population
%  size, $N$, is unknown is a challenging technical problem in
%  capture-recapture models. We confront this problem in almost every
%  chapter of this book. To deal with it we use a technical device
%  called {\it data augmentation} which is extremely useful for
%  analysis of capture-recapture models that are specified
%  ``conditional on $N$'' \citep{royle_etal:2007}.
\end{itemize}

\begin{comment}
Altogether, these different conceptual and methodological elements
provide for a formulation of SCR models that essentially renders them
as variations of generalized linear mixed models (GLMMs). This in a
sense makes them consistent with many important methodologies used in
ecology (e.g., see \citet{zuur_etal:2009, kery_etal:2010}), and
because of the connection with standard modeling concepts, we believe
that the material presented in this book can be understood and used by
most ecologists with some modeling experience.
\end{comment}

