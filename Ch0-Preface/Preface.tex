\begin{verbatim}
"A preface or foreword deals with the genesis, purpose, limitations,
and scope of the book and may include acknowledgments of indebtedness;

an introduction deals with the subject of the book, supplementing and
introducing the text and indicating a point of view to be adopted by
the reader. The introduction usually forms a part of the text [and the
text numbering system]; the preface does not." (In other words, the
arabic numbering of the book (1,2,3) starts with the introduction, if
there is one. The other front matter takes i, ii, iii, etc.)
\end{verbatim}


Spatial capture-recapture models (aka spatially-explicit
capture-recapture) are a relatively new invention but they stand to
revolutionize how people study animal populations. We feel
that SCR models are really important and will come to dominate the
study of animal populations. At this time, ordinary closed population
models are essentially obsolete in our view, except as a conceptual
device. The literature has exploded on these things -- and people are
keenly interested -- so the purpose of this book is to bring together
all of the recent developments over the last few years and give people
practial options for analyzing their own data. 

In this book we
provide a survey of the models that are out there now, even some new
ones, and  how people use them in practice. Some of this is really new
stuff that we wrote in response to gaps in the literature. Things we
had  said from time to time that SCR models should be useful for,
but had not been established yet. 

CR and SCR have been thought of mostly as ways to ``estimate density''
with not so much of a direct link to understanding ecological
processes. So one of the things that motivated us here was to
elaborate on, and develop, some ideas related to modeling ecological
processes (movement, space usage, landscape connectivity) in the
context of SCR models. 
So we do have a lot of material on density estimation -- this is
problem \# 1 in applied ecology, but we try to cover a lot more on
spatial aspects of populations as they related to SCR.
The 2nd thing that reallly motivated this book was to provide a
broad-based framework for {\it implementation} using R/WinBUGS/JAGS and
using both Bayesian and classical modes of inference. 
Why both? (see below) 

\section*{Statistical Modeling}

SCR models are a type of statistical model. Therefore, 
the book is basically an applied statistical modeling book.

In essence, SCR
are simple hierarchical models. 
Hierarchical models are..............
SCR models condition on a latent variable ${\bf s}$ which we interpret
as an explicit biological process -- the outcome of animals
distributing themselves over a landscape.  Some conceptual and
methodological aspects of hierarchical modeling can be found in
\citet{royle_dorazio:2008}, we rehash some of those ideas as we need
to but we don't focus too much on the broader framework of
hierarchical modeling.
We do, however, 
 provide some background material on inference strategies and,
 especially, Bayesian analysis using the  {\bf BUGS} variants {\bf
   JAGS} and {\bf WinBUGS}. We introduce Bayesian inference in some
 detail because people are less likely to have had a class in that and
 we also wanted to produce a stand-alone thing.   We also do
 likelihood analysis of many models and we provide an introduction to
 the relevant elements of likelihood analysis in Chapt. XXX and the
 implementation of SCR models in the package secr \citep{efford_etal:2009euring}.

We develop SCR models using a dual inference paradigm involving both
Bayesian and classical methods based on likelihood inference.
We are not strong advocates of either Bayesian or likelihood-based
inference but rather we tihnk either works in most problems and one or
the other works betters in some cases. 
We show examples throughout the
book for which the specific problem solved using Bayesian methods
would be difficult using likelihood methods and vice versa.


\section*{Computing}

Emphasis on doing things in the 
R programming language which by now is something that almost every
ecologists not only knows about, but actually uses.  There are some
great books out there including Bolker (XXXX) and XXXXX. We use a
large number of R packages in our analyses, we address them in
Appendix 1.

We do Bayesian analysis almost exclusively in the BUGS language, using
WinBUGS and JAGS. Mostly we are transitioning to use of JAGS but we
still like WinBUGS a lot. Oold habits die hard..... WinBUGS is not in
active develpment anymore, but JAGS is. So is OpenBUGS but we didn't
want to use all of them (whats the point?). We love the BUGS language
because, as Marc Kery said, it ``frees the modeler in you''.
If you can express your model algebraically, in the BUGS language,
then JAGS or WinBUGS or OpenBUGS will do the MCMC for you. Thats
pretty handy.

We do a limited amount of developing our own custom MCMC algorithms
(see chapt. XXXX) which we think is really handy for certain
problems. In fact, there are problems that WinBUGS or JAGS can't do,
and so we have had to develop our own custom algorithms (e.g.,
Sollmann et al. 2012; Chandler and Royle 2013). This is really handy
because you can then exploit large linux or windows clusters to
distribute your computing efficienctly. There are R packages for that
(snowfall XXXX and others). 

We do a fair amount of likelihood analysis in this book. We have a
chapter on how to do this, and several chapters use likelihood methods
exclusively. We rely 
on the R package \mbox{\tt secr} \citep{efford_etal:2009euring} for
many analyses and we think people should use this because it {\it is}
fairly general and has the usual R summary methods and considerably
capability for doing analysis from start to finish. In some chapters
we discuss models that we have to use likelihood methods for, but
which are not implemented in secr  (e.g., Chapt XXXX). This is a good
example of why it
is useful to understand the principles and to be able to implement
these methods yourself. 


\section{The R package scrbook}

As we were developing content for the book it was obvious that it
would be useful if the
R/BUGS scripts and data were availble for readers to reprdocue the
analyses and also to modify so that they can do their own analysis. 
Almost every analysis that we did is in the R package. The R package
should be a very dynamic thing, as we plan to 
 continue to update and expand it.

The purpose of the package is  not meant to be general-purpose and
flexible software for doing SCR models but, rather, a set of examples
and templates to see how specific things are done.   Because we use so
many different software packages and computing platforms, we think its
impossible to put all of what is convered in this book into a single
integrated package. 

We give many other examples
that are not meant to be general and flexible, but rather meant so that
 interested readers can better understand the computational
aspects of the problems, and can modify it for their purposes.


\section{Organization}

We didn't
think to write a 550 page book on SCR models because there wasn't even
that much material to work when when the project was started back in
the early part of 2009. But the project great, and new things were
produced, and we developed new models and concepts taht led to new
material for the book, including models of resource selection,
landscape connectivity, and methods for dealing with unmarked
individuals. There are at least 10 chapters in the book that we
couldn't have thought about 4 years ago. 

In the following chapters we develop a comprehensive synthesis and extension of
spatial capture-recapture models.
Roughly the first third of the book is introductory material --
In Chapt. \ref{chapt.glms} we provide the basic analysis tools to understand and
analyze SCR models - namely generalized linear models (GLMs) with random effects, and their
analysis in {\bf R} and {\bf WinBUGS}.  Because SCR models represent extensions of
basic closed population models, we cover ordinary closed population
models in Chapt. \ref{chapt.closed} wherein, along with Chapts. \ref{chapt.scr0} and \ref{chapt.poisson-mn}
\footnote{might ought to put Modeling Encounter Probability
  as chapter 5 instead}, provides the basic introduction
to capture-recapture models and their spatial extension.... this
covers XXXXX.
observation models related to the type of encounter device (Chapt. \ref{chapt.poisson-mn}), models for encounter probability
(Chapt. \ref{chapt.covariates}),
The middle part of the book we extend capture-recapture to SCR models
and discuss a number of different conceptual and technical topis
including 
tools for likelihood (Chapt. \ref{chapt.mle}), sampling design
(Chapt. XXXXX) and analysis of model fit and model selection
(Chapt. \ref{chapt.gof}). 

The last part of the book, which is really about 40\% or more by page
count, is ``advanced'' stuff. This covers a chapter on developing your
own MCMC algorithsm for SCR models. We do this because many advanced
models require you to do this and we think much of the future of SCR
will require MCMC methods. 
We also have a number of chapters on spatial modeling aspects related
to SCR including modeling 
space usage or resource selection (Chapt \ref{chapt.rsf}), modeling
landscape connectivity (Chapt. \ref{chapt.ecoldist}), and modeling
spatial variation in density. 
We cover open population models (Chapt. \ref{chapt.open}),
a bunch of stuff on unmarked individuals either entirely (Chapt. \ref{chapt.scr-unmarked})
or partially marked samples (Chapt. \ref{chapt.partialID}), 
a chapter on search-encounter methods XXXX, a chapter on multi-session
models XXXXXX (probably move this to chapter 10?)


\section*{Point of View}
What is the approach or philosophy -- the overriding set of ideas that
motivates how we developed and organized conent?
[I stole this from chapter 2]:
In our experience, students in ecology and even many established
scientists simply cannot separate what they need to do from how to do
it.  They cannot distinguish clearly (either conceptually or actually)
the difference between the model for their data, and the actual
procedure of how to estimate parameters of that model, or make
predictions - ie., how to do the calculations. Sometimes this issue
raises itself in an email from some hapless grad student wondering
``what is the right statistical test for this type of data?''  In a
sense it is this view that drives our approach to developing elements
of this book.

In contemporary statistical ecology, models and methods are sometimes
obscured by named procedures often that are completely uninformative,
the technical details of which hide in obscurity in some black boxes
such as MARK, PRESENCE, DISTANCE, etc., known only by the few
specialist experts in the field. While it is sometimes convenient to
refer to a type or class of models by a name (logistic regression or
even ``model Mh'') in order to emphasize a broad concept or
methodological area, this is only useful if the fundamental
statistical and mathematical structure underlying that name is
clear. As such, we try to focus on model development and keep the
model development distinct from how to combine our data with the model
to produce estimates and so forth. We talk a lot about hypothetical
data we wish we could observe - complete data sets - data sets as if
$N$ were known, etc.. We talk about the model in precise terms and
then break down various ways for analyzing the model either using
likelihood methods or Bayesian methods or some black-box that does one
or the other.

To fit models, we rely heavily on the various implementations of the
{\bf BUGS} language including {\bf WinBUGS} \citep{lunn_etal:2000},
{\bf JAGS} \citep{plummer:2003}
 and {\bf OpenBUGS} \citep{thomas_etal:2006}. We really like
the {\bf BUGS} language, not merely  as a computational device for
fitting models but because it emphasizes
understanding of what the model is and fosters understanding how to
build models - as Kery XYZ XYZ says ``it frees the modeler in you.''  (direct
citation for this would be nice).  However, in addition to using the
{\bf BUGS} language and its various implementations, we also develop our own
{\bf R} code both for doing MCMC
and maximum likelihood, for which we also use the R
package \mbox{\tt secr} \citep{efford:2011}. In addition, we have
created an {\bf R} package to go with this book, \mbox{\tt scrbook},
which contains the data sets, {\bf R} and {\bf BUGS} scripts, and {\bf
  R} code for doing summary analyses, and some likelihood and MCMC
functions written solely in {\bf R}.


\section*{Who should read this book}

This book is not a book about Bayesian analysis, not a book about
hierarchical models, not a book about capture-recapture, and not about
programming in R. In a sense though, our book integrates elements of
all of these things into what we hope is a coherent package for
analyzing data from this class of data collection methods
that produce spatially-explicit capture-recapture data.   As such, we
expect that people have a basic understanding of statistical models
and classical inference (What is frequentist inference? what is a
likelihood? Generalized linear model? Generalized linear mixed
model?), 
 Bayesian analysis (what is s a prior distribution and a
posterior distribution?),
certainly have used the {\bf R} programming environment,
and maybe even a little bit
of Bayesian
computation (MCMC and perhaps the BUGS language).
The ideal candidate for reading this book has basic knowledge of these
topics. However, we do provide introductory chapters on the necessary
components which we hope can serve as a brief and cursory tutorial for
those who might have only limited technical knowledge, e.g., many
carnivore biologists who implement field sampling programs but do not
have extensive experience analyzing data.

\section*{Why should you read this book?}

The future of studying animal populations is SCR. There is not doubt
about that. 

\section{What sucks about this book?}

Nothing. Right?



