\begin{verbatim}
"A preface or foreword deals with the genesis, purpose, limitations,
and scope of the book and may include acknowledgments of indebtedness;

an introduction deals with the subject of the book, supplementing and
introducing the text and indicating a point of view to be adopted by
the reader. The introduction usually forms a part of the text [and the
text numbering system]; the preface does not." (In other words, the
arabic numbering of the book (1,2,3) starts with the introduction, if
there is one. The other front matter takes i, ii, iii, etc.)
\end{verbatim}

\begin{comment}
RS:
According to what we discussed a few days ago, the 4 big blocks of the Preface should be
1. Why are we writing this book?
2. What are the themes?
3. Software/computing/modeling - something about the practical aspects
4. Organization of the book
I tried commenting on the below material referring to this overall orgainzation we agreed on.
\end{comment}

%RS:This is the 'Why' part of the preface. 
%I think Richard made this comment further down, but we should stress more the importance of 'space', both in ecology and in sampling
Spatial capture-recapture (SCR) models (aka spatially-explicit
capture-recapture) are a relatively new invention that %but they
stand to
revolutionize how people study animal populations.
% XXXX Need to first say something about how they are great for
% studying ecological processes, then mention that they improve upon
% non-spatial CR.
SCR models present
an important improvement over traditional, %i.e.
non-spatial CR models
because they explicitly describe exposure of individuals to encounter
that results from the juxtaposition of sampling devices or traps with
individuals, as well as the ecologically intuitive link between
abundance and area, both of which are ignored by traditional CR
models. Including spatial processes, these models can be adapted and
expanded to directly address many questions related to animal ecology
and conservation. With such advanced tools at hand, we believe that,
but for some specific situations, ordinary closed population models
are largely obsolete, % XXXX I think this is too extreme. How about,
                      % "we believe that virtually all non-spatial CR
                      % analyses can be improved upon by using the
                      % spatial information inherent to CR data to
                      % make inference about spatiai variation in density in capture
                      % probability."
except as a conceptual device.
%RS: I agree with Richard that this is too extreme a view and we should give the regular models some more credit. 
%Even if it is 'just' as a an important step in the development of more realistic models, like SCR. 
%In so many ways we draw on the regular CR stuff, I don't think we should ignore or downplay that. 


The recent literature on SCR models has exploded, %and the interest in
%their application has exploded. XXXX avoiding 2 "exploded"s
as has interest in applying these models for ecology and conservation.
Our purpose in writing this
book is to bring together all of the developments over the last few
years and give people practial options for analyzing their own data
using the large and growing class of SCR models.  We provide a survey
of the models that are available now, and even some new ones, and
demonstrate how they can be applied in practice. Some of the material
is really new, only just now appearing in the peer-reviewed
lieterature, and some of it we developed along the way in response to
gaps in the literature.

CR and SCR have been thought of mostly as ways to ``estimate density''
with not so much of a direct link to understanding ecological
processes. So one of the things that motivated us here was to
elaborate on, and develop, some ideas related to modeling ecological
processes (movement, space usage, landscape connectivity) in the
context of SCR models.  So, while we do have a lot of material on
density estimation -- this is problem \# 1 in applied ecology -- we
worked hard to cover a lot more of the spatial aspect of population
analysis as they relate to SCR. 
%RS: I don't like the following sentence. Of course we have more than any other book about spatial aspects of populations as they relate to SCR - it's the first book on SCR.
% We think we have more in this book
% than in any other book on spatial analysis of popualtions. 
There are a
lot of books out there that cover spatial analysis of popualtion
structure which are more theoretical or mathematical, and there are a
lot of books out there that cover sampling and estimation, but that
are {\it not} spatial. Our book bridges these two huge ideas as much
as was possible roughly mid-2012.  The 2nd thing that really
motivated this book was our interest in doing SCR, teaching people to
do SCR, and therefore to provide a framework for {\it implementation}
using \textbf{R}/\textbf{WinBUGS}/\textbf{JAGS} and using both Bayesian and classical modes of
inference.

%RS: I think this section should be moved to block 3: Software/computing/modeling
\section*{Statistical Modeling}

SCR models are a type of statistical model. Therefore, a reasonable
view of this book is that it is basically an applied statistical
modeling book.

We develop SCR models using a dual inference paradigm involving both
Bayesian and classical methods based on likelihood inference.  We are
not strong advocates of either Bayesian or likelihood-based inference
but rather we think either works in most problems and one or the other
works betters in some cases.  We show examples throughout the book for
which the specific problem solved using Bayesian methods would be
difficult using likelihood methods and vice versa.
% XXXX I really like the previous sentence, but I think we contradict
% it in several places later in the book

We expect that people have a basic understanding of statistical models
and classical inference (What is frequentist inference? What is a
likelihood? Generalized linear model? Generalized linear mixed
model?), Bayesian analysis (what is s a prior distribution and a
posterior distribution?), certainly have used the {\bf R} programming
environment, and maybe even a little bit of Bayesian computation (MCMC
and perhaps the \textbf{BUGS} language).  The ideal candidate for reading this
book has basic knowledge of these topics. However, we do provide
introductory chapters on the necessary components which we hope can
serve as a brief and cursory tutorial for those who might have only
limited technical knowledge, e.g., many biologists who implement field
sampling programs but do not have extensive experience analyzing data.


xxxxxxxxxxxxxxxxxxxxxxxxxxxxxxxxxxxxxxxxxxxxxxxxxxxxxxxxxxxxx

% XXXX Some of this gets said in next section
In essence, SCR are
simple hierarchical models.  %Hierarchical models are models that
The hierarchical models that we find most relevant in ecology % XXXX added
consist of a formal description of a state process -- a process that
characterizes what the true state of our study system looks like and
usually the process we are utimately interested in -- and a separate
observation process, which describes how our data were generated
conditional on the state process. This acknowledges that, usually, we
cannot directly observe the state process, but rather have a partially
observed or biased picture of the state we're interested in.  SCR
models condition on a latent variable ${\bf s}$ which we interpret as
an explicit biological process -- the outcome of animals distributing
themselves over a landscape.  Some conceptual and methodological
aspects of hierarchical modeling can be found in
\citet{royle_dorazio:2008}. We rehash some of those ideas as we need
to but we don't focus too much on the broader framework of
hierarchical modeling.  We do, however, provide some background
material on inference strategies and, especially, Bayesian analysis
using the {\bf BUGS} variants {\bf JAGS} and {\bf WinBUGS}.



%RS: This should be block 2: Themes or scope of this book
\section{Scope of this Book}

In this book, we try to achieve a broad methodological scope from
basic closed population models for inference about population density,
movement, space usage and resource selection, on up to open population
models for inference about vital rates such as survival and
recruitment.  Much of the material is a synthesis of recent research
but we also expand SCR models in a number of useful directions,
including to accommodate unmarked individuals
(Chapt. \ref{chapt.scr-unmarked}), use of telemetry information
(Chapt. \ref{chapt.rsf}), and developing explicit models of landscape
connectivity based on ecological or least-cost distance
(Chapt. \ref{chapt.ecoldist}), and many other new topics that have
only recently appeared in the literature, or not at all.  Our intent
is to provide a comprehensive resource for ecologists interested in
understanding and applying SCR models to solve common problems faced
in the study of populations.  To do so, we make use of hierarchical
models \citep{royle_dorazio:2008}, which allow extraordinary
flexibility in accommodating many types of capture-recapture data. We
present many example analyses, of real and simulated data using
likelihood-based and Bayesian methods---examples that readers can
replicate using the code presented in the text and the resources made
available on-line and in our accompanying {\bf R} package {\tt
  scrbook}.

Although we aim to reach a broad audience, at times we go into details
that may only be of interest to advanced practitioners who need to
extend capture-recapture models to unique situations.  We hope that
these advanced topics will not discourage those new to these methods,
but instead out intent is to allow readers to advance their own
understanding and become less reliant on restrictive tools and
software.
A number of conceptual and methodological
themes unify the main topical coverage of this book, and those are:

% XXXX Most of these are technical and/or non-ecological. What about
% something like this:
% (1) Spatial ecology: much (half?) of ecology is about spatial variation in
% density and the mechanisms (e.g. habitat selection, movement) that
% determine this variation. The other half is temporal variation,
% which we also cover, but in less depth.
% (2) Spatial observation error: Observation error is omnipotent in
% ecology, especially in the study of free-ranging vertebrates, and in
% fact the entire 100+ year history of capture-recapture studies have
% been devoted to estimating key demographic parameters in the
% presence of observation error because we simply cannot observer all
% the individuals in a population, plot, whatever... And we can't know
% their fates even if we mark them all. What has been missing in most
% of the capture-recapture methods is an acknowledgement of the
% spatial context of sampling and the fact that capture (or detection)
% probability will virtually always be a function of the distance
% between traps and animals (or their home ranges).
% (3) Hierarchical modeling are the perfect tool for modeling spatial
% processes, especially those of the type covered in this book, where
% one process (the ecological process) is conditionally
% related to another (the capture process). We make use of HMs
% throughout this book, and we do so using both Bayesian and classical
% (frequentist) modes of inference. These tools allow us to mold our hypotheses into
% probability models which can be used for description, testing, and
% prediction. ...

%RS: I like Richard's suggestions. In our meeting we had a 4th item which is model implementation. 

\begin{itemize}
\item[(1)] Hierarchical modeling. We develop hierarchical models
  consisting of explicit models for both the observation process and
  the underlying ``ecological process'' which describes the
  organization of individuals in space.

\item[(2)] Spatial processes in capture-recapture. We emphasize the
  linkage of capture-recapture data to underlying ecological processes
  including density or distribution of individuals, space usage,
  resource selection and movement.

\item[(3)] Formal inference using both classical (frequentist,
  likelihood-based) and Bayesian methods. We often emphasize Bayesian
  analysis because this allows us to focus the technical formulation
  of models, and spatial capture-recapture is mainly concerned with
  modeling random effects and estimating functions of random
  effects. However, we also explore likelihood methods using existing
  software such as the {\bf R} package \mbox{\tt secr}
  \citep{efford:2011}, as well as development of custom solutions
  along the way.

\item[(4)] In developing Bayesian analyses of SCR models, we emphasize
  the use of the {\bf BUGS} language for describing models. The {\bf
    BUGS} langage emphasizes the syntactic description of the
  essential assumptions of models in a special kind of pseudo-code
  language, which is used in software ({\bf WinBUGS}, {\bf JAGS}, {\bf
    OpenBUGS}) to devise Markov chain Monte Carlo (MCMC) algorithms
  for Bayesian analysis of models. The {\bf BUGS} language focuses
  your thinking on model development and lets you develop an
  understanding of models at the level of their basic assumptions and
  structure.  Despite our focus on describing models in {\bf BUGS}, we
  also show readers how to devise their own MCMC algorithms for
  Bayesian analysis of SCR models, which can be convenient (even
  necessary) in some practical situations.

%\item[(4)] Data augmentation -- dealing with the fact that population
%  size, $N$, is unknown is a challenging technical problem in
%  capture-recapture models. We confront this problem in almost every
%  chapter of this book. To deal with it we use a technical device
%  called {\it data augmentation} which is extremely useful for
%  analysis of capture-recapture models that are specified
%  ``conditional on $N$'' \citep{royle_etal:2007}.
\end{itemize}


Altogether, these different conceptual and methodological elements
provide for a formulation of SCR models that essentially renders them
as variations of generalized linear mixed models (GLMMs). This in a
sense makes them consistent with many important methodologies used in
ecology (e.g., see \citet{zuur_etal:2009, kery_etal:2010}), and
because of the connection with standard modeling concepts, we believe
that the material presented in this book can be understood and used by
most ecologists with some modeling experience.


%RS: Thinking about this next section, I think it might could be merged with the previous one (Scope/themes), and be part of the introduction before we lay out the 4 major themes of the book.

\section*{Motives/Point of View}

What is the approach or philosophy -- the overriding set of ideas that
motivates how we developed and organized conent?

In our experience, students in ecology
and even many established scientists sometimes have a difficult time
distinguishing betwee what they need to do from how to do it.  They
cannot distinguish clearly (either conceptually or actually) the
difference between the model for their data, and the actual procedure
of how to estimate parameters of that model, or make predictions -
ie., how to do the calculations. Sometimes this issue raises itself in
an email from some hapless grad student wondering ``what is the right
statistical test for this type of data?''  In a sense it is this view
that drives our approach to developing elements of this book.

In contemporary statistical ecology, models and methods are sometimes
obscured by named procedures often that are completely uninformative,
the technical details of which hide in obscurity in some black boxes
such as MARK, PRESENCE, DISTANCE, etc., known only by the few
specialist experts in the field.  {\bf XXXX I think we should say
  something to the extent that these black boxes are very useful but
  their user-friendlyness comes with the risk of abuse - the risk that
  you can run an analysis and get a result without really knowing what
  you're doing XXXXXX}  While it is sometimes convenient to refer to a
type or class of models by a name (logistic regression or even ``model
Mh'') in order to emphasize a broad concept or methodological area,
this is only useful if the fundamental statistical and mathematical
structure underlying that name is clear. As such, we try to focus on
model development and keep the model development distinct from how to
combine our data with the model to produce estimates and so forth. We
talk a lot about hypothetical data we wish we could observe - complete
data sets - data sets as if $N$ were known, etc.. We talk about the
model in precise terms and then break down various ways for analyzing
the model either using likelihood methods or Bayesian methods or some
black-box that does one or the other.



%RS: Accroding to our meeting this is block 3 - software/computing (and I think the stuff on hierarchical models should go here, as a more technical part). We should mention the scrbook pacakge here
\section*{Computing}

We rely heavily on data processing and analysis in the {\bf R}
programming language which, by now, is something that very many
ecologists not only knows about, but probably uses extensively. We
adopt {\bf R} because
it is free; because it has a large community that constantly develops
code for new applications; and because it gives us great flexibility.
There are some great books out there including Bolker (XXXX) and XX
other XXXX XXXXX. We use a large number of {\bf R} packages in our
analyses, we address them in Appendix 1, and moreover, we provide an
{\bf R} package containing scripts and functions for all of our
analyses (see below).

We rely heavily on the various implementations of the {\bf BUGS}
language including {\bf WinBUGS} \citep{lunn_etal:2000}, {\bf JAGS}
\citep{plummer:2003}.  Mostly we are transitioning to use of {\bf
  JAGS} but we still use {\bf WinBUGS}.  WinBUGS is not in active
development anymore, but JAGS is. Sometimes models run better or mix
better in one or the other. We don't have much experience with {\bf
  OpenBUGS} \citep{thomas_etal:2006}. We really like the {\bf BUGS}
language, not merely as a computational device for fitting models but
also because it emphasizes understanding of what the model is and
fosters understanding how to build models -- as our good friend and
colleague Marc K\'{e}ry says, ``BUGS frees the modeler in you.''
(XXXXX direct citation for this would be nice XXXXX).
%However, in addition to using the
%{\bf BUGS} language and its various implementations, we also develop our own
%{\bf R} code both for doing MCMC.
%and maximum likelihood, for which we also use the R
%package \mbox{\tt secr} \citep{efford:2011}.
%In addition, we have
%created an {\bf R} package to go with this book, \mbox{\tt scrbook},
%which contains the data sets, {\bf R} and {\bf BUGS} scripts, and {\bf
%  R} code for doing summary analyses, and some likelihood and MCMC
%functions written solely in {\bf R}.
While we mostly the the {\bf BUGS} implemenations, we do a limited
amount of developing our own custom MCMC algorithms (see
Chapt. \ref{chapt.mcmc}) which we think is really handy for certain
problems where {\bf BUGS}/{\bf JAGS} fail or prove to be inefficient.
This can be really handy because you can exploit large Linux or
windows clusters to distribute your computing efficienctly.

We don't just do Bayesian analysis in this book! Here and there you'll
find a fair amount of likelihood analysis, and we have a chapter that
provides the conceptual and technical background for how to do this,
and several chapters use likelihood methods exclusively. We use the R
package \mbox{\tt secr} \citep{efford_etal:2009euring} for many
analyses and we think people should use this because it is polished,
easy to use, fairly general and has the usual {\bf R} summary methods,
and considerable capability for doing analysis from start to
finish. In some chapters we discuss models that we have to use
likelihood methods for, but which are not implemented in \mbox{\tt
  secr} (e.g., Chapts. \ref{chapt.rsf} \ref{chapt.ecoldist}). This is
a good example of why it is useful to understand the principles and to
be able to implement these methods yourself.

\section*{The R package scrbook}

As we were developing content for the book it was obvious that it
would be useful if the {\bf R}/{\bf BUGS} scripts and data were availble for
readers to reproduce the analyses and also to modify so that they can
do their own analysis.  Almost every analysis we did is included as an
{\bf R} script in the \mbox{\tt scrbook} package.
The {\bf R} package should be a very dynamic thing, as we plan to
continue to update and expand it.

The purpose of the package is not meant to be general-purpose and
flexible software for doing SCR models but, rather, a set of examples
and templates to see how specific things are done. Code can be used
by the reader to develop methods tailored to his/her situation, or
possibly even more general methods.  Because we use so many different
software packages and computing platforms, we think its impossible to
put all of what is convered in this book into a single integrated
package.  The \mbox{\tt scrbook} package is for educational purposes
and not for production or consulting work.

%RS: This is the last, 4th, block of the preface - organization of the book. I think some of the stuff that's written in the 'Where did we go wrong' part could go here - the justification why we spend 100 pages on intro material. Maybe some of the 'Ideal reader' stuff should go here, too, not in the Modeling section.
\section*{Organization of This Book}

We introduce Bayesian inference in some detail because people are less
likely to have had a class in that and we also wanted to produce a
stand-alone product.  We also do likelihood analysis of many models
and we provide an introduction to the relevant elements of likelihood
analysis in Chapt. \ref{chapt.mle}, and the implementation of SCR
models in the package \mbox{\tt secr} \citep{efford_etal:2009euring}.

We didn't think to write a 550 page book on SCR models because there
wasn't even that much material to work when when the project was
started back in the early part of 2009. At that time, we envisioned
maybe 250 pages or so. But during the project great and new things
were produced, and we developed new models and concepts taht led to
new material for the book, including models of resource selection,
landscape connectivity, and methods for dealing with unmarked
individuals. There are at least 10 chapters in the book that we
couldn't have thought about 4 years ago. Actually, more ideas kept
coming up and we ended up not even being able to cover all of them
extensivvely in this book; so we also, in places, provide ideas of how
SCR models could be developed further.

In the following chapters we develop a comprehensive synthesis and
extension of spatial capture-recapture models.  Roughly the first
third of the book is introductory material. In Chapt. \ref{chapt.glms}
we provide the basic analysis tools to understand and analyze SCR
models, namely generalized linear models (GLMs) with random effects,
and their analysis in {\bf R} and {\bf WinBUGS}.  Because SCR models
represent extensions of basic closed population models, we cover
ordinary closed population models in Chapt. \ref{chapt.closed}
wherein, along with Chapts. \ref{chapt.scr0} and
\ref{chapt.poisson-mn} \footnote{might ought to put Modeling Encounter
  Probability as chapter 5 instead}, provides the basic introduction
to capture-recapture models and their spatial extension.... this
covers XXXXX.  observation models related to the type of encounter
device (Chapt. \ref{chapt.poisson-mn}), models for encounter
probability (Chapt. \ref{chapt.covariates}). In the middle part of the
book we extend capture-recapture to SCR models and discuss a number of
different conceptual and technical topis including tools for
likelihood inference (Chapt. \ref{chapt.mle}), sampling design
(Chapt. XXXXX) and analysis of model fit and model selection
(Chapt. \ref{chapt.gof}).

The last part of the book, which is really about 40\% or more by page
count, is ``advanced'' stuff. This covers a chapter on developing your
own MCMC algorithsm for SCR models. We do this because many advanced
models require you to do this and we think much of the future of SCR
will require MCMC methods.  We also have a number of chapters on
spatial modeling aspects related to SCR including modeling space usage
or resource selection (Chapt \ref{chapt.rsf}), modeling landscape
connectivity (Chapt. \ref{chapt.ecoldist}), and modeling spatial
variation in density.  We cover open population models
(Chapt. \ref{chapt.open}), a bunch of stuff on unmarked individuals
either entirely (Chapt. \ref{chapt.scr-unmarked}) or partially marked
samples (Chapt. \ref{chapt.partialID}), a chapter on search-encounter
methods XXXX, a chapter on multi-session models XXXXXX (probably move
this to chapter 10?)


\begin{comment}
\section*{Where did we go wrong?}

The latex \mbox{\tt ~} is about useless as tits on a boar.

Ok, there will be some obvious, even embarrasing, problems that arise
once the book hits the street. Those are unintnentional. Besides
those, there are some deficiencies with the book that we can identify
{\it before} the book hits the street. For one thing, it's a little
bit long-winded. We added some introductory/background material that
covers more than 100 pages. We don't really get to SCR models until
around page 130.  So you have some reading to do, or else skip over
it, or whatever. Our intent was to provide all of the material you
need in one place. When you start outlinging or enumerating the
``requirements'' and you flesh each of those out a little bit, then
you have basically a short chapter on what you need to know in order
to get through SCR models. But then that short chapter is just
referencing a bunch of other material and so you start fleshing it out
a little bit more, and the next thing you know you have 3 chapters of
background content that amounts to 100 or more pages.  So, there you
have it.

As a second thing, we're tying together data analysis, R scripts, and
BUGS code from 4 different developers and we find, every day, certain
inconsistencies with notation and so on. We hope to have that all
nailed down in the final, edited, version of the book. But some of the
R package material will appear inconsistent with the book because the
package is dynamic, but the book is static. Or, in some cases, the
package development is lagging the printed book page. (this should
soon be fixed!).
\end{comment}





