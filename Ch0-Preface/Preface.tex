\begin{comment}
\begin{verbatim}
"A preface or foreword deals with the genesis, purpose, limitations,
and scope of the book and may include acknowledgments of indebtedness;

an introduction deals with the subject of the book, supplementing and
introducing the text and indicating a point of view to be adopted by
the reader. The introduction usually forms a part of the text [and the
text numbering system]; the preface does not." (In other words, the
arabic numbering of the book (1,2,3) starts with the introduction, if
there is one. The other front matter takes i, ii, iii, etc.)
\end{verbatim}
\end{comment}

\begin{comment}
RS:
According to what we discussed a few days ago, the 4 big blocks of the Preface should be
1. Why are we writing this book?
2. What are the themes?
3. Software/computing/modeling - something about the practical aspects
4. Organization of the book
I tried commenting on the below material referring to this overall organization we agreed on.
\end{comment}

%RS:This is the 'Why' part of the preface.
%I think Richard made this comment further down, but we should stress more the importance of 'space', both in ecology and in sampling

% XXXX RC says: I like the opening paragraph, but it doesn't have much
% kick to it. I tried to hype it up a little, but wasn't so
% successful, and so I left this stuff commented out:

%XXX RS: Actually, I like that new first paragraph.
%XXX Beth - I added in Richards.

Capture-recapture (CR) models have been around for well
over a century, and in that time they have served as the primary means of
estimating population size and demographic parameters in ecological
research. The development of these methods has never ceased, and
each year new and useful extensions are presented in ecological and
statistical journals. The seemingly steady clip of development was
recently punctuated with the introduction of spatial capture
recapture (SCR; a.k.a.  spatially explicit capture-recapture models, or SECR)
 models, which in our view stand to revolutionize the
study of animals populations. The importance of this new class of
models is rooted in the fact that they acknowledge that both
ecological processes and observation processes are inherently
spatial. The purpose of this book is to explain this statement, and
to bring together all of the developments over the last few years
while offering researchers practical options for analyzing their own
data using the large and growing class of SCR models.

%There is a long and well documented history of the development and
%application of capture-recapture (CR) models in the ecological
%literature dating back decades (even centuries).  However, there %is
%has been a recent explosion of spatial capture-recapture (SCR) models
%over the last several years, and interest in these models to answer
%questions of basic and applied ecology and conservation has increased
%dramatically over this period.  Our purpose in writing this book is to
%bring together all of the developments over the last few years and
%give researchers practical options for analyzing their own data using
%the large and growing class of SCR models.

CR and SCR have been thought of mostly as ways to ``estimate density''
with not so much of a direct link to understanding ecological
processes. So one of the things that motivated us in writing this book was to elaborate on,
and develop, some ideas related to modeling ecological processes
%XXX RS: Should we say 'spatial ecological processes'? 
(movement, space usage, landscape connectivity) in the context of SCR
models.  The incorporation of spatial ecological processes is where SCR models present an important improvement
over traditional, non-spatial CR models.  SCR models explicitly
describe exposure of individuals to encounter 
%XXX RS: exposure to encounter sounds odd to me; maybe to sampling?
that results from the
juxtaposition of sampling devices or traps with individuals, as well
as the ecologically intuitive link between abundance and area, both of
which are unaccounted for by traditional CR models. By including
spatial processes, these models can be adapted and expanded to
directly address many questions related to animal population and
landscape ecology, wildlife management and conservation.  As such, SCR
models 
stand to revolutionize how researchers study animal populations.  With
such advanced tools at hand, we believe that, but for some specific
situations, traditional closed population models are largely obsolete,
except as a conceptual device.

So, while we do have a lot of material on density estimation in this
book -- this is problem \# 1 in applied ecology -- we worked hard to
cover a lot more of the spatial aspect of population analysis as
relevant to SCR.  There are a lot of books out there that cover
spatial analysis of population structure which are more theoretical or
mathematical, and there are a lot of books out there that cover
sampling and estimation, but that are {\it not} spatial. Our book
bridges these two major ideas as much as is possible as of, roughly,
mid-late 2012.

%XXX RS: Somewhere along the line of edits we seem to have lost the statement that 'space' is one of the central aspects of much of ecological research. I think it's worth emphasizing that in a sentence or two. 

% XXXX RC: I commented out the next paragraph because we brought up the
% implementation bit in the first paragraph, and we emphasize it heavily
% below.
\begin{comment}
The 2nd thing that %really
motivated this book was our interest in
analyzing SCR data, teaching people to analyze SCR data, and
 therefore to provide a
framework for {\it implementation} using
\textbf{R}/\textbf{WinBUGS}/\textbf{JAGS} and using both Bayesian and
classical modes of inference.  To that extent, we provide a survey of
the models that are available now, and even some new ones, and
demonstrate how they can be applied in practice. Some of the material
is really new, only just now appearing in the peer-reviewed
literature, and some of it we even developed along the way in response to
gaps in the literature (and gaps in this book's outline!).
\end{comment}


\begin{comment}
%XXX I'm not sure what to do with this section, I kind of don't like it.
XXX andy sez: i'm ok leaving it out............

I'm not sure how it fits into our preface.    The second paragraph is
okay....

What is the approach or philosophy -- the overriding set of ideas that
motivates how we developed and organized content?

In our experience, students in ecology
and even many established scientists sometimes have a difficult time
distinguishing what they need to do from how to do it.  They
cannot articulate clearly (either conceptually or actually) the
difference between the model for their data, and the actual procedure
of how to estimate parameters of that model, or make predictions -
i.e., how to do the calculations. Sometimes this issue raises itself in
an email from some hapless grad student wondering ``what is the right
statistical test for this type of data?''  In a sense it is this view
that drives our approach to developing elements of this book.

In contemporary statistical ecology, models and methods are sometimes
obscured by named procedures often that are completely uninformative,
the technical details of which hide in obscurity in some black boxes
such as MARK, PRESENCE, DISTANCE, etc., known only by the few
specialist experts in the field.  {\bf XXXX I think we should say
  something to the extent that these black boxes are very useful but
  their user-friendlyness comes with the risk of abuse - the risk that
  you can run an analysis and get a result without really knowing what
  you're doing XXXXXX}  While it is sometimes convenient to refer to a
type or class of models by a name (logistic regression or even ``model
Mh'') in order to emphasize a broad concept or methodological area,
this is only useful if the fundamental statistical and mathematical
structure underlying that name is clear. As such, we try to focus on
model development and keep the model development distinct from how to
combine our data with the model to produce estimates and so forth. We
talk a lot about hypothetical data we wish we could observe - complete
data sets - data sets as if $N$ were known, etc.. We talk about the
model in precise terms and then break down various ways for analyzing
the model either using likelihood methods or Bayesian methods or some
black-box that does one or the other.

\end{comment}


\section*{Themes of this Book}

In this book, we try to achieve a broad
conceptual and
methodological scope from
basic closed population models for inference about population density,
movement, space usage and resource selection, on up to open population
models for inference about vital rates such as survival and
recruitment.  Much of the material is a synthesis of recent research
but we also expand SCR models in a number of useful directions,
including to
the development of explicit models of landscape
connectivity based on ecological or least-cost distance
(Chapt. \ref{chapt.ecoldist}),
use of telemetry information to model resource selection with SCR
(Chapt. \ref{chapt.rsf}), and to
 accommodate unmarked individuals
(Chapt. \ref{chapt.scr-unmarked}),
and many other new topics that have
only recently, or not yet at all, appeared in the literature.  Our intent
is to provide a comprehensive resource for ecologists interested in
understanding and applying SCR models to solve common problems faced
in the study of populations.  To do so, we make use of hierarchical
models \citep{royle_dorazio:2008}, which allow great
flexibility in accommodating many types of capture-recapture data. We
present many example analyses, of real and simulated data using
likelihood-based and Bayesian methods---examples that readers can
replicate using the code presented in the text and the resources made
available on-line and in our accompanying {\bf R} package {\tt
  scrbook}.

% XXXX RC: I moved this down
% Although we aim to reach a broad audience, at times we go into details
% that may only be of interest to advanced practitioners who need to
% extend capture-recapture models to unique situations.  We hope that
% these advanced topics will not discourage those new to these methods,
% but instead out intent is to allow readers to advance their own
% understanding and become less reliant on restrictive tools and
% software.

%A number of conceptual and methodological themes unify the
%main topical coverage of this book, and those are:

The conceptual and methodological themes of this book can be
summarized as follows:

\begin{itemize}
\item[(1)]  Spatial ecology: Much of ecology is about spatial variation in
  processes (e.g., density) and the mechanisms (e.g., habitat selection, movement) that
  determine this variation. Temporal variation is also commonly of interest and
  we cover this as well, but in less depth.

\item[(2)] Spatial observation error: Observation error is omnipotent in
  ecology, especially in the study of free-ranging vertebrates, and in
  fact the entire 100+ year history of capture-recapture studies have
  been devoted to estimating key demographic parameters in the
  presence of observation error because we simply cannot observe all
  the individuals that are present,
  % in a population, plot,whatever... A
  and we can't know
  their fates even if we mark them all. What has been missing in most
  of the capture-recapture methods is an acknowledgment of the
  spatial context of sampling and the fact that capture (or detection)
  probability will virtually always be a function of the distance
  %XXX RS: Maybe I'm being picky here, but maybe we should replace distance with a vaguer term, like spatial relationship, or something. 
  between traps and animals (or their home ranges).

\item[(3)] Hierarchical modeling: Hierarchical models (HM) are the
  perfect tool for modeling spatial processes, especially those of the
  type covered in this book, where one process (the ecological
  process) is conditionally related to another (the observation
  process). We make use of HMs throughout this book, and we do so
  using both Bayesian and classical (frequentist, likelihood-based)
  modes of inference. These tools allow us to mold our hypotheses into
  probability models which can be used for description, testing, and
  prediction.

\item[(4)] Model implementation: We consider proper implementation of
  the models to be very important throughout the book.  We explore
  likelihood methods using existing software such as the {\bf R}
  package \mbox{\tt secr} \citep{efford:2011}, as well as development
  of custom solutions along the way. In Bayesian analyses of SCR
  models, we emphasize the use of the {\bf BUGS} language for
  describing models.  We also show readers how to devise their own
  MCMC algorithms for Bayesian analysis of SCR models, which can be
  convenient (even necessary) in some practical situations.

\end{itemize}

Altogether, these %different conceptual and methodological
elements
provide for a formulation of SCR models that will allow the reader to
learn the fundamentals of standard modeling concepts and ultimately
implement complex hierarchical models.
We also believe that
while the focus of the book is spatial capture-recapture (that is,
in fact, the title), the reader will be able to apply the general
principles that we cover in the introductory material (e.g., principles
of Bayesian analysis) and even the advanced material (e.g., building
your own MCMC algorithm) to a broad array of topics in general ecology
and wildlife science.
Although we aim to reach a broad audience, at times we go into details
that may only be of interest to advanced practitioners who need to
extend capture-recapture models to unique situations.  We hope that
these advanced topics will not discourage those new to these methods,
but instead
%out intent is to
will
allow readers to advance their own
understanding and become less reliant on restrictive tools and
software.

%...................
%[XXXXXXXXXXXXXXX NOTE SURE HOW TO END THIS HERE XXXXXXXXXXXXXXXXXXXX]
% Beth - I tried above to end this section.  Not sure if it works.

 % essentially renders them
 % as variations of generalized linear mixed models (GLMMs). This in a
 % sense makes them consistent with many important methodologies used in
% ecology (e.g., see \citet{zuur_etal:2009, kery_etal:2010}), and
% because of the connection with standard modeling concepts, we believe
% that the material presented in this book can be understood and used by
%most ecologists with some modeling experience.

\section*{Computing}

We rely heavily on data processing and analysis in the {\bf R}
programming language, which by now is something that many ecologists
not only know about, but use frequently.  We adopt {\bf R} because it
is free, has a large community that constantly develops code for new
applications, and it gives the user flexibility in data processing and
analyses.  There are some great books out there, including %(e.g.,
\citet{venables_ripley:2002}, \citet{bolker:2008} and \citet{zuur_etal:2009}, and we encourage those
new to \R~to read through the manuals that come with the software.
We use a number of {\bf
  R} packages in our analyses, which are described in Appendix 1, and
moreover, we provide an {\bf R} package containing the scripts and
functions for all of our analyses (see below).

We also rely on the various implementations of the {\bf BUGS}
language including {\bf WinBUGS} \citep{lunn_etal:2000} and {\bf JAGS}
\citep{plummer:2003}.  Because {\bf WinBUGS} is not in active development
any more, we are transitioning to mainly using {\bf JAGS}.  Sometimes
models run better or mix better in one or the other. As a side note,
we don't have much experience with {\bf OpenBUGS}
\citep{thomas_etal:2006}, but our code for {\bf WinBUGS} should run
just the same in {\bf OpenBUGS}. The {\bf BUGS} language provides not
only a computational device for fitting models but it also emphasizes
understanding of what the model is and fosters understanding of how to
construct models.
As our good colleague Marc K\'{e}ry wrote
\citep[][p. 30]{kery:2010}
``{\bf BUGS} frees the modeler in you.''
While we mostly use {\bf BUGS} implementations,
we do a limited amount of developing our own custom MCMC algorithms
(see Chapt. \ref{chapt.mcmc}) which we find very helpful for certain
problems where {\bf BUGS}/{\bf JAGS} fail or prove to be inefficient.

You will find a fair amount of likelihood analysis throughout the book,
 and we have a chapter that
provides the conceptual and technical background for how to do this,
and several chapters use likelihood methods exclusively. We use the
\R~package \mbox{\tt secr} \citep{efford_etal:2009euring} for many
analyses, and we think people should use this tool because it is polished,
easy to use, fairly general, has the usual {\bf R} summary methods,
and has considerable capability for doing analysis from start to
finish. In some chapters we discuss models that we have to use
likelihood methods for, but which are not implemented (at the time
when we wrote this book) in \mbox{\tt
  secr} (e.g., Chapts. \ref{chapt.ecoldist}, \ref{chapt.rsf}). This is
a good example of why it is useful to understand the principles and to
be able to implement these methods yourself.

\subsection*{The R package {\tt scrbook}}

As we were developing content for the book it became clear that it
would be useful if the tools and data were available for
readers to reproduce the analyses and also to modify so that they can
do their own analysis.  Almost every analysis we did is included as an
{\bf R} script in the \mbox{\tt scrbook} package.
The {\bf R} package will be very dynamic, as we plan to
continue to update and expand it.

The %purpose of the
package is not meant to be general-purpose, % and
flexible software for doing SCR models but, rather, a set of examples
and templates %to see
illustrating how specific things are done. Code can be used
by the reader to develop methods tailored to his/her situation, or
possibly even more general methods.  Because we use so many different
software packages and computing platforms, we think it's impossible to
put all of what is covered in this book into a single integrated
package.  The \mbox{\tt scrbook} package is for educational purposes
and not for production or consulting work.


\section*{Organization of This Book}

We expect that readers have a basic understanding of statistical
models and classical inference (What is frequentist inference? What is
a likelihood? Generalized linear model? Generalized linear mixed
model?), Bayesian analysis (what is s a prior distribution? and a
posterior distribution?), and have used the {\bf R} programming
environment and maybe even %a little bit of Bayesian computation (MCMC
%and perhaps
the \textbf{BUGS} language.  The ideal candidate for
reading this book has basic knowledge of these topics; however, we do
provide introductory chapters on the necessary components which we
hope can serve as a brief and cursory tutorial for those who might
have only limited technical knowledge, e.g., many biologists who
implement field sampling programs but do not have extensive experience
analyzing data.

To that extent, we introduce Bayesian inference in some detail because
we think readers are less likely to have had a class in that and we
also wanted to produce a stand-alone product.  Because we do
likelihood analysis of many models, there is an introduction to the
relevant elements of likelihood analysis in Chapt. \ref{chapt.mle},
and the implementation of SCR models in the package \mbox{\tt secr}
\citep{efford:2011}.  
%XXX RS: Above we use a different citation for the package (efford:2011); should prob. standardize.
Our intent was to provide all of the
material you need in one place, but naturally this led to one of the
deficiencies with the book: it's a little bit long-winded, especially in the first, introductory part.
% XXX RS: I think after the introduction the book isn't long-winded, so added this little modifier. Should we also add a sentence like 'This should not discourage you/the reader and if you alrady have extensive background in the basics of statistical inference, you can skip straight ahead to the specifics of SCR modeling, starting with Chapter 4.

In the following chapters we develop a comprehensive synthesis and
extension of spatial capture-recapture models.  Roughly the first
third of the book is introductory material. In Chapt. \ref{chapt.glms}
we provide the basic analysis tools to understand and analyze SCR
models, namely generalized linear models (GLMs) with random effects,
and demonstrate their analysis in {\bf R} and {\bf WinBUGS}.  Because SCR models
represent extensions of basic %closed population
CR models, we cover
ordinary closed population models in Chapt. \ref{chapt.closed}
%wherein,
which,
along with Chapts. \ref{chapt.scr0}, \ref{chapt.covariates}
and \ref{chapt.poisson-mn}, provides the basic introduction to
capture-recapture models and their spatial extension from a Bayesian
point of view.  
% XXX RS: So this I find a little confusing; SCR is already part of the basic intro part, but then it says that in the middle part we extend CR to SCR
In the middle part of the book, we extend
capture-recapture to SCR models and discuss a number of different
conceptual and technical topics including tools for likelihood
inference (Chapt. \ref{chapt.mle}), analysis of model fit and model
selection (Chapt. \ref{chapt.gof}), and sampling design
(Chapt. \ref{chapt.design}).

The 3rd chunk of the book covers more advanced SCR models.  We have a
number of chapters on spatial modeling aspects related to SCR,
including modeling spatial variation in density
(Chapt. \ref{chapt.state-space}, modeling landscape connectivity or
``ecological distance'' using SCR models
(Chapt. \ref{chapt.ecoldist}), and modeling space usage or resource
selection (Chapt \ref{chapt.rsf}), which includes material on
integrating telemetry data into SCR models.  After this there are a
series of 3 chapters that involve some elements of modeling spatially
or temporally stratified populations.  We cover Bayesian multi-session
models in Chapt. \ref{chapt.hscr}, what we call
``search-encounter'' models in Chapt. \ref{chapt.search-encounter}
and, finally, fully open models involving movement or population
dynamics in Chapt. \ref{chapt.open}.  The reason we view the
search-encounter models chapter, Chapt. \ref{chapt.search-encounter},
as a prelude to fully open models is that these models apply to
situations where we observe the animal locations ``unbiased by fixed
sampling locations'' -- so we get to observe clean measurements of
movement outcomes. When this is possible, we can resolve parameters of
explicit movement models free of those that involve encounter
probability.  For example, one such models has two ``scale''
parameters: $\sigma$ that determines the rate of decay in encounter
probability from a sampling point or line and $\sigma_{move}$
% XXXX RC: could we call this $\tau$ instead?
which is
the standard deviation of movements about an individuals activity
center.
% XXXX I don't quite see the connection between
% search-encounter models and open models, since we assume movement in
% all of the closed models as well, and we rarely model explicit
% movement outcomes in open models.

The final conceptual 4th of this book is what we call
``super-advanced stuff.''% -- this is material we barely understand
%ourselves and, in all likelihood, our understanding of this will
%improve greatly right after the book becomes published.
We include a
chapter on developing your own MCMC algorithms for SCR models because
many advanced models require you to do this, or can be run more efficiently than in the {\bf BUGS} language, and we thought some
readers would appreciate a practical introduction to MCMC for ecologists.
%and we think much of the future of SCR will require MCMC methods.
Following the MCMC chapter, we have a number
of topics related to unmarked individuals 
(Chapt. \ref{chapt.scr-unmarked}) or partially marked populations
(Chapt. \ref{chapt.partialID}). This last section of the book contains some research areas that we are
currently developing but lays the foundation for 
further development of novel extensions and applications.
%XXX RS: I'd add something at the end of this paragraph along the line of: 'Some of the material in this last section of the book is still in a somewhat initial phase of development and provides opportunity for the advanced reader to think up novel extensions and applications.' Sounds less like we don't know what we're doing but conveys that some of this stuff might still change. Stupid SMR...

%Beth - added suggested sentence

When this project was begun in 2008, the idea of producing a 550 page
book would have been unimaginable -- there wasn't that much material
to work with.  Optimistically, there was maybe a 250 page monograph that
could have been squeezed out of the literature.  But, during the project,
great and new things appeared in the literature, and we developed new
models and concepts ourselves, in the process of writing the book. This includes models of resource selection,
landscape connectivity, and methods for dealing with unmarked
individuals. There are at least 10 chapters in the book that we
couldn't have thought about 5 years ago. We hope that the result is a
timely summary and a lasting resource.



\begin{comment}
\section*{Where did we go wrong?}

The latex \mbox{\tt ~} is about useless as tits on a boar.

Ok, there will be some obvious, even embarrassing, problems that arise
once the book hits the street. Those are unintentional. Besides
those, there are some deficiencies with the book that we can identify
{\it before} the book hits the street. For one thing, it's a little
bit long-winded. We added some introductory/background material that
covers more than 100 pages. We don't really get to SCR models until
around page 130.  So you have some reading to do, or else skip over
it, or whatever. Our intent was to provide all of the material you
need in one place. When you start outlining or enumerating the
``requirements'' and you flesh each of those out a little bit, then
you have basically a short chapter on what you need to know in order
to get through SCR models. But then that short chapter is just
referencing a bunch of other material and so you start fleshing it out
a little bit more, and the next thing you know you have 3 chapters of
background content that amounts to 100 or more pages.  So, there you
have it.

As a second thing, we're tying together data analysis, \R~scripts, and
{\bf BUGS} code from 4 different developers and we find, every day, certain
inconsistencies with notation and so on. We hope to have that all
nailed down in the final, edited, version of the book. But some of the
\R~package material will appear inconsistent with the book because the
package is dynamic, but the book is static. Or, in some cases, the
package development is lagging the printed book page. (this should
soon be fixed!).



%XXXXXXXXXXXXXXXXXXXXXXXXXXXXXXXXXXXXXXXXXXXXXXXXXXXXXXXXXX

%RS: I think this section should be moved to block 3: Software/computing/modeling
\section*{Statistical Modeling}

SCR models are a type of statistical model. Therefore, a reasonable
view of this book is that it is basically an applied statistical
modeling book.

We develop SCR models using a dual inference paradigm involving both
Bayesian and classical methods based on likelihood inference.  We are
not strong advocates of either Bayesian or likelihood-based inference
but rather we think either works in most problems and one or the other
works betters in some cases.  We show examples throughout the book for
which the specific problem solved using Bayesian methods would be
difficult using likelihood methods and vice versa.
% XXXX I really like the previous sentence, but I think we contradict
% it in several places later in the book

xxxxxxxxxxxxxxxxxxxxxxxxxxxxxxxxxxxxxxxxxxxxxxxxxxxxxxxxxxxxx

% XXXX Some of this gets said in next section
In essence, SCR are
simple hierarchical models.  %Hierarchical models are models that
The hierarchical models that we find most relevant in ecology % XXXX added
consist of a formal description of a state process -- a process that
characterizes what the true state of our study system looks like and
usually the process we are ultimately interested in -- and a separate
observation process, which describes how our data were generated
conditional on the state process. This acknowledges that, usually, we
cannot directly observe the state process, but rather have a partially
observed or biased picture of the state we're interested in.  SCR
models condition on a latent variable ${\bf s}$ which we interpret as
an explicit biological process -- the outcome of animals distributing
themselves over a landscape.  Some conceptual and methodological
aspects of hierarchical modeling can be found in
\citet{royle_dorazio:2008}. We rehash some of those ideas as we need
to but we don't focus too much on the broader framework of
hierarchical modeling.  We do, however, provide some background
material on inference strategies and, especially, Bayesian analysis
using the {\bf BUGS} variants {\bf JAGS} and {\bf WinBUGS}.




\end{comment}





