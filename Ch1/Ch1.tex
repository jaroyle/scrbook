\chapter{
Introduction
}
\markboth{Introduction}{}
\label{chapt.intro}

\vspace{.3in}

The spatial structure of populations, and spatial processes that
contribute to population dynamics, are central to applied and
theoretical population ecology, landscape ecology, conservation
biology and many other ecological disciplines.  For examples,
understanding distribution or spatial variation in density are
important in the management of populations; movement, dispersal, space
usage, are important to understanding landscape connectivity, and
density dependence, and spatial interactions among individuals
contribute to population processes.  At the same time, the inherent
spatial aspect of {\it sampling} populations strongly affects apparent
biases in how we observe population structure.  Books have been
written on spatial processes in animal populations
\citep{tilman_kareiva:1997,hanski:1999} and similarly on how to sample
these populations using capture-recapture methods
\citep{seber:1982,williams_etal:2002}.  However, and despite the
central roll of space and spatial processes to both understanding
population dynamics and how we observe populations, a coherent
framework that integrates spatial population processes with the
spatial nature of how we observe populations has not been fully
realized either conceptually or methodologically.

In particular, capture-recapture methods represent perhaps the most
common technique for studying animal populations, and it is growing in
popularity due to recent technological innovations that expand the
utility of such methods to many taxa that until recently could not be
studied by capture-recapture. 
A major deficiency with classical capture-recapture methods is that
they do not admit the spatial structure of either ecological processes
that give rise to encounter history data, nor the spatial aspect of
sampling populations. While many technical limitations of this lack of
spatial explicitness have been recognized for decades (REF XXXXX), 
it has only been very recently \citep{efford:2004} that
spatially explicit capture-recapture methods -- those which accomodate
space -- 
have been developed. 

These spatial capture-recapture (SCR) methods resolve a host of
technical problems that arise in applying capture-recapture methods to
animal populations.  However, SCR models are not merely an extension
of technique but rather they represent an extention in a much more
profound way in that they make ecological processes explicit in the
model -- processes of density, spatial organization of individuals,
movement and space-usage of individuals.  The practical importance of
SCR models is that they allow ecological scientists to study elements
of ecological theory using observational data that exhibits various
biases relating to the observation mechanisms employed. In the context
of capture-recapture, we observe individual encounter history data
from which we can use SCR models to infer where individual live, how
they organize themselves in space and move around in space and how
they interact with other individuals.  Thus, SCR models enable
ecologists to explicitly integrate biological context and theory with
encounter history data.  We therefore believe that SCR models will
enable ecologists to ask questions of space usage, environmental or
landscape effects, social behavior and other important elements of
ecological theory.


\section{The Study of Populations by Capture-Recapture}

Information about abundance or density of populations, and their vital
rates, is fundamental to applied ecology and conservation biology.  To
that end, a huge variety of statistical methods have been devised, and
among these, the most well-developed are collectively known as
capture-recapture (or capture-mark-recapture) methods. For example,
the volumes by \citet{seber:1982}, \citet{borchers_etal:2002},
\citet{williams_etal:2002}, and \citet{amstrup_etal:2005} are largely
synthetic treatments of such methods, and contributions on modeling
and estimation using capture-recapture are plentiful in the
peer-reviewed ecology literature.  
%Capture-recapture techniques have been the number 1 quantiative method
%in studies of animal populations for decades.
%But they apply basically to fish bowl sampling. Does it make sense
%that methods should apply to both?
Capture-recapture techniques make use of individual {\it encounter
  history} data, by which we mean sequences (usually) of 0's and 1's
denoting if an individual was encountered during sampling over a
certain time period. For example, the encounter history ``010''
indicates that this individual was encountered only during the second
of three trapping occasions. As we will see, these data contain
information about encounter probability, abundance, and other
parameters of interest in the study of population dynamics.

Capture-recapture methods have been important in studies of animal
populations for many decades, and their importance is growing
dramatically in response to technological advances that improve our
ability and efficiency to obtain encounter history data. While such
information was obtainable historically only using physical traps,
which capture and retain animals until visited by a biologist who
removes the individual, marks it, or otherwise molests it in some
scientific fashion, new methods do not require physical capture or
handling of individuals.  A large number of passive detection devices
produce individual encounter history data including camera traps
\citep{karanth_nichols:1998, oconnell_etal:2010}, acoustic recording
devices \citep{dawson_efford:2009}, and methods that obtain DNA
samples such as hair snares for bears \citep{gardner_etal:2010jwm},
scent posts for many carnivores \citep{kery_etal:2010}, and related
methods which allow DNA to be extracted from scat, urine or animal
tissue in order to identify individuals.  This book is concerned with
how such data can be used to carry out inference about animal
abundance or density, and other demographic parameters such as
survival, recruitment, and movement using new classes of
capture-recapture models which utilize auxiliary spatial information
related to the encounter process.  We refer to such methods as spatial
capture-recapture (SCR) models\footnote{In the literature the term
  spatially explicit capture-recapture (SECR) is also used}.

As the name implies, the primary feature of SCR models that
distinguishes them from traditional CR methods is that they make use
of the spatial information inherent to capture-recapture studies. That
is, when encounter histories are associated with auxiliary information
on the location of capture, which we refer to as {\it spatial
  encounter histories}, then such information is informative about
spatial processes such as spatial variation in density, movement,
resource selection, and space usage.  As we will see, this allows us
to overcome critical deficiencies of non-spatial methods, and also
%to use capture-recapture data in the form of
%encounter history data to study explicit ecological hypotheses about
%spatial variation and space usage.
%namely, traditional CR methods cannot be used to formally
%estimate density, include of trap-level covariates of density or
%capture probability, or account for heterogeneity in encounter
%probability that results from the spatial organization of animals and
%traps.  Thus, spatial modeling is not just a fun academic exercise; it
%provides a solution to basic problems in the study of animal
%populations that have been acknowledged for more than 70 years
%\citep{dice:1938}.  More important than just providing a resolution to
%some basic technical problems, SCR models provide a framework for
integrate into capture-recapture models explicit ecological hypotheses
related to spatial processes such as movement, space usage, and the
spatial organization of individuals in a population.  This greatly
expands the practical utility and scientific relevance of
capture-recapture methods and studies based on encounter history data.

\section{Scope of this Book}

In this book, we try to achieve a broad methodological scope from
basic closed population models %using a number of distinct observation
%models
for inference about population density, movement, space usage and resource
selection, on up to open population models for inference about vital
rates such as survival and recruitment. %---spatial versions of
%conventional Jolly-Seber models. %A number of conceptual and
%methodological themes unify the main topical coverage of this book, and
%those are:
Much of the material is a synthesis of recent research but we also
expand SCR models in a number of useful directions, including to
accommodate unmarked individuals (Chapt. \ref{chapt.scr-unmarked}),
use of telemetry information (Chapt. \ref{chapt.rsf}), and developing
explicit models of landscape connectivity based on ecological or
least-cost distance (Chapt. \ref{chapt.ecoldist}), and many other new
topics that have only recently appeared in the literature.  Our intent
is to provide a comprehensive resource for ecologists interested in
understanding and applying SCR models to solve common problems faced
in the study of populations.  To do so, we make use of hierarchical
models \citep{royle_dorazio:2008}, which allow extraordinary
flexibility in accommodating many types of capture-recapture data. We
present many example analyses, of real and simulated data using
likelihood-based and Bayesian methods---examples that readers can
replicate using the code presented in the text and the resources made
available on-line and in our accompanying {\bf R} package {\tt
  scrbook}.

Although we aim to reach a broad audience, at times we go into details
that may only be of interest to advanced practitioners who need to
extend capture-recapture models to unique situations.  We hope that
these advanced topics will not discourage those new to these methods,
but instead out intent is to allow readers to advance their own
understanding and become less reliant on restrictive tools and
software.
%Before discussing the specifics of SCR models, we begin with
%an overview of the methods used to collect capture-recapture data, and
%provide a brief summary of traditional non-spatial capture-recapture
%models.
%In this book we present a diverse array of modeling approaches for
%making inference about density and population dynamics using spatial
%capture-recapture data. 
A number of conceptual and methodological
themes unify the main topical coverage of this book, and those are:

\begin{itemize}
\item[(1)] Hierarchical modeling. We develop hierarchical models
  consisting of explicit models for both the observation process and
  the underlying ``ecological process'' which describes the
  organization of individuals in space.

\item[(2)] Spatial processes in capture-recapture. We emphasize the
  linkage of capture-recapture data to underlying ecological processes
  including density or distribution of individuals, 
space usage, resource selection and movement.

\item[(3)] Formal inference using both classical (frequentist,
  likelihood-based) and Bayesian methods. We often emphasize
  Bayesian analysis because this allows us to focus the technical
  formulation of models, and spatial capture-recapture is mainly
  concerned with modeling random effects and estimating functions of
  random effects. However, we also explore likelihood methods using existing
  software such as the {\bf R} package \mbox{\tt secr} \citep{efford:2011}, as well as
  development of custom solutions along the way.

\item[(4)] In developing Bayesian analyses of SCR models, we emphasize
  the use of the {\bf BUGS} language for describing models. The {\bf BUGS}
langage emphasizes the syntactic description of the essential
  assumptions of models in a special kind of pseudo-code language,
  which is used in software ({\bf WinBUGS}, {\bf JAGS}, {\bf OpenBUGS}) to devise Markov
  chain Monte Carlo (MCMC) algorithms for Bayesian analysis of
  models. The {\bf BUGS} language focuses your thinking on model development
  and lets you develop an understanding of models at the level of
  their basic assumptions and structure.  Despite our focus on
  describing models in {\bf BUGS}, we also show readers how
  to devise their own MCMC algorithms for Bayesian analysis of SCR
  models, which can be convenient (even necessary) in some practical
  situations.

%\item[(4)] Data augmentation -- dealing with the fact that population
%  size, $N$, is unknown is a challenging technical problem in
%  capture-recapture models. We confront this problem in almost every
%  chapter of this book. To deal with it we use a technical device
%  called {\it data augmentation} which is extremely useful for
%  analysis of capture-recapture models that are specified
%  ``conditional on $N$'' \citep{royle_etal:2007}.
\end{itemize}

\begin{comment}
Altogether, these different conceptual and methodological elements
provide for a formulation of SCR models that essentially renders them
as variations of generalized linear mixed models (GLMMs). This in a
sense makes them consistent with many important methodologies used in
ecology (e.g., see \citet{zuur_etal:2009, kery_etal:2010}), and
because of the connection with standard modeling concepts, we believe
that the material presented in this book can be understood and used by
most ecologists with some modeling experience.
\end{comment}


\section{Lions and Tigers and Bears, oh my:  Genesis of
Spatial capture-recapture data}

A diverse number of methods and devices exist for producing individual
encounter history data with auxiliary spatial information about
individual locations. Historically, physical ``traps'' have been
widely used to sample animal populations. These include live traps,
leg-hold traps, mist nets, pitfall traps and many other types of
devices. Although these are still widely used, recent technological
advances for obtaining encounter history data non-invasively have made
it possible to study many species that were difficult if not
impossible to study effectively just a few years ago.  These methods
have revolutionized the study of animal populations by
capture-recapture methods, and will lead to their increasing relevance
in the future.  We briefly review some of these here, which we
consider more explicitly in later chapters of this book.

\subsection{Camera trapping}

Considerable recent work has gone into the development of
camera-trapping methodologies. For a historical overview of this
method see \citet{kays_etal:2008} and \citet{kucera_barrett:2011}.
Several recent synthetic works have been published including
\citet{nichols_karanth:2002}, and an edited volume by
\citet{oconnell_etal:2010} devoted solely to camera trapping concepts
and methods. As a method for estimating abundance, some of the
earliest work that relates to the use of camera trapping data in
capture-recapture models originates from Karanth and colleagues
\citep{karanth:1995, karanth_nichols:1998, karanth_nichols:2000}. In
camera trapping studies, cameras are situated along trails or at
baited stations and individual animals are photographed and
subsequently identified either manually by a person sitting behind a
computer, or sometimes now using computational methods. Camera
trapping methods are widely used for species that have unique stripe
or spotting patterns such as tigers \citep{karanth:1995,
  karanth_nichols:1998}, ocelots
\citep{trolle_kery:2003,trolle_kery:2005}, leopards
\citep{balme_etal:2010}, and many other cat species.
% Scientific names
Camera traps are also used for other species such as wolverines
\citep[{\it Gulo gulo}][]{magoun_etal:2011,royle_etal:2011jwm}, and
even species that are less easy to identify uniquely such as mountain
lions \citep{sollmann_etal:inprepjapplecol} and coyotes %add scientific names
(e.g. \citet{kelly_etal:2008}.  We note that even for species that are
not readily identified by pelage patterns, it might be efficient to
use camera traps in conjunction with spatial capture-recapture models
to estimate density (see Chapt.~\ref{chapt.scr-unmarked}).
%, if an initial sample of individuals can be collared
%or tagged in some way so that subsequent encounter by camera-traps can
%yield individual information. In this way, the probability of
%encounter can be estimated from the camera traps based on the
%pre-marked individuals, and this is applied to the frequencies of
%unmarked individuals to estimate density.


\begin{figure}
\begin{center}
\includegraphics[width=5in]{Ch1/figs/wolverinetiger}
\end{center}
\caption{
Left: Wolverine being encounter by a
camera trap ({\it Photo credit: Audrey Magoun}).
Right: Tiger encountered by
camera trap ({\it Photo credit: Ullas Karanth}).
}
\label{fig.wolverinetiger}
\end{figure}

\subsection{DNA Sampling}

DNA obtained from hair, blood or scat is now routinely used to obtain
individual identity and encounter history information about
individuals
 \citep{taberlet_bouvent:1992, 
kohn_etal:1999,
woods_etal:1999,
  mills_etal:2000, schwartz_monfort:2008}.  A common method is based
on the use of ``hair snares'' (Fig. \ref{fig.bearcat}) which are
widely used to study bear populations \citep{woods_etal:1999,
  gardner_etal:2010jwm, garshelis_etal:2006, kendall_etal:2009}.  A
sample of hair is obtained as individuals pass under or around
barbed-wire (or other physical mechanism) to take bait. Hair snares
and scent sticks have also been used to sample felid populations
\citep{garciaalaniz_etal:2010, kery_etal:2010} and other
species. Research has even shown that DNA information can be extracted
from urine deposited in the wild (e.g., in snow; see
\cite{valiere_taberlet:2000}) and as a result this may prove another
future data collection technique where SCR models are useful.

\begin{figure}
\begin{center}
\includegraphics[width=5in]{Ch1/figs/bearcat}
\end{center}
\caption{Left:  Black bear in a hair snare ({\it Photo credit: M. Wegan})
Right: European wildcat loving on a scent stick ({\it Photo credit: Darius
Weber
%, Hintermann \& Weber AG, Ecological Consultancy, Planning \&
%Research, Switzerland
})
}
\label{fig.bearcat}
\end{figure}


\begin{figure}
\begin{center}
\includegraphics[width=5in]{Ch1/figs/beardog}
\end{center}
\caption{Left:
A wildlife research technician for the USDA Forest Service
  holding a male fisher  captured as part of the Kings River Fisher
  Project in the Sierra National Forest, California.
Right: A dog handler surveying for fisher scat in the Sierra National Forest.
{\it Photo credit: Craig Thompson}.}
%, USDA Forest Service,
%Pacific Southwest Research Station.}}
\label{fig.fisherscatdog}
\end{figure}


\subsection{Acoustic surveys}

Many studies of birds \citep{dawson_efford:2009}, bats, and whales
\citep{marques_etal:2009} now collect data using devices that record
vocalizations. When vocalizations can be identified by individual from
multiple recording devices, spatial encounter histories are produced
that are amenable to the application of SCR models
\citep{dawson_efford:2009, efford_etal:2009ecol}. Recently, these
ideas have been applied to data on direction or distance to
vocalizations by multiple simultaneous observers
\citep{borchers_etal:inprep}.

\subsection{Search-Encounter Methods}

There are other methods which don't fall into a nice clean taxonomy of
``devices''. Spatial encounter histories are commonly obtained by
conducting manual searches of geographic sample units such as
quadrats, transects or road or trail networks.  For example, DNA-based
encounter histories can be obtained from scat samples located along
roads or trails or by specially trained dogs \citep{mackay_etal:2008}
searching space (Fig. \ref{fig.fisherscatdog}). This method has been
used in studies of martens, fishers \citep{thompson_etal:2012},
lynx, coyotes, birds \citet{kery_etal:2010}, and many other species.
Similar data structure arises from the use of standard territory or
spot mapping of birds \cite{bibby_etal:1992} or area sampling in which
space is searched by observers to physically capture individuals.
This is common in surveys that involve reptiles and amphibians, e.g.,
we might walk transects picking up box turtles \citep{hall_etal:1999},
or desert tortoises \citep{zylstra_etal:2010}, or search space for
lizards \citep{royle_young:2008}.

These methods don't seem like normal capture-recapture in the
sense that the encounter of individuals is not associated with
specific trap location, but SCR models are equally relevant for
analysis of such data as we discussed in Chapt. \ref{chapt.search-encounter}.


\section{ Historical Context: A Brief Synopsis}

Spatial capture-recapture is a relatively new methodological
development, at least with regard to formal estimation and
inference. However, the basic problems that motivate the need for
formal spatially-explicit models have been recognized for decades and
quite a large number of ideas have been proposed to deal with these
problems. We review some of these ideas here.

\subsection{Buffering}

The standard approach to estimating density even now is to estimate
$N$ using conventional closed population models \citep{otis_etal:1978}
and then try to associate with this estimate some specific sampled
area, say $A$, the area which is contributing individuals to the
population for which $N$ is being estimated. The strategy is to define
$A$ by placing a buffer of say $W$ around the trap array or some
polygon which encloses the trap array. The historical context is
succinctly stated by \citep{obrien:2011} from which we draw this
description:

\begin{quote}
  ``At its most simplistic, $A$ may be described by a concave polygon
  defined by connecting the outermost trap locations ($A_{tp}$;
  \citet{mohr:1947}).  This assumes that animals do not move from
  outside the bounded area to inside the area or vice versa. Unless
  the study is conducted on a small island or a physical barrier is
  erected in the study area to limit movement of animals, this
  assumption is unlikely to be true. More often, a boundary area of
  width $W$ ($A_{w}$) is added to the area defined by the polygon
  $A_{tp}$ to reflect the area beyond the limit of the traps that
  potentially is contributing animals to the abundance estimate
  \citep{otis_etal:1978}. The sampled area, also known as the
  effective area, is then $A(W) = A_{tp} + A_{w}$. Calculation of the
  buffer strip width ($W$) is critical to the estimation of density
  and is problematic because there is no agreed upon method of
  estimating $W$. Solutions to this problem all involve ad hoc methods
  that date back to early attempts to estimate abundance and home
  ranges based on trapping grids
  \citep[see][]{hayne:1949}. \citet{dice:1938} first drew attention to
  this problem in small mammal studies and recommended using one-half
  the diameter of an average home range. Other solutions have included
  use of inter-trap distances \citep{blair:1940, burt:1943}, mean
  movements among traps, maximum movements among traps
  \citep{holdenried:1940, hayne:1949}, nested grids
  \citep{otis_etal:1978}, and assessment lines
  \citep{smith_etal:1971}.''
\end{quote}

The idea of using 1/2 mean maximum distance moved
\citep[``MMDM''][]{wilson_anderson:1985a} to create a buffer strip seems to be the standard
approach even today, presumably justified by Dice's suggestion to use
1/2 the home range diameter, with the mean over individuals of the
maximum distance moved being an estimator of home range
diameter. Alternatively, some studies have used the full MMDM
(e.g. \citet{parmenter_etal:2003}), because the trap array might not
provide a full coverage of the home range (home ranges near the edge
should be truncated) and so 1/2 MMDM should be biased smaller than the
home range radius.  And, sometimes home range size is estimated by
telemetry \citep{karanth:1995, bales_etal:2005}.
% \footnote{Is this correct cite for this?}.  Karanth used some
% technique, it's hard to tell for sure....but he mentions using 1
% female that is collared to estimate effective trap area.  Bales does
% it too, so I added it for extra value!
Use of MMDM summaries to estimate home range radius is usually
combined with an AIC-based selection from among the closed-population
models in \citet{otis_etal:1978} which most often suggests
heterogeneity in detection (model $M_h$).  Almost all of these early
methods were motivated by studies of small mammals using classical
``trapping grids'' but, more recently, their popularity has increased
with the advent of new technologies and especially related to
non-invasive sampling methods such as camera trapping. In particular,
the series of papers by Karanth and Nichols \citep{karanth:1995,
  karanth_nichols:1998, karanth_nichols:2002} has led to fairly
widespread adoption of these ideas.

\begin{comment}
Some of the heuristic ideas based on buffer strips do have some
technical justification in the sense of estimating parameters of an
underlying movement model from observed movements. For example, if we
let $x$ be a random variable indicating movement outcomes of an
individual about its  home range center, and suppose that $x$ has pdf
$g(x)$ then we can understand properties of MMDM by studying the
properties of the sample order statistics, as the maximum distance
moved is the sample range based on a sample of observations of
individual locations.
\end{comment}


%As an illustration, imagine a 1-dimensional
%system where individuals have a home range that amounts to a line
%segment. Then suppose that individual movements are $\mbox{uniform}(0,A)$. It
%can be shown that the sampling distribution of the sample range, R,
%scaled by $A$, say $R/A$ has a beta distribution, $\mbox{beta}(n-1,2)$
%\citep[][p. 235]{casella_berger:2002}
%and thus the diameter of the home range, i.e. $A$, is
%estimated (biasedly) by$ R/( (n-1)/(n+1) )$. For large $n$ we could then
%say that the sample range, i.e., ''maximum distance moved'' seems like a good estimator of home range diameter and, therefore, $R/2$ is an estimator of home-range radius.

%There are a number of technical issues that arise in attempting to use
%such heuristics to justify the application in practice. For one, the
%moments of the sample order statistics are strongly affected by sample
%size, which is typically quite small (per individual encountered) and
%thus, in general, are biased and estimated with variable precision
%depending on sample size. For example, the expected value of MMDM is
%$k(n)*A$ , i.e., the true home range diameter is related to observed
%MMDM by some function of sample size, $k(n)$, that increases to 1. In
%the case where the underlying movement model is uniform, $k(n) =
%(n-1)/(n+1)$ (from above) which motivates a formula for ``adjusting''
%observed MMDM for small sample size. We suspect that many such
%formulae are obtainable depending on the assumed movement distribution
%\citep[e.g., formula 6.16 in][]{obrien:2011}. We might also think about taking
%the {\it maximum} (over individuals) of the maximum distance moved
%because under the specific model considered here (iid uniform) then
%all individuals have the same home range radius. This increases our
%sample size ($n$) and thus the observed sample range should be more
%accurate.


%%Another issue of somewhat more importance (and less easy to
%rectify) is that the {\it observation} of movement outcomes is biased
%by the locations of traps. We cannot observe movements ``off the
%trapping grid'' (or between traps) and thus our observed movements
%will generally be smaller than expected under any particular model
%(the uniform in this case). Moreover, the trap spacing also induces a
%discreteness to the movements that causes a further level of
%approximation based on hypothetical movement
%distributions. Nevertheless, formal analysis of `` buffering''
%strategies based on sample order statistics under specific models for
%movement does at least provide some heuristic support for specific
%choices.  The interested reader should ponder the distribution of the
%sample minimum, maximum and range under other distributions such as a
%normal (and bivariate normal), exponential distribution and perhaps
%others. In addition, contemplate the effect of censoring of movements
%to some arbitrary limit ($B<A$) to mimic bias in observed movement
%outcomes due to a finite trap grid.
\begin{comment}
\subsection{Trapping webs}

The use of buffer strips is conventional and widespread due to the
heuristic appeal of that idea and its easy implementation, but other
conceptual approaches exist to address specific problems motivated by
the spatial context of capture-recapture data. D.R. Anderson came up
with the idea of the ``trapping web'' \citep{anderson_etal:1983} which
does not seem to have been widely adopted in practice.
% although there
%is a clear mathematical formalization to the trapping web design
%\citep{link_barker:1994}.
One reason for this is the design is somewhat restrictive in the sense
that it requires a large number of traps be organized in close
proximity to one another.
\end{comment}

\subsection{Temporary Emigration}

Another intuitively appealing idea is that by \citet{white_shenk:2000}
who discuss ``correcting bias of grid trapping estimates'' by
recognizing that the basic problem is like random temporary emigration
\citep{kendall_etal:1997, chandler_etal:2011} where individuals flip
a coin with probability $\phi$ to determine if they are ``available''
to be sampled or not.  White and Shenk's idea was to estimate $\phi$
from radio telemetry, as the proportion of time an individual spends
in the study area. They obtain the estimated ``super-population'' size by
using standard closed population models and then obtain density by
$\hat{D} = \hat{N}\hat{\phi}/A$ where $A$ is the nominal area of the
trapping array (e.g., minimum convex hull).  A problem with this
approach is that individuals that were radio collared represent a
biased sample i.e., you fundamentally have to sample individuals
randomly from the population {\it in proportion to their exposure to
  sampling} and that seems practically impossible to accomplish. In
other words, ``in the study area'' has no precise meaning itself and
is impossible to characterize in almost all capture-recapture studies.
Deciding what is ``in the study area'' is effectively the same as
choosing an arbitrary buffer which defines who is in the study area
who who isn't.  That said, the temporary emigration analogy is a good
heuristic for understanding SCR models and has a precise technical
relevance to certain models.

Another interesting idea is that of using some summary of ``average
location'' as an individual covariate in standard capture-recapture
models. \citet{boulanger_mclellan:2001} use distance-to-edge (DTE) as
a covariate in the Huggins-Alho type of model. \citet{ivan:2012} uses
this approach in conjunction with an adjustment to the estimated $N$
obtained by estimating the proportion of time individuals are ``on the
area formally covered by the grid'' using radio telemetry.  We do not
dwell too much on these different variations but we do note that the
use of DTE as an individual covariate amounts to some kind of
intermediate model between simple closed population models and fully
spatial capture-recapture models, which we address directly in
Chapt. \ref{chapt.closed}.
%We note that no adjustment
%based on telemetry information is necessary if one were simply to
%place a prior distribution on the individual covariate (which is not
%to say that telemetry data isn't useful, just that the same objective
%can be achieved without telemetry data).

While these procedures are all heuristically appealing, they are also
essentially ad hoc in the sense that the underlying model remains
unspecified or at least imprecisely characterized and so there is
little or no basis for modifying, extending or generalizing the
methods. These methods are distinctly {\it not} model-based procedures
even though they might well be heuristically appealing under specific
movement models. Despite this, there seems to be an enormous amount of
literature developing, evaluating and ``validating'' these literally
dozens of heuristic ideas that solve specific problems, as well as
various related tweeks and tunings of them and really it hasn't led to
any substantive breakthroughs that are sufficiently general or
theoretically rigorous.



%A classical argument in favor of the HA model is
%that it ``doesn't require assumptions about the covariate'' but the
%assumption is explicit in capture-recapture models and thus it is
%natural to attack inference based on the ``joint likelihood''
%\citep{borchers_etal:2002}. This has proven necessary in certain other
%classes of individual covariate models in which natural models arise
%for the individual covariate, such as time-varying individual
%covariates \citep{bonner_schwarz:2006}, or covariates with measurement
%error (e.g., distance sampling; see
%\citet[][ch. 7]{royle_dorazio:2008}).
%The model-based formulation is easily adapted to standard
%individual covariate models as well \citep{royle:2008}. Throughout
%this book we rely heavily on Bayesian inference of the joint
%likelihood, using the formulation based on data-augmentation
%\citep{royle_etal:2007, royle_young:2008, royle:2009} though we also
%discuss the development of likelihood-based inference in chapter 5 and
%apply those methods in some cases.


\section{Capture-Recapture for Modeling Encounter Probability}

We briefly introduced and reviewed a number of classical techniques
for applying non-spatial capture-recapture models to studies of animal
populations. These techniques, such as buffering, are based on many
heuristically appealing ideas.  But these are just heuristics and do
not resolve the essential, basic problem with conventional
(''non-spatial'') capture-recapture models which is that there is no
linkage {\it in the model} between the quantity being informed by the
data (i.e., $N$) and any stated or prescribed ``area'', $A$.


\subsection{An Example: Fort Drum Bear Study}

Here we confront some of the issues that motivate the need for spatial
capture-recapture models by considering analysis of data from a study
design to estimate black bear abundance on the Fort Drum Military
Installation in upstate New York (see Chapt. \ref{chapt.closed} for
more details). The specific data used here are encounter histories on
47 individuals obtained from an array of 38 baited ``hair snares''
during June and July 2006. The study area and locations of the 38 hair
snares are shown in Fig. \ref{fig.hairsnares}.  Barbed wire traps (see
Fig. \ref{fig.bearcat}) were baited and checked for hair samples each
week for eight weeks.  Analysis of these data appears in
\citet{gardner_etal:2010jwm} and we use the data in a number of
analyses in later chapters.

\begin{figure}
\begin{center}
\includegraphics[height=3in]{Ch1/figs/hairsnares}
\end{center}
\caption{Locations of hair snares on Fort Drum, New York, operated
  during the summer of 2006 to sample black bears.}
\label{fig.hairsnares}
\end{figure}

We regarded this data set as a standard capture-recapture data set -
an encounter history matrix with 47 rows and 8 columns with entries
$y_{ik}$, where $y_{ik}=1$ if individual $i$ was captured in sample
$k$ and $y_{ik}=0$ otherwise. There is a standard closed population
model, colloquially referred to as ``model $M_0$'' (see Chapt. \ref{chapt.closed}), which
assumes that encounter probability $p$ is constant for all individuals
and sample periods.  We fitted model $M_0$ to the Fort Drum data using
traditional likelihood methods, yielding the maximum likelihood
estimate (MLE) of $\hat{N} = 49.19$ with an asymptotic standard error
(SE) of $1.9$.

The key issue in using closed population models with such data is how
on earth do we interpret this estimate of $N=49.19$ bears? Does it
represent the entire population of Fort Drum? Certainly not -- the
trapping array covers less than half of Fort Drum!
(Fig. \ref{fig.hairsnares}). So to get at the total bear population
size of Fort Drum, we'd have to convert our $\hat{N}$ to an estimate
of density and extrapolate. To get at density, then, should we assert
that $N$ applies to the southern half of Fort Drum below some
arbitrary line? Surely bears move on and off of Fort Drum without
regard to hypothetical boundaries. Without additional information
there is simply no way of converting this estimate of $N$ to density,
and hence it is really not meaningful biologically. To resolve this
problem, we will adopt the customary approach of converting $N$ to $D$
by buffering the convex hull around the trap array. The convex hull
has area $157.135$ $km^2$. We follow \citet{bales_etal:2005} in
buffering the convex hull of the trap array by the radius of the mean
female home range size.


%%%%\footnote{Did Bales et al. actually do this?}.
%HERE IS WHAT BALES DID:  First, we created a 95% minimum convex polygon
%for all radiolocations of adult females used in homerange
%analyses (Figure 1). Second,we buffered the
%100% minimum convex polygon for trapping locations
%with the approximate radius of the average
%95% minimum convex polygon home range of adult
%females (n = 13) using ArcView (ESRI, Redlands,
%Calif.).

The mean female home range radius was estimated \citep{wegan:2008} for
our study region to be $2.19$ km,
%\footnote{Is this number right out of Wegan's disseration?}
% YES, this is straight out of his thesis.
and the area of the convex hull buffered by $2.19$ km is $277.01$
km$^2$. ({\bf R} commands to compute the convex hull, buffer it, and
compute the area are given in the {\bf R} package \mbox{\tt scrbook} which
accompanies the book).  Hence, the estimated density here is
approximately $0.178$ bears/km$^2$ using the estimated population size
obtained by model $M_0$.  We could assert that the problem has been
solved, go home, and have a beer.  But then, on the other hand, maybe
we should question the use of the estimated home range radius -- after
all, this is only the female home range radius and the home ranges
change for many reasons. Instead, we may decide to rely on a buffer
width based on one-half MMDM estimated from the actual hair snare data
as is more customary \citep{dice:1938}. In that case the buffer width
is $1.19$ km, and the resulting estimated density is increased to
$0.225$ bears/km$^2$ about 27 \% larger.  But wait -- some studies
actually found the full MMDM \citep{parmenter_etal:2003} to be a more
appropriate measure of movement (e.g
\citet{soisalo_cavalcanti:2006}). So maybe we should use the full MMDM
which is $2.37$ km, pretty close to the telemetry-based estimate and
therefore providing a similar estimate of density ($0.171$
bears/km$^2$). So in trying to decide how to buffer our trap array we
have already generated 3 density estimates. The crux of the matter is
obvious: Although it is intuitive that $N$ should scale with area --
the number of bears should go up as area increases and go down as area
decreases -- in this ad hoc approach of accounting for animal movement
$N$ remains the same, no matter what area we assert was sampled. The
number of bears and the area they live in are not formally tied
together within the model, because estimating $N$ and estimating the
area $N$ relates to are two completely independent analytical steps
which are unrelated to one another by a formal model.

Unfortunately, our problems don't end here. In thinking about the use
of model $M_0$, we might naturally question some of the basic
assumptions that go into that model. The obvious one to question is
that which declares that $p$ is constant. One obvious source of
variation in $p$ is variation {\it among individuals}. We expect that
individuals may have more or less exposure to trapping due to their
location relative to traps, and so we try to model this
``heterogeneous'' encounter probability phenomenon.  To illustrate
this here are the number of traps that each individual was captured
in:
\begin{verbatim}
 #traps:  1   2  3  4  5  6  9
 #bears: 23  13  6  2  1  1  1
\end{verbatim}
suggesting quite a range in traps exposed to by different bears.
%%% #bears: 19 15  5  2  2  1  1  1  1
% But, being captured in different numbers of traps is {\it not}
% inconsistent with a non-spatial model.  That is if individuals
% roamed randomly over space with no ``home range'' then you should
% expect them to be captured in varying numbers of traps also.....
This has led many to consider capture-recapture models that allow for
individual heterogeneity in $p$. Such models have the colloquial name
of ``model $M_h$'' \citep{otis_etal:1978}.  We fitted this model (see
Chapt. \ref{chapt.closed} for details) to the Fort Drum data using
each of the 3 buffer widths previously described (telemetry, 1/2 MMDM
and MMDM), producing the estimates reported in Table
\ref{intro.tab.fdests}. While we can tell by the models' AIC that
$M_h$ is clearly favored by more than 30 units, we might still not be
entirely happy with our results. Clearly there is information in our
data that could tell us something about the exposure of individual
bears to the trap array -- where they were captured, and how many
times -- but since space has no representation in our model, we can't
make use of this information. Model $M_h$ thus merely accounts for
what we observe in our data (some bears were more frequently captured
than others) rather than explicitly accounting for the processes that
generated the data.

XXXXX SCRIPT NEEDS put IN PACKAGE XXXXXXXX

So what are we left with?  Our density estimates span a range from
$0.17$ to $0.43$ bears/km$^2$ depending on which estimator of $N$ we
use and what buffer strip we apply. Should we feel strongly about one
or the other?  Which buffer should we prefer?  AIC favors model $M_h$,
but did it adequately account for the differences in exposure of
individuals to the trap array? Are we happy with a purely
phenomenological model for heterogeneity?  It assumes that all
individuals are iid draws from some distribution but does not account
for the explicit mechanism of induced heterogeneity. And, further, we
have information about that (trap of capture) which model $M_h$
ignores.
% Moreover, we could find more variations of model Mh to choose among,
% but see \citep{link:2003}.
And if we choose one type of buffer, how do we compare our density
estimates to those from other studies that may opt for a different
kind of buffer?  The fact that $N$ does not scale with $A$, as part of
the model, renders this choice arbitrary. The buffer isn't part of the
model.


\begin{table}[ht]
\centering
\caption{Table on estimates of density ($D$, bears/$km^2$) for the Fort Drum data
using models $M_0$ and $M_h$ and different buffers. Model $M_h$ here
is a logit-normal mixture \citep{coull_agresti:1999}.}
\begin{tabular}{ll|cc}
\hline \hline 
model & buffer &  $\hat{D}$ & SE \\ \hline
$M_0$   & telemetry &  0.178 & 0.178 \\
$M_0$    & MMDM     &  0.171 & 0.171\\
$M_0$   & 1/2 MMDM  &  0.225 & 0.225\\
$M_h$ & telemetry &0.341 & 0.144\\
$M_h$ & MMDM    &  0.327 & 0.138\\
$M_h$ & 1/2 MMDM & 0.432 & 0.183\\
\end{tabular}
\label{intro.tab.fdests}
\end{table}


\subsection{Inadequacy of Capture-Recapture}

***Models are not integrated with any ecological theory.****

The parameter $N$ in an ordinary capture-recapture
model  is functionally unrelated
to any notion of sample area, and so we are left
 taking arbitrary guesses at area, and matching it up with
estimates of $N$ from different models that do not have any explicit
biological relevance.  Clearly, there is not a compelling solution to
be derived from this ``estimate $N$ and conjure up a buffer'' approach
and we are left not much wiser about bear density at Fort Drum than we
were before we conducted this analysis, and certainly not confident in
our assessments.

The capture-recapture models that we used apply to truly closed
populations -- a population of goldfish in a fish bowl. Yet here we
are applying them to a population of bears that inhabit a rich
two-dimensional landscape of varied habitats, exposed to trapping by
an irregular and sparse array of traps. It seems questionable that 
the same model that is completely sensible for a population fo goldfish in
a bowl, should also be the right model for this population of bears
distributed over a broad landscape.

More specifically, 
ordinary capture-recapture methods are distinctly
non-spatial. They don't admit spatial indexing of either sampling (the
observation process) or of individuals (the ecological process). This
leads immediately to a number of practical deficiencies: 
(1) Ordinary CR models do
not provide a coherent basis for estimating density.  For
capture-recapture models to provide a coherent framework for inference
about population density, $N$ has to scale, as part of the model, with
$A$ so that the model imposes biological context on $A$ (i.e., as the
area over which the $N$ individuals reside). SCR models achieve this.
(2) Ordinary CR models {\it induce} a form of heterogeneity that can
only at best be approximated by classical models of latent
heterogeneity. SCR models formally accommodate heterogeneity due to
the juxtaposition of individuals with the encounter devices.  (3)
Ordinary CR models do not accommodate trap-level covariates which
exist in a large proportion of real studies. Again, SCR models
formally accommodate heterogeneity due to trap
variation; (4) Ordinary CR models do not accommodate formal
consideration of any spatial process that gives rise to the observed data.


In subsequent chapters of this book, we resolve these specific
technical problems related to density, model-based linkage of N and A,
covariates, spatial variation, and related things all within a
coherent unified framework for spatial capture-recapture. 




\begin{comment}
Some of the open questions at this point:
How do we characterize uncertainty of the buffer ``estimate''?  And,
in what sense is the
buffer even an estimate of something? What is it an estimate of?
The summary here should be that there's not a compelling solution to be derived from
this ``estimate $N$ conjure up a buffer'' approach.
{\bf The main point that N doesn't scale with A is not made
  clearly here.}
\end{comment}




\section{Extension of Closed Population Models}



The deficiency with classical closed population models is that they
have no spatial context. $N$ is just an integer parameter that applies
equally well to some population that exists in a computer, estimating the number
of unique words in a book, or a bucket full of goldfish.  The question
of {\it where} the $N$ items belong is central both to interpretation
of data and estimates from all capture-recapture studies and, in fact,
to the construction of spatial capture-recapture models considered in
this book.  Surely it must matter whether the $N$ items exist as words
in a book, or goldfish in a bowl, or birds in a forest patch! That
classical closed population models have no spatial context leads to a
number of conceptual and methodological problems or limitations as we
have  encountered previously. More important, 
ecologists seldom care only about $N$ -- space is often central to
objectives of many population studies  -- movement, space usage,
resource selection, how individuals are distributed in space and in
response to explicit factors related to landuse or habitat. because
space is central to so many real problems, this is probably the \# 1
reason that most ecologists don't bother with capture-recapture. They
haven't seen that as being able to solve their problems. 

Thus, the essential problem is that classical closed population models
are too simple - they ignore the spatial attribution of traps and
encounter events, movement and variability in exposure of individuals
to trap proximity, and, because ordinary closed population models
possess no notion of ``area'',  they do not yield estimates of {\it
  density}. These problems can be addressed formally by the development of
more general models.



\subsection{Efford's Formulation}

%Spatial capture-recapture models are
%statistical and mathematical models that extend non-spatial
%``ordinary'' capture-recapture models to accommodate the spatial
%structure inherent in sampling animal populations - i.e., trap
%locations, individual locations, and individual use of space.

The solution to the various issues that arise in the application of
ordinary capture-recapture models is to extend the closed population
model so that $N$ becomes spatially explicit.
%A natural way is to
%define a point process \citep{efford:2004} that describes how
%individuals are organized in space and that, when points are
%aggregated over space, the value $N$ is derived in a meaningful way.
%Thus, in this book, we adopt the view that the locations of the $N$
%individuals in the population are a {\it realization of a spatial
%  point process}.
\citet{efford:2004} was the first to formalize an explicit model for
spatial capture-recapture problems in the context of trapping arrays.
He adopted a Poisson point process model to describe the distribution
of individuals and then what is essentially a distance sampling
formulation of the observation model which describes the probability
of detection as a function of individual location, regarded as a
latent variable governed by the point process model. While earlier
(and contemporary) methods of estimating density from trap arrays have
been ad hoc in the sense of lacking a formal description of the
spatial model, Efford achieved a formalization of the model,
describing explicit mechanisms governing the spatial distribution of
individuals and how they are encountered by traps, but
adopted a more or less ad hoc framework for inference under that
spatial model using a simulation based method known as inverse
prediction \citep{gopalaswamy:2012}.

Recently, there has been a flurry of effort devoted to formalizing
inference under this model-based framework for the analysis of spatial
capture-recapture data
\citep{royle_gardner:2011,borchers:2011,gopalaswamy:2012}.  There are
two distinct lines of work which adopt the model-based formulation in
terms of the underlying point process but differ primarily by the
manner in which inference is achieved. One approach
\citep{borchers_efford:2008} is a classical inference approach based
on likelihood (see Chapt. \ref{chapt.mle}), and the other
\citep{royle_young:2008} adopts a Bayesian framework for inference
(Chapts. \ref{chapt.scr0} and \ref{chapt.mcmc}).

\begin{comment}
To motivate the origins and relevance of these approaches, we note
that, fundamentally, spatial capture-recapture models are related to
classical ``individual covariate'' models (colloquially referred to as
Huggins-Alho models) in capture-recapture \citep{huggins:1989,
  alho:1990}.  In particular, the individual covariate\footnote{have
  we mentioned what the individual covariate is, yet?} is observed in
these classical individual covariate models whereas it is not directly
observed in SCR models.  To accommodate that, a prior distribution for
the individual covariate is required.
%In essence then, SCR models are
%similar to a fully model-based formulation of classical Huggins-Alho
%models (see \citet{royle:2009}).
Likelihood analysis
\citep{borchers_efford:2008} proceeds by removing the random effect
from the likelihood by integration whereas Bayesian analysis
\citep{royle_young:2008} proceeds by analyzing the conditional model
directly, usually by methods of Markov chain Monte Carlo (MCMC).
\end{comment}



\subsection{Abundance as the Aggregation of a Point Process}

Spatial point process models represent a major methodological theme in
spatial statistics \citep{cressie:1992} and they are widely applied as
models for many ecological phenomena
\citep{stoyan_penttinen:2000,illian_etal:2008}. Point process models
apply to situations in which the random variable in question
represents the locations of events or objects: trees in a forest,
weeds in a field, bird nests, etc.  As such, it seems natural to
describe the organization of individuals in space using point process
models. SCR models represent the extension of ordinary
capture-recapture models by augmenting the model with a point process
model to describe individual locations.

Specifically, let ${\bf s}_{i}; i=1,2,\ldots,N$ be the locations of all
individuals in the population.  One of the key features of SCR models
is that the point locations are latent, or unobserved, and we only
obtain imperfect information about the point locations by observing
individuals at trap or observation locations.  Thus, the realized
locations of individuals represent a type of ``thinned'' point
process, where the thinning mechanism is not random but, rather,
biased by the observation mechanism.  It is also natural to think about the
observed point process as some kind of a compound or aggregate point
process with a set of ``parent'' nodes being the locations of
individual home ranges or their centroids, and the observed locations
as ``offspring'' - i.e., a Poisson cluster process (PCP). In that
context, density estimation in SCR models is analogous to estimating
the number of parents of a Poisson cluster process
\citep{chandler_royle:2012}.
% Other types of point
% process models for the realized locations have direct relevance to SCR
% models (See \citet{chandler_royle:2012}, discussed in chapter XYZ).

Most of the recent developments in modeling and inference
from spatial encounter history data, including most methods discussed
in this book, are predicated on the view that individuals are
organized in space according to a relatively simple point process
model. More specifically, we assume that the collection of individual
activity centers are independent and identically distributed
(abbreviated ``$iid$'') random variables distributed uniformly
over some region. This is consistent with the assumption that the
activity centers represent the realization of a Poisson point process
or, if the total number of activity centers if fixed, then this is
usually referred to as a binomial point process.  


\subsection{The Activity Center Concept}

In the context of SCR models, and because most animals we study by
capture-recapture are not sessile, there is not a unique and precise
mathematical definition of the point locations ${\bf s}$.
Rather, we imagine these to be the centroid of individual's home
ranges, or the 
centroid of an individual's
activities during the time of sampling. In general, this point is
unknown for any individual but if we could track an individual over
time and take many observations then we could perhaps get a good idea
of where that point is.  We'll think of the collection of these points
as defining the spatial distribution of individuals in the
population. 

%%I think we could shorten the home range paragraph; I like the definition
%%'the centroid of an individual's
%%% activities during the time of sampling'. I think the definition of
%% home range is something like the colleciton of points/sites/areas
%% an animal uses over the course of its lifetime so it's vague anyway
%% and what that definition means for the different forms of home
%%ranges - territory, migratory species etc - is pretty much left open.

We use the terms home range or activity center interchangeably. The
term ``home range center'' suggests that models are only relevant to
animals that exhibit behavior of establishing home ranges or
territories, or central place foragers,
 and since not all species do that, perhaps the
construction of SCR models based on this idea is flawed. However, the
notion of a home range center is just a conceptual device and we don't
view this concept as being strictly consistent with classical notions
of animal territories. Rather our view is that a home range or
territory is inherently dynamic, temporally, and thus it is a
transient quantity - where the animal lived during the period of
study, a concept that is completely analogous to the more conventional
notion of utilization distributions.  Therefore, whether or not
individuals of a species establish home ranges is irrelevant because,
once a precise time period is defined, this defines a distinct region
of space that an individual must have occupied. 
%In other words, the
%definition of ``home range center'' is predicated on the specification
%of a time period over which individuals are studied. A term that might
%be less offensive than ``home range center'' is ``centroid of space
%usage (CSU)'' which should not conflict directly with preconceived
%understandings and interpretations of home range.


\subsection{The state-space}

Once we introduce the collection of activity centers, ${\bf s}_{i};
i=1,2,\ldots,N$, then the question ``what are the possible values of
${\bf s}$?'' needs to be addressed because the individual ${\bf
  s}_{i}$ are {\it unknown}. As a technical matter, we will regard
them as random effects and in order to apply standard methods of
statistical inference we need to provide a distribution for these
random effects.  In the context of the point process model, the
possible values of the point locations referred to as the
``state-space'' of the point process and this is some region or set of
points which we will denote by ${\cal S}$. This is analogous to what
is sometimes called the {\it observation window} for ${\bf s}$ in the point process
literature.
The region ${\cal S}$ serves as a prior distribution for ${\bf
  s}_{i}$ (or, equivalently, the random effects distribution).
%%Don't think prior has come up yet; maybe not that important here?
In animal studies, as a description of
where individuals that could be captured are located, it includes our
study area, and should accommodate all individuals that could have been captured in the study
area.
In the practical application of SCR models, in most cases estimates of
density will be relatively insensitive to choice of state-space which
we discuss further in Chapt. \ref{chapt.scr0} and elsewhere.
\begin{comment}
%%% I also think the rest of this paragraph could be postponed to a later chapter
unless there are meaningful features to the state-space
which should be accommodated. For example, if the region within which
traps are located contains a coastline or a huge body of water then
clipping that out of the state-space will typically have an
effect on density
(see
sec. \ref{scr0.sec.wolverine} for an illustration).
This should be expected because, insofar as the
state-space serves as a prior distribution on the latent variables
${\bf s}_{i}$ then, {\it the state-space is a
  component of the model. } We discuss choosing the state-space in
Chapt. \ref{chapt.scr0}.
\end{comment}

\subsection{Abundance and Density}

When the underlying point process is well-defined, including a precise
definition of the state-space, this in turn induces a precise
definition of the parameter $N$ ``population size'' as the number of
individual activity centers located within the prescribed state-space,
and its direct linkage to density, $D$. That is, if $A({\cal S})$ is
the area of the state-space then
\[
 D = \frac{N}{ A({\cal S})}.
\]
A deficiency with some classical methods of ``adjustment'' is they
attempted to prescribe something like a state-space - a ``sampled
area'' - except absent any precise linkage of individuals with the
state-space. SCR models formalize the linkage between individuals and
space and, in doing so, provide an explicit definition of $N$
associated with a well-defined spatial region, and hence density. That
is, the provide a model in which $N$ scales, as part of the model,
with the size of the prescribed state-space. In a sense, the whole
idea of SCR models is that by defining this point process and its
state-space ${\cal S}$, this gives context and meaning to $N$ which
can be estimated directly for that specific state-space. Thus, it is
fixing ${\cal S}$ that resolves the problem of ``unknown area'' that
we have previously discussed.




\section{A Survey of Spatial Capture-Recapture}

combine with next and cover basic sampling issues and models?


\section{Elements of SCR Models}

Formulation of capture-recapture models conditional on the latent
point process 
is the critical and unifying element of
{\it all} SCR models. However,
 there are many more aspects relevant to the formulation
of SCR models for specific situations. We address various of these
things in later chapters. 

\subsection{ Biological focus}

SCR models differ in how the underlying process model is formulated,
or its complexity. 
Most of the development and application of SCR models has focused on
their use to estimate density and touting the fact that they resolve
certain specific technical problems related to the use of ordinary
capture-recapture models (as noted in sec. xXXXXX above). This is
achieved with a simple process model being a basic point process of
independently distributed points. 
 At the same time, there are
models of CR data that focus exclusively on {\it movement} modeling,
or models with explicit dynamics \citep{ovaskainen:2004,
  ovaskainen_etal:2008}. Conceptually, these are akin to spatial versions of
so-called Cormack-Jolly-Seber (CJS) models in the traditional
capture-recapture literature, except they involve explicit mathematical models of
movement based on diffusion or Brownian motion.
Finally, there are now a very small
number of papers that focus on {\it both} movement and density
simultanesouly \citep{royle_young:2008, royle_etal:2011mee, royle_chandler:2012} or
population dynamics and density \citep{gardner_etal:2010jwm}.

A key thing is that these models, whether focused just on density, or
just on movement, or both, are similar models in terms of the
underlying concepts, the latent
structure, and the observation model. They are rather just different
in what the ecological focus is. 

It is great to focus on elaborate models of movement....  but a strict
focus on developing elaborate movement models will be limited by two
practical considerations: (1) most capture-recapture data e.g., by
camera trapping or whatever, produces only a few observations of each
individual (between 1-5 would be typical). So there is not too much
information about complex movement models.  (2) Typically people have
an interest in density of individuals and therefore you need models
that can be extrapolated from the sample to the unobserved part of the
population.  My sense in looking at some of the movement modeling
papers is that they are focused on "what is this individual doing in
relation to the space it has available" and there is no formal attempt
to extrapolate a sample to the population.  That said, there are
clearly some cases where more elaborate movement models should come
into play. If one has some telemetry data in addition to SCR then
there is additional info on fine-scale movements that should be
useful.
 

\subsection{ Sampling focus}

A second way to characterize SCR models is based on how the
encounter observations arise.

Broadly speaking we differentiate between two situations: Sampling
based on fixed arrays or sampling based on ``search encounter''
methods. The former includes things like camera traps, hair snares,
mist nets and conventional traps. Fixed arrays limit the observation
location to pre-defined points, where traps are located. Using such
methods the model is a little simpler because the ``movement process''
of individuals is confounded with the ``observation process''.  The
2nd type of model -- search encounter models -- typically will allow
locations in continuous space, possibly only restricted by polygon
boundaries \citep{royle_young:2008}.  Search-encounter data usually
allow for the separate modeling and estimation of movement model
parameters from encounter model parameters but not always, depending
on whether replication of the sampling is done. 









\begin{comment}

\subsection{Why is density so important? }

Knowledge of population size is a fundamental piece of information in
conservation. Since the risk of a species/population going extinct is
a function of how many individuals of that species there are, much of
conservation-related research revolves around abundance. Consider, for
example, the concept of minimum viable population size � to assess
whether a population has a good chance of persistence over some time
frame we need to know how big it is to begin with. The idea of a
minimum viable population is reflected in many applied conservation
efforts. For example, in a range-wide assessment of the jaguar�s
population status, researchers were asked to delineate Jaguar
Conservation Units (JCU�s), of which one criterion was ``holding at
least 50 jaguars'' � a number considered a substantial population
\citep{sanderson_etal:2002}.

While the importance of abundance is indisputable, there are some
major issues associated with this measure. First, you cannot compare
mere values of abundance unless they refer to a specific area. If you
look at the IUCN Red List of Endangered Species entry for the
population status of the tiger, it will tell you that there are an
estimated 1700 tigers in India but only about 20 in Cambodia
\citep{chundawat_etal:2011}. Now, this will not automatically make you
lament the state of tiger conservation in Cambodia as compared to
India (although seeing these numbers you might well lament the state
of the tiger in general), because you know these numbers refer to
countries that are extremely different in size. Rather, if you wanted
to know something about where tigers are currently doing better,
you�d probably divide the number of tigers by the countries�
areas and compare tiger densities (turns out India�s tigers are
still doing better, not by a factor of 85, as mere abundances suggest,
but by a factor of 5). Although abundance and density are obviously
directly related to each other, they are different in their
applicability. Particularly, density as a scaled measure lets us
compare results across sites (as we just demonstrated for the tiger
example). In addition, some concepts incorporated in conservation
biology explicitly deal with density. For example, population growth
rate, home ranges or the probability of epidemics/disease spread are
density-dependent; the Allee effect links individual reproductive
success to population density in low-density populations.

Second, going back to the tiger example once more, we may wonder how
researchers even came up with these numbers for total population
size. Tiger abundance can be estimated using camera-traps, because
individuals have distinct stripe patterns so that photographic data
can be analyzed with capture-recapture models. But surely, no-one ever
camera-trapped the whole of India. This is a typical situation, even
on a much smaller scale. Ecologists generally sample only a small
fraction of the area used by a species or population, but want to
estimate total population size, i.e. the number of individuals
occurring in sampled {\it and unsampled} areas. If we can use the data
from sampled area to obtain a density estimate, explicit predictions
of abundance can be made to regions of any size (assuming that density
is constant across the region we are inferring to and equal to density
in the sampled area)\footnote{Note that the way total tiger abundance
  estimates are derived for India is much more complex than just
  looking at tiger density somewhere in India and then extrapolating
  it to the entire country (for details, see \citep{jhala_etal:2011});
  we merely use these numbers here to illustrate the general
  problem.}.

To summarize, density not only influences several ecological
processes, but also allows us to compare population status among
different sites; even where total abundance is of primary interest,
density can help us arrive at a total population estimate even when
we�re unable to survey the total population. Capture-recapture
models were designed to estimate abundance, but they generally cannot
be used to formally estimate density. This limitation of non-spatial
CR models has long been recognized (REF) and several ad hoc approaches
to overcome this problem have been devised. We will discuss those and
their shortcomings in XXX. The great advantage of SCR models over
non-spatial capture-recapture models is that they formally link
abundance and area so that they actually estimate density.


\end{comment}









\section{Summary and Outlook}


Spatial capture-recapture models are an extension of ordinary
capture-recapture models to accommodate the spatial organization of
both individuals in a population and the observation mechanism (e.g.,
locations of traps).  They resolve problems which have been recognized
historically and for which various ad hoc solutions have been
suggested: heterogeneity in encounter probability due to the spatial
organization of individuals relative to traps, the need to model
trap-level effects on encounter, and that a
well-defined sample area does not exist in most studies, and thus
estimates of $N$ using ordinary capture-recapture models cannot be
related directly to density.

But SCR models are much more than an extension of technique that
resolves certain technical problems of ordinary capture-recapture
models.  Rather, for the first time, they provide a coherent, flexible
framework for making ecological processes explicit in models of
individual encounter history data.  Spatial capture-recapture models
show great promise in their ability to integrate explicit ecological
theories directly into the models so that we can directly test
hypotheses about either space usage (e.g.,
Chapt. \ref{chapt.ecoldist}) or movement
(Chapt. \ref{chapt.search-encounter}) or the distribution of
individuals in space (Chapt. \ref{chapt.state-space}). We imagine that
in the near future SCR models will include point process models that
allow for interactions among individuals such as inhibition or
clustering \citep{reich_etal:2012}.  Thus, SCR models are capture-recapture models
that enable ecologists to explicitly integrate biological context and
theory with encounter history data, which is something that has always
been the focus of ``open population'' models but never, until very
recently, has been considered formally in closed population models. We
therefore believe that SCR models will enable ecologists to test
theories of space usage and environmental effects, social behavior and
other important theories.

{\bf key graph:
SCR = holistic framework for studing animal populations -- individual movement,
space usage, population dynamics, and density. Historically these things are all
studied differently using ostensibly unrelated study designs and
statistical procedures. RSF models for resource selection or space
usage, state-space models for movement, density using closed
capture-recapture, and dynamics from a hodge-podge of open extensions
of those.  SCR brings all of these problems together into a single
unified framework for modeling and inference. 
}


In the following chapters we develop a comprehensive synthesis and
extension of spatial capture-recapture models.  Roughly the first
third of the book is introductory material -- In
Chapt. \ref{chapt.glms} we provide the basic analysis tools to
understand and analyze SCR models - namely generalized linear models
(GLMs) with random effects, and their analysis in {\bf R} and {\bf
  WinBUGS}.  This is important material because 
we find that SCR models are essentially
variations of generalized linear mixed models (GLMMs). This in a
sense makes them consistent with many important methodologies used in
ecology (e.g., see \citet{zuur_etal:2009, kery_etal:2010}), and
because of the connection with standard modeling concepts, we believe
that the material presented in this book can be understood and used by
most ecologists with some modeling experience.

Because SCR models represent extensions of basic closed
population models, we cover ordinary closed population models in
Chapt. \ref{chapt.closed} wherein, along with Chapts. \ref{chapt.scr0}
and \ref{chapt.poisson-mn} provides the basic
introduction to capture-recapture models and their spatial extension.
We delve more deepling into the details of both likelihood
 (Chapt. \ref{chapt.mle}) and Bayesian analysis
(Chapt. \ref{chapt.mcmc}) of SCR models. 
In the last third of the book,
we address more advanced stuff including modeling space usage in the
encounter process (Chapt. \ref{chapt.ecoldist}), modeling state-space
covariates, covariates that affect density,
(Chapt. \ref{chapt.state-space}), open population models
(Chapt. \ref{chapt.open}), models that include unmarked individuals
either entirely (Chapt. \ref{chapt.scr-unmarked}) or partially marked
samples (Chapt. \ref{chapt.partialID}). This last third of the book is
largely based on research that has only very recently been published
in the primary literature but we feel is important to provide a full
picture of the importance of SCR models. 







