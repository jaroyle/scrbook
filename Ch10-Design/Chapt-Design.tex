\chapter{
Sampling Design
}
\markboth{Sampling design}{}
\label{chapt.design}

\section{General Considerations}


\subsection{Model-based not design-based}

\subsection{Sampling space or sampling individuals?}

\subsection{Scope of inference vs. state-space}


\section{Study design for (spatial) capture-recapture}



\section{Trap spacing and array size relative to animal movement}

 \begin{table}[ht]
  \centering
      \begin{tabular}{l*{7}{c}}
    \hline
   \hline
    
    \end{tabular}
  \label{design.tab.simres}
\end{table}


\begin{table}[ht]
  \centering
      \begin{tabular}{l p{2.1cm} p{2.3cm}p{2.1cm}p{2.3cm}}
    \hline
   
    \end{tabular}
  \label{design.tab.simdat}
\end{table}


\subsection{Example: Black bears from Pictured Rocks National Lakeshore: }

\begin{table}[ht]
  \centering
 
    \begin{tabular}{lcccc}

	\hline
 
    \end{tabular}
  \label{design.tab.bears}
\end{table}


\subsection {Final musings: SCR models, trap spacing and array size}


\section{ Spacing of traps with telemetered individuals}
\label{design.sec.telemetry}



\section {Sampling over large scales}



\section{Model-based Spatial Design}



\subsection{Formalization of the Design Problem for SCR Studies}




\subsection{An Optimal Design Criterion for SCR}


\begin{equation}
\label{design.eq.theQ}
\end{equation}


\begin{equation}
\label{design.eq.varpbar}
\end{equation}


\begin{equation}
\label{eq.varbeta}
\end{equation}



\subsection{Optimization of the criterion}
\label{design.sec.exchange}


\subsection{Illustration}



\section{Covariate models}





\section {Summary and Outlook}

