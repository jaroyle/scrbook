\chapter{
%Modeling Animal space-usage with 
%Detection Models based on Ecological Distance
Ecological Distance Models in Spatial Capture-Recapture
}
\markboth{Chapter XXX}{}
\label{chapt.implicit}


\vspace{.3in}


Previously we have only considered stationary and symmetric models for
detection probability. While these will often be sufficient for
practical purposes, especially in small data sets, there will
sometimes be interest in developing more complex models of the
detection process as it relates to space usage of individuals.  

These models are simple because they are based on
Euclidean distance -- which are convenient because we know this
distance precisely conditional on individual activity center ${\bf  s}$.
However, animals may not
judge distance in terms of euclidean distance but, rather, according
to local habitat conditions and so forth. 

In this section we develop models for detection probability based on a
distance metric that tries to mimic ecological distance. 

\section{Cost Distance}

We use a cost-weighted distance metric in the package \mbox{\tt
  gdistance} \index{R package!gdistance} which computes the distance between points by
accumulating pixel-specific costs assigned by the user (but we
consider estimating these in Sec. XYZ). The idea is due to XYZ XYZ REF
XYZ. The basic idea is that distance bewtween $x_{i}$ and $x_{j}$ is
\[
 wd_{ij} =  w_{ij} ....\mbox{formula for this?}
\]
where .... and so on....XXX XYZ.
To be consistent with the functioning of \mbox{\tt costdistance} the
incremental cost is assessed for {\it leaving} a pixel, not
arriving. As a consequence, the terminal value is not counted. 
As an example of the cost-weighted distance calculation consider the
following landscape comprised of 9 pixels identified as follows, along
with their costs:
\begin{verbatim}
  pixel ID           cost
  1 2 3             100  1  1 
  4 5 6             100 100 1
  7 8 9             100 100 1
\end{verbatim}
Then we assign low cost of 1 to ``good habitat'' pixels (or pixels we think of as
``highly connected'' by virtue of being in good habitat) and,
conversely, we 
assign high cost (100) to ``bad habitat''. So the cost-weighted distance
between pixels 2 and 3 in this example is just 1 unit, the distance
between pixels 2 and 6 is sqrt(2) units, and the distance between
pixels 4 and 5 is 100 units, while the cost-distance between 4 and 6 is 101.

Do the calculation here:

Alternatively we might label our pixels by transposing the id's so
that pixels 1, 4 and 7 represent the first row:
\begin{verbatim}
 1 4 7
 2 5 8 
 3 6 9
\end{verbatim}
In this case we have to transpose or rotate something along the way. 
Be careful. 

\subsection{Illustration: Example Good vs. Bad habitat}

We made up a polygon which we might think of as representing a habitat
corridor or park unit or something which has been preserved. It is
surrounded by a suburban wasteland of McDonalds and Wal-Marts, much
less hospital habitat for most things.
Figure here. 



\section{Distance Weighting}


\section{Hard Boundary}



\section{Simulation Study}

evaluate effect of ignoring eoclogical distance


\section{Estimating Cost or Resistance Values}




