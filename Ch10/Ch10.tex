\chapter{
%Modeling Animal space-usage with 
%Detection Models based on Ecological Distance
Ecological Distance Models in Spatial Capture-Recapture
}
\markboth{Chapter XXX}{}
\label{chapt.implicit}


\vspace{.3in}


Previously we have only considered stationary and symmetric models for
detection probability. While these will often be sufficient for
practical purposes, especially in small data sets, there will
sometimes be interest in developing more complex models of the
detection process.  These models are simple because they are based on
Euclidean distance -- which we always know. However, animals may not
judge distance in terms of euclidean distance but, rather, according
to local habitat conditions and so forth. 

In this section we develop models for detection probability based on a
distance metric that tries to mimic ecological distance. 


\section{Cost Distance}

We use a cost-weighted distance metric in the package \mbox{\tt
  gdistance}. Imagine a landscape like this:
\begin{verbatim}
  pixel ID           cost

  1 2 3             100  1  1 
  4 5 6             100  1  1
  7 8 9             100 100 1

\end{verbatim}
Then we assign low cost to ``good habitat'' or ``highly connected''
pixels and they receive weight 1 in the distance calculation. We
assign high cost to ``bad habitat''. So the cost-weighted distance
between pixels 2 and 3 in this example is just 1 unit, the distance
between pixels 2 and 6 is sqrt(2) units, and the distance between
pixels 4 and 5 is 100 units.
{\bf NOTE: This is ideally the case (and conceptually accurate) but
\mbox{\tt gdistance} doesn't strictly operate like this for reasons I
don't understand. But I'm ignoring that for now.
}



\section{Example Good vs. Bad habitat}



\section{Distance Weighting}


\section{Hard Boundary}



\section{Simulation Study}

evaluate effect of ignoring eoclogical distance


\section{Estimating Cost or Resistance Values}




