\documentclass[12pt]{article}

\usepackage[total={6.5in,8.75in}, top=2.4cm, left=2.4cm]{geometry}
\usepackage{lineno}
\usepackage{amsmath}
%\usepackage{amssymb}    % used for symbols in figure legends
\usepackage{graphicx}
\usepackage[round,colon,authoryear]{natbib}

\usepackage{bm}
\usepackage{float}
\usepackage{amsmath}
\usepackage{amsfonts}
\usepackage{hyperref}
\usepackage{verbatim}
\usepackage{soul}
\usepackage{color}
\usepackage{setspace}

\bibliographystyle{ecology} % kluwer, plos-natbib, pnas-natbib


%\title{Ecological Distance in Spatial Capture-Recapture Models}
\title{Spatial Capture-Recapture Models for
%Jointly Estimating Population Density and
Studying Landscape Connectivity}

\begin{small}
\author{
{\bf J. Andrew Royle}, {\bf Richard B. Chandler},
{\bf Kimberly D. Gazenski} \\
USGS Patuxent Wildlife Research Center, Laurel MD \\ \\
{\bf Tabitha A. Graves} \\
Northern Arizona University, Flagstaff AZ \\ \\
}
\end{small}


\begin{document}

\maketitle

\date


\linenumbers


\begin{spacing}{1.8}

\begin{flushleft}
{\em \bf Abstract}
Assessing the impacts of habitat fragmentation and habitat loss on connectivity
%on population density and landscape resistance to movement
are high priorities
in applied ecological research. Population density and landscape
resistance to movement are key determinants of connectivity, yet no methods exist for
simultaneously estimating density and resistance parameters.
Recently-developed spatial capture-recapture (SCR) models
provide a framework for estimating density of animal populations, %using
%individual encounter history data with auxiliary spatial information,
but thus far have not been used to study landscape resistance.  Rather, all
applications of SCR models have used encounter probability models
based on the Euclidean distance between traps and animal activity
centers. These models assume that home ranges are stationary and
symmetric, and that they are unaffected by landscape or habitat
structure. In this paper we devise encounter probability models based
on ``ecological distance'', i.e., the least-cost distance between
traps and activity centers, which is a function of both Euclidean
distance and animal movement behavior in %heterogeneous
resistant landscapes. We
integrate least-cost path models into a likelihood-based estimation
scheme for spatial capture-recapture models in order to estimate
population density and parameters of the least-cost encounter
probability model.  Therefore, it is possible to make explicit
inferences about animal density %, distribution,
and landscape connectivity as it relates to animal movement
from standard
capture-recapture data.  Furthermore, a simulation study demonstrated
that ignoring landscape resistance can result in biased density
estimators under the naive SCR model. % can be substantially
%biased when home ranges are not symmetric.

{\em \bf Key words:} animal movement, ecological distance, landscape connectivity,
least-cost path, spatial capture-recapture

\end{flushleft}



\section{Introduction}

Applied ecological research requires robust methods
for assessing how factors such as habitat fragmentation and habitat loss affect
landscape connectivity. Connectivity is a function of both population
density and landscape resistance, i.e. the degree to which landscape
structure impedes movement, and is an important determinant of
population viability \citep{with_crist:1995}. Numerous studies have focused on the
implications of increased landscape resistance resulting from
fragmentation; however, few
formal methods exist for estimating resistance
parameters, and instead, ecologists often rely on expert opinion or
\textit{ad hoc} methods of specifying resistance values
\citep{adriaensen_etal:2003,zeller_etal:2012}. In addition, no
methods are available for simultaneously estimating landscape
resistance parameters and population density. Such methods are needed
to evaluate theory predicting interactive effects of population density
and landscape resistance on population viability
\citep{cushman_etal:2010}, and for purposes such as informing reserve
design \citep{beier_etal:2008}.

Even when landscape resistance is not of direct interest, inference
about population density is challenging for mobile species
because movement can invalidate assumptions of standard models such
distance sampling, which require spatial
independence and population closure.
Spatial capture-recapture (SCR) models are a relatively new class of
models that do not makes these assumptions and were
designed explicitly for estimating population density of mobile
species
%animal density %from capture-recapture data with
%auxiliary information about individual capture locations
\citep{efford:2004,borchers_efford:2008, royle_young:2008, efford_etal:2009ecol,
  royle_etal:2009ecol}. SCR models also resolve other
critical problems with ordinary non-spatial capture-recapture
methods such as ill-defined area sampled and heterogeneity in
encounter probability due to the juxtaposition of individuals with
traps, and hence %While SCR models are a relatively recent innovation,
their use is already becoming widespread \citep{efford_etal:2009ecol,
  gardner_etal:2010jwm, gardner_etal:2010ecol,kery_etal:2010,
  gopalaswamy_etal:2012, foster_harmsen:2012}.
Furthermore,
essentially all capture-recapture studies produce auxiliary spatial
information and therefore SCR models are widely applicable.


In spite of their utility for estimating population
density, SCR  methods are still in their
infancy, and so far, every application %of SCR models
has been based on capture %encounter
probability models in which distance between individual activity
centers and trap locations is defined by a function of simple Euclidean
distance.  [define activity center here, or defer to next section?]
While these simple encounter probability models will often
be sufficient for some
purposes, especially in small data sets, %sometimes developing more
%complex models of the capture %detection
%process as it relates to space usage
%of individuals will be useful.
animals may not judge distance in
terms of Euclidean distance but, rather, according to the quality of local
habitat, %landscape connectivity,
perceived mortality risk, and other
considerations that facilitate or impede movement. Because capture %encounter
probability and the distance
metric upon which it is based represent outcomes of individual
movements about their home range, it is desirable to relax the
Euclidean distance assumption of SCR models such that
% offer a means of evaluating %ecologists might have explicit
hypotheses about how environmental variables affect movement can
be evaluated.
\begin{comment}
distance
metric, and it is therefore desirable to incorporate these hypotheses
directly into SCR models so that they may be formally evaluated
statistically.
As an example of the potential problem of parameterizing SCR models
using Euclidean distance, imagine a forest fragmented into two patches
% study area bisected
by a large semi-permeable barrier. In traditional SCR models, the probability of
capturing an animal in a trap located on the opposite side of the
barrier would simply be a function of distance, whereas in reality it
should be a function of both distance and the permeability of the
barrier.
Such situations are extremely common in capture-recapture
studies where multiple habitats occur in the study area or when
animals use linear features such as trails, corridors, or rivers.
Moreover,
Because capture %encounter
probability and the distance
metric upon which it is based represent outcomes of individual
movements about their home range, ecologists might have explicit
hypotheses about how environmental variables affect the distance
metric, and it is therefore desirable to incorporate these hypotheses
directly into SCR models so that they may be formally evaluated
statistically.
\end{comment}

In this paper we develop models for encounter probability based on
alternative distance metrics that account for ecological
considerations---which, in keeping with the conventions in the
ecological literature, we will call ``ecological distance''. In
particular, we adopt a cost-weighted distance metric (the least-cost path)
used widely in landscape ecology for modeling connectivity,
movement, and gene flow
\citep{adriaensen_etal:2003}. In the
%%%%%%%%%%,manel_etal:2003,mcrae_etal:2008}. In the
context of SCR models we can use this as the basis for computing the
distance between traps and individuals activity centers. In this way
we can explicitly accommodate landscape structure and
account for how animals use space in SCR models. We develop a
likelihood-based inference framework for SCR model parameters using
this new distance metric when the ecological distance function is
known.  We show that the maximum likelihood estimates (MLEs) are approximately unbiased in moderate
sample sizes, as expected, but also that the misspecified model based
on Euclidean distance can produce substantial bias in estimates of $N$
and hence density.  Further, we extend the model to allow for likelihood
estimation of parameters of the cost function, so that direct inference
about movement and connectivity can be made from capture-recapture data without subjective prescription
of resistance or cost surfaces.


\section{Spatial Capture-Recapture}

The basic idea of SCR is to express encounter probability %of individuals
as a function of the distance between an individual's center of
activity, say ${\bf s}$, and a trap location, say ${\bf x}$. The
definition of an activity will be context-specific, but often it will
be the center of an individual's home range, or more generally, the
spatial average of an individual's locations during some time
period. SCR methods regard the activity centers as latent variables
following some spatial point process, such the homogeneous
model ${\bf s} \sim \mbox{Uniform}({\cal
  S})$ where ${\cal S}$ is a spatial region (the ``state-space'' of
${\bf s}$), or a
model in which covariates might affect the spatial distribution of
individuals \citep{borchers_efford:2008}. The state-space ${\cal S}$
defines the potential values for any activity center ${\bf s}$, e.g.,
a polygon defining available habitat or range of the species under study.


%In these models ${\bf s}_{i}$ is regarded as a latent variable and
%conventional methods of statistical inference either based on marginal
%likelihood \citep{borchers_efford:2008} or Bayesian analysis by MCMC
%\citep{royle_young:2008}.

A number of distinct observation models have been proposed for
spatial capture-recapture situations \citep{borchers_efford:2008,
  royle_etal:2009ecol, efford_etal:2009ecol}, including Poisson,
multinomial or binomial observation models.
%The most appropriate
%observation model for a dataset depends upon the the sampling method.
%For example, when animals are physically captured and retained in
%traps, then the multinomial model is appropriate because it does not
%allow for an individual to be captured in multiple traps during a
%single occasion.
Here we focus on the binomial model in which we
suppose that $J$ traps at locations ${\bf x}_{j}$ are operated for $K$
occasions (e.g., nights), although our development of cost-distance
models is directly applicable to other observation models without any
further technical considerations. The binomial model is most directly
relevant to devices such as ``hair snares''
\citep{woods_etal:1999,gardner_etal:2010jwm} or ``scent sticks''
\citep{kery_etal:2010} for which individuals can only be encountered
at most once in a trap per observation occasion.

The observations are individual- and trap-specific counts $y_{ij}$
which are binomial with sample size $K$ and probabilities
$p_{ij}$. The vector of trap-specific counts for individual $i$,
 ${\bf y}_{i} = (y_{i1},\ldots,y_{iJ})$ is its {\it encounter history}.
A standard encounter probability model
\citep{borchers_efford:2008} is the Gaussian model in which

\begin{equation}
log(p_{ij})= \theta_{0} + \theta_{1} d({\bf x}_{j} - {\bf s}_{i})^{2}
\label{eq.encounter}
\end{equation}
or, equivalently,
$p_{ij} = \lambda_{0} exp(-  d({\bf x}_{j} - {\bf s}_{i})^{2}
/(2\sigma^{2}) )$
where $\theta_{0} = log(\lambda_{0})$ and $\theta_{1} =
-1/(2\sigma^2)$ and $d({\bf x}_{j} - {\bf s}_{i})$ is the Euclidean
distance between trap $j$ and activity center $i$.


Alternative detection models are used, and all are
functions of Euclidean distance and so we do not consider them
further. The distance metric we develop subsequently can be
used in conjunction with any other detection model.  In all previous
applications of SCR models the normal Euclidean distance has been
used, i.e., $ d({\bf x}_{j} - {\bf s}_{i}) = ||{\bf x}_{j} - {\bf
  s}_{i}||$, and the parameters $\theta_0$ and $\theta_1$ have been
estimated using standard methods (likelihood or Bayesian).

The critical assumption that motivates our work is that the Euclidean
distance metric is unaffected by habitat or landscape structure, and
it implies that the space used by individuals is stationary (invariant
to translation) and symmetric
which may be unreasonable in some applications.  For example, if the
common detection model based on a bivariate normal probability
distribution function is used, then the implied space usage by {\it
  all} individuals, no matter their location in space or local habitat
conditions, is symmetric with circular contours of usage intensity
(density contours of the probability density).  Subsequently we provide an extension
of this class of SCR models that accommodates alternative distance
metrics that explicitly incorporate information about the landscape so
that a unit of distance is variable depending on identified
covariates. Thus, ``where'' an individual lives on the landscape, and
the state of the surrounding landscape, will determine the character
of its usage of space. In particular, we suggest distance metrics that
imply irregular, asymmetric and non-stationary home ranges of
individuals. An example of this is shown in Fig. \ref{fig.homeranges}
which shows home ranges of 4 individuals for a specific landscape,
described below.



\section{Landscape Resistance Defined by Cost-Weighted Distance}

We adopt the use of a cost-weighted distance metric here which defines
the distance between points by accumulating pixel-specific costs
determined under a cost function defined by the user.  The idea of
cost-weighted distance to characterize animal use of landscapes is
widely used in landscape ecology for modeling connectivity, movement
and gene flow \citep{beier_etal:2008}. As is customary for reasons of
computational tractability we consider a discrete landscape defined by
pixels
of some prescribed resolution. The distance between any two
points ${\bf x}$ and ${\bf x}'$ can be represented by a sequence of
line segments connecting neighboring pixels say ${\bf l}_{1},{\bf
  l}_{2},\ldots,{\bf l}_{m}$. Then the cost-weighted distance between
${\bf x}$ and ${\bf x}'$ is

\begin{equation}
 d({\bf x},{\bf x}')
  =  \sum_{i=1}^{m-1} cost({\bf l}_{i},{\bf l}_{i+1})||{\bf l}_{i} - {\bf l}_{i+1}||
\label{eq.costweighted}
\end{equation}

{\flushleft
where } $cost({\bf l}_{i},{\bf l}_{i+1})$ is the user-defined cost function
to move
from pixel ${\bf l}_{i}$ to neighboring pixel ${\bf l}_{i}$ in the sequence.
Given the ``cost'' of each pixel, it is a simple matter to compute the
cost-weighted distance between any two pixels, along {\it any} path,
by accumulating the incremental  costs weighted by
distances.
In the context of
SCR models (and, more generally, landscape
connectivity) we are concerned with the {\it minimum} cost-weighted
distance, or the {\it least-cost path}, between any two points which
we will denote by $d_{lcp}$, which is
the
sequence ${\bf l}_{1},{\bf l}_{2},\ldots,{\bf l}_{m}$ that minimizes
Eq. \ref{eq.costweighted}. That is,

\begin{equation}
 d_{lcp}({\bf x},{\bf x}')
  =  min_{{\bf l}_{1},\ldots,{\bf l}_{m}}  \sum_{i=1}^{m-1} cost({\bf l}_{i},{\bf l}_{i+1})||{\bf l}_{i} - {\bf l}_{i+1}||
\label{eq.lcp}
\end{equation}

{\flushleft
 Least-cost} path distance can be calculated in
 many geographic information systems and other software packages,
including the {\bf R} package \mbox{\tt
  gdistance} \citep{vanetten:2011}.


The key ecological aspect of least-cost path modeling is the
development
of models for pixel-specific cost.
In this paper we model cost as a log-linear function of covariates
defined on every pixel.
%{\flushleft This} is interpreted as cost of moving {\it through} a
%pixel but, in practice,
Specifically, we model cost of
moving from a pixel ${\bf x}$ to one of its neighbors, say ${\bf
  x}'$, by the weighted average

\begin{equation}
 log(cost({\bf x},{\bf x}'))=  \theta_{2} \frac{z({\bf x})+z({\bf x}')}{2}
\label{eq.cost}
\end{equation}

{\flushleft
where $\theta_{2}$ is a parameter to be estimated.} Thus, if $\theta_{2} = 0$ then substituting $cost({\bf x},{\bf x}')
=exp(0) = 1$ into
Eq. \ref{eq.lcp} will produce the ordinary Euclidean distance
between points.

In practical applications, variables that influence the cost of moving
across the landscape include things like highways
\citep[e.g.,][]{epps_etal:2005}, elevation \citep{cushman_etal:2006},
 snow cover
\citep{schwartz_etal:2009}, distance to escape terrain
\citep{shirk_etal:2010}, or range limitations
\citep{mcrae_beier:2007}.  Together multiple environmental variables
create a resistance surface, which forms the linchpin of all
connectivity planning \citep{spear_etal:2010}.
 Although $\theta_{2}$ is never known in practice,
conservation biologists design linkages that require this resistance
value as input \citep[see][and articles cited
therein]{beier_etal:2008}.
Therefore, investigators often choose a value for $\theta_{2}$
based on
expert opinion \citep{beier_etal:2008}, although recently researchers
have begun to define costs based on resource selection functions,
animal movement \citep{tracy:2006, fortin_etal:2005}, or genetic
distance data \citep{gerlach_musolf:2000,
schwartz_etal:2009}. Our paper represents the first formal approach
to estimating $\theta_{2}$ from capture-recapture data.
%The published
%methods require many assumptions and these estimates have not been
%integrated with capture-recapture methods, nor have parameters been
%formally estimated using capture-recapture data.

To formalize the use of cost-weighted distance in SCR models, we
substitute Eq. \ref{eq.lcp} in the expression for encounter
probability (Eq. \ref{eq.encounter}) and maximize the resulting
likelihood (see {\it Maximum Likelihood Estimation}), including for estimating the
parameter $\theta_{2}$. An example of computing cost-weighted distance
in {\bf R} for a simple landscape is given in Appendix 1.


\subsection{Non-stationarity of home range structure}

When distance is defined by the cost-weighted distance metric given
by Eq. \ref{eq.lcp} then individual space-usage varies
spatially in response to the landscape covariate(s) used in the
distance metric. For example, we use the two landscapes shown in
Fig. \ref{ecoldist.fig.raster100}
in our
our simulation study below. The landscape shown in the right panel, with
distance
metric defined by Eq. \ref{eq.lcp}, produces home ranges such
as those shown in Fig. \ref{fig.homeranges}. Later we simulate data
under the model that produces these home ranges and fit spatial
capture-recapture models to evaluate the efficacy of likelihood
estimation under this model.




\section{Maximum likelihood estimation}
\label{sec.mle}

Here we outline a standard method of parameter estimation based on
marginal likelihood. That is, the likelihood in which the latent
variables ${\bf s}$ are removed by integration \citep{borchers_efford:2008}.
The individual- and trap-specific observations have a binomial
distribution conditional on the latent variable ${\bf s}_{i}$:

\begin{equation}
  y_{ij}| {\bf s}_{i} \sim \mbox{Binomial}(K, p_{\theta}(d_{lcp}({\bf x}_{j},{\bf s}_{i};\theta_{2}); \theta_{0}, \theta_{1})
\label{mle.eq.cond-on-s}
\end{equation}

{\flushleft where} we have indicated the dependence of $p_{ij}$ on the parameters
${\bm \theta}$, and also $d_{lcp}$ which
itself depends on $\theta_{2}$, and the latent variable ${\bf s}$.
%The parameters
%${\bm \theta}$ include whatever parameters are involved in the
%cost-weighted distance function, i.e., at least $\theta_{2}$ from
%Eq. \ref{eq.cost}.
For the random effect we have ${\bf s}_{i} \sim  \mbox{Uniform}({\cal
  S})$.
The joint distribution of the data for individual $i$ is the product
of $J$ binomial terms (i.e., contributions from each of $J$ traps):
$  [{\bf y}_{i} | {\bf s}_{i} , \theta] =
  \prod_{j=1}^{J} \mbox{Binomial}(K, p_{\theta}({\bf x}_{j},{\bf s}_{i}) )$.
%This assumes independence of capture in each trap.
%Conditional on
%${\bf s}_{i}$ this is reasonable in most applications in our view.
 The so-called marginal likelihood is computed by removing
${\bf s}_{i}$, by integration,  from the conditional-on-${\bf s}$
likelihood and regarding the {\it marginal} distribution of the data
as the likelihood. That
is, we compute:

\[
  [y|{\bm \theta}] =
\int_{{\cal S}}  [ {\bf y}_{i} |{\bf s}_{i},{\bm \theta}] g({\bf s}_{i}) d{\bf s}_{i}
\]

{\flushleft where}, under the uniformity assumption, we have
$g({\bf s}) = 1/||{\cal S}||$.
The joint likelihood for all $N$ individuals,
is the product of $N$ such terms:

\[
{\cal L}({\bm \theta} | {\bf y}_{1},{\bf y}_{2},\ldots, {\bf y}_{N}) = \prod_{i=1}^{N}
[{\bf y}_{i}|{\bm \theta}]
\]

Technical details for computing the likelihood and obtaining the MLEs
are given in Appendix 2 where we provide an ${\bf R}$ function
to evaluate the likelihood and obtain the MLEs.
A key practical detail is that the likelihood here is formulated in
terms of the parameter $N$, the population size for the landscape
defined by ${\cal S}$. Given ${\cal S}$, density
is
computed as $D({\cal S}) = N/||{\cal S}||$ where $\|\mathcal{S}\|$ is
the area of the state space. In our simulation study
below we report $N$ as the two are equivalent summaries of the data
set once ${\cal S}$ is defined.


\section{Examples of SCR Models with Landscape Resistance}

In this section we provide examples that we think are typical of how
cost-weighted distance models can be used in real capture-recapture
problems.  We define a $20 \times 20$ pixel landscape with
extent = $[0.5, 4.5] \times [0.5, 4.5]$.
%We define this landscape by
%a single covariate for determing the cost function, and we consider
%two specific covariates.
%purposes of our example, as a coarse landscape covariate, with pixels
%having some arbitrary scaling say, a $2 \times 2$ km resolution. Thus,
%the raster defines a landscape of $40 \times 40$ km and we suppose
We suppose that 16 camera traps are established at the integer coordinates
$(1,1), (1,2), \ldots, (4,4)$. We could think of this as a landscape
within which we're studying a population of ocelots, lynx or some
other cat.

For our analyses, cost is characterized by a single covariate
and we consider two specific cases. First is an increasing trend from
the NW to the SE (``systematic landscape''), where $z(x)$ is defined as
$z(x) = r(x) + c(x)$ where $r(x)$ and $c(x)$ are just the row and
column, respectively, of the landscape.  This might define something
related to distance from an urban area or a gradient in habitat
quality due to land use, or environmental conditions such as
temperature or precipitation gradients.  In the second case we make up
a covariate by generating a field of spatially correlated noise to
emulate a typical patchy habitat covariate (''patchy landscape'') such as
tree or understory density.
\hl{Maybe we should call this a ``fragmented landscape''}
The two covariates are shown in
Fig. \ref{ecoldist.fig.raster100}, along with a sample realization of
$N=100$ individuals (left panel only).  For both covariates we use a
cost function in which transitions from pixel ${\bf x}$ to ${\bf x}'$
is given by:

\[
 log(cost({\bf x},{\bf x}'))=  \theta_2 \frac{z({\bf x}) + z({\bf x}')}{2}
\]

{\flushleft where} $\theta_2 = 1$ for our simulation.
When $\theta_2=0$ the
model reduces to one in which the cost of moving across each pixel is
constant, and therefore distance is Euclidean.

\subsection{Simulation study}

We devised a limited simulation study to evaluate three things: (1)
the general statistical performance of the density estimator under
this new model; (2) the effect of mis-specifying the model with a
normal Euclidean distance metric and (3) the statistical performance
of estimating the relative cost parameter.

We used population sizes of 100 and 200 individuals with activity
centers randomly distributed on the $20 \times 20$ landscape, and subjected them
to encounter by 16 traps arranged in a $4\times 4$ grid according to
the Euclidean distance metric. We fit 3 different models; (i) the
misspecified euclidean distance model; (ii) the true data-generating
model with the cost {\it known} and (iii) the true
data-generating model but estimating the relative cost parameter by
maximum likelihood.  We used the ``systematic'' and ``patchy''
covariates defined previously.

We simulated encounter data for the $N$ individuals using the Gaussian
encounter model with least-cost path distance metric:

\[
log(p_{ij})= \theta_{0} + \theta_{1} d_{lcp}({\bf x}_{j},{\bf
  s}_{i}; \theta_{2})^{2}
\]

{\flushleft We } used here $\theta_{0} = -2$ and $\theta_{1} = 2$, the latter value
corresponding to $\sigma = 0.5$ of a stationary bivariate normal home
range model.  We varied the number of replicate samples $K=3,5,10$
(e.g., nights in a camera trapping study) to produce varying sample
sizes.  Because any simulation study is inherently arbitrary, we have
provided {\bf R} scripts for carrying out simulations in Appendix 2 so
that the interested reader can experiment with their own situations.




\subsection{Simulation Results}

For both landscapes and all simulation conditions (levels of $K$ and
$N$) the average sample sizes of individuals captured are given in
Appendix 4.  The simulation results for estimating $N$
for the prescribed state-space are given in Table 1.
For the ``patchy'' landscape we see
extreme
bias in estimates of $N$ when the Euclidean distance is used. There is
moderate small sample bias of 3-5\% in the MLE of $N$ using the
least-cost distance which becomes negligible as $K$ increases. For
$N=200$ the bias is on the order of 2\% for the lowest sample size
case ($K=3$) but negligible otherwise.  Interestingly, for the
landscape exhibiting systematic trend, there is a persistent bias
in the MLE of $N$ of 1-3\% even for the highest level of $K$. We were
initially surprised by this but, in fact, it results from
the state-space being small relative to the extent of the trap grid, and
sensitivity to a state-space that is too small is expected because the
support of the integrand is truncated. In the particular case of the
systematic landscape, we find that, on some edges of the landscape
where cost of movement is low, individuals use large areas of space,
and the fitted model is under-stating the apparent
heterogeneity in encounter probability for the prescribed landscape.  We
found that the issue is resolved when the traps are moved away from
the boundary (see Appendix 4).

The performance of estimating the cost parameter $\theta_{2}$ mirrors
the results for estimating $N$ for the prescribed area
(Appendix 4). In the
patchy landscape where we don't expect a systematic gradient in space
usage around the edge of the state-space, we find
that $\theta_{2}$ is estimated with
diminishing bias as the sample size increases, but with persistent
bias due to truncation of the likelihood for the systematic
landscape which, as with the MLE of $N$, is resolved by moving the
traps away from the edge of the prescribed landscape ${\cal S}$. Equivalently, in practice,
this could be resolved by expanding ${\cal S}$ away from the trap
locations so that all regions used by animals exposed to capture are
included in ${\cal S}$.



\section{Discussion}


\hl{Focus on connectivity first}

Interest in SCR models has grown rapidly in the last several years
since their formalization using likelihood
\citep{borchers_efford:2008} and Bayesian \citep{royle_young:2008}
inference methods. This is because of both the severe practical
limitations of classical non-spatial capture-recapture models, and
also the ubiquity of auxiliary spatial information in all
capture-recapture studies, and thus the universal applicability of SCR
models.
All applications of SCR models have been based on models for the
encounter probability that are functions of Euclidean
distance between individuals and traps. The obvious limitations are
that it is unaffected by landscape or habitat structure and implies
stationary, isotropic and symmetrical home ranges. These are standard
criticisms of the basic SCR model as universally applied in practice.

\hl{Somewhere make the distinction between within-season movements
  (which still can affect reproduction and other things) and
  dispersal, which is what people typically think about when the think
  about connectivity}

In this paper, we developed the first formal framework for integrating
``ecological distance'' into SCR models, where ecological distance is
defined as the minimum cost-weighted distance (i.e., ``least-cost
path'') between points, and where ``cost'' is characterized by one or
more spatially explicit covariates that are believed to influence
movement or space-usage of individuals.
How animals use space and therefore how distance to a trap is
perceived by individuals is not something that can ever be known. We
can only ever conjure up models to describe this phenomenon and fit
those models to limited data on a sample of individuals during a
limited amount of time.  Here we have shown that there is hope to
estimate parameters, from capture-recapture data, that describe how
animals use space and thereby allow for irregular home range geometry
that is influenced by landscape structure.

Not surprisingly, our simulation study demonstrated
(Table 2) that the MLE of model parameters is
approximately unbiased in moderate sample sizes. Moreover, the effect
of ignoring ecological distance and using normal Euclidean distance in
the model for encounter probability, has the logical effect of causing
negative bias in estimates of $N$.  We expect this because the effect
is similar to failing to model heterogeneity. i.e., if we mis-specify
``model Mh'' with ``Model M0''
\citep{otis_etal:1978} then we will expect to under-estimate $N$. So
the effect of mis-specifying the ecological distance metric with a
standard homogeneous Euclidean distance has the same effect. As a
practical matter, it stands to reason that many previous applications
of SCR models based on homogeneous distance metrics have under-stated
density of the focal population.

In our view, this bias is not the most important reason to
consider models of ecological distance. Rather, inference about the
structure of ecological distance is fundamental to many problems in
applied and theoretical ecology related to modeling landscape
connectivity, corridor and reserve design, population viability
analysis, gene flow, and other phenomena.  Our  model allows
investigators to evaluate landscape factors that influence movement of
individuals over the landscape from non-invasively collected
capture-recapture data.  Therefore SCR models based on ecological
distance metrics might aid in corridor design and understanding other
aspects of space usage and movement in animal populations.

We adopted a standard approach to inference under our model based on
marginal likelihood. In principle,
Bayesian analysis does not pose any unique challenges for this
class of models, except that computing the cost-weighted distance is
computationally intensive and having to do this at each iteration of
an MCMC algorithm may be impractical.
Some additional extensions of the model may be of general interest.
We have used least-cost paths here to represent ecological distance
although other distance metrics could be used, including circuit
resistance distances \citep{mcrae:2006}.
Instead of
characterizing cost with explicit covariates it might be possible to
estimate the ``resistance surface'' as a latent field, much as
\citep{wikle:2003} did in the developing of models of species spread
based on a diffusion process. He defined the spatially-explicit rate
of diffusion, $\delta(x)$, as a Gaussian spatial process and it was
estimated from the data.









\newpage


\bibliography{../AndyRefs_alphabetized}


\end{spacing}



\clearpage

\newpage


\begin{table}[h!]
{\small
\caption{Simulation results for estimating population size $N$ for a
  prescribed landscape with
$N=100$ or $N=200$ and various levels of replication ($K$) chosen to affect the observed sample
size of individuals (see Appendix 1 for details). For each simulated data set, the SCR model was fitted with
standard Euclidean distance (``euclid''), least-cost path assuming the
cost parameter $\theta_2$ is known (``lcp/known''), or allowing it to
be estimated by maximum likelihood (``lcp/est'').
The summary statistics of the
sampling distribution reported are the mean, standard deviation
(``SD'') and quantiles (0.025, 0.50, 0.975).
}
{\bf Systematic trend landscape:} \\
\begin{tabular}{l|rrrrr|rrrrr}
         & \multicolumn{5}{c}{N=100   } & \multicolumn{5}{c}{N=200  }  \\
         &   mean &  SD  & 0.025 & 0.50 & 0.975  & mean  & SD   & 0.025 & 0.50  & 0.975 \\ \hline
K=3      &        &      &       &      &        &       &      &       &       &       \\
euclid   &   63.65& 12.62& 44.77 & 61.17&  90.98 & 126.68& 17.05&  98.93& 124.49& 168.26 \\
lcp/known&   99.28& 20.80& 68.83 & 97.55& 152.59 & 196.47& 27.39& 152.03& 192.96& 259.78\\
lcp/est  &  101.93& 21.68& 67.95 &101.56& 156.21 & 201.58& 28.14& 154.96& 200.15& 263.20\\
K=5      &        &      &       &      &        &       &      &       &       &        \\
euclid   &  64.60 & 7.11 & 51.52 & 63.86&  77.33 & 130.02& 10.25& 113.48& 128.96& 151.32\\
lcp/known&  95.96 &11.64 & 74.21 & 96.16& 117.65 & 193.04& 17.13& 166.84& 191.88& 226.16\\
lcp/est  &  98.94 &12.97 & 74.68 & 99.00& 123.88 & 198.80& 19.60& 166.87& 197.97& 239.46\\
K=10     &        &      &       &      &        &       &      &       &       &       \\
euclid   &  69.24 & 4.83 & 59.37 & 69.47&  79.18 & 139.83&  7.62& 125.65& 139.65& 154.82\\
lcp/known&  94.46 & 7.04 & 81.45 & 94.04& 108.83 & 190.47& 11.55& 170.49& 189.74& 213.19\\
lcp/est  &  97.53 & 8.18 & 82.02 & 97.62& 113.16 & 195.19& 13.28& 171.63& 194.58& 217.96\\ \hline
\end{tabular}
\\
{\bf Patchy "random" landscape: } \\
\begin{tabular}{l|rrrrrrrrrr}
         & \multicolumn{5}{c}{N=100  } & \multicolumn{5}{c}{N=200   }  \\
         &   mean &  SD  & 0.025 & 0.50  & 0.975  & mean  & SD   & 0.025 & 0.50  & 0.975 \\ \hline
K=3      &        &      &       &       &        &       &      &       &       &       \\
euclid   &  78.68 & 18.12& 49.40 & 76.34 & 125.47 & 154.34& 33.74& 107.00& 146.34& 221.43\\
lcp/known& 109.09 & 27.52& 69.50 &104.86 & 183.72 & 207.18& 46.53& 143.31& 198.42& 315.89\\
lcp/est  & 110.96 & 28.65& 69.55 &106.98 & 181.84 & 208.77& 49.29& 141.68& 197.89& 325.77\\
K=5      &        &      &       &       &        &       &      &       &       &        \\
euclid   &  77.85 & 11.55& 59.17 & 77.44 & 101.14 & 153.39& 15.57& 129.31& 149.54& 185.38\\
lcp/known& 103.57 & 15.83& 78.15 &100.58 & 137.48 & 201.57& 21.25& 165.94& 199.95& 243.26\\
lcp/est  & 104.44 & 15.79& 78.38 &101.47 & 139.55 & 200.91& 20.78& 164.42& 200.47& 246.46\\
K=10     &        &      &       &       &        &       &      &       &       &       \\
euclid   &  78.01 & 5.26 & 68.00 & 77.96 & 87.81  & 156.27&  8.51& 142.17& 156.05& 174.55\\
lcp/known&  99.84 & 7.09 & 86.86 & 99.84 & 114.11 & 198.64& 11.04& 181.43& 197.62& 220.45\\
lcp/est  & 100.42 & 7.56 & 86.72 &100.34 & 115.47 & 198.45& 11.44& 180.06& 198.04& 219.52\\ \hline
\end{tabular}
}
\label{tab.results1}
\end{table}





\begin{comment}

\begin{table}[ht]
\centering
\caption{
Mean of sampling distribution of the cost function parameter
$\theta_{2}$ for the different simulation
conditions.
}
\begin{tabular}{l|rrrr}
 & \multicolumn{2}{c}{Patchy} & \multicolumn{2}{c}{Systematic} \\
    & N=100 &  N=200  &   N=100 &  N=200  \\ \hline
K=3 &   1.05&    1.03 &     1.17 & 1.14 \\
K=5 &   1.02&    1.01 &     1.12 &1.12 \\
K=10&   1.01&    1.00 &     1.10 &1.08 \\
\end{tabular}
\label{tab.results2}
\end{table}



\end{comment}






\newpage

{\flushleft \bf LIST OF FIGURES:}

\vspace{.2in}


{\flushleft \bf
Figure 1:}
Typical home ranges for 6 individuals based on the ``patchy'' cost surface.
The black dot indicates the home
  range center and the pixels around each home range center are shaded
according to the probability of encounter, if a trap were located in
that pixel.



\vspace{.2in}


{\flushleft \bf Figure 2:}
Two landscape covariates used for simulations. A hypothetical
  realization of $N=100$ activity centers is superimposed on the left,
along with 16 trap locations.


\newpage


% Figures


\begin{figure}
\begin{center}
\includegraphics[height=5in,width=6in]{figs/home_rangesv2}
\end{center}
\caption{
Typical home ranges for 6 individuals based on the ``patchy'' cost surface.
The black dot indicates the home
  range center and the pixels around each home range center are shaded
according to the probability of encounter, if a trap were located in
that pixel.
}
\label{fig.homeranges}
\end{figure}

\clearpage
\newpage


\begin{figure}
\begin{tabular}{cc}
\includegraphics[height=3.25in,width=3.25in]{figs/raster_withN100}
\includegraphics[height=3.25in,width=3.25in]{figs/raster_krige} &
\end{tabular}
\caption{
Two landscape covariates used for simulations. A hypothetical
  realization of $N=100$ activity centers is superimposed on the left,
along with 16 trap locations.
}
\label{ecoldist.fig.raster100}
\end{figure}




\clearpage

\newpage




\end{document}






