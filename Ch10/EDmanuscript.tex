\documentclass[12pt]{article}

\usepackage[total={6.5in,8.75in}, top=2.4cm, left=2.4cm]{geometry}
\usepackage{lineno}
\usepackage{amsmath}
%\usepackage{amssymb}    % used for symbols in figure legends
\usepackage{graphicx}
\usepackage[round,colon,authoryear]{natbib}

\usepackage{bm}
\usepackage{float}
\usepackage{amsmath}
\usepackage{amsfonts}
\usepackage{hyperref}
\usepackage{verbatim}
\usepackage{soul}
\usepackage{color}
\usepackage{setspace}

\bibliographystyle{../asa} % kluwer, plos-natbib, pnas-natbib


\title{Ecological Distance in Spatial Capture-Recapture Models}

\author{
{\bf J. Andrew Royle}\\
USGS Patuxent Wildlife Research Center, Laurel MD \\ \\
{\bf Richard B. Chandler} \\
USGS Patuxent Wildlife Research Center, Laurel MD\\ \\
{\bf Kimberly D. Gazenski} \\
USGS Patuxent Wildlife Research Center, Laurel MD\\ \\
{\bf Tabitha A. Graves} \\
Northern Arizona University, Flagstaff AZ \\ \\
}



\begin{document}

\maketitle

\date

\newpage

\linenumbers

\begin{flushleft}
{\em \bf Abstract}
Spatial capture-recapture (SCR) models provide a framework for estimating density of 
animal populations using individual encounter history data with auxiliary spatial information. 
To date, all applications of SCR models have used encounter probability models based on 
the Euclidean distance between traps and animal activity centers. These models assume 
that home ranges are stationary and symmetric, and that they are unaffected by landscape 
or habitat structure. In this paper we devise encounter probability models based on 
``ecological distance'', i.e., the least-cost distance between traps and activity centers, 
which is a function of both Euclidean distance and animal movement behavior in heterogeneous 
landscapes. We integrate least-cost path models into a likelihood-based estimation
scheme for spatial capture-recapture models in order to estimate population density and 
parameters of the least-cost encounter probability model.  
Therefore, it is possible to make explicit inferences about animal density, distribution, 
and landscape connectivity as it relates to animal movement from standard capture-recapture data.
Furthermore, a simulation study demonstrated that density estimators under the naive SCR 
model can be substantially biased when home ranges are not symmetric. 

{\em \bf Key words:} animal movement, ecological distance, landscape connectivity,
least-cost path, spatial capture-recapture



\end{flushleft}


\begin{spacing}{1.8}

\section{Introduction}


%% this material is a general introduction for a manuscript
Spatial capture-recapture models are a relatively new class of models
for estimating animal density from capture-recapture data with
auxiliary information about individual capture locations
\citep{borchers_efford:2008, royle_young:2008, efford_etal:2009,
  royle_etal:2009ecol}.
The basic idea is to express encounter probability of
individuals as a function of the distance between individual center of
activity, say ${\bf s}_{i}$, and trap location, say ${\bf x}_{j}$.  In
these models ${\bf s}_{i}$ is regarded as a latent variable and
conventional methods of statistical inference either based on marginal
likelihood \citep{borchers_efford:2008} or Bayesian analysis by MCMC
\citep{royle_young:2008}.

While the models are a relatively recent innovation, their use is
already becoming widespread \citep{efford_etal:2009,
  gardner_etal:2010, kery_etal:2010, efford:2011UO,
  gopalaswamy_etal:2012, foster_harmsen:2012} because they resolve critical problems with using
ordinary non-spatial capture-recapture methods such as ill-defined
area sampled, and heterogeneity in encounter probability due to the
juxtaposition of individuals with traps, and they provide a framework
for modeling of trap-specific covariates.  Furthermore, essentially
all real capture-recapture studies produce auxiliary spatial
information and therefore SCR models are directly relevant to standard
data collected from such studies.
% Indeed, the use of ordinary
%capture-recapture models essentially admits a model misspecification
%(i.e. homogeneous encounter probability) by ignoring the explicit
%spatial information.

Every application of SCR models so far has been based on encounter
probability models in which distance between individual activity
center and trap location is defined by a function of simple Euclidean
distance.  While these will often be sufficient for practical
purposes, especially in small data sets, sometimes developing more
complex models of the detection process as it relates to space usage
of individuals will be useful.  Animals may not judge distance in
terms of Euclidean distance but, rather, according to quality of local
habitat, landscape connectivity, perceived mortality risk, and other
considerations that statisticians might not really
understand.
As an example of the potential problem of parameterizing SCR models
using Euclidean distance, imagine a study area bisected by a large
semi-permeable barrier. In traditional SCR models, the probability of
capturing an animal in a trap located on the opposite side of the
barrier would simply be a function of distance, whereas in reality it
should be a function of both distance and the permeability of the
barrier. Such situations are extremely common in capture-recapture
studies where multiple habitats occur in the study area or when
animals use linear features such as trails, corridors, or rivers.
 Moreover, because encounter probability and the distance
metric upon which it is based represent outcomes of individual
movements about their home range, ecologists might have explicit
hypotheses about how environmental variables affect the distance
metric, and it is therefore desirable to incorporate these hypotheses
directly into SCR models so that they may be formally evaluated
statistically.




In this paper we develop models for encounter probability based on
alternative distance metrics that account for ecological
considerations -- which, in keeping with the conventions in the
ecological literature, we will call ``ecological distance''. In
particular, we adopt a cost-weighted distance metric (the least-cost path)
used widely in landscape ecology for modeling connectivity,
movement and gene flow
\citep{adriaensen_etal:2003,manel_etal:2003,mcrae_etal:2008}. In the
context of SCR models we can use this as the basis for computing the
distance between traps and individuals activity centers. In this way
we can explicitly accommodate landscape structure and
account for how animals use space in SCR models. We develop a
likelihood-based inference framework for SCR model parameters using
this new distance metric when the ecological distance function is
known.  We show that the MLEs are approximately unbiased in moderate
sample sizes, as expected, but also that the misspecified model based
on Euclidean distance can produce substantial bias in estimates of $N$
and hence density.  Further, we extend the model to allow for likelihood 
estimation of parameters of the cost function, so that direct inference
about movement connectivity can be made from capture-recapture data without subjective prescription
of resistance or cost surfaces.


\section{Spatial Capture-Recapture}

A number of distinct observation models have been proposed for
spatial-capture situations \citep{borchers_efford:2008,
  royle_etal:2009ecol, efford_etal:2009ecol}, including Poisson, multinomial or binomial
observation models. The most
appropriate observation model for a dataset depends upon the the
sampling method.
For example, when animals are physically captured and retained
in traps, then the multinomial model is appropriate because it does not
allow for an individual to be captured in multiple traps during a
single occasion. Here we focus on the binomial model in which we
suppose that $J$ traps at locations ${\bf x}_{j}$ are operated for $K$
periods (e.g., nights) although our development of cost-distance
models is directly applicable to other observation models without any
further technical considerations. The binomial model is most directly
relevant to devices such as ``hair snares''
\citep{woods_etal:1999,gardner_etal:2010} or scent sticks
\citep{kery_etal:2010} for which individuals can only be encountered a
single time per observation occasion.

The observations are the individual- and trap specific counts $y_{ij}$
which are binomial counts with sample size $K$ and probabilities
$p_{ij}$. The vector of trap-specific counts for an individual $i$,
 ${\bf y}_{i} = (y_{i1},\ldots,y_{iJ}$ is its {\it encounter history}.
A standard encounter probability model
\citep{borchers_efford:2008} is the Gaussian model in which

\begin{equation}
log(p_{ij})= \theta_{0} + \theta_{1} d({\bf x}_{j} - {\bf s}_{i})^{2}
\label{eq.encounter}
\end{equation}

{\flushleft or, equivalently, }
\[
p_{ij} = \lambda_{0} exp(-  d({\bf x}_{j} - {\bf s}_{i})^{2}
/(2\sigma^{2}) )
\]
where $\theta_{0} = log(\lambda_{0})$ and $\theta_{1} =
-1/(2\sigma^2)$.


Alternative detection
models are commonly used, but they are all functions of Euclidean
distance and so we do not consider them further. However, the
distance metric we consider subsequently can be used in conjunction
with any other detection model.
In all previous applications of SCR models the normal Euclidean
distance has been used, i.e., $ d({\bf x}_{j} - {\bf s}_{i}) =
||{\bf x}_{j} - {\bf s}_{i}||$, and the parameters $\theta_0$ and
$\theta_1$ have been estimated using standard methods (likelihood or
Bayesian). These methods regard ${\bf s}_{i}$ as realizations of a
point process (i.e., latent variables) and remove them from the
likelihood either under a model of ``uniformity'' in which ${\bf s}
\sim \mbox{Uniform}({\cal S})$ where ${\cal S}$ is the state-space of
the point process, or a model in which covariates might affect the spatial
distribution of individuals \citep{borchers_efford:2008}. The
state-space
${\cal S}$ defines the potential values for any activity center ${\bf
  s}$.
It could be a polygon coverage containing all possible habitat in a
region, or a regular rectangle containing the trap array that defines
the limits beyond which individuals are unlikely to be encountered. 

The critical assumption  that motivates our work is that
 Euclidean distance metric is unaffected by habitat or landscape
structure, and it implies that space usage of individuals is
stationary and symmetric which may be unreasonable in most
applications.  For example, if the common detection model based on a
bivariate
normal probability distribution function is used, then the implied
space
usage by {\it all} individuals, no matter their location in space or
local habitat conditions, is symmetric with circular contours of
usage intensity (density contours of the pdf).
Subsequently we provide an extension of this class of SCR models that
accommodates alternative distance metrics that explicitly incorporate
information about the landscape so that a unit of distance is variable
depending on identified covariates. Thus, ``where'' an individual
lives on the landscape, and the state of the surrounding landscape,
will determine the character of its usage of space. In particular, we
suggest distance metrics that imply irregular, asymmetric and
non-stationary home ranges of individuals. An example of this is shown
in Fig. \ref{fig.homeranges} which shows home ranges of 6 individuals under
a specific model, which is described below.

\begin{figure}
\begin{center}
\includegraphics[height=6in,width=3.75in]{figs/home_ranges}
\end{center}
\caption{
Typical home ranges for 6 individuals based on the cost surface shown in
  Fig. \ref{ecoldist.fig.raster100} with $\theta_{2}=1$. The black dot indicates the home
  range center and the pixels around each home range center are shaded
according to the probability of encounter, if a trap were located in
that pixel. 
}
\label{fig.homeranges}
\end{figure}


\section{Cost Distance}

We adopt the use of a cost-weighted distance metric here which
defines the
distance between points by accumulating pixel-specific costs determined under
a cost function defined by the user.
The idea of cost-weighted distance to characterize animal use of
landscapes  is widely used in
landscape ecology for modeling connectivity, movement and gene flow
\citep{beier_etal:2008}. As is customary for reasons of
computational tractability we consider a discrete landscape defined by a
raster of some prescribed resolution. The distance between any two points
${\bf x}$ and ${\bf x}'$ can be represented by a sequence of line segments
connecting neighboring  pixels say
${\bf l}_{1},{\bf l}_{2},\ldots,{\bf l}_{m}$. Then the cost-weighted distance
between ${\bf x}$ and ${\bf x}'$ is

\begin{equation}
 d({\bf x},{\bf x}')
  =  \sum_{i=1}^{m-1} cost({\bf l}_{i},{\bf l}_{i+1})||{\bf l}_{i} - {\bf l}_{i+1}||
\label{eq.costweighted}
\end{equation}

{\flushleft
where } $cost({\bf l}_{i},{\bf l}_{i+1})$ is the user-defined cost function
to move
from pixel ${\bf l}_{i}$ to neighboring pixel ${\bf l}_{i}$ in the sequence.
Given the ``cost'' of each pixel, it is a simple matter to compute the
cost-weighted distance between any two pixels, along {\it any} path,
simply by accumulating the incremental  costs weighted by
distances.
In the context of
spatial capture-recapture models (and, more generally, landscape
connectivity) we are concerned with the {\it minimum} cost-weighted
distance, or the {\it least-cost path}, between any two points which
we will denote by $d_{lcp}$, which is
the
sequence ${\bf l}_{1},{\bf l}_{2},\ldots,{\bf l}_{m}$ that minimizes
the objective function defined by Eq. \ref{eq.costweighted}. That is,

\begin{equation}
 d_{lcp}({\bf x},{\bf x}')
  =  min_{{\bf l}_{1},\ldots,{\bf l}_{m}}  \sum_{i=1}^{m-1} cost({\bf l}_{i},{\bf l}_{i+1})||{\bf l}_{i} - {\bf l}_{i+1}||
\label{eq.lcp}
\end{equation}

{\flushleft
 Least-cost} path distance can be calculated in
 many geographic information systems and other software packages,
including the {\bf R} package \mbox{\tt
  gdistance} \citep{vanetten:2011}.


The key ecological aspect of least-cost path modeling is the
development 
of models for pixel-specific cost. 
In this paper we model cost as a function of covariates
defined on every pixel of the raster according to, for example using a
single covariate $z(x)$:

\begin{equation}
 log(cost({\bf x}))=  \theta_{2} z({\bf x})
\label{eq.cost}
\end{equation}

{\flushleft This} is interpreted as cost of moving {\it through} a pixel but, in practice, it is accumulated 
incrementally as the cost of moving from a pixel ${\bf x}$ to one of its neighbors, say ${\bf x}'$. As such, it is conventional
to express this as an average cost 

\[
 log(cost({\bf x},{\bf x}'))=  \theta_{2} \frac{z({\bf x})+z({\bf x}')}{2}
\]

{\flushleft
Thus,} if $\theta_{2} = 0$ then substituting $cost({\bf x},{\bf x}')$ into
Eq. \ref{eq.lcp} will produce the ordinary Euclidean distance
between points.

In practical applications, variables that
influence the cost of moving across the landscape include things like
highways  \citep[e.g.,][]{epps_etal:2005}, elevation
\citep{cushman_etal:2006},
ruggedness \citep{epps_etal:2007}, snow cover \citep{schwartz_etal:2009},
distance to escape terrain \citep{shirk_etal:2010}, or range limitations
\citep{mcrae_beier:2007}.  Together multiple environmental variables create
a resistance surface, which forms the linchpin of all connectivity
planning \citep{spear_etal:2010}.  Often $\theta_{2}$ is fixed by the
investigator. Although
 $\theta_{2}$ is rarely known,
 conservation
biologists design linkages that require this resistance value as input
\citep[see][and articles cited therein]{beier_etal:2008}.  Typically planners pick a value based on
expert opinion \citep{beier_etal:2008}, although recently researchers
have begun to define costs based on resource selection functions,
animal movement \citep{tracy:2006, fortin_etal:2005}, or genetic
distance data (e.g., \citet{gerlach_musolf:2000};
\citet{epps_etal:2007}; \citet{schwartz_etal:2009}. The published
methods require many assumptions and these estimates have not been
integrated with capture-recapture methods, nor have parameters been
formally estimated using capture-recapture data.

To formalize the use of cost-weighted distance in SCR models, we
substitute Eq. \ref{eq.lcp} in the expression for encounter
probability (Eq. \ref{eq.encounter}) and maximize the resulting
likelihood (see {\it Maximum Likelihood Estimation}), including for estimating the
parameter $\theta_{2}$.


\subsection{Example of Computing Cost-weighted distance}

As an example of the cost-weighted distance calculation consider the
following landscape comprised of 16 pixels with unit spacing
identified as follows, along with the pixel-specific cost:
\begin{center}
\begin{verbatim}
  pixel ID                 Cost
  1  5  9  13          100   1   1  1
  2  6 10  14          100 100   1  1
  3  7 11  15          100 100 100  1
  4  8 12  16          100 100   1  1
\end{verbatim}
\end{center}
Then we assigned low cost of 1 to ``good habitat'' pixels (or pixels we
think of as ``highly connected'' by virtue of being in good habitat)
and, conversely, we assign high cost (100) to ``bad habitat''. So the
shortest cost-weighted distance between pixels 5 and 9 in this example
is just 1 unit, the shortest cost-distance between pixels 5 and 10 is
$\sqrt{2}(1+1)/2 = 1.414214$ units, the shortest distance between pixels 4 and 8 is 100
units, while the shortest cost-distance between 4 and 12 is 150.5. A
tough one is: what is the shortest distance between 7 and 16? An
individual at pixel 7 can move diagonal and pay  $sqrt(2)*(100+1)/2 + 1 =72.41778$.
This simple cost raster is shown in 
Fig. \ref{ecoldist.fig.raster} ({\bf R} commands are given in Appendix
1).
Once the cost raster is created, the least-cost path distances are
computed with just a couple {\bf R} commands, and those can be
inserted directly into the likelihood construction for an ordinary
spatial capture-recapture model (Appendix 1). The {\bf R} package
\mbox{\tt gdistance} uses 
 the implementation of Dijkstra's algorithm
\citep{dijkstra:1959} found in the \mbox{\tt igraph} package
\citep{csardi:2010}.
Using \mbox{\tt gdistance}, (Appendix 1),
we define the incremental cost of moving
from one pixel to another as the distance-weighted {\it
  average} of the 2 pixel costs.


\begin{figure}
\begin{center}
\includegraphics[height=3.25in,width=3.25in]{figs/raster_2values}
\end{center}
\caption{A $4 \times 4$ raster with cost = 1 (white) or 100 (shaded) to represent ease of movement across a pixel.}
\label{ecoldist.fig.raster}
\end{figure}

\begin{comment}
Now we can look at the result and see if it makes sense to us. Here we
print the first 4 columns of this distance matrix and illustration a
couple of examples of calculating the minimum cost-weighted distance
between points:
\begin{verbatim}
> outD[1:5,1:5]
         1         2        3        4         5
1   0.0000 100.00000 200.0000 205.2426  50.50000
2 100.0000   0.00000 100.0000 200.0000  71.41778
3 200.0000 100.00000   0.0000 100.0000 171.41778
4 205.2426 200.00000 100.0000   0.0000 154.74264
5  50.5000  71.41778 171.4178 154.7426   0.00000
\end{verbatim}
\end{comment}

\subsection{Non-stationarity of home range structure}

When distance is defined by the cost-weighted distance metric given
by Eq. \ref{eq.lcp} then individual space-usage varies
spatially in response to the landscape covariate(s) used in the
distance metric. For example, using one of the covariates we use in
our simulation study below (Fig. \ref{ecoldist.fig.raster100}, right
panel) with a Gaussian pdf detection function but having distance
metric defined by Eq. \ref{eq.lcp}, produces home ranges such
as those shown in Fig. \ref{fig.homeranges}. Later we simulate data
under the model that produces these home ranges and fit spatial
capture-recapture models to evaluate the efficacy of likelihood
estimation under this model. 




\section{Maximum likelihood estimation}
\label{sec.mle}

Here we develop a standard method of parameter estimation based on
the marginal likelihood. That is, the likelihood in which the latent
variables ${\bf s}$ are removed by integration \citep{borchers_efford:2008}.

The individual- and trap-specific observations have a binomial
distribution conditional on the latent variable ${\bf s}_{i}$:

\begin{equation}
	y_{ij}| {\bf s}_{i} \sim \mbox{Bin}(K, p_{\theta}(d_{lcp}({\bf x}_{j},{\bf s}_{i};\theta_{2}); \theta_{0}, \theta_{1})
\label{mle.eq.cond-on-s}
\end{equation}

{\flushleft where} we have indicated the dependence of $p_{ij}$ on the parameters
${\bm \theta}$, and also $d_{lcp}$ which
itself depends on $\theta_{2}$, and the latent variable ${\bf s}$.
%The parameters
%${\bm \theta}$ include whatever parameters are involved in the
%cost-weighted distance function, i.e., at least $\theta_{2}$ from
%Eq. \ref{eq.cost}.
For the random effect we have ${\bf s}_{i} \sim  \mbox{Unif}({\cal
  S})$.
The joint distribution of the data for individual $i$ is the product
of $J$ binomial terms (i.e., contributions from each of $J$ traps):

\[
  [{\bf y}_{i} | {\bf s}_{i} , \theta] =
  \prod_{j=1}^{J} \mbox{Bin}(K, p_{\theta}({\bf x}_{j},{\bf s}_{i}) )
\]

{\flushleft This} assumes that encounter of individual $i$ in each
trap is independent of encounter in every other trap. Conditional on
${\bf s}_{i}$ this is reasonable in most applications in our view.
 The so-called marginal likelihood is computed by removing
${\bf s}_{i}$, by integration,  from the conditional-on-${\bf s}$
likelihood and regarding the {\it marginal} distribution of the data
as the likelihood. That
is, we compute:

\[
  [y|{\bm \theta}] =
\int_{{\cal S}}  [ {\bf y}_{i} |{\bf s}_{i},{\bm \theta}] g({\bf s}_{i}) d{\bf s}_{i}
\]

{\flushleft where}, under the uniformity assumption, we have
$g({\bf s}) = 1/||{\cal S}||$.
The joint likelihood for all $N$ individuals, assuming independence of
encounters among individuals, is the product of $N$ such terms:

\[
{\cal L}({\bm \theta} | {\bf y}_{1},{\bf y}_{2},\ldots, {\bf y}_{N}) = \prod_{i=1}^{N}
[{\bf y}_{i}|{\bm \theta}]
\]

The key operation for computing the likelihood is solving the
2-dimensional integration problem to remove ${\bf s}$. There are some
general purpose {\bf R} packages that implement a number of
multi-dimensional integration routines including \mbox{\tt adapt}
\citep{genz_etal:2007} and \mbox{\tt R2cuba} \citep{hahn_etal:2011}.
We won't rely on these extraneous {\bf R} packages but instead will
use perhaps less efficient methods in which we replace the integral
with a summation over an equal area mesh of points on the state-space
${\cal S}$ and explicitly evaluate the integrand at each point. We
invoke the rectangular rule for integration here in which the
integrand is evaluated on a regular grid of points of equal area and
then averaged.  Let $u=1,2,\ldots,nG$ index a grid of $nG$ points,
${\bf s}_{u}$, where the area of grid cell $u$ is constant.  In this
case, the integrand, i.e., the marginal pmf of ${\bf y}_{i}$, is
approximated by

\begin{equation}
         [{\bf y}_{i}|\theta] = \frac{1}{nG} \sum_{u=1}^{nG}  [ {\bf
            y}_{i} |{\bf s}_u, \theta]
\label{mle.eq.intlik}
\end{equation}

To deal with the fact that $N$ is unknown,
there are two key issues that need to be addressed.
First is that
we don't observe the ``all-zero'' encounter histories (i.e., $y_{ij} = 0$ for all $j$)
corresponding to 
uncaptured individuals, so we have to
make sure we compute the probability for that all zero encounter history which
we do operationally by tacking a row of zeros onto the encounter
history matrix. We include the number of such all-zero encounter histories
as an unknown parameter of the model, which we label
 $n_{0}$.
In addition, we have to be sure to include a combinatorial term to
account for the fact that of the $n$ observed individuals there are
${N \choose n}$
 ways to realize a sample of size $n$. The combinatorial term
involves the unknown $n_{0}$ and thus it must be included in the likelihood.

To compute the integral requires that the bounds of integration are
specified, which is equivalent to prescribing the state-space of the
underlying point process, i.e., ${\cal S}$. Given ${\cal S}$, density
is 
computed as $D({\cal S}) = N/||{\cal S}||$. In our simulation study
below we report $N$ as the two are equivalent summaries of the data
set once the state-space is fixed.

We wrote an {\bf R} function to evaluate the likelihood which we optimize
using the {\bf R} function \mbox{\tt nlm}.
The {\bf R} code is given as a script in the Appendix and the {\bf R}
package \mbox{\tt scr}
is available on the website
\url{https://sites.google.com/site/spatialcapturerecapture/}. The
\mbox{\tt scr} package also contains
functions for simulating data.


\section{Examples of SCR models based on ecological distance}

In this section we provide examples that we think are typical of
how cost-weighted distance models can be used in real
capture-recapture problems.
\begin{comment}
In particular, we will typically have a polygon coverage either in the
form of a GIS shapefile or a matrix of points or some other specific
format, and we want to put that polygon on a map and use the polygon
boundary in some way to generate pixel-specific costs. So we want to
see if points or raster pixels are in the polygon, or not, or how far
they are from the polygon boundary (cost might be related to distance)
and similar operations. In the following examples, we confront how to
do some of these operations in {\bf R}.
\end{comment}
We define a $20 \times 20$ pixel covariate raster with extent = $[0.5, 4.5] \times
[0.5, 4.5]$.  We regard for the purposes of this example as a coarse
landscape covariate, with pixels having some arbitrary scaling say, a
$2 \times 2$ km resolution. Thus, the raster therefore defines a
landscape of $40 \times 40$ km and we suppose that 16 camera traps are
established at the integer coordinates $(1,1), (1,2), \ldots,
(4,4)$. We could think of this as a landscape within which we're
studying a population of ocelots, lynx or some other cat.

For our analyses, cost is characterized by a
single covariate raster and we consider two specific cases. First is an
increasing trend from the NW to the SE (''systematic raster''), where $z(x)$ is defined as
$z(x) = r(x) + c(x)$ where $r(x)$ and $c(x)$ are just the row and
column, respectively, of the raster.  This might define something
related to distance from an urban area or a gradient in habitat
quality due to land use, or environmental conditions.  In the second
case we make up a covariate by generating a field of spatially
correlated noise to emulate a typical patchy habitat covariate (''patchy raster''). The
two covariates are shown in Fig. \ref{ecoldist.fig.raster100}, along
with a sample realization of $N=100$ individuals (left panel only).
 For
both covariates we use a cost function in which transitions from
pixel ${\bf x}$ to ${\bf x}'$ is given by:

\[
 log(cost({\bf x},{\bf x}'))=  \theta_2 \frac{z({\bf x}) + z({\bf x}')}{2}
\]

{\flushleft where} $\theta_2 = 1$ for simulating the observed data.
 With $\theta_2=0$ then the
model reduces to one in which the cost of moving across each pixel is
constant, and therefore Euclidean distance is operative.

\subsection{Simulation studies}

We devised a limited simulation study to evaluate three things: (1)
the general statistical performance of the density estimator under
this new model; (2) the effect of mis-specifying the model with a
normal Euclidean distance metric and (3) the statistical performance
of estimating the relative cost parameter.

We used population sizes of 100 and 200 individuals with activity
centers randomly distributed on the $20 \times 20$ landscape, and subjected them
to encounter by 16 traps arranged in a $4\times 4$ grid according to
the Euclidean distance metric. We fit 3 different models; (i) the
misspecified euclidean distance model; (ii) the true data-generating
model with the relative cost raster {\it known} and (iii) the true
data-generating model but estimating the relative cost parameter by
maximum likelihood.  We used the ``systematic'' and ``patchy''
covariates defined previously. 

We simulated encounter data for the $N$ individuals using the Gaussian
encounter model with least-cost path distance metric:

\[
log(p_{ij})= \theta_{0} + \theta_{1} d_{lcp}({\bf x}_{j},{\bf
  s}_{i}; \theta_{2})^{2}
\]

{\flushleft We } used here $\theta_{0} = -2$ and $\theta_{1} = 2$, the latter value
corresponding to $\sigma = 0.5$ of a stationary bivariate normal home
range model.  We varied the number of replicate samples $K=3,5,10$
(e.g., nights in a camera trapping study) to produce varying sample
sizes.  Because any simulation study is inherently arbitrary, we have
provided {\bf R} scripts for carrying out simulations in Appendix 2 so
that the interested reader can experiment with their own situations.

\begin{figure}
\begin{tabular}{cc}
\includegraphics[height=3.25in,width=3.25in]{figs/raster_withN100}
\includegraphics[height=3.25in,width=3.25in]{figs/raster_krige} &
\end{tabular}
\caption{Two covariate rasters used for simulations. A hypothetical
  realization of $N=100$ activity centers is superimposed on the left,
along with 16 trap locations. }
\label{ecoldist.fig.raster100}
\end{figure}

\subsection{Simulation Results}

For both landscapes and all simulation conditions (levels of $K$ and
$N$) the average sample sizes of individuals
captured are given in Tab. \ref{tab.samplesize}.
The simulation results for estimating $N$ for the prescribed
state-space 
are presented in Table \ref{tab.results1} below. 
For the ``patchy'' landscape we see 
extreme bias in estimates of $N$ when the Euclidean distance is
used. There is moderate small sample bias of 3-5\% in the MLE of $N$
which becomes marginal as $K$ increases. For $N=200$ the bias is
on the order of 2\% for the lowest sample size case ($K=3$) but
negligible otherwise. 
Interestingly, for the landscape exhibiting systematic structure, 
there is a persistent bias in the MLE of $N$ of 1-3\% even for the
highest level of $K$. We were initially surprised by this but then
realized this is due to the fact that the state-space is small
relative to the extent of the trap grid and sensitivity to a
state-space  that is too small is well-known. In the NW corner of the
raster where cost of movement is low, individuals should be moving all
over the place and the fitted model is under-stating the apparent
heterogeneity in encounter probability. 


\begin{table}[ht]
\centering
\caption{
Expected sample sizes of captured individuals under each configuration of 
$N$ (population size for the prescribed state-space) and $K$ (number of replicate samples). 
}
\begin{tabular}{l|rrrr}
 & \multicolumn{2}{c}{Systematic} & \multicolumn{2}{c}{Patchy}  \\
    & N=100 &  N=200  &   N=100 &  N=200  \\ \hline
K=3 &  38.69 &   78.17  &   37.30 &   74.93  \\
K=5 &  51.10 &  103.18  &   51.89 &  103.71 \\
K=10&  65.81 &  132.39  &   69.44 &  138.76 \\
\end{tabular}
\label{tab.samplesize}
 \end{table}


\begin{table}[ht]
{\small
\caption{Simulation results for estimating population size $N$ for a prescribed state-space with
$N=100$ or $N=200$ and various levels of replication ($K$) chosen to affect the observed sample
size of individuals (Tab. \ref{tab.samplesize}). For each simulated data set, the SCR model was fitted with
standard Euclidean distance, least-cost path assuming the cost parameter $\theta_2$ is known, or allowing it to
be estimated by maximum likelihood.}
{\bf Systematic trend raster:} \\
\begin{tabular}{l|rrrrr|rrrrr}
         & \multicolumn{5}{c}{N=100   } & \multicolumn{5}{c}{N=200  }  \\
         &   mean &  SD  & 0.025 & 0.50 & 0.975  & mean  & SD   & 0.025 & 0.50  & 0.975 \\ \hline
K=3      &        &      &       &      &        &       &      &       &       &       \\
euclid   &   63.65& 12.62& 44.77 & 61.17&  90.98 & 126.68& 17.05&  98.93& 124.49& 168.26 \\
lcp/known&   99.28& 20.80& 68.83 & 97.55& 152.59 & 196.47& 27.39& 152.03& 192.96& 259.78\\
lcp/est  &  101.93& 21.68& 67.95 &101.56& 156.21 & 201.58& 28.14& 154.96& 200.15& 263.20\\
K=5      &        &      &       &      &        &       &      &       &       &        \\
euclid   &  64.60 & 7.11 & 51.52 & 63.86&  77.33 & 130.02& 10.25& 113.48& 128.96& 151.32\\
lcp/known&  95.96 &11.64 & 74.21 & 96.16& 117.65 & 193.04& 17.13& 166.84& 191.88& 226.16\\
lcp/est  &  98.94 &12.97 & 74.68 & 99.00& 123.88 & 198.80& 19.60& 166.87& 197.97& 239.46\\
K=10     &        &      &       &      &        &       &      &       &       &       \\
euclid   &  69.24 & 4.83 & 59.37 & 69.47&  79.18 & 139.83&  7.62& 125.65& 139.65& 154.82\\
lcp/known&  94.46 & 7.04 & 81.45 & 94.04& 108.83 & 190.47& 11.55& 170.49& 189.74& 213.19\\
lcp/est  &  97.53 & 8.18 & 82.02 & 97.62& 113.16 & 195.19& 13.28& 171.63& 194.58& 217.96\\ \hline
\end{tabular} 
\\
{\bf Patchy "random" raster: } \\
\begin{tabular}{l|rrrrrrrrrr}
         & \multicolumn{5}{c}{N=100  } & \multicolumn{5}{c}{N=200   }  \\
         &   mean &  SD  & 0.025 & 0.50  & 0.975  & mean  & SD   & 0.025 & 0.50  & 0.975 \\ \hline
K=3      &        &      &       &       &        &       &      &       &       &       \\
euclid   &  78.68 & 18.12& 49.40 & 76.34 & 125.47 & 154.34& 33.74& 107.00& 146.34& 221.43\\
lcp/known& 109.09 & 27.52& 69.50 &104.86 & 183.72 & 207.18& 46.53& 143.31& 198.42& 315.89\\
lcp/est  & 110.96 & 28.65& 69.55 &106.98 & 181.84 & 208.77& 49.29& 141.68& 197.89& 325.77\\
K=5      &        &      &       &       &        &       &      &       &       &        \\
euclid   &  77.85 & 11.55& 59.17 & 77.44 & 101.14 & 153.39& 15.57& 129.31& 149.54& 185.38\\
lcp/known& 103.57 & 15.83& 78.15 &100.58 & 137.48 & 201.57& 21.25& 165.94& 199.95& 243.26\\
lcp/est  & 104.44 & 15.79& 78.38 &101.47 & 139.55 & 200.91& 20.78& 164.42& 200.47& 246.46\\
K=10     &        &      &       &       &        &       &      &       &       &       \\
euclid   &  78.01 & 5.26 & 68.00 & 77.96 & 87.81  & 156.27&  8.51& 142.17& 156.05& 174.55\\
lcp/known&  99.84 & 7.09 & 86.86 & 99.84 & 114.11 & 198.64& 11.04& 181.43& 197.62& 220.45\\
lcp/est  & 100.42 & 7.56 & 86.72 &100.34 & 115.47 & 198.45& 11.44& 180.06& 198.04& 219.52\\ \hline
\end{tabular}
}
\label{tab.results1}
\end{table}


mean of sampling distn of $\theta_{2}$

     patchy landscape     systematic landscape
      N=100   N=200        N=100   N=200
K=3    1.05    1.03
K=5    1.02    1.01
K=10   1.01    1.00




\section{Discussion}

Interest in SCR models has grown rapidly in the last several years
since their formalization using likelihood \citep{borchers_efford:2008}
and Bayesian \citep{royle_young:2008} inference methods. This is because
of both the severe practical limitations of classical non-spatial
capture-recapture models, and also the ubiquity of auxiliary spatial
information in all capture-recapture studies, and thus the universal
applicability of SCR models.

All applications of SCR models have been based on models for the
encounter probability that are functions of
the standard Euclidean distance between
individuals and traps. The obvious limitations are that it is
unaffected by landscape or habitat structure and implies
stationary,  isotropic and symmetrical home ranges. These are standard
criticisms of the basic SCR model as universally applied in practice.
In this paper,  we developed the first formal framework
for integrating
``ecological distance'' into SCR models, where ecological distance is
defined as the minimum cost-weighted distance (i.e., ``least-cost path'') between points, and where ``cost''
is characterized by one or more spatially explicit covariates that are
believed to influence movement or space-usage of individuals.

How animals use space and therefore how distance to a trap is
perceived by individuals is not something that can ever be known. We
can only ever conjure up models to describe this
phenomenon and fit those models to limited data on a sample of individuals
during a limited amount of time.
Here we have shown that there is hope to
estimate parameters, from capture-recapture data,
 that describe how animals use space and thereby
allow for irregular home range geometry that is influenced by
landscape structure.

Not surprisingly, our simulation study demonstrated 
(Tab. \ref{tab.results1})
that the MLE of model
parameters is approximately unbiased in moderate sample
sizes. Moreover, the
effect of ignoring ecological distance and using normal Euclidean
distance in the model for encounter probability, has the
logical effect of causing negative bias
in estimates of $N$. 
We expect this because the effect is similar to
failing to model heterogeneity. i.e., if we mis-specify ``model Mh'' \citep{otis_etal:1978} with
``Model M0'' \citep{otis_etal:1978} then we will expect to under-estimate $N$. So the effect of
mis-specifying the ecological distance metric with a standard
homogeneous Euclidean distance has the same effect. As a practical matter,
it stands to reason that many previous applications of SCR models based
on homogeneous distance metrics have under-stated density of the focal
population.

In our view, this bias is not really the most important reason to consider models of
ecological distance. Rather, inference about the structure of
ecological distance is fundamental to many problems in applied and
theoretical ecology related to modeling landscape
connectivity, corridor and reserve design, population viability
analysis, gene flow, and other phenomena.
Our new model
allows investigators to  evaluate landscape
factors that influence movement of individuals over the landscape. 
Therefore SCR models based on ecological distance metrics might aid in
corridor design and understanding other aspects of space usage and
movement in animal populations.

We adopted a standard approach to inference under our model based on
marginal likelihood \citep{borchers_efford:2008}. In principle,
Bayesian analysis does not pose any unique challenges for this new
class of models, except that computing the cost-weighted distance is
computationally intensive.
 So, having to do this at each iteration of an
MCMC algorithm may be impractical using existing algorithms.
A related issue is that the size of the raster slows things down. For
very large rasters, even likelihood analysis can be computationally
challenging and methods for efficient calculation of the ecological distance
given the raster covariate(s) and parameters might be needed.
Some additional extensions of the model may be of general
interest. 
We have used least-cost paths here to represent ecological distance although
other distance metrics could be used, including circuit resistance distances
\citep{mcrae:2006} which allows that animals can move through multiple paths 
in the landscape.
Instead of characterizing cost with 
explicit covariates it might be possible to estimate the
``resistance surface'' as a latent field, much as \citep{wikle:2003}
 did in the developing of models of species spread based on a
 diffusion process. He defined the spatially-explicit rate of
diffusion, $\delta(x)$, as a Gaussian spatial process and it was
estimated from the data.  












%\end{spacing}

\newpage


\bibliography{../AndyRefs_alphabetized}
\newpage

%%\bibliography{EDmanuscript}
% Tables


\end{spacing}

% Figures


\section*{Appendix 1: {\bf R} code for computing least-cost path
  distance and likelihood analysis of the SCR model}


The {\bf R} commands to create a raster containing the pixel-specific
costs are given as follows for a simple $4\times 4$ raster containing
values either $z=1$ (good habitat) or $z=100$ (bad habitat):
\begin{verbatim}
r<-raster(nrows=4,ncols=4)
projection(r)<- "+proj=utm +zone=12 +datum=WGS84"
extent(r)<-c(.5,4.5,.5,4.5)
costs1<- c(100,100,100,100,1,100,100,100,1,1,100,1,1,1,1,1)
values(r)<-matrix(costs1,4,4,byrow=FALSE)
\end{verbatim}
The helper functions \mbox{\tt transition}, which
calculates the cost of moving between neighboring pixels, and
\mbox{\tt geoCorrection} which modifies the costs of moving diagonally
by the additional distance, produce output which feeds into the
function \mbox{\tt costDistance} to compute the pair-wise distance
matrix. For that, we define the center points of each raster.  The
commands altogether are as follows:
\begin{comment}
# The transition function specifies that we'd like to use the mean of
# 2 neighboring pixels as the weight and then takes the inverse to convert
# the costs into conductances, which algorithms within gdistance and
# igraph require. We use the argument directions= 8 to specify that
# all 8 touching pixels are neighbors.)
\end{comment}
\begin{verbatim}
values(r)<-matrix(cost1,20,20,byrow=FALSE)
tr1<-transition(r,transitionFunction=function(x) 1/mean(x),directions=8)
tr1CorrC<-geoCorrection(tr1,type="c",multpl=FALSE,scl=FALSE)
pts<-cbind( sort(rep(1:4,4)),rep(4:1,4))
costs1<-costDistance(tr1CorrC,pts)
outD<-as.matrix(costs1)
\end{verbatim}





\end{document}






