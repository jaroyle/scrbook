\documentclass[12pt]{article}

\usepackage[total={6.5in,8.75in}, top=2.4cm, left=2.4cm]{geometry}
\usepackage{lineno}
\usepackage{amsmath}
%\usepackage{amssymb}    % used for symbols in figure legends
\usepackage{graphicx}
\usepackage[round,colon,authoryear]{natbib}

\usepackage{bm}
\usepackage{float}
\usepackage{amsmath}
\usepackage{amsfonts}
\usepackage{hyperref}
\usepackage{verbatim}
\usepackage{soul}
\usepackage{color}
\usepackage{setspace}

\bibliographystyle{ecology} % kluwer, plos-natbib, pnas-natbib


\title{Modeling Resource Selection in 
 Spatial Capture-Recapture Models}

\begin{small}
\author{
{\bf J. Andrew Royle}, xyz, 
USGS Patuxent Wildlife Research Center, Laurel MD \\ \\
}
\end{small}


\begin{document}

\maketitle

\date


\linenumbers


\begin{spacing}{1.8}

\begin{flushleft}
{\em \bf Abstract}


{\em \bf Key words:} animal movement, ecological distance, 
landscape connectivity,
resource selection functions, 
 spatial capture-recapture

\end{flushleft}



\section{Introduction}

SCR is the shit

RSF is the shit

SCR + RSF? !!!!





\section{Model Formulation}

No landscape on your computer is continuous.

RSF model:
 telemetry fixes produces n(i,j) = number of fixes for individual i in
 pixel j.  The standard RSF model is that, conditional on the total
 number of fixes, these are multinomial random variables with
 probabilities
\[
 \pi_{i,j} = \frac{ exp( -\beta z(x) ) }{ \sum_{x} exp(-\beta z(x))} 
\]
this is the same as saying
\[
 n(i,j) \sim Poisson( \lambda_{ij})
\]
where
\[
 log(\lambda_{ij}) = \beta_{0}  + \beta_{1} z(x)
\]
Intercept here is completely fixed. We decide on that. How many fixes
do we wish to have?



So the key to combing RSF data with SCR data is to think about this
Poisson model. In SCR data the total encounter frequency is a random
variable and, in fact, we only observe the event $n_{ij}>0$ but it
should have the same mean, just a different intercept. Thus when we
observe $y_{ijk} = 1$ this occurs if the individual visited the pixel
containing a trap, i.e.,  $m_{ij}>0$, and we detected it. in this
construction $m_{ij}$ is a ``detectable'' visit to a pixel, say of an
individual wandering in the vicinity of a trap or a trail that is
sampled. 
We imagine a hierarchy like this:
\[
m_{ij} \sim Bin(n_{ij}, p_{0})
\]
therefore
\[
m_{ij} \sim Poisson( \lambda_{ij} p_{0} )
\]
and
we observe the binary event $y_{ijk} = 1$ if $m_{ijk}>0$. Therefore 
this occurs with probability 
\[
 p_{ij} = 1-exp(- \lambda_{ij} p_{0} ) 
\]
we see that $p_{0}$ and a baseline rate of use  $exp(\beta_{0})$ are
all balled up together which makes sense. 
think of this as follows: we have some detector sitting there in a
pixel and we sample continuously and $p_{0}$ is the rate of encounter
given that an individuals visits a pixel. 




one thing about this is if you have no covariates at all then
the telemetry RSF function is proportional exactly to the SCR 
detection model? Well that really isn't quite true. But if 
we use a half-normal detection model then , in that case, 
\[
p_{ij} \propto RSF
\]
but we can't just set $p_{ij} = RSF$ because RSF's add to 1 over space
but $p_{ij}$ don't. 



\section{Maximum likelihood estimation}
\label{sec.mle}

Here we outline a standard method of parameter estimation based on
marginal likelihood. That is, the likelihood in which the latent
variables ${\bf s}$ are removed by integration \citep{borchers_efford:2008}.
The individual- and trap-specific observations have a binomial
distribution conditional on the latent variable ${\bf s}_{i}$:

\begin{equation}
	y_{ij}| {\bf s}_{i} \sim \mbox{Bin}(K, p_{\theta}(d_{lcp}({\bf x}_{j},{\bf s}_{i};\theta_{2}); \theta_{0}, \theta_{1})
\label{mle.eq.cond-on-s}
\end{equation}

{\flushleft where} we have indicated the dependence of $p_{ij}$ on the parameters
${\bm \theta}$, and also $d_{lcp}$ which
itself depends on $\theta_{2}$, and the latent variable ${\bf s}$.
%The parameters
%${\bm \theta}$ include whatever parameters are involved in the
%cost-weighted distance function, i.e., at least $\theta_{2}$ from
%Eq. \ref{eq.cost}.
For the random effect we have ${\bf s}_{i} \sim  \mbox{Unif}({\cal
  S})$.
The joint distribution of the data for individual $i$ is the product
of $J$ binomial terms (i.e., contributions from each of $J$ traps):
$  [{\bf y}_{i} | {\bf s}_{i} , \theta] =
  \prod_{j=1}^{J} \mbox{Bin}(K, p_{\theta}({\bf x}_{j},{\bf s}_{i}) )$.
%This assumes independence of capture in each trap.
%Conditional on
%${\bf s}_{i}$ this is reasonable in most applications in our view.
 The so-called marginal likelihood is computed by removing
${\bf s}_{i}$, by integration,  from the conditional-on-${\bf s}$
likelihood and regarding the {\it marginal} distribution of the data
as the likelihood. That
is, we compute:

\[
  [y|{\bm \theta}] =
\int_{{\cal S}}  [ {\bf y}_{i} |{\bf s}_{i},{\bm \theta}] g({\bf s}_{i}) d{\bf s}_{i}
\]

{\flushleft where}, under the uniformity assumption, we have
$g({\bf s}) = 1/||{\cal S}||$.
The joint likelihood for all $N$ individuals, 
is the product of $N$ such terms:

\[
{\cal L}({\bm \theta} | {\bf y}_{1},{\bf y}_{2},\ldots, {\bf y}_{N}) = \prod_{i=1}^{N}
[{\bf y}_{i}|{\bm \theta}]
\]

Technical details for computing the likelihood and obtaining the MLEs
are given in Appendix 2 where we provide an ${\bf R}$ function
to evaluate the likelihood and obtain the MLEs.
A key practical detail is that the likelihood here is formulated in
terms of the parameter $N$, the population size for the landscape
defined by ${\cal S}$. Given ${\cal S}$, density
is
computed as $D({\cal S}) = N/||{\cal S}||$. In our simulation study
below we report $N$ as the two are equivalent summaries of the data
set once ${\cal S}$ is defined.


\section{Illustration}

In this section we provide examples that we think are typical of how
cost-weighted distance models can be used in real capture-recapture
problems.  We define a $20 \times 20$ pixel landscape with 
extent = $[0.5, 4.5] \times [0.5, 4.5]$.  
%We define this landscape by
%a single covariate for determing the cost function, and we consider
%two specific covariates.
%purposes of our example, as a coarse landscape covariate, with pixels
%having some arbitrary scaling say, a $2 \times 2$ km resolution. Thus,
%the raster defines a landscape of $40 \times 40$ km and we suppose
We suppose that 16 camera traps are established at the integer coordinates
$(1,1), (1,2), \ldots, (4,4)$. We could think of this as a landscape
within which we're studying a population of ocelots, lynx or some
other cat.

For our analyses, cost is characterized by a single covariate 
and we consider two specific cases. First is an increasing trend from
the NW to the SE (''systematic landscape''), where $z(x)$ is defined as
$z(x) = r(x) + c(x)$ where $r(x)$ and $c(x)$ are just the row and
column, respectively, of the landscape.  This might define something
related to distance from an urban area or a gradient in habitat
quality due to land use, or environmental conditions such as
temperature or precipitation gradients.  In the second case we make up
a covariate by generating a field of spatially correlated noise to
emulate a typical patchy habitat covariate (''patchy landscape'') such as
tree or understory density. The two covariates are shown in
Fig. \ref{ecoldist.fig.raster100}, along with a sample realization of
$N=100$ individuals (left panel only).  For both covariates we use a
cost function in which transitions from pixel ${\bf x}$ to ${\bf x}'$
is given by:

\[
 log(cost({\bf x},{\bf x}'))=  \theta_2 \frac{z({\bf x}) + z({\bf x}')}{2}
\]

{\flushleft where} $\theta_2 = 1$ for our simulation.
When $\theta_2=0$ the
model reduces to one in which the cost of moving across each pixel is
constant, and therefore distance is Euclidean.






\section{Discussion}


could use non-euclidean distance with this stuff i guess.








\newpage


\bibliography{../AndyRefs_alphabetized}


\end{spacing}






\end{document}






