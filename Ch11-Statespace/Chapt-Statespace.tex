\chapter{
Modeling Spatial Variation in Density
}
\markboth{Spatial Variation in Density}{}
\label{chapt.state-space}

\vspace{0.3cm}

Underlying every SCR model is a spatial point process
that describes the number and distribution of animal activity
centers. Spatial point processes are characterized by two key
elements: a spatial domain, or state-space $\mathcal{S}$, and an
intensity function which returns the
expected density of points at any location in $\mathcal{S}$.
%in the case of SCR models, this intensity parameter is population
%density.
If the intensity is constant throughout $\mathcal{S}$,
%density is constant throughout $\mathcal{S}$ and
the point process is said to be homogeneous.
Thus far we have focused our attention on homogeneous %binomial
point processes whose realized values are the locations of the $N$
activity centers.
When a Poisson prior is
placed on $N$, the model is known as a homogeneous Poisson point process, which
is the classic model of ``complete spatial randomness.''
%because the point process intensity is constant
%and the activity centers do not interact with one another.
A similar model, that we often use in conjunction with data
augmentation and MCMC, places a binomial prior on $N$. This is also a
model of spatial randomness, and in this chapter we will compare and
contrast the two.

The spatial randomness assumption is often viewed as restrictive
because ecological processes such as
%territoriality and
habitat selection can result in non-uniform
distributions of organisms. We have argued, however, that this
assumption is less restrictive than may be recognized because a
homogeneous point process actually allows for infinite
possible ``point patterns'', or realized configurations of activity
centers. Furthermore, given enough data,
the uniform prior will have very little influence on the estimated
locations of activity centers. Nonetheless, a homogeneous point
process does not allow one to model population density using
covariates, which is an important objective in much ecological research.
For example, even when assuming a homogeneous point process for
the activity centers, an estimated density surface may strongly
suggest that individuals are more abundant in one habitat than
another; however, such results do not provide the basis for formally testing
hypotheses about spatial variation in density, and they could not be
used to make predictions about habitat-specific abundance in other
regions. A more direct approach is to replace the homogeneous model
with an inhomogeneous model in which the point process intensity
is allowed to vary spatially.

In this chapter, we cover methods % we present a method
for fitting inhomogeneous Poisson and binomial
point process models so that density can be modeled as a
function of covariates in much the same way as is done in generalized linear
models. The covariates we consider differ
from those covered in previous chapters, which were typically
attributes of the animal (e.g. sex or age) or the trap (e.g. baited or
not) and were used to model movement or encounter
rate. In contrast, here we wish to model covariates that are defined
at all points in $\cal S$, and so we will refer to them as
state-space covariates or density covariates. These may
include continuous covariates such as elevation, or categorical
covariates such as habitat type. Typically, these
state-space covariates are formatted
as raster images with a prescribed resolution and extent.

One thing to keep in mind when modeling density
is that the SCR definition of density is %actually
different than what is perhaps a more common definition of density in
ecology. In SCR models, density is defined as the number (or expected
number) of \textit{activity centers} in some region, whereas in other
ecological studies, density is often defined as the number of
\textit{individuals} in some region at some instant in
time. The latter definition is closer to the quantity being estimated
in distance sampling studies. So which definition is better? Does it
make more sense to contemplate activity centers or individuals at an
instant in time? From our perspective, either definition may suffice
for a given objective, but we note that there exists a formal
relationship between the two since an activity center is the
\textit{average} of an individual's locations during some time period. As
such, an activity center may be a better descriptor of an individual's
preferences than is a location during a single instant in
time. Moreover, with SCR models we can model both the distribution of
activity centers (as we will do in this chapter) as
well as the distribution of individuals during specific instances in
time, as is demonstrated in Chapt.~\ref{chapt.search-encounter}.

Inhomogeneous Poisson point process models were discussed in the original
formulation of SCR models by \citet{efford:2004} and were described in
more detail by \citet{borchers_efford:2008}. We will show that an
inhomogeneous point process with a binomial prior on $N$ is quite similar to the Poisson
model, but is more easily implemented in MCMC algorithms. To do so, we
will define the data augmentation parameter $\psi$ in terms of the point
process intensity function, and we will replace the uniform prior on the
activity centers with a prior that is also derived from the intensity
function. Development of this prior, which does not have a
standard form, is a central component of this chapter. First we
begin with a review of homogeneous point process models.


\section{Homogeneous point process revisited}

The homogeneous Poisson point process is \textit{the} model of complete
spatial randomness and is often used in ecology as a null model
to test for departures from randomness
\citep{cressie:1992, diggle:2003, illian_etal:2008}.
%Given its central role in spatial ecology, it is helpful to describe it briefly and
%compare it with the binomial model that we use in when conducting
%Bayesian analysis of SCR models.
The Poisson model asserts that the number of points in $\mathcal{S}$ is
Poisson distributed: $N \sim \text{Poisson}(\mu|\mathcal{S}|)$ where $\mu>0$ is
the intensity parameter and $|\mathcal{S}|$ is the area of the
state-space. The intensity parameter $\mu$
is the density of points, and thus multiplying the intensity by the area
of some region yields the expected number of points in that region.
As with all homogeneous point process models, the $N$ points are
distributed uniformly, which implies that they do not interact with each other in
any way -- for example, they neither attract nor repel one another.

Unlike the Poisson point process, the
binomial point process assumes that $N$ is fixed, not random.
%In other words, the binomial point process conditions on $N$,
The distinction is illustrated by this simple \R~code that generates
realizations from Poisson and binomial point processes in the unit
square ($\mathcal{S} = [0,1]\times[0,1]$):
%\begin{samepage}
\begin{small}
\begin{verbatim}
> Area <- 1                          # Area of unit square
> muP <- 4                           # intensity
> nP <- rpois(1, muP*Area)           # number of points: random
> PPP <- cbind(runif(nP), runif(nP)) # Poisson point pattern
> nB <- 4                            # number of points: fixed
> muB <- nB/Area                     # intensity
> BPP <- cbind(runif(nB), runif(nB)) # binomial point pattern
\end{verbatim}
\end{small}
%\end{samepage}
%{\flushleft
Both of these models are homogeneous because the intensity parameter
is constant ($\mu=4$ in both cases) and the locations of $N$ the
points are mutually independent and uniformly distributed.
with each other. %This results from the fact that the locations of the
%points follow a uniform distribution on the plane.
The key distinction
is that $N$ is random in the former and fixed in the latter.%}

Another difference between the Poisson and binomial models is that if the
state-space is divided into $K$ disjunct regions, the number of points in each
region $n(B_k): k=1,\dots,K$; are independent and identically
distributed (i.i.d.) under the Poisson model but not under the
binomial model. In the Poisson case,
the counts are %simply distributed as
$n(B_k) \sim
\text{Poisson}(\mu|B_k|)$, where $|B_k|$ is the area of the region
$B_k$. For the binomial model, $n(B_k) \sim
\text{Binomial}(N, \pi(B_k))$ where $\pi(B_k)$ is the proportion of
the state-space in $B_k$; however, these counts are not
i.i.d. because the number of points in one region is informative
about the number of points in another region. For example, if
$N=10$ %, which would be known for a binomial point process,
and if %we know that
there are 7 points outside the region $B_1$,
then we can say with certainty that $B_1 = 10 - 7 = 3$.

Fig.~\ref{state-space.fig.homo} is meant to further illustrate the characteristics
of the binomial model. The left panel shows a point pattern
realized from a
homogeneous binomial point process with $N=50$. The right panel shows
the same realization, except that the state-space has been discretized
into 25 equally-sized disjunct regions, or pixels, and the counts in each pixel
are shown. Since the pixels are the same
size, we have that $\pi(B_k) = 1/25$, and the expected number of point in each
pixel is $\mathbb{E}(n(B_k)) = N\pi(B_k) = 50/25 = 2$, which
happens to be the empirical mean in this instance. However, as
previously stated, these counts are not
independent realizations from a binomial distribution since $\sum_k
n(B_k) = N$. Rather, the model for the entire vector is multinomial:
$\{n(B_1), n(B_2), \dots, n(B_k)\} \sim \mbox{Multinomial}(N, \{p(B_1), p(B_2), \dots,
p(B_K) \})$ \citep{illian_etal:2008}. If you need a refresher on the
multinomial distribution, refer to Sec.~\ref{modeling.sec.multinom}, and
consider the following \R~code, which generates counts similar to those
seen in Fig.~\ref{state-space.fig.homo}:
\begin{verbatim}
> n.Bk <- rmultinom(1, size=50, prob=rep(1/25, 25))
> matrix(n.Bk, 5, 5)
     [,1] [,2] [,3] [,4] [,5]
[1,]    2    2    2    2    1
[2,]    2    4    0    5    0
[3,]    0    3    2    4    1
[4,]    1    2    1    4    1
[5,]    4    2    4    1    0
\end{verbatim}

\begin{figure}%[ht!]
\centering
%\includegraphics[width=5in,height=2.5in]{Ch13-Statespace/figs/homoPlots}
\includegraphics[width=\textwidth]{Ch11-Statespace/figs/homoPlots}
\label{state-space.fig.homo}
\caption{Homogeneous binomial point process with $N$=50 points
  represented in continuous and discrete space.}
\end{figure}

The dependence among counts has virtually
no practical consequence when the number of pixels is large. For
example, if there are 100 pixels, the number of points in one pixels
carries very little information about the expected number of points in another
pixel. However, if there are only 2 pixels, then clearly the number of
points in one pixel allows one to determine how many points will occur in the
remaining pixel.

The discrete representation of space shown in
Fig.~\ref{state-space.fig.homo} is not only helpful for understanding
the properties of a point process, it is also of practical importance
when fitting SCR models because spatial covariates are almost always
represented as rasters, i.e. grids with predetermined extent and
resolution. In such cases, the definition of the prior for
the point locations can be changed from the probability that a point
occurs at some location in space to the probability that it occurs in
some pixel of the raster. As we will explain in
Sec.~\ref{modeling.sec.discrete}, this typically involves changing the
prior from a uniform distribution to a multinomial or categorical
distribution.

%Up to this point in this chapter we have
Having sketched out the basic characteristics
of homogeneous Poisson and binomial point process models, %. Now we need
%to speak more specifically about
we will now review
their relevance to SCR models before moving on
to the inhomogeneous models. %Much of this has already been in previous
%chapters, but we feel it is important enough to review here.
In a SCR model with a homogeneous point process, the intensity
parameter $\mu$ is interpreted as population density, and $N$ is
interpreted as population size (i.e. the
  number of activity centers in $\mathcal{S}$). These interpretations
are true regardless of whether we consider the
Poisson model or the binomial model, but since $N$ is always unknown, one
might wonder why we are discussing the binomial model at all. %Indeed,
%the binomial model was not mentioned by \citet{efford:2004} or
%\citet{borchers_efford:2008}. Instead, they focused exclusively on the
%Poisson model and estimation of $\mu$, with $N$ being regarded as a
%derived parameter.

In our work, we typically adopt the binomial model simply
because it is easy to implement using MCMC and data
augmentation. And while $N$ is truly unknown, we use an upper bound, $M$,
which is fixed. Thus, the standard point process we use in Bayesian
analyses can be
regarded in two ways. First, it is a binomial point process with $M$
points. Second, in terms of $N$, it is a thinned binomial point
process, where $\psi$ is the thinning parameter.
With this in mind,
the only real difference between the Poisson and binomial models, as
implemented in SCR contexts, is that in the former, we have
$N \sim \text{Poisson}(\mu|\mathcal{S}|)$, and in the latter we have
$N \sim \text{Binomial}(M, \psi)$. In other words, we just have a
different prior on $N$, and when using MCMC, the binomial prior is
much more convenient because it fixes the size of the parameter space
and makes it easy to extend the model in each of the ways discussed in
this book. It is also worth remembering that the Poisson
distribution is the limit of the binomial distribution when $M$ is
very large and $\psi$ is very small (Chapt.~\ref{chapt.modeling}), and
thus the two models are much more similar than may appear.

You might have noticed that the intensity parameter $\mu$ was not shown for the
binomial prior $N \sim \text{Binomial}(M, \psi)$. Instead, we see the
data augmentation parameter $\psi$, which has been used throughout
this book, but without much mention of the point process
intensity. What then is the relationship between $\psi$ and $\mu$?
As first discussed in Chapt.\ref{chapt.scr0}, under data augmentation,
the expected value of $N$ is $\mathbb{E}[N] = M\psi$. But, from this
chapter, we also know that the
expected value of $N$ can be written in terms of $\mu$ as
$\mathbb{E}[N] = \mu|\mathcal{S}|$. Therefore,
$\psi = \mu|\mathcal{S}| / M$ and hence we can directly estimate $\mu$
rather than $\psi$ if we want, as will be demonstrated
in the next
section where the objective is to model $\mu$ as a function of
spatially-referenced covariates. First, %as an exercise, execute the
%following \R~commands to familiarize yourself with some of these
%concepts:
consider the following \R~code, which illustrates some the concepts we
just covered:
\begin{small}
\begin{verbatim}
> Area <- 1                  # Area of state-space
> M <- 100                   # Data augmentation size
> mu <- 10                   # Intensity (points per area)
> psi <- (mu*Area)/M         # Data augmentation parameter (thinning rate)
> N <- rbinom(1, M, psi)     # Realized value of N under binomial prior
> cbind(runif(N), runif(N))  # Point pattern from thinned binomial model
            [,1]       [,2]
 [1,] 0.52779588 0.84306878
 [2,] 0.11529168 0.80635046
 [3,] 0.06777632 0.66072116
 [4,] 0.18694649 0.56761245
 [5,] 0.30176929 0.03159091
 [6,] 0.84352724 0.89691452
 [7,] 0.52766808 0.08871199
 [8,] 0.73007529 0.63184825
 [9,] 0.01119023 0.69807029
\end{verbatim}
\end{small}
%Note that changing $M$ will not have any effect on the expected value
%of $N$.


\begin{comment}
  We conclude this section by pointing out
  Table\ref{state-space.tab.pvb}, which highlights the differences of
  the homogeneous Poisson and binomial point process models as they
  relate to SCR. % XXXX Need to fill in this table XXXX

\begin{table}
  \centering
  \caption{Characteristics of homogeneous point
    processes. Table~\ref{state-space.tab.hetero} describes the
    inhomogeneous models.}
  \begin{tabular}{lccc}
    \hline
    & Prior & $\mathbb{E}[N]$    & $\mu$   \\
    \hline
    Poisson  & $N \sim \text{Poisson}(\mu|\mathcal{S}|)$   &  &  \\
    Binomial & $N \sim \text{Binomial}(M, $   & $M\psi = $   &  &  \\
    \hline
  \end{tabular}
  \label{state-space.tab.homo}
\end{table}
\end{comment}


\section{Inhomogeneous point processes}

The principal difference between homogeneous and inhomogeneous point
processes is that the intensity parameter $\mu$ is allowed to vary spatially
in the inhomogeneous model. Thus, rather than $\mu$ being a fixed constant,
it is now a function defined at each point $\mathbf{x} \in
\mathcal{S}$. A vast number of options exist for modeling spatial
variation in the intensity of a point process
\citep{cox:1955,stoyan_penttinen:2000,illian_etal:2008}, but here we
focus on modeling $\mu$ as a function of
spatially-referenced covariates and a vector of regression
coefficients $\bm \beta$; a function we will denote $\mu(\mathbf{x},
\bm{\beta})$. To be clear, $\mu(\mathbf{x}, \bm{\beta})$, is a
function that returns the expected density of activity centers at
location $\bf x$, given the covariate values at
$\bf x$\footnote{The use of $\bf x$ to denote any point in the state-space
could cause confusion because we use ${\bf x}_j$ as the location of a
trap, but it is standard notation, and the distinction should be
evident by the context.}.
Since the
intensity must be positive, and because the natural logarithm is the
canonical link function of the Poisson generalized linear model
\citep{mccullagh_nelder:1989}, it is natural to consider the following model:
\begin{equation}
  \log(\mu(\mathbf{x}, {\bm \beta})) = \beta_0 + \sum_{v=1}^V \beta_v C_v(\mathbf{x})%, \quad  \mathbf{x} \in \cal{S}
  \label{state-space.eq.loglin}
\end{equation}
which says that there are $V$ covariates and $\beta_v$ is the
regression coefficient for covariate $C_v(\mathbf{x})$. This
covariate, $C_v(\mathbf{x})$, could be any variable defined at all points
in the state-space, such as habitat type or elevation.
Eq.~\ref{state-space.eq.loglin} should look familiar because it is the
standard linear predictor used in Poisson regression. As with other
GLMs, one could consider alternative link functions.

Recall from the previous section that for a homogeneous point process,
the expected number of points in the state-space was simply the
intensity parameter multiplied by area: $\mathbb{E}(N) =
\mu|\mathcal{S}|$. But now that we are regarding the intensity as a
function, rather than a scalar, this equation is not very useful. So
what is $\mathbb{E}(N)$ for an inhomogeneous point process?
%And why do
%we care? Well, hopefully you remember that the reason we need a
%the reason we care is that we want a model that
Contemplating a discrete state-space is useful for figuring this
out. Imagine that the state-space is represented as a raster with many
tiny pixels. In this case, we will associate
$\bf x$ with pixel ID, i.e. $\bf x$ just references some pixel with
$V$ covariates values associated with it. The expected number of
individuals in this pixel, say $\mathbb{E}(n(\mathbf{x}))$, can intuitively be
found by evaluating the intensity function
(Eq.~\ref{state-space.eq.loglin}) and multiplying it by the area of
the pixel. In other words, we compute the expected number of
individuals in a pixel by multiplying the expected value of density
for that pixel by the area of the pixel. If we do this for each pixel in the state-space, then
summing up these values gives us what we are after, the expected value
of $N$. Specifically,
$\mathbb{E}(N) = \sum_{\mathbf{x} \in \mathcal{S}}
\mathbb{E}(n(\mathbf{x}))$.
As the area of the pixels approaches zero, such that we move from discrete
space back to continuous space, the summation must be replaced
with an integration of the form:
\begin{equation}
\mathbb{E}(N) = \int_{\mathcal{S}} \mu(\mathbf{x}, {\bm \beta}) \mathrm{d}\mathbf{x}.
\label{state-space.eq.EN}
\end{equation}
Together, Eqs.~\ref{state-space.eq.loglin} and \ref{state-space.eq.EN}
describe a model for spatial variation in density as well as
population size. The key task in fitting such inhomogeneous point
process models is to estimate the $\bm \beta$
parameters.

We have now described an approach for modeling the point process
intensity, yet in order to define the likelihood or to develop an MCMC
algorithm for the inhomogeneous model, we need to specify the prior
distribution for the activity centers. Recall that under the
homogeneous point process, the prior was
$\mathbf{s}_i \sim \text{Uniform}(\mathcal{S})$, for $i=1,\dots,N$, or
equivalently:
\begin{equation}
  \label{state-space.eq.uprior}
  [\mathbf{s}_i] = 1/|\mathcal{S}|
\end{equation}
where, as before, $|\mathcal{S}|$ is the area of the
state-space. This simply indicates that an activity center is just as
likely to occur at any location as another.
However, if animals exhibit habitat selection or simply
occur in one region more often than another, it would be preferable to
replace this prior with one describing the spatial variation in
density. Clearly this prior should be determined in some way by the
spatially-varying intensity function $\mu({\bf x}, \bm{\beta})$.
%The key to determining this prior is to recall that
Since
the integral of a probability density function (pdf) must be unity,
we can convert $\mu(\mathbf{x}, \bm{\beta})$ into a pdf by dividing it by a
normalizing constant. In this case, the normalizing constant is found by integrating
$\mu({\bf x}, \bm{\beta})$ over the entire state-space. %, which is Eq.~\ref{state-space.eq.EN}.
The probability density function of the new prior is therefore:
\begin{equation}
[\mathbf{s}_i | \bm{\beta}] = \frac{\mu(\mathbf{s}_i, \bm{\beta})}{\int_{\mathcal{S}} \mu(\mathbf{x}, \bm{\beta})\, \mathrm{d}\mathbf{x}}
\label{state-space.eq.pdf.hetero}
\end{equation}
Substituting the uniform prior with this new distribution
allows us to fit inhomogeneous binomial point process
models to spatial capture-recapture data.
%We can also use this
%distribution to obtain the expected number of individuals in any given
%region $B \in \mathcal{S}$. Specifically, the proportion of $N$ expected to occur in
%$B$ is $\pi(B) = \int_B [\mathbf{s} | \bm{\beta}]\, \mathrm{d}x$. These are
%also the conditional-on-$N$ multinomial cell probabilities if the regions are
%disjoint and compose the entire state-space. We provide an example in
%the next section, and in Fig.\ref{state-space.fig.hetero}.

As a practical matter, note that the integral in the
denominator of Eq.~\ref{state-space.eq.pdf.hetero} is evaluated over
space, and since we always regard space as two-dimensional (the
state-space is planar), this is a two-dimensional integral that can
be approximated using the methods discussed in
Chapt.~\ref{chapt.poisson-mn}, which include
Monte Carlo integration and Gaussian quadrature. Alternatively, if
our state-space covariates are in raster format, i.e. they are
in discrete space, the integral can be replaced with a summation over
all the pixels in the raster,
\begin{equation}
[\mathbf{s}_i | \bm{\beta}] = \frac{\mu(\mathbf{s}_i, \bm{\beta})}{\sum_{\mathbf{x} \in \mathcal{S}} \mu(\mathbf{x}, \bm{\beta})}
\label{state-space.eq.pdf.hetero.d}
\end{equation}
where $\bf s$ is now defined as ``pixel ID'' rather than a point in space.

Although the discrete space approach is standard practice, it is
technically unjustified because covariate values must be known for all
points in space, and a raster is simply a set of spatially-referenced
covariate values at an evenly-spaced subset of points (the pixel
centers). This same problem is present anytime that we have a
sample of the spatial covariates, rather than a function defining
their value for all points in space. In such cases, it may be necessary to
interpolate the values of the covariates for points in space where
they were not measured. One option would be to use a Kriging
interpolator, as demonstrated by \citet{rathbun:1996}. Another option
is to sample the spatial covariates using probabilistic sampling
methods, which allow for design-based estimators of their values for
the entire study area \citep{rathbun_etal:2007}. Either option could
be implemented within maximum likelihood or MCMC estimation methods;
however,
%even though such approaches are technically necessary,
we do not demonstrate them here
because it seems likely that they will be inconsequential in most
cases where the raster data are of high resolution, such that the loss
of information is negligible when going from continuous space to
discrete space. Furthermore, the validity of this assertion, and the
level of resolution required to adequately approximate continuous
space can often be assessed by checking the consistency of the
parameter estimates among varying levels of resolution, as was
demonstrated in Chapt.~\ref{chapt.scr0}.

We now have all the tools needed to fit inhomogeneous point process
models. Likelihood-based inference for inhomogeneous Poisson point
process models was described by \citet{borchers_efford:2008} and
reviewed in Chapt.~\ref{chapt.mle}. Another example is demonstrated in
the next section, but first we focus on the binomial
model that we favor when conducting Bayesian inference. In the
previous section we noted that the data augmentation parameter $\psi$
can be expressed in terms of the intensity parameter $\mu$. The same
is true for inhomogeneous models. Specifically, rather than
$\mathbb{E}(N) = \psi M$ as before, we use the expected value of $N$ shown
in Eq.~\ref{state-space.eq.EN} which results in
\begin{equation}
\psi = \frac{\int_{\mathcal{S}} \mu(\mathbf{x},
  \bm{\beta}) \, \mathrm{d}\mathbf{x}}{M}
\label{state-space.eq.psimu}
\end{equation}
Note that the data augmentation limit $M$ must be high enough so that
it is greater than the numerator -- i.e., the expected value
of $N$ must be less than $M$.

\begin{comment}
If we refer to the distribution $[\mathbf{s}_i | \bm{\beta}]$ as
``IPP'', we can write a hierarchical description of a SCR model with a
Binomial encounter process and a half-normal, or Gaussian, encounter model as
\begin{gather*}
z_i \sim \mbox{Bernoulli}(\psi) \\
{\bf s}_i \sim \mbox{IPP}(\mu(\mathbf{x},\beta)) \\
p_{ij} = p_0 \exp(-\|{\bf x}_j - {\bf s}_{i}\|^2/(2\sigma^2)) \\
y_{ij} \sim \mbox{Binomial}(K, p_{ij} z_i)
\end{gather*}
The new prior for $\mathbf{s}_i$ and Eq.~\ref{state-space.eq.psimu}
%use of $\mbox{IPP}(\mu(s, \beta))$ instead of
%$\mbox{Uniform}(\cal{S})$ is the
are the key differences between homogeneous an inhomogeneous
models.

The IPP for the activity centers
results in another IPP for the observation process, $\lambda(s)$, the
expected number of captures for a trap
at point. As was true for the homogeneous model, this
intensity function is a product of the point process intensity
and the encounter rate function, $\lambda(s) = \mu(s, {\bm \beta})
\lambda_{ij}$.
\end{comment}

In the next sections we walk through a few examples, building up from
the simplest case where we actually observe the activity centers as
though they were data. In the second example, we fit the inhomogeneous model to simulated
data in which density is a function of a single continuous
covariate.
%To build upon the developments in the previous chapter, we
%further consider the plausible case where a state-space covariate is also a
%covariate of ecological distance. A small simulation study indicates
%that both effects can be estimated.
The next example shows an analysis in discrete space using
both \secr~\citep{efford:2011} and \jags~\citep{plummer:2003}, and in the
final example, we model the intensity of
activity centers for a real dataset collected on jaguars
(\emph{Panthera onca}) in Argentina.

\section{Observed Point Processes}

In SCR models, the points (activity centers) are not directly
observed, but in other contexts they are. Examples include the
locations of disease outbreaks, the locations of trees in a forest, or
the locations of radio-tracked animals. In such cases, it is
straightforward to fit inhomogeneous point process models and estimate the
%Indeed Eq.~\ref{eq.pdf.ipp}
%has been used extensively in the radio-telemetry literature to model
%so-called ``resource selection functions''
%\citep{manly_etal:2002,lele_keim:2006}. When the point locations are
%directly observed, estimating
the parameters $\bm \beta$ from Eq.~\ref{state-space.eq.loglin}, as we
will do in the following example. % is
%straight-forward as demonstrated in the following example.
%This example also illustrates the fundamental process that we will
%later embed in our MCMC algorithm used to fit SCR models that include
%an inhomogeneous point process.

Suppose we knew the locations of $N$ animal activity
centers, perhaps as the result of an extensive telemetry study. %To
%estimate the intensity surface $\mu(\mathbf{s}, \bm{\beta})$ underlying these
%points, we need to derive the likelihood for the data under this
%model.
If we assume $N$ is Poisson distributed and the points are
mutually independent of one another, we can fit the
inhomogeneous Poisson point process model. The likelihood of this
model has two components: $[\{\mathbf{s}_1, \ldots, \mathbf{s}_N\}|N]$
and $[N]$. The pdf of the first part is given by
%Eq.~\ref{eq.pdf.ipp}
Eq.~\ref{state-space.eq.pdf.hetero},
and with the Poisson assumption we have:
\begin{align*}
\mathcal{L}({\bm \beta} | \{{\bf s}_1, \ldots, \mathbf{s}_N\})
&= [\{s_1, \ldots, \mathbf{s}_N\}|N][N] \\
&= \left\{\prod_{i=1}^N
  \frac{\mu(\mathbf{s}_i,\bm{\beta})}{\int_{\mathcal{S}}\mu(\mathbf{x},\bm{\beta})\mathrm{d}\mathbf{x}} \right\}
  \frac{e^{-\int_{\mathcal{S}}\mu(\mathbf{x},\bm{\beta})\mathrm{d}\mathbf{x}}
    \int_{\mathcal{S}}\mu(\mathbf{x},\bm{\beta})\mathrm{d}\mathbf{x}^N}{N!}.
\end{align*}
This can be simplified by noting that the denominator in the first
component of the model cancels with the corresponding piece in the
numerator of the second component. And, since $N$ is observed and
thus does not depend on the parameters, $N!$ can be omitted as
well. After log-transforming the remaining pieces, we have the
log-likelihood often seen in textbooks, such as \citet[pg. 104]{diggle:2003}:
\[
\ell({\bm \beta} | \{{\bf s}_i\}) = \sum_{i=1}^N
\log(\mu(\mathbf{s}_i, \bm{\beta})) - \int_{\mathcal{S}} \mu(\mathbf{x}, \bm{\beta}) \, \mathrm{d} \mathbf{x} .
\]
Having arrived at the likelihood we could choose a prior distribution for
$\bm \beta$ and obtain the posterior distribution %of $\bm \beta$
using Bayesian methods, or we can find the maximum likelihood
estimates (MLEs) using standard numerical methods as is demonstrated
below.

First, we simulate some data under the model $\mu(\mathbf{x},
\bm{\beta}) = \beta_0 + \beta_1\mathrm{ELEV}(\mathbf{x})$, where
$\mathrm{ELEV}(\mathbf{x})$ is a spatial covariate, say
elevation, and $\beta_0=-6$ and $\beta_1=1$. It is worth emphasizing
that a spatial covariate must be defined at any location in the
state-space, as is true of the following covariate \verb+elev.fn+:
%\begin{samepage}
  \begin{small}
\begin{verbatim}
> elev.fn <- function(x) {          # spatial covriate
+     x <- matrix(x, ncol=2)        # Force x to be a matrix
+     (x[,1] + x[,2] - 100) / 40.8  # Returns (standardized) "elevation"
+ }
> # intensity function
> mu <- function(x, beta0, beta1) exp(beta0 + beta1*elev.fn(x=x))
> beta0 <- -6 # intercept of intensity function
> beta1 <- 1  # effect of elevation on intensity
> # Next line computes integral
> EN <- cuhre(2, 1, mu, beta0=beta0, beta1=beta1,
+             lower=c(0,0), upper=c(100,100))$value
\end{verbatim}
\end{small}
%\end{samepage}
The function \texttt{elev.fn} returns the value of elevation at any
location \verb+x+. The standardization bit is not necessary,
but helps with the model fitting below. The next lines of the code define the
intensity function $\mu(\mathbf{x}, \bm{\beta})$ in terms of elevation
and the regression coefficients. The last line uses the \verb+cuhre+ function in
the {\tt R2Cuba} package \citep{hahn_etal:2011} to compute the
expected value of $N$ in a $[0,100]\times[0,100]$ square state-space, which is the
two-dimensional integral of Eq.~\ref{state-space.eq.pdf.hetero}. This
integral could also be computed using a fine grid of points as we have done in previous
chapters, but it is useful to gain familiarity with more efficient
integration functions in \R.

The \R~code above demonstrates how to obtain the expected value
of $N$ given a spatial covariate and the coefficients defining the
intensity function. Now we need to generate a realized value of $N$
and distribute the $N$ points in proportion to the intensity
function. This is not as simple as it was to simulate data from a homogeneous point process
because the points are no longer uniformly distributed within the
state-space. Instead one must resort to methods such as rejection sampling, which involves
simulating data from a standard distribution and then accepting or
rejecting each point using probabilities defined by the distribution
of interest. For more information, readers should consult an
accessible text such as \citet{robert_casella:2010}. In our example, we
simulate from a uniform distribution and then accept or reject using
the (scaled) probability density function
$[\mathbf{s}_i | \bm{\beta}]$
(Eq.~\ref{state-space.eq.pdf.hetero}). The following \R~commands
demonstrate the use of
rejection sampling to simulate an inhomogeneous point process for the
elevation covariate depicted in
Fig.~\ref{state-space.fig.hetero}.
%\begin{samepage}
  \begin{small}
\begin{verbatim}
> set.seed(31025)
> beta0 <- -6 # intercept of intensity function
> beta1 <- 1  # effect of elevation on intensity
> # Next line computes integral, which is expected value of N
> EN <- cuhre(2, 1, mu, beta0=beta0, beta1=beta1,
+             lower=c(0,0), upper=c(100,100))$value
> EN
[1] 39.96634
> N <- rpois(1, EN) # Realized N
> s <- matrix(NA, N, 2) # This matrix will hold the coordinates
> elev.min <- elev.fn(c(0,0))
> elev.max <- elev.fn(c(100, 100))
> Q <- max(c(exp(beta0 + beta1*elev.min),
+            exp(beta0 + beta1*elev.max)))
> counter <- 1
> while(counter <= N) {
+   x.c <- runif(1, 0, 100); y.c <- runif(1, 0, 100)
+   s.cand <- c(x.c,y.c)
+   pr <- mu(s.cand, beta0, beta1) #/ EN
+   if(runif(1) < pr/Q) {
+     s[counter,] <- s.cand
+     counter <- counter+1
+     }
+   }
\end{verbatim}
  \end{small}
%\end{samepage}
Similar methods are also
implemented in the \R~package \texttt{spatstat} \citep{baddeley_turner:2005}.
\begin{figure}%[ht]
\centering
\includegraphics[width=\textwidth]{Ch11-Statespace/figs/heteroPlots}
\caption{An example of a spatial covariate, say elevation, and a
  realization from an inhomogeneous Poisson point process with
  $\mu(\mathbf{x}, \bm{\beta}) = \exp(\beta_0 + \beta_1
  \mbox{ELEV}(\mathbf{x}))$ where $\beta_0=-6$ and $\beta_1=1$.}
\label{state-space.fig.hetero}
\end{figure}

The 41 simulated points are shown in
Fig~\ref{state-space.fig.hetero}. High elevations
are represented by light gray and low elevations are darker. The
density of points in apparently higher in lighter regions
suggesting that these simulated animals prefer high
elevations.  %Perhaps they are mountain goats.
%The underlying model describing this preference is
%$\log(\mu(s)) = \exp(\beta \times elev(s))$
%where $\beta=2$ is the parameter to be estimated.
Given these points, we will now estimate $\beta_0$ and $\beta_1$ by
minimizing the negative-log-likelihood using \R's \verb+optim+
function.

%\begin{samepage}
\begin{small}
\begin{verbatim}
> nll <- function(beta) {
+     beta0 <- beta[1]
+     beta1 <- beta[2]
+     EN <- cuhre(2, 1, mu, beta0=beta0, beta1=beta1,
+                 lower=c(0,0), upper=c(100,100))$value
+     -(sum(beta0 + beta1*elev.fn(s)) - EN)
+ }
> starting.values <- c(-10, 0)
> fm <- optim(starting.values, nll, hessian=TRUE)
> cbind(Est=fm$par, SE=sqrt(diag(solve(fm$hessian)))) # estimates and SEs
            Est        SE
[1,] -5.9335547 0.2204693
[2,]  0.9545532 0.1771507
\end{verbatim}
\end{small}
%\end{samepage}

Maximizing the Poisson likelihood took a fraction of a second, and we
obtained estimates of $\hat{\beta}_0=-5.93$ and $\hat{\beta}_1=0.95$,
which are very close to the data-generating values. The 95\% confidence
interval for $\hat{\beta}_1$ is [0.61, 1.3] and since it does not
include zero, the null hypothesis that $\beta_1=0$, i.e. that there is
no effect of elevation on density, can be rejected. In addition to testing
hypotheses, these results can be used to predict population size in
new regions or create predicted density surface maps by plugging the
parameter estimates into Eqs.~\ref{state-space.eq.loglin} and~\ref{state-space.eq.EN}.

You might wonder if the results would differ if we assumed a binomial
rather than a Poisson distribution for $N$. This can be checked
using the following code:
\begin{small}
\begin{verbatim}
> nllBin <- function(beta, M=100) {
+     beta0 <- beta[1]
+     beta1 <- beta[2]
+     EN <- cuhre(2, 1, mu, beta0=beta0, beta1=beta1,
+                 flags=list(verbose=0),
+                 lower=c(0,0), upper=c(100,100))$value
+     N <- nrow(s)
+     psi <- EN/M
+     -(sum(beta0 + beta1*elev.fn(s) - log(EN)) +
+       dbinom(N, M, psi, log=TRUE))
+ }
> cbind(Est=fmBin$par, SE=sqrt(diag(solve(fmBin$hessian)))) # est and SE
            Est        SE
[1,] -5.9339490 0.1965479
[2,]  0.9545742 0.1771962
\end{verbatim}
\end{small}
which indicates that the MLEs are almost identical, and
supports the claim that the prior
on $N$ has little influence in SCR models. Notice, however, that the
standard error for $\beta_0$ is smaller under the binomial model than
it was under the Poisson model -- a difference that will dissipate as
$M$ tends toward infinity.

This example demonstrates
that if we had the data we wish we had, i.e. if we knew the
coordinates of the activity centers, we could easily estimate the
parameters governing the underlying point process and make inferences
about spatial variation in density and abundance. Unfortunately, in
virtually all animal ecology studies,
%including in SCR studies,
the locations of the $N$ animals, or the $N$ activity centers,
cannot be directly observed. Thus we need
extra information to estimate the locations of these unobserved
points, which in the case of SCR, comes from the locations where each
animal is captured.

\section{Fitting inhomogeneous point process SCR models}

\subsection{Continuous space}

% One of the nice things about hierarchical models is that they
% %allow us to
% break a complex problem up into a series of simpler conditional
% sub-models. The problem faced in the analysis of SCR data is that the
% underlying point process is not observed, and so an
% observation model is needed to describe how the observed data result from
% the latent activity centers. Thus, in SCR, we can simply add the methods described above into a likelihood
% function or an MCMC algorithm to simulate the posterior distributions of $\beta$ conditional on the
% simulated values of $\mathbf{s}$.

% XXXX Note that data was simulated with Poisson, not binomial
% prior. Discuss distinction between model psi as E(N)/M vs Unif(0,1)

In this example, we will use the same set of points simulated in the
previous section to generate spatial capture-recapture
data. Specifically, we overlay a grid of 49
traps on the map shown in Fig.~\ref{state-space.fig.hetero} and
simulate capture histories conditional on the activity
centers. Then, we will attempt to estimate the activity center
locations as though we did not know where they were, as is the case in
real applications. We will also estimate $\beta_0$ and $\beta_1$ as
before and see how the estimates compare when the points are not
actually observed. The following \R~code simulates encounter histories under a
Poisson observation model (see Chapt. \ref{chapt.poisson-mn}), which could be appropriate in camera
trapping studies or when using other methods in which animals could
be detected multiple times at a trap during a single occasion.
%\begin{samepage}
\begin{small}
\begin{verbatim}
> xsp <- seq(20, 80, by=10); len <- length(xsp)
> X <- cbind(rep(xsp, each=len), rep(xsp, times=len)) # traps
> ntraps <- nrow(X); noccasions <- 5
> y <- array(NA, c(N, ntraps, noccasions)) # capture data
> sigma <- 5  # scale parameter
> lam0 <- 1   # basal encounter rate
> lam <- matrix(NA, N, ntraps)
> set.seed(5588)
> for(i in 1:N) {
+     for(j in 1:ntraps) {
+         # The object "s" was simulated in previous section
+         distSq <- (s[i,1]-X[j,1])^2 + (s[i,2] - X[j,2])^2
+         lam[i,j] <- exp(-distSq/(2*sigma^2)) * lam0
+         y[i,j,] <- rpois(noccasions, lam[i,j])
+     }
+ }
\end{verbatim}
\end{small}
%\end{samepage}

Now that we have a simulated capture-recapture dataset \texttt{y}, %and we have
%augmented it to create the new data object \texttt{yz},
we can simulation the posterior distributions of the model parameters
%estimate the parameters
using MCMC.  A commented Gibbs sampler written
in \R~is available in the accompanying \R~package \scrbook~(see
\texttt{?scrIPP}). This function is not meant to be an all purpose
tool for fitting SCR models using MCMC. Instead, it is presented so
that interested readers can better understand the computational
aspects of the problem and can modify it for their purposes.
The function can be used as so:
%\begin{samepage}
\begin{small}
\begin{verbatim}
> fm1 <- scrIPP(y, X, M=150, 10000, xlims=c(0,100), ylims=c(0,100),
+               space.cov=elev.fn, tune=c(0.4, 0.2, 0.3, 0.3, 7))
> plot(mcmc(fm1$out))
\end{verbatim}
\end{small}
%\end{samepage}
which requests 10000 posterior samples and estimates the effect of the
spatial covariate, elevation, on density.
The argument \verb+space.cov+ accepts any spatial
covariate that returns a real value for any location in the
rectangular state-space defined by the \verb+xlims+ and \verb+ylims+
arguments. Currently, the function
places uniform priors on the parameters $\sigma$, $\lambda_0$,
$\beta_0$ and $\beta_1$, although this could easily be modified.
The \verb+tune+ argument specifies the tuning parameters used in the
Metropolis-within-Gibbs steps of the algorithm. These should be chosen
using trial and error to achieve an acceptance rate of between 0.4 and
0.6, roughly. See Chapt.~\ref{chapt.mcmc} for more details about MCMC.

Results of the analysis are shown in
Fig.~\ref{state-space.fig.fm1post} and
Table~\ref{state-space.tab.simIPP}.
Fig.~\ref{state-space.fig.fm1post}
displays the trace plots of the Markov chains as well as the posterior
distributions for three parameters. The chains appear to
converge rapidly but may need to be run longer to reduce Monte Carlo
error. Summaries of the posterior distributions are presented in
Table~\ref{state-space.tab.simIPP}. The posterior means for $\beta_0$
and $\beta_1$ are quite similar to MLEs from the analysis in the
previous section in which we assumed no observation error. However, we
see that the confidence intervals are wider. With respect to the other
parameters in the model, we see that all of the data
generating parameter fall within the 95\% credible intervals. One
thing to note is that, although the point estimates for
the expected and realized values of $N$ are quite similar, the
posterior for the realized value of $N$ is more precise. This is to be
expected because the uncertainty associated with the realized value of
$N$ is entirely determined by the sampling error. That is,
if we could perfectly detect all of the individuals in $\cal S$, there
would be no uncertainty about $N$. In contrast, the variance for
expected value of $N$ is composed both process error and sampling
error. See Chapt.~\ref{chapt.scr0} and
\citet{efford_fewster:2012} for additional discussion on the
difference between realized and expected values of abundance.


Fitting continuous space inhomogeneous point process models is somewhat
difficult in \bugs~because the ``IPP'' prior $[\mathbf{s}_i | \bm
\beta]$, unlike the uniform prior, is not one of the
available distributions that comes with the software. It is
possible to add new distributions in \bugs, but it is somewhat
cumbersome.  \secr~allows
users to fit continuous space models using linear or polynomial functions of the easting and northing
coordinates, but it does not accept truly continuous covariates that
are functions of space. However, these
are not really important limitations because discrete
space versions of the model are straight-forward, and virtually all spatial
covariates are, or can be, defined as such.


\begin{table}[h!]
\centering
\caption{Summary of posterior distributions from SCR model with
  inhomogeneous point process. }
\begin{tabular}{lrrrr}
\hline
Parameter 	 	& Mean  	& SD    	& 2.5\% 	& 97.5\% \\
\hline
 $\sigma=5$ 	 	&  5.232 	&  0.310 	&  4.681 	&  5.858 \\
 $\lambda_0=1$ 	 	&  0.802 	&  0.119 	&  0.595 	&  1.049 \\
 $\beta_0=-6$ 	 	& -5.856        & 0.254        & -6.376        & -5.393 \\
 $\beta_1=1$ 	 	&  0.985 	&  0.209 	&  0.575 	&  1.378 \\
 $N=41$ 	 	& 47.615 	&  8.041 	& 35.000 	& 66.000 \\
 $\mathbb{E}(N)=39.9$ 	& 47.551 	& 10.992 	& 29.837 	& 71.332 \\
\hline
\end{tabular}
\label{state-space.tab.simIPP}
\end{table}

\begin{figure}[h!]
  \centering
  \includegraphics[width=0.8\textwidth]{Ch11-Statespace/figs/fm1p}
  \caption{Trace plots and posterior distributions from MCMC analysis
    of SCR model with inhomogeneous point process. Analysis was
    conducted using the \texttt{scrIPP} function in the accompanying
    \R~package \scrbook.}
  \label{state-space.fig.fm1post}
\end{figure}



\subsection{Discrete space}
\label{modeling.sec.discrete}

To fit inhomogeneous point process models using covariates in discrete
space, i.e. in raster format, we follow the same steps
as outlined in Chapt.~\ref{chapt.poisson-mn} -- we define ${\bf s}_i$ as
pixel ID, and we use the categorical distribution as a prior. This
effectively changes the problem from estimating the coordinates of an
activity center, to estimating the pixel in which an activity center is
located. As pixel size approaches zero, these two become equivalent. A good
example is found in \citep{mollet_etal:2012}. Here we present
an analysis of the simulated data shown in the %right panel of
Fig.~\ref{state-space.fig.hetero}. The spatial covariate, let's call it
forest canopy height (CANHT), was simulated
using using the code shown on the help page
\verb+ch11+ in \scrbook. The points are the number of
activity centers in each pixel, generated from a single realization of
the inhomogeneous point process model with intensity
$\mu(\mathbf{x}, \bm{\beta}) = \exp(\beta_0 + \beta_1
\text{CANHT}({\bf x}))\times\text{pixelArea}$,
where $\beta_0 = -6$ and $\beta_1 = 1$.
\begin{figure}%[ht]
\centering
\includegraphics[width=0.6\textwidth]{Ch11-Statespace/figs/discrete}
\label{state-space.fig.discrete}
\caption{Simulated activity centers in discrete space. The spatial
  covariate, canopy height, is highest in the lighter areas and
  density increases with canopy height. A single
  activity center is shown as a small circle, and larger circles
  represent two activity centers in a pixel. Trap locations
  are shown as crosses.}
\end{figure}

The \bugs~description of the model is shown in
panel~\ref{state-space.panel1}. The vector \verb+probs[]+ is the prior
probability defined by Eq.~\ref{state-space.eq.pdf.hetero.d}, which is
the probability that an individual's activity center is located at
pixel $\bf x$. \verb+grid+ is the matrix of coordinates for each pixel.

\begin{panel}%[h!]
\centering
%\rule[0.15in]{\textwidth}{.03in}
\rule[0.05in]{\textwidth}{.03in}
\begin{small}
\begin{verbatim}
model{
sigma ~ dunif(0, 20)
lam0 ~ dunif(0, 5)
beta0 ~ dunif(-10, 10)
beta1 ~ dunif(-10, 10)
for(j in 1:nPix) {
  mu[j] <- exp(beta0 + beta1*CANHT[j])*pixArea
  probs[j] <- mu[j]/EN
}
EN <- sum(mu[]) # Expected value of N, E(N)
psi <- EN/M
for(i in 1:M) {
  z[i] ~ dbern(psi)
  s[i] ~ dcat(probs[])
  x0g[i] <- grid[s[i],1]
  y0g[i] <- grid[s[i],2]
  for(j in 1:ntraps) {
    dist[i,j] <- sqrt(pow(x0g[i]-traps[j,1],2) + pow(y0g[i]-traps[j,2],2))
    lambda[i,j] <- lam0*exp(-dist[i,j]*dist[i,j]/(2*sigma*sigma)) * z[i]
    y[i,j] ~ dpois(lambda[i,j])
    }
  }
N <- sum(z[]) # Realized value of N
}
\end{verbatim}
\end{small}
\rule[0.15in]{\textwidth}{.03in}
\caption{\bugs~model specification for the inhomogeneous point process model in
  discrete space. A nearly equivalent formulation would involve
  omitting $\beta_0$ and modeling the expected number of activity
  centers as $\mathbb{E}(N)=M\psi$ with $\psi \sim \text{Uniform}(0,1)$.}
\label{state-space.panel1}
\end{panel}

This model can also be fit in \secr, which refers
to the raster data as a ``habitat mask''. The habitat mask is
essentially a \verb+data.frame+ with attributes. The \verb+data.frame+
itself has 2 columns for the coordinates of each of the pixel
centers. The attributes of the object include information such as the
area of the pixels and the spacing between pixel centers. If there are
covariates, these too are stored as an attribute of the habitat mask,
and are formatted as a \verb+data.frame+ with 1 row per pixel and 1
column per covariate. Once the data have been formatted
correctly, fitting the model in \secr~is as simple as:
\begin{verbatim}
> secr1 <- secr.fit(ch, model=D~canht, mask=msk)
\end{verbatim}
where \verb+D~canht+ indicates that we want to model density as a
function of canopy height, which is defined in the \verb+msk+ object.
\R~code to format the data and fit the models using \secr~and \jags~is
available in \scrbook, found by issuing the command: \verb#help(ch11secr-jags)#.

Results of fitting the model in \jags~and \secr~are shown in Table
\ref{state-space.tab.jagsVsecr} and are
similar as expected. The differences that do exist are likely due to
the differences in Bayesian and frequentist estimation methods, as
discussed in Chapt~\ref{chapt.glms}. Either answer may be ``more
correct'' depending upon one's criteria for correctness!
% explained by a variety of reasons. For one, there exists some Monte
% Carlo error in the Bayesian posterior summaries. There is also the
% fact that posterior summaries can be computed in numerous ways---for
% example, we could have presented posterior modes or medians instead of
% means---or, we could have shown highest posterior density credible
% intervals instead of simple percentiles. The posteriors would also
% differ if we chose more informative priors than the uniform
% distributions used here. We see no reason why these
% issues should be seen as limitations of the Bayesian analysis, rather
% we would argue that the posterior distribution, which describes the
% probability that the parameter equals any particular value, is a
% better descriptor uncertainty than any particular point estimator or
% confidence interval. %Futhermore, most of these differences are minor,
%and hardly worth mention.
% The only exception is that the estimates of
% the expected and realized values of $N$ are closer to the data
% generating values in the Bayesian analysis, and the Bayesian credible
% intervals are narrower than the frequentist confidence intervals. This
% is likely a result of the fact that the Bayesian analysis assumed
% that $N$ was binomial whereas the frequentist analysis
% assumed a Poisson prior for $N$, and the variance of the binomial will
% always be less than or equal to the variance of Poisson distribution
% for a shared expectation.

\begin{table}%[h!]
\centering
\caption{Comparison of \secr~and \jags~results. Point estimates from
  the Bayesian analysis are posterior means. Intervals are lower and
  upper 95\% CIs.}
\begin{tabular}{lrlrrrr}
\hline
Parameter 	& Truth 	& Software 	& Mean 	& SD 	& 2.5\% & 97.5\% \\
\hline
 $\lambda_0$ 	&  1.00 	& \textbf{JAGS} 	&  1.04 	& 0.087 	&  0.88 	&  1.22 \\
                &  1.00 	& \texttt{secr} 	&  1.08 	& 0.089 	&  0.92 	&  1.27 \\
 $\sigma$ 	& 10.00 	& \textbf{JAGS} 	& 10.16 	& 0.373 	&  9.46 	& 10.94 \\
  	        & 10.00 	& \texttt{secr} 	&  9.84 	& 0.350 	&  9.18 	& 10.55 \\
 $\beta_1$ 	&  1.00 	& \textbf{JAGS} 	&  1.20 	& 0.350 	&  0.50 	&  1.88 \\
  	        &  1.00 	& \texttt{secr} 	&  1.09 	& 0.316 	&  0.47 	&  1.71 \\
 $N$ 	        & 30.00 	& \textbf{JAGS} 	& 26.63 	& 2.585 	& 23.00 	& 33.00 \\
  	        & 30.00 	& \texttt{secr} 	& 28.19 	& 3.037 	& 24.49 	& 37.39 \\
 $\mathbb{E}(N)$ 	& 32.30 	& \textbf{JAGS} 	& 26.39 	& 5.048 	& 17.25 	& 36.96 \\
  	        & 32.30 	& \texttt{secr} 	& 28.19 	& 6.117 	& 18.52 	& 42.93 \\
\hline
\end{tabular}
\label{state-space.tab.jagsVsecr}
\end{table}




\section{The Jaguar Data}

Estimating density of large felines has been a priority for many
conservation organizations, but few robust methodologies existed before
the advent of SCR. Distance sampling is not feasible for such rare and
cryptic species, and traditional capture-recapture methods yield
estimates that are highly sensitive to the subjective choice of the
effective survey area. SCR models provide a powerful alternative
because density can be estimated directly and data can be collected
using non-invasive methods such as camera traps or hair snares.

In this example, we demonstrate how readily density can be estimated
for a globally imperiled species using SCR. Furthermore, we show how
inhomogeneous point process models can be used to test important
hypotheses regarding the factors affecting density.
The data come from an 8-year camera-trapping study designed to assess the impacts of poaching
on jaguar density in Argentina, near the borders of Brazil and
Paraguay. Additional information about the study is presented in
\citet{paviolo_etal:2008} and \citet{paviolo_etal:2009}.
%Although
%jaguars themselves are occasionally killed by
%poachers, the larger concern is the influence of poaching on prey
%species.
The expected effect of poaching is a decline in jaguar
density due to the direct removal of individuals and the depletion of its
main prey species.
To conserve jaguars and related species, protected areas have
been established and three levels of protection are
recognized, as depicted in Fig.~\ref{state-space.fig.jaguarCts}.
The dark gray %green XXXX TODO: Convert images to grayscale
area is the Iguaz\'{u} National Park that is patrolled regularly by law enforcement
officials. %The light green areas are officially protected, but due to
%resource limitations, are not patrolled as often. %The beige areas are
%not protected at all, and the gray areas are large soybean
%monocultures, which provide no habitat.
The medium gray areas are not protected and rarely patrolled. Finally, the
light gray areas are large soybean monocultures, cities and dams which
provide no suitable habitat for jaguars

To test for differences in
density between the three regions, we modeled the point process intensity parameter
as a function of protection status (PROTECT), which we treated as an
ordinal variable:
\[
\mu(\mathbf{x}, \bm{\beta}) = \exp(\beta_0 + \beta_1\text{PROTECT}(\mathbf{x}))\times \text{pixelArea}.
\]
We predicted that $\beta_1$
would be greater than zero,
indicating that jaguar density increases with protection status. In
addition to modeling spatial variation in density, we also modeled the
scale parameter of the half-normal (or Gaussian) encounter model as
sex-specific because male cats typically have larger home ranges than
females \citep{sollmann_etal:2011}. Since sex is an
individual-specific covariate, and not observed for the individuals
that were not captured, a prior distribution is required for the sex of uncaptured
individuals. We used a Bernoulli prior with probability 0.5 to
describe our uncertainty about sex ratio. Another equivalent option is
to augment the data with an equal number of males and females and let
the MCMC algorithm determine which of these individuals are actually
members of the population.

\begin{figure}%[ht]
\centering
\includegraphics[width=0.6\textwidth]{Ch11-Statespace/figs/jaguarCountMap}
\caption{Jaguar detections at 46 camera trap stations. The three levels of
  protection status are no protection (light gray), some protection
  (gray), and Iguaz\'{u} National Park (dark gray). Non-habitat
  (soybean monocultures) is shown in white. }
\label{state-space.fig.jaguarCts}
\end{figure}

An additional unique aspect of this study is the highly irregular
state-space. Unlike in the examples of simulated data, the
geometry of this state-space is not a simple rectangular
region. Instead, it is the
%In this example, we restricted the state-space to exclude the large
%soybean monocultures surrounding the study area, and we only
%considered
area south of the Iguaz\'{u} River, which runs along the northern border
of the park shown in dark green in
Fig.~\ref{state-space.fig.jaguarCts}, and it excludes the large
soybean monocultures.
%Rather than restricting the
%state-space, we could have modeled the permeability of the river using
%the methods described in the previous section and in
%Chapter~\ref{chapt.ecoldist}; however, no sampling was conducted on
%the northern side of the river, and ancillary data indicates that
%jaguars rarely forge the waterway.
Fitting models in highly convoluted spatial regions raises
the question: How does one integrate
Eq.~\ref{state-space.eq.pdf.hetero} over this irregular space? Earlier we used the function \verb+cuhre+ in
\R~for the two-dimensional integration, but its \verb+lower+ and
\verb+upper+ arguments essentially assume that the state-space is
square. There are methods of transforming the state-space that might
allow us to work around this problem, but once again we find that it
is most convenient to work in discrete space and sum over all the pixels defining
$\mathcal{S}$.

We fit the model to data from a single year in which 46
camera stations were operational, each consisting of a pair of cameras placed along
roads or small trails. Forty-five detections of 16 jaguars (8 males and 8
females) were made over a 95-day sampling period. The mean number of
sampling days at each camera station was 48.2. The raw capture data
shown in Fig.~\ref{state-space.fig.jaguarCts} suggest that the highest
number of captures was in the national park, but there were also
several traps in the park with no captures. Furthermore, few cameras
were placed far from the protected areas, making it somewhat difficult
to detect differences in density. \R~code to fit the model is
available in \scrbook~on the help page \verb+jaguarDataCh11+.
Parameter estimates are shown in Table~\ref{state-space.tab.jagposts}.


% To assess the influence of poaching on jaguar density, we treated
% protection status as an ordinal variable with 3 levels: no protection,
% some protection, and high protection (national parks). Clearly these
% are ordered, and our
% hypothesis is that poaching pressure should decrease and jaguar
% density should increase with the level of
% protection. Thus, $\beta$ in this example is a ``slope''
% parameter describing the degree to which protection status affects
% jaguar density. We also hypothesized that males would have larger home
% ranges than females  and that the sex ratio may not be
% 1:1.


\begin{table}
\centering
\caption{Summaries of posterior distributions from the model of jaguar
  density. $\sigma$ is the scale parameter of
  the half-normal detection function. $\lambda_0$ is baseline
  encounter rate,
$\beta_1$ is the
  effect of protection status on jaguar density, $\rho$ is the
  sex-ratio,  $N$ is population size. The last three parameters are the density estimates
  (jaguars/100 km$^2$) for the three levels of protection.}
\begin{tabular}{lrrrr}
\hline
& Mean & SD & 2.5\% & 97.5\% \\
\hline
 $N$ 	&   35.819 	&   7.9749 	&   23.0000 	&   54.0000 \\
 $D_\text{low}$ 	&    0.906 	&   0.3265 	&    0.3813 	&    1.6682 \\
 $D_\text{med}$ 	&    0.770 	&   0.2841 	&    0.2698 	&    1.4392 \\
 $D_\text{high}$ 	&    1.370 	&   0.3069 	&    0.8315 	&    1.9955 \\
 $\sigma_\text{female}$ 	& 5501.204 	& 876.8774 	& 4142.2756 	& 7578.5692 \\
 $\sigma_\text{male}$ 	& 6452.570 	& 915.3623 	& 4970.3215 	& 8505.5219 \\
 $\lambda_0$ 	&    0.006 	&   0.0016 	&    0.0034 	&    0.0098 \\
 $\psi$ 	&    0.355 	&   0.0937 	&    0.1998 	&    0.5638 \\
 $\beta_0$ 	&   -4.686 	&   0.2602 	&   -5.2346 	&   -4.2129 \\
 $\beta_1$ 	&    0.174 	&   0.3500 	&   -0.5104 	&    0.8649 \\
 Sex Ratio 	&    0.489 	&   0.0550 	&    0.3824 	&    0.6000 \\
 \hline
\end{tabular}
\label{state-space.tab.jagposts}
\end{table}

The results indicate that efforts to protect jaguars by reducing
poaching in protected areas are not working as well as hoped for. The
posterior probability that $\beta_1 > 0$ was only 0.69, and
the posterior mean of realized density was only 51\% higher in the national park than in the
unprotected area. Fig.~\ref{state-space.fig.Dsurface} shows the estimated
density surfaces. The first map is the expected density in each of
the three values, which was computed by plugging in the posterior mean
values of $\beta_0$ and $\beta_1$ into the log-linear intensity
function. The second map is the realized density surface -- the
conditional-on-$N$ probability distribution of the number of
activity centers in each pixel of the rasterized
state-space. The expected values
would be used if we were interested in making inferences about other
areas or time periods, whereas the realized map is the best
description of the system during the study period.

\begin{figure}%[ht]
\centering
\includegraphics[width=\textwidth]{Ch11-Statespace/figs/reD}
\caption{Estimated density (activity centers / pixel) surfaces from
  the analysis of the jaguar data.}
\label{state-space.fig.Dsurface}
\end{figure}

We note that there is room for improvement in our analysis, and our
results should be considered preliminary. The
political boundaries used to demarcate protected areas are not as
concrete as we might like. In reality poaching pressure is likely
higher near remote park boundaries than in well-guarded park
interiors. One option for addressing this would be to use a continuous
measure of poaching pressure such as distance from the nearest town,
or some other accessibility metric. It would also be worthwhile to
model density separately for each sex because many of the detections outside
of the park were of males, and thus it is possible that the sexes use
habitat differently \citep{conde_etal:2010}. Other extensions warranting
investigation include treating PROTECT as a categorical rather than
ordinal variable, and assessing the effects of
roads and trails on jaguar movement using the methods described in
Chapt.~\ref{chapt.ecoldist}. %This latter extension could be important
%because if jaguar home ranges are not symmetic...
Developing models for these extensions
could be readily accomplished by modifying the fitting functions found
in the \R~package \scrbook.



\section{Summary and Outlook}

One of the distinguishing features of spatial capture-recapture models
is that they allow for inference about spatial variation
in density without relying on ad hoc approaches for determining the
amount of area surveyed. The approach described
in this chapter involves modeling the locations of activity centers as outcomes
of an inhomogeneous point process with intensity determined by
covariates defined at all locations in the state-space. Covariate
effects can be evaluated in exactly the same way as is done in
generalized linear models, making it easy to interpret the results.

All the examples in this section included a single state-space
covariate, but this was for simplicity only. Including multiple
covariates poses no additional challenges. Similarly, additional model
structure such sex-specific encounter rate parameters or behavioral
responses can be accommodated and fit using \secr, \bugs, or by
extending the functions in \scrbook. It is also possible to consider
covariates that affect both density and ecological
distance as will be described in the next chapter. The ramifications of this are enormous for applied
ecological research and conservation efforts because %, for instance,
researchers can use capture-recapture data to identify areas where
both density and landscape connectivity are high
\citep{royle_etal:2012ecol}. Addressing such questions
is simply not possible using standard, non-spatial capture-recapture
methods. %Accomplishing these goals will of course require more data
%than is needed to estimate the parameters of a basic SCR model.

% When maximum likelihood is used,
% it is convenient to replacing the uniform prior on
% the activity centers with a prior based on the point process intensity
% function of covariates. This distribution has been widely used in
% ecology to model point processes as well as resource selection
% probability functions \citep{manly_etal:2002,lele_keim:2006}. In the SCR
% context, use of this new prior results in
% a model for the inhomogeneous point process describing the
% location of activity centers, which can be used to test hypotheses
% about spatial variation in density. In
% rare cases, these covariates are truly continuous in the sense that
% they are defined as a function of space. More often, covariates are
% represented as rasters, which simplifies the analysis. Fitting these
% models can be accomplished using \bugs, \secr, or the custom \R~code
% presented in this chapter and found in the package \scrbook.
% %However,
% %at the time this book was written, \scrbook is only software available
% %for fitting models with covariates of both density and ecological
% %distance.

Although we focused on modeling the point process intensity as a
function of covariates, other options for fitting inhomogeneous models
exist \citep{illian_etal:2008}. Cox processes are models in which the
point process intensity is a function of spatial random effects. Such
methods are useful for accommodating overdispersion, but it seems unlikely that
most SCR datasets could support such complexity. Gibbs processes
are another important class of models that are distinguished by the
interactions of points. Although little work has been done on such
models in the context of SCR studies \citep{reich_etal:2012}, we
expect they will receive more
attention because they can be used to model processes such
territoriality (points repel one another) or aggregation (points
attract one another). Neyman-Scott processes are another option for
modeling aggregation or clustering, and could be useful for studying
gregarious species.

