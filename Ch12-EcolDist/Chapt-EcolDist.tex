\chapter{
%Modeling Animal space-usage with
%Detection Models based on Ecological Distance
%Ecological Distance Models in Spatial Capture-Recapture
Modeling Landscape Connectivity 
}
\markboth{Ecological Distance}{}
\label{chapt.ecoldist}


\section{Shortcomings of Euclidean Distance Models}


\begin{figure}[h]
\centering
\label{fig.distort}
\end{figure}


\section{Least-Cost Path Distance}

\begin{equation}
\label{eq.costweighted}
\end{equation}

\begin{equation}
\label{eq.lcp}
\end{equation}

\begin{equation}
\label{ecoldist.eq.cost}
\end{equation}


\subsection{Example of Computing Cost-weighted distance}


\begin{figure}[h]
\begin{center}
\end{center}
\label{ecoldist.fig.raster}
\end{figure}


\section{Simulating SCR Data using Ecological Distance}
\label{ecoldist.sec.simulating}


\begin{figure}[h]
\begin{tabular}{ll}
\end{tabular}
\label{ecoldist.fig.raster100}
\end{figure}


\begin{figure}
\begin{center}
\end{center}
\label{fig.homeranges}
\end{figure}


\section{Likelihood Analysis of Ecological Distance Models}
\label{ecoldist.sec.mle}

\begin{equation}
\label{ecoldist.eq.cond-on-s}
\end{equation}



\subsection{Example of SCR with Least-Cost Path}


\section{Bayesian Analysis}


\section{Simulation Evaluation of the MLE}


\subsection{Simulation Results}

\begin{table}[h]
\centering
\begin{tabular}{l|rrrr}
\end{tabular}
\label{tab.samplesize}
\end{table}

\begin{table}[htp]
\label{tab.results1}
\end{table}


\begin{table}[htp]
\centering
\begin{tabular}{l|rrrr}
\end{tabular}
\label{tab.results2}
\end{table}




\begin{table}[htp]
\begin{tabular}{l|rrrrr|rrrrr}
\end{tabular}
\label{tab.results3}
\end{table}





\section{Distance In an Irregular Patch}
\label{ecoldist.sec.buffer}


\subsection{Basic Geographic Analysis in R}

\begin{figure}[h]
\begin{center}
\end{center}
\label{ecoldist.fig.corridor}
\end{figure}


\begin{table}
\centering
\begin{tabular}{crrrr}
\end{tabular}
\label{rsf.tab.fakecorridor}
\end{table}


\section{Summary and Outlook}



























