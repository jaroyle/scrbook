\chapter{
%Modeling Animal space-usage with
%Detection Models based on Ecological Distance
%Ecological Distance Models in Spatial Capture-Recapture
Modeling Landscape Connectivity
}
\markboth{Ecological Distance}{}
\label{chapt.ecoldist}


\vspace{.3in}


Every spatial capture-recapture model that we have considered so far
has expressed encounter probability as function of the Euclidean
distance between individual activity centers ${\bf s}$ and trap
locations ${\bf x}$.  As a practical matter, models based on Euclidean
distance imply circular, symmetric, and stationary home ranges of
individuals, and these are not often biologically realistic.  While
these simple encounter probability models will often be sufficient for
practical purposes, especially in small data sets, sometimes
developing more complex models of the detection process as it relates
to space usage of individuals will be useful.  Animals may not judge
distance in terms of Euclidean distance but, rather, according to
the configuration of habitat patches, quality of local habitat, %landscape connectivity,
perceived mortality
risk, and other considerations. % affecting movement behavior.
Together, the degree to which these factors facilitate or impede
movement determines landscape connectivity
\citep{tischendorf_fahrig:2000}, which is widely recognized to be an
important component of population viability
\citep{with_crist:1995,compton_etal:2007}.
Moreover, because encounter probability and the distance metric upon which it is
based represent outcomes of individual movements about their home
range, ecologists might have explicit hypotheses about how
environmental variables affect the distance metric, and it is
therefore desirable to incorporate these hypotheses directly into SCR
models so that they may be formally evaluated statistically.

%Assessing the impacts of habitat fragmentation and habitat loss on
%population density and landscape connectivity are high priorities in
%applied ecological research.  %Landscape connectivity is defined as the
%degree to which landscape structure impedes or facilitates movement
%\citep{tischendorf_fahrig:2000} and is widely recognized to be an
%important component of population viability
%\citep{with_crist:1995}.
Although much theory has been developed to
predict the effects of decreasing connectivity, few empirical studies
have been conducted to test these predictions due to the paucity of
formal methods for estimating connectivity parameters
\citep{cushman_etal:2010}. Instead, ecologists often rely on expert
opinion or \textit{ad hoc} methods of specifying connectivity values,
even in important applied settings
\citep{adriaensen_etal:2003,beier_etal:2008,zeller_etal:2012}. In
addition, no methods are available for simultaneously estimating
population density and connectivity parameters, in spite of theory
predicting interacting effects of density and connectivity on
population viability \citep{tischendorf_etal:2005,cushman_etal:2010}.
In this chapter, following \citet{royle_etal:2012ecol}, we provide a
framework for modeling landscape connectivity using SCR models, by
parameterizing models for encounter probability based on ``ecological
distance''.  A natural candidate framework for modeling ecological
distance is the least-cost path which is used widely in landscape
ecology for modeling connectivity, movement and gene flow
\citep{adriaensen_etal:2003,manel_etal:2003,mcrae_etal:2008}.  In
practical applications, variables that influence landscape
connectivity, or the effective cost of moving across the landscape,
include things like highways \citep[e.g.,][]{epps_etal:2005},
elevation \citep{cushman_etal:2006}, ruggedness
\citep{epps_etal:2007}, snow cover \citep{schwartz_etal:2009},
distance to escape terrain \citep{shirk_etal:2010}, range limitations
\citep{mcrae_beier:2007}, or distance from urban areas, highways,
human disturbance or other factors that animals might avoid.


\citet{royle_etal:2012ecol} provided an SCR
framework based on least-cost path for modeling landscape
connectivity. They parameterized encounter probability
based not on Euclidean distance but, rather,
on the least-cost path between an individual's activity center and a
trap location. This is parameterized in terms of one or more
parameters that relate the {\it resistance} of the landscape to
explicit covariates.  In this way, SCR models can explicitly accommodate
landscape structure and account for connectivity of the landscape.
%For these models based on least-cost path, it is convenient to use a
%likelihood-based inference framework which we follow here in this
%chapter.
Using this
methodological extension of SCR models, it is possible to make formal
statistical inferences about movement and connectivity from
capture-recapture studies that generate sparse individual encounter
history data without subjective prescription of resistance or cost
surfaces. %, which is commonly done in practice. XXXX already said
While we believe there
should be much ecological interest in developing SCR models that
account for landscape connectivity, it is also important for obtaining
more accurate estimates of density; under simple models of landscape
connectivity, incorrectly fitting the basic model SCR0 produces
substantial bias in estimates of $N$ and hence density  \citep{royle_etal:2012ecol}.


\section{Shortcomings of Euclidean Distance Models}

In the standard SCR models, encounter probability is modeled as a
function of Euclidean distance. For example, using the binomial
observation model (Chapt. \ref{chapt.scr0}), let $y_{ij}$ be
individual- and trap-specific binomial counts with sample size $K$ and
probabilities $p_{ij}$. The Gaussian model is
\[
p_{ij} = p_{0} \exp(-  d_{ij}^2 /(2\sigma^{2}) )
\]
where $d_{ij} = ||{\bf x}_{j} - {\bf s}_{i}||$ is Euclidean
distance. As usual, we will sometimes adopt the log-scale
parameterization based on $\log(p_{ij})= \alpha_{0} + \alpha_{1}
d_{ij}^{2}$ where where $\alpha_{0} = \log(p_{0})$ and $\alpha_{1} =
-1/(2\sigma^2)$.

The main problem with the Euclidean distance metric in this encounter
probability model is that it is unaffected by habitat or landscape
structure, and it implies that the space used by individuals is
stationary and symmetric, which may be unreasonable assumptions for
some species. By stationary %here
we mean in the formal sense of
invariance to translation. That is, the properties of an individual
home range centered at some point ${\bf s}$ are exactly the same as
any other point say ${\bf s}'$.  As an example, if the common
detection model based on a bivariate normal probability distribution
function is used, then the implied space usage by {\it all}
individuals, no matter their location in space or local habitat
conditions, is symmetric with circular contours of usage intensity.

In the framework of \citet{royle_etal:2012ecol}, SCR models explicitly
incorporate information about the landscape so that a unit of distance
is variable depending on identified covariates, say
$C({\bf x})$. %$z({\bf x})$.
Thus, where an individual lives on the landscape, and the state of the
surrounding landscape, will determine the character of its usage of
space. In particular, they suggest distance metrics, based on
least-cost path, that imply irregular, asymmetric and non-stationary
home ranges of individuals. As an example, Fig. \ref{fig.distort}
shows a typical symmetric home range (left panel), and a compressed
home range (right panel) resulting from the effect of an environmental
variable (center panel) on an animal's movement behavior. We might
think of the environmental variable as representing an elevation
gradient of a valley and so, for a species that avoids high elevation,
space usage will be concentrated in flatter terrain at lower
elevations and therefore producing the elliptical home range shape.
%We reproduce the application from \citet{royle_etal:2012ecol} later in
%this chapter, in addition to providing an alternative applied context
%that involves computing distances within odd-shaped landscape patches
%(Sec. \ref{ecoldist.sec.buffer}).


\begin{figure}[h]
\centering
\includegraphics[width=5in,height=1.3in]{Ch12-EcolDist/figs/distort}
\caption{A symmetric home range (left), a habitat variable (center)
  such as representing an elevation gradient,
  and a non-symmetric home range (right) resulting from the cost imposed on
  movement by the habitat variable.}
\label{fig.distort}
\end{figure}


\section{Least-Cost Path Distance}

We adopt a cost-weighted distance metric here which defines the
effective distance between points by accumulating pixel-specific costs
determined using a cost function defined by the user.  The idea of
cost-weighted distance to characterize animal use of landscapes is
widely used in landscape ecology for modeling connectivity, movement
and gene flow \citep{beier_etal:2008}. For reasons of computational
tractability we consider a discrete landscape defined by a raster of
some prescribed resolution. The distance between any two points ${\bf
  x}$ and ${\bf x}'$ can be represented by a sequence of line segments
connecting neighboring pixels, say ${\bf l}_{1},{\bf
  l}_{2},\ldots,{\bf l}_{m}$. Then the cost-weighted distance between
${\bf x}$ and ${\bf x}'$ is
\begin{equation}
 d({\bf x},{\bf x}')
  =  \sum_{i=1}^{m-1} \mbox{cost}({\bf l}_{i},{\bf l}_{i+1})||{\bf l}_{i} - {\bf l}_{i+1}||
\label{eq.costweighted}
\end{equation}
where $\mbox{cost}({\bf l}_{i},{\bf l}_{i+1})$ is the user-defined cost to
move from pixel ${\bf l}_{i}$ to neighboring pixel ${\bf l}_{i+1}$ in
the sequence.  Given the cost of each pixel, it is a simple matter to
compute the cost-weighted distance between any two pixels, along {\it
  any} path, simply by accumulating the incremental costs weighted by
distances.  In the context of spatial capture-recapture models (and,
more generally, landscape connectivity) we are concerned with the {\it
  minimum} cost-weighted distance, or the {\it least-cost path},
between any two points which we will denote by $d_{lcp}$, which is the
sequence ${\cal P} = ({\bf l}_{1},{\bf l}_{2},\ldots,{\bf l}_{m})$
that minimizes the objective function defined by
Eq. \ref{eq.costweighted}. That is,
\begin{equation}
 d_{lcp}({\bf x},{\bf x}')
  =  \min_{{\cal P}} \sum_{i=1}^{m-1} \mbox{cost}({\bf l}_{i},{\bf l}_{i+1})||{\bf l}_{i} - {\bf l}_{i+1}||
\label{eq.lcp}
\end{equation}
The least-cost path distance can be calculated in
 many geographic information systems and other software packages,
including the {\bf R} package \mbox{\tt
  gdistance} \citep{vanetten:2011} which we use below.

The key ecological aspect of least-cost path modeling is the
development
of models for pixel-specific cost.
%In this paper we
A natural approach is to
model cost as a function of one or more covariates
defined on every pixel of the according raster. For example, using a
single covariate $C({\bf x})$ we define the cost of moving from some pixel
${\bf x}$ to neighboring pixel ${\bf x}'$ as
\begin{equation}
\log(  cost({\bf x},{\bf x}'))=  \alpha_{2}\left( \frac{C({\bf
      x})+C({\bf x}')}{2}
\right)
\label{ecoldist.eq.cost}
\end{equation}
Thus, if $\alpha_{2} = 0$ then substituting $\mbox{cost}({\bf x},{\bf x}')
=\exp(0) = 1$ into
Eq. \ref{eq.lcp} will produce the ordinary Euclidean distance
between points. Here we assume the covariate $C$ is positive-valued,
and we constrain $\alpha_{2}\ge 0$ so as to avoid
negative costs. While not necessarily problematic from a mathematical
standpoint, negative costs are unrealistic biologically.

The use of least-cost path models to model landscape connectivity has
been around for a long time. And, although $\alpha_{2}$ is rarely
known, conservation biologists design linkages that require this
resistance value as input \citep[see][and articles cited
therein]{beier_etal:2008}.  However, formal inference (e.g.,
estimation) of parameters is not often done.  Instead, in many
existing applications of least-cost path analysis, the parameter
$\alpha_{2}$ is fixed by the investigator, or based on expert opinion
\citep{beier_etal:2008}, although recently researchers have begun to
define costs based on resource selection functions\footnote{We address the integration of resource
selection models based on telemetry data with SCR models in
Chapt. \ref{chapt.rsf}.},
animal movement
\citep{tracy:2006, fortin_etal:2005}, or genetic distance data (e.g.,
\citet{gerlach_musolf:2000}; \citet{epps_etal:2007};
\citet{schwartz_etal:2009}.


To formalize the use of cost-weighted distance in SCR models, we
substitute Eq. \ref{eq.lcp} in the expression for encounter
probability (Eq. \ref{eq.encounter}) and maximize the resulting
likelihood (see  below). In doing so, we can directly
estimate parameters of the least-cost path model, evaluate how
landscape covariate influence connectivity, and test explicit hypotheses
about these things using only individual level encounter history data
from capture-recapture studies.


\subsection{Example of Computing Cost-weighted distance}

As an example of the cost-weighted distance calculation consider the
following landscape comprised of 16 pixels with unit spacing
identified as follows, along with the pixel-specific cost:
\begin{center}
\begin{verbatim}
      pixel ID                 Cost
     4 8 12 16            100   1   1  1
     3 7 11 15            100 100   1  1
     2 6 10 14            100 100 100  1
     1 5  9 13            100 100   1  1
\end{verbatim}
\end{center}
We
assume the scale is such that the distance between neighboring pixels
in any cardinal direction is 1 unit, and the distance between
neighbors on a diagonal is $\sqrt{2}$ units.  We assigned low cost of
1 to ``good habitat'' pixels (or pixels we think of as ``highly
connected'' by virtue of being in good habitat) and, conversely, we
assign high cost (100) to ``bad habitat''.
This simple cost raster is shown in Fig. \ref{ecoldist.fig.raster}.
The {\bf R} commands for creating this simple example are as follows
(which can be run using the {\bf R} script \mbox{\tt SCRed} -- see the
help file for that):
\begin{verbatim}
> library(raster)
> library(gdistance)
> r<-raster(nrows=4,ncols=4)
> projection(r)<- "+proj=utm +zone=12 +datum=WGS84" # Sets the projection
> extent(r)<-c(.5,4.5,.5,4.5) #sets the extent of the raster
> costs1<- c(100,100,100,100,1,100,100,100,1,1,100,1,1,1,1,1)
> values(r)<-matrix(costs1,4,4,byrow=FALSE) #assign the costs to the raster
> par(mfrow=c(1,1))
> plot(r)
\end{verbatim}
This produces Fig. \ref{ecoldist.fig.raster}.

\begin{figure}[h]
\begin{center}
\includegraphics[height=3.25in,width=3.25in]{Ch12-EcolDist/figs/raster_2values}
\end{center}
\caption{A $4 \times 4$ raster depicting a binary cost surface, with cost = 1 (white) or 100 (shaded) to represent ease of movement across a pixel.}
\label{ecoldist.fig.raster}
\end{figure}

For this simple case we
 can easily compute the shortest cost-weighted distance between any
pixels ``by eye''.  For example, the shortest cost-weighted distance between
pixels 5 and 9 in this example is 50.5 units: $1*(100+1)/2 = 50.5$,
the shortest distance between pixels 4 and 8 is also 50.5, while the
shortest cost-distance between 4 and 12 is 51.5.  What is the shortest
distance between 7 and 16? Suppose an individual at pixel 7 can move
diagonal (which has distance $\sqrt{2}$) and pay $\sqrt{2}(100+1)/2$,
and then move once to the right to pay $1$ additional unit cost, for a
total of $72.4$. However, if the individual instead moved one unit to
the right, to pixel 11, and then diagonally, the total cost is
$51.914$ which is the minimum cost-weighted distance in getting from
pixel 7 to 16. These two ways of moving from 7 to 16 have the same
Euclidean distance, but different cost-weighted distances according to
our cost function.

The least-cost path distances can be
computed with just a couple {\bf R} commands, and these commands can
be inserted directly into the likelihood construction for an ordinary
spatial capture-recapture model. The {\bf R} package
\mbox{\tt gdistance} calculates least-cost path using  Dijkstra's algorithm
\citep{dijkstra:1959} (from the \mbox{\tt igraph} package
\citep{csardi:2010}).
%Using \mbox{\tt gdistance}, we
%define the incremental cost of moving from one pixel to another as the
%distance-weighted {\it average} of the 2 pixel costs. We demonstrate
%how to do this subsequently.
To compute the least-cost path, or the minimum cost-weighted distances
between every pixel and every other pixel, we make use of the helper
function \mbox{\tt transition}, which calculates the cost of moving
between neighboring pixels.  It operates on the inverse-scale
(``conductance''), and so the \mbox{\tt transitionFunction} argument
is given as $1/mean(x)$.  The function \mbox{\tt geoCorrection}
modifies this object depending on the projection of the coordinate
system (e.g., it corrects for curvature of the earth's surface if
longitude/latitude coordinates are used).  The result is fed into the
function \mbox{\tt costDistance} to compute the pair-wise distance
matrix. For that, we define the center points of each raster, here
these are just integers on $[1,4] \times [1,4]$.  The commands
altogether are as follows: {\small
\begin{verbatim}
> tr1<-transition(r,transitionFunction=function(x) 1/mean(x),directions=8)
> tr1CorrC<-geoCorrection(tr1,type="c",multpl=FALSE,scl=FALSE)
> pts<-cbind( sort(rep(1:4,4)),rep(4:1,4))
> costs1<-costDistance(tr1CorrC,pts)
> outD<-as.matrix(costs1)
\end{verbatim}
}
Now we can look at the result and see if it makes sense to us. Here we
produce the first 5 columns of this distance matrix to illustrate a
couple of examples of calculating the minimum cost-weighted distance
between points:
\begin{center}
{\small
\begin{verbatim}
> outD[1:5,1:5]
         1        2        3        4        5
1   0.0000 100.0000 200.0000 205.2426 100.0000
2 100.0000   0.0000 100.0000 200.0000 141.4214
3 200.0000 100.0000   0.0000 100.0000 126.1604
4 205.2426 200.0000 100.0000   0.0000 105.2426
5 100.0000 141.4214 126.1604 105.2426   0.0000
\end{verbatim}
}
\end{center}
An interesting case is that between point 1 and 4. Note that simply
taking the shortest Euclidean distance, weighted by cost, produces a
cost-weighted distance of $100 \times 1$ to move from pixel 1 to pixel
2, and similarly from 2 to 3 and 3 to 4, producing a total
cost-weighted distance of $300$. However, the actual {\it least-cost
  path} has cost-weighted distance $205.2426$. See if you can figure
out the shortest path by inspection.

The key point here is that, once we can compute this distance matrix,
we can use it as the distance matrix in computing the encounter
probability between acctivity centers and traps, and we can use our
existing MLE technology (Chapt. \ref{chapt.mle}) to fit models that
are based on ecological distance.



\section{Simulating SCR Data using Ecological Distance}
\label{ecoldist.sec.simulating}

\citet{royle_etal:2012ecol} simulated capture-recapture data
%in the presence of
such that landscape connectivity was governed by %using
a cost function having a
single covariate, and they considered two hypothetical covariate
landscapes
(Fig. \ref{ecoldist.fig.raster100}).
%typical of how
%cost-weighted distance models might be used in real capture-recapture
%problems.
The landscape here is a $20 \times 20$ pixel raster, with
extent = $[0.5, 4.5] \times [0.5, 4.5]$.
For example, think of each pixel as
representing, say, a $1 \times 1$ km grid cell with something like
``percent developed'' or ``trail/road density'' representing the
covariate. For sampling by capture-recapture, imagine
that 16 camera traps are established at the integer coordinates
$(1,1), (1,2), \ldots, (4,4)$.
The two covariates were constructed as follows (see \mbox{\tt
  ?make.EDcovariates} for the {\bf R} commands):
First is an increasing trend from
the NW to the SE (``systematic covariate''), where $C({\bf x})$ is defined as
$C({\bf x}) = row({\bf x}) + col({\bf x})$ and $row({\bf x})$ and $col({\bf x})$ are just the row and
column, respectively, of the raster.  This might mimic something
related to distance from an urban area or a gradient in habitat
quality due to land use, or environmental conditions such as
temperature or precipitation gradients.  In the second case we make up
a covariate by generating a field of spatially correlated noise to
emulate a typical patchy habitat covariate (``patchy covariate'') such as
tree or understory density.

For both covariates we use a
cost function in which transitions from pixel ${\bf x}$ to ${\bf x}'$
is given by:
\[
 \log(\mbox{cost}({\bf x},{\bf x}'))=  \alpha_2 \left( \frac{C({\bf
       x}) + C({\bf x}')}{2} \right)
\]
where $\alpha_2 = 1$ for simulating the observed data.
 Remember that with $\alpha_2=0$ the
model reduces to one in which the cost of moving across each pixel is
constant, and therefore Euclidean distance is operative.
In the left panel of
Fig. \ref{ecoldist.fig.raster100}, a sample realization of
$N=100$ activity centers is shown. While encounter probability is
assumed to be related to landscape connectivity according to the
single-variable cost function, individual activity centers are
assumed to be uniformly distributed, although we can modify this
assumption (See Sec.~\ref{chapt.ecoldist.sec.ssed} below).


\begin{figure}[h]
\begin{tabular}{ll}
\includegraphics[height=2.5in,width=2.5in]{Ch12-EcolDist/figs/raster_withN100}
\includegraphics[height=2.5in,width=2.5in]{Ch12-EcolDist/figs/raster_krige} &
\end{tabular}
\caption{Two covariates (defined on a $20 \times 20$ grid) used in simulations.
 Left panel shows a covariate with systematic structure meant
to mimic distance from some feature, and the right panel shows a ``patchy'' covariate.
A hypothetical realization of $N=100$ activity centers (blue dots) is superimposed
  on the left figure, along with 16 trap locations. }
\label{ecoldist.fig.raster100}
\end{figure}


When distance is defined by the cost-weighted distance metric given
by Eq. \ref{eq.lcp} then individual space-usage varies
spatially in response to the landscape covariate(s) used in the
distance metric.  As a consequence, home range contours are no longer
circular, as in SCR models based on Euclidean distance.
For example, using one of the covariates we use in
our simulation study below (Fig. \ref{ecoldist.fig.raster100}, right
panel) with a Gaussian
%pdf detection function
encounter model but having distance
metric defined by Eq. \ref{eq.lcp}, produces home ranges such
as those shown in Fig. \ref{fig.homeranges}.


\begin{figure}
\begin{center}
\includegraphics[height=6in,width=3.75in]{Ch12-EcolDist/figs/home_ranges}
\end{center}
\caption{
Typical home ranges for 6 individuals based on the cost surface shown in the right panel of
  Fig. \ref{ecoldist.fig.raster100} with $\alpha_{2}=1$. The black dot indicates the home
  range center and the pixels around each home range center are shaded
according to the probability of encounter, if a trap were located in
that pixel.
}
\label{fig.homeranges}
\end{figure}

To simulate data,
 we have to load the \mbox{\tt
scrbook} package and call the function \mbox{\tt make.EDcovariates} to generate
our raster covariates (see the help file for how that is done). We
process the covariate into a least-cost path distance
matrix, and then simulate observed encounter data using standard methods
which we have used many times previously in this book. The complete set
of {\bf R} commands is:
{\small
\begin{verbatim}
### Grab a covariate
library(scrbook)
set.seed(2013)
out<-make.EDcovariates()
covariate<-out$covariate.patchy

### prescribe some settings
N<-200
alpha0<- -2
sigma<- .5
alpha1<- 1/(2*sigma*sigma)
alpha2<-1
K<- 5
S<-cbind(runif(N,.5,4.5),runif(N,.5,4.5))

# make up some trap locations
xg<-seq(1,4,1); yg<-4:1
traplocs<-cbind( sort(rep(xg,4)),rep(yg,4))
points(traplocs,pch=20,col="red")
ntraps<-nrow(traplocs)

### make a raster and fill it up with the "cost"
r<-raster(nrows=20,ncols=20)
projection(r)<- "+proj=utm +zone=12 +datum=WGS84"
extent(r)<-c(.5,4.5,.5,4.5)
cost<- exp(alpha2*covariate)

### compute least-cost path distance
tr1<-transition(cost,transitionFunction=function(x) 1/mean(x),directions=8)
tr1CorrC<-geoCorrection(tr1,type="c",multpl=FALSE,scl=FALSE)
D<-costDistance(tr1CorrC,S,traplocs)
probcap<-plogis(alpha0)*exp(-alpha1*D*D)

# now generate the encounters of every individual in every trap
# discard uncaptured individuals
Y<-matrix(NA,nrow=N,ncol=ntraps)
for(i in 1:nrow(Y)){
 Y[i,]<-rbinom(ntraps,K,probcap[i,])
}
Y<-Y[apply(Y,1,sum)>0,]
\end{verbatim}
}


\section{Likelihood Analysis of Ecological Distance Models}
\label{ecoldist.sec.mle}

Throughout much of this book we rely on Bayesian analysis by MCMC
mostly using {\bf BUGS}, but sometimes (as in Chapt. \ref{chapt.mcmc})
developing our own implementations. However, occasionally we prefer to
use likelihood estimation, such as when we can compare a set of models
directly by likelihood either to do a direct hypothesis test of a
parameter, or to tabulate a bunch of AIC values. For the class of
models that use least-cost path, we also prefer likelihood methods not
because they have any conceptual or methodological benefit, but simply
because they are more computationally efficient to implement
\citep{royle_etal:2012ecol}.

There are no technical considerations in adapting our formulation of
maximum likelihood estimation \citep{borchers_efford:2008} from
Chapt. \ref{chapt.mle} for the class of models based on least-cost
path (see the appendix in \citet{royle_etal:2012ecol} for complete details).
Likelihood analysis is really just a straightforward adaptation in which we
replace the Euclidean distance with least-cost path.  Consider the
Bernoulli model in which the individual- and trap-specific
observations have a binomial distribution conditional on the latent
variable ${\bf s}_{i}$:
\begin{equation}
  y_{ij}| {\bf s}_{i} \sim \mbox{Binomial}(K, p_{\bm \alpha}(d_{lcp}({\bf x}_{j},{\bf s}_{i};\alpha_{2}); \alpha_{0}, \alpha_{1})
\label{ecoldist.eq.cond-on-s}
\end{equation}
where we have indicated the dependence of $p$ on the parameters
${\bm \alpha} =(\alpha_{0},\alpha_{1},\alpha_{2})$, and also $d_{lcp}$ which
itself depends on $\alpha_{2}$, and the latent variable ${\bf s}_i$.
We note that the only difference between likelihood analysis of this
model and the standard Bernoulli model, is the use of $d_{lcp}$ here.
For the random effect we have ${\bf s}_{i} \sim  \mbox{Uniform}({\cal
  S})$, we can easily compute the integrated (marginal) likelihood of
an encounter history.
The likelihood is given in the {\tt scrbook} package as the function
\mbox{\tt intlik3ed}. The help file
provides an example of its usage and for simulating data.
To use this function the cost covariate $C({\bf x})$ has to be of class
\mbox{\tt RasterLayer} which requires packages \mbox{\tt sp} and
\mbox{\tt raster} to manipulate.


\subsection{Example of SCR with least-cost path}

Now we use the {\bf R} function \mbox{\tt nlm} along with
our \mbox{\tt intlik3ed} function to  obtain the MLEs of the
model parameters for the data simulated
in Sec. \ref{ecoldist.sec.simulating}.
 We'll do that for both the standard Euclidean distance
and then for the ecological distance based on the ``patchy''
covariate using the following commands:
{\small
 \begin{verbatim}
> frog1<-nlm(intlik3ed,c(alpha0,alpha1,3)),hessian=TRUE,y=Y,K=K,X=traplocs,
               distmet="euclid",covariate=covariate,alpha2=1)

> frog2<-nlm(intlik3ed,c(alpha0,alpha1,3,-.3),hessian=TRUE,y=Y,K=K,X=traplocs,
               distmet="ecol",covariate=covariate,alpha2=NA)
\end{verbatim}
}
The summary output for the two model fits is shown in Table \ref{ecoldist.tab.results1}.
The model based on least-cost path (the data generating model) appears
to be much preferred in terms of negative log-likelihood.
The output parameter order is $(\alpha_{0}, \alpha_{1}, \log(n_{0}), \text{and}
\log(\alpha_{2}))$ (remember, we want to keep $\alpha_{2}$
positive, so it's logarithm is estimated).
The data generating parameter values were
$\alpha_{0} = - 2$,
$\alpha_{1} = 2$ and $\log(\alpha_{2}) = 0$.
The simulated sampling produced a sample of 96 individuals and so the
number of individuals not captured is
$n_{0} = 104$, and $\log(n_{0}) = 4.64$. We see that the
 MLEs of the least-cost path model are pretty close whereas they are
 not so close under the misspecified model based on Euclidean distance.



\begin{table}
\caption{
Summary output of fitting models based on Euclidean and least-cost
path distance to simulated data using the
 \mbox{\tt intlik3ed} function (see \mbox{\tt ?intlik3ed}). Data were
 simulated based on the least-cost path model using the ``patchy''
 covariate shown in Fig. \ref{ecoldist.fig.raster100}.
}
\begin{tabular}{cccccc} \hline \hline
Distance metric &  -loglik &     $\alpha_0$  & $\alpha_1$  &\mbox{log}($n_{0}$) &  $\alpha_2$ \\ \hline
True value      &          &         -2      &    2       &   4.644 & 1 \\
Euclidean   &133.495&  -1.885 &    1.247 &    3.549 &      -- \\
Least-cost path (truth) & 70.119& -1.780 &  2.471    &    4.459 &
0.046  \\ \hline
\end{tabular}
\label{ecoldist.tab.results1}
\end{table}






\section{Bayesian Analysis}

While implementation of these ecological distance SCR models is
reasonably straightforward, the model cannot be fitted
in the  {\bf BUGS} engines because least-cost path distance cannot be computed.
It would be possible to fit the models
in {\bf BUGS} if the parameter $\alpha_{2}$ was fixed. In that case,
one could compute the distance matrix ahead of time and reference the
required elements for a given ${\bf s}$.
Alternatively, it would be possible to write a custom MCMC routine
using the methods we present in Chapt. \ref{chapt.mcmc}, although we
have not yet developed our own MCMC implementation of SCR models with
ecological distance metrics.


\section{Simulation Evaluation of the MLE}

\citet{royle_etal:2012ecol}
carried-out a limited simulation study to evaluate the
general statistical performance of the density estimator under
this new model, the effect of mis-specifying the model with a
normal Euclidean distance metric, and evaluate the general bias and
precision properties of the MLE using the systematic and patchy
landscapes shown in
Fig. \ref{ecoldist.fig.raster100}.

Their results showed extreme
bias in estimates of $N$ when the misspecified Euclidean distance is
used, and only negligible small-sample
 bias of 3-5\% in the MLE of $N$ using the
least-cost distance which becomes negligible as the expected seample
size increases (either due to increasing $K$, or larger population sizes).
The performance of estimating the other parameters, including the
cost parameter $\alpha_{2}$ mirrors
the results for estimating $N$.
We reproduce a subset of the results from \citet{royle_etal:2012ecol}
%here,
in Table \ref{ecoldist.tab.simresults}.

\begin{table}[htp]
\label{tab.results1}
{\small
\caption{
Simulation results for estimating population size $N$ for a prescribed state-space with
$N=100$ or $N=200$ and various levels of replication ($K$)
using the ``patchy'' landscape shown in Fig.
 \ref{ecoldist.fig.raster100}.
For each simulated
data set, the SCR model was fitted by maximum likelihood with
standard Euclidean distance (``euclid''), or least-cost path
(``lcp''), which was the true data-generating model.
The summary statistics of the
sampling distribution reported are the mean, standard deviation
(``SD'') and quantiles (0.025, 0.50, 0.975).
}
\begin{tabular}{l|rrrrr}
  \hline
         & \multicolumn{5}{c}{N=100  }  \\ \hline
         &   mean &  SD  & 0.025 & 0.50  & 0.975   \\ \hline
$K=3$      &        &      &       &       &         \\
euclid   &  78.68 & 18.12& 49.40 & 76.34 & 125.47  \\
lcp      & 110.96 & 28.65& 69.55 &106.98 & 181.84  \\
$K=5$     &        &      &       &       &         \\
euclid   &  77.85 & 11.55& 59.17 & 77.44 & 101.14  \\
lcp      & 104.44 & 15.79& 78.38 &101.47 & 139.55  \\
$K=10$     &        &      &       &       &         \\
euclid   &  78.01 & 5.26 & 68.00 & 77.96 & 87.81   \\
lcp      & 100.42 & 7.56 & 86.72 &100.34 & 115.47  \\ \hline
        & \multicolumn{5}{c}{N=200   }  \\ \hline
$K=3$      &        &      &       &       &         \\
euclid  154.34& 33.74& 107.00& 146.34& 221.43\\
lcp     208.77& 49.29& 141.68& 197.89& 325.77\\
$K=5$           &      &       &       &        \\
euclid   153.39& 15.57& 129.31& 149.54& 185.38\\
lcp      200.91& 20.78& 164.42& 200.47& 246.46\\
$K=10$           &      &       &       &       \\
euclid   156.27&  8.51& 142.17& 156.05& 174.55\\
lcp      198.45& 11.44& 180.06& 198.04& 219.52\\ \hline
\end{tabular}
}
\label{ecoldist.tab.simresults}
\end{table}














\begin{comment}
For population sizes of 100 and 200, individuals with activity
centers randomly distributed on the $20 \times 20$ landscape, they
subjected individuals
to encounter by 16 traps arranged in a $4\times 4$ grid
using a Gaussian
encounter model with least-cost path distance metric:
\[
\log(p_{ij})= \alpha_{0} + \alpha_{1} d_{lcp}({\bf x}_{j},{\bf
  s}_{i}; \alpha_{2})^{2}
\]
where  $\alpha_{0} = -2$ and $\alpha_{1} = 2$, the latter value
corresponding to $\sigma = 0.5$ of a stationary bivariate normal home
range model.  Different numbers of replicate samples were considered,
$K=3,5,10$
(e.g., nights in a camera trapping study), in order
to produce varying sample
sizes.
For each of the ``systematic'' and ``patchy'' landscapes defined
previously, 200 data sets were simulated and, for each of those, two
different models were fitted: the misspecified Euclidean distance
model; and (ii) the true data-generating model but estimating the
relative cost parameter by maximum likelihood.
\end{comment}


%%%%%%%%%%%% \subsection{Simulation Results}
\begin{comment}
For both landscapes and all simulation conditions (levels of $K$ and
$N$) the average sample sizes of individuals captured are given in
Tab. \ref{tab.samplesize}.
\begin{table}[h]
\centering
\caption{
Expected sample sizes of captured individuals under each configuration of
$N$ (population size for the prescribed state-space) and $K$ (number of replicate samples).
}
\begin{tabular}{l|rrrr}
 & \multicolumn{2}{c}{Systematic} & \multicolumn{2}{c}{Patchy}  \\
    & N=100 &  N=200  &   N=100 &  N=200  \\ \hline
K=3 &  38.69 &   78.17  &   37.30 &   74.93  \\
K=5 &  51.10 &  103.18  &   51.89 &  103.71 \\
K=10&  65.81 &  132.39  &   69.44 &  138.76 \\
\end{tabular}
\label{tab.samplesize}
\end{table}
The simulation results for estimating $N$
for the prescribed state-space are presented in Tab.
\ref{tab.results1}.  For the ``patchy'' landscape we see extreme
bias in estimates of $N$ when the Euclidean distance is used. There is
moderate small sample bias of 3-5\% in the MLE of $N$ using the
least-cost distance which becomes negligible as $K$ increases. For
$N=200$ the bias is on the order of 2\% for the lowest sample size
case ($K=3$) but negligible otherwise.  Interestingly, for the
landscape exhibiting systematic structure, there is a persistent bias
in the MLE of $N$ of 1-3\% even for the highest level of $K$.
As noted by \citet{royle_etal:2012ecol},
this is due to the fact that
the state-space is small relative to the extent of the trapping grid and
sensitivity to a state-space that is too small is expected because the
support of the integrand is truncated. In the particular case of the
systematic landscape, we find that, in the NW corner of the raster
where cost of movement is low, individuals use large areas of space,
and the fitted model is under-stating the apparent
heterogeneity in encounter probability for the prescribed raster.  \citet{royle_etal:2012ecol}
found that the issue is resolved when the traps are moved away from
the boundary (results shown in Tab. \ref{tab.results3}).

The performance of estimating the cost parameter $\alpha_{2}$ mirrors
the results for estimating $N$ for the prescribed state space. In the
patchy landscape where we don't expect a systematic gradient in space
usage around the edge of the state-space, we see
(Table \ref{tab.results2}) that $\alpha_{2}$ is estimated with
diminishing bias as the sample size increases, but with persistent
bias due to truncation of the likelihood under the systematic
landscape which, as with the MLE of $N$, is resolved by moving the
traps away from the edge of the raster. Equivalently, in practice,
this could be resolved by expanding the raster away from the trap
locations so that all regions used by animals exposed to capture are
included in the state-space.
















\begin{table}[htp]
\label{tab.results1}
{\small
\caption{Simulation results for estimating population size $N$ for a prescribed state-space with
$N=100$ or $N=200$ and various levels of replication ($K$) chosen to affect the observed sample
size of individuals (Tab. \ref{tab.samplesize}). For each simulated
data set, the SCR model was fitted by maximum likelihood with
standard Euclidean distance (``euclid''), or least-cost path
(``lcp''), which was the true data-generating model.
The summary statistics of the
sampling distribution reported are the mean, standard deviation
(``SD'') and quantiles (0.025, 0.50, 0.975).
}
{\bf Systematic trend raster:} \\
\begin{tabular}{l|rrrrr|rrrrr}
         & \multicolumn{5}{c}{N=100   } & \multicolumn{5}{c}{N=200  }  \\
         &   mean &  SD  & 0.025 & 0.50 & 0.975  & mean  & SD   & 0.025 & 0.50  & 0.975 \\ \hline
K=3      &        &      &       &      &        &       &      &       &       &       \\
euclid   &   63.65& 12.62& 44.77 & 61.17&  90.98 & 126.68& 17.05&  98.93& 124.49& 168.26 \\
lcp      &  101.93& 21.68& 67.95 &101.56& 156.21 & 201.58& 28.14& 154.96& 200.15& 263.20\\
K=5      &        &      &       &      &        &       &      &       &       &        \\
euclid   &  64.60 & 7.11 & 51.52 & 63.86&  77.33 & 130.02& 10.25& 113.48& 128.96& 151.32\\
lcp      &  98.94 &12.97 & 74.68 & 99.00& 123.88 & 198.80& 19.60& 166.87& 197.97& 239.46\\
K=10     &        &      &       &      &        &       &      &       &       &       \\
euclid   &  69.24 & 4.83 & 59.37 & 69.47&  79.18 & 139.83&  7.62& 125.65& 139.65& 154.82\\
lcp      &  97.53 & 8.18 & 82.02 & 97.62& 113.16 & 195.19& 13.28& 171.63& 194.58& 217.96\\ \hline
\end{tabular}
\\
{\bf Patchy ``random'' raster: } \\
\begin{tabular}{l|rrrrrrrrrr}
         & \multicolumn{5}{c}{N=100  } & \multicolumn{5}{c}{N=200   }  \\
         &   mean &  SD  & 0.025 & 0.50  & 0.975  & mean  & SD   & 0.025 & 0.50  & 0.975 \\ \hline
K=3      &        &      &       &       &        &       &      &       &       &       \\
euclid   &  78.68 & 18.12& 49.40 & 76.34 & 125.47 & 154.34& 33.74& 107.00& 146.34& 221.43\\
lcp      & 110.96 & 28.65& 69.55 &106.98 & 181.84 & 208.77& 49.29& 141.68& 197.89& 325.77\\
K=5      &        &      &       &       &        &       &      &       &       &        \\
euclid   &  77.85 & 11.55& 59.17 & 77.44 & 101.14 & 153.39& 15.57& 129.31& 149.54& 185.38\\
lcp      & 104.44 & 15.79& 78.38 &101.47 & 139.55 & 200.91& 20.78& 164.42& 200.47& 246.46\\
K=10     &        &      &       &       &        &       &      &       &       &       \\
euclid   &  78.01 & 5.26 & 68.00 & 77.96 & 87.81  & 156.27&  8.51& 142.17& 156.05& 174.55\\
lcp      & 100.42 & 7.56 & 86.72 &100.34 & 115.47 & 198.45& 11.44& 180.06& 198.04& 219.52\\ \hline
\end{tabular}
}
\end{table}

















\begin{table}[htp]
\centering
\caption{
Mean of sampling distribution of the cost function parameter
$\alpha_{2}$ for the different simulation
conditions.
}
\begin{tabular}{l|rrrr}
 & \multicolumn{2}{c}{Patchy} & \multicolumn{2}{c}{Systematic} \\
    & N=100 &  N=200  &   N=100 &  N=200  \\ \hline
$K=3$  &   1.05&    1.03 &     1.17 & 1.14 \\
$K=5$  &   1.02&    1.01 &     1.12 &1.12 \\
$K=10$ &   1.01&    1.00 &     1.10 &1.08 \\
\end{tabular}
\label{tab.results2}
\end{table}




\begin{table}[htp]
{\tiny
\caption{Simulation results for estimating population size $N$ for a prescribed state-space with
$N=100$ or $N=200$ and various levels of replication ($K$) chosen to affect the observed sample
size of individuals. These results correspond to those of the
systematic landscape in Table XXXXXXX  except with the traps
moved 0.5 units in from the boundary of the landscape.
Each grouping of 2 rows (for a given value of $K$) summarizes the
performance of $\hat{N}$ under models based on
Euclidean distance  (``euclid'') and
a model based on least-cost path, which was the true data-generating model.
The summary statistics of the
sampling distribution reported are the mean, standard deviation
(``SD'') and quantiles (0.025, 0.50, 0.975).
}
\begin{tabular}{l|rrrrr|rrrrr}
         & \multicolumn{5}{c}{N=100   } & \multicolumn{5}{c}{N=200  }  \\
         &   mean &  SD  & 0.025 & 0.50 & 0.975  & mean  & SD   & 0.025 & 0.50  & 0.975 \\ \hline
K=3      &        &      &       &      &        &       &      &       &       &       \\
euclid   &   84.48& 20.42& 51.16 & 81.51& 140.62 &163.70 &24.55 &126.64 &157.67 &223.63 \\
lcp      &  105.90& 26.19& 65.95 &103.40& 182.30 &201.34 &29.54 &161.88 &192.36 &268.98\\
K=5      &        &      &       &      &        &       &      &       &       &       \\
euclid   & 81.21  &11.33 &61.35  &79.20 & 98.86  &163.27 &13.06 &140.21 &162.97 &185.94\\
lcp      & 100.84 &13.15 &79.96  &99.51 &119.08  &200.25 &16.53 &168.88 &199.29 &227.39\\
K=10     &        &      &       &      &        &       &      &       &       &       \\
euclid   &  80.10 & 7.81 &66.45  &79.14 &93.33   &158.40 & 9.25 &142.74 &157.86 &173.18\\
lcp      & 100.10 & 9.88 &82.31  &100.91&116.27  &197.52 &13.03 &169.49 &200.68 &217.82\\ \hline
\end{tabular}
}
\label{tab.results3}
\end{table}




\end{comment}









\section{Distance In an Irregular Patch}
\label{ecoldist.sec.buffer}

We provide another illustration of how to employ ecological distance
calculations in SCR models. This example is meant to mimic
a situation where we have something like a hard habitat boundary
such as a habitat corridor or park unit or some other block
of relatively homogeneous good-quality habitat for some species. This
particular system (shown in Fig. \ref{ecoldist.fig.corridor}) could
be habitat surrounded by a suburban wasteland of McDonuts and
Beer-Marts, much less hospitable habitat for most species.  For our
purposes, we suppose that individuals live within the buffered
``f-shaped''
region, although we could also imagine the negative of the
situation in which individuals live outside of the region, so that the
polygon represents a barrier (a lake) or bad habitat (an urban area)
or similar.  We describe the steps for creating this landscape
shortly, so that you can use a similar process to generate more
relevant landscapes for your own problems.

In this case we're not going to estimate any parameters of the cost
function (though you could adapt the analyses of the previous sections
to do that) but instead we're going to use ecological
distance ideas only to constrain movement within (or to avoid)
landscape features. Note that, normally, distance ``as the crow
flies'' would not be suitable for irregular habitat patches such as
that shown in Fig. \ref{ecoldist.fig.corridor}.


\subsection{Basic Geographic Analysis in R}

In practical applications our landscape will contain %one more more
polygons which delineate good or bad habitat or other important
characterisetics of the landscape.  These might exist as GIS
shapefiles or merely as a text file with coordinates defining polygon
boundaries. To work with polygons in the context of SCR models we need
to create a raster, overlay the polygon and assign values to each pixel
depending on whether pixels are in the polygon or not, or how far they
are from polygon boundaries. These operations are relatively easy to
do within a GIS system but we need to be able to do them in ${\bf R}$
in order to compute the least-cost paths needed in the likelihood
evaluation. Some additional geographic analyses have been discussed in
Secs. \ref{ecoldist.sec.shapefile} and \ref{mcmc.sec.state-space}
where we talked about reading in the shapefile and doing SCR calculations
on that.

Often we will have GIS shapefiles that define polygons but, here, we
 create a set of polygons by
buffering and joining some line segments.
In the {\bf R} package \mbox{\tt scrbook}, we provide
 a function \mbox{\tt make.seg} which allows you to make such
 lines segments given a
specific trap region.  To use %involve
\mbox{\tt make.seg} we first
create a plot region and then call \mbox{\tt make.seg} which has a
single argument being the number of points used to define the line
segment. The user will click on the visual display until the required
number of points has been obtained by \mbox{\tt make.seg}.
In the following set of commands we generate two line
segments, \mbox{\tt l1} consisting of 9 points and \mbox{\tt l2}
consisting of 5 points, and these reside in a geographic region
enclosedd by $[0,10] \times [0,10]$:
{\small
\begin{verbatim}
library(scrbook)
library(sp)
plot(NULL,xlim=c(0,10),ylim=c(0,10))
l1<-make.seg(9)
plot(l1)
l2<-make.seg(5)
plot(l1)
lines(l2)
\end{verbatim}
}

We used this function as above to create a habitat corridor compose of
line segments of class
\mbox{\tt SpatialLines} from the {\bf R} package \mbox{\tt sp}. The
corridor can be loaded from \mbox{\tt scrbook} by typing the command
\mbox{\tt data(fakecorridor)}.
This data list has 2 line files in it (\mbox{\tt l1} and \mbox{\tt l2}) and a
trap locations file (\mbox{\tt traps}).
We use some functions from the {\bf R} packages \mbox{\tt sp} and
\mbox{\tt rgeos} to join and
buffer (by 0.5 units) the two segments. The commands are as follows
and the result is shown in Fig. \ref{ecoldist.fig.corridor}.

{\small
\begin{verbatim}
data(fakecorridor)
library(sp)
library(rgeos)

buffer<- 0.5
par(mfrow=c(1,1))
aa<-gUnion(l1,l2)
plot(gBuffer(aa,width=buffer),xlim=c(0,10),ylim=c(0,10))
pg<-gBuffer(aa,width=buffer)
pg.coords<- pg@polygons[[1]]@Polygons[[1]]@coords

xg<-seq(0,10,,40)
yg<-seq(10,0,,40)

delta<-mean(diff(xg))
pts<- cbind(sort(rep(xg,40)),rep(yg,40))
points(pts,pch=20,cex=.5)

in.pts<-point.in.polygon(pts[,1],pts[,2],pg.coords[,1],pg.coords[,2])
points(pts[in.pts==1,],pch=20,col="red")
\end{verbatim}
}

\begin{figure}[h]
\begin{center}
\includegraphics[height=3.25in,width=3.25in]{Ch12-EcolDist/figs/corridor}
\end{center}
\caption{A fake wildlife corridor or reserve. The boundary outlines
  a polygon of suitable habitat surrounded by suburban development.}
\label{ecoldist.fig.corridor}
\end{figure}

In this example, we're not going to estimate parameters of the cost
function. Instead, the point is to compute ordinary Euclidean distance
but restricted by the boundaries of the corridor (or patch geometry in
general) and thus not distance ``as the crow flies.''  To do this, we
imagine that animals will tend to severely avoid leaving the buffered
habitat zone. Therefore, we assign $\mbox{\tt cost}=1$ if a pixel is
within the buffer, and $\mbox{\tt cost} = 10000$ if a pixel is outside
of a buffer. Therefore the cost to move to a neighboring pixel outside
of the buffered area is $5000.5$ compared to the cost of 1 to move to
a neighboring pixel inside the buffer.  With this cost specification,
we can compute the least-cost path distance matrix one time and modify
our likelihood code to accept the distance matrix as input. We give
that likelihood in the package \mbox{\tt scrbook} as the function
\mbox{\tt intlik3edv2}.  We note also that this function accepts a
habitat mask in the form of a vector of 0's
and 1's
that define any potential state-space restrictions. i.e., 1 if
the pixel is an element of the state-space and 0 if it is not, and so
additional modifications to the geometry of the region could be made.
However, in the analysis of this simulated data set, we define the
state-space to be the buffered corridor system.  Here we simulate a
population of $N=200$ individuals in the corridor system and so we
restrict our state-space accordingly for purposes of fitting the
model. However we encourage you to refit the model without the
state-space restriction (for fitting the model only) and then compare
the results.  The code for doing all of this is in the help file for
\mbox{\tt intlik3edv2}, which contains the likelihood function and
sample {\bf R} script (\mbox{\tt ?intlik3edv2}).

{\small
\begin{verbatim}
### Define the cost structure
cost<-rep(NA,nrow(pts))
cost[in.pts==1]<-1      # low cost to move among pixels but not 0
cost[in.pts!=1]<-10000  # high cost

### Stuff costs into a raster
library("raster")
r<-raster(nrows=40,ncols=40)
projection(r)<- "+proj=utm +zone=12 +datum=WGS84"
extent(r)<-c(0-delta/2,10+delta/2,0-delta/2,10+delta/2)
values(r)<-matrix(cost,40,40,byrow=FALSE)

# check what it looks like
plot(r)
points(pts,pch=20,cex=.4)

# compute ecological distances:
library("gdistance")
tr1<-transition(r,transitionFunction=function(x) 1/mean(x),directions=8)
tr1CorrC<-geoCorrection(tr1,type="c",multpl=FALSE,scl=FALSE)
costs1<-costDistance(tr1CorrC,pts)
outD<-as.matrix(costs1)
\end{verbatim}
}

In the next block of code we simulate some data and then fit a model
to the simulated data.  Note that the object \mbox{\tt traps} is
loaded with \mbox{\tt data(fakecorridor)} along with the data which
define the f-shaped patch in
Fig. \ref{ecoldist.fig.corridor}:
{\small
\begin{verbatim}
library(scrbook)
traplocs<-traps$loc
trap.id<-traps$locid
ntraps<-nrow(traplocs)

set.seed(2013)
N<-200
S.possible<- (1:nrow(pts))[in.pts==1]
S.id<-sample(S.possible,N,replace=TRUE)
S<- pts[S.id,]

D<- outD[S.id,trap.id]
eD<- e2dist(S,traplocs)
Dtraps<-outD[trap.id,]

alpha0<- -1.5
sigma<- 1.5
alpha1<- 1/(2*sigma*sigma)
K<-10

probcap<-plogis(alpha0)*exp(-alpha1*D*D)
Y<-matrix(NA,nrow=N,ncol=ntraps)
for(i in 1:nrow(Y)){
 Y[i,]<-rbinom(ntraps,K,probcap[i,])
}
Y<-Y[apply(Y,1,sum)>0,]

frog1<-nlm(intlik3edv2,c(-2.5,2,log(4)),hessian=TRUE,y=Y,K=K,X=traplocs,
            S=pts,D=Dtraps,inpoly=in.pts)
frog2<-nlm(intlik3edv2,c(-2.5,2,log(4)),hessian=TRUE,y=Y,K=K,X=traplocs,
            S=pts,D=Deuclid,inpoly=in.pts)
\end{verbatim}
}

These two models fit, with the correctly specified ecological
distance, constrained by the patch boundaries, and that with the
ordinary (misspecified) Euclidean distance are summarized in Table \ref{rsf.tab.fakecorridor}.
We find little difference between the two models. In
particular, 150 individuals were captured and so truth (the number of
uncaptured individuals) is $\log(n_{0}) = 3.9$.
The correct model produces only a slightly more accurate  estimate, and
it is favored by only 0.7 negative log-likelihood units.
Therefore, for this single instance, the results are not too different.
This is primarily because
 the distance between individuals, and traps that they are likely
to be captured in, is well-approximated by %the
Euclidean distance.


\begin{table}
\centering
\caption{
Summary output of fitting models to simulated data in which movement
is restricted by the habitat corridor shown in
Fig. \ref{ecoldist.fig.corridor}. The two models fitted were those
based on distance  constrained by the corridor boundary
(``constrained'') and a misspecified model based on ordinary Euclidean
distance which is ``as the crow flies'', and cuts through some
boundaries.
See \mbox{\tt ?fakecorridor} for the {\bf R} commands to fit these
models.
}
\begin{tabular}{c|rrrr} \hline \hline
Distance    &  neg. LL &    $\alpha_0$   & $\alpha_1$    & $\log(n_0)$ \\ \hline
constrained & -21.892 &  -1.338 & 0.332 & 4.353 \\
Euclidean   & -21.128 &  -1.307 & 0.382 & 4.212 \\ \hline
\end{tabular}
\label{rsf.tab.fakecorridor}
\end{table}


\section{Ecological Distance and Density Covariates}
\label{chapt.ecoldist.sec.ssed}

Habitat characteristics that affect spatial variation in density can
also affect home range size and movement behavior. For example, a
species that occurs at high density in a forest may be reluctant to
venture from a forest patch into an adjacent field. Thus, even if a
trap placed in a field is located very close to an animal's activity
center, the probability of capture may be very
low. In this case, forest cover is a covariate of
both density and encounter probability,
and we could model it as such by combining the methods described in
this chapter with those described in Chapter~\ref{chapt.state-space}.

To demonstrate, we continue with our analysis of the data shown in
Fig~\ref{state-space.fig.discrete}. Once again, we suppose that density
increases with canopy height, but this time, we also allow
home range size to decrease as density increases. This
commonly-observed phenomenon can be explained by numerous factors such
as intra-specific competition \citep{sillett_etal:2004} or optimal
foraging behavior \citep{tufto_etal:1996,said_servanty:2005}.
%To model
%this effect, we
%introduce the parameter $\theta$, which determines the ``cost'' of
%moving between pixels. If $\theta=0$, then the animal perceives
%distance as Euclidean. If $\theta>0$, then least-cost distance (LCD)
%is greater than than Euclidean distance (ED). In most cases, we would
%not expect,
%or should not even consider the possibility of $\theta<0$ because this
%implies that LCD$<$ED, which would mean that an animal could view
%1000km as 1m. In addition to the fact that this is not biologically
%justifiable, it also suggests that the area of the state-space could
%be infinitely large. Thus, one may want to enforce the constraint that
%$\theta$ is $\geq 0$. See Chapter~\ref{chapt.ecoldist} for
%more details.

A question that arises is: Is it possible to estimate the effect of
the covariate on density ($\bm \beta_1$)
and $\alpha_2$ using standard SCR data? In other words, can we model
spatial variation in density and connectivity at the same time,
using standard SCR data? Currently, it is not possible to
model least-cost distance using \jags~or \secr, so we wrote our own
function, \verb+scrDED+, to fit the model using maximum likelihood. An
example analysis is provided on the help page for the function in our
\R~package \scrbook. We briefly note here that the function requires
the capture history data, the trap locations, and the raster data
formatted using the {\tt raster} package
\citep{hijmans_vanetten:2012}. The linear model for the
intensity parameter $\mu(\mathbf{s}, \beta)$ and the least-cost distance
function $\text{lcd}(\theta)$ are specified using \R's formula interface. A
simple function call is
\begin{verbatim}
fm <- scrDED(y, traplocs=X, den.formula=~elev, dist.formula=~elev,
             rasters=elev.raster)
\end{verbatim}
To assess the possibility of estimating both $\bm \beta$ and $\alpha_2$, we
conducted a small simulation study, generating 500 datasets from the
model with both parameters set to 1, which corresponds to the
conditions described above. The results indicate that it is
possible to estimate both parameters
(Fig~\ref{chapt.ecoldist.fig.simDED}).

\begin{figure}[ht]
\centering
\includegraphics[width=4in,height=2in]{Ch12-EcolDist/figs/scrDEDsim}
\caption{Histograms of parameter estimates from 500 simulations under
  the model in which both density and ecological distance are affected
by the same covariate, canopy height. The vertical lines indicate the
data-generating value.}
\label{chapt.ecoldist.fig.simDED}
\end{figure}



\section{Summary and Outlook}


Almost
all published applications of SCR models to date have been based on
models for the encounter probability that are functions of the
Euclidean distance between individual activity centers and traps. The
obvious limitations of such models are that Euclidean
distance is unaffected by landscape or habitat
structure and implies stationary, isotropic and symmetrical home
ranges. These are standard criticisms of the basic SCR model which we
have seen many times in referee reports, or heard in discussions with
colleagues. However, this should not be seen as criticism
that is inherent to the basic conceptual formulation of SCR models because,
as we have shown here, % that
one can modify the Euclidean distance metric
to accommodate more realistic
formulations of distance that allow for inference to be made about
landscape connectivity, and model ``distance'' as a function of
local habitat characterists. As such, effective distance between individual home
range centers and traps varies depending on the local landscape.

How animals use space and therefore how distance to a trap is
perceived by individuals is not something that can ever be known. We
can only ever conjure up models to describe this phenomenon and fit
those models to limited data on a sample of individuals during a
limited amount of time.  Here we have shown that there is hope to
estimate connectivity parameters
that describe how
animals use space,  from capture-recapture data alone,
thereby allowing  for irregular home range geometry
that is influenced by landscape structure.

In the presence of unctional landscape connectivity, misspecification
of the model by an ordinary SCR model based on Euclidean distance
produces biased estimates of model parameters
 \citep{royle_etal:2012ecol}.
 This is expected because the effect is similar
to failing to model heterogeneity, i.e., if we mis-specify ``model
$M_h$'' \citep{otis_etal:1978} with ``model $M_0$''
\citep{otis_etal:1978} then we will expect to under-estimate $N$. So
the effect of mis-specifying the ecological distance metric with a
standard homogeneous Euclidean distance has the same effect.
In our view, this bias is not really the most important reason to
consider models of ecological distance. Rather, inference about the
structure of ecological distance is fundamental to many problems in
applied and theoretical ecology related to modeling landscape
connectivity, corridor and reserve design, population viability
analysis, gene flow, and other phenomena.  Models based on least-cost
path distance allow investigators to evaluate landscape factors that
influence movement of individuals over the landscape from
non-invasively collected capture-recapture data.  Therefore SCR models
based on ecological distance metrics might aid in understanding
aspects of space usage and movement in animal populations and,
ultimately, in addressing conservation-related problems such as
corridor design.





























