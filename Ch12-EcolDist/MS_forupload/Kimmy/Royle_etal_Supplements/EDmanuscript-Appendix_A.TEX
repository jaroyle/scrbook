\documentclass[12pt]{article}

\usepackage[total={6.5in,8.75in}, top=2.4cm, left=2.4cm]{geometry}
\usepackage{lineno}
\usepackage{amsmath}
%\usepackage{amssymb}    % used for symbols in figure legends
\usepackage{graphicx}
\usepackage[round,colon,authoryear]{natbib}

\usepackage{bm}
\usepackage{float}
\usepackage{amsmath}
\usepackage{amsfonts}
\usepackage{hyperref}
\usepackage{verbatim}
\usepackage{soul}
\usepackage{color}
\usepackage{setspace}
\usepackage{caption}


\bibliographystyle{ecology} % kluwer, plos-natbib, pnas-natbib

\title{Ecological Distance in Spatial Capture-Recapture Models}

\author{
{\bf J. Andrew Royle}\\
USGS Patuxent Wildlife Research Center, Laurel MD \\ \\
{\bf Richard B. Chandler} \\
USGS Patuxent Wildlife Research Center, Laurel MD\\ \\
{\bf Kimberly D. Gazenski} \\
USGS Patuxent Wildlife Research Center, Laurel MD\\ \\
{\bf Tabitha A. Graves} \\
Northern Arizona University, Flagstaff AZ \\ \\
}



\begin{document}

\maketitle

\date

\newpage

\linenumbers


\begin{spacing}{1.2}



\section*{Appendix A: Details for computing the marginal likelihood
  and obtaining the MLEs}


The key operation for computing the likelihood is solving the
2-dimensional integration problem to remove ${\bf s}$. There are some
general purpose {\bf R} packages that implement a number of
multi-dimensional integration routines including 
%\mbox{\tt adapt} \citep{genz_etal:2007} and  %% Not on CRAN anymore
\mbox{\tt R2Cuba} \citep{hahn_etal:2011}.
We won't rely on these extraneous {\bf R} packages but instead will
use perhaps less efficient methods in which we replace the integral
with a summation over an equal area mesh of points on the state-space
${\cal S}$ and explicitly evaluate the integrand at each point. We
invoke the rectangular rule for integration here in which the
integrand is evaluated on a regular grid of points of equal area and
then averaged.  Let $u=1,2,\ldots,nG$ index a grid of $nG$ points,
${\bf s}_{u}$, where the area of grid cell $u$ is constant.  In this
case, the integrand, i.e., the marginal pmf of ${\bf y}_{i}$, is
approximated by

\begin{equation}
         [{\bf y}_{i}|\alpha] = \frac{1}{nG} \sum_{u=1}^{nG}  [ {\bf
            y}_{i} |{\bf s}_u, \alpha]
\label{mle.eq.intlik}
\end{equation}

To deal with the fact that $N$ is unknown, there are two key issues
that need to be addressed.  First is that we don't observe the
``all-zero'' encounter histories (i.e., $y_{ij} = 0$ for all $j$)
corresponding to uncaptured individuals, so we have to make sure we
compute the probability for that all zero encounter history which we
do operationally by tacking a row of zeros onto the encounter history
matrix. We include the number of such all-zero encounter histories as
an unknown parameter of the model, which we label $n_{0}$.  In
addition, we have to be sure to include a combinatorial term to
account for the fact that of the $n$ observed individuals there are
${N \choose n}$ ways to realize a sample of size $n$. The
combinatorial term involves the unknown $n_{0}$ and thus it must be
included in the likelihood.

To compute the integral requires that the bounds of integration are
specified, which is equivalent to prescribing the state-space of the
underlying point process, i.e., ${\cal S}$. Given ${\cal S}$, density
is
computed as $D({\cal S}) = N/||{\cal S}||$. In our simulation study
below we report $N$ as the two are equivalent summaries of the data
set once the state-space is fixed.

We wrote an {\bf R} function to evaluate the likelihood which we optimize
using the {\bf R} function \mbox{\tt optim} (or, alternatively,
\mbox{\tt nlm}).





\bibliography{EDmanuscript-Appendix_A.bbl}


\end{spacing}

\end{document}






