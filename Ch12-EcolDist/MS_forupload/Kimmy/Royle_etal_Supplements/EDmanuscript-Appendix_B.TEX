\documentclass[12pt]{article}

\usepackage[total={6.5in,8.75in}, top=2.4cm, left=2.4cm]{geometry}
\usepackage{lineno}
\usepackage{amsmath}
%\usepackage{amssymb}    % used for symbols in figure legends
\usepackage{graphicx}
\usepackage[round,colon,authoryear]{natbib}

\usepackage{bm}
\usepackage{float}
\usepackage{amsmath}
\usepackage{amsfonts}
\usepackage{hyperref}
\usepackage{verbatim}
\usepackage{soul}
\usepackage{color}
\usepackage{setspace}
\usepackage{caption}


\renewcommand{\thetable}{B\arabic{table}}
\renewcommand{\theequation}{A.\arabic{equation}}



\bibliographystyle{ecology} % kluwer, plos-natbib, pnas-natbib


\title{Ecological Distance in Spatial Capture-Recapture Models}

\author{
{\bf J. Andrew Royle}\\
USGS Patuxent Wildlife Research Center, Laurel MD \\ \\
{\bf Richard B. Chandler} \\
USGS Patuxent Wildlife Research Center, Laurel MD\\ \\
{\bf Kimberly D. Gazenski} \\
USGS Patuxent Wildlife Research Center, Laurel MD\\ \\
{\bf Tabitha A. Graves} \\
Northern Arizona University, Flagstaff AZ \\ \\
}



\begin{document}

\maketitle

\date

\newpage

\linenumbers


\begin{spacing}{1.2}


\section*{Appendix B:  MLE summary statistics for reanalysis of the
  systematic landscape}


For both landscapes and all simulation conditions (levels of $K$ and
$N$) the average sample sizes of individuals captured are given in
Table~\ref{tab.samplesize}.  


\begin{table}[htp]
\centering
\caption{Expected sample sizes of captured individuals under each configuration of
$N$ (population size for the prescribed state-space) and $K$ (number of replicate samples).
}
\begin{tabular}{l|rrrr}
 & \multicolumn{2}{c}{Systematic} & \multicolumn{2}{c}{Patchy}  \\
    & N=100 &  N=200  &   N=100 &  N=200  \\ \hline
K=3 &  38.69 &   78.17  &   37.30 &   74.93  \\
K=5 &  51.10 &  103.18  &   51.89 &  103.71 \\
K=10&  65.81 &  132.39  &   69.44 &  138.76 \\
\end{tabular}
\label{tab.samplesize}
\end{table}



\begin{table}[htp]
{\small
\caption{Simulation results for estimating population size $N$ for a prescribed state-space with
$N=100$ or $N=200$ and various levels of replication ($K$) chosen to affect the observed sample
size of individuals. These results correspond to those of the
systematic landscape in Table B2 except with the traps
moved 0.5 units in from the boundary of the raster.
Each grouping of 3 rows (for a given value of $K$) summarizes the
performance of $\hat{N}$ under 3 distance models: (1) A model in which
Euclidean distance was used (``euclid''); (2) A model in which the
least-cost path distance was used, with the coefficient of the cost
function fixed (``lcp/known''); and (3) A model in which the
coefficient was estimated (``lcp/est''). The summary statistics of the
sampling distribution reported are the mean, standard deviation
(``SD'') and quantiles (0.025, 0.50, 0.975).
}
{\bf Systematic trend raster:} \\
\begin{tabular}{l|rrrrr|rrrrr}
         & \multicolumn{5}{c}{N=100   } & \multicolumn{5}{c}{N=200  }  \\
         &   mean &  SD  & 0.025 & 0.50 & 0.975  & mean  & SD   & 0.025 & 0.50  & 0.975 \\ \hline
K=3      &        &      &       &      &        &       &      &       &       &       \\
euclid   &   84.48& 20.42& 51.16 & 81.51& 140.62 &163.70 &24.55 &126.64 &157.67 &223.63 \\
lcp/known&  104.14& 25.49& 65.67 &101.50& 173.19 &200.16 &29.27 &158.65 &191.04 &268.78\\
lcp/est  &  105.90& 26.19& 65.95 &103.40& 182.30 &201.34 &29.54 &161.88 &192.36 &268.98\\
K=5      &        &      &       &      &        &       &      &       &       &       \\
euclid   & 81.21  &11.33 &61.35  &79.20 & 98.86  &163.27 &13.06 &140.21 &162.97 &185.94\\
lcp/known& 99.93  &12.86 &76.97  &99.75 &117.76  &199.80 &16.60 &170.25 &198.23 &227.66\\
lcp/est  & 100.84 &13.15 &79.96  &99.51 &119.08  &200.25 &16.53 &168.88 &199.29 &227.39\\
K=10     &        &      &       &      &        &       &      &       &       &       \\
euclid   &  80.10 & 7.81 &66.45  &79.14 &93.33   &158.40 & 9.25 &142.74 &157.86 &173.18\\
lcp/known& 100.07 & 9.50 &82.99  &100.33&114.81  &197.62 &12.58 &171.95 &199.21 &217.19\\
lcp/est  & 100.10 & 9.88 &82.31  &100.91&116.27  &197.52 &13.03 &169.49 &200.68 &217.82\\ \hline
\end{tabular}
}
\label{tab.results3}
\end{table}






\begin{table}[htp]
\centering
\caption{Mean of sampling distribution of the cost function parameter
$\alpha_{2}$ for the different simulation
conditions.  The simulation for the systematic landscape was repeated
with sample locations concentrated away from the boundary of the
landscape to avoid truncation bias in computing the integrated
likelihood. Results for that case are shown in rows 4-6.
}
\begin{tabular}{l|rrrr}
 & \multicolumn{2}{c}{Patchy} & \multicolumn{2}{c}{Systematic} \\
    & N=100 &  N=200  &   N=100 &  N=200  \\ \hline
K=3 &   1.05&    1.03 &     1.17 & 1.14 \\
K=5 &   1.02&    1.01 &     1.12 &1.12 \\
K=10&   1.01&    1.00 &     1.10 &1.08 \\ \hline
K=3    &       &         &     1.08 & 1.04 \\
K=5    &       &         &     1.02 & 1.02 \\
K=10    &       &         &     1.01 & 1.01 \\
\end{tabular}
\label{tab.results2}
\end{table}






\newpage


%%\bibliography{EDmanuscript-Appendix.bbl}


\end{spacing}

\end{document}






