\chapter{SCR for Stratafied Populations: 
Multi-sesssion and Multi-site Data}
\markboth{Stratified Population Models}{}
\label{chapt.hscr}

\vspace{0.3cm}



The basic idea of this chapter has to do with aggregating data, using
hierarchical models, from multiple study sessions or trapping arrays,
or what we call stratified populations.  In secr, these are referred
to as ``multi-session'' models. 
A number of distinct situations arise:
(1) Multiple sessions:
It is commong to have samples occur over short intervals but then
sampling is repeated again some weeks later, perhaps multiple times in
a year.  Data are collected like a Robust design.
(2) This could also be one session each year and we could view years
as sessions;
(3) We have multiple sample arrays spread out over the landscape.
(4) A mix of those things including possibly all 3 of them.

For multiple years we imagine a fully dynamic -- demographically open
-- model that involves survival and recruitment. We deal with those
models specifically in Chapt. \ref{chapt.js}.
We deal here with a more primative type of model in which the
populations are assumed to be independnet. 

We do this as follows: We have $g=1,2\ldots, G$ groups -- grids or
sessions or grid*sessions, and let $N_{g}$ be the population size for
each group. Then we build models for $N_{g}$. For example, we might
assume that population size is constant across sessions or groups:
\[
 N_{g} \sim Poisson(\lambda)
\]
or we might have measureable covariates that affect N
\[
 \lambda_{g} = \beta_{0} + \beta_{1} z_{g}
\]
A typicaly example is...........

This kind of problem is really common in small mammal trapping stuff
\citep{converse_royle:xxxx} MORE HERE XXXXXXX.
Because of the idea of building independent models for $Ng$ and also
for the encounter process, these were called 
{\it hierarchical capture-recapture} models by \citep{royle_etal:2013}
and then hierarchical SCR models by \citep{converse_royle:2013}. 

Conceptually we can apply models like this which assume Ng are
independent even if they're not... as long as we dont cear about the
underlying dynamics explictly and also possibly with some loss of
eficiency. 

GROUPS, STRATA, POPULATIONS, ETC...????

\section{Model formulation}

Here we note that Ng conditional on total has a multinomial
distribution. We can implement that with a bunch of categorical
variables which are equivalent to multinomial trials.  The
cateogorical is this:
\[
formula
\]
So this categorical model on a ``stratum membership''

Bernoulli model...................

\subsection{Simulating structured data}

So how to simulate data?


\subsection{Other issues}


\section{Multi-sessions}

The case here is we have $g=1,\ldots,G$ samples over time but
individuals are coming and going.
We might capture some individuals over time but we ignore the
individual recaptures across primary periods. (See chapter
\ref{chapt.js}). So instead of modeling the dynamics at the individual
level we just model net change in $N_{g}$.


\section{What does \mbox{\tt secr} do?}



\section{Multi-sites}


stuff from manuscripts





