


\chapter{Spatial Capture-Recapture for Unmarked Populations}
\markboth{Chapter 14 }{}
\label{chapt.scr-unmarked}

\vspace{0.3cm}


Traditional capture-recapture models share the fundamental
assumption that each individual in a population can be uniquely
identified when captured. This can often be accomplished
by marking individuals with color bands, ear tags, or some other
artifical mark that can be read in the field. For other species, such as
tigers or marbled salamanders, individuals can be easily identified
using only their natural markings. In a great number of cases, however,
species do not possess sufficient natural markings and are too
difficult to capture to make it practical to apply artifical marks. So
we must throw up our hands and not study these species. End of
chapter.

When capture-recapture methods are not a viable option, researchers
often collect simple count data or even detection/non-detection data
to estimate population parameters. These data are often analyzed using
Poisson regression or logistic regression, perhaps with random
effects; but when detection is imperfect, as it almost always is,
these methods cannot be used to obtain unbiased estimates of
population size or occurrence probability. Even when these data are
used an index of abundance or occurrence, standard models may yield
unreliable results when covariates affect both the state variable and
detection probability. A classic example is the finding by
\citet{bibby_buckland:1987} who reported that the probability of detecting
songbirds in restocked confier plantations decreased with vegetation
height; whereas population density was positively related to
vegetation height. This intuitive and common phenomenon has led to the
development a vast number of methods to model population size or
density while controlling for factors affecting detection
probability. A review of these models is beyond the scope of this
chapter, but we mention a few deficiencies of existing methods
that warrant the exploration of alternatives.

Distance sampling is perhaps the most widely used method for
estimating population density when individuals are unmarked and
detection probability is less than one. DESCRIBE. This class of methods is known
to work impecibly when estimating the number of stakes in a field or
the number of duck nests in a wetland. In many other situations,
factors such as animal movement and measurement error may result in
substantial bias. In addition, traditional distance sampling methods
assume that individuals are randomly located with respect to the
observer and are available for detection (but see XXXX). Most other
methods, such as double-observer sampling and repeated counts, can be
used to estimate population size, but as with traditional CR methods,
it may be difficult or impossible to covert abundance estimates to
density estimates because the effective area sampled is unknown. We
mention these issues not to suggest that they do not have value, and
indeed we believe that can be used to obtain reliable density
estimates in many situations, but rather to highlight the need for
alternative methods when the assumptions of existing methods cannot be
met.

In this chapter we expand on t






