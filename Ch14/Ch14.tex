


\chapter{Spatial Capture-Recapture for Unmarked Populations}
\markboth{Chapter 14 }{}
\label{chapt.scr-unmarked}

\vspace{0.3cm}


Traditional capture-recapture models share the fundamental
assumption that each individual in a population can be uniquely
identified when captured. This can often be accomplished
by marking individuals with color bands, ear tags, or some other
artifical mark that can be read in the field. For other species, such as
tigers or marbled salamanders, individuals can be easily identified
using only their natural markings. In a great number of cases, however,
species do not possess sufficient natural markings and are too
difficult to capture to make it practical to apply artifical marks. So
we must throw up our hands and not study these species. End of
chapter.

When capture-recapture methods are not a viable option, researchers
often collect simple count data or even detection/non-detection data
to estimate population parameters. These data are often analyzed using
Poisson regression or logistic regression, perhaps with random
effects; but when detection is imperfect, as it almost always is,
these methods cannot be used to obtain unbiased estimates of
population size or occurrence probability. Even when these data are
used an index of abundance or occurrence, standard models may yield
unreliable results when covariates affect both the state variable and
detection probability. A classic example is the finding by
\citet{bibby_buckland:1987} who reported that the probability of detecting
songbirds in restocked confier plantations decreased with vegetation
height; whereas population density was positively related to
vegetation height. This intuitive and common phenomenon has led to the
development a vast number of methods to model population size or
density while controlling for factors affecting detection
probability. A review of these models is beyond the scope of this
chapter, but we mention a few deficiencies of existing methods
that warrant the exploration of alternatives.

Distance sampling, which we briefly introduced in chapter XXXX,
is perhaps the most widely used method for
estimating population density when individuals are unmarked and
detection probability is less than one. This class of methods is known
to work impecibly when estimating the number of stakes in a field or
the number of duck nests in a wetland. It can also work very well in
more interesting situations; however, %In many other situations,
common issues such as animal movement and measurement error may result in
substantial bias. In addition, traditional distance sampling methods
assume that individuals are randomly located with respect to the
observer and are available for detection (but see
\citet{johnson_etal:2010,chandler_etal:2011}). % Add ISSJ paper too
Most other
methods, such as double-observer sampling and repeated counts, can be
used to estimate population size, but as with traditional CR methods,
it may be difficult to covert abundance estimates to
density estimates because the effective area sampled is unknown. We
mention these issues not to suggest that existing models do not have
value---
indeed we believe that they can be used to obtain reliable density
estimates in many situations---rather our aim to highlight the need for
alternative methods when the assumptions of existing methods cannot be
met. Additionally, the model we develop in this chapter serves as the
foundation for a broad class of SCR models in which all or some of the
individuals cannot be uniquely identified.

In this chapter we highlight the work of \citet{chandler_royle:2012}
who demonstrated that the individual recognition assumption of
CR models is not a requirement of spatial capture-recapture
models. The ability to fit
SCR models to data from unmarked populations has important
consequences in several respects. For one, it means that SCR models can
be applied to data collected using methods like points counts in which
observers record simple counts of animals at an array of survey
points. This development also has important implications for
traditional SCR studies because many resulting datasets include some
individuals that cannnot be identified due to poor photo quality or
the indistiguishable natural markings.


In order to apply SCR models to data collected unmarked animals, one
requirement is critial---counts must be spatially correlated. Of
course, this condition holds true in virtually all SCR models since
animals are often detected at more than one trap. In fact, efficient
SCR designs should try to ensure correlation in counts among
neighboring traps because this is the primary source of information
about the encounter rate parameter, $\sigma$.

%Although ``unmarked SCR'' models represent an important development,
%we will see that they are not without their limitations. For example,
%when all individuals are unmarked, we can expect extreme sensitivity
%to the priors as well as skewed posterior distributions in typical
%sample sizes. Nonetheless,






\section{Encounter Histories as Latent Variables}

Just when you thought we ran out of things to treat as latent
variables, we are now going to regard even the data itself as latent.


\section{Data Requirements and Survey Designs}


\section{How Much Correlation Is Enough?}


\section{Northern Parula Example}



\section{On (Im)precision}




\section{Mutants}

\subsection{Other observation models}

\subsection{Linear designs}




\section{Summary}

