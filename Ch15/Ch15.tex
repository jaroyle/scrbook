\chapter{
Spatial capture-recapture models for partially identifiable populations/spatial mark-resight models
}
\markboth{Spatial mark-resight models}{}
\label{chapt.partialID}

\vspace{.3in}


So far, Chapters \ref{chapt.intro} �- \ref{chapt.XX} dealt with the situation where all detected individuals are identifiable, and in Chapter \ref{chapt.scr-unmarked} we introduced and developed an SCR model for non-identifiable populations, a spatial non-capture-recapture model, if you will. These two extremes are common in the study of large mammals and/or birds with non-invasive sampling methods (camera trapping of individually spotted or striped species, genetic sampling, bird point counts). However, there is also an intermediate situation, where a part of the population is tagged and can thus be identified upon recapture, while the untagged portion remains unidentified. In this situation so-called mark-resight models (REF) can be used to estimate population size and density combining data from both the marked and unmarked individuals. 

\subsection{Overview of mark-resight models}
Mark-resight models have been around for a while and are incorporated in standard capture-recapture software such as MARK. Traditionally, capture-recapture studies involved physical capture of individuals throughout the study; new individuals are marked on every re-capture occasion. In mark-resight, a sample of individuals is captured and tagged prior to the resighting surveys during a single marking event. As such, mark-resight models have a major advantage over traditional capture-recapture models in that they only require individuals to be captured and handled once, during marking. The resighting portion of the study can then be implemented using a non-invasive technique (hence the name �resighting�). This reduces field costs and risks for the animals (and potentially the researchers).
Mark-resight models have a set of underlying assumptions: 
(1) a random, representative sample of the study population is tagged, so that inference about individual detection can be made for the whole population from the tagged sample. This issue is usually addressed by using a different method for marking than for resighting. For example, a number of individuals could be physically tagged before a camera trapping study (e.g. bear paper). 
(2)  tags are always correctly reported, � one of the fundamental assumptions of any capture-mark-recapture technique (REF), which also entails that marks are not lost during resight surveys. 
{\bf FINISH!!!}

\subsection{Types of partial ID data}
Before we start exploring mark-resight models themselves in more detail, we need a clear understanding of what types of partial ID data we can have. 
In general, we have (at least) two sets of data: encounter histories for identifiable individuals $i$ at trap $j$ and occasion$k$, $z_{ijk}$, and counts of unidentified records for each $j$ and $k$, $y_{jk}$. Depending on the sampling technique, we can conceive of three slightly different types of partial ID data. 
Imagine tags can always be identified to individual level when a tagged animal is resighted. Under these circumstances, your $y_{jk}$ will only contain records (sightings, pictures,�) of untagged � unknown � individuals and all records of your tagged � known � individuals will be correctly attributed to the right capture histories in {\bf $Z$}. You may fail to resight some of the tagged individuals, but since you know how many there are, you can simply assign them all-0 capture histories
A slight variation of this data type arises when in some instances we can only tell that an individual is tagged, but not who it is. You may be able to see an individual is tagged but the identifying number on the tag may have become unreadable, or may be hidden from view. In this case in addition to $y_{jk}$ and $z_{ijk}$, you have some count of records per sampling device of marked but unidentified individuals, say $x_{jk}$. 

If we suspect that some of the marks may have been lost between tagging and conducting the resighting samples, we obtain a slightly different type of mark-resight data. Here, we do not know the accurate number of marked individuals available for resighting. As a consequence, individuals have to be resighted at least once for us to know they are still tagged and available for sampling. In other words, contrary to the situation where we know how many marked individuals are out there, we cannot observe an all-0 encounter history of a marked individual. 
(A special case of this kind of data can arise from camera trapping. Even when dealing with a species that has no spots or stripes, some individuals in the study population can have natural marks that make them identifiable on pictures, such as scars or some distinct coloration. Again, in this scenario an individual has to be photographed at least once to be known. In case of camera-trapping, the fact that both the �marking� method and the subsequent resighting method are the same (although marking in this case does not involve any actual physical marking) can be cause for concern: our sample of �marked� individuals may not be a random sample of the population but consist of individuals that for some reason are more likely to be photographed.)    

Finally, consider a scat or hair snare survey, where only a part of the samples are analyzed genetically. In this scenario, your $y_{jk}$ can contain both completely unknown individuals that are not represented at all in {\bf $Z$}, but it can also contain samples from individuals that we previously identified. The difference is that in the first two scenarios, part of the population of individuals is identifiable, while in the second scenario, part of the �population� of samples is identifiable. This type of data actually violates one of the basic assumptions of mark-resight models, namely, that tags are always reported correctly. For now, we will ignore this kind of data as well as the special case of camera trapping where some individuals can be identified based on natural marks and focus on the two types of typical mark-resight data:  

\begin{itemize}
\item[(1)]	Known number of tagged individuals 
\item[(2)] Unknown number of tagged individuals, 
\end{itemize}

with the two variations of tagged individuals always correctly being identified to individual level, where tagged individuals are not always identified to individual level (but whether an individual is tagged or not is always correctly recorded).

\section {A brief history of mark-resight models}
Initially, mark-resight methods focused on radio-tagged individuals to estimate population size (White and Shenk 2001). Radio-collars provide a means of determining which of the animals were in the study area and available for sampling, i.e. determining the number of marked individuals in the population. Knowing this number was a prerequisite for earlier mark-resight approaches. The oldest mark-resight model is the good old Lincoln-Petersen estimator, where individuals are marked and a single resight/recapture occasion is carried out (Krebs 1989?). We need not identify individuals, but only tell apart marked from unmarked individuals. Let $m_1$ be the number of marked individuals in the population, $m_2$ the number of marked individuals seen on the resighting occasion, and $n_2$ the total number of marked and unmarked individuals observed during resighting.  Abundance $N$ is then estimated as 
\[
N = m_1 * n_2 / m_2
\]

A suite of more elaborate models using individual capture histories over several resighting occasions were developed in the 1980ies and 90ies and compiled into the program NOREMARK (White 1996). Apart from a model for the basic case with known number of marked individuals and no individual variation in resighting probabilities (joint hypergeometric maximum likelihood estimator; Bartmann et al. 1987, White and Garrott 1990, Neal 1990, Neal et al. 1993), NOREMARK contains models that account for lack of geographical population closure (Neal et al. 1993), individual heterogenenity in resighting rates and sampling with replacement (i.e. individuals can be seen on any occasion more than once, Minta and Mangel 1989, Bowden, 1993). The first mark-resight model for an unknown number of marked individuals was developed by Arnason et al. (1991), but their approach has several limitations. More recently, more reliable and flexible likelihood-based mark-resight models have been developed and incorporated into the program MARK (McClintock and White 2010). These models can account for individual heterogeneity in detection, unknown number of marked individuals and lack of geographical closure, as well as a less than 100\% individual identification rate of tagged individuals, can be applied to sampling with and without replacement and can combine data across several primary sampling occasions in a robust design type of analysis (McClintock et al. 2009a,b, McClintock and White 2009).  
To our knowledge there are currently no models available that account for possible misidentification of individuals (as would be the case in the �partially identifiable population of samples� scenario described above).
In this chapter we will consider the situation of partially identifiable populations in the framework of spatial capture-recapture. The underlying concepts and issues are the same and we will consider two different extensions to the SCR models to analyze spatial mark-resighting data.


\section {Spatial mark-resight model � known number of marked individuals}
Let�s begin with the traditional data situation: a known number of individuals representing a random sample from the population are marked and a series of resight samples are conducted following marking. All individuals are correctly identified as marked or unmarked, and marked individuals are 100 \% correctly identified to individual level. 
Under these conditions, the information obtained from the individually recognizable animals can easily be included into the spatial non-capture-recapture model we presented in Chapt. \ref{chapt.scr-unmarked}. Recall that without individual identity, the observed counts at trap $j$ and occasion $k$, $n_{jk}$, represent the sum of all latent individual detections at $j$ and $k$, $\displaystyle\sum\limits_{i=1}^{N}  z_{ijk}$, where $z_{ijk}$ are the individual encounter histories. We can model photo counts as

\[
n_{jk} \sim \mbox{Poisson}( \Lambda_{j} )
\]


Under this formulation we do not need to update the individual $z_{ijk}$ in our model. However, we can also formulate it as conditional on the �- latent -� $z_{ijk}$. This is useful because if we have $n$ individually known animals in our study population, than those $n$ $z_{ijk}$ are no longer latent, but fully observed and can easily be included in the analysis. 
The formulation conditional on $z_{ijk}$ basically brings us back to the original SCR model, where individual site and occasion specific counts, $z_{ijk}$, are modeled as
\[
z_{ijk} \sim \mbox{Poisson}(\lambda _{ij})
\] 
and
\[
\lambda _{ij} = \lambda_0 * exp(-D_{ij}^2/(2 \sigma^2))
\]

Unobserved $z_{ijk}$ are updated using their full conditional distribution, which is multinomial with sample size $y_{jk}$.  
While in the non-spatial mark-resight analysis known individuals provide direct information about individual detection probability (or rate), in the spatial setting they also inform the movement parameter $\sigma$. Including known individuals into the analysis help estimate model parameters more accurately and precisely, especially $N$, which is usually the objective of inference. We will address the relationship between the number of marked individuals and accuracy of the estimated parameters in section \ref{sect.XX}. 

\subsection{MCMC for the spatial mark-resight model with known number of marked individuals}
Just as for the model without individual identification, for the partial ID model, knowing how to write your own MCMC algorithm comes in extremely handy. You will find that we only have to make relatively simple modifications to the MCMC code for the model without any individual identification presented in Chapt. \ref{chapt.scr-unmarked}, which, in turn, has much in common with the algorithms we developed for regular SCR models in Chapt. \ref{chapt.mcmc}. You can find the full MCMC code in the accompanying {\bf R} package scrbook (mcmcZdatSim.txt). 

\begin{comment}
Essentially, we have to modify the updating step for Z, w and psi to reflect our knowledge of these parameters for the n tagged individuals. Before we filled the array holding the latent individual encounter histories using (at site r and occasion t)
            Z[,r,t] <- rmultinom(1, y[r,t], lam[,r]*w)

Now, we have to fill the first n rows of the array with the n observed individual encounter histories, then update Z for the unknown individuals only:
\begin{verbatim}
# set up placeholders    
Z <- array(NA, c(M,R,T))
    nMarked <- 0
    marked <- rep(FALSE, M)
#Zknown holds the individual encounter histories
    if(!missing(Zknown) || !is.null(Zknown)) {
        nMarked <- nrow(Zknown)
        marked[1:nMarked] <- TRUE
        Z[1:nMarked,,] <- Zknown
    }
#set w = 1 for known individuals
    w[marked] <- 1
    Zdata <- !is.na(Z)
    
#the following lines apply to both setting up the Z array for the #first time and updating it
for(r in 1:R) {
        for(t in 1:T) {
            if(y[r,t]==0) {
                Z[,r,t]  <- 0
                next
            }
#fill Z for unknown individuals only
            unmarked <- !Zdata[,r,t]
            nUnknown <- y[r,t] - sum(Z[!unmarked,r,t])
            if(nUnknown < 0)
                browser()
            probs <- lam[,r]*w
            probs <- probs[unmarked]
            Z[unmarked,r,t] <- rmultinom(1, nUnknown, probs)
\end{verbatim}

Next, let�s look at w. When updating w, we do not need to update the known individuals, since we know they are part of our population. Thus, the updater for w becomes
\begin{verbatim}
      wUps <- 0
      seen <- apply(Z>0, 1, any)

      for(i in 1:M) {
#skip update for individuals marked or seen
if(marked[i]|seen[i])
next
          wcand <- ifelse(w[i]==0, 1, 0)
		ll.w <- sum(dpois(Z[i,], lam[i,]*w[i], log=TRUE) )
          llcand <- sum(dpois(Z[i,], lam[i,]*wcand, log=TRUE) )
          prior <- dbinom(w[i], 1, psi, log=TRUE)
          prior.cand <- dbinom(wcand, 1, psi, log=TRUE)
          if(runif(1) < exp((llcand+prior.cand) - (ll.w+prior))) {
              w <- wcand
              ll <- llcand
              wUps <- wUps+1
          }
      }
\end{verbatim}

Finally, under this formulation our known individuals are no longer part of the �population� of w�s that are updated conditional on psi. Since we can observe all-0 encounter histories for marked individuals, we do not need to estimate how many of them there are. We only need to estimate the w�s for M-n unknown individuals (while we skip the update of w for the �seen� individuals, seen is defined based on Z and Z is updated at each iteration, so the w for the �seen� but unmarked individuals are still updated) and this needs to be reflected in our update for psi. In the full conditional Beta distribution we have to replace M with (M-n) and sum(w) with sum(w)-n:

\begin{verbatim}
  psi<-rbeta(1,1+sum(w[!marked]),1+sum(!marked)-sum(w[!marked]))   
\end{verbatim}

The last subtle change concerns the acceptance rate of w. Again, we no longer divide sUps by M, but by M-n. The remainder of the code is identical to the MCMC code for the model without individual identity.  
\end{comment}
%%% I don�t think we need these details since this is not the MCMC chapter any more. Maybe just some well commented code in the R package

\subsection{Example � Canada geese in North Carolina} 
We applied this spatial mark-resight model to a data set of Canada geese collected by XXX from North Carolina State University during her PhD dissertation (Rutledge 2012). 750 individual geese were captured and tagged with neck and leg bands in XXXX, North Carolina. Geese were resighted over a period of almost two years (during 81 resightin events) at 87 different locations. In addition to the banded geese, the number of unmarked geese was also recorded at each resighting event. Here, we only looked at a subset of the data, from July to October 2008, the first part of the post molt season, when Canada geese start to move around again after molting (WHAT IS IT THAT DEFINES THIS SEASON?).  
In this time block XX of the 750 marked geese were resighted XXX times, and XXX sightings of unmarked geese were recorded. Since an individual could only be observed once on a given occasion at a given site, we adopted a binomial encounter model to model detection. So far, we have only worked with Poisson encounter models for partially identifiable or unmarked populations; when we use the binomial model we have to make a slight modification to how we update the latent $z_{ijk}$, to ensure that a hypothetical individual receives at most a single observation at a given trap and occasion from the pool of $y_{jk}$ pictures. Effectively, we move from a multinomial situation to a sampling without replacement situation (an individual drawn once at j and k cannot be drawn again); here is how we implement this in our MCMC algorithm:

\begin{verbatim}
####updating latent Z's

    for(r in 1:R) {
        for(t in 1:T) {
            if(y[r,t]==0) {
                Z[,r,t] <- 0
                next
            }
            unmarked <- !Zdata[,r,t]
            nUnknown <- y[r,t] - sum(Z[!unmarked,r,t]) 
            if(nUnknown < 0)
                browser()
            probs <- lam[,r]*w *EffAr[,r,t]
            probs <- probs[unmarked]
	guyIDs <- sample((nMarked+1):M, nUnknown, replace=FALSE, prob=probs) 		
	noGuys<-(1:M)[-c(1:nMarked,guyIDs)]
            Z[guyIDs,r,t] <- 1
		Z[noGuys,r,t] <- 0
        }
    }
\begin{verbatim}

In the case of the Canada geese, we also allowed $\sigma$ to vary between males and females. This only requires minor changes to the MCMC sampler and you can check these out by calling the XXX function from the scrbook package.
We augmented the data set with 4500 - $n$ all-zero encounter histories and ran 100000 iterations -- we provide all the data and functions for you to repeat this analysis but be aware that given the large data set this will take you anywhere between 5 and 7 days -- and removed a burn-in of XXX iterations. The model results are shown in Tab. \ref{XXX}. We see that BLABLABLA, discuss results.  

\subsection  {Individual identification rate of tagged animals <100 \%}
Often during resighting, it may be possible to see that an individual is tagged but impossible to determine the individual identity of the tag. As discussed above, in such a situation we have three sets of data: Individual site-specific encounter histories of marked and identified individuals, site specific counts of unmarked individuals and site specific counts of marked but not identified individuals. Here, the individual encounter histories are essentially incomplete, and if we used these incomplete data to inform the detection parameter of the model, we are likely to underestimate detection/trap encounter rate and overestimate abundance.
McClintock et al. (2009) suggest an intuitive means of correcting for this bias in a non-spatial model framework: In their model formulation the expected number of records of unmarked individuals, $E(n_u)$ is

{\bf EXPAND, THIS MAKES NO SENSE STANDING ALON}

\[
E(n_u) = (N--n) { lam , eta/n}
\]
where $N$ is the total population size, $n$ is the known number of marked individuals, $\lambda$ is the individual encounter rate estimated from the known resighted individuals and $\eta$ is the number of records of marked but unidentified individuals. So essentially, because $\lambda$ is known to be too low, the average number of unidentified pictures per known individual is added as a correction factor. This procedure assumes that the inability to identify a marked individual occurs at random throughout the population, which seems to be a reasonable assumption under most circumstances.

We can relatively easily translate this concept to our spatial mark-resight models.  In the spatial model framework we are interested in the individual and trap specific encounter rate, $lambda_{ij}$. Further, we do not look at the sum of all records of unmarked individuals, but formulate the model conditional on the latent individual encounter histories. Thus, instead of using the number of unidentified pictures per marked individual as a correction factor, we need something that applies at the individual and trap level. If we take the sum of all correctly identified records of marked individuals and divide it by the total number of records of marked individuals, we get the average rate of correct individual identification for marked individuals, say, $c$. We can then multiply $\lambda_0$ times $c$ in the model for the marked individuals, to account for the fact that we observe incomplete individual encounter histories. For example, if on average we are able to assign 80\% of the records of marked to the individual, we would multiply $\lambda_0 * 0.8$ in the part of the model for the marked individuals. The model for unmarked individuals remains as before. Observe that now, in addition to assuming that failure to identify tagged individuals occurs at random throughout the population, we also assume that it occurs at random throughout space, i.e. our success of identifying a tagged individual does not depend on the trap we encounter it in. 

{\bf Example ?}

\section {Spatial mark-resight models for unknown numbers of marked individuals}
\subsection{Lincoln-Petersen extension of the SCR model }
Now let us consider the case where we do not know the exact number of tagged individuals available for resighting. As mentioned before, in this case we have to capture an individual at least once to be sure that it is available. As a consequence, we cannot observe all-0 encounter histories for the marked individuals. In a non-spatial setting, this situation requires a model where detection rates of known individuals are modeled as coming from a zero-truncated distribution. If we did not account for the fact that 0�s are unobservable, our estimates of detection rates would be artificially inflated and estimates of population size would be negatively biased. 
Under these circumstances there is a relatively simple way to implement a mark-resight model in a, say, partially spatial framework by using a Lincoln-Petersen extension of the regular SCR model. 

{\bf Maybe raccoons to see if we can estimate number of tags?}
%%mention analogy with McClintock and Hoeting paper with 0-inflation and logit-normal mixture

\subsection{Fully spatial formulation }
We can also develop a fully spatial model for the case where the number of marked individuals is unknown, making use of the spatial autocorrelation in the counts of the unmarked individuals. In the previous example, where we knew the number of marked individuals, we essentially removed the marked individuals from the augmented population by fixing the indicator variable of whether an individual is part of the population or not (z? w?) at 1, and have psi refer only to the unmarked population, M-nmarked. All we have to do in the spatial mark-resight model with unknown number of marked individuals is to let our marked individuals be part of the augmented population again. 
{\bf maybe geese again, ignoring that we know number of tags? Other example data set?}

\section{How many marks do we need and how much information do unmarked guys contribute?}
Richard�s simulation study

\section{ SCR models for a partially identifiable �population of samples�}
If we can figure that out�

\section{summary}
