


\chapter{Spatial Capture-Recapture with Distance Sampling Data}
\markboth{Chapter 17}{}
\label{chapt.scrds}

\vspace{0.3cm}



In SCR models, the locations of animal activity
centers are unknown and must be inferred from the trap locations where
individuals are captured. Intuitively, the more
we know about the locations of activity centers, the more precise will
be our density estimates, and thus we strive to increase the
number of spatial recaptures. That is, we want to catch each
individual at multiple points in space so that we can pinpoint
its activity center. However, obtaining a large
number of spatial recaptures can be difficult due to the associated costs
of traps and the labor required to set and check them. This is true
even of ``cheap'' methods like camera traps which can easily run
$>$200\$ a pop.

A much easier way to record
animal locations in space is to go traverse a transect or stand at some
point and directly record the coordinates of the animal at some
instant in time. This is the approach taken in distance sampling,
although typically only the distance to the animal is
recorded. Since distance sampling is so popular and relatively cheap
compared to capture-recapture methods, it is natural to wonder how
distance data could be included in a SCR analysis. Before doing so, a better
question is why bother with SCR when a distance sampling analysis would
be straight-forward. In some cases, there probably is no
need to use SCR if the sole quantity of interest is density, and the
assumptions of distance sampling can be met. However, SCR let's us
study more than just density. Think of space use.

Use of distance data
forces us to consider movement a little more explictily than we had
previously. However, it has been argued that it is better to avoid
explicit movement models in the context of SCR
\citep{borchers:2010}. We agree that this can be
convenient and appropriate in many contexts, but movement is an
inherent part of the problem, and interesting in
its own right as shown by the rapidly growing literature on the
subject.

Currently, we know of only two papers that have attempted
to estimate explicit movement parameters
\citep{royle_young:2008,royle_etal:2009jae}. The underlying approach
is as follows. First, define $\bf u$ to be an individual's location in
space at some instant in time. We now need a movement model that links
the activity center $\bf s$ to $\bf u$ and finally a detection model that
is a function of $\bf \| u - x\|$ instead of $\bf \|s - x\|$,
\emph{i.e.}, a function of distance between the observer
at point $\bf x$ and the animal at point $\bf u$---just as in distance
sampling. A natural movement model is the bivariate normal, but we
will consider other options in this chapter. In addition, we will
consider pragmatic issues such as what to do if not all individuals
are marked.


\section{Model Development}

Although we refer to the models in this chapter as spatial
capture-recapture distance-sampling (SCRDS) models, we will actually
make use of more than just the distance between observer and
animal. What we really want is the exact location of the animal $\bf
u$ at some point in time. This data is often collected even in
standard distance sampling situations, such as when an observer
records the distance and bearing to an individual while walking a line
transect. Recently developed distance sampling methods even require
exact locations of individuals so that inhomogeneous point process
models can be fitted \citep{johnson_etal:2010}. Thus there is nothing
too unconventional about collecting location data when sampling using
point-counts or a line-transects.






\section{Everybody is Marked}



\begin{verbatim}

model {
sigHome ~ dunif(0, 5)
sigObs ~ dunif(0, 5)
tauHome <- 1/pow(sigHome,2)
tauObs <- 1/pow(sigObs,2)
psi ~ dunif(0, 1)
for(i in 1:M) {
  w[i] ~ dbern(psi)
  sx[i] ~ dunif(0, 15)
  sy[i] ~ dunif(0, 15)
  for(r in 1:R) {
    ux[i,r] ~ dnorm(sx[i], tauHome)
    uu[i,r] ~ dnorm(sy[i], tauHome)
    d2[i,r] <- pow(X[r,1]-ux[i,r], 2) + pow(X[r,2]-uy[i,r], 2)
    p[i,r] <- exp(-d2[i,r]/(2*sigObs*sigObs)) * w[i]
    y[i,r] ~ dbern(p[i,r])
    }
  }
N <- sum(w[])
}

\end{verbatim}







\section{Nobody is marked}





\section{Partially-marked populations}




\section{What if We Only Have Distance Data, not Exact Coordiantes?}


POSSIBLE ANALYSIS OF NOPA DATA





\section{An Implicit SCRDS Model Without Distance Data}

The idea here is to convolve two Gaussian kernels, one for the animal
(movement) and one for the observer (detection | distance). This is
just another two-parameter observation model I guess.



\section{Summary}

What have we gained by using distance data? First, we can apply SCR
models to a wider range of sampling methods and existing
datasets. Second, we can model space use and home range size, not just
density. Third, we made our model more explicit by directly addressing
movement, rather than balling it up with the detection process.