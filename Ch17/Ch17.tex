


\chapter{Spatial Capture-Recapture with Distance Sampling Data}
\markboth{Chapter 17}{}
\label{chapt.scrds}

\vspace{0.3cm}



In SCR models, the locations of animal activity
centers are unknown and must be inferred from the trap locations where
individuals are captured. Intuitively, the more
we know about the locations of activity centers, the more precise will
be our density estimates, and thus we strive to increase the
number of spatial recaptures. That is, we want to catch each
individual at multiple points in space so that we can pinpoint
its activity center. However, obtaining a large
number of spatial recaptures can be difficult due to the associated costs
of traps and the labor required to set and check them. This is true
even of ``cheap'' methods like camera traps which can easily run $>$
200\$ a pop.

Distance-samplers, of course, know that a much easier way to record an
animal's location in space is to go traverse a transect or stand at some
point and directly record the coordinates of the animal at some
instant in time\footnote{Generally only the distance between the
  observer and the animal is used in the analysis, but the exact
  coordinates are often recorded.}. \hl{Hmm, this make you wonder
  about telemetry data too}. This is cheap and easy data, which
partially explains the popularity of distance sampling.


Given the ease
with which one can record distance data, it is natural to wonder how
it could be included in a SCR analysis. Before doing so, a better
question is why bother with SCR when a distance sampling analysis would
be straight-forward. Good point. In some cases, there probably is no
need to use SCR if the sole quantity of interest is density, and the
assumptions of distance sampling can be met. However, SCR let's us
study more than just density. Think of space use. Think up a ship,
think up a long trip, think up the Vipper, the Vipper of Biff. Oh the
thinks you can think up if only you try (Seuss 1960s).

Some regard movement as a nuisance that is
best left untouched \citep{borchers:2010}. This is
convenient, but movement is actually a part of the problem, not to
mention a central focus of an enormous
branch of ecology.

Use of data on animal locations at some distance from the observer
forces us to consider movement a little more explictily than we had
previously. Currently, only two papers that we know of have attempted
to estimate explicity movement parameters
\citep{royle_young:2008,royle_etal:2009jae}. The underlying approach
is simple. First, define $\bf u$ to be an individual's location in
space at some instant in time. We now need a movement model that links
the activity center $\bf s$ to $\bf u$ and finally a detection model that
is a function of $\bf \| u - x\|$ instead of $\bf \|s - x\|$,
\emph{i.e.}, a function of distance between the observer
at point $\bf x$ and the animal at point $\bf u$---just as in distance
sampling. A natural movement model is the bivariate normal, but we
will consider other options in this chapter. In addition, we will
consider pragmatic issues such as what to do if not all individuals
are marked.







\section{Everybody is marked}



\begin{verbatim}

model {
sigHome ~ dunif(0, 5)
sigObs ~ dunif(0, 5)
tauHome <- 1/pow(sigHome,2)
tauObs <- 1/pow(sigObs,2)
psi ~ dunif(0, 1)
for(i in 1:M) {
  w[i] ~ dbern(psi)
  sx[i] ~ dunif(0, 15)
  sy[i] ~ dunif(0, 15)
  for(r in 1:R) {
    ux[i,r] ~ dnorm(sx[i], tauHome)
    uu[i,r] ~ dnorm(sy[i], tauHome)
    d2[i,r] <- pow(X[r,1]-ux[i,r], 2) + pow(X[r,2]-uy[i,r], 2)
    p[i,r] <- exp(-d2[i,r]/(2*sigObs*sigObs)) * w[i]
    y[i,r] ~ dbern(p[i,r])
    }
  }
N <- sum(w[])
}

\end{verbatim}







\section{Nobody is marked}





\section{Partially-marked populations}





\section{An Implicit SCRDS Model Without Distance Data}

The idea here is to convolve two Gaussian kernels, one for the animal
(movement) and one for the observer (detection | distance). This is
just another two-parameter observation model I guess.
