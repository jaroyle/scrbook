\documentclass[12pt]{article}

\usepackage[pdftex]{hyperref}
\usepackage{parskip}
\usepackage{amsmath}

\begin{document}

A (possibly bad) idea for modeling data when the number of marks, $m$, is
unknown. This assumes that there are no all-zero encounter histories.

Let $v_j$ be the number of \textit{known} guys known to be marked on
occasion $j$. Let $q_{ij}$ be a binary indicator of whether or not individual $i$
($i=1,\ldots,M$) is marked. The reason why it has $J$ columns is that
the number of \textit{known} marked guys changes over time. So part of
$q_{ij}$ is observed data, but the observed part changes over time.
However, once a
guy is known to be marked, his $q$ is always 1 in subsecquent occasions.

Here is the model:

\begin{align*}
q_{ij} &\sim \text{Bern}(\omega_j) \\
m &= \sum_i I((\sum_j q_{ij})>0) \\
z_i &\sim \text{Bern}(\psi) \\
y_{ijk} &\sim \text{Pois}(\lambda_{ij}z_{i}) \\
n_{jk} &= \sum_i y_{ijk} \\
N &= \sum_i z_i
\end{align*}

With a Beta(1,1) prior, the full conditional of $\omega_j$ is:
\[
\Pr(\omega_j|\cdot) = \text{Beta}(1+\sum_i^{m-v_j} q_{ij}, 1+m-v_j-\sum_i^{m-v_j}q_{ij})
\]

With a Beta(1,1) prior, the full conditional of $\psi$ is:
\[
\Pr(\psi|\cdot) = \text{Beta}(1+\sum_i^{N-m} z_{i}, 1+N-m-\sum_i^{N-m}z_i)
\]



\end{document}
