\documentclass[12pt]{article}

\usepackage[pdftex]{hyperref}
\usepackage{parskip}
\usepackage{amsmath}

\begin{document}

A (possibly bad) idea for modeling data when the number of marks, $m$, is
unknown. %This assumes that there are no all-zero encounter histories.

\vspace{1cm}

Model:
\begin{align*}
  N \sim \text{DUnif}(0, M) \\
  m \sim \text{DUnif}(0, M) \\
\end{align*}


$E(m) = M\psi$ \\
$E(u) = M\omega$ \\
$E(N) = M - M\psi + M\omega$ \\
psi = 1 - (omega+psi)

Definitions: \\
$M$ = Data-augmentation size \\
$\psi = \Pr(z_i=1)$ \\ %for $i=1,\ldots,M$ \par
$\omega = \Pr(q_i=1)$ \\
$N = \sum_{i=1}^M z_i$ = Number of real individuals in $\mathcal{S}$ \\
$m = \sum_{i=1}^M q_i$ = unknown number of marked guys \\
$u = N - m$ = number of unmarked guys

\vspace{1cm}

The following contraint is important:
\[
M = m + u + fakers
\]


\newpage

Let $v_j$ be the number of \textit{known} guys known to be marked on
occasion $j$. Let $q_{ij}$ be a binary indicator of whether or not individual $i$
($i=1,\ldots,M$) is marked. The reason why it has $J$ columns is that
the number of \textit{known} marked guys changes over time. So part of
$q_{ij}$ is observed data, but the observed part changes over time.
However, once a
guy is known to be marked, his $q$ is always 1 in subsecquent occasions.

Here is the model:

\begin{align*}
q_{ij} &\sim \text{Bern}(\omega_j) \\
q_{i.} &= I((\sum_j q_{ij})>0) \quad \text{Was guy marked?}\\
m &= \sum_i q_{i.} \\
z_i &\sim \text{Bern}(\psi) \\
y_{ijk} &\sim
\text{Pois}(\lambda_{ij}z_{i}q_{ij} + \lambda_{ij}z_i(1-q_{ij})) \quad
\text{For marked guys, no like. contrib. until first obs}\\
n_{jk} &= \sum_i y_{ijk} \\
N &= \sum_i z_i
\end{align*}

With a Beta(1,1) prior, the full conditional of $\omega_j$ is:
\[
\Pr(\omega_j|\cdot) = \text{Beta}(1+\sum_i^{M-m-v_j} q_{ij}, 1+M-m-v_j-\sum_i^{M-m-v_j}q_{ij})
\]

With a Beta(1,1) prior, the full conditional of $\psi$ is:
\[
\Pr(\psi|\cdot) = \text{Beta}(1+\sum_i^{M-m} z_{i}, 1+M-m-\sum_i^{M-m}z_i)
\]

NOTE: Need to make sure that if $q_{ij}=1$ then all subsequent
$q_{i.}=1$.


\end{document}
