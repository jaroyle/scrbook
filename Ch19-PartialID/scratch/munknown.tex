\documentclass[12pt]{article}

\usepackage[pdftex]{hyperref}
\usepackage{parskip}
\usepackage{amsmath}
\usepackage{bm}

\begin{document}

Spatial mark-resight models with $m$ unknown.

\vspace{0.5cm}



Case 1: marked guys are sampled randomly in the polygon
$\mathcal{B}$.

The population in $\mathcal{S}$ is $N$. $N_1$ is
population in $\mathcal{B}$. The key idea is to recognize that if the
$N$ activity centers are uniformly distributed in $\mathcal{S}$, then
neither the marked or unmarked guys are not uniformly distributed in
$\mathcal{B}$.

The random sampling assumption leads to this model:
\begin{align*}
  {\bf s}_i &\sim \text{Unif}(\mathcal{S}) \\
  \{N_1, N_2\} &\sim \text{MN}(N, \{A1/A, A2/A\}) \\
  m &\sim \text{Bin}(N1, \theta) \\
\end{align*}


To implement it using DA, you can do this:
\begin{align*}
  {\bf s}_i &\sim \text{Unif}(\mathcal{S}) \\
  \phi_1 &= I({\bf s}_i \in \mathcal{B}) \theta \psi \\
  \phi_2 &= I({\bf s}_i \in \mathcal{B}) (1-\theta) \psi \\
  \phi_3 &= I({\bf s}_i \notin \mathcal{B}) \psi \\
  \phi_4 &= 1 - \psi \\
  \text{h}_i &\sim \text{Cat}(\bm{\phi}) \\
\end{align*}

Definitions: \\
$\psi$ = Pr(guy is real) \\
$\theta$ = Pr(guy in B is marked) \\
$h$ = indicator of a guy's state: (1)marked in B, (2) unmarked in B,
(3) unmarked outside B, and (4) fake guy


\newpage


Case 2: The density of marked guy's decreases smoothly with distance
from traps.


To implement it using DA, you can do this:
\begin{align*}
  {\bf s}_i &\sim \text{Unif}(\mathcal{S}) \\
  \phi_1 &= \theta \psi \\
  \phi_2 &= (1-\theta) \psi \\
  \phi_4 &= 1 - \psi \\
  \text{h}_i &\sim \text{Cat}(\bm{\phi}) \\
\end{align*}


Definitions: \\
$\theta = \exp(\tau d)$ = Pr(guy is marked, given his distance from
trap array centroid??) \\
$h$ = indicator of a guy's state: (1)marked, (2) unmarked, and (3)
fake guy



\end{document}
