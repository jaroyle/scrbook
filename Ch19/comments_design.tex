%  template.tex for Biometrics papers
%
%  This file provides a template for Biometrics authors.  Use this
%  template as the starting point for creating your manuscript document.
%  See the file biomsample.tex for an example of a full-blown manuscript.

%  ALWAYS USE THE referee OPTION WITH PAPERS SUBMITTED TO BIOMETRICS!!!
%  You can see what your paper would look like typeset by removing
%  the referee option.  Because the typeset version will be in two
%  columns, however, some of your equations may be too long. DO NOT
%  use the \longequation option discussed in the user guide!!!  This option
%  is reserved ONLY for equations that are impossible to split across 
%  multiple lines; e.g., a very wide matrix.  Instead, type your equations 
%  so that they stay in one column and are split across several lines, 
%  as are almost all equations in the journal.  Use a recent version of the
%  journal as a guide. 
%  
\documentclass[useAMS,referee]{biom}
%documentclass[useAMS]{biom}
%
%  If your system does not have the AMS fonts version 2.0 installed, then
%  remove the useAMS option.
%
%  useAMS allows you to obtain upright Greek characters.
%  e.g. \umu, \upi etc.  See the section on "Upright Greek characters" in
%  this guide for further information.
%
%  If you are using AMS 2.0 fonts, bold math letters/symbols are available
%  at a larger range of sizes for NFSS release 1 and 2 (using \boldmath or
%  preferably \bmath).
% 
%  Other options are described in the user guide. Here are a few:
% 
%  -  If you use Patrick Daly's natbib  to cross-reference your 
%     bibliography entries, use the usenatbib option
%
%  -  If you use \includegraphics (graphicx package) for importing graphics
%     into your figures, use the usegraphicx option
% 
%  If you wish to typeset the paper in Times font (if you do not have the
%  PostScript Type 1 Computer Modern fonts you will need to do this to get
%  smoother fonts in a PDF file) then uncomment the next line
%  \usepackage{Times}

%%%%% PLACE YOUR OWN MACROS HERE %%%%%
\usepackage{graphicx}
\usepackage{lineno}
\usepackage{float}
\usepackage{framed}
\def\bSig\mathbf{\Sigma}
\newcommand{\VS}{V\&S}
\newcommand{\tr}{\mbox{tr}}
\floatstyle{plain}
\floatname{panel}{Panel}
\newfloat{panel}{h}{txt}

%  The rotating package allows you to have tables displayed in landscape
%  mode.  The rotating package is NOT included in this distribution, but
%  can be obtained from the CTAN archive.  USE OF LANDSCAPE TABLES IS
%  STRONGLY DISCOURAGED -- create landscape tables only as a last resort if
%  you see no other way to display the information.  If you do do this,
%  then you need the following command.

%\usepackage[figuresright]{rotating}

%%%%%%%%%%%%%%%%%%%%%%%%%%%%%%%%%%%%%%%%%%%%%%%%%%%%%%%%%%%%%%%%%%%%%

%  Here, place your title and author information.  Note that in 
%  use of the \author command, you create your own footnotes.  Follow
%  the examples below in creating your author and affiliation information.
%  Also consult a recent issue of the journal for examples of formatting.

\title[Spatial Capture-Recapture]{
Spatial Design of Capture-Recapture Studies
}

%  Here are examples of different configurations of author/affiliation
%  displays.  According to the Biometrics style, in some instances,
%  the convention is to have superscript *, **, etc footnotes to indicate 
%  which of multiple email addresses belong to which author.  In this case,
%  use the \email{ } command to produce the emails in the display.

%  In other cases, such as a single author or two authors from 
%  different institutions, there should be no footnoting.  Here, use
%  the \emailx{ } command instead. 

%  The examples below corrspond to almost every possible configuration
%  of authors and may be used as a guide.  For other configurations, consult
%  a recent issue of the the journal.

%  Single author -- USE \emailx{ } here so that no asterisk footnoting
%  for the email address will be produced.

%\author{John Author\emailx{email@address.edu} \\
%Department of Statistics, University of Warwick, Coventry CV4 7AL, U.K.}

%  Two authors from the same institution, with both emails -- use
%  \email{ } here to produce the asterisk footnoting for each email address

%\author{John Author$^{*}$\email{author@address.edu} and
%Kathy Authoress$^{**}$\email{email2@address.edu} \\
%Department of Statistics, University of Warwick, Coventry CV4 7AL, U.K.}

%  Exactly two authors from different institutions, with both emails  
%  USE \emailx{ } here so that no asterisk footnoting for the email address
%  is produced.

\author{
J. Andrew Royle$^{1,*}$\email{aroyle@usgs.gov}
\\
$^{1}$USGS PWRC \\
}

%\and
%Kathy Author\emailx{anotherauthor@address.edu} \\
%Department of Biostatistics, University of North Carolina at Chapel Hill, 
%Chapel Hill, North Carolina, U.S.A.}

%  Three or more authors from same institution with all emails displayed
%  and footnoted using asterisks -- use \email{ } 

%\author{John Author$^*$\email{author@address.edu}, 
%Jane Author$^{**}$\email{jane@address.edu}, and 
%Dick Author$^{***}$\email{dick@address.edu} \\
%Department of Statistics, University of Warwick, Coventry CV4 7AL, U.K}

%  Three or more authors from same institution with one corresponding email
%  displayed

%\author{John Author$^*$\email{author@address.edu}, 
%Jane Author, and Dick Author \\
%Department of Statistics, University of Warwick, Coventry CV4 7AL, U.K}

%  Three or more authors, with at least two different institutions,
%  more than one email displayed 

%\author{John Author$^{1,*}$\email{author@address.edu}, 
%Kathy Author$^{2,**}$\email{anotherauthor@address.edu}, and 
%Wilma Flinstone$^{3,***}$\email{wilma@bedrock.edu} \\
%$^{1}$Department of Statistics, University of Warwick, Coventry CV4 7AL, U.K \\
%$^{2}$Department of Biostatistics, University of North Carolina at 
%Chapel Hill, Chapel Hill, North Carolina, U.S.A. \\
%$^{3}$Department of Geology, University of Bedrock, Bedrock, Kansas, U.S.A.}

%  Three or more authors with at least two different institutions and only
%  one email displayed

%\author{John Author$^{1,*}$\email{author@address.edu}, 
%Wilma Flinstone$^{2}$, and Barney Rubble$^{2}$ \\
%$^{1}$Department of Statistics, University of Warwick, Coventry CV4 7AL, U.K \\
%$^{2}$Department of Geology, University of Bedrock, Bedrock, Kansas, U.S.A.}


\begin{document}
\linenumbers
%\raggedright
%\setlength{\parindent}{.25in}

%  This will produce the submission and review information that appears
%  right after the reference section.  Of course, it will be unknown when
%  you submit your paper, so you can either leave this out or put in 
%  sample dates (these will have no effect on the fate of your paper in the
%  review process!)

\date{{\it Received xyz} 2009. {\it Revised xyz} 2009.  {\it
Accepted xyz} 2009.}

%  These options will count the number of pages and provide volume
%  and date information in the upper left hand corner of the top of the 
%  first page as in published papers.  The \pagerange command will only
%  work if you place the command \label{firstpage} near the beginning
%  of the document and \label{lastpage} at the end of the document, as we
%  have done in this template.

%  Again, putting a volume number and date is for your own amusement and
%  has no bearing on what actually happens to your paper!  

\pagerange{\pageref{firstpage}--\pageref{lastpage}} 
\volume{xy}
\pubyear{200x}
\artmonth{May}

%  The \doi command is where the DOI for your paper would be placed should it
%  be published.  Again, if you make one up and stick it here, it means 
%  nothing!

\doi{10.1111/j.1541-0420.2005.00454.x}

%  This label and the label ``lastpage'' are used by the \pagerange
%  command above to give the page range for the article.  You may have 
%  to process the document twice to get this to match up with what you 
%  expect.  When using the referee option, this will not count the pages
%  with tables and figures.  

\label{firstpage}

%  put the summary for your paper here

\begin{abstract}
\end{abstract}
%  Please place your key words in alphabetical order, separated
%  by semicolons, with the first letter of the first word capitalized,
%  and a period at the end of the list.
%

\begin{keywords}
\end{keywords}

%  As usual, the \maketitle command creates the title and author/affiliations
%  display 

\maketitle

%  If you are using the referee option, a new page, numbered page 1, will
%  start after the summary and keywords.  The page numbers thus count the
%  number of pages of your manuscript in the preferred submission style.
%  Remember, ``Normally, regular papers exceeding 25 pages and Reader Reaction 
%  papers exceeding 12 pages in (the preferred style) will be returned to 
%  the authors without review. The page limit includes acknowledgements, 
%  references, and appendices, but not tables and figures. The page count does 
%  not include the title page and abstract. A maximum of six (6) tables or 
%  figures combined is often required.''

%  You may now place the substance of your manuscript here.  Please use
%  the \section, \subsection, etc commands as described in the user guide.
%  Please use \label and \ref commands to cross-reference sections, equations,
%  tables, figures, etc.
%
%  Please DO NOT attempt to reformat the style of equation numbering!
%  For that matter, please do not attempt to redefine anything!






\linenumbers
%\raggedright
%\setlength{\parindent}{.25in}

\section{Introduction}



Let ${\cal X}$, the {\it design space}, denote some region within
which sampling could occur and let ${\bf X} = {\bf x}_{1},\ldots, {\bf
  x}_{J}$ denote the {\it design}, the set of sample locations (e.g.,
of camera traps) which henceforth will be referenced as ``traps.'' The
technical problem addressed in this paper is how to choose the
locations ${\bf X}$ in a manner that is statistically efficient for
estimating abundance or density.  The design space, ${\cal X}$, which
determines potential design points, will have to be prescribed.  This
could be some polygon describing a park or forest unit from which
we may choose trap locations.  Further, while ${\cal X}$ maybe be
continuous, 
in practice it will be sufficient to represent ${\cal X}$ by a
discrete collection of points.  This is especially convenient when
the geometry of ${\cal X}$ is complicated and irregular (which would
be in most practical applications).

The models considered here are based on the notion that individuals
have a static spatial location about which their movements are
concentrated and which can be related to trap-specific encounter
probabilities by distance alone.  The models regard the population of
$N$ such individual ``activity centers'' as the outcome of a point
process.  Denote the home range center of an individual by the coordinate
${\bf s}$ which is regarded
as the outcome of a random variable
uniformly distributed over the state-space ${\cal S}$, some
2-dimensional region.  The importance of ${\cal S}$ is obvious as it
defines a population of individuals (i.e., activity centers) and, in
practice, it is not usually the same as ${\cal X}$ due to the fact
that animals move freely over the landscape and the location of traps
is typically restricted by policies, ownership and other
considerations. That ${\cal X}$ and ${\cal S}$ are not the same is the
basic problem of geographic non-closure of the population for which
spatial capture-recapture models have been devised (Efford 2004; 
Borchers and
Efford 2008; Royle and Young 2008; Royle and Gardner 2009).

We adopt a class of models
which assume that, conditional on an individual's activity center,
${\bf s}_{i}$, the encounter frequency for individual $i$ in trap $j$
is a Poisson random variable
\[
 y_{ij}|{\bf s}_{i} \sim \mbox{Poisson}( \lambda(s_{i}, x_{j}))
 \]
where
\[
log(\lambda(s,x)) = \alpha_{0} + \alpha_{1} || s_{i} - x_{j} ||^2
\]
Conditional on $s_{i}$, this is a basic log-linear model with
\begin{equation}
  log(\lambda(s,x))  =  {\bf m}({\bf s}_{i})' {\bm \alpha}
\label{eq.linearpredictor}
\end{equation}
for ${\bf m}({\bf s}_{i})' = (1,||{\bf s}_{i} - {\bf x}_{j}||^{2})$
and regression parameters ${\bm \alpha} =( \alpha_{0}, \alpha_{1})$.

Two comments:
(1) We use a Poisson encounter model here although this is less common
than a Bernoulli model. They two should be about equivalent as long as 
$p_{0} = exp(\alpha_{0})$ is small.
(2) We are operating conditional on the latent variables ${\bf s}$. That is, as
if they are known. It is convenient to devise a variance expression in this
case and we discuss how to uncondition shortly. 


\section{An Estimator of $N$ and Variance criteria}

Consider a conditional estimator of $N$ of the form
\[
  \tilde{N}  =  \frac{n}{\bar{p}}
\]
where $\bar{p}$ is the probability that an individual appears in the
sample of size $n$. In SCR models 
 an individual is captured if it is captured in {\it
  any} trap and therefore
\[
 \bar{p}_{i} = 1 - \prod_{j=1}^{J} (1- p_{ij})
\]
keeping in mind that this is conditional on ${\bf s}_{i}$ and also the
trap locations ${\bf x}_{j}$ in the sense
that $p_{ij}$ depends on both.

The approach we take here is we develop the variance of $\tilde{N}$
conditional on knowing the locations of all $N$ individuals and then
we suggest to unconditon on the realized point process by taking a
Monte Carlo average over realizations of ${\bf s}$ under a suitable
model for ${\bf s}$. The variance criterion we propose here is based
on a delta approximation $Var(n/\bar{p})$:
\[
 Var(\tilde{N}(\alpha) | \{ {\bf s}_{i} \}_{i=1}^{N} ) = 
\frac{N^{2} Var(\bar{p})}{\bar{p}^{2}}  + N\frac{(1-\bar{p})}{\bar{p}}
\]
It is important to note that this is the sum of two parts which are
essentially those due to (1) estimation of $\bar{p}$ from the sample
and and (2) the variance of $n$. We see that generally the criterion is improved
(decreases) as we do a better job estimating $\bar{p}$ and also as $n$
approaches $N$, i.e., as $\bar{p}$ increases to 1. Thus, good designs
should generate information about detection probability {\it and}
produce large samples of individuals. 

In order to work with this expersssion we will
have to do some analysis of $Var(\bar{p})$ which we take up in the
next section.

We emphasize  that the above variance expression
is {\it conditional} on the realization  ${\bf s}_{1},\ldots, {\bf
  s}_{N}$ which is, in the 
context of design, not observable.  We will therefore develop design
criteria which are unconditional on $\{ {\bf s} \}$.
The total variance expression is unconditional on ${\bf s}$ is:
\[
Var(\tilde{N}) = E_{s} Var(\tilde{N}|s) + Var_{s} E(\tilde{N}|s)
\]
if we assert that sample sizes will be large enough so that our
estimator is unbiased, then the 2nd term will be close to 0 and we can
ignore it. 
Therefore to evaluate the unconditional variance we need to solve an
$N-$fold integration to average over ${\bf s}_{1},\ldots, {\bf
  s}_{N}$.
We can do this by taking a monte carlo average or perhaps we can
simplify some math and do it numerically.... this is to be determined
shortly. 




\subsection*{Analysis of $Var(\bar{p})$}
Remember that $\bar{p}$ is not estimated in SCR models directly but,
rather, is a function of the vector parameter ${\bm \alpha} =
(\alpha_{0},\alpha_{1})$. We can express the variance of $\bar{p}$ as
a function of the variance of $\hat{\bm \alpha}$ using a delta
approximation. This yields
\[
\left( \frac{d \bar{p}}{d {\bm \alpha} } \right)'  Var(\hat{\bm
  \alpha})
\left( \frac{d \bar{p}}{d {\bm \alpha} } \right)
\]
where
$\left( \frac{d \bar{p}}{d {\bm \alpha} } \right)$ is a vector having
elements
\[
\left( \frac{d \bar{p}}{d {\bm \alpha} } \right)  =   \left(
\begin{array}{c}
\frac{d \bar{p}}{d \alpha_{0} } \\
\frac{d \bar{p}}{d \alpha_{1} }
\end{array}
\right)
\]

Ok so now we have to compute each of these two pieces.  We are going
to do this for the {\it Poisson} encounter model as described above
XXXXX. This is slightly easier on the calculus-bone although we can do
it for any prescribed model. The results in terms of the optimal
design should be hardly affected in my view. 

Anyhow, under the Poisson model,
\[
\bar{p}_{i} = 1-\exp(-\sum_{j} \lambda_{ij})
\]
NOTE TO ANDY: THIS IS SLIGHTLY WRONG. ``i'' SHOULD NOT BE A SUBSCRIPT
ON BAR P HERE BECAUSE BAR P IS THE SPATIAL AVERAGED!  MAYBE I'M
SCREWED. 

Its a little bit confusing what i'm going to do now. We see that
$\bar{p}$ is expressed conditional on each ${\bf s}$ and we want to
take the average of that over all possible values of ${\bf s}$. This
is easy to do. We just do that integraiton as a summation and it works
out that we just average the derivatives in the above expressions. 
This is strange because we keep the 2x2 var-cov matrix conditional on
the point process but we have marginizled the derivative.....


\section{Moving on}


To make further progress we have to deal with 
two 
things: (1) This expression is conditional $z_{i}$ (i.e., in Eqs. \ref{eq.del0}
and \ref{eq.del1}) 
Since we're doing this in the context of design where we don't
actually observe {\it any} individuals, we don't have the $z_{i}$
variables in these expressions. For that we need to take the expected
value of this over $z_{i}$. We cannot just plug in its 
expectation because $z_{i}$ appears as a quartic term, there is  a
quadratic inherented from the evaluation of ${\bf v}_{i}$ and then
another quadratic term that comes out of the $Cov(\hat{\alpha}|{\bf
  z})$. 
(2) The variance is conditional on $\{ {\bf s}_{i} \}$. These latent
variables are not observed (even for captured individuals!). 


To deal with item (1) first, we note that $E[z_i^{4}] = \phi_{i}$ (the
4th raw moment of a Bernoulli random variable is the same as the
expected value). Therefore, the {\it unconditional} variance is:
\[
Var(\tilde{N}(\alpha)) =  E_{{\bf z}} Var(\tilde{N}(\alpha) | {\bf z}) = \sum_{i=1}^{N}
\frac{ (1-\phi_{i})^{2} }{\phi_{i}^{3}} 
(F_{i}({\bf X}), G_{i}({\bf X}))'   
({\bf M}'{\bf D}(\hat{\bm \alpha}){\bf M})^{-1}
(F_{i}({\bf X}), G_{i}({\bf X}))
\]


Ok  lets stop and clarify where we're at here. We started out with an
expression for the variance of $\tilde{N}(\hat{\bm \alpha}) | {\bf
  z})$ which is {\it conditional} on ${\bf z}$, and now its
unconditional on ${\bf z}$ and , as a result, it depends on
$\phi_{i}$.
We emphasize that the variance is conditional on the latent variables
${\bf s}_{i}$ which we will do something about shortly.


This variance expression is unconditional on ${\bf z}$ but we're
missing some variation because we know that
\[
Var(N) = E Var (N|z) + Var E (N|z)
\]
and thus the expression above XXX is only the first part of this, it
is $E(Var(N|z))$. So we need to figure out what is
$Var(E(\tilde{N}|z))$. This is (Alho 1990):
\[
Var E (N|z) =  \sum_{i=1}^{N} \frac{ \phi_{i}(1-\phi_{i}) }{\phi_{i}^{2}}
\]
which again is conditional on ${\bf s}_{i}$ variables. 

Because $\{ {\bf s}_{i} \}$ are unknown, we should take the expected
value of this variance over all possible values of ${\bf s}_{i}$:
\begin{eqnarray*}
 Var(\tilde{N}(\alpha) ) &= &
\sum_{i=1}^{N} 
\int_{s_{i} \in {\cal S}}
\frac{ (1-\phi_{i})^{2} }{\phi_{i}^{3}} 
(F_{i}({\bf X}), G_{i}({\bf X}))'   
({\bf M}'{\bf D}(\hat{\bm \alpha}){\bf M})^{-1}
(F_{i}({\bf X}), G_{i}({\bf X}))
 d{\bf s}_{i}  \\ 
& + &
\int_{s_{i} \in {\cal S}}  \frac{ \phi_{i}(1-\phi_{i}) }{\phi_{i}^{2}}
 d{\bf s}_{i} 
\end{eqnarray*}
which, in practice, we approximate on a finite grid of points and
compute the integral as a summation over each possible value of ${\bf s}_{i}$.

In practice, 
We have to be careful because mean of $(1-phi)^2$ is not $(1-bar.phi)^2$!!!!
Therefore I have to redo the calculations to properly compute these
functions of $phi_i$.



The idea is to develop design criteria
based on this asymptotic variance. Because it is conditional on 
the design points ${\bf X}$, we might think about trying to 
find the set of design poins that minimizes some function of this
matrix. e.g., if we focus on its
determinant,  
this is the standard ``D-optimal'' criterion (Mitchell 1974;
Hardin and Sloane 1993; Box and Draper 1997) widely used in classical
experimental design for estimating response surfaces. We have to deal
with two technical issues: First, that the random variables ${\bf s}$
are unknown and also that there are $N$ individuals (each with its own
${\bf s}$).

Given that the variance is expressed conditional on ${\bf s}$, a
latent variable, a natural thing to do is to average this variance
over possible values of ${\bf s}$, and seek to locate design points to
minimize the integrated variance of the regression coefficients. In
the present context, we would integrate over ${\cal S}$.  Thus, a
heuristically appealing 
design criterion is
\begin{equation}
Q({\bf X}) =  \sum_{ {\bf s} \in {\cal S} }
|\mbox{Var}(\hat{\bm beta}|{\bf X},{\bf s})|
\label{eq.Q}
\end{equation}
This is easy enough to evaluate for a discrete ${\cal S}$. In fact,
the determinant can be computed analytically and the calculations
vectorized for efficient calculation. It seems reasonalbe that
designs that have small values of 
Eq. (\ref{eq.Q}) should be preferred. Ideally, we might hope to find 
the design that minimizes Eq. (\ref{eq.Q}) over all possible designs
of a fixed dimension. This is discussed in the following Section.







\subsection{Some remarks}

The variance criterion is the sum of two parts: One that relates to
how well parameters are being estimated, and one that relates to how
many individuals are being captured.  Both components are themselves
sensible design criteria. In a pilot study , for example, we might
choose to focus on geting good estimates of the parameters but don't
care about estimating N and therefore having a large sample of
individuals is less important.  On the other hand, we might also
choose to design solely based on minimizing the variance of $n$ which
effectively locates design points so as to maximize the expected
capture probability. This is very much in a similar vane as
space-filling designs or classical designs which seek to optimize the
prediction variance and similar.
In general, designs which are good for the joint criterion thus seek a
kind of compromise between obtaining information about the parameters
and capturing a large sample of individuals. 


 clearly it is reasonable that 
the criterion should be sensitive to
the size of the state-space ${\cal S}$ because this defines the
population of individuals that is the focus of inference.  Design
points are going to try to minimize the expected variance, where the
expectation is taken over space (i.e., ${\cal S}$). The design will
try to maximize exposure of all points in ${\cal S}$. 
This is typical of 
other spatial design problems. For example, for designs based on
optimal prediction variance using random field models, ${\cal S}$ will
represent the region over which predictions are desired or otherwise
defines the scope of inference (see Discussion).  This is problematic
from a practical point of view in capture-recapture studies because we
may not be able to delineate a population of interest in a precise
way. In that case then we might restrict attention to the design space
and set ${\cal S} = {\cal X}$.  








\section{Optimization of the criterion}

In formulating the optimization problem note that we have $J$ sample
locations corresponding to rows of ${\bf X}$.  The problem is a $2J$
dimensional optimization problem which, for $J$ small, could be solved
using standard numerical optimization algorithms as exist in almost
every statistical computation environment.  However, $J$ will almost
always be large enough so as to preclude effective use of such
algorithms. This is a common problem in experimental design, design
for response surface estimation, computer experiments, spatial
sampling designs and other disciplines for which sequential exchange
or swapping algorithms can be used (e.g., Wynn 1970; 
Fedorov 1972; Mitchell 1974; Meyer and
Nachtsheim 1995). The basic idea is to pose the problem as a sequence
of 1-dimensional optimization problems in which the objective function
is optimized over 1 or several coordinates at a time.

In the present case, we consider swapping out ${\bf x}_{j}$ for some
point in ${\cal X}$ that is nearby ${\bf x}_{j}$ (e.g., a 1st order
neighbor). The objective function is evaluated for all possible swaps
(at most 4 in the case of 1st order neighbors) and whichever point
yields the biggest improvement is swapped for the current value.  The
algorithm is iterated over all $J$ design points and this continues
until convergence is achieved. Such algorithms may yield local optima
and optimization for a number of random initial designs can yield
incremental improvements. We implemented this swapping algorithm in
{\bf R}, using the basic strategy employed elsewhere (e.g., Nychka et
al. 1997; Royle and Nychka 1998).  A version of a swapping
algorithm used to optimize a space-filling criterion is implemented
in the {\bf R} package {\bf
  fields} (Fields Development Team 2006).  We developed an
  implementation for
a discrete representation of ${\cal S}$ (an aribtrary matrix of
coordinates), and the required inputs are the grid of coordinates
${\cal S}$ and also the potential sample locations ${\bf D}$, which
may or may not be elements of ${\cal S}$. 
For each design point being considered for exchange, only 
the nearest neighbors (the number is specified) in the design space
${\bf D}$ are  considered for
swapping into the design.

While swapping algorithms are convenient to implement, and efficient
at reducing the criterion in very high dimensional problems, they will
not usually yield the global optimum.  In practice, as in the examples
below, it is advisable to apply the algorithm to a number of
random starting designs. Our experience is that essentially
meaningless improvements are realized after searching through a few
dozen random starts.


\section{Illustration}

I think what I want to do is something along the following lines:

 small number of points with sigma = .5, 1, 2, 3

 largeish number of points with sigma = .5, 1, 2, 3

Look at each componet of the criteria and then the joint criteria
which should dominate things.




Consider designing a study for camera traps in a square region defined
by the square $[10,20] \times [10, 20]$ and with ${\cal X} = {\cal
  S}$.  For this illustration I assumed $\beta_{0} = log(\lambda_{0})
= -2.7$ and $\beta_{1} = 1/(\sigma^{2}) = 1/4$, $1/9$ and $1/16$, so
$\sigma = 2,3,4$. (this was dumb - note that $\sigma$ is really 2
times the standard deviation of a normal distribution. Oh well!).
Designs of size 9 and 10 were computed for each value of $\sigma$
using many random starting designs.  The putative optimal designs
(henceforth ``best'') are shown\footnote{My intention is to provide
  many of these results in an Appendix in order to reduce the length
  of the paper.} in Figure \ref{fig.fig1}.  For J=9, $\sigma =2$, the
best design was produced in 180 out of 1000 random starts.  For
$\sigma = 3$ (row 2, left panel) the best design was produced in about
88\% of all optimizations from random starting values.
%reason is that there are a lot of points in the interior that interact
%relatively little with the design and these ``holes'' tend to cause
%the algorithm to get caught in a local optimum (my interpretation) of
%the objective function.  Or, consider this, with sigma = small the
%design can probably be translated a little bit in space .... this is
%what I think happens.  
For $J=10$, and $\sigma =2$ (row 1, right panel), the best design was
found about 24\% of the time (from random starts). 
The $\sigma = 3$ best design (row 2, right
panel; 14\% of random starts) clusters 2 points in the center.
Finally, consider the $\sigma =4$ case (last row of
Fig. \ref{fig.fig1}).  We have two irregular looking designs and 
the design points cluster in various ways. 
% For $J=9$ this was produced
%only 1 time whereas only 8 instances of 1000 produced the
%best design for $J=10$.  We might thus have little confidence in that
%result\footnote{subsequent analyses have failed to find a better
%  design.}. 

I computed the best designs using the same settings but inreasing the
size of ${\cal S}$ relative to ${\cal X}$.  In particular, I nested
${\cal X}$ into $[9,21] \times [9,21]$ (Figure \ref{fig.fig2})
 and then $[8,22]^{2}$ (Figure \ref{fig.fig3}).
The obvious effect of this is
that the best designs move points toward the edge of the design space
${\cal X}$ so as to provide more exposure to points in ${\cal S}$.
The effect is more pronounced, obviously, as you provide more area
outside of ${\cal X}$ that is allowed to influence the design.

As a final example, 
consider placing 20 camera traps in this region. Where do they
go? Look at the 3 buffers, 3 values of sigma, thats 9 total designs
(use a single panel).
An interesting feature of the designs is that they are not
regular. Traps occur in clusters of several traps close together
with the clusters more widely spaced.


\begin{figure}
\centering
%\includegraphics[height=6in,width=4in]{fig1-buff0.pdf}
\caption{J=9 and J=10 designs for 
 ${\cal X} = {\cal S}$.... no buffer.
}
\label{fig.fig1}
\end{figure}


\begin{figure}
\centering
%\includegraphics[height=6in,width=4in]{fig2-buff1.pdf}
\caption{
Best J=9 and J=10 designs for 
${\cal S}$ equal to 
${\cal X}$ buffered by 1.
}
\label{fig.fig2}
\end{figure}

\begin{figure}
\centering
%\includegraphics[height=6in,width=4in]{fig3-buff2.pdf}
\caption{
Best J=9 and J=10 designs with a 2 unit buffer.
}
\label{fig.fig3}
\end{figure}



\begin{figure}
\centering
%\includegraphics[height=6in,width=4in]{fig4-buff02.pdf}
\caption{
Optimal designs for J=20 design points and 0 or 2 buffer.
}
\label{fig.fig4}
\end{figure}

\subsection{Effectiveness of the sequential exchange algorithm}

The sequential swapping algorithm always converges but may frequently
find a local optimum of the $2J$-dimensional objective function. In
some cases it is instructive to inspect different local optima and
this seems to usually indicate that they are variations on a theme,
with the points in clusters but locked into a slightly sub-optimal
configuration. This is partially
aided by changing the number of nearest neighbors considered for
swapping at each iteration and I suspect other modifications can be
made to improve the effectiveness.

If multiple equivalent solutions are found based on random starts,
then I interpret this as evidence that the global optimum was
found. e.g., if based on 1000 starts we find the same best design many
times then it may be reasonable to regard the modal design as the
global optimum.  Otherwise we can examine the relative closelness
of the local optima and it seems to always be the case that, when
there is not a modal best design, 
the best solutions vary by fractional percentages of the
criterion. At that point we probably don't care too much given
logistical and practical constraints on trap placement anyways. e.g.,
if a point falls in a lake or river it wouldn't make too much sense to
put it there.

Most of the optimizations used to compute the figures XXX-YYY were
based on at least 1000 random starts. For cases where the best design
occurred only once, I ran a follow-up run with 10000 random starts.
I summarized the
frequency with which the best solution was found and the
corresponding criteroin value using 4 nearest neighbors considered for
swapping into the design (Table XXXX).
We see 
that when sigma is small then....................
each point kind of creates a bubble around it so that a point can't
get to the other side if it has to. By inreasing nn then it can make a
big jump and get over there.  seems that higher nn or mixing them up
might be good. The algorithm is highly amenable to parallelization
etc... but I used a brute force implementation under which it would
usually take several hours to run through 2000 iterations.......
For the n=20 designs.... high clustering..... causes minor differences....
 However, I think the 2nd, 3rd, etc.. best designs are
almost irrelevant as they differ in fractional percentages of the
variance criterion. Thus, as a practical matter, it is not so
important to be conifdent that we have {\it the} optimal design.

\begin{verbatim}
             Buffer=0               Buffer=1          Buffer = 2
sig=2,J=9  180/1000  3.851908  15/1000   5.586599  1/1000    7.676768
sig=2,J=10 240/1000  3.498840  100/1000  4.879272  11/1000   6.708044
sig=3,J=9  880/1000  0.3682018  1/1000    0.9686344  11/1000   1.707353
sig=3,J=10 141/1000  0.1503701 557/1000  0.72635    612/1000  1.410881
sig=4,J=9,   1/1000 -1.423150  213/1000 -1.072478  726/1000 -0.6637247
sig=4,J=10   8/1000 -1.635006  33/1000  -1.276037  133/1000 -0.885592

case 5 buff=0 was run for 10000 and produced 17 cases of the same
optimal

case 3, buff=1 was run for 10000 and produced 7 cases of the same
design

case 1 buffer=2 was run for 10000 and produced 12 instances of the
same optimal design.


Note that cases with 1/1000 were rerun with 10k iters as was the 2nd
case for buffer=2 to see what happens.

J=20,
              FALSE  TRUE 
sig=2, buf 0   999     1  1.691365
sig=2, buf 2   986    14  3.277961
sig=3, buf 0   998     2  -1.235308
sig=3, buf 2   961    39  -0.1629501
sig=4, buf 0   999     1  -3.02408
sig=4, buf 2   999     1  -2.273385
\end{verbatim}







\begin{figure}
\centering
\includegraphics[height=6in,width=4in]{fig5-buff0.pdf}
\caption{
% name of file is misleading, has 0 and 2 buffer
Optimal designs for J=9 using conditional variance criterion,
corrected for E[n].... using 2 buffers (0 and 2). We see that buff = 0
favors interior points and vice versa. As it should.  However, as buff
gets large resulting designs should be insensitive to this.... need to
do one set of sims to check this out (For J=9 and J=10, use buffer=5
or something).
}
\label{fig.fig5}
\end{figure}


\begin{figure}
\centering
\includegraphics[height=6in,width=4in]{fig6-buff2.pdf}
\caption{
% name of file is misleading, has 0 and 2 buffer
Optimal designs for J=20 using conditional variance criterion,
corrected for E[n].... using 2 buffers (0 and 2). We see that buff = 0
favors interior points and vice versa. As it should. 
}
\label{fig.fig6}
\end{figure}


\section{Case Study: Mink sampling on the Hudson river}

Some of the examples analyzed previously were recomputed using this
new criterion...............

Optimal designs for J=9 using conditional variance criterion,
corrected for E[n].... using 2 buffers (0 and 2). We see that buff = 0
favors interior points and vice versa. As it should.  However, as buff
gets large resulting designs should be insensitive to this.... need to
do one set of sims to check this out (For J=9 and J=10, use buffer=5
or something).

Optimal designs for J=20 using conditional variance criterion,
corrected for E[n].... using 2 buffers (0 and 2). We see that buff = 0
favors interior points and vice versa. As it should. 



\section{Discussion}

Almost all capture-recapture studies involve spatial sampling yet only
recently has the spatial dimension of such studies been formalized
using spatially explicit statistical models and contemporary methods
of inference (Borchers and Efford 2008; Royle and Young 2008).  Given
the formalization of the modeling and inference framework, it is
natural to consider using such models to inform sampling, as is done
in many branches of classical statistics (Box and Draper 1997; Fedorov
and Hackl 1997; M$\ddot{u}$ller 2007).

In this paper, we developed a framework for constructing spatial
sampling designs which are optimal for estimating model parameters
under a class of spatial capture-recapture models.  The class of
models assumes that encounter probabilities are a function of the 
distance between sample locations (i.e., traps) and an individual
latent 
variable representing home range or activity center.  

Adapting results from Alho (XXXX) we devise the asymptotic
variance-covariance matrix for the regression parameters  and for the
conditional estimator of $N$. We formulate design criteria based on
the {\it expectated value} of the  variances, averaged over the
state-space of the underlying random variables ${\bf s}$. Thus, the
resulting design criterion only depends on the sample locations.

As a result, the variance of model parameters (or
functions of them) can be minimized over any set of candidate sample
locations -- the design space -- ${\cal X}$.  I considered choosing
the sample points such that the generalized variance of the model
parameters is minimized.  This design will correspond to the optimum
design for estimating abundance or density also based on the results
of Sanathanan (1972).  The case where $N$ is not known, in which case
the all-zero individual encounter histories are not observable, can be
resolved by properly conditioning on the event that an individual is
observed in at least one trap. The probability mass function of the
observations in this case is a scaled (or truncated) Poisson, which is
also a member of the exponential family of distributions. The
asymptotic variance-covariance matrix resolves to an intuitive form in
which the ordinary variance criterion is weighted approximately by the
probability that an individual is encountered.
A useful property of optimal designs under this model
is that they are invariant to the amount of
effort or temporal sampling (length of time the traps are operational or 
replicates) by each trap.

I have evaluated the design framework 
under regular shaped regions which is not so
common in practice. However, a virtue of the framework is that arbitrary
configurations of ${\cal X}$ and ${\cal S}$ can be specified.
Even with regular ${\cal X}$ and ${\cal S}$, resulting designs may be
irregular (not symmetric). Moreover, a general feature of optimal
designs is that they 
often involve clusters of sample locations -- 2 or more traps forming
a cluster, and sometimes with  more of a systmatic distribution of clusters.
This is contrary to
practice where investigators seldom deviate from a regular, systematic
trap spacing.
A basic result of design for spatial capture-recapture, shared with
other spatial design contexts is that both ${\cal X}$ and ${\cal S}$
are important in determining the optimal design.  In classical spatial
design, as in spatial capture-recapture, the area over which
predictions are desired (corresponding to ${\cal S}$) might typically
be distinct from the region within which samples can be collected
(${\cal X}$).  
This is not something that has ever been
noted, or considered in practice to the best of my knowledge. Rather,
biologists just consider ${\cal X}$ in deciding where to place traps.
%Conversely, the relevance
%yet its relevance to {\it estimation} has been (e.g., Borchers
%and Efford 2008; Royle and Young 2008; Royle et al. 2009).  
Even under the conditional formulation of the model which,
heuristically, weights points in ${\cal S}$ by their distance to the
design, the size and configuration of ${\cal S}$ relative to ${\cal
  X}$ will influence the design.  For example, if a portion of ${\cal
  X}$ is bordered by a large body of water, then the resulting design
will be affected by that.
I have scarcely considered the influence of the number of traps on the
design criterion.
However, the proposed framework makes explicit this consideration
by the specification of a variance criterion that is influenced
directly, in a manner amenable to study, by both the number and
configuration of traps.
As such, it 
allows formal assessments to be made of the influence of different numbers
of traps under their optimal configuration which should be a component
of any study.  For example, 20 traps configured in a highly efficient
manner might be preferred to more traps configured in a standard, 
systematic manner. 


The framework is model based and so it is natural to question the
sensitivity or robustness of the design to model misspecification.
Two options arise in this case: It seems possible to extend the
framework to accomodate multiple models if one could pose such models
and, secondly, we could use designs that have a heuristic basis but
are not model-based. For example, it is clear that 
sampling design for spatial capture-recapture studies has much in
common with classical ``spatial design'' (M$\ddot{u}$ller 2007),
which are often used as ``exploratory'' designs when little is
known about the underlying form of the model.
The criterion for the class of models considered here is structurally
and conceptually similar to distance-based criteria used to construct
``space-filling'' designs (Johnson et al. 1990; Hardin and Sloane
1993; Nychka et al. 1997; M$\ddot{u}$ller 2007, Sec. 4.2).  For example, it seems sensible in the
context of spatial capture-recapture studies to attempt to minimize
the expected distance of an individual to the collection of sample
locations.  Thus, we might construct a criterion based on the total
distance of individual $i$ to the ``design'' (i.e., all trap
locations):
\begin{equation}
d({\bf s}_{i}, {\bf X}) =  \sum_{j}   ||{\bf s}_{i} - {\bf x}_{j} ||
\label{eq.d1}
\end{equation}
or perhaps the minimum distance to the trap array:
\begin{equation}
 d({\bf s}_{i},{\bf X}) = \stackrel{min}{{\bf u} \in {\bf X}} ||{\bf
   s} -{\bf u}||.
\label{eq.d2}
\end{equation}
Then the ``expected exposure'' (to the design) of individuals
in the state space ${\cal S}$ is
\begin{equation}
 Q({\bf X}) = 
\left(
 \sum_{{\bf s} \in {\cal S}} d({\bf  s},{\bf X}) \right)
\label{eq.spacefilling}
\end{equation}
Under distance metric (\ref{eq.d2}) this criterion is the so-called
uniform coverage design
(Fang and Wang 1994)
which is the U-criterion 
computed in PROC
OPTEX (SAS, 2004). Both metrics are special cases of a generalized
criterion (See Nychka et al. 1997) implemented in an {\bf R} library
{\tt Fields} (Fields Development Team 2006).  We see that elements of
the asymptotic variance-covariance matrix of $\hat{\bm \beta}$ are
structurally similar to this purely distance-based criterion, but with
an exponential weighting scheme.  Some investigation into the
connections between space-filling designs and those for spatial
capture-recapture models seems warranted.  As a practical matter, it
might be sensible to consider using space filling designs by default
as they may be reasonable compromises between alternative models, or
perhaps robust to model misspecification. Designs that minimize the
average distance between points in the state-space ${\cal S}$ and the
design should produce high total exposure of individuals to traps
under regular encounter models (e.g., that are monotone decreasing in
distance). In addition, space-filling designs may be useful as initial
designs (i.e., in 1st year studies) before data are available to
inform model parameters.







\section*{Literature Cited}


\newcommand{\rf}{\vskip .1in\par\sloppy\hangindent=1pc\hangafter=1
                  \noindent}






%\rf Anderson, D., Burnham, K., White, G. \& Otis, D. (1983)
%Density estimation of small-mammal populations using a trapping web and 
%distance sampling methods. {\it Ecology}, {\bf 64}, 674-680.
 
\rf Borchers, D.L. and M.G. Efford. 2008.
Spatially explicit maximum likelihood methods for capture-recapture studies.
{\it Biometrics} 64:377-385.

\rf Box, GEP and N.R. Draper 1987. Empirical model-building with response
surfaces.
Wiley, New York.

\rf Box, GEP and N.R. Draper 1959. A basis for the selection of a response
surface design. J. Amer. Stat. Associ. 54, 622-654.

%\rf Buckland, S., Anderson, D., Burnham, K., Laake, J., Borchers,
%D. \& Thomas, L. (2001) Introduction to distance sampling: estimating
%abundance of biological populations: Oxford University Press.




\rf Efford, M. 2004. Density estimation in live-trapping studies. 
{\it Oikos} {\bf 106}:598-610.

\rf Fang, K.-T. and Y. Wang. 1994. Number-theoretic Methods in
Statistics.
Chapman \& Hall, London.

\rf  Fedorov, V.V. 1972. Theory of Optimal Experiments. Academic
Press, New York.

\rf Fedorov, V.V. and P. Hackl. 1997. Model-oriented design of
experiments. 1997. Springer.

\rf Fields Development Team (2006). fields: Tools for Spatial Data. National Center for Atmospheric Research, Boulder, CO. URL http://www.cgd.ucar.edu/Software/Fields. 

\rf  Gardner, B., J. Reppucci, M. Lucherini and J.A. Royle. in review.
Spatially-explicit  inference  for open  populations:  estimating  demographic
parameters from camera-trap studies. 

\rf Gardner, B., J.A. Royle and M.T. Wegan.  2009. 
Hierarchical models for
estimating density from DNA mark-recapture studies. {\it Ecology} 90:1106-XXXYYYZZZZ.

\rf Hardin, R.H. and N.J.A. Sloane. 1993. A New Approach to the Construction
of Optimal Designs. J. Statistical Planning and Inference, vol. 37,
1993, pp. 339-369 

%\rf Henschel, P. and J. Ray. 2003.
%Leopards in African Rainforests: Survey and Monitoring Techniques.
%WCS Global Carnivore Program website.

\rf Johnson, M.E., Moore, L.M., and Ylvisaker, D. (1990). Minimax and 
maximin distance designs. Journal of Statistical Planning and
Inference 
26, 131-148. 

\rf Kiefer, J. 1959. Optimal experimental designs (with
discussion). {\it Journal of the Royal Statistical Society, Series B}
272-319.



\rf Karanth, K.U. 1995. Estimating tiger (Panthera tigris)
populations from camera-trap data using capture-recapture models. {\it
  Biological Conservation} {\bf 71}:333-338.
 
%\rf Karanth, K.U., Chundawat, R., Nichols, J.D. \& Kumar, N.S. (2004a) 
%Estimation of tiger densities in the tropical dry forests of Panna, Central 
%India, using photographic capture-recapture sampling. 
%{\it Animal Conservation}, {\bf 7}, 285-290.
 
\rf Karanth, K. U. and J.D. Nichols. 1998. Estimation of tiger densities
 in India using photographic captures and recaptures. 
 {\it Ecology} {\bf 79}:2852-2862.
 
%\rf Karanth, K.U, J.D. Nichols, N.S. Kumar, W.A. Link,  and J.E. Hines.
%2004.
%Tigers and their prey: Predicting carnivore densities from prey abundance. 
%{\it Proceedings of the National Academy of Sciences} {\bf 101}:4854-4858.

%\rf Karanth, K.U., J.D. Nichols, N.S. Kumar, and J.E. Hines. 2006.
%Assessing tiger population dynamics using photographic capture-recapture 
%sampling. {\it Ecology} {\bf 87}:2925-2937.
 
\rf Karanth, K.U. and J.D. Nichols. 2002. Monitoring tigers and their 
prey: a manual for researchers, managers and conservationists in 
Tropical Asia. Centre for Wildlife Studies, Bangalore, India. Editors.

\rf Maffei, L., E. Cuellar and A. Noss. 2004. 
One thousand jaguars (Panthera onca) 
in Bolivia's Chaco? Camera trapping in the Kaa-Iya National Park. 
{\it Journal of Zoology} 262:295-304.

%\rf Magoun, A.J., P. Valkenburg and R.E. Lowell. 2008.
%Habitat associations and movement patterns of reproductive
%female wolverines (Gulo gulo luscus) on the Southeast
%Alaska mainland. Wildlife Research Annual Progress Report, 
%Alaska Dept. of Fish and Game. Petersburg, Alaska. 29pp.

\rf Meyer, R.K., Nachtsheim, C.J., 1995. The coordinate-exchange algorithm for 
constructing exact optimal
experimental designs. Technometrics 37 (1), 60-69.

\rf Mitchell, T.J. 1974.
An algorithm for the construction of" D-optimal" experimental designs.
Technometrics.
203-210.

\rf Morrison, M.L., M.D. Strickland, W.M. Block, B.A. Collier, and 
M.J. Peterson. 2008. Wildlife Study Design. Springer.

\rf M$\ddot{u}$ller, W.G. 2007. 
Collecting spatial data: optimum design of experiments for random fields.
Springer.

\rf Nichols, J.D. and K.U. Karanth. 2002.  Statistical concepts:
estimating absolute densities of tigers using capture-recapture
sampling.  In Karanth, K.U. and J.D. Nichols, Eds., ``Monitoring
tigers and their prey: a manual for researchers, managers and
conservationists in Tropical Asia,'' Centre for Wildlife Studies,
Bangalore, India.

\rf  Nychka, D., Q. Yang, and J.A. Royle. 1997. Constructing spatial designs 
for monitoring air pollution using regression subset selection,
pp. 131-154 in V. Barnett and K.F. Turkman (eds.), Statistics for the 
Environment 3: Pollution Assessment and Control, Wiley, New York.

\rf R Development Core Team. 2004.
R: A language and environment for statistical computing.
R Foundation for Statistical Computing Vienna, Austria.

\rf  Royle, J.A. 2002. Exchange algorithms for constructing large spatial 
designs. Journal of Statistical Planning and Inference, 100(2), pp. 121-134

\rf Royle, J.A. 2008.
Analysis of capture-recapture models with individual 
covariates using data augmentation. {\it Biometrics} (in press).
Published online 16 Apr 2008
\mbox{\tt  http://dx.doi.org/10.1111/j.1541-0420.2008.01038.x}

%\rf Royle, J.A., R.M. Dorazio and W.A. Link. 2007. Analysis of multinomial 
%models with unknown index using data augmentation. 
%{\it Journal of Computational and Graphical Statistics} {\bf 16}:67-85.

%\rf Royle, J.A. and R.M. Dorazio. 2008.
%Hierarchical Modeling and Inference in Ecology. Academic Press.

\rf Royle, J.A., J.D. Nichols, K.U. Karanth, and A. Gopalaswamy. 
2009.
A hierarchical model for estimating density in camera trap studies.
{\it Journal of Applied Ecology} 46:118-127.

\rf  Royle, J.A. and D. Nychka. 1998. An algorithm for the construction of 
spatial coverage designs with implementation in SPLUS, {\it Computers \& 
Geosciences} 24:479-488. 

\rf  Royle, J.A. and K.V. Young. 2008. 
A hierarchical model for spatial capture-recapture data.
{\it Ecology} {\bf 89}:2281-2289.

\rf Sacks, J., W.J. Welch, T.P. Mitchell, and H. Wynn. 1989. Design and
analysis of
computer experiments. {\it Statistical Science} 4:409-435.

\rf SAS/QC 9.1 User's Guide. 2004. SAS Institute Inc. SAS Campus
Drive, 

\rf Sanathanan, L. 1972. Estimating the size of a multinomial population.
{\it The Annals of Mathematical Statistics} 43:142-152.

%\rf Stickel, L. (1954) A Comparison of Certain Methods of Measuring 
%Ranges of Small Mammals. {\it Journal of Mammalogy}, {\bf 35}, 1-15.
 
%\rf Trolle, M. and M. K\'{e}ry. 2003.  Estimation of ocelot density in the 
%Pantanal using capture-recapture analysis of camera-trapping data. 
%{\it Journal of Mammalogy} {\bf 84}:607-614.
 
%\rf Trolle, M. and M. K\'{e}ry, M. 2005. 
%Camera-trap study of ocelot and other secretive mammals in the northern Pantanal.
%{\it Mammalia} {\bf 69}:409-416.

%\rf Wallace, R., H. Gomez, G. Ayala, and F. Espinoza. 2003. 
%Camera 
%trapping for jaguar (Panthera onca) in the Tuichi Valley, Bolivia. 
%{\it Journal of Neotropical Mammalogy} {\bf 10}:133-139.

\rf Wegge, P., Pokheral, C.Pd. \& Jnawali, S.R. 2004. Effects of trapping 
effort and trap shyness on estimates of tiger abundance from camera trap 
studies. {\it Animal Conservation} {\bf 7}, 251-256.
 
%\rf  Wikle, C. K. and J.A. Royle. 1999. Space-time dynamic design of 
%environmental monitoring networks. J. Agr. Biol. and Env. Statistics. 
%4(4):489-507.

\rf Wilson, K.R. and D.R. Anderson. 1985a. Evaluation of two density 
estimators of small mammal population size. {\it Journal of Mammalogy}
{\bf 66}:13-21.

\rf Wilson, K.R. and D.R. Anderson. 1985b. Evaluation of a nested 
grid approach for estimating density. {\it Journal of Wildlife Management}
{\bf 49}:675-678.

\rf Wynn, H.P.  1970. The sequential generation of D-optimum
experimental designs.
{\it Annals of Mathematical Statistics} 41:1655-1664.

%\rf Wolpert, R.L. and K. Ickstadt. 1998.
%Poisson/gamma random field models for spatial statistics. 
%Biometrika {\bf 85}:251-267.






\end{document}
