\chapter{
Hierarchical Models for Spatial Capture-Recapture Data
}
\markboth{Modeling}{}
\label{chapt.modeling}


\vspace{.3in}


\subsection{Conceptual and Technical Scope of this Book}

In this book, we try to achieve a broad methodological scope from
basic closed population models using a number of distinct observation
models on up to open population models - spatial versions of
conventional Jolly-Seber models. A number of conceptual and
methodological thems unify the main topical coverage of this book, and
those are:

\begin{itemize}
\item[(1)] Hierarchical modeling. We develop hierarchical models
  consisting of explicit models for both the observation process and
  the underlying ``ecological process'' which describes the
  organization of individuals in space.

\item[(2)] Formal inference using both classical (frequentist,
  likelihood-based) and Bayesian methods. We often emphasize
  Bayesian analysis because this allows us to focus the technical
  formulation of models, and spatial capture-recapture is mainly
  concerned with modeling random effects and estimating functions of
  random effects. However, we also explore likelihood methods using existing
  software such as the R package SECR \citep{efford:2011}, as well as
  development of custom solutions along the way.

\item[(3)] In developing Bayesian analyses of SCR models, we emphasize
  the use of the BUGS language for describing models. The BUGS
  language emphasizes the syntactic description of the essential
  assumptions of models in a special kind of pseudo-code language,
  which is used in software (WinBUGS, JAGS, OpenBUGS) to devise Markov
  chain Monte Carlo (MCMC) algorithms for Bayesian analysis of
  models. The BUGS language focuses your thinking on model development
  and lets you develop an understanding of models at the level of
  their basic assumptions and structure.  Despite our focus on
  describing models using the BUGS language, we also show readers how
  to devise their own MCMC algorithms for Bayesian analysis of SCR
  models, which can be convenient (even necessary) in some practical
  situations.

\item[(4)] Data augmentation -- dealing with the fact that population
  size, $N$, is unknown is a challenging technical problem in
  capture-recapture models. We confront this problem in almost every
  chapter of this book. To deal with it we use a technical device
  called {\it data augmentation} which is extremely useful for
  analysis of capture-recapture models that are specified
  ``conditional on $N$'' \citep{royle_etal:2007}.
\end{itemize}

Altogether, these different conceptual and methodological elements
provide for a formulation of SCR models that essentially renders them
as variations of generalized linear mixed models (GLMMs).This in a
sense makes them consistent with many important methodologies used in
ecology (e.g., see \citet{zuur_etal:2009, kery_etal:2010}), and
because of the connection with standard modeling concepts, we believe
that the material presented in this book can be understood and used by
most ecologists with some modeling experience. Our intent is to
provide a comprehensive resource for ecologists interested in
understanding and applying the SCR models to solve common problems
faced in the study of population dynamics. Although we aim to reach a
broad audience, at times we go into details that may only be of
interest to advanced practitioners who need to extend these models for
unique situations.  We hope that these advanced topics will not
discourage those new to these methods, but instead we believe this
material will allow readers to advance their understanding and become
less reliant on restrictive tools and software.

This book is not a book about Bayesian analysis, not a book about
hierarchical models, not a book about capture-recapture, and not about
programming in R. In a sense though, our book integrates elements of
all of these things into what we hope is a coherent package for
analyzing data from this enormous class of data collection methods
that produce spatially-explicit capture-recapture data.   As such, we
expect that people have a basic understanding of statistical models
and classical inference (What is frequentist inference? what is a
likelihood? Generalized linear model? Generalized linear mixed
model?),
{\bf R} programming,
 Bayesian analysis (what is s a prior distribution and a
posterior distribution?),
and maybe even a little bit
of Bayesian
computation (MCMC and perhaps the BUGS language).
The ideal candidate for reading this book has basic knowledge of these
topics. However, we do provide introductory chapters on the necessary
components which we hope can serve as a brief and cursory tutorial for
those who might have only limited technical knowledge, e.g., many
carnivore biologists who implement field sampling programs but do not
have extensive experience analyzing data.




\section{Hierarchical Models and Inference}

The term hierarchical modeling (or hierarchical model) has become
something of a buzzword over the last decade with hundreds of papers
published in ecological journals using that term.  So then, what
exactly is a hierarchical model, anyhow? Obviously, this term stems
from the root ``hierarchy'' which means:

\vspace{.1in}

{\flushleft
Definition: {\it hierarchy} (noun) -- a series of ordered groupings of people or things within a system;
}

\vspace{.1in}

In the case of a hierarchical model (hierarchical being the adjective
form of hierarchy), the ``things'' are probability distributions, and
they are ordered according to their conditional probability structure.
Thus, a hierarchical model is {\it an ordered series of models,
  ordered by their conditional probability structure}.

If we declare that the random variable $y = $ \# of times an
individual is encountered in a trap out of $K=10$ days has a
$\mbox{Bin}(10, p)$ distribution then this is but a single model and,
thus, not a hierarchical model. If, however, we declare that
\[
y \sim \mbox{Bin}(10,p)
\]
{\it and}
\[
p \sim \mbox{beta}(1,1)
\]
which is the same as the previous model but with a ``flat'' prior
distribution on $p$, then this is kind of a cheap pedestrian
hierarchical model according to our definition although it is barely
more interesting than the previous non-hierarchical model.
%% I think here in the intro you could remove this 'pedestrian hierarchical model'
%% For the readers who are not familiar (yet) with distributions and what a prior is, I think it would be mroe helpful
%% to only use the following example and explain briefly what the
%% p_{i}\sim \mbox{beta}(\mu, \tau) stands for
On the
other hand, suppose we have some meaningful group structure in this
problem such that the data arise by observing repeated Bernoulli
trials on {\it individuals}, e.g., they are eggs hatching from a
common nest (or parentage). So let $y_{i}$ be the outcomes for
individuals $i=1,2,...,N$ with
\[
y_{i} \sim \mbox{Bin}(K, p_{i})
\]
 and
\[
p_{i}\sim \mbox{beta}(\mu, \tau).
\]
Because of the meaningful group structure, this is a more interesting
hierarchical model. In fact, in the context of capture-recapture this
is a specific version of ``Model Mh'' (see Chapt. 3 and
\citet{dorazio_royle:2003}).  We should consider this a type of a
hierarchical model although we will make a further conceptual
distinction shortly that further dichotimizes the space of
hierarchical models.

A canonical hierarchical model in ecology is this
elemental model of species occurrence or distribution
\citep{mackenzie_etal:2002, tyre_etal:2003, kery:2011}:
\[
y_{i}|z_{i} \sim \mbox{Bin}(K,z_{i} \,  p)
\]
\[
z_{i} \sim \mbox{Bern}(\psi)
\]
where  $y_{i} = $ observation of presence/absence at a site $i$ and
$z_{i} = $ occurrence status ($z_{i}=1$ if a species occurs at  site
$i$ and $z_{i}=0$ if not).  This model has an important conceptual
distinction between the hierarchical model shown just previously
(Model Mh) and also other types of classical multi-level models such
as repeated measures on subjects, in that $z_{i}$ is an actual state
of nature. In that sense, $z$ is a random variable that is the outcome of a
``real'' process.   \citet{royle_dorazio:2008} used the term {\it
  explicit} hierarchical model to describe this type of model to
distinguish from hierarchical models ({\it implicit} hierarchical
models) where the latent variables don't
correspond to an actual state of nature -- but rather just soak up
variation that is unmodeled by explicit elements of the model.
At best, latent variables in such models
are a a surrogate for something of ecological relevance
(``time effects'', ``space effects'' etc.).


With these examples,
we expand on our definition of a hierarchical model as we will use it
in this book: \newline
{\flushleft {\bf Definition}: {\it Hierarchical Model}: A model with
  explicit component models that describe variation in the data due to
  (spatial/temporal) variation in {\it ecological process}, and due to
  {\it imperfect observation} of the process.
}



%\subsection{Anatomy of a hierarchical model}
%Interesting hierarchical models in ecology typically
%contain the following components:
%\begin{itemize}
%\item[{\bf 1.}] {\it Observations}, $y(s,t)$ -- ``data''
%\item[{\bf 2.}] {\it Observation model} $[y|z,\theta_1]$
%\item[{\bf 3.}] {\it State variable}, $z(s,t)$: outcome of ecological {\it process} of interest
%\item[{\bf 4.}] {\it Process model}  $[z|\theta_2]$
%\item[{\bf 5.}] {\it Parameters}, $\theta_1$, $\theta_2$, that govern
%  the observation and state processes
%\end{itemize}


\subsection{Spatial Capture-Recapture models as hierarchical models}

Most models considered in this book describe the encounter of
individuals conditional on the ``activity center'' of the individual,
which is a latent variable (i.e., unobserved random effect).  The
collection of these latent variables represents the outcome of an
ecological process describing how individuals distribute themselves
over the landscape. Moreover, how individuals are encountered in traps
is, in some cases, the result of a model governing movement.  As such,
these models are examples of hierarchical models that contain formal
model components representing both ecological process and also the
observation of that process. That is, they are explicit hierarchical
models \citep{royle_dorazio:2008} as opposed to implicit hierarchical
models.




\section{Characterization of SCR models}
\label{intro.sec.characterization}

For the purposes of this book, an SCR model is any ``individual
encounter model'' (not just ``capture-recapture''!) where auxiliary
spatial information is also obtained. To be more precise we could as
well use the term ``Spatial capture and/or recapture'' but that is
slightly unwieldy and, besides, it also abbreviates to SCR. The class
of SCR models includes traditional capture-recapture models with
auxiliary spatial information and even some
models that do not even require ``recapture'' (e.g., distance
sampling).  There is even a class of models (Chapt. \ref{chapt.scr-unmarked})
which don't require unique capture or
identification of individuals.

Conceptually SCR models involve a collection of random
variables, ${\bf s}$, ${\bf u}$ and $y$ where ${\bf s}$ are the
activity or home range centers, ${\bf u}$ is the location of the
individual at the time of sampling (i.e., where the observer records
the animal) which we think of as realizations from some movement
model, and $y$ is the ``response variable'' - what the observer
records. E.g., $y=1$ means ``detected'' and $y=0$ means ``not
detected'' but many other types of responses are possible.
A broad class of models for estimating density are unified by a
hierarchical model involving explicit models for
animal home range centers ${\bf s}$, movement outcomes ${\bf u}$, and
encounter data $y$.  In some cases, we don't observe $y$ but rather
summaries of $y$, say $n(y)$, yet it might be convenient in such cases
to retain an explicit focus on $y$ in terms of model construction.
We thus introduce a sequence of models - a hierarchical model -
to relate these random variables and it goes something like this:
{\small
\begin{verbatim}
# NEED A graphic made out of this somehow
# possibly a Directed Acyclic Graph with some parameters,
# Fixed nodes, and stochastic nodes, might look cool.

Home range center    movement model   observation model  [data summarization]
   g(s)                  h(u|s)            f(y|u)	        n(y)
\end{verbatim}
}
Thus, models covered in this book all have distinct
characteristics related to the following decomposition as a
hierarchical model:
\[
f(y|{\bf u})h({\bf u}|{\bf s})g({\bf s}).
\]

Every model we talk about in this book has either all of these
components or a subset of them. Examples
of such models are:
\begin{enumerate}
\item[$\bullet$] Classical distance sampling
\item[$\bullet$] Spatial capture-recapture models with fixed arrays of traps
  \citep{efford:2004, borchers_efford:2008, royle_etal:2009ecol,
    royle_etal:2009jwm, gardner_etal:2010ecol}
\item[$\bullet$] Search-encounter models \citep{royle_young:2008, royle_etal:2011mee}.
\item[$\bullet$] Capture-recapture distance-sampling \citep{borchers_etal:1998}.
\end{enumerate}
In some classes of models, components for ${\bf u}$ and ${\bf s}$ will be confounded.
e.g., if ${\bf s}$ are uniform in space and ${\bf u}$ is
a random draw from some distribution centered at ${\bf s}$, then we might as
well define ${\bf u}^{*}=\Pr({\bf u})=\int_{s} [{\bf u}|{\bf s}][{\bf
  s}]ds$ which will itself by uniform
for reasonable choices of $[{\bf u}|{\bf s}]$.  Some examples
of typical spatial capture-recapture models and how
these various model components are manifest in specific cases is
described as follows:
\begin{itemize}
\item[1.] {\bf Distance sampling -- } The last 2 stages of the hierarchy
  are confounded (implicitly) and so analysis is based on the model
  $[y|u*] [u*]$. The ``process model'' is that of ``uniformity'': ${\bf u}^{*}
  \sim Unif({\cal S})$. Sometimes it is argued that distance sampling
  estimators are ``pooling robust'' which is a way of saying they are
  (or may be)
  relatively insensitive to this assumption. That may be true, but the
  construction of distance sampling estimators makes explicit the
  uniformity assumption as a mathematical fact.

\item[2.] {\bf Spatial capture-recapture model with a fixed array of traps} --
SCR models appear to have little in common with distance sampling
because observations are made only at a pre-defined set of discrete
locations -- where traps are placed. However, the models are closely
related in terms of our hierarchical representation above\footnote{Really
they're kind of like point-count distance sampling where the identity
of individuals is preserved across point samples , and distance is a
latent variable. i.e., SCR-DS. I feel like this point should be
emphasized somehow. Here? Later?}
In SCR models based on fixed arrays,
we cannot estimate both
$\Pr(y=1|{\bf u})$ and $\Pr({\bf u}|{\bf s})$ -- the probability  that
an individual ``moves to ${\bf u}$'' cannot be seperated from the
probability that it is detected given that it moves to ${\bf u}$,
because of the fact that the observation locations are fixed by
design.
Formally, such SCR models confound $[y|{\bf u}]$  with $[{\bf
  u}|{\bf s}]$ so that the observation model arises as:
\[
 [y|{\bf s}] = \int_{u} [y|{\bf u}][{\bf u}|{\bf s}] du
\]
This confounding happens because SCR sampling is spatially biased -
restricted to a fixed pre-determined set of locations.

Conversely,
distance sampling confounds $[{\bf u}|{\bf s}][{\bf s}]$ because, essentially, there is
only a single realization of the encounter process.

It is probably
reasonable to assume that $\Pr(y=1|{\bf u})=1$ or at least it is locally
constant for most devices (e.g., cameras, etc..), and thus the
detection model will have the interpretation in terms of movement (see
chapter XXX.YY).

\item[3.] {\bf Search-encounter models -- } What we call
  ``search-encounter'' models \citep{royle_etal:2011mee}
  are kind of a hybrid model - combining features of SCR models and
  features of distance sampling. Like distance sampling they allow for
  encounters in continuous space which provide direct observations
  from $[{\bf u}|{\bf s}]$.
Thus, the
  hierarchical model is fully identified.

\item[4.] {\bf Capture-recapture/distance-sampling -- } See
  \citet{borchers_etal:1998}. As with the search-encounter models the
full hierarchical model is identified:
$[y|{\bf u}][{\bf u}|{\bf s}][{\bf s}]$ but the quantities don't
really mean the same thing as before.

To understand this, we expand the model to accommodate imperfect
measurements of ${\bf u}$. Let ${\bf u}_{obs}$ be an observation of
${\bf u}$ (i.e., made with error). A larger hiearchical model is this:
\[
[y|{\bf u}][{\bf u}_{obs}|{\bf u}][{\bf u}|{\bf s}][{\bf s}]
\]
If we make replicate ``instantaneous'' observations of location, then
information is provided about
 $[{\bf u}_{obs}|{\bf u}]$ (i.e., measurement error). However, in a normal
 distance sampling application, with instantaneous sampling, we don't
 learn anything about $[{\bf u}|{\bf s}]$,
in effect, we are again confounding $[{\bf u}|{\bf s}]$ and $[{\bf
  s}]$: ${\bf u}^{*} = \int_{s} [{\bf u}|{\bf s}][{\bf s}] ds$. So the CR-DS model focuses on:
\[
[y|{\bf u}^{*}][{\bf u}_{obs}|{\bf u}^{*}][{\bf u}^{*}].
\]
Structurally, this is the same basic model as the search-encounter
model notwithstanding (1) that it is usually talked about in terms of
repeated measures of distance instead of location and (2) the 2nd
component of the hierarchy is not movement (an ecological process) but
rather ``measurement error'' and (3) the third component is not a home
range center but rather a movement outcome (``instantaneous
location'').  Thus, while the models are structurally identically, the
meaning and interpretation of quantities are distinct.

%These are
%mostly all semantic and conceptual distinctions which are easy to
%define in a convenient table:
%\begin{table}[ht]
%\centering
%\title{What things mean in each model.}
%\begin{tabular}{c|cc}
%           &   Search encounter models     &  CR-DS  \\  \hline
%  $\sigma$    &  movement       &   measurement error  \\
% ${\bf s}$ & activity center & instaneous location \\
%\end{tabular}
%\end{table}
\end{itemize}

\subsection{Other elements of SCR models}

Broadly speaking we differentiate
between two situations: Sampling based on fixed arrays or sampling
based on ``search encounter'' methods. The former includes things like
camera traps, hair snares, mist nets and conventional traps. Fixed
arrays limit the observation location to pre-defined points, where
traps are located. Using such methods the model is a little simpler
because the ``movement process'' of individuals is confounded with the
``observation process''.
The 2nd type of model -- search encounter models -- typically
will allow locations in continuous space, possibly only restricted by
polygon boundaries \citep{royle_young:2008}.
Search-encounter data
usually allow for the separate modeling and estimation of movement
model parameters from encounter model parameters but not always,
depending on whether replication of the sampling is done.  The
classical distance sampling model with no replication (i.e., $t=1$) is a basic model
which confounds the two processes.


Depending on the type of device being considered, certain restrictions
on the observable variable are induced which suggest specific
probability models for the observable random variable, suggesting
either binomial, Poisson or multinomial (and possibly other)
observation models.
One type of a
device is what we think of as the classical ``camera trap'' and which
\citet{efford:2011} refers to as a ``proximity detector''. We can take
pictures of or detect any number of individuals and an individual can
be caught in any number of traps, and an arbitrary number of
times. Iid Bernoulli model is convenient but if you think the
re-encounters are valuable then you can have a frequency model.  Bear
hair snares are slightly different because you cannot differentiate
re-encounters.
The standard observation model that applies for ``single-catch''
\citep{efford_etal:2004} traps posits that individuals are encountered
in at most one trap per sample occasion and traps only hold one
individual.  Unfortunately we're really screwed in the single-catch
situation.
A ``multi-catch'' is like a mist-net or other things - individual is
captured and restrained but traps hold > 1 individual. In this case,
the observation model is a multinomial. There are
many variations on all of these models and new models.

\begin{comment}
other things I have trouble working in here:

state-space: continuous, discrete, polygons???

observation locations: continuous, discrete, mapped to a polygon, or
something else?

auxiliary information e.g., from telemetry?

\subsubsection{5. modeling distance}

An important part of the observation model of every SCR model is the
manner in which {\it distance} between individuals and observation
locations (``sampler'') enters into the model.
This depends on whether the sampling is done at a point, along a line,
or uniformly over some polygon.

Observation location $x$ is a point.  We record individual location at
point $x$ or at some other location $u$. Distance is either $||x-s||$
or $||x-u||$.

Observation location is a ``line'':
(i) if we note point on line {\it and} $u$ then radial distance $||x - u||$
(ii) We don't observe point on line we might use ``closest distance''
(standar distance sampling)
(iii) we may or may not observe ${\bf u}$.
(iv) we might only record ${\bf u}$ but not the point on the line. In
that case we could use closest-distance or ``total hazard''.

More often, distance samplers approximate this with the closest linear
distance to the line, i.e., $min_{x} ||u-x||$.  Someone told me that
this has been found to work better in practice but clearly it is not the
correct description of the observation mechanism (when $x$ is available).
An alternative is to use the cumulative hazard (from Buckland and Hayes paper)
where we sum the hazard from the start of the line up to point $x$.
\end{comment}





\section{Analysis of spatial capture-recapture models}



\begin{comment}
\subsection{Models don't have political views!}

Whereas hierarchical modeling is a conceptual and framework for
formulating models, the method of inference is independent of model
formulation. Hierarchical models can be analyzed by Bayesian and
non-Bayesian methods. A model is not Bayesian or frequentist -- what
you do to that model is Bayesian or frequentist!
\[
\bullet \mbox{"Hierarchical model"} \ne  \mbox{"Bayesian"}!!!
\]
Thus, analysis of hierarchical models is easily achieved using either
Bayesian or classical (likelihood, frequentist) methods. By
``analysis'' we mean any type of estimation, characterization of
uncertainty, prediction, model selection, or evaluation and we are not
dogmatic about our choice of inference methods. That said, we do
recognize a benefit of the Bayesian approach which is that it
emphasizes model construction and not the construction of
procedures. The Nobel prize\footnote{called something else besides
  Nobel, officially} winning econometrician Christopher Sims (Slides
from the Hotelling Lecture 6/29/2007 at Duke University - cite his
webpage) said it this way: ``Bayesian inference is a way of thinking,
not a basket of 'Methods''' Conversely, ecologistis that are subjected
to a classical statistical curriculum often have only a vague sense of
what the model is that any particular procedure is employing.  We
agree with Little \citet{little:2006}
that there should be more emphasis on understanding statistical
modeling, and less emphasis on statistical methods.  Toss the ``basket
of methods'' out the window and learn how to model!

\end{comment}

We rely strictly on principles and procedures of {\it parametric
  inference} in our analysis of hierarchical models in general and,
specifically, of spatial capture-recapture models. Parametric
inference is that in which we make explicit probability assumptions
about how the data were generated. Inference procedures are then
developed under the assumption that the model is truth, because formal
parametric inference procedures that we understand the joint
probability distribution of everything that is a realization of a
random variable. {\bf There is something missing in the previous sentence. require, maybe?}There are two popular flavors of parametric
inference: {\bf Classical inference}: The joint probability distribution
of observations is the {\bf likelihood}. We maximize it to obtain MLEs
and do other fun things to it. We evaluate procedures by thinking
about what would happen over replicate realizations of data to which
our procedures are applied.  {\bf Bayesian inference} is based on the
posterior distribution, which is the joint probability distribution of
the data and also parameters and possibly other quantities including
latent variables or random effects.

Because SCR models contain a collection of latent variables - random
effects -- a natural framework for classical analysis of the models is
based on integrated likelihood \citep{laird_ware:1982,berger_etal:1999}. That is, while
the observation model is conceptualized conditional on the random
effects (the locations of individuals), classical inference is
formally based on the likelihood constructed from the {\it marginal}
probability distribution of the observations (i.e., {\it
  unconditional} on the random effect). The random effects are removed
from the conditional likelihood by integration (which is accomplished
numerically in spatial capture-recapture models). This approach to
inference has been formalized in the context of SCR models by
\citet{borchers_efford:2008, efford:2011}, and implemented for some
classes of models in the software package {\bf DENSITY} \citep{efford:2004}
and the {\bf R} package \mbox{\tt secr} \citep{efford:2011}.

Bayesian analysis is another natural framework for the analysis of
models containing latent variables or random effects.  Under this
approach, analysis of the model is based on Monte Carlo simulation
from the posterior distribution, which is the product of the
conditional likelihood, the distribution of the random effects, and
perhaps other distributions.  This approach was developed by
\citet{royle_young:2008}, and was motivated by work focused on
modeling individual effects in capture-recapture models. In
particular, a convenient reparameterization of individual covariate
models can be obtained using a method known as data augmentation
\citep{royle_etal:2007}, see \citet{royle:2008} for an application to
classical individual covariate models in \citet{royle:2008}. The close
similarity between individual covariate models and spatial
capture-recapture models, with the individual's activity center ${\bf
  s}$ being the individual covariate, led to the application of the
data augmentation method described by \citet{royle_young:2008} and subsequent papers.

These two technical formulations (classical inference based on
integrated likelihood and Bayesian inference) both provide rigorous solutions to
the inference problems posed by spatial capture-recapture data.  There
are also minor distinctions to be aware of. For example,
as a technical matter, \citet{borchers_efford:2008} and related work, assume
a Poisson point process that is unconditional on $N$ whereas Royle and
Young (2008) and related work assume a binomial point process model
which is conditional on $N$.
%More importantly, Borchers and Efford
%develop the analysis in a way that is unconditional on the point
%process (which is removed from the conditional likelihood by
%integration).  Conversely, the analysis of \citet{royle_young:2008} is
%conditional on the underlying point process. As a technical matter,
%Bayesian analysis allows us to analyze the model that is conditional
%on the underlying point process and will otherwise have more
%flexibility - open populations, using telemetry data, etc.. as will be
%demonstrated in later chapters.

We tend to favor Bayesian inference for conceptual and philosphical
reasons but we also think that  integrated likelihood
for complex point process models may prove difficult. On the other
hand, we suspect that Bayesian
analysis by MCMC
of the model that is conditional on the underlying point process will
prove to be more versatile and generalizable for complex point process
models. We say this only
tentatively and throughout this book we are not exclusive in our views
of inference and use Bayesian and classical methods of inference
interchangeable and opportunistically in this book.
We don't want to get too much into the technical foundations of
Bayesian analysis because there are many good books now including
\citet{link_barker:2009}.  \citet{kery:2010, mccarthy:2007,
  king_etal:2009} and probably others by the time this book is
finished. That said,  Bayesian analysis is introduced at a
level required to get through this book in Chapter 2.


\subsection{Implementing hierarchical models}

In our experience, students in ecology and even many established
scientists simply cannot separate what they need to do from how to do
it.  They cannot distinguish clearly (either conceptually or actually)
the difference between the model for their data, and the actual
procedure of how to estimate parameters of that model, or make
predictions - ie., how to do the calculations. Sometimes this issue
raises itself in an email from some hapless grad student wondering
``what is the right statistical test for this type of data?''  In a
sense it is this view that drives our approach to developing elements
of this book.

In contemporary statistical ecology, models and methods are sometimes
obscured by named procedures often that are completely uninformative,
the technical details of which hide in obscurity in some black boxes
such as MARK, PRESENCE, DISTANCE, etc., known only by the few
specialist experts in the field. While it is sometimes convenient to
refer to a type or class of models by a name (logistic regression or
even ``model Mh'') in order to emphasize a broad concept or
methodological area, this is only useful if the fundamental
statistical and mathematical structure underlying that name is
clear. As such, we try to focus on model development and keep the
model development distinct from how to combine our data with the model
to produce estimates and so forth. We talk a lot about hypothetical
data we wish we could observe - complete data sets - data sets as if
$N$ were known, etc.. We talk about the model in precise terms and
then break down various ways for analyzing the model either using
likelihood methods or Bayesian methods or some black-box that does one
or the other.

To fit models, we rely heavily on the various implementations of the
{\bf BUGS} language including {\bf WinBUGS} \citep{lunn_etal:2000},
{\bf JAGS} \citep{plummer:2003}
 and {\bf OpenBUGS} \citep{thomas_etal:2006}. We really like
the {\bf BUGS} language, not merely  as a computational device for
fitting models but because it emphasizes
understanding of what the model is and fosters understanding how to
build models - as Kery XYZ XYZ says ``it frees the modeler in you.''  (direct
citation for this would be nice).  However, in addition to using the
{\bf BUGS} language and its various implementations, we also develop our own
{\bf R} code both for doing MCMC
and maximum likelihood, for which we also use the R
package \mbox{\tt secr} \citep{efford:2011}. In addition, we have
created an {\bf R} package to go with this book, \mbox{\tt scrbook},
which contains the data sets, {\bf R} and {\bf BUGS} scripts, and {\bf
  R} code for doing summary analyses, and some likelihood and MCMC
functions written solely in {\bf R}.


\subsection{SCR models as GLMMs}

Most SCR models covered in this book
have a formulation that closely resembles generalized linear
models (GLMs) in which the individual activity centers appear as a
random effect \citep{royle_etal:2009ecol, royle_gardner:2011}. That is,
they appear conceptually and structurally similar to so-called
generalized linear {\it mixed} models (GLMMs) which are pervasive in
the literature of many disciplines including ecology.  This actually
has, we think, some profound consequences both conceptually and
practically (in terms of implementation). For example, as a result,
SCR models should by conceptually accessible to practitioners with
some basic statistical understanding and experience.  And, we believe
that practitioners will have some flexibility in developing models
that fit their specific situation. Quite simply - given existing
software, it is easy for practitioners to specify models, even if they
lack the technical know-how to devise their own fitting routines.





\section{Summary: The Promise of Spatial Capture-Recapture}

Spatial capture-recapture models are an extension of ordinary
capture-recapture models to accommodate the spatial organization of
both individuals in a population and the observation mechanism (e.g.,
locations of traps).  They resolve problems which have been recognized
historically and for which various ad hoc solutions have been
suggested: heterogeneity in encounter probability due to the spatial
organization of individuals relative to traps, the ability to model
trap-level effects on encounter, and that a
well-defined sample area does not exist in most studies, and thus
estimates of $N$ using ordinary capture-recapture models cannot be
related directly to density.

However, SCR models are not merely an extension of technique but
rather they represent an extention in a much more
profound way in that they make ecological processes explicit in the
model -- processes of spatial organization of individuals, movement
and space-usage of individuals. While capture-recapture models have
existed for decades this is a completely new element of
closed capture-recapture models.
This is so profoundly important because
ecological scientists study elements of ecological theory using
observational data that exhibits various biases relating to the
observation mechanisms employed. In the context of capture-recapture,
we observe individual encounter history data from which we can use SCR
models to infer where individual live, how they organize themselves in
space and move around in space and how they interact with other
individuals.  Moreover, SCR models show great promise in their ability
to integrate explicit ecological theories directly into the models so
that we can directly test hypotheses about either space usage (e.g.,
Chapter XYZ) or movement (Chapter. XYZ) or the distribution of
individuals in space (Chapter XYZ). We imagine that in the near future
SCR models will include point process models that allow for
interactions among individuals such as inhibition or  clustering.

Thus, SCR models are capture-recapture models that enable ecologists
to explicitly integrate biological context and theory with encounter
history data, which is something that has always been the focus of
``open population'' models but never, until very recently, has been
considered formally in closed population models. We therefore believe
that SCR models will enable ecologists to test theories of space usage
and environmental effects, social behavior and other important
theories.


In chapter 2 we provide the basic analysis tools to understand and
analyze SCR models - namely GLMs with random effects, and their
analysis in R and WinBUGS.  Because SCR models represent extensions of
basic closed population models, we cover ordinary closed population
models in chapter 3 wherein, along with chapter 4, we will see that
SCR models are a type of individual covariate model, which are
conceptual and technical intermediates between Model Mh and classical
individual covariate models.  In subsequent chapters we will cover a
bunch of different types of SCR models related to the type of
encounter process - e.g., type of trap - and also different
embellishments of the basic model structure as alluded to in section
XYZ above.  We will consider many different extensions of SCR models
to accommodate covariates on encounter probability, and density. We
also consider important practical extensions such as SCR for open
populations (Chapter xyz), combining SCR data with auxiliary
information from telemetry (chapter XYZ) and multiple encounter
methods (chapter XYZ).
