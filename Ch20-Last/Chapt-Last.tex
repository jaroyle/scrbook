\chapter{
2012: A Spatial Capture-Recapture Odyssey
}

\markboth{The End}{}
\label{chapt.final}

\vspace{0.3cm}






\begin{figure}[h!]
\centering
\includegraphics[height=3.5in]{Ch20-Last/fisher.jpg}
\caption{
Fisher assaulting tree \# 8-12.
{\it Credit: NYSDEC (New York State Department of Environmental Conservation),
A Fuller/NYSDEC camera trap and hair snare study of fishers in
southern NY}
}
\label{last.fig.fisher}
\end{figure}

\begin{figure}[h!]
\centering
\includegraphics[height=3.5in]{Ch20-Last/weasel.jpg}
\caption{
A weasel taking bait on a hair snare, A. Fuller southern NY fisher study
{\it Photo credit: Marty DeLong}.
%These are weasels. Cute, blood-thirsty weasels.
%weasels wobble, but they don't fall over
}
\label{last.fig.weasels}
\end{figure}



\begin{figure}[h!]
\centering
%\includegraphics[height=3.5in]{Ch20-Last/lynx.jpg}
\includegraphics[width=\textwidth]{Ch20-Last/lynx.jpg}
\caption{
Canada Lynx, ear-tagged and radio collared, producing high quality
data in the name of science.
{\it Credit: A Fuller, Cornell University} }
\label{last.fig.lynx}
\end{figure}

You've finally made it to the last chapter and we realize it's been a
long journey to get here. Congratulations! %(congratulations)!
We hope this book has provided you with many ideas on how to conduct
ecological studies, address specific questions that were previously
thought difficult or impossible to answer, and given you a solid
foundation for carrying out SCR analyses
%on your own
using either Bayesian or classical methods of statistical inference.
However, we believe this journey is only just beginning, and we leave
you now with a few thoughts on what we see as the future of SCR
methods.

Let us first briefly consider how we got here. Over a century ago,
capture-recapture methods were first being developed and the study of
populations including spatially varying density was being introduced
by Pierre-Simon Laplace and others in France around 1786 (this was of
course regarding human population demography, but still, the
foundation of how we would go on to study animal populations was being
laid out then and there).  The Lincoln-Petersen method was articulated
by
% XX 'in'?
the 1930s and development of capture-recapture models began to grow
rapidly %more and more
starting in the 1950s.  Soon, capture-recapture methods had become a
cornerstone of ecological and wildlife modeling and analysis. And now,
spurred on by the advent of new non-invasive technologies like DNA
sampling, camera trapping, acoustic sampling, and other methods,
capture-recapture is more relevant and widely used than ever before
(see also Sec. \ref{last.sec.growth}). These new survey methods allow
researchers to use capture-recapture for species that could not be
studied efficiently even a few years ago, especially those that are
difficult to capture or handle including most felids
(Fig. \ref{last.fig.lynx}), bears,
%species,
mustelids such as
mink,
fishers (Fig. \ref{last.fig.fisher}),
 weasels (Fig. \ref{last.fig.weasels}),
and many other species.

With these new sampling techniques, like many commonly used
capture-recapture sampling methods,
%auxiliary
spatial information about location of capture is
collected. %As such, c
Classical capture-recapture models ignore this information, and in
doing so fail to provide a formal method for modeling spatial
variation in density and encounter
probability. %are usually not the right tool for the job of analyzing
             %such data and thus arose
It was these deficiencies that motivated %the need to develop
%models that can directly deal with the
%spatial context of capture-recapture studies.
%The result, as you have seen in this book, was
the development of SCR models (starting around
2003 - 2004).

%So what does the future look like?
We have seen a great increase in the number of papers that use or cite SCR models, and to
articulate quantify this growth,
%the point of where we see SCR usage going in the future,
we did a Google Scholar search on
March 6, 2013 using the terms:
\begin{small}
\begin{verbatim}
``spatial capture recapture'' OR ``spatially explicit capture recapture''
\end{verbatim}
\end{small}
The results from this literature search are shown
 in Table~\ref{last.tab.cites}.
We see the number of citations involving SCR rapidly increasing,
with growth in citation counts after 2004 fueled by publication of
\citet{efford:2004} and the release of the software DENSITY
\citep{efford_etal:2004}. In 2012 there were 84 articles published and
27 through the first 9 weeks of 2013.
%With 274 papers appearing since 2002,
The results, we think, suggest a bright future
for the development and application of spatial capture-recapture
models. Most (but not all) of these papers are about the type of SCR models
discussed in this book, although a handful had to
do with other types of spatial analysis as related to
capture-recapture models.

\begin{table}[ht]
\caption{Google Scholar citations by year based on a search of
%\mbox
{\tt ``spatial capture recapture'' OR ``spatially explicit
capture recapture''} conducted on March 6, 2013. The estimated finite
rate of increase for this population of papers was 33.4\%.
}
\begin{tabular}{lcl} \hline \hline
Time period & Cumulative cites & Cites in year previous \\ \hline
since 2002 & 274 cites & \\
since 2003 & 274 cites &0 articles published in 2002 \\
since 2004 & 271 cites &3 articles published in 2003 \\
since 2005 & 269 cites &2 articles published in 2004 \\
since 2006 & 264 cites &5 articles published in 2005 \\
since 2007 & 261 cites &3 articles published in 2006 \\
since 2008 & 253 cites &8 articles published in 2007 \\
since 2009 & 242 cites &11 articles published in 2008 \\
since 2010 & 222 cites &20 articles published in 2009 \\
since 2011 & 176 cites &46 articles published in 2010 \\
since 2012 & 111 cites &65 articles published in 2011 \\
since 2013 & 27 cites &84 articles published in 2012 \\
& &27 published so far in 2013, since March 6
\\ \hline
\end{tabular}
\label{last.tab.cites}
\end{table}

We believe that use and growth of SCR modeling in conservation biology,
management, wildlife, fisheries, and many other disciplines that we place under
the general umbrella of ecology will only continue.
This prediction is based our belief that SCR provides a robust
framework for
studying spatial and temporal variation in ecological processes while acknowledging the fact
that these processes are almost always
observed imperfectly.
% that by explicitly linking the space occupied by individuals with the
%spatial location or region of sampling, SCR models
%provide a holistic framework for answering ecological questions
%related to the spatial structure of populations. % -- movement, space
%usage, spatial variation in density, landscape connectivity, and other
%things.
%And now, after laying out how we got here and where we think SCR models are
%going, we will discuss some of the research areas we see as forthcoming.
%In this book we summarized, synthesized and extended recent
%developments of spatial capture-recapture models.
The the ``big idea'' of SCR,
if you could distill the whole thing into one idea, is based on
extending closed population models by augmenting them with a point
process model that describes the distribution of individuals
\citep{efford:2004} in space.
% As a conceptual matter then, the
%underlying point locations are regarded as an individual covariate in
%the encounter part of the capture-recapture model.
In a sense, that is
really all there is to it. But the relevance is much bigger and more
profound because, once we have made space explicit in the model,
we can think about building population models that embody explicit
spatial processes.

We covered many ecological processes that can be studied using SCR,
such as landscape connectivity, resource
selection, and spatial variation in density. These are all by
themselves profound extensions of the basic capture-recapture method,
and they broaden and expand the relevance and utility of
capture-recapture for studying animal populations.
Although we filled almost 600 book pages (mostly) with SCR methods,
there remains much to be done in the continued development of SCR
models. In the following section, we highlight some emerging topics that show promise or might be in
need of further development. Finally, we end with a few remaining
thoughts on the use of SCR models in the future.



\section{Emerging Topics}

% XXX RC: somewhat redundant with previous paragraph
In this book, we provided an overview and synthesis of
capture-recapture methods as known to us around the end of 2012. There
are many emerging topics which we have not covered either because of
lack of technical knowledge, lack of time for satisfactory
development, or lack of a good framework for implementation. Here we
present some of those topics. This is not a complete list by any means,
just a subset of topics that we or our colleagues are currently working on, or that we think
might make good PhD, Masters or other research projects.


\subsection{Modeling territoriality}
\label{last.sec.ipp}

% XXX RC: I would switch to the phrase "territory center" here because
% a territory is defined as the defended portion of a home range, and
% anal people love to pick on this point.
In currently developing work, \citet{reich_etal:2012} propose a model
that accounts for spatial variation in %home range
density and
potential interactions between individuals' activity centers.
%home ranges.
Under this model,
%lets
the activity centers follow an inhomogeneous Strauss process
\citep{strauss:1975}, which %allows for spatial variation in the home
%range intensity, and
includes a parameter that determines the strength
of repulsion between territory centers. %home ranges.
The idea is based on the notion
that territorial species would have well defined (and defended)
territories
%home ranges
and thus activity centers may be more regular on the landscape
than predicted by a homogeneous point process.
%as individuals would avoid one another. % and/or home ranges would be
%less likely to overlap.
%The development of this work includes a
A simulation study demonstrated %, where it was found
that properly accounting for
interactions between individuals can %provide a
substantially improvement
in %estimating
population size estimates in terms of bias and precision relative to
the usual independence model. %And thus far, for simulated data
%generated with interaction, the usual independence model has a
%significant bias for the population size, and generally has larger
%uncertainty for the population size than the proposed Strauss process
%model.

While the Strauss model is intuitive and shows great potential, it
presents computational challenges. The first challenge is that the
likelihood includes a high-dimensional integral that has no closed
form. To address this issue, \citet{reich_etal:2012} developed an
approximation to the Strauss likelihood which allows for posterior
sampling, extending related work for categorical Markov random fields
\citep{green_richardson:2002,smith_smith:2006}. The second challenge
is that $N$ is treated as an unknown parameter to be updated and hence
$N$ varies and so does the dimension of the likelihood, and thus the
posterior.
In this case, the dimension-changing problem can be
overcome by using an auxiliary variable scheme in the Markov chain
Monte Carlo algorithm.
% XX RS: Does that mean you don't use DA?
While the results from an initial analysis of simulated data verifies
that this computational approach leads to reliable inference, there
are still many areas to be explored in using the Strauss model and
other models of clustering.


\subsection{Combining data from different surveys}

In some instances, researchers apply different survey techniques to
the population of interest, because they yield complementary
information. For example, camera trapping is the prime tool for
estimating population size/density and other demographic parameters
for uniquely marked species, while genetic surveys can yield additional
information on the genetic diversity and health of a population that
cannot be studied using camera traps. At the same time, genetic surveys,
when samples are analyzed to the individual level, also yield spatial
capture recapture data (see Chapt. \ref{chapt.search-encounter}). In
this situation, we have two data sets at hand that carry information
on animal density, and we should be able to get more precise estimates
of density if we combine these two data sets into a single SCR model.

\citet{gopalaswamy_etal:2012mee} developed two approaches to combining
data from different survey types. In the first case, both surveys are
carried out at the same time, so that we can assume that they both
sample the same -- closed -- animal population, i.e., there are no
possible changes in population density between the two surveys. For
camera trapping and genetic surveys, we cannot match records of
individuals between the two data sets. However, models for the
distinct sample methods may share some parameters (e.g., $\sigma$ of
the encounter probability model) and, if the studies were conducted
simultaneously, they share a common population size $N$.

A second approach of using information from one survey in the analysis
of a second survey (that maybe does not yield quite as much data as
the primary survey) is by analyzing the %your
primary data set alone, then
taking the posterior distribution of a parameter both surveys share
and using it as an informative prior distribution in the analysis of
the second data set. \citet{gopalaswamy_etal:2012mee} refer to this as
the stepwise approach, and they implemented this approach by equating
the mean and variance of the posterior distribution of $\psi$ and
$\sigma$ from the photographic survey to the mean and variance of a
beta and a gamma prior for these parameters, respectively, for the
genetic survey. The authors found that this approach produced almost
identical density estimates compared to the combined model approach
described above.

In summary, no matter which approach is chosen, combining data across
surveys can help researchers %to
obtain more precise population size or
density
estimates, which is especially valuable when dealing with rare and
elusive species like big cats that almost always will produce sparse
individual data sets.
The paper by \citet{gopalaswamy_etal:2012mee} considers the
situation where we have two SCR data sets, but we can imagine
combining SCR data with other sources of information, such as
telemetry data (see Chapt. \ref{chapt.partialID} and
Chapt. \ref{chapt.rsf} for examples), and possibly opportunistic
observations, although to our knowledge this latter issue has not been
tackled in the context of SCR, yet.


%\subsection{Imperfect identification of individuals}
\subsection{Misidentification}

Imperfect identification of individuals can happen in a variety of
ways. In genetic surveys there is usually some probability of
misidentification %of individuals
due to genotyping error
(e.g. \citet{lukacs_burnham:2005}). In camera trap survey a different
type of imperfect identification can occur when only the only one
flank of an animal is recorded in a detection event and %that imagine
cannot be matched to any of the individuals identified by both
flanks.
% XXX RC: Wait, do we mean "cannot be matched to other individuals
% identified by a just one flank"? If an animals is identified by both
% flanks, then a picture of just 1 side should be good enough right?
In that case, we can match single-flank pictures with the same
side flank pictures, but not with opposite side flank pictures and
thus cannot construct definite encounter histories for these
single-flank individuals (a right flank and a left flank picture could
be the same individual, or could be from two distinct
individuals). Finally, in Chapt. \ref{chapt.partialID}, in the context
of mark-resight models, we discussed the case where individuals can
either not definitely be identified as marked or not -- a violation of
a basic mark-resight assumption, and developed an approach to dealing
with the situation where we can always tell if an animal is marked or
not, but we are not always able to ascertain its individual identity.

In non-spatial capture recapture some efforts have been made to
formally deal with misidentification. \citet{stevick_etal:2001}
address this problem by double-sampling to derive an error rate for
genetic identification, and then including this error rate as a known
constant into a Lincoln-Petersen estimator of
abundance. \citet{lukacs_burnham:2005} develop an approach that
includes an additional parameter in the model -- the probability of a
genotype being identified correctly, which is estimated as part of the
model likelihood. \citet{link_etal:2010} developed an approach towards
solving the same problem implemented in a Bayesian framework that
relaxes some of the assumptions of the initial approach.
\citet{yoshizaki_etal:2009} deal with misidentification from camera
trap pictures due to evolving marks (i.e., natural marks that change
over time, such as scars). This situation is different from the
genotyping error one. Here, a change in marks creates a supposedly
`new' individual that can be recaptured several times, while the
original individual is never captured again (its mark is no longer in
the population). In contrast, in genotyping error it is assumed that
misidentification creates a `new' individual that is never observed
again, because each error leads to a new unique
genotype. \citet{yoshizaki_etal:2009} approach this situation
similarly, by including a parameter describing the probability of
correctly identifying an individual upon recapture (the parameter can
also be interpreted as the probability that a mark does not change
between capture occasions). Because of the dependencies between true
and false detection histories (when a `new' individual is created, the
`real' one can no longer be recaptured), the standard multinomial
approach to coming up with a model likelihood does not work and
implementing the model in a maximum likelihood framework is
difficult. The authors instead demonstrate an implementation of the
model based on minimizing a function of the squared differences
between the observed and expected frequencies of the observed capture
histories.

To our knowledge no attempts have been made to deal with
misidentification in an SCR framework. While all of the mis-ID cases
described above require distinct approaches, we believe that there is
one unifying theme to all of them: the capture locations of the %un or
potentially mis-identified records
should be informative about identity.
%and the resulting probabilities of
%belonging to certain individuals conditional on their activity
%centers.
For example, a right flank and a left flank camera trap
picture that are taken at two neighboring camera traps
should be %are
more likely
to belong to the same individual that a right and a left flank picture
taken at cameras located at opposing ends of the trap array, especially if
animal movement is smaller than the extent of the trap array. %; an
%unidentified record of a marked individual in a mark-resight survey is
%more likely to belong to a marked individual whose activity center is
%close by, than to an individual whose activity center is located far
%away; and so forth.
%Formally developing misidentification models
%should be a focus for future SCR model development.
SCR models provide a natural way of using this additional information
to reduce the uncertainty arising from misidentification.

\begin{comment}
\subsection{Three dimensional space}

Throughout this book we have treated space as
two-dimensional, meaning that activity centers are assumed to occur on
the real plane. This approximation of reality is reasonable for many
terrestrial species, but aquatic organisms, especially marine animals
move about in three-dimensional space. Treating space as
three-dimensional could also conceivably be useful in studies of
flying organisms, aquatic organisms,
or species that use multiple strata of tall forests; however, we
suspect that two-dimensional models of space should suffice in such
contexts. Regardless, a three-dimensional view of space requires that
activity centers $\bf s_i$ are indexed by
$x,y,z$ coordinates. In theory, this presents no problem whatsoever. In
practice, estimation based on integrated likelihood methods must
involve a three-dimensional integration. This will clearly be more
computationally demanding, but it should be possible using packages
such as {\tt R2Cuba}.
\end{comment}

\subsection{Gregarious species}

One of the key assumptions of the SCR models that we described
throughout this book is that the activity centers are independent of
one another, but this assumption will be violated for %There are good biological reasons why this should not be
%the case, and one of those is when
species that associate in pairs, family groups, or any other type of
aggregation. %with one another
%as a pair, or family group.
%Even species regarded as solitary often join family
%groups for some portion of their life cycle.
However, we believe that general
models %for non-independence of activity centers
can be developed for
use in studies of gregarious species.
%(see Sec. \ref{last.sec.ipp} above).

The two issues that must be addressed are that (1)
%There are two consequences of individuals that exist as a pair or
%group. For one, the
detections are not independent -- a trap that
catches one of the individuals is likely to capture others in the
group, and (2)
% XX RS: Didn#t Robin also show that this kind of clustering doesn't really have an effect on estimates, at least for family groups up to some size that I do't remember...
%The other consequence is that
the activity centers ${\bf s}_{i}$
should appear clustered or, in fact, completely redundant in some
cases. A possible way to account for this is to change our definition
of ${\bf s}_i$ from the location of an individual's activity center,
to the location of a group's activity center
\citep{russell_etal:2012}. Ideally, to accommodate unknown group size,
the SCR model would be expanded to include a model component for group size,
so that formal estimation of both group density and group size would
be possible.
% XXX RC: Could mention Thomas process or other clustered point
% process models here
%could we add something like: similar to density estimation
%in distance sampling when groups are included??
%% hmm, how exactly do they do this in distance sampling? They use a
%% covariate in a Huggins-Alho type of model, I don't think we would
%% do it that way.


% XXXX RC: I commented this out b/c it is so brief doesn't include any
% specific ideas
\begin{comment}
  \subsection{Design}

  In Chapt. \ref{chapt.design}, we briefly introduced %you to
  concepts of model-based spatial design \citep{muller:2007}, and we
  feel that these ideas extend readily to the design of
  capture-recapture studies.  Clearly more work can be done on this
  problem and we think at some point not too far into the future,
  there will be flexible platforms for building capture-recapture
  sampling plans that are optimal (or nearly optimal) under specific
  models for estimating density.

  A number of specific design problems require further
  investigation. These include the design of studies for sampling
  large landscapes, when uniform coverage with traps cannot be
  achieved, sampling design in the context of modeling spatial
  variation in density, and the effect on design of having telemetry
  or other auxiliary data.
\end{comment}

\begin{comment}
\subsection{New models for passive detectors}
%should we just add his abstract to the bib and cite it?  I included it below
%Also maybe we should say his passive detector stuff with the uncertain recapts might be
%somehow combined with the unmarked stuff of richards? (i might be talking trash here)
At the 2012 ISEC, Dr. David Borchers talked about a number of interesting
developments and applications of SCR based on passive detector
arrays. Passive detectors are said to be devices such as microphones,
human observers, or camera-traps.  These devices not only collect spatial
capture recapture data, but auxiliary information such as signal strength or
exact time of detection or multiple observers ``triangulating'' on a source.
 Including such information into SCR models appears to
show great promise based on simulations and applications that have been conducted
so far.  Dr. Borchers does also suggest that there are costs associated to these
data types which include uncertainty in determining if detections are new captures
or recaptures.

%% Andy sez: I just commented the whole thing out.  That said, I think
%% because of the importance of the technology, maybe it should be in
%% here. Maybe the above paragraph but with a sentence emphasizing the technology?
\end{comment}

\begin{comment}
Here's the abstract:

Spatially Explicit Capture-Recapture with Passive Detectors

David Borchers 1, Darren Kidney1, Len Thomas1, Tiago Marques2, Greg Distiller1
1University of St Andrews, St Andrews, Fife, UK, 2University of Lisbon, Lisbon, Portugal

Spatially explicit capture-recapture (SECR) methods are seeing increasing use, and one particular area in which we anticipate growth is the use of SECR methods with passive detectors (which do not physically catch animals), such as microphones, camera-traps or human observers. Passive detector arrays can be much more efficient than trap arrays, partly because they can gather data at low cost but partly also because they can generate supplementary data (in addition to capture histories) that is not available from trap arrays. Such data include exact times of detection, received signal strengths, and various kinds of imperfect observations of the locations of detected animals. We develop a general SECR methodology for inclusion of such data, and by simulation and application to a variety of real datasets, we demonstrate the sometimes very substantial benefits of their inclusion. The benefits do not always come without costs, and primary among these is loss of fidelity of recapture information: with passive detectors it is usually more difficult to tell with certainty whether detections are new captures or a recaptures. We discuss methods for dealing with this, and illustrate one method by application to a capture-recapture survey dataset.

\end{comment}

\subsection{Single Catch Traps}

In Chapt. \ref{chapt.poisson-mn} we covered %talked about
multinomial models in
which an individual's probability of being captured in a trap is
independent of all other individuals.
%encounter of individuals is independent for all
%individuals.
This is the multi-catch type of device in which traps
never fill-up, but an individual can only be caught in one trap. We
suggested (following \citet{efford_etal:2009euring}) that the
multi-catch, independent multinomial model, could be used for ``single
catch'' traps (traps that hold a single individual or ``fill up'') and
that bias associated with mis-specifying the model would be low under
certain conditions (i.e., when the proportion of occupied traps is
low).

As discussed in Chapt. \ref{chapt.poisson-mn},
Sec. \label{poisson-mn.sec.singlecatch},
we recognize that the {\it time}, or order, of capture of an
individual in any trapping interval will affect the encounter
probability of subsequently captured individuals. Thus if
the order of capture was known, then this information could be
used to write the likelihood exactly. In practice, the order of capture
is almost never known, but %we do think that if you could
it should be possible regard capture order as a latent variable and consider all
possible orderings.
%, then you would have a close approximation to the likelihood. All
%possible orderings of capture
This would be %quite
computationally intense and so we are working on a solution that selects
an arbitrary ordering of the captures as a practical approximation to the
single-catch process. This will hopefully lead to a formal model
for the the single catch trap problem.
% XX RC: Beth, do you have an "in prep" paper you could cite?

\subsection{Model Fit and Selection}

Evaluation of model adequacy or ``fit'' is an important part of any
applied analysis. In Chapt. \ref{chapt.gof}, we offered up a number of
ideas based on standard considerations and adapted and applied them
to SCR models. However, these ideas have not been widely applied, or
evaluated, and much work needs to be done. In particular, some basic
analysis of their power under meaningful alternatives would increase
their relevance and possibly lead to insights for devising better
methods. This applies to both Bayesian and likelihood-based methods,
for which there are even fewer published applications of
goodness-of-fit assessment.

Similarly, we discussed model selection strategies using more-or-less
conventional ideas based on AIC/DIC, and model indicator variables
using the \citet{kuo_mallick:1998} method. Calibration of these
methods under alternatives is needed, along with some analysis of
sensitivity to density estimates to misspecification of certain model
components.


%\subsection{Dynamics}
\subsection{Explicit movement models}


%BETH: do you want me to put in more from the open chapter?  I basically only
%speculate here about what you could do and what would be cool

We briefly discussed the topics of dispersal,
transience, and migration in Chapts. \ref{chapt.searchencounter} and \ref{chapt.open} and sketched out
a few ideas that allow for dynamics related to movement or migration.
%%to be dynamic. %other chapters? RSF?
Temporary emigration and transiency are two topics where
a significant amount of work has been accomplished in non-spatial closed and open capture-recapture
models \citep{kendall_etal:1997, pradel_hines:1997, hines_etal:2003,
clavel_etal:2008, gilroy_etal:2012,chandler_etal:2011}.
Additionally, models for dispersal (e.g., \citet{clobert_etal:2001,
ovaskainen:2004, ovaskainen_etal:2008} and
and other forms of movement (e.g., \cite{jonsen_etal:2005, johnson_etal:2008b,
mcclintock_etal:2012}) have received quite a bit of attention and development in
ecology.

With the formulation of SCR models, the framework is already in place to provide
a formal integration of the movement dynamics governing the processes
of dispersal, emigration, and transiency. Movement modeling can be done
in some cases within a closed
population model, but more commonly is carried out within an open population model.
The development of SCR models for open populations \citep{gardner_etal:2012} now sets
the stage to incorporate
movement estimation directly with population demography. What remains now is for the
movement models
of dispersal, temporary emigration, and transiency to be incorporated into SCR models.
The standard SCR study design allows researchers to start better understanding movement patterns of
the species of interest.  However, models that can capture longer distance or broader scale movements,
may require incorporating telemetry data, multi sampling arrays on the landscape, or other
methods to better understand the movement and distance patterns of the study species.

Dispersal and emigration can also be related to the life stage of an individual in
a certain population.  Ultimately, combining multi-state models where the states are
age classes or breeding status categories, with open population SCR models and explicitly
modeling patterns of movement like dispersal as a function of state (e.g., age class) is
another area wide open for development.  The main restriction on these models in terms
of application has mostly been the lack of data, such a model is data hungry and many studies
will struggle to acquire enough data.


\begin{comment}
\subsection{Miscellaneous topics}

Models for unmarked or partially marked individuals integrated with RSF data from telemetry

Occupancy and counts data + SCR data (AOAS and Sollmann et al.)

Spatial genetics -- can use SCR to study gene flow, related things....

SCR on dendritic networks (streams and trails).
%Beth's random thought below
A number of sampling techniques involves linear or dendritic networks such as
trails or stream systems.  SCR models can easily be applied to these systems.
If the animal is constrained to the system, such is the case with fish in stream,
then care must be taken to use a model that is appropriate for non-Euclidean distances
\citep{peterson_etal:2013}.
\end{comment}


% XXX RC: Or maybe: \section{Final Remarks}
\section{The Future of SCR}

Everything in ecology is spatial, and now so too are capture-recapture
models.
Historically the main use of capture-recapture was to obtain population
size estimates, but
SCR models move the problem from one of estimation to one of
formalizing hypotheses about spatial and temporal variation in
ecological processes. % such %allow you to address basic and applied questions of
%population ecology from individual encounter history data -- problems
%having to do with movement,
These processes include resource selection, landscape connectivity, and
how individuals organize themselves in space. SCR models allow for
this formalization by borrowing methods from spatial statistics,
%thought of as belonging to a larger class of spatial models used in
%ecology, % is all done using
%advanced spatial models,
but unlike many spatial models,
% used in ecology,
SCR models include key demographic parameters such as density
and survival and thus allow for mechanistic rather than just
phenomenological descriptions of natural variation.
For these reasons, we believe SCR models will continue to be developed and
extended, and their use will continue to grow.
% XXX RC: I moved the growth rate stuff back up
%While there are not
%a huge number of applications of SCR models right now, we showed that
%their use has expanded rapidly based on simple citation counts.
%Considering the number of citations as an ordinary population (without
%spatial context!), we used these data to make a projection of the
%number of published articles that involve SCR methods into the future.
%We fitted an exponential growth curve to these data and estimate the
%annual rate of growth to be 33.4\%, accounting for the partial year of
%data observed in 2013.  This shows the great potential for SCR models in
%ecology in the near future, although such growth might not be sustainable
%over the long term (we did not
%consider important notions like carrying capacity).
%We envision that SCR models will be useful in helping
%ecologists ``do science'' by developing explicit models of spatial
%processes, space usage, connectivity, etc., often through the use of ordinary,
%inexpensive, and
%easy to obtain individual encounter history data.
However, much work still needs to
be done to improve computational feasiability, to address many
technical or methodological holes in the literature,
%(see previous sections),
and to make these methods more accessible to practitioners.
% XXX RC: Am I bordering on hyperbole?
We look forward to these developments and hope that this book will
help catalize further exploration on this nascent odyssey.
% This book is
%just the beginning and there are many frontiers still
%left to explore on this odyssey. % -- we expect to see more books, \R~packages,
%and papers in the next 5 years on SCR, beyond that, we look forward to seeing
%the new directions that will arise from the topics covered here.

% XXXX RC: I would get rid of the predictions.
% Andy sez: is this too pretentious? haha. Probably is....

%We used the model to project the number of
%publications that involve SCR models into the future
%(Fig. \ref{last.fig.expgrowth}). This includes the observed data
%(solid circles) (omitting the partial year 2013) and projections up to
%2020. We conclude that the future of SCR is bright, with 800 or 900
%publications predicted to occur in 2020. While this is a long time in
%the future to be predicting based on an exponential growth model, we
%can do some model checking along the way, as the prediction of 85 SCR
%publications in 2013 and 119 in 2014 can be assessed in short order,
%giving us some short-term feed-back on the state of the system. Of
%course, this model is missing some important things. One of those is
%carrying capacity. The model predicts $>800$ publications in 2020, but
%there may not be that much capacity to do capture-recapture studies!


%\begin{figure}[ht]
%\centering
%\includegraphics[width=4in,height=4in]{Ch20-Last/exp_growth.png}
%\caption{
%exponential growth projection of population size of published articles
%that involve SCR models.
%}
%\label{last.fig.expgrowth}
%\end{figure}













