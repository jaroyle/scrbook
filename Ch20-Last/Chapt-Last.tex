


\chapter{
 2012: A Spatial Capture-Recapture Odyssey
 }

\markboth{The End}{}
\label{chapt.final}

\vspace{0.3cm}


\vspace{2in}

Capture recapture methods have been a cornerstone of ecological
modeling and analysis for decades.  Yet there are essentially no real
capture-recapture data sets that come {\it without} auxiliary spatial
information about location of capture (but sometimes such information
is thrown into the trashcan).  As such, classical capture-recapture
models are really not the right tool for the job of analyzing real
data sets. Instead, biologists need methods -- spatial
capture-recapture methods -- that make use of spatial
information in their capture-recapture data sets. 

Spatial capture-recapture methods resolve the essential problem of
capture-recapture, that of estimating population size, but, by
explicitly linking space occupied by individuals with the spatial
location or region of sampling, they do so in a manner that provides a
holistic framework for answering ecological questions related to the
spatial structure of populations -- movement, space usage, spatial
variation in density, landscape connectivity, and other things
(probably).  Thus, spatial capture-recapture methods enable ecologists
to integrate science with their capture-recapture models based on
individual level encounter data.

We provided an overview and synthesis of what is currently known (or,
rather, published in the literature),
around mid-2012, related to spatial capture-recapture methods. There
are many emerging topics which we have not covered either because lack
of technical knowledge, or satisfactory development, or no good
framework for implementation......

\section{Emerging Topics}

\begin{comment}

Strauss paper....beth currently working on this 

In currently developing work, we propose a model that accounts for spatial variation in home range density and potential interactions between individuals' home ranges.  This model lets the activity centers follow an inhomogeneous Strauss process \citep{strauss:1975,handbook:2010}, which allows for spatial variation in the home range intensity, and includes a parameter that determines the strength of repulsion between home ranges.  We show via a simulation study that properly accounting for interactions between individuals can provide a substantial improvement in estimating population size.  For simulated data generated with interaction, the usual independence model has a significant bias for the population size, and generally has larger uncertainty for the population size than the proposed Strauss process model.

While the Strauss model is intuitive and shows great potential, it presents computational challenges.  First, the likelihood includes a high-dimensional integral that has no closed form.   Extending related work for categorical Markov random fields \citep{green:2002,smith:2006}, we develop an approximation to the Strauss likelihood which allows for posterior sampling.  Second, in our Bayesian analysis, the population size is treated as an unknown parameter to be updated using the data.  As the population size varies, so does the dimension of the likelihood, and thus the posterior.  We overcome this dimension-changing problem using an auxiliary variable scheme in the Markov chain Monte Carlo algorithm.  Our analysis of simulated data verifies that this computational approach leads to reliable inference.

\end{comment}

\subsection{Combining data from different surveys}
In some instances, researchers apply different survey techniques to the population of interest, because they yield complementary information. For example, 
Camera trapping is the prime tool for estimating population size/density and other demographic parameters for uniquely marked species; genetic surveys can yield additional information on the genetic diversity and health of a population that we cannot study using camera traps. At the same time, genetic surveys, when samples are analyszed to the individual level, also yield spatial capture recapture data (see Chapt. \ref{chapt.search-encounter}). In this situation, we have two data sets at hand that carry information on animal density, and we should be able to get more precise estimates of density if we combine these two data sets into a single SCR model.

\citet{gopalaswamy_etal:2012} developed two approaches to combining  data from different survey types. In the first case, both surveys are carried out at the same time, so that we can assume that they both sample the same -- closed -- animal population, i.e., there are no possible changes in population density between the two surveys. For camera trapping and genetic surveys, we cannot match records of individuals between the two data sets. As a consequence, in the combined model there are two sets of $z_i$, say, $z^{C}_{i}$ for the camera trap data and $z^{G}_{i}$ for the genetic survey data. But defining a single state-space around both sets of survey locations, we can define a single data augmentation paramter, $\psi$, that refers to both these sets of $mathbf{z}$.    
XXX ANDY; WE STILL HAVE TWO ESTIMATES OF N AND THUS D; HOW DID YOU CHOOSE? OR DID YOU AVERAGE? I DONT FIND ANY INFORMATION ON THAT IN ARJUNS MS XXXXX
Further, since the scale paramter of the trap encounter model (in \citet{gopalaswamy_etal:2012} the Gaussian model}), $\sigma$, is related to animal movement, we can expect this parameter to be the same for both surveys and share it across both data sets. Finally, we need to define separate baseline detection parameters, say $\lambda_{0}^{C}$ and $\lambda_{0}^{G}$, because the processes governing detection by the survey methods should be different. \citet{gopalaswamy_etal:2012} found that this combined model did indeed produce a more precise density estimate, compared to single-data set models. 
 We can, of course, imagine other parameterizations for this combined model -- we could specify both $\psi$ or $\sigma$ as survey specific, if we have reason to believe they changed between surveys, and we refer you to \citet{gopalaswamy_etal:2012} for more details on these alternative parameterizations.
 
A second approach of using information from one survey in the analysis of a second survey (that maybe does not yield quite as much data as the primary survey) is by analyzing your primary data set alone, then taking the posterior distribution of a parameter both surveys share and using it as an informative prior distribution in the analysis of the second data set. \citet{gopalaswamy_etal:2012} refer to this as the stepwise approach, and they implemented this approach by equating the mean and variance of the posterior distribution of $\psi$ and $\sigma$ from the photographic survey to the mean and variance of a beta and a gamma prior for these paramters, respectively, for the genetic survey. The authors found that this approach produced almost identical density estimates compared to the combined model approach descibed above. 

In summary, no matter which approach is chosen, combining data across surveys can help researchers to obtain more precise population estimates, which is especially valuable when dealing with rare and elusive species like big cats that almost always will produce sparse individual data sets. Some thought has to go into the assumptions underlying combining data -- for example, if there is a chance that population size may have changed between surveys, it might not be approapriate to have $\psi$ be a parameter shared across surveys; but the assumption that $\sigma$ remains reasonably constant might still be valid. The paper by \citet{gopalaswamy_etal:2012} considers the situation where we have two SCR data sets, but we can imagine combining SCR data with other sources of information, such as telemetry data (see Chapt. \ref{chapt.partialID} and Chapt. \ref{chapt.rsf} for examples), and possibly opportunistic observations, although to our knowledge this latter issue has not been tackeld in the context of SCR, yet. 


\subsection{Imperfect identification of individuals}
Imperfect identification of individuals can happen in a variety of ways. In genetic surveys there is usually some probability of mis-identification of individuals due to genotyping error (e.g. \citet{lukacs_burnham:2005}). In camera trap survey a different type of imperfect identification can occur when only the only one flank of an animal is recorded in a detection event and that imagine cannot be matched to any of the individuals identified by both flanks. If that case, we can match single-flank pictures with the same side flank pictures, but not with opposite side flank pictures and thus cannot construct definite encounter histories for these single-flank individuals (a right flank and a left flank picture could be the same individual, or could be from two distinct individuals. Finally, in Chapt. \ref{chapt.partialID}, in the context of mark-resight models, we discussed the case where individuals can either not definitely be identified as marked or not -- a violation of a basic mark-resight assumption, and developed an approach to dealing with the situation where we can always tell if an animal is marked or not, but we are not always able to ascertain its individual identity. 

In non-spatial capture recapture some efforts have been made to formally deal with misidentification. \citet{stevick_etal:2001} address this problem by double-sampling to derive an error rate for genetic identification, and then including this error rate as a knwon constant into a Lincoln-Petersen estimator of abundance. \citet{lukacs_burnham:2005} develop an approach that includes an additional parameter in the model -- the probability of a genotype being identified correctly, which is estimated as part of the model likelihood. \citet{link_etal:2010} developed an approach towards solving the same problem implemented in a Bayesian framework that relaxes some of the assumptions of the initial approach.
\citet{yoshizaki_etal:2009} deal with misidentification from camera trap pictures due to evolving markes (i.e., natural marks that change over time, such as scars). This situation is different from the genotyping error one. Here, a change in marks creates a supposedly `new' individual that can be recaptured several times, while the original individual is never captured again (its mark is no longer in the population). In contrast, in genotyping error it is assumed that misidentification creates a `new' individual that is never observed again, because each error leads to a new unique genotype. \citet{yoshizaki_etal:2009} approach this situation similarly, by including a parameter describing the probability of correctly identifying an individual upon recapture (the parameter can also be interpreted as the probability that a mark does not change between capture occasions). Because of the dependencies between true and false detection histories (when a `new' individual is created, the `real' one can no longer be recaptured), the standard multinomial approach to coming up with a model likelihood does not work and implementing the model in a maximum likelihood framework is difficult. The authors instead demostrate an implementation of the model based on minimizing a function of the squared differences between the observed and expected frequencies of the observed capture histories. 

To our knowledge no attempts have been made to deal with misidentification in an SCR framework. While all of the mis-ID cases described above require distinct approaches, we believe that there is one unifying theme to all of them: the location of the un or potentially mis-identified records and the resulting probabilities of belonging to certain individuals conditional on their activity centers. For example, a right flank and a left flank camera trap picture that are taken at two neighboring camera traps are more likely to belong to the same individual that a right and a left flank picture taken at cameras at opposing ends of the trap array, especially if animal movement is smaller than the extent of the trap array; an unidentified record of a marked individual in a mark-resight survey is more likely to belong to a marked individual whose activity center is close by, than to an individual whose activity center is located far away; and so forth. Formally developing misidentification models should be a focus for future SCR model development. 

Design.......

Borchers acoustic stuff .....

Mark-resight models ......


\section{10 thesis or dissertation topics}

Future research directions:

Modeling dynamics of the point process. Transient individuals. 
Dispersal. 

Calibration of GoF under meaningful alternatives

Calibration of AIC/DIC 

Models for non-uniform point processes that exhibit clustering or
repulsion, Strauss processes, Markov point processes, 

The trapping web as an SCR model 

Formalization of the model for single catch systems. 

Models for unmarked or partially marked individuals integrated with
RSF data from telemetry

Occupancy and counts data + SCR data (AOAS and Sollmann et al.)

Spatial genetics  -- can use SCR to study gene flow, related things....

SCR on dendritic networks (streams and trails).

Model based design: Formal connection between SCR models and spatial
models, suggests the infrastructure can be borrowed.

\subsection{Three dimensional space}

Throughout this book we have treated space as
two-dimensional, meaning that activity centers are assumed to occur on
the real plane. This approximation of reality is reasonable for many
terrestrial species, but aquatic organisms, especially marine animals
move about in three-dimensional space. Treating space as
three-dimensional could also conceivably be useful in studies of flying organisms
or species that use multiple strata of tall forests; however, we
suspect that two dimensional models of space should suffice in such
contexts. Regardless, a three-dimensional view of space requires that
activity centers $\bf s_i$ are indexed by
$x,y,z$ coordinates. In theory, this presents no problem whatsoever. In
practice, estimation based on integrated likelihood methods must
involve a three-dimensional integration. This will clearly be more
computationally demanding, but it should be possible using packages
such as {\tt R2Cuba}.




\subsection{Gregarious species}

Many social species move about in large groups rather than as single
individuals. Even species regarded as solitary often join family
groups for some portion of their life cycle. The consequences of
gregariousness?? are x-fold....

To account for this, we change our definition of $s_i$ from the
location of an individual's activity center, to the location of a
group's activity center. We then expand our model to include a
submodel for group size, and we can estimate both the density of group
activity centers and total population size.


\section{Grand Summary}


SCR models allow you to address basic and applied questions of
population ecology from individual encounter history data. Problems
having to do with movement, space usage, landscape connectivity,
etc... This is a BFD. 

