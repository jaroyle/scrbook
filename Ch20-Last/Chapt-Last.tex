


\chapter{
 2012: A Spatial Capture-Recapture Odyssey
 }

\markboth{The End}{}
\label{chapt.final}

\vspace{0.3cm}


\vspace{2in}

Capture recapture methods have been a cornerstone of ecological
modeling and analysis for decades.  Yet there are essentially no real
capture-recapture data sets that come {\it without} auxiliary spatial
information about location of capture (but sometimes such information
is thrown into the trashcan).  As such, classical capture-recapture
models are really not the right tool for the job of analyzing real
data sets. Instead, biologists need methods -- spatial
capture-recapture methods -- that make use of spatial
information in their capture-recapture data sets. 

Spatial capture-recapture methods resolve the essential problem of
capture-recapture, that of estimating population size, but, by
explicitly linking space occupied by individuals with the spatial
location or region of sampling, they do so in a manner that provides a
holistic framework for answering ecological questions related to the
spatial structure of populations -- movement, space usage, spatial
variation in density, landscape connectivity, and other things
(probably).  Thus, spatial capture-recapture methods enable ecologists
to integrate science with their capture-recapture models based on
individual level encounter data.

We provided an overview and synthesis of what is currently known (or,
rather, published in the literature),
around mid-2012, related to spatial capture-recapture methods. There
are many emerging topics which we have not covered either because lack
of technical knowledge, or satisfactory development, or no good
framework for implementation......

\section{Emerging Topics}

\begin{comment}

Strauss paper....beth currently working on this 

In currently developing work, we propose a model that accounts for spatial variation in home range density and potential interactions between individuals' home ranges.  This model lets the activity centers follow an inhomogeneous Strauss process \citep{strauss:1975,handbook:2010}, which allows for spatial variation in the home range intensity, and includes a parameter that determines the strength of repulsion between home ranges.  We show via a simulation study that properly accounting for interactions between individuals can provide a substantial improvement in estimating population size.  For simulated data generated with interaction, the usual independence model has a significant bias for the population size, and generally has larger uncertainty for the population size than the proposed Strauss process model.

While the Strauss model is intuitive and shows great potential, it presents computational challenges.  First, the likelihood includes a high-dimensional integral that has no closed form.   Extending related work for categorical Markov random fields \citep{green:2002,smith:2006}, we develop an approximation to the Strauss likelihood which allows for posterior sampling.  Second, in our Bayesian analysis, the population size is treated as an unknown parameter to be updated using the data.  As the population size varies, so does the dimension of the likelihood, and thus the posterior.  We overcome this dimension-changing problem using an auxiliary variable scheme in the Markov chain Monte Carlo algorithm.  Our analysis of simulated data verifies that this computational approach leads to reliable inference.

\end{comment}




Design.......

Borchers acoustic stuff .....

Mark-resight models ......


\section{10 thesis or dissertation topics}

Future research directions:

Modeling dynamics of the point process. Transient individuals. 
Dispersal. 

Calibration of GoF under meaningful alternatives

Calibration of AIC/DIC 

Models for non-uniform point processes that exhibit clustering or
repulsion, Strauss processes, Markov point processes, 

The trapping web as an SCR model 

Formalization of the model for single catch systems. 

Models for unmarked or partially marked individuals integrated with
RSF data from telemetry

Occupancy and counts data + SCR data (AOAS and Sollmann et al.)

Spatial genetics  -- can use SCR to study gene flow, related things....

SCR on dendritic networks (streams and trails).

Model based design: Formal connection between SCR models and spatial
models, suggests the infrastructure can be borrowed.

\subsection{Three dimensional space}

Throughout this book we have treated space as
two-dimensional, meaning that activity centers are assumed to occur on
the real plane. This approximation of reality is reasonable for many
terrestrial species, but aquatic organisms, especially marine animals
move about in three-dimensional space. Treating space as
three-dimensional could also conceivably be useful in studies of flying organisms
or species that use multiple strata of tall forests; however, we
suspect that two dimensional models of space should suffice in such
contexts. Regardless, a three-dimensional view of space requires that
activity centers $\bf s_i$ are indexed by
$x,y,z$ coordinates. In theory, this presents no problem whatsoever. In
practice, estimation based on integrated likelihood methods must
involve a three-dimensional integration. This will clearly be more
computationally demanding, but it should be possible using packages
such as {\tt R2Cuba}.




\subsection{Gregarious species}

Many social species move about in large groups rather than as single
individuals. Even species regarded as solitary often join family
groups for some portion of their life cycle. The consequences of
gregariousness?? are x-fold....

To account for this, we change our definition of $s_i$ from the
location of an individual's activity center, to the location of a
group's activity center. We then expand our model to include a
submodel for group size, and we can estimate both the density of group
activity centers and total population size.


\section{Grand Summary}


SCR models allow you to address basic and applied questions of
population ecology from individual encounter history data. Problems
having to do with movement, space usage, landscape connectivity,
etc... This is a BFD. 

