


\chapter{Spatial Capture-Recapture: The Final Frontier}

\markboth{The End}{}
\label{chapt.final}

\vspace{0.3cm}

Alternative: 2013: A spatial capture-recapture odyssey. 

Future research directions:

 Modeling dynamics of the point process. Transient individuals. 
 Dispersal. Things like that.

The big point is that we provide a framework for spatial analysis of animal populations from
individual encounter data:
MOVEMENT, SPACE USAGE, SPATIAL VARIATION IN DENSITY -- 
much to be done: how do individuals interact? how is space usage changing over time, etc...


Topics to discuss:

 (1) Strauss process model
 (2) Need for general purpose software.... all of the spatial stuff + 
open populations in one big model. 
 (3) Efficient computation is still an issue.
 (4) Fit and model selection will continue to be important practical
 issues.

\section{10 thesis or dissertation topics}

Calibration of GoF under meaningful alternatives

Calibration of AIC/DIC and efficacy study

Models for non-uniform point processes that exhibit clustering or
repulsion







\section{Three dimesional space}

Throughout this book we have treated space as
two-dimensional, meaning that activity centers are assumed to occur on
the real plane. This approximation of reality is reasonable for many
terrestrial species, but aquatic organisms, especially marine animals
move about in three-dimensional space. Treating space as
three-dimensional could also conceivably be useful in studies of flying organisms
or species that use multiple strata of tall forests; however, we
suspect that two dimensional models of space should suffice in such
contexts. Regardless, a three-dimensional view of space requires that
activity centers $\bf s_i$ are indexed by
$x,y,z$ coordinates. In theory, this presents no problem whatsoever. In
practice, estimation based on integrated likelihood methods must
involve a three-dimensional integration. This will clearly be more
computationally demanding, but it should be possible using packages
such as {\tt R2Cuba}.




\section{Gregarious species}

Many social species move about in large groups rather than as single
individuals. Even species regarded as solitary often join family
groups for some portion of their life cycle. The consequences of
gregariousness?? are x-fold....

To account for this, we change our definition of $s_i$ from the
location of an individual's activity center, to the location of a
group's activity center. We then expand our model to include a
submodel for group size, and we can estimate both the density of group
activity centers and total population size.
