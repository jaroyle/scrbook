%%% TO DO  as of 12/29/11

 %%% Spell check document

 %%% Change "beta" to "theta"

 %%% Fix up R scripts and consolidate for R package
 %%% R commands to process wolverine data need included in that section

 %%% Run Wolverine 2k 4k and 8k grids in JAGS compare to WinBUGS
 %%%     insert those results in text

 %%%  For discrete state-space stuff, convert BUGS output to JAGS and
 %%%  figure out MC errors
 %%% Finish Table that has those results in it

 %% pick up all hard references to chapters and make float


\chapter{Fully Spatial Capture-Recapture Models}
\markboth{Chapter 4 }{}
\label{chapt.scr0}

\vspace{.3in}

In previous sections we discussed some classes of models that could be
viewed as primitive spatial capture-recapture models. We looked at a
basic distance sampling model and we also considered a classical
individual covariate modeling approach in which we defined a covariate
to be the distance from (estimated) home range center to the center of
the trap array. These were spatial in the sense that they included
some characterization of where individuals live but, on the other
hand, only a primitive or no characterization of trap location.  That
said, very little distinguishes these two models from spatial
capture-recapture models that we consider in this chapter which fully
recognize the spatial attribution of both individual animals {\it and}
the locations of encounter devices.

Fully spatial capture-recapture models must accommodate the spatial
organization of individuals and the encounter devices because the
encounter process occurs at the level of individual traps.  Failure to
consider the trap-specific collection of data is the key deficiency
with classical ad-hoc approaches which aggregate encounter information
to the resolution of the entire trap array. We have  previously
addressed some problems that this induces including induced
heterogeneity in encounter probability, imprecise notation of ``sample
area'' and not being able to accommodate trap-specific
effects.
In this chapter we resolve these issues by developing 
our first fully spatial capture-recapture
model which turns out to be precisely the model considered in sec. \ref{closed.sec.indcov}
 but instead of defining the individual covariate to be distance
to centroid of the array we define $J$ individual covariates - the
distance to {\it each} trap. And, instead of using estimates of
individual locations ${\bf s}$, we consider a fully hierarchical model in
which we regard ${\bf s}$ as a latent variable and impose a prior
distribution on it.  We can think of having $J$ independent
capture-recapture studies generating one data set for each trap, and
applying the individual covariate model with random activity centers,
and that is all the basic SCR model is.

In the following sections of this chapter we investigate the basic
spatial capture-recapture model, which we refer to as ``model SCR0'',  and address some important
considerations related to its analysis in {\bf WinBUGS}. We also demonstrate
how to summarize posterior output for the purposes of producing
density maps or spatial predictions of density.

\section{Sampling Design and Data Structure}

In our development here, we will assume a standard sampling design in
which an array of $J$ traps is operated for $K$ time periods (say,
nights) producing encounters of $n$ individuals.  Because sampling
occurs by traps and also over time, the most general data structure
yields encounter histories for {\it each individual} that are
temporally {\it and} spatially indexed. Thus a typical data set will
include an encounter history {\it matrix} for each individual.  For
the most basic model, there are no time-varying covariates that
influence encounter, there are no explicit individual-specific
covariates, and there are no covariates that influence density we will
develop models in this chapter for encounter data that are aggregated
over the temporal replicates. For example, suppose we observe 6
individuals in sampling at 4 traps over 3 nights of sampling then a
plausible data set is the $6 \times 4$ matrix of encounters, out of 3,
of the form:
\begin{verbatim}
      trap1 trap2 trap3 trap4
 [1,]     1     0     0     0
 [2,]     0     2     0     0
 [3,]     0     0     0     1
 [4,]     0     1     0     0
 [5,]     0     0     1     1
 [6,]     1     0     1     0
\end{verbatim}

We develop models in this chapter for devices such as ``hair snares''
or other DNA sampling methods \citep{kery_etal:2010,
  gardner_etal:2010jwm} and related types of sampling devices in which
(i) effective ``traps'' may capture any number of individuals (i.e.,
they don't fill up; This is referred to as a ``multi-catch'' type of
sampling \citep{efford_etal:2009ecol}); (ii) an individual may be
captured in any number of traps during each occasion but (iii)
individuals can be encountered at most 1 time in a trap during any
occasion.  The statistical assumptions are that individual encounters
within and among traps are independent, and this allows us to regard
individual- and trap-specific encounters as $iid$ Bernoulli trials
(see next section).  These basic (but admittedly at this point
somewhat imprecise) assumptions define the basic spatial
capture-recapture model, which we will refer to as ``SCR0'' 
so that we may use that model as a point of reference without having
to provide a long-winded enumeration of assumptions and sampling
design each time we do. We will make things more precise as we develop
a formal statistical definition of the model shortly.

While the model is mostly directly relevant
to hair snares and other DNA sampling methods for which multiple
detections of an individual are not distinguishable,
we will also make use of the model for data that arise from
camera-trapping studies. In practice, with camera trapping,
individuals might be photographed several times in a night but we will
typically distill such data into a single binary encounter event for
reasons discussed later in Chapt. \ref{chapt.poisson-mn}.


\section{The binomial observation model }

We assume that the individual and trap-specific encounters, $y_{ij}$,
are mutually independent outcomes of a binomial random variable:
\begin{equation}
	y_{ij} \sim \mbox{Bin}(K, p_{ij})
\label{scr0.eq.bin}
\end{equation}
This is the basic model underlying ``logistic regression'' (Chapt. \ref{chapt.glms})
as well as standard closed population models
(Chapt. \ref{chapt.closed}). The key
element of the model is that the encounter probability $p_{ij}$ is
indexed by (i.e., depends on) both individual and trap. In a sense,
then, we can think of each {\it trap} as producing individual level
encounter history data of the classical variety - an $\mbox{\tt nind}
\times \mbox{\tt nreps}$
matrix of 0's and 1's (this is the ``encountered at most 1 time''
assumption).


As we did in sec. \ref{closed.sec.indcov}, we will make explicit the notion that
$p_{ij}$ is defined conditional on ``where'' individual $i$
lives. Naturally, we think about defining an individual home range and
then relating $p_{ij}$ explicitly to the centroid of the individuals
home range, or its center of activity \citep{efford:2004,
  borchers_efford:2008, royle_young:2008}.  Therefore, define ${\bf
  s}_{i}$, a two-dimensional spatial coordinate, to be the activity
center for individual $i$. Then, the SCR model postulates that
encounter probability, $p_{ij}$, is a decreasing function
of distance between ${\bf s}_{i}$ and the location of trap $j$, ${\bf x}_{j}$.
 Naturally, if we think of modeling binomial counts using
logistic regression, we might specify the model according to:
\begin{equation}
	\mbox{logit}(p_{ij}) = \alpha_{0} + \alpha_1 ||{\bf s}_{i}-{\bf x}_{j} ||
\label{scr0.eq.logit}
\end{equation}
where, here, $||{\bf s}_{i}-{\bf x}_{j}||$ is the distance between
${\bf s}_{i}$ and ${\bf x}_{j}$. We sometimes write $||{\bf
  s}_{i}-{\bf x}_{j}|| = dist({\bf s}_{i},{\bf x}_{j}) =
d_{ij}$. Alternatively, if we think about distance sampling then we
might use the ``half-normal'' model of the form:
\[
p_{ij} = p_{0}*\exp(-\alpha_{1} *||{\bf s}_{i}-{\bf x}_{j}||^2)
\]
Or any of a large number of standard detection models that are
commonly used (we consider more in Chapt. \ref{chapt.covariates}). The half-normal model implies
\begin{equation}
\log(p_{ij})  = \log(p_{0}) - \alpha_{1} *||{\bf s}_{i}-{\bf x}_{j}||^2
\label{scr0.eq.norm}
\end{equation}
%We would always like to be clear that encounter probability depends on individual activity
%centers {\it and} trap locations {\it and} parameter(s) $\theta$, and
%so it would be ideal to write $p({\bf s}_{i},{\bf x}_{j}; \theta)$ or
%something similar. However, this can be extremely unwieldy and
%clutter up what are otherwise extremely simple mathematical
%expressions and formulae. As such, we will usually abbreviate these
%various dependencies by writing $p_{ij}$ or sometimes $p_{\theta,ij}$,
%understanding that $p_{ij}$ is actually a function of the various important
%quantities.
We probably expect that the parameter $\alpha_{1}$ in
Eq. \ref{scr0.eq.logit} or \ref{scr0.eq.norm} should be negative, so
that the probability of encounter decreases with distance between the
trap and individual home range center.  
Whatever model encounter probability we choose, we should always keep
in mind that the model is described conditional on ${\bf s}_{i}$,
which is an unobserved random variable.  Thus, to be precise about
this, we should write the observation model as
\[
y_{ij}|{\bf s}_{i} \sim \mbox{Bin}(K, p({\bf s}_{ij};\alpha_{1}))
\]


The joint likelihood for the
data, conditional on the collection of individual activity centers,
can therefore be expressed as
\[
{\cal L}(\alpha_{1} | \{ {\bf y}_{i},{\bf s}_{i} \}_{i=1}^{N})
 =  \prod_{i} \prod_{j} \mbox{Bin}(y_{ij}|p_{ij}(\alpha_{1}))
\]
Which, if we switch the indices on the product operators, this shows
the SCR likelihood (conditional on ${\bf s}$) to be the product of $J$
{\it independent} capture-recapture likelihoods - one for each trap.
However, the data have a distinct ``repeated measures'' type of structure, with
each of the $j$ likelihood contributions for each individual being
grouped by individual. Thus, we cannot analyze the model
meaningfully by $J$ trap-specific models. In classical repeated measures
types of models, we accommodate the group structure of the data using
random effects (random individual or group level variables). For SCR
models we take the same basic approach, which we develop subsequently.

\subsection{Distance as a latent variable}

If we knew precisely every ${\bf s}_{i}$ in the population (and how
many, $N$), then the model specified by eqs. \ref{scr0.eq.bin} and
\ref{scr0.eq.logit} is just an ordinary logistic
regression type of a model which we learned how to fit using {\bf
  WinBUGS} previously (Chapt. \ref{chapt.glms}), with a covariate $d_{ij}$. However,
the activity centers are unobservable even in the best possible
circumstances. In that case, $d_{ij}$ is an unobserved variable,
analogous to classical random effects models. We need to therefore
extend the model to accommodate these random variables with an
additional model component. A standard, and perhaps not unreasonable,
assumption is the so-called ``uniformity assumption'' which is to say
that the ${\bf s}_{i}$ are uniformly distributed over space (the
obvious next question ``which space?'' is addressed below).  This
uniformity assumption amounts to a uniform prior distribution on ${\bf
  s}_{i}$, i.e., the pdf of ${\bf s}_{i}$ is constant, which we may
express
\begin{equation}
	\Pr({\bf s}_{i}) \propto \mbox{\tt const}
\label{scr0.eq.sprior}
\end{equation}
 As it turns out, this assumption is usually not precise
enough to fit SCR models in practice for reasons we discuss in the
following section.  We will give another way to represent this prior
distribution that is more concrete, but it depends on specifying the
``state-space'' of the random variable ${\bf s}_{i}$. The term
state-space is a technical way of saying ``possible outcomes''.

To summarize the preceeding model developing, a basic SCR model is
defined by 3 essential components:
\begin{itemize}
\item[(1)] Observation model: $y_{ij}|{\bf s}_{i} \sim \mbox{Bin}(K, p_{ij})$
\item[(2)] Encounter probability: $\mbox{logit}(p_{ij}) = \alpha_{0} +
  \alpha_{1}*||{\bf s}_{i}-{\bf x}_{j}||$
\item[(3)] Point process model: $\Pr({\bf s}_{i} ) \propto \mbox{\tt const}$
\end{itemize}
Therefore, the SCR model is little more than an ordinary
capture-recapture model for closed populations. It is such a model,
but augmented with a set of ``individual effects'', ${\bf s}_{i}$,
which relate some sense of individual location to encounter
probability. 

\section{ The Binomial Point-process Model}

The collection of individual activity centers ${\bf s}_{1},\ldots,
{\bf s}_{N}$ represent a realization of a {\it binomial point process}
\citep[][p. xyz]{illian_etal:2008}.  The binomial point process (BPP)
is analogous to a Poisson point process in the sense that it
represents a ``random scatter'' of points in space - except that the
total number of points is {\it fixed}, whereas, in a Poisson point
process it is random (having a Poisson distribution).  As an example,
we show in Fig. \ref{scr0.fig.bpp} locations of 20 individual activity
centers (black dots) in relation to a grid of 25 traps. For a Poisson
point process the number of such points in the prescribed state-space
would be random whereas often we will simulate fixed numbers of
points, e.g., for evaluating the performance of procedures such as how
well does our estimator perform of $N=50$?
\begin{figure}
\begin{center}
\includegraphics[height=2.5in]{Ch4/figs/binomialpoint}
\end{center}
\caption{Realization (small circles) of a binomial point process with $N=20$. The
  large circles represent trap locations.}
\label{scr0.fig.bpp}
\end{figure}

It is natural to consider a binomial point process in the context of
capture-recapture models because it preserves $N$ in the model and thus
preserves the linkage directly with closed population models. In fact,
under the binomial point process model then model $M_0$ and other closed
models are simple limiting cases of SCR models, i.e., as the
coefficient on distance tends to 0.
In addition, use of
the BPP model allows us to use data augmentation for Bayesian analysis
of the models as in Chapt. \ref{chapt.closed}, thus yielding a methodologically
coherent approach to analyzing the different classes of
models. Despite this, making explicit assumptions about $N$, such as
Poisson, is convenient in some cases (see Chapt. \ref{chapt.hscr}).

One consequence of having fixed $N$, in the BPP model, is that the
model is not strictly a model of ``complete spatial randomness''. This
is because if one forms counts $n(A_{1}),\ldots, n(A_{k})$ in any set
of disjoint regions say $A_{1}, \ldots, A_{k}$, then these counts are
{\it not} independent.  In fact, they have a multinomial distribution
\citep[see][p. XYZ]{illian_etal:2008}. Thus, the BPP model introduces
a slight bit of dependence in the distribution of points. However, in
most situations this will have no practical effect on any inference or
analysis and, as a practical matter, we will usually regard the BPP
model as one of spatial independence among individual activity centers
because each activity center is distributed independently of each
other activity center. Despite this implicit independence we see in
Fig. \ref{scr0.fig.bpp} that {\it realizations} of randomly distributed
points will typically exhibit distinct non-uniformity. Thus,
independent, uniformly distributed points will almost never appear
regularly, uniformly or systematically distributed. For this reason,
the basic binomial (or Poisson) point process models are enormously
useful in practical settings.  More relevant for SCR models is that we
actually have a little bit of data for some individuals and thus the
resulting posterior point pattern can deviate strongly from
uniformity, a point we come back to repeatedly in this book.
The uniformity hypothesis is only
a {\it prior} distribution which is directly affected by the quantity
and quality of observations, to produce a posterior distribution which
may appear distinctly non-uniform.


\subsection{Definition of home range center}

Some will be offended by our use of the concept of ``home range
center'' and thus will have difficulty in believing that the resulting
model is really useful for anything.  Indeed, the idea of a home range
or activity center is a vague concept anyway, a purely
phenomenological construct.  Despite this, it doesn't really matter
whether or not a home range makes sense for a particular species -
individuals of any species inhabit {\it some} region of space and we
can define the ``home range center'' to be the center of the space
that individual was occupying (or using) during the period in which
traps were active. Thinking about it in that way, it could even be
observable (almost) as the centroid of a very large number of radio
fixes over the course of a survey period or a season.  Thus, this
practical version of a home range center in terms of space usage is a well-defined construct
regardless of whether one thinks the home range concept is meaningful,
even if individuals are not particularly territorial.  This is why we
usually use the term ``activity center'' or maybe even ``centroid of
space usage'' and we recognize that this construct is a transient
thing which applies only to a well-defined period of study.



\subsection{The state-space of the point process}

Shortly we will focus on Bayesian analysis of this model with $N$
known so that we can directly apply what we learned in
Chapt. \ref{chapt.glms} to 
this situation. To do this, we note that the individual effects ${\bf
  s}_{i},\ldots, {\bf s}_{N}$ are unknown quantities and we will need
to be able to simulate each ${\bf s}_{i}$ in the population from the
posterior distribution.  It should be self-evident that we cannot
simulate the ${\bf s}_{i}$ unless we describe precisely the region
over which they are uniformly distributed. This is
the quantity referred to above as the state-space, denoted henceforth
by ${\cal S}$, which is a region or a set of points comprising the
potential values of ${\bf s}_{i}$. Thus, an equivalent explicit
statement of the ``uniformity assumption'' is
\[
{\bf s}_{i} \sim \mbox{Unif}({\cal S})
\]
where ${\cal S}$ is a precisely defined region. e.g., in Fig. 
\ref{scr0.fig.bpp}, ${\cal S}$ is the square defined by $[-1,7] \times
[-1, 7]$. Thus each of the $N=20$ points were generated by randomly
selecting each coordinate on the line $[-1, 7]$. 


\subsubsection{Prescribing the state-space}

Evidently, we need to define the state-space, ${\cal S}$. How can we
possibly do this objectively? Prescribing any particular ${\cal S}$
seems like the equivalent of specifying a ``buffer'' which we
criticized previously as being ad hoc. How is it, then, is choosing a
state-space is {\it not} ad hoc? As a practical matter, it turns out
that estimates of density are insensitive to choice of the
state-space. As we observed in Chapt. \ref{chapt.closed}, it is true that $N$ increases
with ${\cal S}$, but only at the same rate as the area of ${\cal S}$
increases under the
prior assumption of constant density. As a result, we say that density
is invariant to ${\cal S}$ as long as ${\cal S}$ is sufficiently
large. Thus, while choice of ${\cal S}$ is (or can be) essentially
arbitrary, once ${\cal S}$ is chosen, it defines the population being
exposed to sampling, which scales appropriately with the size of the
state-space.

For our simulated system developed previously in this chapter, we
defined the state space to be a square within which our trap array was
centered. For many practical situations this might be an
acceptable approach to defining the state-space. We provide an example
of this in sec. \ref{scr0.sec.wolverine} below in which the trap array is
irregular and also situated within a realistic landscape that is
distinctly irregular.  In general, it is most practical to define the
state-space as a regular polygon (e.g., rectangle) containing the trap
array without differentiating unsuitable habitat. Although defining
the state-space to be a regular polygon has computational advantages
(e.g., we can implement this more efficiently in {\bf WinBUGS} and
cannot for irregular polygons), a regular polygon induces an apparent
problem of admitting into the state-space regions that are distinctly
non-habitat (e.g., oceans, large lakes, ice fields, etc.).  It is
difficult to describe complex sets in mathematical terms that can be
used in {\bf BUGS}. As an alternative, we can provide a
representation of the state-space as a discrete set of points (sec.
\ref{scr0.sec.discrete}) that will allow specific points to be deleted
or not depending on whether they represent habitat, or we can define
the state-space as an arbitrary  collection of polygons stored as a GIS
shapefile
which can be analyzed easily using MCMC
(see sec. \ref{mcmc.sec.state-space}), but not so easily in the {\bf
  BUGS} variants.  In what follows below we provide an
analysis of the camera data defining the state-space to be a regular
continuous polygon (a rectangle).


\subsection{Invariance and the State-space as a model assumption}
\label{scr0.sec.invariance}

We will assert for all models we consider in this book that density is
invariant to the size and extent of ${\cal S}$, if ${\cal S}$ is
sufficiently large as long 
as our model relating $p_{ij}$ to ${\bf  s}_{i}$ is a decreasing
function of distance.  
We can prove this easily by drawing an analogy with a 1-d case such as
in distance sampling.  Let $y_{j}$ be the number of individuals
captured in some interval $[d_{j-1},d_{j})$, and define $d_{J} = B$
for some large value of $B$.  By choosing $B$ large enough we
guarantee that $E[y_{J+1}] = 0$ and therefore this ``last cell'' 
contributes nothing to
the likelihood
in regular situations in which the detection function decays
monotonically with distance and prior density is constant.  


Sometimes
our estimate of density can be influenced if we make ${\cal S}$ too small but
this might be sensible if ${\cal S}$ is naturally well-defined. As we discussed
in chapter 1, {\bf choice of ${\cal S}$ is part of the model and thus it makes
  sense that estimates of density might be sensitive to its definition
  in problems where it is natural to restrict ${\cal S}$}.
One could imagine
however that in specific cases where you're studying a small
population with well-defined habitat preferences that a problem could
arise because changing the state-space around based on differing
opinions and GIS layers really changes the estimate of total
population size. But this is a real biological problem and a natural
consequence of the spatial formalization of capture-recapture models -
a feature, not a bug or some statistical artifact - and it should be
resolved with better information, research, and thinking.
 For situations where there is not a natural
choice of ${\cal S}$, we should default to choosing ${\cal S}$ to be very large in order
to achieve invariance or otherwise evaluate sensitivity of density
estimates by trying a couple of different values of ${\cal S}$. This is a
standard ``sensitivity to prior'' argument that Bayesians always have
to be conscious of.  We demonstrate this in our analysis of section
\ref{scr0.sec.wolverine}
below. Note that $area({\cal S})$ affects data augmentation. If you
increase $area({\cal S})$ then there are more individuals to account for and
therefore the size of the augmented data set $M$ must increase.

We have been told that one can carry-out non-Bayesian analyses of SCR
models without having to specify the state-space of the point process
or perhaps while only specifying it imprecisely.  This assertion is
incorrect. We assume people are thinking this because {\it they} don't
have to specify it explicitly because someone else has done it for
them in a package that does integrated likelihood. Even to do
integrated likelihood (see Chapt. \ref{chapt.mle}) we have to integrate the
conditional-on-${\bf s}$ likelihood over some 2-dimensional space.  It might
work that the integration can be done from $-\infty$ to $+\infty$ but
that is a mathematical artifact of specific detection functions, and
an implicit definition of a state-space that doesn't make biological
sense, even though it may in fact be innocuous;


\subsection{Connection to Model  $M_h$}  \label{scr0.sec.scrmh}

SCR models are closely related to heterogeneity models. In SCR models,
heterogeneity in encounter probability is induced by both the effect
of distance in the model for detection probability and also from
specification of the state-space. Hence, the state-space  is an
explicit element of the model. 
To understand this, suppose we have a random
effect with some prior distribution:
\[
{\bf s} \sim \mbox{Unif}({\cal S})
\]
And $p({\bf s}) = p(y=1|{\bf s})$ is some function of ${\bf
  s}$. Therefore, for any specific $g(p)$ and ${\cal S}$ we can work
out what the implied heterogeneity model is for example, the mean,
variance or other moments of the population distribution of $p$ can be
evaluated by integrating $p({\bf s})$ over the state-space of ${\bf
  s}$.  We
show an illustration in Fig. \ref{scr0.fig.buffereffect} which
shows a histogram of $p$ for a hypothetical population of 100000
individuals on a state-space enclosing our $5 \times 5$ trap array
above, under the logistic model for distance. {\bf R} code is
provided in the {\bf R} package \mbox{\tt scrbook} to produce this analysis for the
logistic and half-normal models. The histogram shows the encounter
probability under buffers of 0.2, 0.5 and 1.0. We see the mass shifts
to the left as the buffer increases, implying more individuals
 with lower encounter probabilies, as their home range
centers increase in distance from the trap array.


\begin{figure}
\begin{center}
\includegraphics[width=5in]{Ch4/figs/buffereffect}
\end{center}
\caption{Implied population distribution of $p_{i}$ for a population
  of individuals as a function of the size of the state-space buffer
  around a trap array. The trap array is fixed and centered within a
  square state-space.}
\label{scr0.fig.buffereffect}
\end{figure}

Another way to understand this is by representing ${\cal S}$ as a set
of discrete points on a grid. In the coarsest possible case where
${\cal S}$ is a single arbitrary point, then every individual has
exactly the same $p$. As we increase the number of points in ${\cal
  S}$ then more distinct values of $p$ are possible. As such, when
${\cal S}$ is characterized by discrete points then SCR models are
precisely a type of finite-mixture model \citep{norris_pollock:1996,
  pledger:2000}, except, in the case of SCR models, we have some information about which
group an individual belong (i.e., where their activity center is), as
a result of their captures in traps.

This context suggests the problem raised by \citet{link:2003}. He
showed that in most practical situations $N$ may not be identifiable
across classes of mixture distributions which in the context of SCR
models is the pair $(g, {\cal S})$.  The difference, however, is that
we do obtain some direct information about ${\bf s}$ in SCR models and
therefore it may be reasonable to expect that
$N$ is identifiable across models characterized by $(g,{\cal
  S})$.

\subsection{Connection to Distance Sampling}

It is worth emphasizing that the basic SCR model is a binomial
encounter model in which distance is a covariate. As such, it is
striking similarity to a classical distance sampling model. Both have
distance as a covariate but in classical distance sampling problems
the focus is on the distance between the observer and the animal at an
instant in time, not the distance between a trap and an animal's home
range center. As a practical matter, in distance sampling, ``distance'' is {\it
  observed} for those individuals that appear in the
sample. Conversely, in SCR problems, it is only imperfectly observed
(we have partial information in the form of trap observations).
Clearly, it is preferable to observe distance if possible, but 
distance sampling requires field methods that
are often not practical in many situations, e.g. when surveying
tigers. Furthermore, SCR models allow us to relax many of the
assumption made in classical distance sampling, and SCR models allow
for estimates of quantities other than density, such as home range
size, and space usage (see Chapt. \ref{chapt.ecoldist}).


\section{Simulating SCR Data}

It is always useful to simulate data because it allows you to
understand the system that you're modeling and also calibrate your
understanding with the parameter values of the model. That is, you can
simulate data using different parameter values until you obtain data
that ``looks right'' based on your knowledge of the specific situation
that you're interested in. Here we provide a simple script to
illustrate how to simulate spatial encounter history data. In this
exercise we simulate data for 100 individuals and a 25 trap array laid
out in a $5 \times 5$ grid of unit spacing.  The specific encounter model is
the half-normal model given above and we used this code to simulate
data used in subsequent analyses.  The 100 activity centers were
simulated on a state-space defined by a $8 \times 8$ square within which the
trap array was centered (thus the trap array is buffered by 2
units). Therefore, the density of individuals in this system is fixed
at $100/64$.

{\small
\begin{verbatim}
	set.seed(2013)
# create 5 x 5 grid of trap locations with unit spacing
traplocs<- cbind(sort(rep(1:5,5)),rep(1:5,5))
Dmat<-e2dist(traplocs,traplocs) # in cases where speed doesn't matter, it might be
                                # clearer to just show the slow for-loop.
                                # Plus, people will want to copy/paste this stuff
ntraps<-nrow(traplocs)

# define state-space of point process. (i.e., where animals live).
# "delta" just adds a fixed buffer to the outer extent of the traps.
delta<-2
Xl<-min(traplocs[,1] - delta)
Xu<-max(traplocs[,1] + delta)
Yl<-min(traplocs[,2] - delta)
Yu<-max(traplocs[,2] + delta)

N<-100   # population size
K<- 20    # number nights of effort

sx<-runif(N,Xl,Xu)    # simulate activity centers
sy<-runif(N,Yl,Yu)
S<-cbind(sx,sy)
D<- e2dist(S,traplocs)  # distance of each individual from each trap

alpha0<- -2.5      # define parameters of encounter probability
sigma<- 0.5        #
alpha1<- 1/(2*sigma*sigma)
probcap<- expit(-2.5)*exp( - alpha1*D*D)    # probability of encounter
# now generate the encounters of every individual in every trap
Y<-matrix(NA,nrow=N,ncol=ntraps)
for(i in 1:nrow(Y)){
   Y[i,]<-rbinom(ntraps,K,probcap[i,])
}
\end{verbatim}
}


Subsequently we will generate data using this code packaged in an {\bf
  R}
function called \mbox{\tt simSCR0.fn} in the package \mbox{\tt
  scrbook} which takes a number of
arguments including \mbox{\tt discard0} which, if \mbox{\tt TRUE}, will return
only the encounter histories for captured individuals.  A second
argument is \mbox{\tt array3d} which, if \mbox{\tt TRUE}, returns the 3-d
encounter history array instead of the aggregated \mbox{\tt nind}
$\times \mbox{\tt ntraps}$ encounter frequencies (see below). Finally
we provide a random number seed, \mbox{\tt sd = 2013} to ensure
repeatability of the analysis here. We obtain a data set as above using the
following command:
\begin{verbatim}
data<-simSCR0.fn(discard0=TRUE,array3d=FALSE,sd=2013)
\end{verbatim}
The {\bf R} object \mbox{\tt data} is a list, so let's take a look at
what's in the list and then harvest some of its elements for further
analysis below.
\begin{verbatim}
> names(data)
[1] "Y"        "traplocs" "xlim"     "ylim"     "N"        "alpha0"   "beta"
[8] "sigma"    "K"
> Y<-data$Y
> traplocs<-data$traplocs
\end{verbatim}


\subsection{Formatting and manipulating real data sets}
\label{scr0.sec.formats}

Conventional capture-recapture data are easily stored and manipulated
as a 2-dimensional array, an $\mbox{\tt nind} \times \mbox{\tt
  nperiod}$ matrix, which is maximally informative for any
conventional capture-recapture model, but not for spatial
capture-recapture models.  For SCR models we must preserve the spatial
information in the encounter history information. We will routinely
analyze data from 3 standard formats:
\begin{itemize}
\item[(1)] The basic 2-dimensional data format, which is an \mbox{\tt
    nind} $\times$ \mbox{\tt ntraps} encounter frequency matrix such
  as that simulated previously; These are the total encounters in each
  trap, summed over replicate samples.
\item[(2)] The maximally informative 3-dimensional array which we
  establish here the convention that it has dimensions \mbox{\tt nind}
  $\times$ \mbox{\tt nperiods} $\times$ \mbox{\tt ntraps} and
\item[(3)] We use a compact format - the ``SCR flat format'' - which
  we describe below in section \ref{scr0.sec.wolverine}.
\end{itemize}
To simulate data in the most informative format - the ``3-d array'' -
we can use the {\bf R} commands given previously but replace the last
4 lines with the following:
{\small
\begin{verbatim}
Y<-array(NA,dim=c(N,K,ntraps))
for(i in 1:nrow(Y)){
for(j in 1:ntraps){
 Y[i,1:K,j]<-rbinom(K,1,probcap[i,j])
}
}
\end{verbatim}
}
We see that a collection of $K$ binary encounter events are generated
for {\it each} individual and for {\it each} trap.  The probabilities
have those Bernoulli trials are computed based on the distance from
each individuals home range center and the trap (see calculation
above), and those are housed in the matrix \mbox{\tt probcap}. Our data simulator
function \mbox{\tt simSRC0.fn} will return the full 3-d array if
\mbox{\tt array3d=TRUE} is specified in the function call.  To recover
the 2-d matrix from the 3-d array, and subset the 3-d array to
individuals that were captured, we do this:
{\small
\begin{verbatim}
Y2d<- apply(Y,c(1,3),sum) # sum over the ``replicates'' dimension (2nd margin of the array)
ncaps<-apply(Y2d,1,sum)   # compute how many times each individual was captured
Y<-Y[ncaps>0,,]           # keep those individuals that were captured
\end{verbatim}
}

\section{Fitting an SCR Model in BUGS}
\label{scr0.sec.winbugs1}

Clearly if we somehow knew the value of $N$ then we could fit this
model directly because, in that case, it is a special kind of logistic
regression model - one with a random effect, but that enters into the
model in a peculiar fashion - and also with a distribution (uniform)
which we don't usually think of as standard for random effects models.
So our aim here is to analyze the known-$N$ problem, using our
simulated data, as an incremental step in our progress toward fitting
more generally useful models.

To begin, we use our simulator to grab a data set and then harvest the
elements of the resulting object for further analysis.
\begin{verbatim}
data<-simSCR0.fn(discard0=FALSE,sd=2013)
y<-data$Y
traplocs<-data$traplocs
nind<-nrow(y)
X<-data$traplocs
J<-nrow(X)
y<-rbind(y,matrix(0,nrow=(100-nrow(y)),ncol=J ) )
Xl<-data$xlim[1]
Yl<-data$ylim[1]
Xu<-data$xlim[2]
Yu<-data$ylim[2]
\end{verbatim}

Note that we specify \mbox{\tt discard0 = FALSE} so that we have a
"complete" data set, i.e., one with the all-zero encounter histories
corresponding to uncaptured individuals. Now, within an {\bf R} session, we
can create the {\bf BUGS} model file and fit the model using the following
commands. 
{\small
\begin{verbatim}
cat("
model {
alpha0~dnorm(0,.1)
logit(p0)<- alpha0
alpha1~dnorm(0,.1)
for(i in 1:N){
 s[i,1]~dunif(Xl,Xu)
 s[i,2]~dunif(Yl,Yu)
for(j in 1:J){
d[i,j]<- pow(pow(s[i,1]-X[j,1],2) + pow(s[i,2]-X[j,2],2),0.5)
y[i,j] ~ dbin(p[i,j],K)
p[i,j]<- p0*exp(- alpha1*d[i,j]*d[i,j])
}
}

}
",file = "SCR0a.txt")
\end{verbatim}
}
This model describes the half-normal detection model but it
would be trivial to modify that to various others including the
logistic described above. One consequence of using the half-normal is
that we have to constrain the encounter probability to be in $[0,1]$
which we do here by defining \mbox{\tt alpha0} to be the logit of the
intercept parameter \mbox{\tt p0}. Note that the distance covariate is
computed within the {\bf BUGS} model specification given the matrix of trap
locations, \mbox{\tt X}, which is provided to {\bf WinBUGS} as data.

Next we do a number of organizational activities including bundling
the data for {\bf WinBUGS}, defining some initial values, the parameters to
monitor and some basic MCMC settings.  We choose initial values for
the activity centers ${\bf s}$ by generating uniform random numbers in
the state-space but, for the observed individuals, we replace those
values by each individual's mean trap coordinate for all encounters
{\small
\begin{verbatim}
sst<-cbind(runif(nind,Xl,Xu),runif(nind,Yl,Yu))  # starting values for s
for(i in 1:nind){
if(sum(y[i,])==0) next
sst[i,1]<- mean( X[y[i,]>0,1] )
sst[i,2]<- mean( X[y[i,]>0,2] )
}

data <- list (y=y,X=X,K=K,N=nind,J=J,Xl=Xl,Yl=Yl,Xu=Xu,Yu=Yu)
inits <- function(){
  list (alpha0=rnorm(1,-4,.4),alpha1=runif(1,1,2),s=sst)
}

library("R2WinBUGS")
parameters <- c("alpha0","alpha1")
nthin<-1
nc<-3
nb<-1000
ni<-2000
out <- bugs (data, inits, parameters, "SCR0a.txt", n.thin=nthin,
n.chains=nc, n.burnin=nb,n.iter=ni,debug=TRUE,working.dir=getwd())
\end{verbatim}
}
There is little to say about the preceding basic operations other than
to suggest that the interested reader explore the output and
additional analyses by running the script provided in the {\bf R}
package \mbox{\tt scrbook}.
 We ran $1000$ burn-in and $1000$ after burn-in, 3 chains,
to obtain 3000 posterior samples.  Because we know $N$ for this
particular data set we only have 2 parameters of the detection model
to summarize (\mbox{\tt alpha0} and \mbox{\tt alpha1}).  When the
object \mbox{\tt out} is produced we print a summary of the results as
follows:
{\small
\begin{verbatim}
> print(out,digits=3)
Inference for Bugs model at "SCR0a.txt", fit using WinBUGS,
 3 chains, each with 2000 iterations (first 1000 discarded)
 n.sims = 3000 iterations saved
            mean     sd    2.5%     25%    50%     75%   97.5%  Rhat n.eff
alpha0    -2.496  0.224  -2.954  -2.648  -2.48  -2.340  -2.091 1.013   190
alpha1     2.442  0.419   1.638   2.145   2.44   2.721   3.303 1.005   530
deviance 292.803 21.155 255.597 277.500 291.90 306.000 339.302 1.006   380

For each parameter, n.eff is a crude measure of effective sample size,
and Rhat is the potential scale reduction factor (at convergence, Rhat=1).

DIC info (using the rule, pD = Dbar-Dhat)
pD = -138.8 and DIC = 154.0
DIC is an estimate of expected predictive error (lower deviance is better).
\end{verbatim}
}

We know the data were generated with \mbox{\tt alpha0} $= -2.5$ and
\mbox{\tt alpha1 = -2}. The estimates look reasonably close to those
data-generating values and we probably feel pretty good about the
performance of the Bayesian analysis and MCMC algorithm that {\bf WinBUGS}
cooked-up based on our sample size of 1 data set.  It is worth noting
that the Rhat statistics indicate reasonable convergence but, as a
practical matter, we might choose to run the MCMC algorithm for
additional time to bring these closer to 1.0 and to increase the
effective posterior sample size (\mbox{\tt n.eff}). Other summary output includes
``deviance'' and related things including the deviance information
criterion (DIC). We discuss these things in Chapts. \ref{chapt.mcmc}
and \ref{chapt.gof}.



\section{Unknown N}
\label{scr0.sec.unknownN}

In all real applications $N$ is unknown and that fact is kind of an
important feature of the capture-recapture problem!  We handled this
important issue in Chapt. \ref{chapt.closed} using the method of data augmentation
which we apply here to achieve a realistic analysis of model SCR0. As
with the basic closed population models considered previously, we
formulate the problem here by augmenting our observed data set with a
number of ``all zero'' encounter histories - what we referred to in
Chapt. \ref{chapt.closed} as potential individuals. If $n$ is the number of observed
individuals, then let $M-n$ be the number of potential individuals in
the data set. For the basic $y_{ij}$ data structure (individuals x
traps encounter frequencies) we simply add additional rows of ``all
0'' observations to that data set. This is because such
``individuals'' are unobserved, and therefore necessarily have
$y_{ij}=0$ for all $j$.  A data set, say with 4 traps and 6 individuals,
augmented with 4 pseudo-individuals therefore might look like this:
{\small
\begin{verbatim}
      trap1 trap2 trap3 trap4
 [1,]     1     0     0     0
 [2,]     0     2     0     0
 [3,]     0     0     0     1
 [4,]     0     1     0     0
 [5,]     0     0     1     1
 [6,]     1     0     1     0
 [7,]     0     0     0     0
 [8,]     0     0     0     0
 [9,]     0     0     0     0
[10,]     0     0     0     0
\end{verbatim}
}
We typically have more than 4 traps and, if we're fortunate, many more
individuals in our data set.

For the augmented data, we introduce a set of binary latent variables
(the data augmentation variables), $z_{i}$, and the model is extended
to describe $\Pr(z_{i} = 1)$ which is, in the context of this problem,
the probability that an individual in the augmented data set is a
member of the population that was sampled. In other words, if $z_{i}=1$
for one of the ``all zero'' encounter histories, this is implied to be
a sampling zero whereas observations for which $z_{i}=0$ are
``structural zeros'' under the model.

How big does the augmented data set have to be? We discussed this
issue in Chapt. \ref{chapt.closed} where we noted that the size of the data set is
equivalent to the upper limit of a uniform prior distribution on $N$.
Practically speaking, it should be sufficiently large so that the
posterior distribution for $N$ is not truncated. On the other hand, if
it is too large then unnecessary calculations are being done. An
approach to choosing $M$ by trial-and-error is indicated. You can take
a ballpark estimate of the probability that an individual is captured
at all during the study, say $\tilde{p}$, which is related to the
``per sample'' encounter probability, $p$, by $\tilde{p} = 1-(1-p)^{K}$, obtain $N$ as $n/\tilde{p}$, and then set $M =
2*N$, as a first guess. Do a short MCMC run and then consider whether
you need to increase $M$. See Chapt. \ref{chapt.mcmc} for an
example of this. \citet[][ch. 6]{kery_schaub:2011}
 provide an assessment of choosing $M$ in closed population models.

Analysis by data augmentation removes $N$ as an explicit parameter of
the model. Instead, $N$ is a derived parameter, computed by $N=
\sum_{i=1}^{M} z_{i}$. Similarly, {\it density}, $D$, is also a
derived parameter computed as $D=N/area({\cal S})$. For our
simulator, we're using an $8 \times 8$ state-space and thus we will
compute $D$ as $D=N/64$.

\subsection{Analysis using data augmentation in WinBUGS}

As before we begin by obtaining a data set using our \mbox{\tt
  simSCR0.fn} routine and then harvesting the required data objects
from the resulting data list.  Note that we use the \mbox{\tt
  discard0=TRUE} option this time so that we get a ``real'' data set
with no all-zero encounter histories. After harvesting the data we
produce the {\bf WinBUGS} model specification which now includes $M$
encounter histories including the augmented potential individuals, the
data augmentation parameters $z_{i}$, and the data augmentation
parameter $\psi$.
{\small
\begin{verbatim}
data<-simSCR0.fn(discard0=TRUE,sd=2013)
y<-data$Y
traplocs<-data$traplocs
nind<-nrow(y)
X<-data$traplocs
J<-nrow(X)
Xl<-data$xlim[1]
Yl<-data$ylim[1]
Xu<-data$xlim[2]
Yu<-data$ylim[2]

cat("
model {
alpha0~dnorm(0,.1)
logit(p0)<- alpha0
alpha1~dnorm(0,.1)
psi~dunif(0,1)

for(i in 1:M){
 z[i] ~ dbern(psi)
 s[i,1]~dunif(Xl,Xu)
 s[i,2]~dunif(Yl,Yu)
for(j in 1:J){
d[i,j]<- pow(pow(s[i,1]-X[j,1],2) + pow(s[i,2]-X[j,2],2),0.5)
y[i,j] ~ dbin(p[i,j],K)
p[i,j]<- z[i]*p0*exp(- alpha1*d[i,j]*d[i,j])
}
}
N<-sum(z[])
D<-N/64
}
",file = "SCR0a.txt")
\end{verbatim}
}

To prepare our data we have to augment the data matrix \mbox{\tt y}
with $M-n$ all-zero encounter histories, we have to create starting
values for the variables $z_{i}$ and also the activity centers ${\bf
  s}_{i}$ of which, for each, we require $M$ values. Otherwise the
remainder of the code for bundling the data, creating initial values
and executing {\bf WinBUGS} looks much the same as before except with more
or differently named arguments.
{\small
\begin{verbatim}
## Data augmentation stuff
M<-200
y<-rbind(y,matrix(0,nrow=M-nind,ncol=ncol(y)))
z<-c(rep(1,nind),rep(0,M-nind))

sst<-cbind(runif(M,Xl,Xu),runif(M,Yl,Yu))  # starting values for s
for(i in 1:nind){
if(sum(y[i,])==0) next
sst[i,1]<- mean( X[y[i,]>0,1] )
sst[i,2]<- mean( X[y[i,]>0,2] )
}
data <- list (y=y,X=X,K=K,M=M,J=J,Xl=Xl,Yl=Yl,Xu=Xu,Yu=Yu)
inits <- function(){
  list (alpha0=rnorm(1,-4,.4),alpha1=runif(1,1,2),s=sst,z=z)
}

library("R2WinBUGS")
parameters <- c("alpha0","alpha1","N")
nthin<-1
nc<-3
nb<-1000
ni<-2000
out <- bugs (data, inits, parameters, "SCR0a.txt", n.thin=nthin,n.chains=nc,
 n.burnin=nb,n.iter=ni,debug=TRUE,working.dir=getwd())
\end{verbatim}
}

{\bf Remarks}:  (1) Note the differences in this new {\bf WinBUGS} model
with that appearing in the known-$N$ version.  (2) Also the input data
has changed - the augmented data set has more rows of
all-zeros. Previously we knew that $N=100$ but in this analysis we
pretend not to know $N$, but think that $N=200$ is a good upper-bound;
(3) Population size $N({\cal S})$ is a derived parameter, being computed by
summing up all of the data augmentation variables $z_{i}$ (as we've
done previously in Chapt. \ref{chapt.closed}); (4) Density, $D\equiv D({\cal S})$, is also a derived
parameter. Summarizing the output from {\bf WinBUGS} produces:
{\small
\begin{verbatim}
> print(out1,digits=2)
Inference for Bugs model at "SCR0a.txt", fit using WinBUGS,
 3 chains, each with 2000 iterations (first 1000 discarded)
 n.sims = 3000 iterations saved
           mean    sd   2.5%    25%    50%    75%  97.5% Rhat n.eff
alpha0    -2.57  0.23  -3.04  -2.72  -2.56  -2.41  -2.15 1.01   320
alpha1     2.46  0.42   1.63   2.16   2.46   2.73   3.33 1.02   120
N        113.62 15.73  86.00 102.00 113.00 124.00 147.00 1.01   260
D          1.78  0.25   1.34   1.59   1.77   1.94   2.30 1.01   260
deviance 302.60 23.67 261.19 285.47 301.50 317.90 354.91 1.00  1400

For each parameter, n.eff is a crude measure of effective sample size,
and Rhat is the potential scale reduction factor (at convergence, Rhat=1).

DIC info (using the rule, pD = var(deviance)/2)
pD = 279.9 and DIC = 582.5
DIC is an estimate of expected predictive error (lower deviance is better).
\end{verbatim}
}

The column labeled ``MC error'' is the Monte Carlo error - the error
inherent in the attempt to compute these posterior summaries by
MCMC
(see secs.  for discussion of this quantity
\ref{glms.sec.convergence} \ref{mcmc.sec.mcmcsummary}).
It is desirable to run the Markov chain algorithm long enough so
as to reduce the MC error to a tolerable level. What constitutes
tolerable is up to the investigator. Certainly less than 1\% is called
for. As a general rule, Rhat gets closer to 1 and MC error decreases
toward 0 as the number of iterations increases.  We see that the
estimated parameters ($\alpha_0$ and $\alpha1$) are comparable to the
previous results obtained for the known-$N$ case, and also not too
different from the data-generating values. The posterior of $N$
overlaps the data-generating value substantially with a mean of
$113.62$.  To obtain these results we fitted the true data-generating
model, that based on the half-normal detection model, to a single
simulated data set. For fun and excitement we fit the {\it wrong}
model, one with a logistic-linear detection model
(Eq. \ref{scr0.eq.logit}),
to the same  
data set. This is easily achieved by modifying the {\bf WinBUGS} model
specification above, although we provide the {\bf R} script in the
{\bf R} package \mbox{\tt scrbook}.
Those results are given below. We see that the estimate of
$N$, the main parameter of interest, is very similar to that obtained
under the correct model, convergence is worse (as measured by Rhat)
which may not have anything to do with the model being wrong,
and the posterior deviance favors the correct model (it is smaller) while the DIC does not.
We consider 
 the effectiveness of DIC for carrying-out model selection in chapter
\ref{chapt.gof}.
{\small
\begin{verbatim}
> print(out2,digits=2)
Inference for Bugs model at "SCR0a.txt", fit using WinBUGS,
 3 chains, each with 2000 iterations (first 1000 discarded)
 n.sims = 3000 iterations saved
           mean    sd   2.5%    25%    50%    75%  97.5% Rhat n.eff
alpha0    -1.59  0.27  -2.16  -1.77  -1.58  -1.42  -1.07 1.05    60
beta       3.77  0.43   2.92   3.48   3.79   4.05   4.66 1.04    70
N        122.57 18.67  90.00 109.00 122.00 135.00 163.00 1.00  3000
D          1.92  0.29   1.41   1.70   1.91   2.11   2.55 1.00  3000
deviance 312.67 22.43 271.00 297.20 311.50 327.00 359.60 1.02   130

For each parameter, n.eff is a crude measure of effective sample size,
and Rhat is the potential scale reduction factor (at convergence, Rhat=1).

DIC info (using the rule, pD = var(deviance)/2)
pD = 247.5 and DIC = 560.1
DIC is an estimate of expected predictive error (lower deviance is better).
\end{verbatim}
}

\subsection{Use of other BUGS engines: JAGS}

There are two other popular {\bf BUGS} engines in widespread use: {\bf
  OpenBUGS} \citep{thomas_etal:2006} and {\bf JAGS}
\citep{plummer:2003}. Both of these are easily called from {\bf
  R}. {\bf OpenBUGS} can be used instead of {\bf WinBUGS} by changing
the package option in the \mbox{\tt bugs} call to \mbox{\tt
  package=OpenBUGS}.  {\bf JAGS} can be called using the function
\mbox{\tt jags()} in package \mbox{\tt R2JAGS} which has nearly the
same arguments as \mbox{\tt bugs()}.  We prefer to use the {\bf R}
library \mbox{\tt rjags} \citep{plummer:2009} which has a slightly
different implementation that we demonstrate here as we reanalyze the
simulated data set in the previous section (note: the same {\bf R}
commands are used to generate the data and package the data, inits and
parameters to monitor). The function \mbox{\tt jags.model} is used to
initialize the model and run the MCMC algorithm for an adaptive
burn-in period.  Then the Markov chains are updated using \mbox{\tt
  coda.samples()} to obtain posterior samples for analysis, as
follows:
\begin{verbatim}
jm<- jags.model("SCR0a.txt", data=data, inits=inits, n.chains=nc,
                 n.adapt=nb))
jm<- coda.samples(jm, parameters, n.iter=ni-nb, thin=nthin)
\end{verbatim}
We find that {\bf JAGS} seems to be 20-30\% faster for the basic SCR
model which the reader can evaluate using the function \mbox{\tt
  SCR0bayes} in the {\bf R} package \mbox{\tt scrbook}.



\section{Wolverine Camera Trapping Study}
\label{scr0.sec.wolverine}

We provide an analysis here of A. Magoun's wolverine data
\citep{magoun_etal:2011, royle_etal:2011jwm}. The study took place in SE
Alaska (Fig. \ref{scr0.fig.wolverinelocs}) where 37 cameras were
operational for variable periods of time (min = 5 days, max = 108
days, median = 45 days).  A consequence of this is that the binomial
sample size $K$ (see Eq. \ref{scr0.eq.bin})
 is variable for each camera. Thus, we
must provide a matrix of sample sizes as data to {\bf BUGS} and modify the
model specification in sec. \ref{scr0.sec.unknownN}
accordingly. Our treatment of the
data here is based on the analysis of  \citet{royle_etal:2011jwm}.

\begin{figure}
\begin{center}
\includegraphics[height=3in]{Ch4/figs/wolverinelocs}
\end{center}
\caption{Wolverine camera trap locations from \citet{magoun_etal:2011}.}
\label{scr0.fig.wolverinelocs}
\end{figure}

To carry-out an analysis of these data, we require the matrix of trap
coordinates and the encounter history data.  We store data in an the
``scr flat format'' (see sec.  \ref{scr0.sec.formats} above), an
efficient file format which is easily manipulated and also used as the
input file format in {\bf SPACECAP} \citep{gopalaswamy_etal:2012} and
in the {\bf R} package \mbox{\tt SCRbayes} \citep{russell_etal:2012}.
To illustrate this format, the wolverine data are available in the
package \mbox{\tt scrbook} by typing:
\begin{verbatim}
data(wolverine)
\end{verbatim}
which contains a list having elements \mbox{\tt wcaps} and
\mbox{\tt wtraps}.
The ``encounter data file''
\mbox{\tt wcaps}  has 3 columns and 115 rows, each representing a
unique encounter event including the trap identity, the individual
identity and the sample occasion index (\mbox{\tt sample}).
The first 10 rows of this matrix are as
follows:
{\small
\begin{verbatim}
> wolverine$wcaps[1:10,]
       trapid individual sample
  [1,]      1          2    127
  [2,]      1          2    128
  [3,]      1          2    129
  [4,]      1         18    130
  [5,]      2          3    106
  [6,]      2         18    104
  [7,]      5          5     73
  [8,]      5          5     89
  [9,]      6         18    117
 [10,]      6         18    118
\end{verbatim}
}
Each row is a unique 
individual/trap encounter, and the 3 variables (columns) are: 
\mbox{\tt trapid} -- an
integer that runs from \mbox{\tt 1:ntraps}, \mbox{\tt individual} runs from
\mbox{\tt 1:nind} and
\mbox{\tt sample} 
runs from \mbox{\tt 1:nperiods}. Often (as the case here) \mbox{\tt
  sample} 
will
correspond to daily sample intervals. The variable \mbox{\tt trapid} will have to
correspond to the row of a matrix containing the trap coordinates - in
this case the file \mbox{\tt wtraps} which we describe further below.

Note that the information provided in this encounter data file
\mbox{\tt wcaps}
does not represent a completely informative summary
of the data. For example, if no individuals were captured in a certain
trap or during a certain period, then this compact data format will
have no record. Thus we will need to know ntraps and nperiods when
reformatting this SCR data format into a 2-d encounter frequency
matrix or 3-d array. In addition, the encounter data file does not
provide information about which periods each trap was operated. This
additional information is also necessary as the trap-specific sample
sizes must be passed to {\bf BUGS} as data. We provide this information in a
2nd data file, along with the trap coordinates, in the 
 ``trap deployment'' file which is described
below.

For our purposes we
need to convert the \mbox{\tt wcaps} file
into the $n \times J$ array of
binomial encounter frequencies, although more general models might
require an encounter-history formulation of the model which requires a
full 3-d array.  To obtain our $n \times J$ encounter frequency
matrix, we do this the hard way by first converting the encounter data
file into a 3-d array and then summarize to trap totals. We have a
handy function \mbox{\tt SCR23darray.fn} which takes the compact
encounter data file with optional arguments ntraps and nperiods, and
converts it to a 3-d array, and then we use the {\bf R} function
\mbox{\tt apply} to summarize over the ``sample'' period dimension (by
convention here, this is the 2nd dimension). To apply this to the
wolverine
data in order to compuate the 3-d array we do this:
{\small
\begin{verbatim}
y3d <-SCR23darray.fn(wolverine$wcaps,wolverine$wtraps)
y <- apply(y3d,c(1,3),sum)
\end{verbatim}
}
See the help file for more information on \mbox{\tt SCR23darray.fn}.
The 3-d array is necessary to fit certain types
of models (e.g., behavioral response) and this is why we sometimes
will require this maximally informative 3-d data format but, here, we
analyze the summarized data.



The other important information needed to fit SCR models is the
``trap deployment'' file
which provides the additional information
not contained in the encounter data file. The traps file has \mbox{\tt
  nperiods} $+ 3$ columns. The first column is assumed to be a trap identifier,
columns 2 and 3 are the easting and northing coordinates (assumed to
be in a Euclidean coordinate system), and columns 4 to (\mbox{\tt nperiods} + 3)
are binary indicators of whether each trap was operational in each
time period. The first 10 rows (out of 37) and 10 columns (out of 167)
of the trap deployment file for the wolverine data are:
{\small
\begin{verbatim}
> wolverine$wtraps[1:10,1:10]

   Easting Northing 1 2 3 4 5 6 7 8 
1   632538  6316012 0 0 0 0 0 0 0 0
2   634822  6316568 1 1 1 1 1 1 1 1
3   638455  6309781 0 0 0 0 0 0 0 0
4   634649  6320016 0 0 0 0 0 0 0 0
5   637738  6313994 0 0 0 0 0 0 0 0
6   625278  6318386 0 0 0 0 0 0 0 0
7   631690  6325157 0 0 0 0 0 0 0 0
8   632631  6316609 0 0 0 0 0 0 0 0
9   631374  6331273 0 0 0 0 0 0 0 0
10  634068  6328575 0 0 0 0 0 0 0 0
\end{verbatim}
}
This tells us that trap 2 was operated in periods (days) 1-7 but the other
traps were not operational during those periods. It is extremely
important to recognize that each trap was operated for a variable
period of time and thus the binomial "sample size" is different for
each, and this needs to be accounted for in the {\bf BUGS} model specification.
To compute the vector of sample sizes $K$, and extract the trap
locations,  we do this:
\begin{verbatim}
traps<- wolverine$wtraps
traplocs<- traps[,1:2]
K<- apply(traps[,3:ncol(traps)],1,sum)
\end{verbatim}
This results in a matrix traplocs which contains the coordinates of
each trap and a vector $K$ containing the number of days that each trap
was operational. We now have all the information required to fit a
basic SCR model in {\bf BUGS}.

Summarizing these data files for the wolverine study, we see that 21
unique individuals were captured a total of 115 times. Most
individuals were captured 1-6 times, with 4, 1, 4, 3, 1, and 2
individuals captured 1-6 times, respectively.  In addition, 1
individual was captured each 8 and 14 times and 2 individuals each
were captured 10 and 13 times.  The number of unique traps that
captured a particular individual ranged from 1-6, with 5, 10, 3, 1, 1,
and 1 individual captured in each of 1-6 traps, respectively, for a
total of 50 unique wolverine-trap encounters.  These numbers might be
hard to get your mind around whereas some tabular summary is often
more convenient. For that it seems natural to tabulate individuals by
trap and total encounter frequencies. The spatial information in SCR
data is based on multi-trap captures\footnote{I will add more 
context here on revision about spatial recaptures, lost recaptures,
ordinary recaptures. Function \mbox{\tt SCRsmy} in \mbox{\tt
  scrbook}}, 
and so, it is informative to
understand how many unique traps each individual is captured in. At
the same, it is useful to understand how many total captures we have
of each individual because this is, in an intuitive sense, the
effective sample size.  So, we reproduce Table 1 from
\citet{royle_etal:2011jwm} which shows the trap and total encounter
frequencies:

\begin{table} [htp]
  \caption{Individual frequencies of capture for wolverines captured
    in camera traps in Southeast Alaska in 2008. Rows index unique
    trap frequencies and columns represent total number of captures
    (e.g., we captured 4 individuals 1 time, necessarily in only 1
    trap; we captured 3 individuals 3 times but in 2 different traps)}
\centering
\begin{tabular}{c c c c c c c c c c c}
\hline
 & & & & & & & &  No.&of&captures \\
\hline
No. of traps & 1 & 2 & 3 & 4 & 5 & 6 & 8 & 10 &13 &14 \\
\hline
1 & 4 & 1 & 0 & 0 & 0 & 0 & 0 & 0 & 0 & 0 \\
2 & 0 & 0 & 3 & 3 & 0 & 2 & 1 & 2 & 0 & 0 \\
3 & 0 & 0 & 1 & 1 & 0 & 0 & 0 & 0 & 0 & 1 \\
4 & 0 & 0 & 0 & 0 & 0 & 0 & 0 & 0 & 1 & 0 \\
5 & 0 & 0 & 0 & 0 & 1 & 0 & 0 & 0 & 0 & 0 \\
6 & 0 & 0 & 0 & 0 & 0 & 0 & 0 & 0 & 1 & 0 \\
\hline
\end{tabular}
\end{table}

\subsection{Fitting the model in WinBUGS}

For illustrative purposes here we fit the simplest SCR model with the
half-normal distance function although we revisit these data with more
complex models in later chapters. The model is summarized by the
following 3 components:
\begin{itemize}
\item[(1)] $y_{ij}|{\bf s}_{i} \sim \mbox{Bin}(K, z_{i}\; p_{ij})$
\item[(2)] $p_{ij} = p_{0} \exp(-\alpha1 \; ||{\bf s}_{i}-x_{j}||^2)$
\item[(3)] $ {\bf s}_{i} \sim \mbox{Unif}({\cal S})$
\item[(4)] $ z_{i} \sim \mbox{Bern}(\psi)$
\end{itemize}
We assume customary flat priors on the structural (hyper-) parameters
of the model, $\alpha_{0} = \mbox{logit}(p_{0})$, $\alpha1$ and $\psi$.  It remains to define the
state-space ${\cal S}$. For this, we nested the trap array (Fig.
\ref{scr0.fig.wolverinelocs}) in a
a rectangular state-space extending $20$ km beyond the traps in each cardinal
direction.  We also considered larger state-spaces up to 50 km to
evaluate that choice.  The buffer of the state space should be larger
enough so that individuals beyond the state-space boundary are not
likely to be encountered. Thus some knowledge of typical space usage
patterns of the species is useful.  For the analysis, 
we scaled the coordinate system 
so that a unit distance was equal to $10$ km, producing a rectangular
state-space of dimension $9.88 \times 10.5$ units ($area = 10374$ km$^2$)
within which the trap array was nested. As a general rule, we
recommend scaling the state-space so that it is defined near the
origin $(x,y)=(0,0)$. While the scaling of the coordinate system is
theoretically irrelevant, a poorly scaled coordinate system can
produce Markov chains that mix poorly.  For the scaled coordinate
system we fit models for various choices of a rectangular state-space
based on 
buffers from 1.0 to 5.0 units on the scaled coordinate system (10 km to
50 km). In the {\bf R} package \mbox{\tt scrbook} we provide a
function
\mbox{\tt wolvSCR0.fn} which will fit the basic SCR model. For
example, to fit the model in 
{\bf WinBUGS} using data augmentation with $M=300$ potential individuals,
using 3 Markov chains each of 12000 total iterations, discarding the
first 2000 as burn-in, we execute the following {\bf R} commands:
{\small
\begin{verbatim}
library("scrbook")
data(wolverine)
traps<-wolverine$wtraps
y3d <-SCR23darray.fn(wolverine$wcaps,wolverine$wtraps)
toad<-wolvSCR0.fn(y3d,traps,nb=12000,ni=2000,delta=1,M=300)
\end{verbatim}
}
The argument $\delta$ determines the buffer size of the state-space.
Note that this analysis takes 
between 1-2 hours on many machines so we recommend trying it out with
lower values of $M$ and fewer iterations.
The output
follows (note, we have a parameter ``sigma'' which we discuss
shortly)\footnote{Final as of 1/11/2012. 
output saved in \mbox{\tt wolv-buffer-study.txt}}:

{\small
\begin{verbatim}
All based on 3 chains, 12k iters, 2k burn, 30k total
Buffer = 10 km
           mean    sd   2.5%    25%    50%    75%  97.5% Rhat n.eff
psi        0.13  0.03   0.08   0.11   0.13   0.15   0.20    1 10000
sigma      0.65  0.06   0.55   0.61   0.64   0.68   0.76    1  1800
p0         0.06  0.01   0.04   0.05   0.06   0.06   0.08    1 20000
N         39.63  6.70  29.00  35.00  39.00  44.00  54.00    1  7100
D          5.92  1.00   4.33   5.22   5.82   6.57   8.06    1  7100
beta       1.23  0.21   0.85   1.08   1.22   1.36   1.66    1  1800
deviance 410.05 12.06 388.70 401.50 409.20 417.80 435.60    1 22000

Buffer = 15 km
 n.sims = 30000 iterations saved
           mean    sd   2.5%    25%    50%    75%  97.5% Rhat n.eff
psi        0.16  0.04   0.10   0.14   0.16   0.19   0.25    1  3800
sigma      0.64  0.06   0.54   0.60   0.64   0.67   0.76    1   510
p0         0.06  0.01   0.04   0.05   0.06   0.06   0.08    1 17000
N         48.77  9.19  34.00  42.00  48.00  54.00  69.00    1  3300
D          5.78  1.09   4.03   4.98   5.69   6.40   8.18    1  3300
beta       1.25  0.21   0.86   1.10   1.24   1.39   1.70    1   510
deviance 411.00 12.16 389.50 402.40 410.30 418.70 437.00    1  5400

Buffer = 20 km
           mean    sd   2.5%    25%    50%    75%  97.5% Rhat n.eff
psi        0.20  0.05   0.12   0.17   0.20   0.23   0.30    1 16000
sigma      0.64  0.06   0.54   0.60   0.63   0.67   0.76    1  1200
p0         0.06  0.01   0.04   0.05   0.06   0.06   0.08    1  1900
N         59.84 11.89  40.00  51.00  59.00  67.00  86.00    1 20000
D          5.77  1.15   3.86   4.92   5.69   6.46   8.29    1 20000
beta       1.26  0.21   0.87   1.11   1.25   1.40   1.71    1  1200
deviance 411.01 12.36 389.10 402.30 410.20 418.80 437.50    1  1500

Buffer = 25 km
           mean    sd   2.5%    25%    50%    75%  97.5% Rhat n.eff
psi        0.24  0.05   0.15   0.20   0.24   0.28   0.36    1  3400
sigma      0.64  0.05   0.54   0.60   0.63   0.67   0.75    1  3600
p0         0.06  0.01   0.04   0.05   0.06   0.06   0.08    1  5000
N         72.40 14.72  47.00  62.00  71.00  81.00 105.00    1  2700
D          5.79  1.18   3.76   4.96   5.67   6.47   8.39    1  2700
beta       1.26  0.21   0.88   1.12   1.25   1.40   1.71    1  3600
deviance 411.35 12.23 389.70 402.70 410.55 419.20 437.20    1 30000

Buffer = 30 km
           mean    sd   2.5%    25%    50%    75%  97.5% Rhat n.eff
psi        0.29  0.06   0.18   0.24   0.28   0.33   0.43    1  3100
sigma      0.63  0.05   0.54   0.60   0.63   0.67   0.75    1  5600
p0         0.06  0.01   0.04   0.05   0.06   0.06   0.08    1 11000
N         86.42 17.98  56.00  74.00  85.00  97.00 126.02    1  3900
D          5.82  1.21   3.77   4.98   5.72   6.53   8.49    1  3900
beta       1.27  0.21   0.88   1.12   1.26   1.41   1.71    1  5600
deviance 411.06 12.37 389.20 402.50 410.20 418.90 437.60    1 10000

Buffer = 35 km
           mean    sd   2.5%    25%    50%    75%  97.5% Rhat n.eff
psi        0.34  0.08   0.21   0.29   0.34   0.39   0.50    1 30000
sigma      0.63  0.05   0.54   0.60   0.63   0.67   0.75    1  4500
p0         0.06  0.01   0.04   0.05   0.06   0.06   0.08    1 24000
N        101.79 21.54  65.00  87.00 100.00 115.00 148.00    1 30000
D          5.85  1.24   3.74   5.00   5.75   6.61   8.51    1 30000
beta       1.27  0.21   0.89   1.12   1.25   1.40   1.70    1  4500
deviance 411.10 12.20 389.50 402.40 410.30 418.90 437.20    1 22000

Buffer = 40 km
           mean    sd   2.5%    25%    50%    75%  97.5% Rhat n.eff
psi        0.39  0.09   0.24   0.33   0.39   0.45   0.60 1.01   480
sigma      0.64  0.05   0.54   0.60   0.63   0.67   0.75 1.01   410
p0         0.06  0.01   0.04   0.05   0.06   0.06   0.08 1.00 21000
N        118.05 26.14  75.00 100.00 116.00 133.00 178.00 1.01   450
D          5.87  1.30   3.73   4.97   5.76   6.61   8.84 1.01   450
beta       1.27  0.21   0.89   1.12   1.25   1.40   1.72 1.01   410
deviance 411.37 12.35 389.30 402.60 410.60 419.30 437.50 1.00  9700

Buffer = 45 km
           mean    sd   2.5%    25%    50%    75%  97.5% Rhat n.eff
psi        0.45  0.10   0.28   0.38   0.44   0.51   0.66    1  3600
sigma      0.64  0.05   0.54   0.60   0.63   0.67   0.75    1 10000
p0         0.06  0.01   0.04   0.05   0.06   0.06   0.08    1  8100
N        134.43 28.68  85.00 114.00 132.00 153.00 196.00    1  3300
D          5.83  1.24   3.68   4.94   5.72   6.63   8.50    1  3300
beta       1.26  0.21   0.88   1.11   1.24   1.39   1.69    1 10000
deviance 411.36 12.19 389.60 402.70 410.60 419.10 437.30    1  9400

Buffer = 50 km
           mean    sd   2.5%    25%    50%    75%  97.5% Rhat n.eff
psi        0.51  0.11   0.31   0.43   0.50   0.57   0.74    1  3200
sigma      0.63  0.05   0.54   0.60   0.63   0.67   0.75    1  4700
p0         0.06  0.01   0.04   0.05   0.06   0.06   0.08    1  3300
N        151.61 31.65  96.00 129.00 149.00 172.00 221.00    1  3400
D          5.79  1.21   3.66   4.92   5.69   6.56   8.43    1  3400
beta       1.27  0.21   0.89   1.12   1.25   1.40   1.70    1  4700
deviance 410.81 12.18 389.20 402.30 410.10 418.50 436.70    1 30000

Buffer = 55 km 
           mean    sd   2.5%    25%    50%    75%  97.5% Rhat n.eff
psi        0.56  0.12   0.35   0.48   0.55   0.64   0.82 1.01   260
sigma      0.64  0.05   0.54   0.60   0.63   0.67   0.76 1.00  1600
p0         0.06  0.01   0.04   0.05   0.06   0.06   0.08 1.00 30000
N        169.28 35.81 108.00 143.00 166.00 192.00 247.00 1.01   260
D          5.73  1.21   3.66   4.84   5.62   6.50   8.36 1.01   260
beta       1.25  0.21   0.88   1.11   1.24   1.39   1.69 1.00  1600
deviance 411.28 12.38 389.40 402.60 410.50 419.10 437.50 1.00 26000
\end{verbatim}
}

We see that the estimated density is roughly consistent as we increase
the state-space buffer from $15$ to $50$ $km$. We do note that the data
augmentation parameter $\psi$ (and, correspondingly, $N$) increase with
the size of the state space in accordance with the deterministic
relationship $N= D*A$. However, density is constant more or less as we
increase the size of the state-space beyond a certain point.  For the
10 $km$ state-space buffer, we see a slight effect on the posterior
distribution of $D$. This is not a bug but rather a feature. As we noted
above, the state-space is part of the model.


\subsection{Thoughts on the Wolverine Analysis}

Our point estimate of wolverine density from this study, using the
posterior mean from the state-space based on the 20
$km$ buffer, is 
approximately $5.77$ individuals/1000 $km^2$ with  a 95\% posterior
interval of $[3.86, 8.29]$. Density is estimated imprecisely
which might not be surprising given the low sample size ($n=21$
individuals!). This seems to be a basic feature of carnivore studies
although it should not (in our view) preclude the study of their
populations nor attempts to estimate density or vital rates.

One thing we haven't talked about yet is that we can calibrate the
desired size of the state-space by looking at the estimated home range
radius of the species. For some models it is possible to convert the
parameter $\alpha1$ directly into the home range radius (sec. 
XXX MISSING XYZ). For the half-normal model we interpret the half-normal scale
parameter $\sigma$ which is related to $\alpha1$ by $\alpha1 =
1/(2\sigma^2)$ as the radius of a bivariate normal movement model. 
In this case $\sigma = 1.82$ standardized units = 18.2 $km$ which 
translates into a home range area of XXXX MISSING XXXXX. 

It is worth thinking about this model, and these estimates, computed
under a rectangular state space roughly centered over the trapping
array (Fig. \ref{scr0.fig.wolverinelocs}).
Does it make sense to define the state-space to
include, for example, ocean? What are the possible consequences of
this? What can we do about it?  There's no reason at all that the
state space has to be a regular polygon -- we defined it as such here
strictly for convenience and for ease of implementation in {\bf WinBUGS}
where it enables us to specify the prior for the activity centers as
uniform priors for each coordinate.  While it would be possible to
define a more realistic state-space using some general polygon GIS coverage, it
might take some effort to implement that in the {\bf BUGS} language
but it is not difficult to devise custom MCMC algorithms to do that
(see Chapt. \ref{chapt.mcmc}).
Alternatively, we recommend
using a discrete representation of the state-space -- i.e., approximate
${\cal S}$ by a grid of $G$ points. We discuss this in sec. 
\ref{scr0.sec.discrete}.


\section{Constructing Density Maps}
\label{scr0.sec.mapping}

One of the most useful aspects of SCR models is that they are
parameterized in terms of individual locations - i.e., {\it where}
each individual lives -- and, thus, we can compute many useful or
interesting summaries of the activity centers.  For example, we can
make a spatial density plot by tallying up the number of activity
centers ${\bf s}_{i}$ in boxes of arbitrary size and then producing a
nice multi-color spatial plot of those which, we find, increases the
acceptance probability of your manuscripts by a substantial amount.
We discussed in Chapt. \ref{chapt.glms} the idea of estimating derived
parameters from MCMC output. In SCR models, there are many derived
parameters that are functions of the latent point locations $({\bf
  s}_{1},\ldots, {\bf s}_{N})$. In the present context, the number of
individuals living in any well-defined polygon is a derived
parameter. Specifically, let $B({\bf x})$ indicate a box centered at
${\bf x}$ then
\[
N({\bf x})=\sum_{i} I({\bf s}_{i} \in B({\bf x}))
\]
is the population size of box $B({\bf x})$, and $D({\bf x}) = N({\bf
  x})/||B({\bf x})||$ is the local density. These are just ``derived
parameters'' (see Chapt.  \ref{chapt.glms}) which are estimated from
MCMC output using the appropriate Monte Carlo average. One thing to be
careful about, in the context of models in which $N$ is unknown, is
that, for each MCMC iteration $m$, we only tabulate those activity
centers which correspond to individuals in the sampled
population. i.e., for which the data augmentation variable $z_{i} =
1$.  In this case, we take all of the output for MCMC iterations
$m=1,2,\ldots,\mbox{\tt niter}$ and compute this summary:
\[
   N({\bf x},m) = \sum_{z_{i,m}=1} I(s_{i,m} \in B({\bf x}))
\]
Thus, $N({\bf x},1),N({\bf x},2),\dots,$ is the Markov chain for
parameter $N({\bf x})$.  In what follows we will provide a set of {\bf
  R} commands for doing this calculations and making a basic image
plot from the MCMC output.

{\flushleft \bf Step 1:} Define the center points of each box, $B({\bf
  x})$, or point at which local density will be estimated:
\begin{verbatim}
xg<-seq(Xl,Xu,,50)
yg<-seq(Yl,Yu,,50)
\end{verbatim}

{\flushleft \bf Step 2:} Extract the MCMC histories for the activity
centers and the data augmentation variables.  Note that these are each
$N \times \mbox{\tt niter}$ matrices:
\begin{verbatim}
Sxout<-out$sims.list$s[,,1]
Syout<-out$sims.list$s[,,2]
z<-out$sims.list$z
\end{verbatim}

{\flushleft \bf Step 3:} We associate each coordinate with the proper
box using the {\bf R} command \mbox{\tt cut()}. Note that we keep only
the activity centers for which $z=1$ (i.e., individuals that belong to
the population of size $N$):
\begin{verbatim}
Sxout<-cut(Sxout[z==1],breaks=xg,include.lowest=TRUE)
Syout<-cut(Syout[z==1],breaks=yg,include.lowest=TRUE)
\end{verbatim}

{\flushleft \bf Step 4:} Use the \mbox{\tt table()} command to tally
up how many activity centers are in each $B(x)$:
\begin{verbatim}
Dn<-table(Sxout,Syout)
\end{verbatim}

{\flushleft \bf Step 5:} Use the \mbox{\tt image()} command to display
the resulting matrix.
\begin{verbatim}
image(xg,yg,Dn/nrow(z),col=terrain.colors(10))
\end{verbatim}
Praise the Lord! This map is somewhat useful or at least it looks
pretty and will facilitate the publication of your papers.

It is worth emphasizing here that density maps will not usually appear
uniform despite that we have assumed that activity centers are
uniformly distributed. This is because the observed encounters of
individuals provide direct information about the location of the
$i=1,2,\ldots,n$ activity centers and thus their ``estimated''
locations will be affected by the observations. In a limiting sense,
were we to sample space intensely enough, every individual would be
captured a number of times and we would have considerable information
about all $N$ point locations. Consequently, the uniform prior would
have almost no influence at all on the estimated density surface in
this limiting situation. Thus, in practice, the influence of the
uniformity assumption increases as the fraction of the population
encountered decreases.

{\bf On the non-intuitiveness of \mbox{\tt image()} } -- the {\bf R}
function \mbox{\tt image()} might
not be very intuitive to some -- it plots $M[1,1]$ in the lower left
corner. If you want $M[]$ to be plotted ``as
you look at it'' then $M[1,1]$ should be in the upper left corner.  We
have a function \mbox{\tt rot()} which does that. If you do \mbox{\tt image(rot(M))} then it
puts it on the monitor as if it was a map you were looking at.  You
can always specify the $x$ and $y-$ labels explicitly as we did above.

{\bf Spatial dot plots } -- Now here is a cruder version based on the
``spatial dot map'' function \mbox{\tt spatial.plot}, which uses
the function \mbox{\tt image.scale()}.
The \mbox{\tt spatial.plot} function requires arguments of point
locations and the resulting value to be displayed:
\begin{verbatim}
spatial.plot<- function(x,y){
 nc<-as.numeric(cut(y,20))
 plot(x,pch=" ")
 points(x,pch=20,col=topo.colors(20)[nc],cex=2)
 image.scale(y,col=topo.colors(20))
}
# To execute the function do this:
spatial.plot(cbind(xg,yg), Dn/nrow(z))
\end{verbatim}

\subsection{Example: Wolverine density map. }

The {\bf R} commands for producing density maps from MCMC output of
spatial capture-recapture models is provided in the {\bf R} function
\mbox{\tt SCRdensity} in the package \mbox{\tt scrbook}. 
We used the posterior output from the wolverine model fitted previous
to compute a relatively coarse version of a density map, using a $10 \times
10$ grid (Fig. \ref{scr0.fig.density10x10}) and using a $30 \times 30$
grid (Fig. \ref{scr0.fig.density20x20}). The {\bf R} commands for
producing such a plot (for short MCMC run) are as follows:
{\small
\begin{verbatim}
library("scrbook")
data(wolverine)
traps<-wolverine$wtraps
y3d <-SCR23darray.fn(wolverine$wcaps,wolverine$wtraps)
# this takes 341 seconds on a standard CPU circa 2011
unix.time(bln<-wolvSCR0.fn(y3d,traps,nb=1000,ni=2000,delta=1,M=100))
Sx<-bln$sims.list$s[,,1]
Sy<-bln$sims.list$s[,,2]
w<- bln$sims.list$w
obj<-list(Sx=Sx,Sy=Sy,w=w)
tmp<-SCRdensity(obj,scalein=100,scaleout=100)
\end{verbatim}
In these figures density is
expressed in units of individuals per $100$ $km^2$, while the area of
the pixels is about 103.7 $km^2$ and 11.5 $km^2$, respectively. That
calculation is based on:
\begin{verbatim}
> total.area<- (Yu-Yl)*(Xu-Xl)*100
> total.area/(10*10)
[1] 103.7427
> total.area/(30*30)
[1] 11.52697
\end{verbatim}

A couple of things are worth noting: First is that as we move away
from ``where the data live'' - away from the trap array - we see that
the density approaches the mean density. This is a property of the
estimator as long as the ``detection function'' decreases sufficiently
rapidly as a function of distance.
Relatedly, it is also a property of statistical smoothers
such as splines, kernel smoothers, and regression smoothers -
predictions tend toward the global mean as the influence of data
diminishes. Another way to think of it is that it is a consequence of
the prior - which imposes uniformity, and as you get far away from the
data, the predictions tend to the prior. The other thing to note about
this map is that density is not $0$ over water (although the coastline
is not shown). This might be perplexing
to some who are fairly certain that wolverines do not like
water. However, there is nothing about the model that recognizes water
from non-water and so the model predicts over water {\it as if} it
were habitat similar to that within which the array is nested. But,
all of this is ok as far as estimating density goes and, furthermore,
we can compute valid estimates of $N$ over any well-defined region which
presumably wouldn't include water if we so choose.

\begin{figure}
\begin{center}
\includegraphics[height=3in,width=3.375in]{Ch4/figs/density10x10}
\end{center}
\caption{Needs a caption}
\label{scr0.fig.density10x10}
\end{figure}

\begin{figure}
\begin{center}
\includegraphics[height=3in,width=3.375in]{Ch4/figs/density30x30}
\end{center}
\caption{Needs a caption}
\label{scr0.fig.density20x20}
\end{figure}

\section{Discrete State-Space}
\label{scr0.sec.discrete}

The SCR model developed previously in this chapter assumes that
individual activity centers are distributed uniformly over the
prescribed state-space. Clearly this will not always be a reasonable
assumption. In chapter \ref{chapt.state-space} we talk about developing models
that allow explicitly for non-uniformity of the activity centers by
modeling covariate effects on density. A simpler method of affecting
the distribution of activity centers, which we address here, is to
modify the shape of the state-space explicitly. For example, we might
be able to classify the state-space into distinct blocks of habitat
and non-habitat. In that case we can remove the non-habitat from the
state-space and assume uniformity of the activity centers over the
remaining portions judged to be suitable habitat.  There are two ways
to approach this: We can use a regular grid of points to represent the
state-space, i.e., by the set of coordinates ${\bf s}_1, \ldots, {\bf
  s}_{G}$, and assign a equal probabilities to each possible value, or
we can retain the continuous formulation of the state-space but use
basic polygon operations to induce constraints on the state-space We
focus here on the formulation of our basic SCR model in terms of a
discrete state-space but later on (chapter \ref{chapt.mcmc} and also
Appendix XYZ) we demonstrate the latter approach based on using
polygon operations to define an irregular state-space.

Use of a discrete state-space can be computationally expensive in {\bf
  WinBUGS}. That said, it isn't too difficult to do the MCMC
calculations in {\bf R} which we discuss briefly in chapter
\ref{chapt.mcmc}. The {\bf R} package {\tt SPACECAP}
\citep{gopalaswamy_etal:2011} arose from the {\bf R} implementation
developed for the application in \citet{royle_etal:2009}.  As we will
see in chapter \ref{chapt.mle}, we must prescribe the state-space by a
discrete mesh of points in order to do integrated likelihood and so if
we are using a discrete state-space this can be accommodated directly
in our code for obtaining MLEs.

While clipping out non-habitat seems like a good idea, its not obvious
that we accomplish any biologically reasonable objective by doing
so. We might prefer to do it when non-habitat represents a clear-cut
restriction on the state-space such as a reserve boundary or a lake,
ocean or river. It makes sense in those situations.  Unfortunately,
having the capability to do this also causes people to start defining
``habitat'' vs. ``non-habitat'' based on their understanding of the
system whereas it can't be known whether the animal being studied has
the same understanding. Moreover, differentiating of the landscape by
habitat or habitat quality probably affects the geometry and
morphology of home ranges much more than the plausible locations of
activity centers. That is, a home range centroid could, in actual
fact, occur in a walmart parking lot if there is pretty good habitat
around walmart, so there is probably no sense to cut out the walmart
lot and preclude it as the location for an activity center.  It would
generally be better to include some definition of habitat quality in
the model for the detection probability (see chapter XYZ).


\subsection{Evaluation of Coarseness of Discrete Approximation}

The coarseness of the state-space should not really have much of an
effect on estimates if the grain is sufficiently fine relative to
typical animal home range sizes.  Why is this?  We have two analogies
that can help us understand this. First is the relationship to Model
$M_{h}$.  As noted in section \ref{scr0.sec.scrmh} above, we can think
about SCR models as a type of finite mixture
\citep{norris_pollock:1996, pledger:2000} where we are fortunate to be
able to obtain direct information about which ``group'' individuals
belong to (group being location of activity center).  In the standard
finite mixture models we typically find that only 1 or a very small
number of groups (e.g., 2 or 3 at the most) can explain really high
levels of heterogeneity and are adequate for most data sets of small
to moderate sample sizes. We therefore expect a similar effect in SCR
models when we discretize the state-space.
We can also
think about discretizing the state-space as being related
to numerical integration where we find (see
chapter \ref{chapt.mle}) that we don't need a very fine
grid of support points to evaluate the integral to a reasonable
level of accuracy. We demonstrate this here by reanalyzing simulated
data using a state-space defined by a different numbers of support points.
We provide an R script called \mbox{\tt simSCR0discrete.fn} in the
{\bf R} package \mbox{\tt scrbook}.  We note that for this comparison
we generated the actual activity centers as a continuous random
variable and thus the discrete state-space is, strictly speaking, an
approximation to truth. That said, we regard all state-space
specifications as approximations to truth because they are all,
strictly speaking, models of some unknown truth. Thus the use of any
specific discrete state-space is not intrinsically more ``wrong'' than
any specific continuous representation.


We used {\bf JAGS} from the \mbox{\tt rjags} function to obtain the results
for $6 \times 6$, $9 \times 9$, $12 \times 12$, $15\times 15$,
$20\times 20$, $25 \times 25$ and $30 \times 30$ state-space grids.
We used 2000 burn, 12000 total iters with 3 chains, therefore a total
of 30000 posterior samples.
For {\bf WinBUGS} we used 3 chains of 5k total with 1k burnin means 12k
total posterior samples.
Summary results for these analyses are shown in
Table XYZ\footnote{Andy to finish later. }.

\begin{verbatim}
Table XYZ.
             Mean       SD    NaiveSE  Time-seriesSE  runtime
6    N     109.7717 15.98959 0.0923160    0.377737    1239
9    N     114.4621 16.72025 0.0965344    0.468659    1267
12   N     115.4309 17.12403 0.098866     0.464830    1576
15   N     114.7699 17.0242  0.0982894    0.425238    1638
20   N     116.0370 17.10686 0.0987665    0.486867    1647
25   N     116.3228 16.98323 0.0980527    0.465527    1661
30   N     116.4252 17.4078  0.100504     0.533735    1806
WinBUGS
             Mean       SD    NaiveSE  Time-seriesSE  runtime
6    N     111.67    16.61                             2274
9    N     114.23    17.99                             4300
12   N     115.98    17.38                             7100
15   N     115.38    17.94                            13010

Note: WinBUGS based on fewer samples too!

To get SE and time-series SE do this:
You can use as.mcmc.list() to convert to a coda object. Then use summary.�
\end{verbatim}

The results in terms of the posterior summaries are, as we
expect, very similar using {\bf WinBUGS}. However, it was interesting
to note that {\bf WinBUGS} runtime is much worse (note the number of
iterations is lower for {\bf WinBUGS} yet the runtime is much longer)
and, furthermore, it seems to scale with the size of the
discrete state-space grid. While that was expected, it was unexpected
that the runtime of {\bf JAGS} would seem relatively consistent
as we increase the grid size.
We suspect that {\bf WinBUGS} is evaluating the full-conditional for
each activity center at all $G$ possible values whereas it may be that
{\bf JAGS} is evaluating the full-conditional only at a subset of
values or perhaps using previous calculations more effectively.

While this might suggest that one should always use {\bf JAGS} for
this analysis, we found in our analysis of the wolverine (next
section) that {\bf JAGS} could be extremely sensitive to starting
values, producing MCMC algorithms that sometimes simply did not work.

\subsection{Analysis of the wolverine camera trapping data}

We reanalyzed the wolverine data using discrete state-space grids with points spaced by 2,
4 and 8 km (depicted in Fig. \ref{scr0.fig.wolvgrids}). These were
constructed from
the 40 km buffered state-space, and deleting the points over water \citep[see][]{royle_etal:2011jwm}.
 Our interest in doing this was
to evaluate the relative influence of grid resolution on estimated
density because the coarser grids will be more efficient from a
computational stand-point and so we would prefer to use them, but perhaps not
if there is a strong influence on estimated density.

{\bf Note}: Results from WinBUGS are given below -- these are updated
based on longer MCMC runs and replace prelim results as of Jan 1 2012
or so. 
To be done: density map.



\begin{figure}
\begin{center}
\includegraphics[height=2.5in,width=5in]{Ch4/figs/wolvgrids}
\end{center}
\caption{2 km 4 km and 8km wolverine state-space grids extending about
40 km from the vicinity of the trap array. }
\label{scr0.fig.wolvgrids}
\end{figure}

{\small
\begin{verbatim}
This took about 6 days in WinBUGS. Terrible mixing for the 2km and
8km. Why is this? We may never know!

> print(out.2km,digits=2)
Inference for Bugs model at "modelfile.txt", fit using WinBUGS,
 3 chains, each with 11000 iterations (first 1000 discarded)
 n.sims = 30000 iterations saved
       mean    sd  2.5%   25%   50%   75%  97.5% Rhat n.eff
psi    0.43  0.09  0.27  0.37  0.43  0.49   0.63 1.00   560
sigma  0.62  0.05  0.54  0.59  0.62  0.65   0.73 1.01   160
lam0   0.05  0.01  0.04  0.04  0.05  0.06   0.07 1.01   320
p0     0.05  0.01  0.03  0.04  0.05  0.05   0.06 1.01   320
N     86.56 16.94 57.00 75.00 85.00 97.00 124.00 1.00   510
D      8.78  1.72  5.78  7.60  8.62  9.83  12.57 1.00   510

For each parameter, n.eff is a crude measure of effective sample size,
and Rhat is the potential scale reduction factor (at convergence, Rhat=1).
> print(out.4km,digits=2)
Inference for Bugs model at "modelfile.txt", fit using WinBUGS,
 3 chains, each with 11000 iterations (first 1000 discarded)
 n.sims = 30000 iterations saved
       mean    sd  2.5%   25%   50%    75%  97.5% Rhat n.eff
psi    0.45  0.09  0.28  0.38  0.44   0.50   0.64    1  1300
sigma  0.61  0.04  0.53  0.58  0.61   0.64   0.71    1  1600
lam0   0.05  0.01  0.04  0.05  0.05   0.06   0.07    1  2500
p0     0.05  0.01  0.03  0.04  0.05   0.05   0.07    1  2500
N     89.25 17.44 59.00 77.00 88.00 100.00 127.00    1  1100
D      9.01  1.76  5.96  7.77  8.88  10.10  12.82    1  1100

For each parameter, n.eff is a crude measure of effective sample size,
and Rhat is the potential scale reduction factor (at convergence, Rhat=1).
> print(out.8km,digits=2)
Inference for Bugs model at "modelfile.txt", fit using WinBUGS,
 3 chains, each with 11000 iterations (first 1000 discarded)
 n.sims = 30000 iterations saved
       mean    sd  2.5%   25%   50%   75%  97.5% Rhat n.eff
psi    0.42  0.09  0.26  0.36  0.41  0.47   0.61 1.00   940
sigma  0.68  0.05  0.59  0.64  0.67  0.71   0.77 1.01   220
lam0   0.05  0.01  0.03  0.04  0.05  0.05   0.06 1.00   560
p0     0.05  0.01  0.03  0.04  0.04  0.05   0.06 1.00   560
N     83.18 16.14 56.00 72.00 82.00 93.00 119.00 1.00   700
D      8.28  1.61  5.57  7.17  8.16  9.26  11.84 1.00   700

For each parameter, n.eff is a crude measure of effective sample size,
and Rhat is the potential scale reduction factor (at convergence, Rhat=1).
\end{verbatim}
}

The density is a bit different depending on the grid size. Also the
effectiveness of the MCMC algorithsm is pretty remarkably different. 
We did the analysis in JAGS also. The results are shown below. {\bf Note}: I
am going to run these again but for longer to finalize the results.

{\small
\begin{verbatim}
 ### 01/10/2012 -- need to rerun these JAGS runs but use more
iterations and check results.


2km
Iterations = 7001:13000
Thinning interval = 1
Number of chains = 3
Sample size per chain = 6000

          Mean        SD  Naive SE Time-series SE
N     86.28522 16.950626 1.263e-01      0.4878973
lam0   0.04807  0.007512 5.599e-05      0.0002199
p0     0.04581  0.006820 5.083e-05      0.0001996
psi    0.28904  0.062117 4.630e-04      0.0017481
sigma  0.62769  0.043596 3.249e-04      0.0018724

4km
          Mean        SD  Naive SE Time-series SE
N     85.53139 16.998966 1.267e-01      0.5181297
lam0   0.04636  0.007542 5.621e-05      0.0002382
p0     0.04425  0.006867 5.118e-05      0.0002172
psi    0.28650  0.061922 4.615e-04      0.0018276
sigma  0.64281  0.048321 3.602e-04      0.0022911

8km
          Mean        SD  Naive SE Time-series SE
N     83.97039 16.508146 1.230e-01      0.4548782
lam0   0.04519  0.006919 5.157e-05      0.0001738
p0     0.04319  0.006319 4.710e-05      0.0001589
psi    0.28146  0.060653 4.521e-04      0.0016555
sigma  0.66956  0.040989 3.055e-04      0.0015070
\end{verbatim}
}

\subsection{SCR models as multi-state models}

While we invoke a discrete state-space artificially, by gridding the
underlying continuous state-space, sometimes the state-space is more
naturally discrete. Consider a situation in which discrete patches of
habitat are searched using some method and it might be convenient (or
occur inadvertently) to associate samples to the patch level instead
of recording observation locations. In this case we might use a model
${\bf s}_{i} \sim dcat(probs[])$  where $probs[]$ are the probabilities that
an individual inhabits a particular patch. We consider such a case
study in chapter XXPoissonXXX from \citet{mollet_etal:2012} who
obtained a population size estimate of a large grouse species known as
the capracaillie. Forest patches were searched for scat which was
identified to individual by DNA analysis.
Even when space is {\it not}
naturally discrete, measurements are often made at a fairly coarse
grain (e.g., meters or tens of meters along a stream), or associated
with spatial quadrats for scat searches and therefore the state-space
may be effectively discrete in many situations.

This discrete formulation of SCR models suggests that SCR models are
related to ordinary multi-state models \citep[][ch. 9]{kery_schaub:2011}
which are also parameterized in terms of a discrete state
variable which is often defined as a spatially-indexed state related
either to location of capture or breeding location. While many
multi-state models exist in which the state variable is not related to
space, multi-state models have been extremely useful in development
models of movements among geographic states and indeed this type of
problem motivated their early developments by \citet{arnason:1972,
  arnason:1973} and \citet{hestbeck_etal:1991}.  We pursue this
connection a little bit more in chapter XXX XYZ.




\section{ Summary and Outlook }

A point we tried to emphasize in this chapter is that the basic SCR
model is not much more than an ordinary capture-recapture model for
closed populations -- it is simply that model but augmented with a set
of ``individual effects'', ${\bf s}_{i}$, which relate encounter
probability to some sense of individual location. SCR models are
therefore a type of individual covariate model (as introduced in
chapter \ref{chapt.closed} -- but with imperfect information about the
individual covariate. In other words, they are GLMM type models when
$N$ is known or, when $N$ is unknown, they are zero-inflated GLMMs
(see \citet{royle:2006}).  Another class of capture-recapture models
that SCR models are closely related to is so-called ``Model $M_{h}$.''
The effect of introducing a spatial location for individuals is that
it induces heterogeneity in detection probability, as in Model
$M_{h}$. However, unlike Model $M_{h}$, we obtain some information
about the individual effect which is completely latent in Model
$M_{h}$. If the state-space of the random effect ${\bf s}$ is discrete
then the SCR model resembles more closely the finite-mixture class of
heterogeneity models \citep{norris_pollock:1996} which parameterizes
heterogeneity by assuming that individuals belong to discrete classes
or groups (e.g., high, medium, low). In the context of SCR models we
obtain some information about the ``group membership'' in the
locations where individuals are captured.  Given the direct
relationship of SCR models with so many standard classes of models, we
find that they are really quite easy to analyze using standard MCMC
methods encased in black boxes such as {\bf WinBUGS} or {\bf JAGS} and
possibly other packages. They are also easy to analyze using classical
likelihood methods, which we address in chapter \ref{chapt.mle}.

Formal consideration of the collection of individual locations $({\bf
  s}_{1}, \ldots, {\bf s}_{N})$ in the model is fundamental to all of
the models considered in this book. In statistical terminology, we
think of the collection of points $\{ {\bf s}_{i} \}$ as a realization of a
point process and part of the promise, and ongoing challenge, of SCR
models is to develop models that reflect interesting biological
processes, for example interactions among points or temporal dynamics
in point locations.  Here we considered the simplest possible point
process model - the points are independent and uniformly
(``randomly'') distributed over space. Despite the simplicity of this
assumption, it should suffice in many applications of SCR models
although we do address generalizations of this model in later
chapters. Moreover, even though the {\it prior} distribution on the
point locations is uniform, the realized pattern may deviate markedly
from uniformity as the observed encounter data provide information to
impart deviations from uniformity. Thus, the estimated density map
will typically appear distinctly non-uniform.  As a general rule,
information in the data will govern estimates of individual point
locations so even fairly complex patterns of non-independence or
non-uniformity will appear in the data. That is, we find in
applications of the basic SCR model that this simple {\it a priori}
model can effectively reflect or adapt to complex realizations of the
underlying point process.  For example, if individuals are highly
territorial then the data should indicate this in the form of
individuals not being encountered in the same trap - the resulting
posterior distribution of point locations should therefore reflect
non-independence.  Obviously the complexity of posterior estimates of
the point pattern will depend on the quantity of data, both number of
individuals and captures per individual.  Because the point process is
such an integral component of SCR models, the state-space of the point
process plays an important role in developing SCR models. As we tried
to emphasize in this chapter, the choice of the stat-espace is part of
the model. It can have an influence on parameter estimates and other
inferences such as model selection (see chapter \ref{chapt.gof}). We
emphasize however that this is not an arbitrary decision like
``buffering'' because the model induces an explicit interpretation of
parameters and statistical effect on estimators.

We showed how to conduct inference about the underlying point process
including calculation of density maps from posterior output. We can do
other things we normally do with spatial point processes such as
compute ``K-functions'' and test for ``complete spatial randomness''
(CSR) which we develop in chapter \ref{chapt.gof}.  Modifying and
applying point process methods to SCR problems seems to us to be a
fruitful area of research.

An obvious question that might be floating around in your mind is why
should we ever go through all of this trouble when we could just use
{\bf MARK} or {\bf CAPTURE} to get an estimate of $N$ and apply $1/2$
MMDM methods?  The main reason is that these conventional methods are
predicated on models that represent explicit misspecifications of both
the observation and ecological process - they are wrong!  Not just
wrong, because of course all models are wrong, but they're not even
{\it plausible} models! Thus while we might be able to show adequate
fit or whatever, we think as a conceptual and philosophical model one
should not be using models that are not even plausible data-generating
models -- even if the plausible ones don't fit!  Perhaps more
charitably, these ordinary non-spatial models are models of the wrong
system. They do not account for trap identity. They don't account for
spatial organization or ``clustering'' of individual encounters in
space. And, ``density'' is not a parameter of those models because
density has no meaning absent an explicit representation of space. If
we do define space explicitly, e.g., as a buffered minimum convex
hull, then the normal models ($M_{0}$, $M_{h}$, etc..) assume that
individual capture-probability is not related to space, no matter how
we define the buffer.  Conversely, the SCR model is a model for
trap-specific encounter data - how individuals are organized in space
and interact with traps. SCR models provide a coherent framework for
inference about density or population size and also, because of the
formality of their derivation, can be extended and generalized to a
large variety of different situations, as we demonstrate in subsequent
chapters.

In the next few chapters we continue to work with this basic SCR
design and model but consider some important extensions of the basic
model.  For example, we consider
extensions
to  include covariates that vary by individual, trap, or over time
(chapter \ref{chapt.covariates}), spatial covariates on density
(chapter \ref{chapt.state-space}),
 open populations (chapter \ref{chapt.open}), model assessment and
 selection (chapter \ref{chapt.gof}) and other topics.
We also consider technical details of Bayesian (chapter
\ref{chapt.mcmc}) and  maximum
likelihood (chapter \ref{chapt.mle}) estimation so that the interested
reader can develop or extend their own methods to suit their needs.
