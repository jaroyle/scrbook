The secr package does likelihood analysis of SCR models for most
classes of models using the methods described by
\citep{borchers_efford:2008}.

The formulation of the likelihood in \citet{borchers_efford:2008}
deviates only slightly from what we presented in Sec. XXXXX
(eq. XXXX) above.
In fact, our formulation of the likelihood is precisely
the binomial form mentioned on p. 379 of \citet{borchers_efford:2008}.
However, the likelihood that \mbox{\tt secr} implements is that based on removing
$N$ from the likelihood by integrating the binomial likelihood
(Eq. XXX above) over a Poisson prior for $N$. 
Specifically, lets take our likelihood which we expressed as:
\[
  [{\bf y}_{i}|{\bm \alpha}] = 
\int_{{\cal S}} \mbox{Binomial}({\bf y}_{i} |{\bf s}_{i}, {\bm \alpha})g({\bf s}_{i}) d{\bf s}_{i}
\]
where $g({\bf s})$ is a uniform pmf and to account for the unobserved
individuals we get  NOTE: PUT THIS FORMULA IN CH 5!!! (also formula
for pi_{0})
\[
 {\cal L} = \frac{N!}{n! n_{0}!} 
 \left\{ \prod_{i}  [{\bf y}_{i}|{\bm \alpha}] \right\}
 (\pi_{0})^{n_{0})
\]
now lets assume that $N \sim \mbox{Poisson}(\Lambda)$ and do a further level
of marginalization over this prior distribution:
\[
\sum_{n0=0}^{\infty}  this * Poisson
\frac{N!}{n! n_{0}!} 
 \left\{ \prod_{i}  [{\bf y}_{i}|{\bm \alpha}] \right\}
 (\pi_{0})^{n_{0})
\]
Thus we note there are two marginalizations
 going on here -- the
integration to remove the latent variables ${\bf s}$ and then
summation to remove the parameter $N$. 
This produces exactly this likelihood:
\[
{\cal L} = 
 \left\{ \prod_{i}  [{\bf y}_{i}|{\bm \alpha}] \right\}  \Lambda^{n}   \exp( - \Lambda \pi_{0} )
\]
which is Eq. 2 of \citet{borchers_efford:2008} except for notational
differences, and this 
looks like our likelihood Eq. XXXX except with 
$\Lambda^{n}   exp( - \Lambda \pi_{0} )$ replacing the combinatorial
term and the $\pi_{0}^{n_{0}}$ term.

So the essential distinction between our MLE and Borchers and Efford
as implemented in \mbox{\tt secr}
is whether you keep $N$ in the model or remove it by
integration over a Poisson prior. If you have prescribed a state-space explicitly then
there is no difference at all, or at least there shouldn't be. That
is, the estimators should be very similar. Perhaps the difference is: N
is the actual realized population size whereas $\lambda$ is the
expected value of that. We probably want to estimate N in most
cases. Numerically we think these will be the same under the Poisson
model used to compute the marginal likelihood but perhaps the error is
different?
I will make up a Poisson likelihood and we will test it out.
Both models assumes ${\bf s}$ is uniformly distributed over space, but
for the binomial model we  make no additional assumption about N
whereas we assume $N$ is Poisson using the formulation in \mbox{\tt
  secr} from 
\citep{borchers_efford:2008}.

Using data augmentation we could do a similar kind of integration but 
integrate $N$ over a binomial (M,\psi) prior. So obviously this is
approximately the same as M gets large. However, doing a bayesian
analysis by MCMC ,  we obtain an
estimate of both $N$ and the parameter controlling its expected value
$\psi$ which are, in fact, both identifiable from the data even using
likelihood analysis \citep{royle_etal:2007}.   That said we can integrate N
out completely and just estimate $\psi$ as we noted in Section XXX
above (and from RD book). 



