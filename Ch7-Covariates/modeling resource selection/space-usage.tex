\documentclass[12pt]{article}

\usepackage[total={6.5in,8.75in}, top=2.4cm, left=2.4cm]{geometry}
\usepackage{lineno}
\usepackage{amsmath}
%\usepackage{amssymb}    % used for symbols in figure legends
\usepackage{graphicx}
\usepackage[round,colon,authoryear]{natbib}

\usepackage{bm}
\usepackage{float}
\usepackage{amsmath}
\usepackage{amsfonts}
\usepackage{hyperref}
\usepackage{verbatim}
\usepackage{soul}
\usepackage{color}
\usepackage{setspace}

\bibliographystyle{ecology} % kluwer, plos-natbib, pnas-natbib


\title{Modeling Resource Selection in Spatial Capture-Recapture Models}

\begin{small}
\author{
%{\bf J. Andrew Royle}, {\bf Richard B. Chandler}, 
%
%
}
\end{small}


\begin{document}

\maketitle

\date


\linenumbers


\begin{spacing}{1.8}

\begin{flushleft}
{\em \bf Abstract}

{\em \bf Key words:} animal movement, ecological distance, landscape connectivity,
least-cost path, spatial capture-recapture

\end{flushleft}



\section{Introduction}


In SCR models we model the probability of capture as a function of distance between
individual activity center ${\bf s}$ and trap locations ${\bf x}$. Define a variable
 y(x)=1 if an individual is encountered at trap $x$ and $y(x)=0$ otherwise.
 Then the probability of encounter is:
$Pr(y(x)=1|s)$ and various models are used to dscribe the relationship between encounter
probability and $s$. For example the normal kernel model is common:
\[
\Pr(y(x)=1|s) =  p0*exp(- || x- s||^{2} / (2 \sigma^{2}))
\]

Resource selection functions model space usage from location information usually obtained
by telemetry data.
A sample of locations $x\_{1},\ldots,x\_{n}$ is obtained and this is viewed as a point
process realization. The conditional intensity function is modeled as a function of covariates.
For example if we have a single covariate,
$z(x)$, then the conditional intensity function is
\[
\pi(x) = \frac{ exp( \beta z(x) ) }{
\sum_{x} exp( \beta z(x) ) 
}
\]
where the sum in the denominatory is taken over all pixels $x$ in the landscape. This is the
standard notation but really it should be regarded as being conditional on the individual's
activity center s. i.e., $\pi(x|s)$, but in practice distance is not modeled as an explicit
factor in RSFs so far as I know.


In this paper we integrate these two ideas into a unifed model that allows us to estimate
``resource selection'' from spatial capture-recapture data.
This is a huge advance because SCR data give us a {\it biased} sample of observation locations.
That is, we can only observe locations where traps are placed. 
Conversely, RSFs estimated by telemetry data can obtain individual locations anywhere in space. 

Do we need to sample randomly?  We probably need to do that with traps to make this work,
but we'll leave this for a discussion point. 

The basic idea here is that detection of an animal at a trap occurs when (1) an individual selects
to use that pixel $x$ and (2) is encountered by the trap. therefore the probability of encounter
should be
\[
 \Pr(y(x)=1|s) = Pr(y(x) = 1| s)\pi(x|s)
\]
This is so obvious its surprising that we didn't think of it before. 
(note: this is really the joint distribution of y(x) and x, need to think about this).

In the case of the normal kernel model for encounter probability then the two functions
combine:
\begin{eqnarray*}
\Pr(y(x)=1|s) &=&  p0*exp(- || x- s||^{2} / (2 \sigma^{2}))  exp( \beta z(x))/N(x)  \\
\Pr(y(x)=1|s) &=&  (p0/N(x))*exp( \alpha_{1} || x- s||^{2} + \beta z(x) )
\end{eqnarray*}
so we see that the normalizing constant $N(x)$ just gets abosrbed with p0, so we don't even
need to know the whole background.
This is kewl as shit. 


Note: This actually isn't quite right because the baseline pr(encounter) gets extremely small,
so you can't really generate reasonable data like this.
I think you have to model Pr(encounter) as a function of ``total exposure'' and then allocate
to traps, ala the original approach I had.  Not sure how to resolve this in a completely
elegant way right now. 


Could just do this:
\[
 logit(p\_{ij}) = a0 + a1*dist(x,s) + a2*z(s)
\]




\section{Simulation}

We simulated this using a hypothetical landscape generating as a multivariate normal to allow
for ``patchiness'' (same landscap as royle 














\newpage


\bibliography{../AndyRefs_alphabetized}


\end{spacing}







\end{document}






