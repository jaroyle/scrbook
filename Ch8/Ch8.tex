\chapter{
Model Selection and Assessment
}
\markboth{Model Assessment}{}
\label{chapt.gof}

\vspace{.3in}

There are 4 main inference tasks that we do with a model: estimate 
parameters, make predictions of unobserved random variables
(e.g., of density), evaluate the relative 
merits of different models or choose a best model (model selection), 
and we check whether a specific model is reasonable or not (assessment). 
Conceptually, we can think of this last thing as follows: if we simulate 
under that model, do the simulated realizations look like our data? 

In a sense, it is that last element that is most relevant to science.
That is beyond the basic construction of the model which presumably
contains some basic scientific hypotheses and thus requires some
understanding and thinking about the system. After that it is the
assessment that provides what is essentially a basis for falsification
of the model, and hence learning..

The Gelman and Shalizi (2001?)  paper requires extensive citation
here. It is not just that models are wrong but they are manifestly
so. - good quote on that in his paper.  Also the fact that
falsification is the main core of science here.

In this chapter, we focus on the last two of these inference tasks:
model selection (which model or models seem preferred), and model
assessment (do the data seem consistent with a particular model?).
Bayesian model assessment has not been terribly popular in the
literature because it is a bit more tedious computationally and,
furthermore, with complex hierarchical models it is not always so
obvious what should be a decent fit statistic. Conversely, in
classical inference it is sometimes possible to identify a single test
statistic for assessing goodness-of-fit.  While this is more typical
for the most basic models (linear models, ANOVA, frequency data,
etc..), it is also easy, in classical statistics, to assess goodness
of fit using bootstrap methods which is more or less the standard
approach, in general.

In the context of SCR models we focus in this chapter on a couple of 
specific model selection and assessment problems including
 selecting among various models of the detection process which 
could involve different "detection functions", link functions related 
p to s, or additional covariates or other model structure such as a 
behavioral response.  We use DIC and AIC for this.

For model checking or assessment we rely exclusively on Bayesian p-values 
(REF XYZ) to evaluate "goodness-of-fit". Part of the challenge thus far is 
coming up with good fit statistics, we suggest a few here and evaluate them 
by simulation. We break the problem up into 2 components which we attack 
separately (Royle et al. 2011): Does the observation model fit? Does the 
uniformity assumption appear adequate? The latter has a huge amount of 
precedent in the ecological literature as it is equivalent to the test of 
``complete spatial randomness'' which Royle et al. (2011) 
applied to SCR models. Really this is a test of ``spatial randomness''
because CSR really means Poisson point process.

A few specific questions to resolve:
\begin{itemize}
\item[(1)]  Can AIC or DIC choose among detection models and does it matter?
\item[(2)]  Can AIC or DIC choose among more substantial models?  We should
 assume it does. Can we rig up a test?  Suggest: (1) trap level
 covariate and trying DIC; (2) Model Mb vs. not
\item[(3)] Simultaneous to above, see if we can do that....... 
Posterior model probabilities: Can we demo this?   With detection models?
 Don't do much analysis of the situation, just demo it maybe?
\item[(4)]  Basic framework for Bayesian GoF. Power under alternatives. 
(clustering, regularity, uniformity) and misspecified encounter models
-- do they show a lack of fit?
\end{itemize}




\begin{comment}
\subsection{The Role of model assumptions}

Referees are quick to point out that bivariate normal distributions as
models for home range are simply wrong. As if using wrong models
somehow invalidates something. (is that a statistical paradigm that I
missed?). The issues are whether its an adequate model, whether we can
refute it, whether resulting inferences are sensitive to it, and
whether the model is useful for its intended purpose regardless.  
If  we really wanted a great description of a home range then we would do 
something different besides conduct an SCR study. The purpose of most 
SCR models is not to study home range geometry and morphology but, rather, 
to estimate density and possibly other vital rates such as survival and 
recruitment. In addition, it is my experience that the same people who 
criticize models as implying overly simple models actually cannot describe 
alternative models except using jargon and procedures like "kernel 
smoothing", "utilization distributions" and "neural networks". So, on 
the one hand, the bivariate normal distribution is overly simple, and 
therefore we should use "neural networks"?  What irritates me even more 
is referees who emphatically assert that "real home ranges aren't bivariate 
normal" and then they go on to cite references who somehow "proved" they 
are not. These people have no clue what statistics is good for and what 
the point of SCR models is, nor can they grasp the relevance of the home 
range model - namely, as a model for explaining heterogeneity in 
capture-probability. To be sure, the bivariate normal model is an 
over-simplification but so far it has not been shown to be ineffective 
in SCR problems, and besides  substantially more flexible models have been 
developed \citep{royle_etal:2012ecol}. 

What we care about in models of capture-recapture data is whether the 
bivariate normal (or any model) provides an adequate description of the 
encounter process. That is the question we will evaluate here. 
\end{comment}



\section{Model Selection: General Strategies}

If you're doing a classical analysis based on likelihood then model
selection is easily accomplished using AIC \citep{burnham_anderson:xxxx}
which we demonstrate below. Model selection is somewhat more difficult 
as a Bayesian and there is no such canned all-purpose method like AIC.  
As such we recommend more of a pragmatic approach, in general, for all
problems, based on a number of basic considerations:
\begin{itemize}
\item[(1)] For a small number of fixed effects we recommend a standard 
hypothesis testing ideas -- i.e., if the posterior for a parameter overlaps 
zero or whatever then toss out the effect
\item[(2)] Posterior model probabilities: In some cases we can implement 
the "variable weights" idea [not really sure who to credit for this]. 
The idea is introduce a latent variable $w \sim Bern(.5)$ and then 
construct  your model according to:  
\[
 logit(p) = a + w*b*covariate  
\]
There are examples in Royle and Dorazio (2008) in chapter 3 and later in 
the CJS chapter (perhaps Marc has an example in the new BPA book?)
\item[(3)] DIC: The official word on DIC is that it might not work so 
well, but its hard to say in general. I recommend using it if it leads to 
sensible results but also calibrate it for specific problems.
As always, we 
think a small simulation study would be valuable for any particular case. 
We need to do that here.
\item[(4)] Logical argument: For something like M/F differences it seems 
to make sense to leave the extra parameters in the model no matter what, 
because of course you expect , biologically, there to be a difference in 
these things. Thus the hypothesis test itself is meaningless (Doug Johnson 
wrote a paper on this kind of stuff once) and you would only reject the 
null because of low power anyhow.
\end{itemize}

In all modeling activities, the use of logical argument should not be under-utilized.

\subsection{Omnibugs Model Selection Based on AIC and DIC}

We recommend using AIC and DIC for selection among various classes of
fixed effects models. 

What is AIC? What is the DIC for a SCR model? The models are just 
binomial or Poisson regressions and so we can define the DIC as
 -loglikelihood().  But what about the data augmentation part?
The current Conventional Wisdom is that DIC doesn't really work so 
well in hierarchical models (    ). But there doesn't really seem to be 
general theory to support this. It seems maybe like its ok sometimes. So 
we take a look at its use here to pick among fixed effects models. 


\subsection{Choosing detection models}

HERE WE SET UP A SIMULATION WITH 5 MODELS. FOR EACH MODEL WE SIMULATE
A DATA SET AND FIT ALL 5 MODELS USING MLE AND USING JAGS. THATS 10
MODEL FITS.  WE DO THAT 100 TIMES AND LOOK AT THE OUTPUT. 
THIS IS A TOTAL OF 1000 MODEL FITS FOR EACH ``TRUE MODEL''
SO TO DO THIS FOR ALL 10 MODELS REQUIRES 10,000 FITS.
QUESTION: SHOULD WE FIX ``AVERAGE E[n]''??????? I think so. This needs
evaluated ahead of time. 

For all 5 models choose parameters: (1) same effective home range
size; (2) same net capture probability.  e.g., 60\% of population for
N=100 and N=200  (huge sample sizes). 



The idea is that maybe we should do a very directed simulation 
study to evaluate selection based on DIC, AIC and also perhaps the "variable 
weights" idea if that works. 

I have a 2x2 factorial arrangement in mind for a 7 x 7 grid of traps
with unit spacing:
\begin{verbatim}
                     N=100              N=200

sigma = 0.50   low spatial recaps      low spatial recaps
               low n                   high n
sigma = 1.00   high spatial recaps     high/high
               low n
\end{verbatim}

There could be two dimensions to this study:

(1)	Selection from among different detection models. 
E.g., Comparison of logit vs. cloglog and linear vs quadratic vs. 
log() distance functions.  That's 6 total models, for 
delta = 2, 3 and 4.  Why should DIC work?

(3)	For each unique detection model, we simulate data under that model 
and fit all of the other models and compute DIC, variable weights and 
perhaps use likelihood to compute AIC and see how the results differ.

Since all the detection models give about the same answer there's no
point in acting like we're engaging in some profound act of pure
science by doing model selection on detection functions. 

\subsection{Choosing Biological models}

Behavior, time of year, trap type (biated or not), seasonal variation
in encounter probability, things that make a
biological difference we should be able to choose among

Different model structures: e.g., with/without behavioral response. 
With/without some trap covariate. Other things?

The expectation is that it should make a big difference for covariate
effects such as Mb and so forth which has been demonstrated in
practice repeatedly. So can we choose among those models effectively?


\subsection{Model selection using posterior model probabilities}

This is ok for covariates. Do an example. 

Could also do the indicator function for different models if the 
w[i] = (0,0,1,0) a multinomial draw.





























\section{Evaluating Goodness-of-Fit}


It is sometimes useful to provide an evaluation of model fit. For SCR 
models it is usually possible to devise such an evaluation using the 
so-called Bayesian p-value (Gelman XYZ.XYZ) approach (see section 2.XYZ). 
However, to the best of our knowledge there has not been a definitive or 
general proposal for a fit statistic for use in computing the Bayesian 
p-value although a few specialized implementations of Bayesian p-values 
have been provided (Royle 2009; Gardner et al. 2010 ???).  

One reason that this is challenging in the context of SCR models is that 
there are at least two components to model fit for most models (but 
sometimes more), making an omnibus measure of fit difficult to describe. 
First we can consider whether the model explains the observation 
process - we can evaluate this based on the encounter frequencies of
individuals {\it conditional} on ${\bf s}_{1}, \ldots, {\bf s}_{N}$
Second, we might also be 
interested in evaluating the hypothesis concerning the distribution of
the activity centers. For the simple model developed in this chapter,
this is the assumption of {\it complete spatial randomness} (CSR)
which we consider in section XXX.YYY below. Actually, as we noted in
sec. XXX this is not strictly CSR per se because of the binomial
assumption on $N$ but we refer to it as that because it is essentially
equivalent. And besides we generate the reference distribution within
the context of a Bayesian p-value.


We propose analyzing the separate elements of fit individually.  
To evaluate fit of the detection component of the model, we consider
two distinct fit statistics based on the individual encounter
frequencies and also the trap encounter frequencies

DETAILS HERE��

I think inference about Density is going to be insensitive to
departures from CSR but probably somewhat sensitive to bad models for
the observation process.  Why do I think this? Well spatial variation
is not important to inference about the aggregate population size, but
it is relevant to "within population" inference, such as predicting on
small areas.  Conversely, inference about total population size is
known to be highly sensitive to the observation model (Dorazio and
Royle 2003; Link 2003).


\section{Testing Complete Spatial Randomness (CSR)}

Historically, especially in ecology, there has been a huge amount of
interest in whether a realization of a point process indicates
"complete spatial randomness," i.e., that the points are distributed
uniformly and independently in space.  A good reference for such
things is Cressie (1996; ch. 8) and XYZ.XYZ and I also like Tony
Smith's lecture notes (Univ. of Penn. ESE
502)\footnote{
\url{http://www.seas.upenn.edu/~ese502/NOTEBOOK/Part_I/3_Testing_Spatial_Randomness.pdf}}. In
the context of animal capture-recapture studies, the CSR hypothesis is
manifestly false, purely on biological grounds. Typically individuals
will be clustered or more uniform (for territorial species) than
expected under spatial randomness. That said, it may be a reasonable
approximation to truth in some situations and, as we noted above,
might be largely irrelevant so far as obtaining estimates of density
is concerned. Furthermore, in realistic sample sizes we might expect
relatively low power to detect departures from CSR although this is a
research question worthy of attention (see section XX.YY).

Before proceeding with the development of a framework for evaluating
the point process model - we note that if $N$ is fixed, the resulting
point process is not, strictly speaking, one of "complete spatial
randomness". This is because when $N$ is fixed, a slight bit of
correlation is induced in the number of points within any particular
subset of the state-space. That said, this is negligible for most
purposes and, besides, we use a simulation based approach to testing
in which we simulate under the appropriate model.  But the point is
that CSR is not really the conventional term - maybe we call this
"uniformity".  The basic technical framework for evaluating the CSR
hypothesis is that the cell counts i.e., precisely those we computed
to produce a density map, should have a binomial distribution and we
can use a standard chi-square goodness-of-fit test to evaluate that.
It will usually suffice to approximate the binomial cell counts by
Poisson cell counts in which case we can use the classical
"index-of-dispersion" test (e.g., Illian et al. 2008; p. 87):
\[
   I =  (cells -1)*s^2/\bar{n}
\]
which has approximately a chi-square on $(ncells - 1)$ df.  If $s^2/nbar
> 1$ this suggests clustering and if $s^2/nbar <1$ then this suggests the
point process is too regular.  I think this test is sensitive to the
number and size of the grid cells chosen and I don't know much about
that but it should be researched.

{\it The} important technical issue is that we don't observe the point
process and so the standard statistics for evaluating CSR cannot be
computed directly.  However, using Bayesian analysis, we do have a
posterior sample of the underlying point process and so we suggest
computing the posterior distribution of the chi-square statistic and
seeing how it compares to 1. As an alternative to the chi-square
statistic based on CSR there are various "nearest-neighbor"
methods. For example EXAMPLE XXXXXXXXX which is also easy to compute.

To evaluate the uniformity assumption (i.e.,
``complete spatial randomness'') for the activity centers, we used a
standard chi-square goodness-of-fit test statistic, based on gridding
the state-space of the point process into $g=1,2,\ldots,G$ cells, and
we tabulated $n(g)$ the number of activity centers in each grid
cell. A standard goodness-of-fit statistic for CSR is based on the
ratio of the variance of $n(g)$ to the mean (see Table 8.3 in Cressie
1996). In particular, in parametric likelihood theory, $I = (G
-1)*s^2/\bar{n}$ should have a chi-square on $(G - 1)$ df under the
CSR hypothesis.  However, we used this statistic as the basis for our
Bayesian p-value calculation, comparing a posterior sample of $I$
computed using each posterior realization of the point process with a
value of $I$ obtained by simulating ${\bf s}$ under CSR.

\subsection{Computing the cell counts}

In chapter 4 we talked about computing summaries of individual
locations, such as for producing a density map. We need these for the
GoF analysis�..In the present context, the number of individuals
living in any well-defined polygon is a derived
parameter. Specifically, let $B(x)$ indicate a pixel
 centered at x then
$N(x)=sum_{i} I(s[i] in B(x))$ is the population size of box B(x), and
$D(x) = N(x)/||B(x)||$ is the local density. These are just "derived
parameters" (see Chapt. 2) which are estimated from MCMC output using
the appropriate Monte Carlo average. Note that we are assuming in our
illustration that N is known and so this is easily done by taking all
of the output for MCMC iterations m=1,2,�, and doing this:
\[
   N(x,m) = sum_{z[i,m]=1} I(s[i,m] in B(x))
\]
Thus, N(x,1),N(x,2),, is the Markov chain for parameter N(x).  

It is worth emphasizing here that density maps will not usually appear
uniform despite that we have assumed that activity centers are
uniformly distributed because the observed encounters of individuals
provide direct information about the location of the i=1,2,�.,n
activity centers and thus their "estimated" locations will be affected
by the observations. In a limiting sense, were we to sample space
intensely enough, every individual would be captured a number of times
and we would have considerable information about all N point
locations. Consequently, the uniform prior would have almost no
influence at all.

\section{Is SCR0 adequate for the wolverine data?} 

We used the posterior output from the wolverine model fitted previous
to compute a relatively coarse version of a density map, using a 10 x
10 grid (Figure XXX.YYY) and using a 30 x 30 grid (Figure XXYYZZ). A
couple of things are noteworthy: First is that as we move away from
"where the data live" - away from the trap array - we see that the
density approaches the mean density (lambda = XX per grid cell �
XXX.YYY how is this computed?). This is a property of the estimator
when the "detection function" decreases sufficiently
rapidly. Relatedly, it is also a property of statistical smoothers
such as splines, kernel smoothers, and regression smoothers -
predictions tend toward the global mean as the influence of data
diminishes. Another way to think of it is that it is a consequence of
the prior - which imposes uniformity, and as you get far away from the
data, the predictions tend to the prior. The other thing to note about
this map is that density is not 0 over water. This might be perplexing
to some who are fairly certain that wolverines do not like
water. However, there is nothing about the model that recognizes water
from non-water and so the model predicts over water {\it as if} it
were habitat similar to that within which the array is nested. But,
all of this is ok as far as estimating density goes and, furthermore,
we can compute valid estimates of N over any well-defined region which
presumably wouldn't include water if we so choose.
 
\section{Omnibus and Encounter Model Testing}

trap frequencies

individual frequencies

trap x individual frequencies (\# traps x \# caps total)


\section{Is SCR0 adequate for the Fort Drum black bear data?} 

\section{A simulation study under alternatives}

The point of this section is to evaluate whether we can effectively reject goodness of fit under various alternatives. In particular, we consider two specific modes of lack of fit:

(1)	Can we detect a lack of fit in the detection model?

(2)	Can we detect a lack of fit in the point process model?

4 or 5 detection models and 3 ponit process models.

7 x 7 grid, set to up to yield 50 individuals out of 100. Also try 49 pseudo-random or systematic points.


\subsection{Point Process Alternatives}

1.	Extreme regularity
2.	Somewhat? Regular
3.	Extreme clustering (multiple guys with same home range center)
4.	Somewhat clustered
5.	Complete randomness
6.	Spatial drift in the form of a spatial covariate or trend


We conducted a limited simulation study to evaluate the basic SCR
model under point processes that deviate from complete spatial
randomness. (To be concise we might use the term "uniformity" because
the abbreviation CSR looks too much like SCR.) Specifically, we
consider two deviations from complete spatial randomness: clustered
points, and points that are more systematically distributed than under
complete spatial randomness. We evaluate two questions: (1) how
sensitive is the density estimate to violation of complete spatial
randomness? (2) how much power does the GoF test have?


It is clear that there are many possible influences of both power and
effect of tests for CSR.  For example, N and lambda interact to affect
the expected size of the data set (n individuals) and also sigma
interacts with trap configuration and spacing to affect the number of
unique traps that individuals are captured in. It would be impossible
to catalog an exhaustive set of "what ifs" and so, instead, we focus
on the limited situation where N=100 on a state-space with parameters
set to obtain about 44 individuals on average. We simulated encounters
on a 5x5 grid of traps, unit spacing, with a 2 unit buffer defining
the state-space.

We simulated 3 point process models: Model 1 is CSR. Model 2 is a
Poisson cluster process (PCP) and Model 3 is a point process generated
to be more systematically distributed than CSR. For the Poisson
cluster process, 100 "parents" were distributed randomly over the
state space and the number of offspring for each parent was assumed to
be Poisson with mean 4. Offspring locations were generated according
to a bivariate normal distribution around the parent location, with a
standard deviation of 0.5.  This generated a list of many more than
100 final offspring locations within the state-space, and so we kept
the first 100 points within the state-space and discarded the
remaining. Effectively the last cluster is truncated if the cumulative
sum of offspring within the state-space is not precisely equal to 100.
For the systematic point process we generated 100 uniformly
distributed points and used that as a starting point to obtain the
"space-filling design" (Royle and Nychka XXXX) which is implemented in
the R package {\it fields} (Nychka et al XXXX) using the function {\it
  cover.design}.


Things to vary�.

Point process�. [random, clustered, regular]   3 levels 

Trapping grid: 5 x 5 unit spacing. Also try random traps?   2 levels 

State-space: buffered by 2 or 3 units.   2 levels

[N,lambda] fixed at a single value.

[sigma] = .75 

GoF grid size = 100 cells, 225 cells, 400 cells with 0, 1 or 2 buffer?

Table:
                      Mode N  Mean N     RejectCSR   meanN(reject) modeN(reject)
CSR
PCP
SYS


4.6.2. GoF under complete spatial randomness.

4.6.3. GoF under the Poisson cluster process.

4.6.4. GoF under the inhibition process.




\section{ Summary and Outlook  }


Model fit: other ideas:
Richard brings up a good point which is related to using
cross-validation.  Given that SCR models are a type of model for
spatialy correlation, modeling spatial dependence in counts, similar
to the Wolpert and Ickstadt idea, why not use techniques commonly used
in spatial models for SCR models?  Cross-validation is a standard
method of fitting spatial models such as kriging and splines. So we
could as well use cross-validation based on the TRAP-SPECIFIC
frequencies or some other measure.  We should try that here.

\subsection{What to do if CSR rejected?}

What can we do? We don't know�.. depends on the laternative and how sensitive estimates are?









