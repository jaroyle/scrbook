\documentclass{book}

\usepackage{amsmath}
\usepackage{amsfonts}
\usepackage{graphicx}
\usepackage{lineno}

\linenumbers

\begin{document}

\chapter{State-space Covariates}

Underlying all spatial capture recapture models is a point process
model describing the distribution of individual activity
centers (${\bf s}_i$) within the state space ($\cal{S}$). So far we have focused our
discussion on the homogeneous binomial point process,
${\bf s}_i \sim Uniform({\cal S}), i=1,2,\dots,N$, where $N$ is the
size of the population. This is often referred to as a model of
``complete spatial randomness'' (CSR) because the intensity of the
activity centers is constant across the study area and the activity
centers are distributed independently of each other.

The CSR assumption is often viewed as restrictive
because ecological processes such as
territoriality and habitat selection typically result in non-random
distributions of organisms. We have argued, however, that the CSR
assumption is less restrictive than may be recognized because the
homogeneous point process actually allows for infinite
possible configurations of activity centers. Furthermore, given enough data,
the uniform prior will have very little influence on the estimated
locations of activity centers. Nonetheless, the homogeneous point
process model does not allow one to model population density using
covariates---a central objective of much ecological research.
For example, a homogeneous point process model
may result in a density surface map indicating that individuals were
more abundant in one habitat than another, but it does not do so
explicitly. A more direct approach would be to model density using
covariates as is done in generalized linear models (GLMs), using a
link function to connect the intensity parameter to the linear predictor.

In this chapter we will present a method
for fitting inhomogeneous binomial point process models using
covariates in much the same way as is done using GLMs. The
covariates we consider differ
from those covered in previous chapters, which were typically
attributes of the animal ({\it eg} sex, age) and were used to model movement or encounter
rate. In contrast, here we wish to
model covariates that are defined for all points in the
the state-space, which we will refer to as
state-space covariates, or spatially-referenced covariates. These may
include continuous covariates such as elevation, or discrete
covariates such as habitat type.

citet{borchersefford:2010} were the first to propose an
inhomogeneous point process model for SCR models, and our approach is
similar to theirs with the exception that we will use a binomial
rather than a Poisson model because the binomial model is
easily integrated into our data augmentation scheme and is consistent
with the objective of determining how a {\it fixed} number of activity
centers are distributed with respect to covariates.

The method we use to accomodate inhomogeneous binomial point process
models within our MCMC algorithm is simple---we
replace the uniform prior with a prior describing the
distribution of
the $N$ activity centers conditional on the covariates. Development of
this prior, which does not have a
standard form, is a central component of this chapter.


\section{Homogeneous point process revisited}

The homogeneous Poisson point process may be the most commonly-used model of
spatial randomness in ecology, thus it is helpful to compare it with
the binomial model that we will expand upon in this chapter. The
primary difference between the two models is that the binomial model
conditions on $N$, the number of points to be simulated; whereas under
the Poisson model $N$ is random. Here is some simple R code to
illustrate this difference.

\begin{verbatim}
mu <- 4                            # intensity
Np <- rpois(1, mu)                 # Np is random
PPP <- cbind(runif(Np), runif(Np)) # Poisson point process

Nb <- 4
BPP <- cbind(runif(Nb), runif(Nb)) # Binomial point process
\end{verbatim}

Note that in both models, the $N$ points are independent
of one another and distributed uniformly
throughout $\mathcal{S}$. Thus, the ``point process
intensity'' at any point $x \in \cal{S}$ is $\mu(x) = 1 /
A(\mathcal{S})$ where $A(\mathcal{S})$ denotes the area of the
state-space. For example, if the area of our state-space is 4 km$^2$,
under a homogeneous model, the intensity is $\mu(x) = 1/4$.

Although the Poisson model is typically described in terms of $\mu(x)$,
the binomial model is not; rather, it
is more common to consider a discrete state space, which we mention
here for clarity. Suppose that $\mathcal{S}$ is divided into $K$ non-overlapping
regions, the number of points in each region $B$ is $n(B) \sim Bin(N, p)$
where $p(B) = A(B)/A(\cal{S})$, ie $p(B)$ is simply the fraction of
the state-space area in $B$.

One additional property of the binomial model is that the $K$
realizations of $n$ are not independent since they must sum to
$N$. Instead, the model for the entire vector
is ${\bf n(B)} \sim Multinomial(N, {\mathbf{\pi}} = (p_1, p_2, \dots, p_K))$.

\section{Inhomogeneous binomial point process}

As with the homogeneous model, the inhomogeneous binomial point process
model is developed conditional on $N$. The primary distinction is that
the uniform distribution is replaced with another distribution
allowing for the intensity parameter to vary spatially. To arrive at
this new distribution, define $\mu(x,\mathbf{\alpha})$ to be a function of
spatially-referenced covariates ($\mathbf{\alpha}$) available at all regions of the state
space.  Subsequently we will drop the vector of cofficients from our
notation to be concise. Since an intensity must be strictly
positive, it is natural to model $\mu(x)$ using the log-link.
\[
\log(\mu(x)) = \sum_{j=1}^J \alpha_j v_j(x), \quad  x \in \cal{S}
\]
where $\alpha_j$ is the regression coefficient for covariate
$v_j(x)$. To be clear, $v(x)$ is the value of any covariate, such as
habitat type or elevation, at location $x$.  This equation should look
familiar because it is the standard linear model used in log-linear
GLMs with the exception that we have no need
for an intercept because it would be entirely confounded with
$N$. This is intuitive since an intercept would
represent the expected value of $N$ when $\alpha=0$, but we already
have a parameter in the model for $E[N]$, namely $E[N] =
\psi M$. Thus an intercept would be
redundant, and without it we are still able to achieve our goal of
describing the distribution of $N$ activity centers as a function of
spatial covariates.

Now that we have a model of the intensity parameter $\mu(x)$,
we need to develop the associated probability density function to use
as a prior in the place of the uniform prior used in the homogeneous
model. Remembering that
the integral of a pdf must be unity, we can create a pdf by dividing
$\mu(x)$ by a normalizing constant, which in this case is the integral
of $\mu(x)$ evlauated over the entire
state-space. The probability density function is therefore
\begin{equation}
f(x) = \frac{\mu(x)}{\int_{\mathcal{S}} \mu(x)\, \mathrm{d}x}, \quad x \in
\mathcal{S}
\label{eq:pdf-ipp}
\end{equation}
Substituting this distribution for the
uniform prior allows us to fit inhomogeneous binomial point process
models to spatial capture-recapture data. We can also use this
distribution to obtain the expected number of individuals in any given
region. Specifically, the proprotion of $N$ expected to occur in any
region $B$ when heterogeneity in density is present is $p(B) = \int_B
f(x)\, \mathrm{d}x$. Once again these are
also the multinomial cell probabilities if the regions are
non-overlapping and compose the entire state-space.

As a practical matter, note that the integral in the
demoninator of $f(x)$ is evaluated over space, and since we almost always regard as
space as two-dimensional, this is a two-dimensional integral that can
be approximated using the methods discussed in ChXX. These methods include
Monte Carlo integration, Gaussian quadrature, etc... One
issue that often arises is that continuous spatial covariates are
\emph{not} represented as continuous, and instead are
defined on discrete grids, called ``rasters'' in GIS-speak. In such
cases, the integral in the denominator can be replaced with a sum over
all pixels citep(diggle:2003), which is much more efficient
computationally.

The inhomogeneous point process model for the activity centers results
in another point process model for the
observation process, which we have previously called $\lambda(x)$. As
a reminder, $\lambda(x)$ is the expected number of captures for a trap
at point $x$. As was true for the homogeneous model, this
intensity function is a convolution of the point process intensity
($\mu(x)$) and the encounter rate function ($g(x,s)$),
$\lambda(x) = \mu(x) g(x,s)$.

In the next section we walk through a few examples, building up from
the simplest case where we actually observe the activity centers as
though they were data. In the second example, we fit our new model to simulated
data in which density is a function of a single continuous
covariate. In the last example, we model the intensity of activity
centers for a real dataset collected on tigers.

\section{Examples}

\subsection{Simulation and analysis of inhomogeneous point processes}

In SCR models, the point process is not directly observed, but in
other contexts the data in hand are the point locations
themselves. Examples include the locations of disease
outbreaks or the locations of trees in a forest. Fitting inhomogeneous
point process models to such data is straight-forward and illustrates
the fundamental process that we will later embed in our MCMC algorithm
used to fit SCR models.

Suppose that we knew the locations of 100 animals' activity
centers. To estimate the intensity surface $\mu(x)$ underlying these points, we
need to express the likelihood of our data for various values of
$\alpha$. Given the pdf $f(x)$, if we assume that the points are
conditionally independent of one another, we may write
the likelihood as the product
of $R$ such terms, where $R=100$ is the sample size in this case,
\emph{ie} the observed number of activity centers.
\[
\mathcal{L}({\bf \alpha} | {\bf x}_i) = \prod_{i=1}^R f(x_i)
\]
Having defined the likelihood we may now obtain the posterior for
$\alpha$ using Bayesian methods, or we can find the maximum likelihood
estimates (MLEs) using standard numerical methods as is demonstrated
below.

Simulating data under an inhomogeneous point process model is often
accomplished using indirect methods such as rejection
sampling. Rejection sampling proceeds by
simulating data from a standard distribution and then accepting or
rejecting each sample using probabilities defined by the distribution
of interest. For more information, readers should consult an
accessible text like citet{robertcasella:2010}. In our example, we
simulate from a uniform distribution and then accept or reject using
the (scaled) probability density function $f(x)$. Note that we first define a
spatial covariate (elevation) that is a simple function of the spatial
coordinates increasing from the southwest to the northeast of our
state-space. It should be obvious that such functional forms of
covariates are rarely available, which is why continuous spatial
covariates are more often measured on a discrete grid. Nonetheless, we
will proceed with our truly continuous covariate for illustrative
purposes. However, to evaluate the integral we end up discretizing the
state-space anyway.

%\newpage

\begin{small}
\begin{verbatim}
# spatial covariate
# Elevation as a function of the coordinates at point x
elev.fn <- function(x) x[,1]+x[,2]

# 2-dimensional integration over [-1, 1] square
int2d <- function(alpha, delta=0.02) {
  z <- seq(-1+delta/2, 1-delta/2, delta)
  len <- length(z)
  cell.area <- delta*delta
  S <- cbind(rep(z, each=len), rep(z, times=len))
  sum(exp(alpha*elev.fn(S)) * cell.area)
  }

# Simulate PP using rejection sampling
set.seed(395)
N <- 100
count <- 1
s <- matrix(NA, N, 2) # matrix to hold simulated activity centers
alpha <- 2 # parameter of interest
Q <- max(c(exp(alpha*elev.min) / int2d(alpha),
           exp(alpha*elev.max) / int2d(alpha))) # Rejection sampling bound
while(count <= 100) {
  x.c <- runif(1, -1, 1); y.c <- runif(1, -1, 1) # proposed activity center
  s.cand <- cbind(x.c,y.c)
  elev.min <- elev.fn(cbind(-1,-1)); elev.max <- elev.fn(cbind(1,1))
  pr <- exp(alpha*elev.fn(s.cand)) / int2d(alpha)
  if(runif(1) < pr/Q) {
    s[count,] <- s.cand # accepted proposals
    count <- count+1
    }
  }
\end{verbatim}
\end{small}


\begin{figure}
\centering
\includegraphics[width=7cm,height=7cm]{figs/elevMap}
\label{fig:elevMap}
\end{figure}

The simulated data are shown in Fig~\ref{fig:elevMap}. High elevations
are represented by light green and low elevations by dark green. The
activity centers of one hundred animals are shown as
points, and it is clear that these simulated animals prefer the high
elevations.  The underlying model describing this preference is
$\log(\mu(x)) = exp(\alpha \times Elevation(x))$
where $\alpha=2$ is the parameter to be estimated and $Elevation(x)$
is a function of the coordinates at $x$, as displayed on the map.

Given these points, we will now estimate $\alpha$ by minimizing the
negative-log-likelihood using \verb+R+'s \verb+optim+ function. Since,
we only have one parameter to estimate, we use method = ``Brent''.

\begin{small}
\begin{verbatim}
# Negative log-likelihood
nll <- function(beta) {
  -sum(beta*cov(S[,1], S[,2]) - log(int2d(beta)))
  }
starting.value <- 0
fm <- optim(starting.value, nll, method="Brent",
            lower=-5, upper=5, hessian=TRUE)
c(Est=fm$par, SE=sqrt(1/fm$hessian)) # estimates and SEs
\end{verbatim}
\end{small}


Maximizing the likelihood took a small fraction of a second, and we
obtained an estimate of $\hat{\alpha}=2.01$. Not bad! We could plug in
this estimate to our linear model at each point in the state-space to
obtain the MLE for the intensity surface.

This example demonstrates
that if we had the data we wish we had, {\it ie} if we knew the
coordinates of the activity centers, we could easily estimate the
parameters governing the underlying point process. Unfortunately, in
SCR models, the activity centers cannot be directly observed, and thus
are latent variables that we must either estimate or at least integrate out
of a likelihood. The good news is that capturing an individual at
multiple locations in space provides us with the information needed to
estimate the location of its activity.

\subsection{Fitting inhomogeneous point process SCR model}

As we have stated before, one
of the nice things about hierarchical models is that they allow us to
break a problem up into a series of simple conditional
relationships. Thus,
we can simply add the methods described above into our existing MCMC
algorithm to simulate the posteriors of $\alpha$ conditional on the
simulated values of $\mathbf{s}_i$. To demonstrate, we will continue with
the previous example. Specifically, we will overlay a grid of
traps upon the map shown in Fig.~\ref{fig:elevMap}. We will then
simulate capture histories conditional upon the activity centers shown
on the map. Then, we will attempt to estimate the activity center
locations as though we did not know where they were.

\begin{small}
\begin{verbatim}
# Create trap locations
xsp <- seq(-0.8, 0.8, by=0.2)
len <- length(xsp)
X <- cbind(rep(xsp, each=len), rep(xsp, times=len))

# Simulate capture histories, and augment the data
ntraps <- nrow(X)
T <- 5
y <- array(NA, c(N, ntraps, T))

nz <- 50 # augmentation
M <- nz+nrow(y)
yz <- array(0, c(M, ntraps, T))

sigma <- 0.1  # half-normal scale parameter
lam0 <- 0.5   # basal encounter rate
lam <- matrix(NA, N, ntraps)

set.seed(5588)
for(i in 1:N) {
    for(j in 1:ntraps) {
        distSq <- (s[i,1]-X[j,1])^2 + (s[i,2] - X[j,2])^2
        lam[i,j] <- exp(-distSq/(2*sigma^2)) * lam0
        y[i,j,] <- rpois(T, lam[i,j])
    }
}
yz[1:nrow(y),,] <- y # Fill
\end{verbatim}
\end{small}

Now that we have a simulated capture-recapture dataset $y$, and we have
augmented it to create the new data object $yz$, we are ready to
begin sampling from the posteriors. A commented Gibbs sampler written in R is
available online. You will see that only two small parts of the R code
were changed. First, we need to update the parameter $\alpha$
conditional on all other parameters in the model. The code to do so is:

\begin{small}
\begin{verbatim}
D1 <- int2d(beta1, delta=.05)
beta1.cand <- rnorm(1, beta1, tune[3])
D1.cand <- int2d(beta1.cand, delta=0.05)
ll.beta1 <- sum(  beta1*cov(S[,1],S[,2]) - log(D1) )
ll.beta1.cand <- sum( beta1.cand*(S[,1]+S[,2]) - log(D1.cand) )
if(runif(1) < exp(ll.beta1.cand - ll.beta1) )  {
    beta1<-beta1.cand
}
\end{verbatim}
\end{small}

Next, we need to use $\alpha$ in the prior for the activity centers:

\begin{small}
\begin{verbatim}
#ln(prior), denominator is constant
prior.S <- beta1*cov(S[i,1], S[i,2]) # - log(D1)
prior.S.cand <- beta1*(Scand[1] + Scand[2]) # - log(D1)
if(runif(1)< exp((ll.S.cand+prior.S.cand) - (ll.S+prior.S))) {
    S[i,] <- Scand
    lam <- lam.cand
    D[i,] <- dtmp
    }
\end{verbatim}
\end{small}

Applying this modified sampler to our data we obtain posterior
distributions summarized in Table~\ref{tab:simIPP}. Mixing is good, and as usual,
life is very nice when we are working with simulated data.

\begin{table}
\centering
\begin{tabular}{lccccc}
Parameter & Mean & SD  & q0.025 & q0.5 & q0.975 \\
\hline
$\alpha$    &&&&& \\
$\lambda_0$  &&&&& \\
$\sigma$    &&&&& \\
$N$        &&&&& \\
Density     &&&&& \\
\hline
\end{tabular}
\label{tab:simIPP}
\end{table}

It is worth noting that, although this method of fitting inhomogeneous
point process models does not require much modification of our custom MCMC
code, it is not so trivial to
implement these models in BUGS. The reason being
that the prior we use is not a standard distribution available by
default. It is, however, possible to use arbitrary distribution in
BUGS using the ??-trick.. Anyone remember how to do this? Here is an
example.


\subsection{The tiger data}

Hopefully Arjun can send me something.


\section{Summary}

When spatially-referenced covariates are available, we can model
density by replacing the uniform prior on the activity centers with a
prior based on a log-linear function of covariates.



\end{document}