\chapter{
Modeling Encounter Probability
%%%%Modeling Covariate Effects in SCR Models
}
\markboth{Encounter probability}{}
\label{chapt.covariates}

\vspace{.3in}

\section{Introduction}

% In previous chapters we showed how to fit basic spatial
% capture-recapture models using Bayesian analysis (in {\bf WinBUGS} or
% {\bf JAGS};
% Chapt. \ref{chapt.scr0}) or by classical likelihood methods
% (Chapt. \ref{chapt.mle} or using \mbox{\tt secr}).  These basic models involved only constant
% parameter values that did not vary in response to covariates of any
% type.  However, in practice, investigators are invariably concerned
% with explicit factors or covariates that might influence variation in
% parameters. Traditionally, in the non-spatial capture recapture
% literature, such models were called ``model $M_t$'', ``model
% $M_h$'', or ``model $M_b$'', identifying models that account for
% variation in detection probability as a function of time, ``individual
% heterogeneity'' or ``behavior'', where behavior describes
% whether or not an individual had been previously captured.  In SCR
% models, more complex covariate models are possible because we might
% also have trap-specific covariates, or covariates that vary spatially
% over the landscape.

% So far, we have only covered 
% how to use only the basic model in various software packages and 
% the suite of possible encounter models
% (e.g., the Binomial, Poisson, and Multinomial encounter models) for
% dealing with different types of sampling.  However, we have not
% considered different detection functions or covariates that can affect
% the parameters of the detection function, including those that may
% arise from the individual or the trap device.
% Most detection functions include a baseline encounter
% rate termed $\lambda_0$ (or $g_0$ for the detection probability when
% we use a logit link for the detection function) and a shape parameter
% termed $\sigma$, which takes on different interpretations depending on
% the selected function.  Such covariates
% include time (e.g., day of year, or season), behavior (e.g., has the
% individual been previously captured), sex of the individual, and trap
% type (e.g., various camera types, or different constructions for hair
% snares).


In previous chapters we showed how to fit basic spatial
capture-recapture models using Bayesian analysis (in {\bf WinBUGS} or
{\bf JAGS};
Chapt. \ref{chapt.scr0}) or by classical likelihood methods
(Chapt. \ref{chapt.mle} or using \mbox{\tt secr}). We covered a suite of possible encounter models
(e.g., the Binomial, Poisson, and Multinomial) for
dealing with different types of sampling. We have not, however,
considered different detection functions or covariates that can affect
the parameters of the detection function, including those that may
arise from the individual or the trap device. In practice, 
investigators are invariably concerned
with explicit factors or covariates that might influence variation in
parameters. Such covariates
include time (e.g., day of year, or season), behavior (e.g., is there an 
effect of trapping on subsequent capture probabilities), sex of the individual, and trap
type (e.g., various camera types, or different constructions for hair
snares). Traditionally, in the non-spatial capture recapture
literature, such models were called ``model $M_t$'', ``model
$M_h$'', or ``model $M_b$'', identifying models that account for
variation in detection probability as a function of time, ``individual
heterogeneity'' or ``behavior'', where behavior describes
whether or not an individual had been previously captured. In SCR
models, more complex covariate models are possible because we might
also have trap-specific covariates, or covariates that vary spatially
over the landscape, and because we generally have more than one 
parameter describing the detection function:
Most detection functions include a baseline encounter
rate termed $\lambda_0$ XXX $p_0$ ??? XXX and a shape parameter
termed $\sigma$, which takes on different interpretations depending on
the selected function. 


In this chapter, we generalize the basic SCR model to accommodate both 
alternative detection functions as well as
many different kinds of covariates. We focus on the Binomial encounter
model used in Chapts. \ref{chapt.scr0} and \ref{5} and the half-normal 
detection function (although we do show you some alternatives to the half-normal),
but the extension to other encounter and detection models is
straightforward.  Specifically, we consider three distinct types of
covariates - those which are fixed, partially observed or completely
unobserved (latent).  Fixed covariates are those that are fully
observed; for example, the date of all sampling occasions.  Partially
observed covariates are those which are not known for all
observations; for example, the sex of an individual cannot always be
determined from photos taken during camera trapping.  Even if we are
able to observe the sex of all individuals sampled, we cannot know it
for those individuals never observed during the study.  And finally,
unobserved covariates are those which we cannot observe at all, for
example, the home range size of individuals, or unstructured random
``individual effects''.


We will see that models containing these different types of
covariates are relatively easy to describe in the {\bf WinBUGS} or
{\bf JAGS} languages, and
therefore to analyze using Bayesian analysis of the joint likelihood
based on data augmentation thus providing a coherent and flexible
framework for inference for all classes of SCR models.  Throughout the
chapter, we will continue to develop the analysis of the black bear
study introduced in \ref{Chapter 3}, using the software
\jags.  We also
consider the likelihood analysis of many of these models; to do so, we
will demonstrate the use of the R package \secr~ and how to do model
comparison with AIC (this section is at the end of the chapter \ref{likelihood.secr}).
There are other types of covariates that we do {\it not} cover in this
chapter; for example, covariates that vary across the
landscape might affect density and we consider these covariates in
Chapt. \ref{chapt.state-space}.
Alternatively, these landscape covariates might affect the way individuals use
space. There are probably very few circumstances under which animals use all 
space equally and we develop more realistic models of encounter
probability in which covariates affect space usage in Chapt. \ref{chapt.ecoldist}.


\section{Detection Functions}

In Chapt. \ref{chapt.scr0}, we developed the basic spatial capture
recapture model using a standard distance function based on the kernel
of a normal (Gaussian) probability distribution:
\[
p_{ij} = p0 \exp(-\alpha_{1} *||{\bf s}_{i}-{\bf x}_{j}||^2)
\]
where $||{\bf s}_{i}-{\bf x}_{j}||$ is the distance between ${\bf
  s}_{i}$ and ${\bf x}_{j}$ and
\[
\alpha_{1} = 1/(2*\sigma^2).
\]
We argued (sec. \ref{scr0.sec.implied}) that this model corresponds to
an explicit model of space usage -- namely, that individual locations
are draws from a bivariate normal distribution. We also mentioned that
other detection models are possible, including a logit model of the
form:
\begin{equation}
	\mbox{logit}(p_{ij}) = \alpha_{0} + \alpha_1 ||{\bf s}_{i}-{\bf x}_{j} ||.
\label{covariates.eq.logit}
\end{equation}

However, there's nothing preventing us from constructing a myriad of
other models for encounter probability, as long as detection decreases
with distance (otherwise $N$ would be unidentifiable). The most
commonly used detection functions are also those used in the distance
sampling literature: the half-normal, the hazard, and the negative
exponential.  The negative exponential model is: 
\[
p_{ij} = p0*\exp(-\alpha_{1} *||{\bf s}_{i}-{\bf x}_{j}||)
\]
or we could use the general power model which includes both the
Gaussian and exponential models \citep{russell_etal:2012}: XXX What is
this model called? CITE from Buckland et al. XXXXX
XX hazard rate is Hayes and Buckland (1983), but not sure which Buckland we want hereXXX
\[
p_{ij} = p0*\exp(-\alpha_{1} *||{\bf s}_{i}-{\bf x}_{j}||^{\theta} )
\]
The {\bf R} package
{\tt secr} allows the user to access 12 different detection models, of which
some are only used for simulating data (see \ref{covariates.tab.detmodels}). These detection
functions can  also be coded in {\bf R}, {\bf WinBUGS},
{\bf JAGS} etc.

Insofar as all these detection functions are symmetric and stationary, they are pretty
coarse descriptions of space usage by real animals. But this is not to
say they are inadequate descriptions of the data XXX this is no longer true, Andy
is showing that they might be inadequate now, right?? XXX
We don't believe in doing too much
selection among these different detection models because there is no biological basis for
choosing any one of over any other. We do describe more
realistic models in Chapt. XXXXX.


\begin{table}[ht]
\centering
\caption{Distance functions available in \secr.  (Table taken from the \secr~
help files). Notation deviates from that used in the text.
In this table $g_{0}$ is the baseline encounter rate or probability
parameter used in \secr~ but this is equivalent to our $p_{0}$ or
$\lambda_{0}$ depending on context. $d$ is distance defined as we have done throughout,
as the distance between the activity center and the trap.
One can read more on this specific table by loading the \secr~ package and using the
{\tt help} command in {\bf R} ({\tt ?detectfn}).
}
\begin{tabular}{cccl}
\hline \hline
Code & Name & Parameters & Function  \\ \hline
0 & half-normal &$g_0$, $\sigma$          &  $g(d) = g_0 * exp\{-d^2 / (2  \sigma^2) \}$  \\
1 &hazard rate  & $g_0$, $\sigma$, z      &  $g(d) = g_0 * (1 - exp(- (d / \sigma) ^(-z) ))$ \\
2 &exponential   &$g_0$, $\sigma$    &  $g(d) = g_0 * exp(- d / \sigma)$ \\
3 &compound half-normal  & $g_0$, $\sigma$, z & $g(d) = g_0 * [1 - \{1 - exp(-d^2 / (2 \sigma^2))]^z\}$ \\
4 &uniform     & $g_0$, $\sigma$     &
\parbox[t]{2in}{ $g(d) = g_{0}, d \leq \sigma$; \\
                 $g(d)= 0$, otherwise
} \\
5 &w exponential            & $g_0$, $\sigma$, w &
\parbox[t]{2in}{ $g(d) = g_{0}, d < w$; \\
                 $g(d) = g_{0} \exp(- (d - w) / \sigma)$, otherwise
} \\
6 &annular normal           & $g_0$, $\sigma$, w & $g(d) = g_0 * exp(-(d-w)^2 / (2 \sigma^2))$ \\
7 &cumulative lognormal     & $g_0$, $\sigma$, z & $g(d) = g_0 [1 -F{(d-\mu)/s)}]$  \\
8 &cumulative gamma         & $g_0$, $\sigma$, z  & $g(d) = g_0 \{ 1 - G (d; k,  \theta) \}$  \\
9 &binary signal strength   & $b_0$, $b_1$       & $g(d) = 1 - F \{- (b_0 + b_1 * d) \}$ \\
10&signal strength          & $\beta_0$, $\beta_1$, sdS  &
  $g(d) = 1 - F[ \{c - (\beta_0 + \beta_1 * d)\} / sdS]$  \\
11&signal strength spherical&  $\beta_0$, $\beta_1$, sdS & 
\parbox[t]{2in}{ $g(d) = 1 - F[\{c - (\beta_0 + \beta_1 * (d-1)- 10 * log10 ( d^2 ) ) \} / sdS ]$ 
} \\
%$g(d) = 1 - F[\{c - (\beta_0 + \beta_1 * (d-1)$ \\ 
%& & & $- 10 * log10 ( d^2 ) ) \} / sdS ]$ \\

\end{tabular}
\label{covariates.tab.detmodels}
\end{table}

By changing the detection function and the specification of
$\alpha_1$, we can basically create any distance function for the
data. It is important to note that $\sigma$ is not comparable under
these different distance functions for detection.  Additionally, the
relationship between $\sigma$ and home range radius does not have
a precise definition under distance functions other than the standard half-normal.  We
demonstrate how to fit different distance functions under the Bayesian framework here
and then provide a section on the likelihood analysis (in \secr~) in a 
separate section below.

\section{Bayesian Analysis with {\tt bear.JAGS}}

To demonstrate how to incorporate various types of covariates using
{\bf JAGS}, we will again return to the data collected during the
Ft. Drum bear study.  This data set was first introduced in Chapt. \ref{chapt.closed},
but to refresh your memory, there were 38 baited hair snares that were run
between June and July 2006.  The snares were checked each week for a
total for $K=8$ sample occasions and $n=47$ individual bears were
encountered at least once.  The data are provided in the {\bf R}
package \mbox{\tt scrbook} and an {\bf R} function called {\tt bear.JAGS} allows the user to easily
pick which model to analyze. 
 The function {\tt bear.JAGS}
will set up the data, write the model, define the MCMC specifications 
(e.g., initial values, etc.) and, finally, 
run the selected model in {\bf JAGS}. In addition to choosing which model to run, the user can also specify the number of chains, iterations and length of the burn-in phase. Calling the function will provide all the code to 
implement the models independently as well.
In the following sections we will present the 
model code and output for the most commonly employed models; for all analyses we ran 3 chains with a burn-in of 500 iterations and 20000 saved iterations.  

\section{Bayesian Analysis of detection functions}
% We start here by presenting the basic SCR model with no covariates and
% the half normal distance function.

In panel \ref{covariates.panel.basicSCR}, 
we present the basic SCR model and show how to specify
the negative exponential detection function.  To call each of these from the function {\tt bear.JAGS}
set {\tt model='SCR0'} or  {\tt model='SCRexp'} in the function call, respectively. 

\begin{panel}[htp]
\centering
\rule[0.1in]{\textwidth}{.03in}
{\small
\begin{Verbatim}[commandchars=\\\{\}]
model \{
alpha0 \mytilde dnorm(0,.1)
logit(p0)<- alpha0
alpha1<-1/(2*sigma*sigma)
sigma \mytilde dunif(0, 15)
psi \mytilde dunif(0,1)

for(i in 1:M)\{
 z[i] \mytilde dbern(psi)
 s[i,1] \mytilde dunif(Xl,Xu)
 s[i,2] \mytilde dunif(Yl,Yu)
for(j in 1:J)\{
d[i,j]<- pow(pow(s[i,1]-X[j,1],2) + pow(s[i,2]-X[j,2],2),0.5)
y[i,j] ~ dbin(p[i,j],K)
p[i,j]<- z[i]*p0*exp(- alpha1*d[i,j]*d[i,j])
# p[i,j]<- z[i]*p0*exp(- alpha1*d[i,j])   #exponential distance function
\}
\}
N<-sum(z[])
D<-N/area
\}

\end{Verbatim}
}

\rule[-0.1in]{\textwidth}{.03in}
\caption{
\jags~ model specification for the basic SCR model with half-normal distance
function and the alternative exponential distance function.}
\label{covariates.panel.basicSCR}
\end{panel}

Applying the basic SCR model with the half-normal
distance function provides an estimate of $D = 0.167$ bears per $km^2$ and with the
negative exponential distance function the estimate is virtually the 
same $D = 0.167$.   We can also see that the 97.5\%
percentile for $N$ is 628 (\label{covariates.tab.SCR0exp}), thus not reaching our $M=650$ value, but
close enough that we may want to check that $N$ is not truncated by
this level of data augmentation.  In addition to checking the percentiles, 
we can also see from \ref{covariates.fig.Nscr}, the histogram of $N$, that $N$ 
shows no signs of truncation.  We could also increase
$M$ as another check that our data augmentation is sufficient.

\begin{table}[ht]
\centering
\caption{Posterior summaries of parameter estimates from different SCR models for the Fort Drum black bear data set.}
\begin{tabular}
\hline
Parameter & Mean & SD & 2.5 & 97.5 \\
\hline

$\mbox{SCR}_0$&   &      &           &        \\
\hline
$D$    &   0.17   & 0.022     &  0.122 & 0.207  \\
$N$    &  500.63 & 66.652  & 371   & 628       \\
$p_0$  &    0.11  & 0.014   & 0.081& 0.135     \\
$\psi$  &   0.77  & 0.104    & 0.566 & 0.966    \\
$\sigma$ &  1.99 & 0.131  &1.762 & 2.275     \\

$\mbox{SCR}_{exp}$&   &      &          &        \\
\hline
$D$  &   0.17 &  0.023  &				& 		 \\
$N$   &  511.55 &  68.730  &			& 		 \\
$p_0$  &    0.34 &   0.055  &			& 		 \\
$\psi$  &   0.79  &  0.107  &			& 		 \\
$\sigma$ &  1.12  & 0.094  &			& 		 \\
\end{tabular}
\label{covariates.tab.SCR0exp}
\end{table}

A very important note when using different distance functions is the
interpretation of $\sigma$.  The estimate of $\sigma$ under the negative exponential model is
$1.12$, which is distinct from our
estimate of $\sigma$ under the half normal model, $\sigma = 1.996$.
The interpretation
of $\sigma$ in the two models is really quite distinct. In the normal
model it can be interpreted as the standard deviation of a bivariate
normal movement model whereas the manner in which $\sigma$ relates to
``area used'' for the negative exponential model has nothing to do
with a bivariate normal model of movement.  This highlights that it is
important for the user to know what distance function is used and what
the interpretation of $\sigma$ might be in relation to the home range size.
This relationship was discussed in \ref{chapt.scr0}


We leave the detection functions for now and move onto incorporating
covariates into the model using the {\bf JAGS}
language.  For this part, we will stick with the half-normal distance
model shown in the panel \ref{covariates.panel.basicSCR} above.


\section{Modeling Covariate Effects}


The basic strategy for modeling covariate effects is to include them
on the baseline encounter rate or probability parameter, $p_{0}$ (or
$\lambda_{0}$), or the scale parameter of the encounter model,
$\sigma$.

Broadly speaking, we recognize (here) 2 types of covariates XXXX ABOVE YOU SAY 3 - OBSERVED; PARTIALLY LATENT AND LATENT XXXXXX: Fixed
covariates which are fully observable and might vary by trap alone
(e.g., type of trap, baited or not, disturbance regime, even habitat),
sample occasion (e.g., day of season or weather conditions), or both
(e.g., behavior, weather - if over a large region).  The other class
of covariates are those which vary at the level of the individual (and
possibly also over time).  As a technical matter, and as noted before,
these are different from fixed covariates because we cannot see all of
the individuals and the covariates are almost always incompletely
observed (if at all).  The lone exception is the behavioral response to capture
which is known for all individuals, captured or not (an animal never
captured/observed has never been captured before).  We noted may times
before that space itself (i.e., the activity centers) is a type of
individual covariate and this notation actually helped us derive the
fully spatial capture-recapture model from the traditional,
non-spatial model (Chapt. \ref{chapt.scr0}). We do not get to observe
the activity center for any individuals, but for individuals that are
encountered we get to observe some information about it in the form of
which traps the individual was encountered in.


To develop covariate models, we assume a standard sampling design in which an
array of $J$ traps is operated for $K$ time periods, which produces
encounter histories for $n$ individuals.  For the basic model, there
are no time-varying covariates that influence encounter, there are no
explicit individual-specific covariates, and there are no covariates
that influence density.  For fixed effects, those which we observe
fully, we can easily incorporate these into the encounter probability
model, just as we would do in any standard GLM or GLMM, on some
suitable scale for the encounter probability $p_{ijk}$. For example,
\[
\mbox{logit}(p0_{iji}) = \alpha_0 + \alpha_2*C_{ijk}
p_{ijk} = p0_{ijk} \exp(- \alpha_1*||{\bf s}_{i}-{\bf x}_{j}||^2)
\]

where $C_{ijk}$ is some covariate and
$\alpha_2$ is the coefficient to be estimated.
 How we define specific covariates (e.g., trap specific
versus individual specific) will influence exactly how we include them
in the model. We will see specific examples in the following sections.  
%For example, $C$ might vary by
%individual, trap, sampling occasion or combinations of these.

\subsection{Date and Time}

Often, researchers are interested in the effect of date on the detection
probability. For example, in a long term hair snare study, we may
expect that seasonal shedding will influence our detection
probabilities.  Or we may expect reproductive behaviors to influence
the detection of certain species at certain times of year.  There are
two common ways to incorporate date or time information into a model
for encounter probability. For cases with a small number of sampling
occasions we can fit a time-specific intercept (analogous to ``model
$M_{t}$'' in classical capture-recapture
\citep{otis_etal:1978}). That is, there are $K$ sampling occasion-specific
parameters to reflect potential variation in sampling effort or other
factors that might vary across samples.    Alternatively, we can model parametric
functions of date or time such as polynomial or sinusoidal functions.

In the first case, 
 we allow each sampling
occasion, $k$, to have its own baseline detection probability, e.g.,
\[
\mbox{logit}(p_{0,k}) = \alpha_{0,k}
\]
so that
\[
p_{ijk} = p_{0,k} \exp(- \alpha_1*||{\bf s}_{i}-{\bf x}_{j}||^2).
\]
This description of the model includes $k$ occasion-specific baseline
detection probabilities.  Thus, if we had 4 sampling occasions, we
would have 4 different baseline detection probabilities.  We imagine
that time-specificity of $p_{0}$ would be most useful in situations
where we have just a few sampling occasions (if we have many this
formulation will dramatically increase the number of parameters to be
estimated) or we do not expect a pattern in the timing of the
occasions.

To implement this in \jags, $\alpha_0$ has to be
estimated for each time period $k$ either using an index vector or
dummy variables (as described in Chapt. \ref{chapt.modeling}) and this can be done by only 
changing only a few lines in 
panel \ref{covariates.panel.basicSCR}.
\[
alpha0[k] \mytilde dnorm(0,.1)
logit(p0[k])<- alpha0[k]
.......
y[i,j,k] ~ dbin(p[i,j,k],K)
p[i,j,k]<- z[i]*p0[k]*exp(- alpha1*d[i,j]*d[i,j])
\]

Since the model estimates
a parameter effect for each time period, the encounter histories must be time-dependent. 
Thus, a 3-d data array (called \mbox{\tt bearArray} in our
code), with dimensions
$\mbox{\tt nind} \times \mbox{\tt ntraps} \times \mbox{\tt nreps}$ is required. In addition
to using the 3-d data array, the initial values must be updated so that 
there are $K$ values generated for $\alpha_0$.  And finally, this
means that we have to put in another nested for loop in the
code to account for the $K$ sample occasions.  
A side note: the computation time will increase quite a bit (this model
for the bear data may take up to 15 hours or more on your machine to obtain a sufficient
posterior sample).

Running this model from the function {\tt bear.JAGS} by setting {\tt model=SCRt}, returns results similar to those from the basic
model \ref{covariates.tab.SCRt}, but now we can examine the variation in detection across time.
Detection seems to
increase for the first few time periods before stabilizing around $0.14$, 
dropping off again at the end of the study.
The differences in detection from the first time periods to the others might
actually be due to something like a behavioral response (see below) or possibly seasonal
differences in the efficiency of the sampling technique.
Researchers have found that hair snares are more effective at different
times of the year (even within season) due to shedding \citep{wegan_etal:inpress}.
In this particular example, our density estimates are
similar to the base model, likely because the differences in detection between occasion were not that large.
In a longer term study or in one with greater variation in the detection probability,
the implication of such differences might have a bigger impact on the estimates of density
and $\sigma$.

\begin{table}[ht]
\centering
\caption{Posterior summaries of parameter estimates from a SCR model with time-dependent baseline detection for the Ft. Drum black bear data set.}
\begin{tabular}
\hline
Parameter & Mean & SD & 2.5 & 97.5 \\
\hline
$D$           &    0.17     &  0.02    & 0.13 & 0.21 \\
$N$           &   509.24 &  66.13  & 381  & 632  \\
$p_0 (t=1)$  &    0.06     & 0.02     & 0.03  & 0.10  \\
$p_0 (t=2)$  &    0.05  & 0.02  &      0.02 & 0.09  \\
$p_0 (t=3)$  &    0.15 &  0.03  &     0.09 & 0.22  \\
$p_0 (t=4)$  &    0.14 &  0.03  &     0.09 & 0.21  \\
$p_0 (t=5)$  &    0.15 &  0.03  &    0.09 &  0.22  \\
$p_0 (t=6)$  &    0.12 &  0.03  &    0.07 & 0.19  \\
$p_0 (t=7)$  &    0.15 &  0.03  &    0.09 & 0.22  \\
$p_0 (t=8)$  &    0.08 &  0.02  &    0.04 & 0.13  \\
$\psi$  &   0.78 &  0.10  &  0.58 & 0.97  \\
$\sigma$ & 1.96 &  0.12  &   1.73 & 2.22  \\
\end{tabular}
\label{covariates.tab.SCRt}
\end{table}

Fitting an occasion specific time effects model might not be the most 
appropriate for all scenarios (and could require the estimation of many parameters
if we had many sampling occasions, take the wolverine example from Chapt. \ref{XXXX} 
where there were 165 sampling
occasions).  In some cases, such as the wolverine study, 
we might sample over longer time frames and
expect the date to have a strong impact on detection for a variety of reasons.  For
example, if we have camera traps running for an entire year and we
expect mating behavior or denning behavior to change the behavioral
patterns of individuals. Instead of fitting a model with $K$ baseline
detection probabilities, we can include date as a linear (or
quadratic, \ldots) effect. An example can be found in
\citet{kery_etal:2011} who incorporated a day of year covariate, both
as a linear and a quadratic effect, into their SCR model of European
wildcats; the data had been collected over a year long period and cat
behavior was expected to vary seasonally thus influencing the
detection probabilities.  In these cases, we would specifically
incorporate day of year (Date) as a continuous covariate as:
\[
\mbox{logit}(p0_{ijk}) = \alpha_0 + \alpha_2*\mbox{\tt Date}_{k}
p_{ijk} = p0_{ijk} \exp(- \alpha_1*||{\bf s}_{i}-{\bf x}_{j}||^2)
\]
or a quadratic effect of day-of-year:
\[
\mbox{logit}(p0_{ijk}) = \alpha_0 + \alpha_2*\mbox{\tt Date}_{k}
 + \alpha_3*\mbox{\tt Date}_{k}^{2}
p_{ijk} = p0_{ijk} \exp(- \alpha_1*||{\bf s}_{i}-{\bf x}_{j}||^2)

\]
where the variable $\mbox{\tt Date}$ is an integer coding of
day-of-year, indexed to some arbitrary start point in time.

XXXX Did we put in a time trend model to the function?  I can write one
if we want to XXXXXXX


\subsection{Trap-specific covariates}

In some studies it makes sense to model encounter probability as a
function of local or trap-specific covariates. These can be one of two
types: genuine trap covariates that describe the trap or encounter site,
such as whether a trap is baited or not, or how many traps were set at a sampling location,
or what kind of bait was used, etc.. Or they can be local covariates that
describe the likelihood that an animal would use the habitat in the
vicinity of the trap.
We imagine that these covariates, of either type, should affect
baseline encounter probability.
%There are a variety reasons that traps may have a different baseline
%detection probability including if the trap is baited or not, if trap
%type varies (e.g., different camera models are used in a camera
%trapping study), or because of the habitat type (e.g., if the trap is
%located on a road/trail).  
For example, \citet{sollmann_etal:2011}
found a large difference in the detection probability of jaguars due to traps
being located on roads, which the animals were using to travel along, as
opposed to traps placed off of roads.  In this case, the trap
type is a binary variable - on/off road,
(another binary variable could be baited/non-baited).  We can write this such that:
\[
\mbox{logit}(p0_{j}) = \alpha_{0,type_j} 
p_{ijk} = p0_{j} \exp(- \alpha_1*||{\bf s}_{i}-{\bf x}_{j}||^2).
\]
Here, we use an indicator variable, ``type'', that will be a numeric
value for the trap-specific covariate.  Thus for our example of on/off
road, we would have $type_j = 1$ if trap $j$ is on a road and $type_j =
2$ otherwise, and we would estimate two separate $alpha_{0}$ - one for
on-road and one for off-road cameras.  This general set up also allows
for more than 2 categories, say if 3 or 4 different camera models were
used, in which case we would estimate as many $alpha_{0}$ as we have
categories in ``type''. Note that instead of modeling several
intercepts, we could also specify the model in terms of effects of
trap type; in this case, for the 2-category example above, we would
specify our ``type'' vector as $type_j = 0$ if trap j is on a road and
$type_j = 1$ otherwise, and write our model such that
\[
\mbox{logit}(p0_{ijk}) = \alpha_{0} + \alpha_{2} \mbox{\tt type}_{j}
\]
Now, $\alpha_{0}$ is the baseline detection probability (on the logit
scale) for traps with $type_j = 0$ and $\alpha_{2}$ is the effect on
baseline detection of a trap being of type = 1. While these models
yield identical results, sometimes one parameterization might work
better than the other in {\bf WinBUGS} or {\bf JAGS}
\citep{kery:2010}.
 
XXX ANDY - Can you clarify the second part of this paragraph??
If the trap-specific covariate describes local conditions, e.g., of
the habitat, this suggests a more fundamental role of encounter
probability as being related to space usage or resource
selection. Indeed, we show in Chapt. \ref{chapt.rsf} how, explicit
formulation of classical resource selection models yield precisely an
additive model of the XXXWHAT FORM?XXX form here, where the covariate
$z({\bf x})$ is the same as that contained in the resource selection
model. XXX ISNT Z OUR INDICATOR OF PART OF THE POPULATION??XXX


\subsection{Behavior or Trap Response by Individual}

One of the most basic of encounter models is that which accommodates a
change in encounter probability as a result of initial encounter.
This is colloquially referred to as ``trap happiness'' or ``trap shyness'', or in other words, a
bahvioral response of individuals to being captured. If a trap is baited
with a food source, an individual might come back for
more. On the other hand, if being captured is traumatic then an
individual might learn to avoid traps. Both of these types of
responses can occur in most species depending on the type of encounter
mechanisms being employed. Moreover, behavioral response can be either
global \citep{gardner_etal:2010} or local \citep{royle_etal:2009jwm}.
The local response is a trap-specific response 
%which likely makes more sense in most spatial situations. 
while a global response suggests that
initial capture provides a net increase or decrease in subsequent probabilities of capture (across all
traps). A behavioral response must not be enduring (i.e., persist for the entire study after the individual has been captured/observed for the first time) but can also be ephemeral, if, for example, an animal only avoids a trap on the occasion immediately after it was captured \citep{royle:2008B, yang_chao:2005}. While we will focus the examples in this chapter on enduring behavioral effects, extending such a model to the case of an ephemeral response should not pose any difficulties. 

To describe these behavioral models we need to create a binary matrix that indicates
if an individual has been captured previously.  For the global
behavioral response, define the $n \times K$ matrix,
${\bf C}$ where $C_{ik} =1$
if individual $i$ was captured at least once prior to session
$k$, otherwise $C_{ik} = 0$.
\[
logit(p0_{ik}) = \alpha_{0} + \alpha_2*C_{ik}
p_{ijk} = p0_{ik} \exp(- \alpha_1*||{\bf s}_{i}-{\bf x}_{j}||^2)
\]
For the local behavioral response, which is trap specific, we create
an array, $C_{ijk}$, that indicates if an individual $i$ has been
previously captured in trap $j$ at time $k$.  We then include this in
the model in the exact same form as above (with the sole difference that both $C$ and $p$ 
are now also indexed by $k$):
\[
logit(p0_{ijk}) = \alpha_{0} + \alpha_2*C_{i,j,k}
p_{ijk} = p0_{ijk} \exp(- \alpha_1*||{\bf s}_{i}-{\bf x}_{j}||^2)
\]

Since the behavioral response is occasion specific, to implement either the local or global response model in \jags~, we will have to use the 3-d array of the
capture histories ($\mbox{\tt nind} \times \mbox{\tt ntraps} \times
\mbox{\tt nreps}$) as we did for the time-varying encounter probability
model above. The code must loop over each sampling occasion, but otherwise, the model
varies only a little from the basic SCR model show in panel \ref{covariates.panel.basicSCR}.  
Here is the specification of the the occasion specific (k) loop:
{\small
\begin{verbatim}
for(k in 1:K){
logit(p0[i,j,k])<- alpha0 + alpha2*C[i,j,k]
y[i,j,k] ~ dbin(p[i,j,k],1)
p[i,j,k]<- z[i]*p0[i,j,k]*exp(- alpha1*d[i,j]*d[i,j])
}

Despite the minor changes to the code, this model can require quite a bit of time and computational 
effort to carry out the local response model.  

XXXX NEED TO FILL IN RESULTS FOR THE BLACK BEAR ANALYSIS HERE - RUNNING XXXXX


\subsection{Individual Covariates}

Individual covariates are those which are measured (or measurable) on
individuals, so we get to observed them only for the captured
individuals. Sex is a simple example of an individual covariate. 
% It can be the
% case that it is missing for some captured individuals because
% frequently, in practice, we only imperfectly determine gender of many
% species.  
The sex of an individual can influence many aspects of its ecology and behavior, 
including, for example its home range size. This is
common in studies of carnivores where females often have smaller home ranges than
males \citep{gardner_etal:2010jwm, sollmann_etal:2011}. Additionally, we may find differences in
the baseline detection between males and females because females may move around less frequently, 
or possibly because they are less likely to use landscape structures that researchers may target 
with sampling devices in order to increase sample size, 
such as roads \citep[e.g.][]{salom-perez_etal:2007}. 


Thus, we can imagine that sex may impact both the baseline encounter
probability $\alpha_{0}$ and the typical home range
size, so that
$\alpha_{1}$ might be sex-specific also.  The fully sex-specific model is:
\[
logit(p0_{i}) = \alpha_{0,sex_{i}}
p_{ijk} = p0_{i} \exp(- \alpha_{1, sex_{i}}*||{\bf s}_{i}-{\bf x}_{j}||^2)
\]

where $sex_{i}$ is a binary vector indicating the sex of
each individual (1 = male, 2 = female).  While we might know the sex of all
individuals observed in the study, we will never know the
sex of individuals that are not observed,
resulting in missing values \citep{gardner_etal:2010jwm} XXX check tag XXXX.
It is also possible that we may not be able to determine the sex of
individuals that are observed during the study. For example photographic
captures do not necessary result in pictures that allow the sex to be absolutely
determined, thus sometimes resulting in missing values of this covariate for animals
captured in the study.   We deal with this slightly differently based on the framework
that we select (Bayesian or likelihood).  Here we demonstrate the Bayesian implementation 
and we discuss the likelihood approach using \secr~ in detail
below in sec. \ref{covariates.secr.sex}.


Specifying a fully sex specific model for {\bf JAGS} is similar to the time-specific model shown above, 
we need to use an index or dummy variable to let $\alpha_0$ and/or $\alpha_1$ be estimated 
separately for males and females. The main difference arise from the fact that we do not observe 
sex for the augmented individuals, which means, we have missing observations of the covariate for those 
augmented individuals. As a result, sex has to be estimated for those unobserved individuals as part 
of the model. In Bayesian inference, missing data are essentially treated like parameters, which means that we need to specify a prior distribution for the missing data. With only two possible outcomes, sex can be regarded as a Bernoulli random variable where $[sex = male] \sim Bernoulli(\pi)$ (we could equally well specify it as $[sex = female] \sim Bernoulli(\pi)$). The parameter $\pi$ can be seen as the sex ratio of the population and since it is something we want to estimate (a hyperparameter of the model), we again have to specify a prior for $\pi$. To express our lack of knowledge about the parameter and account for the fact that being a probability, it can only take values between 0 and 1, we choose $\pi \sim Uniform(0,1)$.  The model specification in panel \ref{covariates.panel.SCRsex} demonstrates how to 
incorporate a partially observed covariate, here, specifically sex.  

In both {\bf JAGS} or {\bf BUGS} missing data are specified as {\tt NA}. In terms of setting up the data that means we need to create a vector of  
length $M$ with the first $n$ elements being 0 if individual $i$ is a female, or 1 if $i$ is a male (for the Ft. Drum black bear data the function {\tt bear.JAGS} extracts this information automatically from the {\tt beardata} object), and the subsequent $M-n$ elements being {\tt NA}.  It is generally a 
good idea to provide starting values for the missing data, but we cannot provide starting values for observed data; in this case where one vector 
(or other object) contains both observed and missing data, initial values for the observed data have to be specified as {\tt NA}. 
The code snippet below shows you how to set up the data including the{\tt sex} vector and the initial values function, 
as well as the {\jags} model specifications (the remainder of the code is identical to what we've shown before).


{\small
\begin{verbatim}
sex<-beardata$sex  #we read in the sex data for captured individual
SEX<-c(sex-1, rep(NA, nz)) # in the data sex is coded as 1 and 2, so we recode it to 0 and 1
data<-list(y=y,SEX=SEX, M=M,K=K, J=ntraps, Xl=Xl, Yl=Yl, Xu=Xu, Yu=Yu, X=X, area=areaX)
params<-c('psi','p0','N', 'D', 'sigma', 'pi')
inits =  function() {list(z=c(rep(1,nind), rbinom(nz,1,0.5)),psi=runif(1), s=cbind(runif(M, Xl, Xu), runif(M, Yl, Yu)), 
		SEX=c(rep(NA, nind), rbinom(nz,1,0.5)),pi=runif(1), sigma=runif(2,2,3),alpha0=runif(2)) }

cat("
model {

psi~dunif(0,1)
pi~dunif(0,1)

for(t in 1:2){
alpha0[t]~dnorm(0,.1)
logit(p0[t])<- alpha0[t]
alpha1[t]<-1/(2*sigma[t]*sigma[t])
sigma[t]~dunif(0, 15)
}

for(i in 1:M){
 z[i] ~ dbern(psi)
 SEX[i]~dbern(pi)
SEX2[i]<-SEX[i] + 1
 s[i,1]~dunif(Xl,Xu)
 s[i,2]~dunif(Yl,Yu)

for(j in 1:J){
d[i,j]<- pow(pow(s[i,1]-X[j,1],2) + pow(s[i,2]-X[j,2],2),0.5)
y[i,j] ~ dbin(p[i,j],K)
p[i,j]<- z[i]*p0[SEX2[i]]*exp(-alpha1[SEX2[i]]*d[i,j]*d[i,j])
}
}
N<-sum(z[])
D<-N/area
}
",file = "SCRsex.txt")
\end{verbatim}
}


Our estimate of density under the fully sex specific model is still very similar to the previous models (Table \ref{covariates.tab.SCRsex}), and while the baseline detection was not very different
between males and females, we can see that they had very different $\sigma$ estimates (note that the BCIs do not
overlap).  As usual, you can reproduce this analysis by calling the function {\
tt bear.JAGS} and set {\tt model='SCRsex'}. 

\begin{table}[ht]
\centering
\caption{Posterior summaries of parameter estimates from sex specific SCR models for the Ft. Drum black bear data set.}
\begin{tabular}
\hline
Parameter & Mean & SD & 2.5 & 97.5 \\
\hline
$D$  &     0.168 & 0.022 & 0.12 & 0.21  \\
$N$   &   509.982 & 66.355 & 376 & 631 \\
$p_0 (female)$ & 0.136 & 0.025 & 0.09 & 0.19 \\
$p_0 (male)$ & 0.092 & 0.017 & 0.06 & 0.13 \\
$\pi$ &  0.310 & 0.068 & 0.19 & 0.45 \\
$\psi$  & 0.784 & 0.103 & 0.58 & 0.97 \\
$\sigma_{female}$ & 1.542 &  0.132 & 1.31 & 1.83 \\
$\sigma_{male}$ & 2.682 & 0.389 & 2.09 & 3.62 \\
\end{tabular}
\label{covariates.tab.SCRsex}
\end{table}



\section{Individual heterogeneity.}

Finally, we consider individual heterogeneity, 
a covariate that is completely latent. 
Capture-recapture models with individual heterogeneity in detection
probability, so-called model $M_{h}$, have a long history in classical
capture recapture models and they have
special relevance to SCR as we noted in sec. \ref{closed.sec.modelmh}.  We
note that their use has been called into question by \citet{link:2003}
who noted that $N$ may not be identifiable across arbitrary classes of
mixture models.  One possible way to get around this problem is to
identify explicit sources of heterogeneity in detection probability
and model those directly. For example, we can do this by using
individual covariate models (e.g., sec. \ref{closed.sec.indcov}). Of
course, spatial capture-recapture models are such a class of models
which seek to explain heterogeneity in detection by describing the
underlying mechanism explicitly. In particular, that mechanism is the
juxtaposition of individuals with traps and the resulting
heterogeneity that is induced by heterogeneity in exposure to
trapping.

While the advent of SCR models may appear to have rendered the use of
classical model $M_h$ obsolete (because the heterogeneity is being
accounted for explicitly) we may still wish to consider
heterogeneity models for other biological reasons.
It is reasonable
to expect in real populations that there exists
heterogeneity in home range size and so we think that $\alpha_{1}$
could exhibit heterogeneity among individuals. 

Here we develop and evaluate a class of spatial capture-recapture
models which allow for individual heterogeneity in encounter
probability.  In particular, one class of models we propose explicitly
admits individual heterogeneity in home range {\it size}. In addition,
we consider a standard representation for heterogeneity in which an
additive individual-specific random effect is included in the linear
predictor for baseline encounter probability.  

\subsection{Models of Heterogeneity}

An obvious model extends the SCR model by including an additive individual effect, analogous to classical ``Model $M_{h}$''. We'll call this model ``SCR+Mh'':
\[
\mbox{logit}(p0_{i}) = \alpha_0 + \eta_{i}
p_{ijk} = p0_{i} \exp(- \alpha_1*||{\bf s}_{i}-{\bf x}_{j}||^2)
\]
where $\eta_{i}$ is an individual random effect having distribution
$g(\eta|\theta)$.  A popular class of models arises by assuming
$\eta_{i} \sim Normal(0,\tau^{2})$ (\citet{coull_agresti:1999};
\citet{dorazio_royle:2003}; etc.).  Many other random effects
distributions are possible. \citet{norris_pollock:1996} propose a
finite mixture of point supports which has been addressed considerably
in the literature \citep{pledger:2003, dorazio_royle:2003, link:2003}. We demonstrate how to fit 
finite-mixture models using \secr~ in sec. \ref{covariates.sec.secrH2}.  However, 
heterogeneity seems naturally continuous unless one expects the heterogeneity to be 
due to meaningful biological groupings in which case such information would normally 
be collected if possible.  Even so the more likely scenario is that heterogeneity 
is due to a lot of different sources contributing independent components of variation, 
and so the normal model seems sensible in that regard.  

{\bf Heterogeneity Induced by Variation in Home Range Size} -- An alternative heterogeneity model, one that has more of a direct biological motivation and interpretation, describes heterogeneity in home range size among individuals. This is manifest in the scale parameter of the detection function $\sigma^{2}$ or its inverse $\alpha_1 = 1/\sigma^{2}$. We might
thus assume a distribution for either $\sigma^{2}$ or its inverse,
$\alpha_1$.  We thus propose ``Model SCR + Ah'' (Ah for area-induced
heterogeneity).
\[
\mbox{logit}(p0) = \alpha_0 
p_{ijk} = p0 \exp(- \alpha_{1,i}*||{\bf s}_{i}-{\bf x}_{j}||^2)
\]
This model is a model of heterogeneity in home range area. For example
if we assume that $\alpha_{1,i} \sim \mbox{Normal}(\beta_0,\tau^{2})$
with $\beta_{0} = 2$ and $\tau = 0.50$. Then the population
distribution of $\sigma$ in this case is given in Figure
\ref{covariates.fig.one}. The motivating point of this model is that we expect
such variability in natural populations. Thus we suggest this
biologically sensible model of heterogeneity, which fills a
methodological gap in the literature in the sense that SCR models have
all been homogeneous with respect to their explicit treatment of home
range morphology.

\begin{figure}[ht]
%%\centerline{\psfig{figure=fig1.ps,height=4in,width=4in}}
\caption{
Population distribution of $\sigma$ if $(1/\sigma^{2}) \sim \mbox{Normal}(2, 0.50)$.
}
\label{fig.one}
\end{figure}

XXX Andy, do you still think this is necessary? XXX
It is interesting to consider alternative distributions for $\alpha_{1,i}$, here we 
use a Normal prior which might result in a negative value for $\alpha_{1,i}$ which is
not expected but also not entirely nonsensical.  We could also use the the Inverse-Gamma
distribution, but it is not conjugate in the present context and so
there is no compelling reason to do that.  Also important to note is that if A[i] 
is the home range area of individual $i$, and we have related area to $\sigma$ through a specified
function, then we can move back and forth between distributions for
A[i], $\sigma_i$, and $\alpha_{1,i}$.


{\bf Approximation: }
Note that ``SCR + Mh'' might be a good approximation to ``SCR + Ah''.  If we write $\alpha_{1,i} =
\alpha_{1} + \eta_{i}$ then
we can take the expectation over  $\alpha_{i}$ to arrive at
\[
\mbox{logit}(p0) = \alpha_0 
p_{ijk} = p0 \exp(- \alpha_{1}*||{\bf s}_{i}-{\bf x}_{j}||^2) +  \eta_{i}*||{\bf s}_{i}-{\bf x}_{j}||^2)
\]

Which has this additive individual effect that varies also by trap. It might be that approximating
this by SCR+Mh is better than nothing and it could also be viewed as suggesting an 
over-dispersed count model for encounter frequencies.

In panel \ref{covariates.panel.SCRMhjags}, we show how to implement the SCR + Mh model.  


\begin{panel}[htp]
\centering
\rule[0.1in]{\textwidth}{.03in}
{\small
\begin{Verbatim}[commandchars=\\\{\}]
model \{
alpha0 \mytilde dnorm(0,.1)
alpha1<-1/(2*sigma*sigma)
sigma \mytilde dunif(0, 15)
psi \mytilde dunif(0,1)
tau_p \mytilde dgamma(.001,.001)

for(i in 1:M)\{
  eta[i]\mytilde dnorm(0, tau_p)
    z[i] \mytilde dbern(psi)
  s[i,1] \mytilde dunif(Xl,Xu)
  s[i,2] \mytilde dunif(Yl,Yu)
   for(j in 1:J)\{
   d[i,j]<- pow(pow(s[i,1]-X[j,1],2) + pow(s[i,2]-X[j,2],2),0.5)
   y[i,j] \mytilde dbin(p[i,j],K)
 logit(p0[i,j])<-alpha0 + eta[i]
    p.eff[i,j]<- z[i]*p0[i,j]*exp(- alpha1*d[i,j]*d[i,j])
\}
\}
N<-sum(z[])
D<-N/area
\}

\end{Verbatim}
}

\rule[-0.1in]{\textwidth}{.03in}
\caption{
\jags~ model specification for the SCR + Mh model with half-normal distance
function.}
\label{covariates.panel.SCRMhjags}
\end{panel}







\section{Likelihood analysis in \secr}
\label{likelihood.secr}

Previously in the book, we have introduced the \R package \secr~ and described in 
detail the likelihood based approach taken in the package(see sec. \ref{mle.sec.secrguts}).
Here we cover how to implement a series of covariate models in \secr~ and show briefly an 
example of model selection using AIC.
In general,
 \secr~ allows the user to simulate data and fit a
suite of models with various detection functions and covariate
responses.  As we saw in Chapt. \ref{chapt.mle}, \secr~ uses the standard \R model
specification framework, defining the dependent and independent
variable relationship using tildes
(e.g., \Verb+y ~ x+).
Thus, in \secr~ we
might have \verb+g0 ~ behavior+ or \verb+sigma ~ time+;
when left unspecified or set
to 1 (e.g., \verb+g0 ~ 1+), this will default to a model with no
covariates
(i.e., constant parameter values).
Additionally, \secr~ allows us to specify covariates 
on density (see Chapt. \ref{chapt.ipp}),
which are set for example as \verb+D ~ habitat+.

To demonstrate
a suite of models with various types of covariates using \secr, we
continue using the data collected on black bears in Ft. Drum.
% , NY, USA
% which we previously analyzed in Chapt. \ref{chapt.closed}. XXXX DONT WE USE THIS DATA SET IN THE PRECEDING SECTION ON BUGS TOO? XXXX  
We include
in the \scrbook package a function called {\tt secr.bear} that will 
format the data (see Chapt. \ref{chapt.mle} for the \secr~ data format) and then fit and compare 8 models (details shown in panel 
\ref{covariates.panel.secrfn} ).  We have described all of these models in the previous sections, so we only briefly comment here on how to fit
certain models in \secr~ and compare them using AIC, and give a few helpful notes.  
% Below, 
% we have a few subsections on the sex-specific effects, heterogeneity,
% and AIC which describe in more detail how we fit these models within
% \secr.

\subsection{Notes for fitting standard models}

In the \secr~ package, the detection functions are specified
by changing the ``\mbox{\tt detectfn}'' option (an integer code)
within the \mbox{\tt secr.fit} command.  Table
\ref{covariates.tab.detmodels} shows the possible detection functions
that \mbox{\tt secr} will fit; the default is that based on the kernel
of a bivariate normal
(``half-normal'') model and the
(negative) exponential is \mbox{\tt detectfn = 2}.  See model 2 in 
panel \ref{covariates.panel.secrfn} for how to fit the exponential 
model to the bear dataset. 

The \mbox{\tt secr} package easily fits a range of SCR equivalents of standard capture-recapture models.
The package has pre-defined versions of the classic
$M_{t}$ where each
occasion has its own detection probability, as well as a linear
trend in baseline detection over occasions (in a spatial modeling framework $\sigma$ could also be an occasion specific parameter, but having detection change with time seems like the more common case). For the classical
time-effects type of model with $K$ distinct parameters \secr~ uses 't' to denote
this in the command call (see model 3 in 
panel \ref{covariates.panel.secrfn}); whereas, for a linear
trend over occasions \secr~ uses 'T'.

It is similarly easy to incorporate a global trap response model (model $M_{B}$), 
or a trap specific behavioral response (model $M_{b}$, as in
\citet{royle_etal:2009}). Despite our notion here, \secr~ uses ``b" for the global response model and
 ``bk" for the local trap response model
(see models 4 and 5 in 
panel \ref{covariates.panel.secrfn}; note that to fit the trap specific behvioral response model you need version 2.3.1 or newer of \secr).

\begin{panel}[htp]
\centering
\rule[0.1in]{\textwidth}{.03in}
{\small
\begin{Verbatim}[commandchars=\\\{\}]
1. null model with a bivariate normal detection function
bear_0=secr.fit (bear.cap, model=list(D \mytilde 1, g0 \mytilde 1, sigma \mytilde 1),buffer = 20000)

2. null model with an exponential detection function
bear_0exp=secr.fit (bear.cap, model=list(D \mytilde 1, g0 \mytilde 1, sigma \mytilde 1),buffer = 20000, detectfn=2)

3. model with fixed time effects
bear_t=secr.fit (bear.cap, model=list(D \mytilde 1, g0 \mytilde t, sigma \mytilde 1),buffer = 20000)

4. global behavioral model
bear_B=secr.fit (bear.cap, model=list(D \mytilde 1, g0 \mytilde b, sigma \mytilde 1),buffer = 20000)

5. trap specific behavioral response
bear_b=secr.fit (bear.cap, model=list(D \mytilde 1, g0 \mytilde bk, sigma \mytilde 1),buffer = 20000)

6. global behavior model with fixed time effects
bear_bt=secr.fit (bear.cap, model=list(D \mytilde 1, g0 \mytilde b+t, sigma \mytilde 1),buffer = 20000)

7. sex-specific model
bear_sex=secr.fit (bear.cap, model=list(D \mytilde session, g0 \mytilde session, sigma \mytilde session),buffer = 20000)

8. heterogeniety model
bear_h2=secr.fit(bear.cap, model=list(D \mytilde 1, g0 \mytilde h2, sigma \mytilde h2),buffer = 20000)
\end{Verbatim}
}

\rule[-0.1in]{\textwidth}{.03in}
\caption{
Models called from secr.bear function. }
\label{covariates.panel.secrfn}
\end{panel}




\subsection{Sex Effects}
\label{covariates.secr.sex}

Specifying models with sex specific parameters is not as straight forward in \secr~ as the other models. 
Incorporating sex effects into models with \secr~ can be done a few
different ways.  In all cases, individuals that
are of unknown sex must be removed from the dataset (recall that in a Bayesian framework we can keep these individuals in the data set and the model will estimate their sex based on the underlying distribution, e.g. {\mbox Bernoulli($\pi$)} and $\pi$ = probability of being a male).
The most common way to include
sex is to code it into ``session'', providing two sessions that
represent males and females (see model 7 in 
panel \ref{covariates.panel.secrfn}).  This method provides two separate density
estimates, which can then be combined into a total density.  

There are at least two other ways that one could
specify the model (M. Efford, pers. comm).  One way is that we could
list sex as a categorical individual covariate and then maximize the
conditional likelihood XXX{\bf Is this NOT what secr does in the above
  approach? explain...} XXXXX.  The second way is that we could treat males and 
females as groups (in \secr~ denoted by ``g''), specify the
model as $model = list(D~g, g0~g, \sigma~g)$ and list \mbox{\tt groups = 'sex'}
where we have specified sex as a 2-level individual covariate XXX How
is this different then the first way? XXXXX.  


\subsection{Individual heterogeneity}
\label{covariates.sec.secrH2}

To incorporate heterogeniety, \secr~ fits the
``finite mixture'' models \citep{norris_pollock:1996,
  pledger:2000}. These are expensive in terms of parameters and not
very typically used outside of their use in \secr~ and a few other
specialized software packages that do capture-recapture
things. Historically they were adopted because they are easy to
compute with. More recently, continuous mixtures have been adopted in
many settings because they are natural extensions of standard GLMs. We
don't favor the use of finite mixtures. Despite this we give some
examples here using \secr.

Using \secr,  individual heterogeneity can be incorporated
into the detection parameters as either a 2-part or 3-part finite
mixture model \citep{norris_pollock:1996,
  pledger:2000} with the use of ``\mbox{\tt h2}'' or
``\mbox{\tt h3}'', respectively, in the
model call.   The 2-part mixture is shown in model 8 of panel
\ref{covariates.panel.secrfn} and the 3-part mixture can easily be fit by
substituting \mbox{\tt h3} for \mbox{\tt h2}.  
It is important to note that this specification of
individual heterogeneity is different from that which we demonstrated in Sec. 
XX insert ref XX above.  Here, a finite mixture model is used, which effectively
puts the individuals into one of two (or three) latent classes and
then assigns each class a distribution for the specified detection
parameter.   This same model can be fit in \jags or \bugs, but we only showed
the SCR + Mh continuous mixture in the version above.



\subsection{Model selection in \secr~ with AIC}

One practical advantage to using the \secr~ package, or likelihood
inference in general, is the convenience of automatic model selection
using AIC \citep{burnham_anderson:1998}.
Running the function {\tt secr.bear}, which calls all of the models we
have described, will return, in addition to all model results, 
an AIC table with all of the summarized results including the AIC values,
delta AIC, and model weights (see Table \ref{covariates.tab.secrAIC} or reproduce results in {\tt R} using{\tt out<- secr.bear(); out$AIC.tab}). 

It is important to note here that because of the way we specified the sex-specific model
as  ``sessions'', we 
coded the dataset as such when first loading it to \secr~.  
Thus, all model results will list separate parameter estimates for each session, even the null model with no covariates (but estimates are the same for both ``sessions'' in all but the sex specific model). This
coding of session allows us to compare all the models via AIC XXXX I THINK THIS NEEDS TO BE EXPLAINED A BIT BETTER XXXX.  There
is an issue with the AIC values for models with and without groups
that has not been resolved so the reader should be cautious when using
this latter option (M. Efford, pers. comm).


\begin{table}[ht]
\centering
\caption{Log-likelihood, AIC, deltaAIC and AIC weight for several models run in secr for the Fort Drum black bear data set.}
\begin{tabular}
\hline
model     &  logLik   &   AIC    &   AICc   & dAICc  & AICwt \\
bear_b    & -641.7215 & 1291.443 & 1292.395 & 0.000  &  1 \\
bear_h2   & -653.8382 & 1319.676 & 1321.776 & 29.381 &  0 \\
bear_0exp & -663.9152 & 1333.830 & 1334.389 & 41.994 &  0 \\
bear_B    & -677.6175 & 1363.235 & 1364.187 & 71.792 &  0 \\
bear_bt   & -668.3044 & 1358.609 & 1366.152 & 73.757 &  0 \\
bear_sex  & -677.7151 & 1367.430 & 1369.530 & 77.135 &  0 \\
bear_t    & -674.4134 & 1368.827 & 1374.938 & 82.543 &  0 \\
bear_0    & -686.2455 & 1378.491 & 1379.049 & 86.654 &  0 \\
\end{tabular}
\label{covariates.tab.secrAIC}
\end{table}


The results from this AIC analysis are pretty easy to interpret; the model
with a local trap response on detection, ``bk'', has a model weight of 1 and thus, according to AIC, 100\% support.
The 2-part finite mixture model for $g0$ and $\sigma$ has the second lowest
AIC, but considering the large dAICc compared to the local trap response model we would probably not consider it any further.  
% Using the AIC provides a fast and convenient mechanism for
% conducting model comparisons. 


\section{Summary and Outlook}

There are endless covariates and detection functions that can be defined
and our goal in this chapter was to introduce some basic concepts for how
to construct and implement such models.  Essentially, SCR's are GLMMS and
therefore it is easy to incorporate covariates into the model for encounter
probability.

The move to spatially explicit models in capture-recapture studies has not rendered the
basic CR models obsolete, instead, we continue to use those models and incorporate
space into them. Now, we can include not only these standard CR models, 
e.g., $M_0$, $M_t$, $M_b$, and $M_h$, but also new models that allow for trap-specific
information such as "baited/not-baited" or "on/off road".  In Chapt. XXX, we continue to
incorporate new methods for explaining variation in encounter probability (and density) 
by also considering spatial covariates.  

Researchers are often concerned with describing the factors or covariates that influence
variation in detection or encounter probability, particularly as this can directly influence
other parameters in the model such as density.  These covariates have various levels - specific
to individual, trap, sampling occasion and can be fully or partially observed as well as 
completely latent.  In SCR models, these more complex covariate models are fairly easy to fit, 
though one should take caution not to over parameterize the models particularly when a study
has produced a limited dataset.

\begin{comment}

Different detection models: We can make up detection models {\it all fucking
day}, to no end, with no point, and with no biological justification for
any single model. To us this would be bad practice and so we think it is
perfectly fine to pick a model ahead of time and stick with it.

we note that underlying these different models is basically something
to do with the 2nd moment structure of some correlated spatial process...
i.e., correlation functions (Higdon et al. 1998; etc...) and , insofar
as chooing detection functions is like choosing a correlation function,
it probably wont have much affect on inferences.
\end{comment}