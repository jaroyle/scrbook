\chapter{
Preface
}
\markboth{Preface}{}
\label{chapt.preface}


Stuff on hierarchical models and how great those are.

Bayesian and likelihood analysis.....

\section{R programming language}




\section{BUGS Language}




\section{R package scrbook}


\section{Organization of This Book}

In the following chapters we develop a comprehensive synthesis and extension of 
spatial capture-recapture models.
Roughly the first third of the book is introductory material -- 
In Chapt. \ref{chapt.glms} we provide the basic analysis tools to understand and
analyze SCR models - namely generalized linear models (GLMs) with random effects, and their
analysis in {\bf R} and {\bf WinBUGS}.  Because SCR models represent extensions of
basic closed population models, we cover ordinary closed population
models in Chapt. \ref{chapt.closed} wherein, along with Chapts. \ref{chapt.scr0} and \ref{chapt.poisson-mn}
\footnote{might ought to put Modeling Encounter Probability 
as chapter 5 instead}, provides the basic introduction 
to capture-recapture models and their spatial extension.
We will see that
SCR models are a 
conceptual and technical intermediates between the class of models referred to as 
model $M_h$, and so-called
individual covariate models.  
We develop technical tools for likelihood (Chapt. \ref{chapt.mle})
and Bayesian analysis (Chapt. \ref{chapt.mcmc}).
The middle part of the book expands set of models that we can deal with to include alternative
observation models related to the type of encounter device (Chapt. \ref{chapt.poisson-mn}), models for encounter probability
(Chapt. \ref{chapt.covariates}), [should include search-encounter models right after Poisson-mn type models?] and provides basic tools for model fit and selection (Chapt. \ref{chapt.gof}).
[should include the design chapter right here].
Finally in the last third of the book we address more advanced stuff including modeling
space usage in the encounter process (Chapt. \ref{chapt.ecoldist}), modeling state-space covariates, covariates
that affect density, (Chapt. \ref{chapt.state-space}), open population models (Chapt. \ref{chapt.open}),
models that include unmarked individuals either entirely (Chapt. \ref{chapt.scr-unmarked}) 
or partially marked samples (Chapt. \ref{chapt.partialID}).

In Chapter XXXX We cover a mish-mash of ideas: using telemetry data, multiple encounter methods, alternative
point-process models, and other topics that are useful but that are not fully developed or that we don't have
room for in this book. 





