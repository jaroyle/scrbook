\documentclass{book}

%\usepackage{elsst-book}
\usepackage{float}
\usepackage{amsmath}
\usepackage{amsfonts}
\usepackage{graphicx}
\usepackage{lineno}
\usepackage{natbib}
\usepackage{hyperref}
\usepackage{verbatim}
\usepackage{soul}
\usepackage{color}

\bibliographystyle{asa}

\usepackage{makeidx,bm,amsmath,url}
\makeindex

\floatstyle{plain}
\floatname{panel}{Panel}
\newfloat{algorithm}{h}{txt}[chapter]
\newfloat{panel}{h}{txt}[chapter]


\newcommand{\R}{\textbf{R}}
\newcommand{\bugs}{\textbf{BUGS}}
\newcommand{\jags}{\textbf{JAGS}}
\newcommand{\secr}{\mbox{\tt secr}}
\newcommand{\scrbook}{\mbox{\tt scrbook}}


\linenumbers

\begin{document}

\title{ Spatial Capture-Recapture Models }
%\subtitle{ A hierarchical approach }
\author{The Four Horsemen (and women) }

%\affiliation{First Author Short Address\\ Second Author Short Address}
%\address{
%USGS Patuxent Wildlife Research Center \\
%North Carolina State University
%}

\maketitle

\newpage

\setcounter{tocdepth}{2}
\tableofcontents

\chapter{Introduction}
\label{chapt.intro}

\chapter{
Bayesian Analysis of GL(M)Ms Using R/WinBUGS
}
\markboth{Bayesian Analysis of GLMMS}{}
\label{chapt.glms}

\vspace{.3in}

%%%% STUFF TO DO
%%% 1. Prior lack of invariance to transformation stuff: Reference and Figure
%%% 2. Full conditional example from ch. 6 copy notation
%%% 3. Check out algorithm environment
%%% 4. reference for sampling from f() with bounded support
%%% 5. need refs on choosing prior disributions
%%% 6. Check Bayesian p-value definition
%%% 7. FIX parameter notation! I have beta0 beta1 , alpha beta, and a,
%%%     b in the same chapter!   Use alpha beta probably?
%%% 8. spell check this document

A major theme of this book is that spatial capture-recapture models
are, for the most part, just generalized linear models (GLMs) wherein
the covariate, distance between trap and home range center, is
partially or fully unobserved  -- and therefore regarded as
a random effect. Such models
are usually referred to as Generalized Linear Mixed Models (GLMMs)
and, therefore, SCR models can be thought of as a specialized type of
GLMM. Naturally then, we should consider analysis of these slightly
simpler models in order to gain some experience and, hopefully,
develop a better understanding of spatial capture-recapture models.

In this chapter, we consider classes of GLM models - Poisson and
binomial (i.e., logistic regression) GLMs - that will prove to be
enormously useful in the analysis of capture-recapture models of all
kinds. Many readers are probably familiar with these models because
they represent probably
the most generally useful models in all of Ecology and, as
such, have received considerable attention in many introductory and
advanced texts. We focus on them here in order to introduce the
readers to the analysis of such models in {\bf R} and {\bf WinBUGS},
which we will
translate directly to the analysis of SCR models in subsequent
chapters.

Bayesian analysis is convenient for analyzing GLMMs because it allows
us to work directly with the conditional model -- i.e., the model that
is conditional on the random effects, using computational methods
known as Markov chain Monte Carlo (MCMC). Learning how to do Bayesian
analysis of GLMs and GLMMs in {\bf WinBUGS} is, in part, the purpose
of this chapter.  While we use {\bf WinBUGS} to do the Bayesian
computations, we organize and summarize our data and execute {\bf
  WinBUGS} from within {\bf R} using the useful package \mbox{\tt
  R2WinBUGS} \citep{sturtz_etal:2005}.  \citet{kery:2010}, and
\citet{kery_schaub:2011} provide excellent introductions to the basics
of Bayesian analysis and GLMs at an accessible level. We don't want to
be too redundant with those books and so we avoid a detailed
treatement of Bayesian methodology - instead just providing a cursory
overview so that we can move on and attack the problems we're most
interested in related to spatial capture-recapture.  In addition,
there are a number of texts that provide general introductions to
Bayesian analysis, MCMC, and their applications in Ecology including
\citet{mccarthy:2007}, \citet{kery:2010}, \citet{link_barker:2009},and
\citet{king_etal:2009}.


While this chapter is about Bayesian analysis of GLMMs, such models
are routinely analyzed using likelihood methods too, as discussed by
\citet{royle_dorazio:2008}, and \citet{kery:2010}. Indeed, likelihood
analysis of such models is the primary focus of many applied
statistics texts, a good one being \citet{zuur_etal:2009}. Later in
this book, we will use likelihood methods to analyze SCR models but,
for now, we concentrate on providing a basic introduction to Bayesian
analysis because that is the approach we will use in a majority of
cases in later chapters.


\section{ Notation}

We will sometimes use conventional ``bracket notation'' \index{bracket
  notation} to refer to
probability distributions. If $y$ is a random variable the $[y]$
indicates its distribution or its probability density/mass function
(pdf, pmf) depending on context. If $x$ is another random variable
then $[y|x]$ is the conditional distribution of $y$ given $x$, and
$[y,x]$ is the joint distribution of $y$ and $x$. To differentiate
specific distributions in some contexts we might label them $g(y)$,
$g(y|\theta)$, $f(x)$, or similar. We will also write $y \sim
\mbox{Normal}(\mu,\sigma^{2})$ to indicate that $y$ ``is distributed as'' a normal
random variable with parameters $\mu$ and $\sigma^{2}$. The expected value
or mean of a random variable is $E[y] = \mu$ ,and $Var[y] = \sigma^{2}$ is
the variance of $y$.  To indicate specific observations we'll use an
index such as ``$i$''. So, $y_{i}$ for $i=1,2,\ldots,n$ indicates
observations for $n$ individuals. Finally, we write $\Pr(y)$ to indicate specific probabilities, i.e., of events ``$y$'' or similar.


To illustrate these concepts and notation, suppose $z$ is a binary
outcome (e.g., species occurrence) and we might assume the model: $z
\sim \mbox{Bern}(p)$ for observations.  Under this model $\Pr(z=1) =
\psi$, which is also the expected value $E[z] = \psi$. The variance is
$Var[z] = \psi*(1-\psi)$ and the probability mass function (pmf) is $[z]
= \psi^{z} (1-\psi)^{1-z}$. Sometimes we write $[z|\psi]$ when it is
important to emphasize the conditional dependence of $z$ on $\psi$. As
another example, suppose $y$ is a random variable denoting whether or
not a species is detected if an occupied site is surveyed. In this
case it might be natural to express the pmf of the observations $y$
{\it conditional} on $z$. That is, $[y|z]$. In this case, $[y|z=1]$ is
the conditional pmf of $y$ given that a site is occupied, and it is
natural to assume that $[y|z=1] = \mbox{Bern}(p)$ where $p$ is the
``detection probability'' - the probability that we detect the
species, given that it is present. The model for the observations $y$
is completely specified once we describe the other conditional pmf
$[y|z=0]$. For this conditional distribution it is sometimes
reasonable to assume $\Pr(y=1|z=0) = 0$ (\citet{mackenzie_etal:2002};
see also \citet{royle_link:2006}). That is, if the species is absent,
the probability of detection is 0. This implies that
$\Pr(y=0|z=0)=1$. To allow for situations in which the true state $z$
is unobserved, we  assume that $[z]$ is Bernoulli with parameter
$\psi$.  In this case, the marginal distribution of $y$ is
\[
 [y] = [y|z=1]Pr(z=1) + [y|z=0]Pr(z=0)
\]
because $[y|z=0]$ is a point mass at $y=0$, by assumption, then
\[
\Pr(y=1) = p \psi
\]
And
\[
\Pr(y=0) = (1-p)*\psi + (1-\psi)
\]


\section{
GLMs and GLMMs}
We have asserted already that SCR models work out most of the time to
be variations of GLMs and GLMMs. Some of you might therefore ask: What
are GLMs and GLMMs, anyhow?   These models are covered extensively in
many very good applied statistics books and we refer the reader
elsewhere for a detailed introduction. We think \citet{kery:2010},
\citet{kery_schaub:2011}, and \citet{zuur_etal:2009} are all
accessible treatments of considerable merit. Here, we'll give the 1
minute
treatment of GLMMs, not trying to be complete but rather only
to preserve a coherent organization to the book.


The generalized linear model (GLM) is an extension of standard linear
models by allowing the response
variable to have some distribution from the exponential family of
distributions (i.e., not just normal). This includes the normal
distribution but also dozens of others such as the Poisson, binomial,
gamma, exponential, and many more. In addition, GLMS allow the
response variable to be related to the predictor variables (i.e.,
covariates) using a
link function, which is usually nonlinear.  Finally, GLMs typically
accommodate a relationship between the mean and variance. The
classical reference for GLMs is \citet{nelder_wedderburn:1972} and
also \citet{mccullagh_nelder:1989}.
The GLM consists of three components:
\begin{itemize}
\item[1.] A probability distribution for the dependent variable $y$,
from a class of probability distributions known as the exponential family.
\item[2.] A ``linear predictor'' $\eta = {\bf X}{\bm \beta}$  .
\item[3.] A link function $g$ that relates $E[y]$ to the linear predictor, $E[y] = \mu = g^{-1}(\eta)$. Therefore $g(E[y]) = \eta$.
\end{itemize}

The dependent variable $y$ is assumed to be an outcome from a
distribution of the exponential family which includes many common
distributions including the normal, gamma, Poisson, binomial, and many
others. The mean of the distribution of $y$ is assumed to depend on predictor variables $x$ according to
\[
 g(E[y]) = {\bf x}'{\bm \beta}
\]
where $E[y]$ is the expected value of $y$, and ${\bf x}'{\bm \beta}$
is termed the {\it linear predictor}, i.e., a linear function of the
predictor variables with unknown parameters ${\bm \beta}$ to be
estimated.  The function $g$ is the link function. In standard GLMs,
the variance of $y$ is a function $V$ of the mean of $y$: $Var(y) =
V(\mu)$ (see below for examples).

A Poisson GLM posits that $y \sim \mbox{Poisson}(\lambda)$ with $E[y]
=\lambda$ and usually the model for the mean is specified using the
{\it log link function} by
\[
log(\lambda_{i}) = \beta_0 + \beta_{1}*x_{i}
\]
The variance function is $\mbox{V}(y_{i}) = \lambda_{i}$.  The
binomial GLM posits that $y_{i} \sim \mbox{Binomial}(K,p)$ where $K$
is the fixed sample size parameter and $E[y_{i}] = K*p_{i}$. Usually
the model for the mean is specified using the {\it logit link
  function} according to
\[
 logit(p_{i}) = \beta_{0} + \beta_{1}*x_{i}
\]
Where $logit(u) = log(u/(1-u))$.  The inverse-logit function, $g^{-1}$ ,
is a function we will refer to as ``expit'', so that $expit(u) =
exp(u)/(1+exp(u))$.

A GLMM is the extension of GLMs to accommodate ``random
effects''. Often this involves adding a normal random effect to the
linear predictor, and so a simple example is:
\[
 \log(\lambda_{i}) = \alpha_{i} + \beta_{1}*x_{i}
\]
where
\[
 \alpha_{i} \sim \mbox{Normal}(\mu,\sigma^{2})
\]
%Many other probability distributions and formulations of the linear
%predictor might be considered.  It is not widely appreicated that
%the link function and
%distribution of the random effect interact directly to affect the
%implied probability distribution of the linear predictor. For the
%Poisson case just considered, $\lambda_{i}$ has a log-normal
%distribution. However, if we set $\lambda_{i} = \alpha_{i}exp(\beta*x_{i})$
%where $\alpha_{i}$ has a Gamma distribution, then $\lambda_{i}$ has
%similarly a gamma distribution with modified scale parameter.  These
%different model assumptions are seldom evaluated formally in practice
%although in many practical situations (in ecology), they imply
%specific things about the ecological process being studied
%(e.g., see \citet{royle_dorazio:2008} section XYZ on occupancy
%logit/cloglog etc..).



\section{Bayesian Analysis}

Bayesian analysis is unfamiliar to many ecological researchers because
older cohorts of ecologists were largely educated in the classical
statistical paradigm of frequentist inference. But advances in
technology and increasing exposure to benefits of Bayesian analysis
are fast making Bayesians out of people or at least making Bayesian
analysis an acceptable, general, alternative to classical, frequentist
inference.

Conceptually, the main thing about Bayesian inference is that it uses
probability directly to characterize uncertainty about things we don't
know.  ``Things'', in this case, are parameters of models and, just as
it is natural to characterize uncertain outcomes of stochastic
processes using probability, it seems natural also to characterize
information about unknown ``parameters'' using probability. At least
this seems natural to us and, we think, most ecologists either
explicitly adopt that view or tend to fall into that point of view
naturally.  Conversely, frequentists use probability in many different
ways, but never to characterize uncertainty about
parameters\footnote{To hear this will be shocking to some readers
  perhaps.} Instead, frequentists use probability to characterize the
behavior of {\it procedures} such as estimators or confidence
intervals (see below), which can lead to some inelegant or unnatural
interpretations of things.  It is paradoxical that people readily
adopt a philosophy of statistical inference in which the things you
don't know (i.e., parameters) should {\it not} be regarded as random
variables, so that, as a consequence, one cannot use probability to
characterize ones state of knowledge about them.


\subsection{Bayes Rule}

As its name suggests, Bayesian analysis makes use of Bayes' rule in
order to make direct probability statements about model
parameters. Given two random variables $z$ and $y$, Bayes rule relates
the two conditional probability distributions $[z|y]$ and $[y|z]$ by
the relationship:
\[
[z|y] = [y|z][z]/[y]
\]
Bayes' rule itself is a mathematical fact and there is no debate in
the statistical community as to its validity and relevance to many
problems. Generally speaking, these distributions are characterized as
follows: $[y|z]$ is the conditional probability distribution of $y$
{\it given} $z$, $[z]$ is the marginal distribution of $z$ and $[y]$
is the marginal distribution of $y$. In the context of Bayesian
inference we usually associate specific meanings in which $[y|z]$ is
thought of as ``the likelihood'', $[z]$ as the ``prior'' and so on. We
leave this for later because here the focus is on this expression of
Bayes rule as a basic fact of probability.

As an example of a simple application of Bayes rule,
consider the problem of determining species presence at a sample
location based on imperfect survey information. Let $z$ be a binary
random variable that denotes species presence $(z=1)$ or absence
$(z=0)$, let $\Pr(z=1) = \psi$ where $\psi$ is usually called
occurrence probability, ``occupancy'' \citep{mackenzie_etal:2002} or ``prevalence''.
Let $y$ be the {\it observed} presence
($y=1$) or absence ($y=0$), and let $p$ be the probability that a
species is detected in a single survey at a site given that it is
present. Thus, $\Pr(y=1|z=1)=p$. The interpretation of this is that,
if the species is present, we will only observe presence with
probability $p$. In addition, we assume here that $\Pr(y=1|z=0) =
0$. That is, the species cannot be detected if it is not present which
is a conventional view adopted in most biological sampling problems (but
see \citet{royle_link:2006}).
If we survey a site $T$ times but never detect the species,
then this clearly does not imply that the species is not present
($z=0$) at this site. Rather, our degree of belief in $z=0$ should be
made with a probabilistic statement
$\Pr(z=1|y_1=0,\ldots,y_{T}=0)$. If the $T$ surveys are independent so
that we might regard $y_{t}$ as $iid$ Bernoulli trials, then the total
number of detections, say $y$, is Binomial with probability $p$ then
we can use Bayes rule to compute the probability that it is present
given that it is not detected in $T$ samples. In words, the expression
we seek is:
\[
\Pr(\mbox{present} | \mbox{not detected}) = \frac{\Pr(\mbox{not detected} |
  \mbox{present})\Pr(\mbox{present})}{\Pr(\mbox{detected})}
\]
Mathematically, this is
\begin{eqnarray*}
\Pr(z=1|y=0) &= &\Pr(y=0|z=1)\Pr(z=1)/\Pr(y=0)  \\
             &= & [(1-p)^{T} \psi]/[ (1-p)^T \psi + (1-\psi) ].
\end{eqnarray*}
To apply this,
suppose that $T=2$ surveys are done at a wetland for a species of
frog, and the species is not detected there. Suppose further that $\psi
= .8$ and $p = .5$ are obtained from a prior study.  Then the
probability that the species is present at this site is
$.25*.8/(.25*.8 + .2) = 0.50$. That is, there seems to be about a
50/50 chance that the site is occupied despite the fact that the
species wasn't observed there.

In summary, Bayes' rule provides a simple linkage between the
conditional probabilities $[y|z]$ and $[z|y]$ which is useful whenever
one needs to deduce one from the other.
Bayes' rule as a basic fact of probability is not disputed.


\subsection{Bayesian Inference}


What is controversial to some is the scope and manner in which Bayes
rule is applied by Bayesian analysts. Bayesian analysts assert that
Bayes rule is relevant, in general, to all statistical problems by
regarding all unknown quantities of a model as realizations of random
variables - this includes ``data'', latent variables, and also
``parameters''. Classical (non-Bayesian) analysts sometimes object to
regarding ``parameters'' as outcomes of random variables. Classically,
parameters are thought of as ``fixed but unknown'' (using the
terminology of classical statistics). Of course, in Bayesian analysis
they are also unknown and, in fact, there is a single data-generating
value and so they are also fixed. The difference is that this fixed
but unknown value is regarded as having been generated from some
probability distribution. Specification of that probability
distribution is necessary to carryout Bayesian analysis, but it is not
required in classical frequentist inference.


To see the general relevance of Bayes rule in the context of
statistical inference, let $y$ denote observations - i.e., ``data'' -
and let $[y|\theta]$ be the observation model (often colloquially
referred to as the ``likelihood'').  Suppose theta is a parameter of
interest having (prior) probability distribution $[\theta]$. These are
combined to obtain the posterior distribution using Bayes' rule, which
is:
\[
 [\theta|y]= [y|\theta][\theta]/[y]
\]
Asserting the general relevance of Bayes rule to all statistical
problems, we can conclude that the two main features of Bayesian
inference are that: (1) ``parameters'' $\theta$ are regarded as realizations of
a random variable and, as a result, (2) inference is based on the
probability distribution of the parameters given the data,
$[\theta|y]$,
which is
called the posterior distribution. This is the result of using Bayes
rule to combine ``the likelihood'' and the prior distribution.  The
key concept is regarding parameters as realizations of a random
variable because, once you admit this conceptual view, this leads
directly to the posterior distribution, a very natural quantity upon
which to base inference about things we don't know -  including
parameters of statistical models.  In particular, $[\theta|y]$ is a
probability distribution for $\theta$ and therefore we can make direct
probability statements to characterize uncertainty about
$\theta$.

The denominator of our invocation of Bayes rule, $[y]$,
is the marginal distribution of the data $y$.  We note without further
remark right now that, in many practical problems, this can be an
enormous pain to compute. The main reason that the Bayesian paradigm
has become so popular in the last 20 years or so is because methods
exist for characterizing the posterior distribution that do not
require that we possess a mathematical understanding of $[y]$, i.e.,
we never have to compute it or know what it looks like, or know
anything specific about it.

A common misunderstanding on the distinction between Bayesian and
frequentist inference goes something like this ``in frequentist
inference parameters are fixed but unknown but in a Bayesian analysis
parameters are random.'' At best this is a sad caricature of the
distinction and at worst it is downright wrong. What is true is that,
to a Bayesian, parameters are random variables. However, a Bayesian
assumes, just like a frequentist, that there was a single
data-generating value of that parameter - a fixed, and unknown value
that produced the given data set.
The distinction between Bayesian and frequentist approaches is that
Bayesians regard the parameter as a random variable, and its value as
the outcome of a random value, on par with the observations. This
allows Bayesians to use probability to make direct probability
statements about parameters. Frequentist inference procedures do not
permit direct probability statements to be made about parameter
values -- because parameters are not random variables!

While we can understand the conceptual basis of Bayesian inference
merely by understanding Bayes rule -- that's really all there is to it
-- it is not so easy to understand the basis of classical
``frequentist'' inference which is mostly
like\footnote{Characterization from Sims REF XYZ} a ``basket of
methods'' with little coherent organization. What is mostly coherent
in frequentist inference is the manner in which items in this basket
of methods are evaluated -- the performance of a given procedure is
evaluated by ``averaging over'' hypothetical realizations of $y$,
regarding the {\it estimator} as a random variable. For example, if
$\hat{\theta}$ is an estimator of $\theta$ then the frequentist is
interested in $E_{y}[\hat{\theta}|y]$ which is used to characterize
bias. If the expected value of $\hat{\theta}$, when averaged over
realizations of $y$, is equal to $\theta$, then $\hat{\theta}$ is
unbiased.

The view of parameters as fixed constants and estimators as random variables
leads to interpretations that are not so straightforward. For
example confidence intervals having the interpretation ``95\%
probability that the interval contains the true value" and p-values
being "the probability of observing an outcome as extreme or more than
the one observed.'' These are far from intuitive interpretations to
most people.  Moreover, this is conceptually probblematic to some
because the hypothetical realizations that characterize the
performance of our procedure we will never get to observe.

While we do tend to favor Bayesian inference for the conceptual
simplicity (parameters are random, posterior inference), we mostly
advocate for a pragamatic non-partisian approach to inference because,
frankly, some of these ``bucket of methods'' are actually very
convenient in certain situations as we will see in later chapters.


\subsection{Prior distributions}


The prior distribution $[\theta]$ is an important feature of Bayesian
inference. As a conceptual matter,
the prior distribution characterizes ``prior beliefs'' or ``prior
information'' about a parameter. Indeed,
an oft-touted benefit of Bayesian analysis is the ease with which
prior information can be included in an analysis.
However, more commonly, the prior is chosen to
express a lack of prior information, even if previous studies have
been done and even if the investigator does in fact know quite a bit
about a parameter.
This is because
the manner in which prior information is embodied in a prior (and the
amount of information) is
usually very subjective and thus the result can wind up being very
contentious, e.g., different investigators might report different
results based on subjective assessments of things. Thus it is usually
better to ``let the data speak'' and use priors that reflect absence
of information beyond the data set being analyzed.

But still the need occasionally arises to embody prior information or
beliefs about a parameter formally into the estimation scheme.
 In SCR models we often have a parameter that is closely linked
to ``home range radius'' and thus auxiliary information on the home
range size of a species can be used as prior information (e.g., see
\citet{chandler_royle:2012} ; also chapter XYZ).

XXXXXXXX
you gonna add something about priors and their potential to truncate posteriors here?
XXXXXXXX

XXXXXXXX

noninformative prior on one scale is informative on another scale.
e.g., flat prior on logit(p) is very different from uniform(0,1) on
p...
show graphic......

reference to non-invariance of prior distributions to transformation......

XXXXXXXX

\subsection{Posterior Inference}

In Bayesian inference, we are not focusing on estimating a single
point or interval but rather on characterizing a whole distribution --
the posterior distribution -- from which one can report any summary of
interest. A point estimate might be the posterior mean, median, mode,
etc..  In many applications in this book, we will compute 95\%
Bayesian intervals using the 2.5\% and 97.5\% quantiles of the
posterior distribution. For such intervals, it is correct to say
$\Pr(L < \theta < U) = 0.95$. That is, "the probability that $\theta$
is between $L$ and $U$ is $0.95$". 

As an
example, suppose we conducted a Bayesian analysis to estimate
detection probability of some species at a study site (p), and we
obtained a posterior distribution of beta(20,10) for the parameter
p. The following R commands demonstrate how we make inferences based
upon summaries of the posterior distribution. Fig. \ref{densityvsdetection.fig} shows the
posterior along with the summary statistics.

\begin{verbatim}
> (post.median <- qbeta(0.5, 20, 10))
[1] 0.6704151
> (post.95ci <- qbeta(c(0.025, 0.975), 20, 10))
[1] 0.4916766 0.8206164
\end{verbatim}

Thus, we can state that there is a 95\% probability that $\theta$ lies
between 0.49 and 0.82.

\begin{figure}
\begin{center}
\includegraphics[height=2.5in]{Ch2/figs/densityvsdetection}
%get figure file from Ch7 folder
\end{center}
\caption{Probability density plot of a hypothetical posterior distribution of beta(20,10); dashed lines indicate mean and upper and lower 95\% interval}
\label{densityvsdetection.fig}
\end{figure}

It is not a subtle thing that this
cannot be said using frequentist methods - although people tend to say
it anyway and not really understand why it is wrong or even that it is
wrong. This is actually a failing of frequentist ideas and the
inability of frequentists to get people to overcome their natural
tendency to use probability - which is something that, as a
frequentist, you simply cannot do in the manner that you would like
to.



Posterior inference is the main practical element of Bayesian
analysis. We get to make an inference conditional on the data that we
actually observed - i.e., what we actually know.  To us, this seems
logical - to condition on what we know. Conversely, frequentist
inference is based on considering average performance over
hypothetical unobserved data sets (i.e., the ``relative frequency''
interpretation of probability).  Frequentists know that their
procedures work well when averaged over all hypothetical, unobserved,
data sets but no one ever really knows how well they work for the
specific data set analyzed. That seems like a relevant question to
biologists who oftentimes only have their one, extremely valuable,
data set.  This distinction comes into play a lot in exposing
philosophical biases in the peer review of statistical analyses in
ecology in the sense that, despite these opposing conceptual views to
inference (i.e. conditional on the data you have, or averaged over
hypothetical realizations), those who conduct a Bayesian analysis are
often (in ecology, almost always) required to provide a frequentist
evaluation of their Bayesian procedure.

\subsection{Small sample inference}

Using Bayesian inference, we obtain an estimate of the posterior
distribution which is an exhaustive summary of the state-of-knowledge
about an unknown quantity. It is the posterior distribution - not an
estimate of that thing. It is also not, usually, an approximation
except to within Monte Carlo error (in cases where we use simulation
to calculate it).  One of the great virtues of Bayesian analysis which
is not really appreciated is that it is completely valid for any
particular sample size. i.e., it is $[\theta|y]$, as precise as we
claim it to be based on our ability to do calculations, for the
particular sample size and observations that we have even if we have
only a single datum $y$.  The same cannot be said for almost all
frequentist procedures in which estimates or variances are very often
(almost always in practice) based on ``asymptotic approximations'' to
the procedure which is actually being employed.

There seems to be a prevailing view in statistical ecology that
classical likelihood-based procedures are virtuous because of the
availability of simple formulas and procedures for carrying out
inference, such as calculating standard errors, doing model selection
by AIC, and assessing goodness-of-fit.  In large samples, this may be
an important practical benefit, but the theoretical validity of these
procedures cannot be asserted in most situations involving small
samples.  This is not a minor issue because it is typical in many
wildlife sampling problems - especially in surveys of carnivores or
rare/endangered species - to wind up with a small, sometimes extremely
small, data set. For example, a recent paper on the fossa
(Cryptoprocta ferox), an endangered carnivore in Madagascar, estimated
an adult density of 0.18 adults / km sq based on 20 animals captured
over 3 years \citep{hawkins_racey:2005}. A similar paper on the
endangered southern river otter (Lontra provocax) estimated a density
of 0.25 animals per river km based on 12 individuals captured over 3
years \citep{sepulveda_etal:2007}. \citet{gardner_etal:2010} analyzed
data from a study of the Pampas cat, a species for which very little
is known, wherein only 22 individual cats were captured .during the
two year period.  \citet{trolle_kery:2005} reported only 9 individual
ocelots captured and \citet{jackson_etal:2006} captured 6 individual
snow leopards using camera trapping. Thus, studies of rare and/or
secretive carnivores necessarily and flagrantly violate one of Le
Cam's Basic Principles, that of ``If you need to use asymptotic
arguments, do not forget to let your number of observations tend to
infinity.''\citep{lecam:1990}.

The biologist thus faces a dilemma with such data. On one hand, these
datasets, and the resulting inference, are often criticized as being
poor and unreliable. Or, even worse\footnote{Actual quote from a
  referee}, ``the data set is so small, this is a poor analysis.''  On
the other hand, such data may be all that is available for species
that are extraordinarily important for conservation and management.
The Bayesian framework for inference provides a valid, rigorous, and
flexible framework that is theoretically justifiable in arbitrary
sample sizes. This is not to say that one will obtain precise
estimates of density or other parameters, just that your inference is
coherent and justifiable from a conceptual and technical statistical
point of view. That is, we report the posterior probability
$\Pr(D|data)$ which is easily interpretable and just what it is
advertised to be and we don't need to do a simulation study to
evaluate how well some approximate $\Pr(D|data)$ deviates from the
actual $\Pr(D|data)$ because they are precisely the same quantity.



\section{Characterizing posterior distributions by MCMC simulation}

In practice, it is not really feasible to ever compute the marginal
probability distribution $\Pr(y)$, the denominator resulting from
application of Bayes' rule. For decades this impeded the adoption of
Bayesian methods by practitioners. Or, the few Bayesian analyses done
were based on asymptotic normal approximations to the posterior
distribution. While this was useful stuff from a theoretical and
technical standpoint and, practically, it allowed people to make the
probability statements that they naturally would like to make, it was
kind of a bad joke around the Bayesian water-cooler to, on one hand,
criticize classical statistics for being, essentially, completely ad
hoc in their approach to things but then, on the other hand, have to
devise various approximations to what they were trying to
characterize. The advent of Markov chain Monte Carlo (MCMC) methods
has made it easier to calculate posterior distributions for just about
any problem to arbitrary levels of precision.

Broadly speaking, MCMC is a class of methods for drawing random
numbers (sampling or simulating) from the target posterior
distribution.  Thus, even though we might not recognize the posterior
as a named distribution or be able to analyze its features
analytically, e.g., devise mathematical expressions for the mean and
variance, we can use these MCMC methods to obtain a large sample from
the posterior and then use that sample to characterize features of the
posterior. What we do with the sample depends on our intentions --
typically we obtain the mean or median for use as a point estimate,
and take a confidence interval based on Monte Carlo estimates of the
quantiles.  These are estimates, but not like frequentist
estimates. Rather, they are Monte Carlo estimates with an associated
Monte Carlo error which is largely determined arbitrarily by the
analyst. They are not estimates qualified by a sampling distribution
as in classical statistics. If we run our MCMC long enough then our
reported value of $E[\theta|y]$ or any feature of the posterior
distribution is precisely what we say it is. There is no ``sampling
variation'' in the frequentist sense of the word.  In summary, the
MCMC samples provide a Monte Carlo characterization of {\it the}
posterior distribution.


\section{What Goes on Under the MCMC Hood}

We will develop and apply MCMC methods in some detail for spatial
capture-recapture models in chapter \ref{chapt.mcmc}. Here we provide
a simple illustration of some basic ideas related to the practice of MCMC.

A type of MCMC method relevant to most problems is Gibbs sampling (REF
XYZ XYZ),
which is based on the idea of iterative simulation from the ``full
conditional'' distributions (also called conditional posterior
distributions). The full conditional distribution for an unknown
quantity is the conditional distribution of that quantity given every
other random variable in the model - the data and all other
parameters. For example, for a normal regression model with $y \sim
\mbox{Normal}(\alpha + \beta x , 1)$ then the two full conditionals are, in
symbolic terms,
\[
[\alpha|y,\beta]
\]
 and
\[
[\beta|y,\alpha].
\]
We might use our knowledge of probability to identify these
mathematically. In particular, by Bayes' Rule, $[\alpha|y,\beta] =
[y|\alpha,\beta][\alpha|\beta]/[y|\beta]$ and similarly for
$[\beta|y,\alpha]$. For example, if we have priors for $[\alpha]$ and $[\beta]$
which are also normal distributions, some algebra reveals that
XXXX COPY NOTATION FFROM CH. 6 XXXXX
\[
[\alpha|y,\beta] = Normal(ybar,...weighted variance here...).
\]
Similarly,
\[
 [\beta|y,\alpha] is normal(........)
\]

The MCMC algorithm for this model has us simulate in succession,
repeatedly, from those two distributions. See \citet{gelman_etal:2004}
for more examples of Gibbs sampling for the normal model. A
conceptual representation of the MCMC algorithm for this simple model
is therefore:
XXXX Check out ALGORITHM environment XXXXX
\begin{verbatim}
 Algorithm

       0. Initialize $\alpha$ and $\beta$

       Repeat{
           1. Draw a new value of $\alpha$ from Eq. \ref{xyz}

           2. Draw a new value of $\beta$ from Eq. \ref{xyz}
       }
\end{verbatim}

As we just saw for this simple ``normal-normal'' model it is sometimes
possible to specify the full conditional distributions
analytically. In general, when certain so-called conjugate prior
distributions are chosen, the form of full conditional distributions
is similar to that of the observation model. In this normal-normal
case, the normal distribution for the mean parameters is the conjugate
prior under the normal model, and thus the full-conditional
distributions are also normal. This is convenient because, in such
cases, we can simulate directly from them using standard methods (or
{\bf R}
functions).  But, in practice, we don't really ever need to know such
things because most of the time we can get by using a simple
algorithm, called the Metropolis-Hastings (henceforth ``MH'')
algorithm, to obtain samples from these full conditional distributions
without having to recognize them as specific, named, distributions.
This gives us enormous freedom in developing models
and analyzing them without having to resolve them mathematically
because to implement the MH algorithm we need only identify the full
conditional distribution up to a constant of proportionality, that
being the marginal distribution in the denominator (e.g., $[y|\beta]$
above).

We will talk about the Metropolis-Hastings algorithm shortly, and we
will use it extensively in the analysis of SCR models (e.g., chapter
\ref{chapt.mcmc}).

\subsection{Rules for constructing full conditional distributions}
\label{glms.sec.rules}

The basic strategy for constructing full-conditional distributions for
devising MCMC algorithms can be reduced conceptually to a couple of
basic steps summarized as follows:
\begin{itemize}
\item[(step 1)] Collect all stochastic components of the model;
\item[(step 2)] Recognize and express the full conditional in question
  as proportional to the product of all components;
\item[(step 3)] Remove the ones that don't have the focal parameter in them.
\item[(step 4)] Do some algebra on the result in order to identify the resulting pdf or pmf.
\end{itemize}
Of the 4 steps, the last of those is the main step that requires quite
a bit of statistical experience and intuition because various
algebraic tricks can be used to reshape the mess into something
noticeable - i.e., a standard, named distribution. But step 4 is not
necessary if we decide instead to use the Metrpolis-Hastings algorithm
as described below.

To illustrate for computing $[\alpha|y,\beta]$ we first apply step 1
and identify the model components as: $[y|\alpha, \beta]$, $[\alpha]$
and $[\beta]$. Step 2 has us write $[\alpha|y,\beta] \propto
[y|\alpha,\beta][\alpha][\beta]$.  Step 3: We note that $[\beta]$ is not a
function of alpha and therefore we remove it to obtain $[\alpha|y,\beta]
\propto [y|\alpha,\beta][\alpha]$. Similarly we obtain $[\beta|y,\alpha]
\propto [y|\alpha,\beta][\beta]$. We apply step 4 and manipulate
these algebraically to arrive at the result or, alternatively, we can
sample them indirectly using the Metropolis-Hastings algorithm (see
below).


\subsection{Metropolis-Hastings algorithm}

The Metropolis-Hastings (MH) algorithm is a completely generic method for
sampling from any distribution, say $f(\theta)$. In our applications,
$f(\theta)$ will typically be the full conditional distribution of
$\theta$.
While we sometimes use Gibbs sampling, we seldom
use ``pure'' Gibbs sampling because we might use MH to sample from one
or more of the full conditional distributions.
When the MH algorithm is used to sample from  full
conditional distributions of a Gibbs sampler the resulting hybrid algorithm is
called
 {\it Metrpolized Gibbs sampling} or
more commonly {\it Metropolis-within-Gibbs}.
Shortly we will
actually construct such an algorithm for a simple class of models.

The MH algorithm generates candidates from some
proposal or candidate-generating distribution, that may be conditional
on the current value of the parameter, denoted by
$h(\theta^{*}|\theta^{t})$. Here, $\theta^{*}$ is the {\it candidate}
or proposed
value and $\theta^{t}$ is the current value, i.e., at iteration $t$ of
the MCMC algorithm.  The proposed value
is accepted with probability

\[
r = \frac{ f(\theta^{*}) h(\theta^{t}|\theta^{*})}
    {f(\theta^{t}) h(\theta^{*}|\theta^{t}) }
\]
which we call the MH acceptance probability.
This ratio can sometimes be $>1$ in which case we set it equal to
1. It is useful to note that $h()$ can be anything at all. No matter
the choice of $h()$, we can evaluate this ratio numerically because
the marginal $f(y)$ cancels from both the numerator and
denominator, which is the magic of the MH algorithm.


\section{Practical Bayesian Analysis and MCMC}

There are a number of really important practical issues to be
considered in any Bayesian analysis and we cover some of these briefly
here.

\subsection{Choice of prior distributions}

{\bf XXX integrate this material with previous section on prior
distributions XXXXXX}

Bayesian analysis requires that we choose prior
distributions for all of the structural parameters of the model (we
use the term structural parameter to mean all parameters that aren't
customary thought of as latent variables). We will strive to use
priors that are meant to express little or no prior information -
default or customary ``non-informative'' or diffuse priors. This will
be $\mbox{Unif}(a,b)$ priors for parameters that have a natural
bounded support and, for parameters that live on the real line we use
either (1) diffuse normal priors; (2) ``improper'' uniform priors or
(3) sometimes even a bounded $\mbox{Unif}(a,b)$ prior if that greatly
improves the performance of {\bf WinBUGS} or other software doing the MCMC
for us.  In {\bf WinBUGS} a prior with low ``precision'', $\tau$, where
$\tau = 1/\sigma^2$, such as $\mbox{Norm}(0,.01)$ will typically be
used. Of course $\tau = 0.01$ ($\sigma^{2} = 100$) might be very
informative for a regression parameter that has a high
variance. Therefore, we recommend that predictor variables {\it
  always} be standardized. Clearly there are a lot of choices for
ostensibly non-informative priors, and the degree of
non-informativeness depends on the parameterization. For example, a
natural non-informative prior for the intercept of a logistic
regression
\[
\mbox{logit}(p_{i}) = \alpha + \beta x_{i}
\]
Would be $[\alpha] = \mbox{const}$ which is the same as saying $a \sim
Unif(\infty,infty)$, the customary improper uniform prior.
However, we might also use a prior on the parameter $p0
= logit^{-1}(a)$, which is $Pr(y=1)$ for the value $x=0$. Since $p0$ is a
probability a natural choice is $p0 \sim Unif(0,1)$. These two priors can
affect results (see Chapter 3.XYZ), yet they are both sensible
non-informative priors. Choice of priors and parameterization is
very much problem-specific and often largely subjective. Moreover, it
also affects the behavior of MCMC algorithms and therefore the analyst
needs to pay some attention to this issue and possibly try different
things out.
XXX REFS on prior distributions XXXXXX

\subsection{Convergence and so-forth}

Once we have carried-out an analysis by MCMC, there are many other
practical issues that we have to confront.  One of the most important
is ``have the chains converged?'' Since we do not know what the stationary posterior distribution of our Markov chain should look like (this is the whole point of doing an MCMC approximation), we effectively have no means to assess whether it has truly converged to this desired distribution or not. Most MCMC algorithms only guarantee
that, eventually, the samples being generated will be from the target
posterior distribution, but no-one can tell us how long this will take. Also, you only now the part of your posterior distribution that the Markov chain has explored so far– for all you know the chain could be stuck in a local maximum, while other maxima remain completely undiscovered.  Acknowledging that there is truly nothing we can do to ever proof convergence of our MCMC chains, there are several things we can do to increase the degree of confidence we have about the convergence of our chains. Some problems are easily detected using simple plots.  Typically a period of transience is
observed in the early part of the MCMC algorithm, and this is usually
discarded as the ``burn-in'' period. The quick diagnostic to whether convergence has been achieved is that
your Markov chains look ``grassy'' -- see Fig.  \ref{glms.fig.grassy}
below.  
Another way to check
convergence is to update the parameters some more and see if the
posterior changes. Yet another option, and one generally implemented in WinBUGS, is to run several Markov chains and to start them off at different initial values that are overdispersed relative to the posterior distribution. Such initial values help to explore different areas of the parameter space simultaneously; if after a while all chains oscillate around the same average value, chances are good that they indeed converged to the posterior distribution. \footnote{Running several parallel chains is computationally expensive. But extra computational demands are not the only and by no means the major concern some people voice when it comes to running several parallel MCMC chains to assess convergence. Again, consider the fact that we do not know anything about the true form of the posterior distribution we are trying to approximate. How do we, then, know how to pick overdispersed initial values? We don’t – all we can do is pick overdispersed values relative to our expectations of what the posterior should look like. To use a quote from the home page of Charlie Geyer, a Bayesian statistician from the University of Minnesota, ``If you don't know any good starting points [...], then restarting the sampler at many bad starting points is [...] part of the problem, not part of the solution.'' (http://users.stat.umn.edu/~charlie/mcmc/diag.html). His suggestion is that your only chance to discover a potential problem with your MCMC sampler is to run it for a very long time. But again, there is no way of knowing how long is long enough.
It is up to you to decide, which school of thoughts appeals more to you – one long versus several parallel Markov chains. Irrespectively, part of developing an MCMC sampler should be to make sure (within reasonable limits) that you are not missing regions of high posterior density because of the way you specify your starting values. Once you have explored the behavior of your chain under a – reasonable – range of starting values, you may feel comfortable enough to run only one long chain.} Gelman and Rubin came up with the so-called``R-hat'' statistic ($\hat{R}$) or Brooks-Gelman-Rubin statistic
 that essentially compares within-chain and between-chain variance to check for convergence of multiple chains (\citep{gelman_etal:1996}). $\hat{R}$ should be close to 1 if the Markov
chains have converged and sufficient posterior samples have been
obtained. In practice, $\hat{R} = 1.2$ is probably good enough for
some problems.  For some models you can't actually realize a low
$\hat{R}$. E.g., if the posterior is a discrete mixture of distributions
then you can be misled into thinking that your Markov chains
have not converged when in fact the chains are just jumping back and
forth in the posterior state-space. 
So, for example, model
selection methods (section XYZ) sometimes suggests non-convergence.
Another situation is when one of the parameters is on the boundary of
the parameter space which might appear to be very poor mixing, but all
within some extreme region of the parameter space.\footnote{it would
  be nice if we could compile examples of this later in the book and
  reference back to this point}.
This
kind of stuff is normally ok and you need to think really hard about
the context of the model and the problem before you conclude that your
MCMC algorithm is ill-behaved.

Some models exhibit ``poor mixing'' of the Markov chains or what
people might also say ``have not coverged'' (or ``slow convergence'')
which is a term we would disagree with because the samples might well
be from the posterior (i.e., the Markov chains have converged to the
proper stationary distribution) but simply mix around the posterior
rather slowly. Anyway, poor mixing can happen for a huge number of
reasons -- when parameters are highly correlated (even confounded), or
barely identified from the data, or the algorithms are very terrible
and probably many other reasons.  Slow mixing equates to high
autocorrelation in the Markov chain - the successive draws are highly
correlated, and thus we need to run the MCMC algorithm much longer to
get an effective sample size that is sufficient for estimation - or to
reduce the MC error to a tolerable level.  A strategy often used to
reduce autocorrelation is ``thinning'' - i.e., keep every $m^{th}$
value of the Markov chain output. However, thinning is necessarily
inefficient from the stand point of inference - you can always get
more precise posterior estimates by using all of the MCMC output
regardless of the level of autocorrelation
\citep{maceachern_berliner:1994}. Practical considerations might
necessitate thinning, even though it is statistically inefficient. For
example, in models with many parameters or other unknowns being
tabulated, the output files might be enormous and unwieldy to work
with. In such cases, thinning is perfectly reasonable. In many cases,
how well the Markov chains mix is strongly influenced by
parameterization, standardization of covariates, and the prior
distributions being used. Some things work better than others, and the
investigator should experiment with different settings and
remain calm when things don't work out perfectly. MCMC is an
art, and a science.


{\bf Is the posterior sample large enough?}  The subsequent samples generated from a Markov chain are not iid samples from the posterior distribution, due to the correlation amongst samples introduced by the Markov process and the sample size has to be adjusted to account for the autocorrelation in subsequent samples (see Chapter 8 in \citet{robert_casella:2010} for more details). This adjusted sample size is referred to as the effective sample size. Checking the degree of autocorrelation in your Markov chains and estimating the effective sample size your chain has generated should be part of evaluating your model output. WinBUGS will automatically return the effective sample size for all monitored parameters. If you find that your supposedly long Markov chain has only generated a very short effective sample, you should consider a longer run. What exactly constitutes a reasonable effective sample size is hard to say. A more palpable measure of whether you've run your chain for enough iterations is the time-series or Monte Carlo error – the 'noise' introduced into your samples by the stochastic MCMC process. The MC error is printed by default in
summaries of BUGS output. You want that to be smallish relative to the
magnitude of the parameter and this might depend on the purpose of the
analysis. For a preliminary analysis you might settle for a few
percent whereas for a final analysis then certainly less than 1\% is
called for, but you can run
your MCMC
algorithm as long as it takes. A consequence of the MC error is that even for the exact same model, results will always be different. Thus, as a good rule
of thumb you should never report
MCMC results to more than 2 decimal places.
Note that MC error in summaries of the
posterior is not the same as having an ``approximate'' solution in a
standard likelihood analysis or similar.  The approximate SE in
likelihood inference is actually wrong in its actual value.... XYZ.


\subsection{Bayesian confidence intervals}

The 95\% Bayesian interval based on percentiles of the posterior
is not a unique interval - there are many of them - and the so-called
``highest posterior density'' (HPD) interval is the narrowest
interval. We might compute that frequently because it is easy to do
with an integer parameter which $N$ is (See the next chapter). The
95 \% HPD is not often exactly 95\% but usually slightly more
conservative than nominal because it is the narrowest interval that
contains at least 95\%  of the posterior mass.

\subsection{Estimating functions of parameters}

A benefit of analysis by MCMC is that we can seamlessly estimate
functions of parameters by simply tabulating the desired function of
the simulated posterior draws. For example, if $\theta$ is the
parameter of interest and let $\theta^{(i)}$ for $i=1,2,\ldots,M$ be
the posterior samples of $\theta$. Let $\eta = exp(\theta)$, then a
posterior sample of $\eta$ can be obtained simply by computing
$exp(\theta^{(i)})$ for $i=1,2,\ldots,M$. We give another example in
section
\ref{glms.sec.xopt}
below and throughout this book.
Almost all SCR models in this book involve at least 1 derived
parameter. For example, density $D$ is a derived parameter, being a
function of population size $N$ and the area $A$ of the underlying
state-space of the point process (see chapter \ref{chapt.scr0}).

\section{Bayesian Analysis using WinBUGS}

We won't be too concerned with devising our own MCMC algorithms for
every analysis
although we will do that a few times for fun.  More often, we
will rely on the freely available software package {\bf WinBUGS} or
{\bf JAGS}
for doing this.  We will always execute these {\bf BUGS} engines from
within {\bf R} using the \mbox{\tt R2WinBUGS} (REF XYZ XYZ) or
\mbox{\tt rjags} pacages. {\bf WinBUGS} and {\bf JAGS} are  MCMC black boxes
that takes a pseudo-code description (i.e., written in the {\bf BUGS}
language) of all of the relevant stochastic
and deterministic elements of a model and generates an MCMC algorithm
for that model. But you never get to see the algorithm. Instead,
{\bf WinBUGS}/{\bf JAGS} will run the algorithm and just return the Markov chain output
- the posterior samples of model parameters.

The great thing about using the {\bf BUGS} language is that it forces
you to become intimate with your statistical model - you have to write
each element of the model down, admit (explicitly) all of the various
assumptions, understand what the actual probability assumptions are
and how data relate to latent variables and data and latent variables
relate to parameters, and how parameters relate to one another.

While we normally use
{\bf WinBUGS} or {\bf JAGS} in this book, we note that {\bf
 OpenBUGS} is the current active development tree of the {\bf BUGS}
language. See \citet[][ch.xyz]{kery:2010} and
\citet[][appendix xyz]{kery_schaub:2011} for more on practial analysis
in {\bf WinBUGS}.
That book should also be consulted
for a more comprehensive introduction to using {\bf WinBUGS}. In this
example, we're going to accelerate pretty fast.

\subsection{Linear Regression in WinBUGS}

We provide a brief introductory example of a normal regression model
using a small simulated data set. The following commands are executed
from within your R workspace, the command line being indicated by
\mbox{\tt ``>''}. First, simulate a covariate $x$ and observations $y$ having
prescribed intercept, slope and variance:
\begin{verbatim}
> x<-rnorm(10)
> mu<- -3.2+ 1.5*x
> y<-rnorm(10,mu,sd=4)
\end{verbatim}
The {\bf BUGS} model specification for a normal regression model is
written within {\bf R} as a character string input to the command
\mbox{\tt cat()} and
then dumped to a text file named \mbox{\tt normal.txt}:
\begin{verbatim}
> cat("
model {
   for (i in 1:10){
      y[i]~dnorm(mu[i],tau)        # the "likelihood"
      mu[i]<- beta0 + beta1*x[i]   # the linear predictor
     }
   beta0~dnorm(0,.01)              # prior distributions
   beta1~dnorm(0,.01)
   sigma~dunif(0,100)
   tau<-1/(sigma*sigma)            # tau is a derived parameter
}
",file="normal.txt")
\end{verbatim}
Alternatively, you
can write the model specifications directly within a text file and
save it in your current working directory, but we do not usually take
that approach in this book.

{\bf Remarks:} {\bf 1.} {\bf WinBUGS} parameterizes the normal in
terms of the mean and inverse-variance, called the precision. Thus,
\mbox{\tt dnorm(0,.01)} implies a variance of 100;
{\bf 2.} We typically use diffuse normal priors for mean parameters, $\beta_0$ and $\beta_1$ in this case, but sometimes we might use uniform priors with suitable bounds -B and +B.
{\bf 3.} We typically use a $\mbox{Unif}(0,B)$ prior on standard
deviation parameters
(Gelman XXX 2006 XXXX). But sometimes we might use a gamma prior on the precision parameter $\tau$.
{\bf 4.} In a {\bf WinBUGS} model file, every variable referenced in
the model description has to be
either data, which will be input (see below), a random variable which
must have a probability distribution associated with it using the
``\verb#~#'', or it has to be a derived parameter connected to variables and
data using ``\mbox{\tt <-}''.


To fit the model, we need to describe various data objects to {\bf
  WinBUGS}. In particular,
we create an {\bf R} list object called \mbox{\tt data} which
are the data objects identified in the BUGS model file.
 In the example, the
data consist of two objects which exist as $y$ and $x$ in the {\bf R}
workspace and also in the {\bf WinBUGS} model definition.
 We also have to create an {\bf R} function
that produces a list of starting values \mbox{\tt inits} that get sent to
{\bf WinBUGS}.
 Finally, we identify
the names of the parameters (labeled correspondingly in the {\bf WinBUGS}
model specification) that we want {\bf WinBUGS} to save the MCMC output
for. In this example, we will ``monitor'' the parameters
$\beta_0$, $\beta_1$, $\sigma$ and $\tau$.
{\bf WinBUGS} is executed using the {\bf R} command
\mbox{\tt bugs()}.
We set the option \mbox{\tt debug=TRUE} if we want the {\bf WinBUGS}
GUI to stay open (useful for analyzing MCMC output and looking at the
{\bf WinBUGS} error log). Also, we set \mbox{\tt working.dir=getwd()}
so that {\bf WinBUGS} output files and the log file are saved in the
current {\bf R} working directory.
  All of these activities look like this:
{\small
\begin{verbatim}
 library("R2WinBUGS")    # "attach" the R2WinBUGS library
 data <- list ( "y","x")
 inits <- function()
  list ( beta1=rnorm(1),beta0=rnorm(1),sigma=runif(1,0,2) )
 parameters <- c("beta0","beta1","sigma","tau")
 out<-bugs (data, inits, parameters, "normal.txt", n.thin=2, n.chains=2,
             n.burnin=2000, n.iter=6000, debug=TRUE,working.dir=getwd())
\end{verbatim}
}

{\bf Remarks:} A common question is ``how should my data be
formatted?'' That depends on how you describe the model in the {\bf
  BUGS} language, how your data are input into {\bf R} and
subsequently formatted.  There is no unique way to describe any
particular model and so you have some flexibility. We talk about data
format further in the context of capture-recapture models and SCR
models in chapter \ref{chapt.scr0} and elsewhere.  In general,
starting values are optional but we recommend to always provide
reasonable starting values for structural parameters, but are not
always necessary for random effects.  Note that the previously created
objects defining data, initial values and parameters to monitor are
passed to the function \mbox{\tt bugs()}.  In addition, various other
things are declared: The number of Markov chains (\mbox{\tt
  n.chains}), the thinning rate (\mbox{\tt n.thin}),
the number of burn-in iterations (\mbox{\tt n.burnin}) and the total
number of iterations
(\mbox{\tt n.iter}).
To develop a detailed understanding of the various parameters and
settings used for MCMC, consult a basic reference such as
\citet{kery:2010}.



You should execute all of the commands given above and then look at
the resulting output. Kill the {\bf WinBUGS} GUI and the data will be
read back into {\bf R} (or specify \mbox{\tt debug=FALSE}).  We don't
want to give instructions on how to navigate and use the GUI - see XYZ
REF (XYZ) for that.
The object \mbox{\tt out} prints important
summaries by default (this is slightly edited):

{\small
\begin{verbatim}
> print(out,digits=2)
Inference for Bugs model at "normal.txt", fit using WinBUGS,
 2 chains, each with 6000 iterations (first 2000 discarded), n.thin = 2
 n.sims = 4000 iterations saved
          mean   sd  2.5%   25%   50%   75% 97.5% Rhat n.eff
beta0    -2.43 1.84 -6.21 -3.50 -2.42 -1.34  1.27    1  4000
beta1     2.62 1.54 -0.42  1.68  2.62  3.57  5.67    1  4000
sigma     5.29 1.66  3.11  4.14  4.95  6.05  9.39    1  4000
tau       0.05 0.02  0.01  0.03  0.04  0.06  0.10    1  4000
deviance 59.85 3.24 56.18 57.47 59.00 61.37 68.32    1   840

For each parameter, n.eff is a crude measure of effective sample size,
and Rhat is the potential scale reduction factor (at convergence, Rhat=1).

DIC info (using the rule, pD = Dbar-Dhat)
pD = 2.6 and DIC = 62.4
\end{verbatim}
}

{\bf Remarks:} (1) convergence is assessed using the $\hat{R}$
statistic -- which we might sometimes write ``$Rhat$''. A value of $Rhat$ near 1
indicates convergence; (2) DIC is the
``deviance information criterion'' \citep{spiegelhalter_etal:2002}
(see section \ref{glms.sec.modsel})
 which
some people use in a manner similar to AIC although it is recognized
to have some problems in hierarchical models \citep{millar:2009}. We
evaluate this in the context of SCR models in chapter XYZ XYZ.

\subsection{Inference about functions of model parameters}
\label{glms.sec.xopt}

Using the MCMC draws for a given model we can easily obtain the
posterior distribution of any function of model parameters.  We showed
this in the above example by providing the posterior of $\tau$ when
the model was parameterized in terms of standar deviation $\sigma$.
 As another example, suppose that the
normal regression model above had a quadratic response function of the
form
\[
	E(y_i) = \beta_0 + \beta_1 x_i + \beta_2 x_{i}^{2}
\]
Then the optimum value of $x$, i.e., that corresponding to the optimal
expected response, can be found by setting the derivative of
this function to 0 and solving for $x$. We find that
\[
df/dx = \beta_1 +
2*\beta_2 x = 0
\]
yields that $x_{opt} = -\beta_1/(2*\beta_2)$.  We can just
take our posterior draws for $beta_1$ and $beta_2$ and obtain a
posterior sample of $x_{opt}$ by this simple calculation. As an exercise, take
the normal model above and simulate a quadratic response and then
describe the posterior distribution of $x_{opt}$.


\section{Model Checking and Selection}
\label{glms.sec.modsel}

In general terms model checking - or assessing the adequacy of the
model - and model selection are quite thorny issues and, despite
contrary and, sometimes, strongly held belief among practitioners, there are not
really definitive, general solutions to either problem. We're against
dogma on these issues and think people need to be open-minded about
such things and recognize that models can be useful whether or not
they pass certain statistical tests. Some models are intrinsically
better than others because they make more biological sense or foster
understanding or achieve some objective that some  bootstrap
or other goodness-of-fit test can't decide for you. That said, it
gives you some confidence if your model seems adequate and we try to
provide some fit assessment in most real applications of SCR models
We provide a very brief overview of concepts here, but provide more
detailed coverage in chapter \ref{chapt.gof}.
See also
\citet[][ch. xyz]{kery:2010} and
\citet[][ch. xyz]{link_barker:2009}
for specific context related to Bayesian
model checking and selection.

\subsection{Goodness-of-fit}

Goodness-of-fit testing is an important element of any analysis
because  our model represents a general set of hypotheses
about the ecological and observation processes that generated our
data. Thus, if our model ``fits'' in some statistical or scientific
sense, then we believe it to be consistent with the hypotheses that
went into the model. More formally, we would conclude that the data
are {\it not inconsistent} with the hypotheses, or that the model
appears adequate. If we have enough
data, then of course we will reject any set of statistical hypotheses.
Conversely, we can always come up with a model that fits by making the
model extremely complex. Despite this paradox, it seems to us that
simple models that you can understand should usually be preferred even
if they don't fit, for example if they embody essential mechanisms
central to our understanding of things, or
if we think that some contributing factors to lack-of-fit are minor or
irrelevant to the scientific context and intended use of the model.
In other words, models can be useful irrespective of whether they fit
according to some formal statistical test of fit.  Yet
the tension is there to obtain fitting models, and this comes naturally at
the expense of models that can be easily interpreted and studied and
effectively used.
Moreover, conducting goodness-of-fit tests is
not always so easy to do. Moreover, it is never really easy (or
especially convenient) to decide if your goodness-of-fit test is worth
anything. It might have 0 power!
Despite this,
we recommend attempting to assess model fit in real applications,
as a general rule, and we provide some basic guidance here and some more
specific to SCR models in
chapter \ref{chapt.gof}.

To evaluate goodness-of-fit in Bayesian analyses, we will most often
use the Bayesian p-value \citep{gelman_etal:1996}.  The basic idea is to define
a fit statistic or ``discrepancy measure'' and compare the posterior distribution of that
statistic to the posterior predictive distribution of that statistic
for hypothetical perfect data sets for which the model is known to be correct. For
example, with count frequency data, a standard measure of fit is the
sum of squares of the ``Pearson residuals'',
\[
D(y_i,\theta) = \frac{(y_i - E(y_i))^{2}}{Var( y_{i} )}
\]
The fit statistic based on the squared residuals is
\[
FIT = \sum_{i} D(y_{i},\theta)^{2}
\]
which can be computed at each iteration of a MCMC algorithm given the
current values of parameters that determine the
 response distribution.  At the same time (i.e., at each MCMC
 iteration),
the equivalent statistic is computed for a
``new'' data set, simulated using the current parameter values. The
Bayesian p-value is simply the posterior probability $\Pr(\mbox{Fit} >
\mbox{Fit}_{new})$\footnote{Check this definition!}
 which should be close to $0.50$ for a good model -- one that
 ``fits'' in the sense that the observed data set is
 consistent with realizations simulated under the model being fitted
 to the observed data. In practice
we judge ``close to 0.50'' as being ``not too close to 0 or 1'' and,
as always, closeness is somewhat subjective. We're happy with anything
$>.1$ and $<.9$ but might settle for $>.05$ and $<0.95$. In summary,
the Bayesian p-value seems like a bootstrap idea, is easy to compute,
and widely used as a result.

Another useful fit statistic is the Freeman-Tukey
statistic\footnote{Ref for this?}, in which
\[
D({\bf y},\theta) = \sum_{i} ( \sqrt(y_{i}) - \sqrt(e_{i}) )^2
\]
\citep{brooks_etal:2000}, where $y_{i}$ is the observed value of
observation $i$ and $e_{i}$ its expected value. In contrast to a
chi-square discrepancy, the Freeman-Tukey statistic removes the need
to pool cells with small expected values.


\subsection{Model Selection }

For model selection we typically use three different methods: First
is, let's say, common sense. If a parameter has posterior mass
concentrated away from 0 then it seems like it should be regarded as
important - that is, it is ``significant.''  This approach seems to
have fallen out of favor with all of the interest over the last 10 or
15 years on model selection in ecology. It seems reasonable to us.


For regression problems we sometimes use the factor weighting idea
which is to introduce a set of binary variables $w_{k}$ for variable
$k$, and express the model as, e.g., for a single covariate model:
 \[
 E(y_i) = \alpha + w \beta x_{i}
\]
where $w$ is given a Bernoulli prior distribution with some prescribed
probability. E.g., $w \sim Bern(0.50)$ to provide a prior probability
of 0.50 that variable $x$ should be an element of the linear
predictor. The posterior probability of the event $w=1$ is a gauge of
the importance of the variable $x$. i.e., high values of $\Pr(w=1)$
indicate stronger evidence to support that ``$x$ is in the model''
whereas values of $\Pr(w=1)$ close to 0 suggest that $x$ is less
important.



This idea seems to be due to \citet{kuo_mallick:1998}\footnote{ Is
  this also what people call Zellner's G-priors?} and see
\citet[][ch. XXXX]{royle_dorazio:2008} for an example in the context
of logistic regression. This approach seems to even work sometimes
with fairly complex hierarchical models of a certain form. E.g.,
\citet{royle:2008} applied it to a random effects model to evaluate
the importance of the random effect component of the model.  The main
problem with this approach is that its effectiveness and results will
typically be highly sensitive to the prior distribution on the
structural parameters (e.g., see \citet[][table xyz]{royle_dorazio:2008}).
The reason for this is obvious: If $w = 0$ for the current
iteration of the MCMC algorithm, so that $\beta$ is sampled from the
prior distribution, and the prior distribution is very diffuse, then
extreme values of $\beta$ are likely. Consequently, when the current value of
$\beta$ is
far away from the mass of the posterior when $w=1$, then the Markov
chain may only jump from $w=0$ to $w=1$ infrequently.  One seemingly
reasonable solution to this problem (Aitken XYZ FIND THIS
XXXXX\footnote{see Royle 2008 paper for reference}) is to fit the full
model to obtain posterior distributions for all parameters, and then
use those as prior distributions in a ``model selection'' run of the
MCMC algorithm.  This seems preferable to more-or-less arbitrary restriction of
the prior support to improve the performance of the MCMC algorithm.

A third method that that we advocate is subject-matter
context. It seems that there are some situations -- some models -- where one should not
have to do model selection because it is necessitated by the specific
context of the problem, thus rendering a formal hypotesis test pointless
\citep{johnson:1999}.
SCR models are such an example. In SCR models, we will see that
``spatial location'' of individuals is an element of the model. The
simpler, reduced, model is an ordinary capture-recapture model which
is not spatially explicit (i.e., chapter \ref{chapt.closed}),
but it seems silly and pointless to think about actually using the
reduced model even if we could concoct some statistical test to refute
the more complex model.  The simpler model is manifestly wrong but,
more importantly, not even a plausible data-generating model!
Other examples are when effort, area or
sample rate is used as a covariate. One might prefer to have such things in
models regardless of whether or not they pass some statistical litmus
test (although one can always find referees to argue for pedantic procedure
over thinking).


Many problems can be approached using one of these methods but there
are also broad classes of problems that can't and, for those, you're
on your own. In later chapters we will address model selection in
specific contexts and we hope those will prove useful for a majority
of the situations you encounter.


\section{Poisson GLMs}
\label{glms.sec.poisson}

The Poisson GLM (also known as ``Poisson regression'') is probably the
most relevant and important class of models in all of ecology. The
basic model assumes observations $y_{i}; i=1,2,...,n$ follow a Poisson
distribution with mean $\lambda$ which we write
\[
 	y_{i} \sim \mbox{Poisson}(\lambda)
\]
Commonly $y_{i}$ is a count of animals or plants at some point in
space and lambda might depend on i. For example, $i$ might index point
count locations in a forest, BBS route centers, or sample quadrats, or
similar.  If covariates are available it is typical to model them as
linear effects on the log mean. If $x(i)$ is some measured covariate
associated with observation $i$. Then,
\[
 	log(x(i)) = \alpha  + \beta*x(i)
\]

While we only specify the mean of the Poisson model directly, the
Poisson model (and all GLMs) has a ``built-in'' variance which is
directly related to the mean. In this case, $\mbox{Var}(y) = \mbox{E}(y) =
\lambda$. Thus the model accommodates a linear increase in variance
with the mean.

\subsection{Important properties of the Poisson distribution}
\label{glms.sec.properties}

There are two properties of the Poisson distribution
that make it extremely useful in ecology. First
is the property of {\it compound additivity}. If $y_1$ and $y_2$ are
Poisson random variables with means $\lambda_1$ and $\lambda_2$,
then their sum $N=y_1+y_2$ is Poisson with mean $\lambda_1+\lambda_2$. Thus,
if the observations can be viewed as an aggregate of counts over some
finer unit of measurement, then the mean aggregates in a corresponding
manner.
Secondly, the Poisson distribution has a direct relationship to the multinomial.
If $y_1$ and $y_2$ are $iid$ Poisson then,
conditional on their sum $N = y_1 + y_2$, their joint distribution is multinomial
 with sample size $N$ and cell probabilities
$\lambda_1/(\lambda_1+\lambda_2)$ and
$\lambda_2/(\lambda_1+\lambda_2)$.  As a result of this, most
multinomial models can be analyzed as a Poisson GLM and {\it vice versa}.

\subsection{Example: Breeding Bird Survey Data}

As an example we consider a classical situation in ecology where
counts of an organism are made at a collection of spatial
locations. In this particular example, we have mourning dove counts
made along North American Breeding Bird Survey (BBS) routes in
Pennsylvania, USA. A route consists of 50 stops separated by 0.5
mile. For the purposes here we are defining $y_i$ = route total count
and he sample location will be marked by the center point of the BBS
route.  The survey is run annually and the data set we have is
1966-1998. BBS data can be obtained online at \mbox{\tt http:\//\//www.pwrc.usgs.gov\//bbs\//}.
We will make use of the whole data set shortly but for now we're going
to focus on a specific year of counts -- 1990 -- for the sake of
building a simple model.
 For 1990 there were 77 active routes. We have the data stored
in a \mbox{\tt .csv} file\footnote{check this data format} where rows index the unique route, column 1 is the
route ID, columns 2-3 are the route coordinates (longitude/latitude),
column 4 is a habitat covariate ``forest cover'' (standardized, see
below) and the remaining columns are the yearly counts. Years for
which a route was not run are coded as ``\mbox{\tt NA}'' in the data matrix. We
imagine that this will be a typical format for many ecological
studies, perhaps with more columns representing covariates.  To read
in the data and display the first few elements of this matrix, do
this:
{\small
\begin{verbatim}
> a<-read.csv("pa-bbsdovedata-all.csv")
> data[1:2,1:6]
      X     lon    lat    habitat X66 X67
1 72002 -80.445 41.501 -0.3871372  NA  24
2 72003 -80.347 41.214 -1.0171629  NA  NA
\end{verbatim}
}

It is useful to display the spatial pattern in the observed counts. For that we use a
spatial dot plot - where we plot the coordinates of the observations
and mark the color of the plotting symbol based on the magnitude of
the count.  We have a special plotting function for that which is
called \mbox{\tt spatial.plot()} and it is available with the
supplemental {\bf R} package.
Actually, what we want to do here is plot the
log-count (+1 of course) which (Fig. \ref{glms.fig.padovecounts}) displays a notable pattern that could
be related to something. The {\bf R} commands for obtaining this figure are:
{\small
\begin{verbatim}
data<-read.csv("pa-bbsdovedata-all.csv")
y<-data[,29]  # pick out 1990
notna<-!is.na(y)
y<-y[notna]
spatial.plot(data[notna,2:3],y)
\end{verbatim}
}
 We can ponder the potential effects that
might lead to dove counts being high....corn fields, telephone wires,
barn roofs along with misidentification of pigeons, these could all
correlated reasonably well with the observed count of mourning doves.
Unfortunately we don't have any of that information.

\begin{figure}
\begin{center}
\includegraphics[height=2.75in]{Ch2/figs/PA1}
\end{center}
\caption{Needs a caption}
\label{glms.fig.padovecounts}
\end{figure}

We do have a measure of forest cover in the vicinity of each point
which is contained in the data set (variable ``habitat''). This was derived
from a larger GIS coverage of the state (provided in the data file
``\mbox{\tt pahabdata.csv}'') which can be plotted using the \mbox{\tt spatial.plot} function
using the following commands
{\small
\begin{verbatim}
> map('state',regions="penn",lwd=2)
> spatial.plot(pahabdata[,2:3],pahabdata[,"dfor"],cx=2)
> map('state',regions="penn",lwd=2,add=TRUE)
\end{verbatim}
}
where the result appears in Fig. \ref{glms.fig.paforest}.
We see a prominent pattern that indicates high forest coverage in the
central part of the state and low forest cover in the SE.  Inspecting
the previous figure of log-counts suggests a relationship between
counts and forest cover which is perhaps not surprising.
\begin{figure}
\begin{center}
\includegraphics[height=2.75in]{Ch2/figs/PA2}
\end{center}
\caption{Needs a caption}
\label{glms.fig.paforest}
\end{figure}

\subsection{Doing it in WinBUGS}

Here we demonstrate how to fit a Poisson GLM in {\bf WinBUGS} using the
covariate $x_{i} =$ forest cover. It is advisable that $x_i$ be
standardized in most cases as this will improve mixing of the Markov
chains. Recall that the data we have stored include a standardized
covariate (forest cover) and so we don't have to worry about that
here.  To read the BBS data into {\bf R} and get things set up for
{\bf WinBUGS}
we issue the following commands:
{\small
\begin{verbatim}
data<-read.csv("pa-bbsdovedata-all.csv")
y<-data[,29]                        # pick out 1990
notna<-!is.na(y)
y<-y[notna]                         # discard missing
habitat<-data[notna,4]              # get habitat data
library("R2WinBUGS")                # load R2WinBUGS
data <- list ( "y","M","habitat")   # bundle data for WinBUGS
\end{verbatim}
}
Now we write out the Poisson model specification in {\bf WinBUGS}
pseudo-code, provide initial values, identify parameters to be
monitored and then execute {\bf WinBUGS}:
{\small
\begin{verbatim}
cat("
model {
    for (i in 1:M){
      y[i]~dpois(lam[i])
      log(lam[i])<- beta0+beta1*habitat[i]
     }
 beta0~dunif(-5,5)
 beta1~dunif(-5,5)
}
",file="PoissonGLM.txt")

inits <- function()  list ( beta0=rnorm(1),beta1=rnorm(1))
parameters <- c("beta0","beta1")
out<-bugs (data, inits, parameters, "PoissonGLM.txt", n.thin=2,n.chains=2,
                n.burnin=2000,n.iter=6000,debug=TRUE,working.dir=getwd())
\end{verbatim}
}
{\bf Remarks:} (1) Note the close correspondence in how the model is
specified here compared with the normal regression model
previously. As an exercise you should discuss the specific differences
between the {\bf BUGS} model specifications for the normal and Poisson
models.
{\small
\begin{verbatim}
> print(out,digits=3)
Inference for Bugs model at
``PoissonGLM.txt'', fit using WinBUGS,
 2 chains, each with 4000 iterations (first 1000 discarded), n.thin = 2
 n.sims = 3000 iterations saved
             mean     sd     2.5%      25%      50%      75%    97.5%  Rhat n.eff
beta0       3.151  0.025    3.102    3.135    3.151    3.168    3.199 1.001  2300
beta1      -0.498  0.021   -0.539   -0.512   -0.498   -0.484   -0.457 1.001  3000
fit       869.930 19.856  835.500  855.700  868.600  881.900  913.602 1.002  1600
fitnew     76.709 12.519   54.098   68.107   76.215   84.510  102.602 1.001  3000
deviance 1116.605  2.014 1115.000 1115.000 1116.000 1117.000 1122.000 1.001  3000
\end{verbatim}
}

We might wonder whether this model provides an adequate fit to our
data.  To evaluate that, we used a Bayesian p-value analysis with fit
statistic based on the Freeman-Tukey residual by replacing the model
specification above with this:
{\small
\begin{verbatim}
cat("
model {
    for (i in 1:M){
      y[i]~dpois(lam[i])
      log(lam[i])<- beta0+beta1*habitat[i]
      d[i]<-  pow(pow(y[i],0.5)-pow(lam[i],0.5),2)   #

      ynew[i]~dpois(lam[i])
      dnew[i]<-pow( pow(ynew[i],0.5)-pow(lam[i],0.5),2)

     }
 fit<-sum(d[])
 fitnew<-sum(dnew[])
 beta0~dunif(-5,5)
 beta1~dunif(-5,5)
}
",file="PoissonGLM.txt")
\end{verbatim}
}
The Bayesian p-value is the proportion of times $fitnew > fit$ which,
for this data set, is 0, which was 1.0 in this case (calculation
omitted). This suggests that the basic Poisson model does not fit
well.

\subsection{ Constructing your own MCMC algorithm}

At this point it might be helpful to suffer through an example
building a custom MCMC algorithm. Here, we develop an MCMC algorithm
for
the Poisson regression model, using a Metropolis-within-Gibbs sampling framework. Building MCMC algorithms is covered in more detail in Chapt. \ref{chapt.mcmc} where you can also find step-by-step instructions for Metropolis-within-Gibbs samplers, should the following section move through all this stuff too quickly.  

We will assume that the two parameters have diffuse
normal priors, say $[\alpha] = \mbox{Norm}(0,100)$ and
$[\beta]=\mbox{Norm}(0,100)$ where each has {\it standard deviation}
100 (recall that {\bf WinBUGS} parameterizes the normal in terms of $1/\sigma^{2}$).
We need to assemble the relevant elements of the model which are these
two prior distributions and the
likelihood $[{\bf y}|\alpha,\beta] = \prod_{i} [y_i|\alpha \beta] $ which is,
mathematically, the product of the Poisson pmf evaluated at each $y_i$,
given particular values of $\alpha$ and $\beta$.
Next, we need to identify the full conditionals
$[\alpha|\beta, {\bf y}]$ and $[\beta|\alpha,{\bf y}]$.  We use the all-purpose
rule for constructing full conditionals
(section \ref{glms.sec.rules})
 to discover that:
\[
 [\alpha|\beta,{\bf y}] \propto \left\{ \prod_{i} [y_{i}|\alpha,\beta]\right\}[\alpha]
\]
and
\[
 [\beta|\alpha,{\bf y}] \propto \left\{ \prod_{i}
   [y_{i}|\alpha,\beta]\right\} [\beta]
\]
Remember, we could replace the ``$\propto$'' with ``$=$'' if we
put $[y|\beta]$ or $[y|\alpha]$ in the denominator. But, in general,
$[y|\alpha]$ or $[y|\beta]$ will be quite a pain to compute and, more
importantly, it is a constant as far as the operative parameters
($\alpha$ or $\beta$,
respectively) are concerned. Therefore,
the MH acceptance probability will be the ratio of the
ful-conditional evaluated at a candidate draw to that evaluated at the
current draw, and so the denominator required to change $\propto$ to $=$
winds up canceling from the MH acceptance probability.
Here we will
use the so-called random walk candidate generator, which is a Normal proposal distribution, so that, for example,
 $\alpha^{*} \sim \mbox{Normal}(\alpha^{t},\delta)$ where $\delta$ is
 the standard-deviation of the proposal distribution, which is just a
 tuning parameter that is set by the user and adjusted to achieve efficient mixing of chains (see Section XX in Chapt. \ref{chapt.mcmc}) \footnote{
It would help
lots of people out to see a non-symmetric proposal distribution, and
the extra step needed to account for it. RS: We can include this in the MCMC chapter
}.
We remark also that calculations are often done on the log-scale to
preserve numerical integrity of things when quantities evaluate to
small or large numbers, so keep in mind, for example,
$a*b = exp(log(a) + log(b))$.
 The ``Metropolis within
Gibbs'' algorithm for a Poisson regression turns out to be  remarkably simple:
{\small
\begin{verbatim}
set.seed(2013)

out<-matrix(NA,nrow=1000,ncol=2)   # matrix to store the output
alpha<- -1                         # starting values
beta <- -.8

# begin the MCMC loop ; do 1000 iterations
for(i in 1:1000){

# update the alpha parameter
lambda<- exp(alpha+beta*habitat)
lik.curr<- sum(log(dpois(y,lambda)))
prior.curr<- log(dnorm(alpha,0,100))
alpha.cand<-rnorm(1,alpha,.05)         # generate candidate
lambda.cand<- exp(alpha.cand + beta*habitat)
lik.cand<- sum(log(dpois(y,lambda.cand)))
prior.cand<- log(dnorm(alpha.cand,0,100))
mhratio<- exp(lik.cand +prior.cand - lik.curr-prior.curr)
if(runif(1)< mhratio)
     alpha<-alpha.cand

# update the beta parameter
lik.curr<- sum(log(dpois(y,exp(alpha+beta*habitat))))
prior.curr<- log(dnorm(beta,0,100))
beta.cand<-rnorm(1,beta,.25)
lambda.cand<- exp(alpha+beta.cand*habitat)
lik.cand<- sum(log(dpois(y,lambda.cand)))
prior.cand<- log(dnorm(beta.cand,0,100))
mhratio<- exp(lik.cand + prior.cand - lik.curr - prior.curr)
if(runif(1)< mhratio)
     beta<-beta.cand

out[i,]<-c(alpha,beta)             # save the current values
}


plot(out[,1],ylim=c(-1.5,3.3),type="l",lwd=2,ylab="parameter value",
     xlab="MCMC iteration")
lines(out[,2],lwd=2,col="red")
\end{verbatim}
}
{\bf XXX Andy I removed the bad tuning example and the respective exercise here and added it in Ch7 XXXX}
The first 300 iterations of the MCMC history of each parameter
are shown in Fig. \ref{glms.fig.poissonmcmc2}. These chains are
not very appealing but a couple of things are evident: 
We see
that the burn-in takes about 250 iterations and that after that chains seem to mix reasonably well, although this is not so clear given the scale of the y-axis.
We generated 10,000 posterior samples,
discarding the first 500 as burn-in, and the result is shown in
Fig. \ref{glms.fig.grassy}, this time seperate panels for each
parameter.
The ``grassy''
look of the MCMC history is diagnostic of Markov chains that are
well-mixing and we would generally be very satisfied with results that
look like this.

\begin{figure}
\begin{center}
\includegraphics[height=3in,width=4in]{Ch2/figs/poissonmcmc2}
\end{center}
\caption{Same as previous fig but with $\delta = 0.05$.}
\label{glms.fig.poissonmcmc2}
\end{figure}

\begin{figure}
\begin{center}
\includegraphics[height=4in,width=5in]{Ch2/figs/poissonmcmc3}
\end{center}
\caption{nice grassy mcmc output, longer run of previous with $\delta
  = 0.05$.}
\label{glms.fig.grassy}
\end{figure}

{\bf Remarks:} (1) We used a specific set of starting values for these
simulations. It should be clear that starting values closer to the
mass of the posterior distribution might cause burn-in to occur
faster. As an exercise, evaluate that.  
(2) For the flat normal prior distributions
here we could leave the prior contribution out of the full conditional
evaluation since it is locally constant, i.e., constant in the vicinity of
the posterior mass, and thus has no practical effect. Removing the
prior contribution from the MH acceptance probability is equivalent to
saying that the parameters have an improper uniform prior, i.e.,
$\alpha \sim \mbox{const}$, which is commonly used for mean parameters
in practice.
Note also that we have
used a different prior than in our {\bf WinBUGS} model specification
given previously. As an
exercise, evaluate whether this seems to affect the result.

\section{Poisson GLM with Random Effects}

What we will be doing in most of this book is dealing with random
effects in GLM-like models - similar to what
are usually referred to as generalized
linear mixed models (GLMMs). We provide a brief introduction by way of
example, extending our Poisson regression model to include a random effect.

ANDY STOPPED HERE

{\bf The Log-Normal mixture:} The classical situation involves a GLM
with a normally distributed random effect that is additive on the
linear predictor. For the Poisson case, we have:
\[
 	log(\lambda_{i}) = \alpha  + \beta x_{i} + \eta_{i}
\]
where $\eta_{i} \sim \mbox{Normal}(0,\sigma^{2})$.  A natural
alternative is to have multiplicative gamma-distributed noise,
$exp(\eta_{i}) \sim  \mbox{Gamma}(a,b)$ which would correspond to a
negative binomial kind of over-dispersion, implying a different
mean/variance relationship to the log-normal mixture (the interested
reader should work that out).   Choosing between such possibilities is
not a topic we will get into here because it doesn't seem possible to
provide general guidance on it.
For this model we carried-out a goodness-of-fit evaluation using the
Bayesian p-value based on a Pearson residual statistic. See also
\citep[][ch. 18]{kery:2010}
for an example involving a binomial mixed model\footnote{Kery has
  noticed that such tests probably have 0 power. Should use the
  marginal frequency of the data}.
 Anyhow, it is really amazingly simple
to express this model in {\bf WinBUGS} and have {\bf WinBUGS}  draw samples from the posterior distribution using the following code for the BBS dove counts:
{\small
\begin{verbatim}
data<-read.csv("pa-bbsdovedata-all.csv")
locs<-data[,2:3]
habitat<-data[,4]
y<-data[,29]     # grab year 1990
M<-length(y)

set.seed(2013)

cat("
model {
  for (i in 1:M){
     y[i]~dpois(lam[i])
     log(lam[i])<- alpha+ beta*habitat[i] + eta[i]
     frog[i]<-beta*habitat[i] + eta[i]
     eta[i] ~ dnorm(0,tau)
     d[i]<-  pow(pow(y[i],0.5)-pow(lam[i],0.5),2)

     ynew[i]~dpois(lam[i])
     dnew[i]<- pow(pow(ynew[i],0.5)-pow(lam[i],0.5),2)
   }
 fit<-sum(d[])
 fitnew<-sum(dnew[])

 alpha~dunif(-5,5)
 beta~dunif(-5,5)
 sigma~dunif(0,10)
 tau<-1/(sigma*sigma)
}

",file="model.txt")
data <- list ( "y","M","habitat")
inits <- function()
  list ( alpha=rnorm(1),beta=rnorm(1),sigma=runif(1,0,4))
parameters <- c("alpha","beta","sigma","tau","fit","fitnew")
library("R2WinBUGS")

out<-bugs (data, inits, parameters, "model.txt", n.thin=2,n.chains=2,
 n.burnin=1000,n.iter=5000,debug=TRUE)
\end{verbatim}
}
This produces the following posterior summary statistics:
{\small
\begin{verbatim}
> print(out,digits=2)
Inference for Bugs model at "model.txt", fit using WinBUGS,
 2 chains, each with 5000 iterations (first 1000 discarded), n.thin = 2
 n.sims = 4000 iterations saved
           mean    sd   2.5%    25%    50%    75%  97.5% Rhat n.eff
alpha      2.98  0.08   2.82   2.93   2.98   3.03   3.12 1.00  1400
beta      -0.53  0.07  -0.68  -0.58  -0.53  -0.49  -0.38 1.01   350
sigma      0.60  0.06   0.49   0.56   0.59   0.64   0.73 1.00  2000
tau        2.88  0.57   1.88   2.47   2.86   3.24   4.12 1.00  2000
fit       26.58  3.72  19.87  23.96  26.37  29.01  34.46 1.00  4000
fitnew    26.83  3.90  19.60  24.12  26.68  29.36  35.04 1.00  4000
deviance 445.94 12.18 424.00 437.40 445.20 453.90 471.50 1.00  4000

[... some output deleted ...]
\end{verbatim}
}
The Bayesian p-value for this model is
\begin{verbatim}
> mean(out$sims.list$fit>out$sims.list$fitnew)
[1] 0.4815
\end{verbatim}
indicating a pretty good fit. Given the site-level random effect, it
would be surprising for this model to not fit! One thing we notice is
that the posterior standard deviations of the regression parameters
are much higher, a result of the excess variation. Wwe would also
notice much less precise predictions of hypothetical new
observations.


ANDY STOPPED HERE.




\section{Binomial GLMs}

Another extremely important class of models in ecology are
binomial models. We use binomial models for count data whenever the
observations are counts or frequencies and it is natural to condition
on a ``sample size'', say $K$, the maximum frequency possible in a sample.
 The random variable, $y \le K$, is then the
frequency of occurrences out of $K$ ``trials''. The parameter of the binomial
models is $p$, often called ``success probability'' which is related
to the expected value of $y$ by $E(y) = pK$. Usually we are interested
in modeling covariates that affect the parameter $p$, and such models
are called binomial GLMs , binomial
regression models or logistic regression, although logistic regression
 really only applies when the logistic link is used to model
the relationship between $p$ and covariates (see below).

One of the most typical binomial GLMs occurs when the sample size
equals 1 and the outcome, $y$, is ``presence'' ($y=1$) or ``absence''
($y=0$) of a species. This is a classical ``species distribution''
modeling situation. A special situation occurs when presence/absence
is observed with error \citep{mackenzie_etal:2002,tyre_etal:2003}.
In that case, $K>1$ samples
are usually needed for effective estimation of model parameters.

 In standard binomial regression problems the sample size
is fixed by design but interesting models also arise when the sample
size is itself a random variable. These are the $N$-mixture models
\citep{royle:2004, kery_etal:2005, royle_dorazio:2008, kery:2010}
and related models (in this case, $N$ being the sample size,
which we labeled $K$ above)\footnote{Some of the jargon is actually a little
bit confusing here
because the binomial index is customarily referred to as ``sample size''
but in the context of $N$-mixture models $N$ is actually the
``population size''}.
Another
situation in which the binomial sample size is ``fixed'' is closed
population capture-recapture models in which a population of
individuals is sampled $K$ times.  The number of times each individual
is encountered is a binomial outcome with parameter - encounter
probability -- $p$, based on a sample of size $K$.  In addition, the
total number of unique individuals observed, $n$, is also a binomial
random variable based onpopulation size $N$.  We consider such
models in the chapter \ref{chapt.closed}.


\subsection{Binomial regression}

In binomial models, covariates are modeled on a suitable
transformation (the link function) of the binomial success
probability, $p$.  Let $x_{i}$ denote some measured covariate for
sample unit $i$ and let $p_{i}$ be the success probability for unit $i$.
The standard choice is the ``logit'' link function which is:
\[
log(p_i/(1-p_i)) = \alpha + \beta*x_{i}.
\]
The inverse-logit (or ``expit'') is
\[
p_{i} = \mbox{expit}(\alpha + \beta*x_{i}) =
 \frac{ \exp(\alpha + \beta*x_{i})}
      {1 + \exp(\alpha + \beta*x_{i} ) }
\]
There are many other possible link functions. However, ecologists seem
to adopt the logit link function without question in most
applications\footnote{a notable exception is distance sampling, which
  is all about choosing among link functions}.  We sometimes use the
``complementary log-log'' (= ``cloglog'') link function in ecological
applications because it arises naturally in many situations
\citep[][p. 150]{royle_dorazio:2008}. For example, consider the
``probability of observing a count greater than 0'' under a Poisson
model: $\Pr(y>0) = 1-exp(- \lambda)$. In that case,
\[
cloglog(p) =log(- log(1-p)) = log(\lambda)
\]
So that if you have covariates in your linear predictor for $E(y)$
under a Poisson model then they are linear on the complementary
log-log link of $p$.
In models of species occurrence it seems natural to view occupancy as
being derived from local abundance $N$
\citep{royle_nichols:2003,royle_dorazio:2006,dorazio:2007}.
Therefore,
models of local abundance in which $N \sim \mbox{Poisson}(A \lambda)$
for a habitat patch of area $A$ implies a model for occupancy $\psi$
of the form
\[
 cloglog(\psi) = log(A) + log(\lambda).
\]
We will use the cloglog link in some analyses of
SCR models in chapter \ref{chapt.scr0} and elsewhere.


\subsection{ Example: Waterfowl Banding Data}

It would be easy to consider a standard ``distribution modeling''
application where $K=1$ and the outcome is occurrence ($y=1$) or not
($y=0$) of some species. Such examples abound in books (e.g.,
\citet[][ch. 3]{royle_dorazio:2008}; \citet[][ch. 21]{kery:2010};
\citet[][ch. 13]{kery_schaub:2011}) and in the literature.
Instead, we will
consider an example involving band returns of waterfowl which were
analyzed by \citet{royle_dubovsky:2001}\footnote{I hate this example.
  Anyone got a better one thats not distribution modeling?}.

For these data, $y_i$ is the number of waterfowl bands recovered out
of $B_i$ birds banded at some location ${\bf s}_{i}$. In this case $B_{i}$ is
fixed. Thinking about recovery rate as being proportional to harvest
rate, we use these data to explore geographic gradients in recovery rate
resulting from variability in harvest pressure experienced by
populations depending on their migration ecology. As such, we fit a
basic binomial GLM with a linear response to geographic coordinates
(including an interaction term). The data are provided with the {\bf
  R} package \mbox{\tt scrbook}. Here we
 provide the part of the script for creating the model and fitting the
 model in
{\bf WinBUGS} using the \mbox{\tt bugs} function.
There are few structural differences between this model and the
Poisson GLM fitted previously. The main things are due to the data
structure (we have a matrix here instead of a vector) and otherwise we
change the main distributional assumption to binomial (specified with
\mbox{\tt dbin}) and then use the \mbox{\tt logit} function to relate
the parameter $p_{it}$ to the covariates.  Here is the script:

{\small
\begin{verbatim}
load("mallarddata")  # not sure how this will look

sink("model.txt")
cat("
model {
 for(t in 1:5){
    for (i in 1:nobs){
       y[i,t] ~ dbin(p[i,t], B[i,t])
       logit(p[i,t]) <- alpha0[t] + alpha1*X[i,1] + alpha2*X[i,2] + alpha3*X[i,1]*X[i,2]
     }
}
	alpha1~dnorm(0,.001)
	alpha2~dnorm(0,.001)
	alpha3~dnorm(0,.001)
	for(t in 1:5){
 	alpha0[t] ~ dnorm(0,.001)
 }
}
",fill=TRUE)
sink()

data  <- list(B=mallard.bandings, y=mallard.recoveries,
             nobs=nrow(banding.locs),X=banding.locs)
inits <- function(){
      list(alpha0=rnorm(5),alpha1=0,alpha2=0,alpha3=0) }
parms <- list('alpha0','alpha1','alpha2','alpha3')
out   <- bugs(data,inits, parms,"model.txt",n.chains=3,
 	n.iter=2000,n.burnin=1000, n.thin=2,debug=TRUE)
\end{verbatim}
}


Posterior summaries of model parameters are as follows:
{\small
\begin{verbatim}
> print(out,digits=3)
Inference for Bugs model at "model.txt", fit using WinBUGS,
 3 chains, each with 2000 iterations (first 1000 discarded), n.thin = 2
 n.sims = 1500 iterations saved
              mean    sd     2.5%      25%      50%      75%    97.5%  Rhat n.eff
alpha0[1]   -2.346 0.036   -2.417   -2.370   -2.346   -2.323   -2.277 1.001  1500
alpha0[2]   -2.356 0.032   -2.420   -2.379   -2.356   -2.335   -2.292 1.001  1500
alpha0[3]   -2.220 0.035   -2.291   -2.244   -2.219   -2.197   -2.153 1.001  1500
alpha0[4]   -2.144 0.039   -2.225   -2.169   -2.143   -2.116   -2.068 1.000  1500
alpha0[5]   -1.925 0.034   -1.990   -1.949   -1.924   -1.901   -1.856 1.004   570
alpha1      -0.023 0.003   -0.028   -0.025   -0.023   -0.022   -0.018 1.001  1500
alpha2       0.020 0.006    0.009    0.016    0.020    0.024    0.031 1.001  1500
alpha3       0.000 0.001   -0.002   -0.001    0.000    0.000    0.002 1.001  1500
deviance  1716.001 4.091 1710.000 1713.000 1715.000 1718.000 1726.000 1.001  1500

[... some output deleted ...]
\end{verbatim}
}

The basic result suggests a negative east-west gradient and a positive
south to north gradient but no interaction. A map of the response
surface is shown in Fig. \ref{glms.fig.bandrecovery}.
 We did an additional MCMC run where we saved the binomial
parameter $p$ and computed the Bayesian p-value (double use of ``p''
here is confusing, but I guess that happens sometimes!)
using a fit statistic based on the Freeman-Tukey
statistic (see Section XXX above). The result indicates that the
linear response surface model does not provide an adequate fit of the
data. The reader should contemplate whether this invalidates the basic
interpretation of the result.


\begin{figure}
\begin{center}
\includegraphics[height=2.75in]{Ch2/figs/responsesurface}
\end{center}
\caption{Predicted recovery rate of bands.}
\label{glms.fig.bandrecovery}
\end{figure}

\section{ Summary and Outlook}

GLMs and GLMMs are the most useful statistical methods in all of
ecology. The principles and procedures underlying these methods are
relevant to nearly all modeling and analysis problems in every branch
of ecology. Moreover, understanding how to analyze these models is
crucial in a huge number of diverse problems. If you understand and
can conduct classical likelihood and Bayesian analysis of Poisson and
binomial GLM(M)s, then you will be successful analyzing and
understanding more complex classes of models that arise. We will see
shortly that spatial capture-recapture models are a type of GLMM
and thus having a basic
understanding of the conceptual origins and formulation of GLM(M)s and
their analysis is extremely useful.

We note that GLM(M)s are routinely
analyzed by likelihood methods but we have focused on Bayesian
analysis here in order to develop the tools that are less familiar to
most ecologists.  In particular, Bayesian analysis of models with random
effects is relatively straightforward because the models
are easy to analyze conditional on the random effect, using methods of
MCMC.  Thus, we will often analyze SCR models in later chapters by
MCMC, explicitly adopting a Bayesian inference framework.
In that regard, the various {\bf BUGS} engines ({\bf WinBUGS}, {\bf
  OpenBUGS}, {\bf JAGS}) are enormously useful because they
provide an accessible platform for
carrying out  analyses by MCMC by just
describing the model, and not having to worry about how to actually
build MCMC algorithms.  That said, the {\bf BUGS} language is more important
than just to the extent that it enables one to do MCMC - it is useful
as a modeling tool because it fosters understanding, in the sense that
it forces you to become intimate with your model. You have to write
down all of the probability assumptions, the relationships between
observations and latent variables and parameters. This is really a
great learning paradigm that you can grow with.

While we have emphasized Bayesian analysis in this chapter, and make
primary use of it through the book, we
we will provide an introduction to likelihood analysis in chapter
\ref{chapt.mle} and use those  methods also from time to time.
 Before getting to that, however, it will be useful to
talk about more basic, conventional closed population
capture-recapture models and these are the topic of the next chapter.


\chapter{Closed population models}
\label{chapt.closed}

\chapter{
 Closed Population Models
}
\markboth{Chapter 3}{}
\label{chapt.closed}

\vspace{.3in}
%%Andy, I really like connecting a new chapter to the previous ones with a few words, so I added this half sentence
Having covered the basics of hierarchical models and their implementation, in this chapter we will consider ordinary capture-recapture (CR)
models for estimating population size in closed populations. We will
see that such models are closely related to binomial (or logistic)
regression type models. In fact, when $N$ is known, they are precisely
such models.  We consider some important extensions of ordinary closed
population models that accommodate various types of ``individual
effects'' --- either in the form of explicit covariates (sex, age,
body mass) or unstructured ``heterogeneity'' in the form of an
individual random effect. In general, these models are variations of
generalized linear or generalized linear mixed models (GLMMs).
Because of the paramount importance of this concept, we focus mainly
on fairly simple models in which the observations are individual
encounter frequencies, $y_{i}$ = the number of encounters of
individual $i$ out of $K$ replicate samples of the population which,
for the models we consider here, is the outcome of a binomial random
variable.  Along the way, we consider the spatial context of
capture-recapture data and models and demonstrate that density cannot
be formally estimated when spatial information is ignored. We also
review some of the informal methods of estimating density using CR
methods, and consider some of their limitations.  We will be exposed
to our first primitive spatial capture-recapture models which arise as
relatively minor variations of so-called ``individual covariate
models'' (of the \citet{huggins:1989} and \citet{alho:1990}
variety). In a sense, the point of this chapter is to establish that
linkage XX between non-spatial and spatial capture-recapture models XXX in a direct and concise manner beginning with the basic
``Model $M_0$'' and extensions of that model to include individual
heterogeneity and also individual covariates. A special type of
individual covariate models is distance sampling, which could be
thought of as the most primitive spatial capture-recapture model.  In
later chapters we further develop and extend ideas introduced in this
chapter.

We emphasize Bayesian analysis of capture-recapture models and we
accomplish this using a method related to classical ``data
augmentation'' from the statistics literature XXX SOMETHING WRONG WITH BRACKETS IN REF XXX
\citet[e.g.,][]{tanner_wong:1987}).  This is a general concept in
statistics but, in the context of capture-recapture models where $N$
is unknown, it has a consistent implementation across classes of
capture-recapture models and one that is really convenient from the
standpoint of doing MCMC \citep{royle_etal:2007}. We use data
augmentation throughout this book and thus emphasize its conceptual
and technical origins and demonstrate applications to closed
population models.  We refer the reader to
\citet[][ch. 6]{kery_schaub:2011} for an accessible and complimentary
development of ordinary closed population models.


\section{The Simplest Closed Population Model: Model $M_0$}

To start looking at the simples capture-recapture model, let's suppose that there exists a population of $N$ individuals which we
subject to repeated sampling, say over $K$ nights, where individuals
are captured, marked, and subsequently recaptured.  We suppose that
individual encounter histories are obtained, and these are of the form
of a sequence of 0's and 1's indicating capture $(y=1)$ or not $(y=0)$
during any sampling occasion (``sample'').  As an example, suppose
$K=5$ sampling occasions, then an individual captured during sample 2
and 3 but not otherwise would have an encounter history of the form
${\bf y}=(0,1,1,0,0)$. Thus, the observation ${\bf y}_{i}$ for each
individual $(i)$ is a vector having elements denoted by $y_{ik}$ for
$k=1,2,..,K$. Usually this is organized as a row of a matrix with
elements $y_{ik}$, see Table \ref{tab.3.1}.  Except where noted
explicitly, we suppose that observations are independent within
individuals and among individuals.  Formally, this allows us to say
that $y_{ik}$ are $iid$ Bernoulli random variables and we may write $y_{ik}
\sim \mbox{Bern}(p)$.  Consequently, for this very simple model in
which $p$ is in fact constant, then we can declare that the individual
encounter frequencies (total captures), $y_{i} = \sum_{k} y_{ik}$,
have a binomial distribution based on a sample of size $K$. That is
\[
y_{i}  = \sum_{k} y_{ik} \sim \mbox{Bin}(p,K)
\]
for every individual in the population. This is a remarkably simple
model that forms the cornerstone of almost all of classical
capture-recapture models, including most spatial capture-recapture
models discussed throughout this book.  

Evidently, the basic
capture-recapture model structure is precisely a simplistic version of
a logistic-regression model with only an intercept term
($\mbox{logit}(p) = \mbox{constant}$).  To say that all
capture-recapture models are just logistic regressions is only
slightly inaccurate. In fact, we are proceeding here ``conditional on
$N$'', i.e., as if we knew $N$. In practice we don't, of course, and
that is kind of the point of capture-recapture models as estimating
$N$ is the central objective. But, by proceeding conditional on $N$,
we can specify a simple model and then deal with the fact that $N$ is
unknown using standard methods that you are already familiar with
(i.e., GLMs - see chapter 2).
\begin{table}
\centering
\caption{a capture-recapture data set with $n=6$ observed individuals
and $K=5$ samples.}
\begin{tabular}{r|ccccc|c}
&  \multicolumn{5}{c}{Sample occasion} &  \\ \hline
 indiv $i$ &  1 & 2 & 3 & 4 & 5 & $y_{i}$ \\ \hline
  1 &     1 & 0 & 0 & 1 & 0  & 2   \\
  2 &     0 & 1 & 0 & 0 & 1  & 2   \\
  3 &     1 & 0 & 0 & 1 & 0  & 2   \\
  4 &     1 & 0 & 1 & 0 & 1  & 3   \\
  5 &     0 & 1 & 0 & 0 & 0  & 1   \\
  $n=6$ & 1 & 0 & 0 & 0 & 0  & 1   \\ \hline
\end{tabular}
\label{tab.3.1}
\end{table}

Assuming individuals of the population are observed independently, the
joint probability distribution of the observations is the product of
$N$ binomials
\begin{eqnarray*}
  \Pr(y_1, \ldots, y_N | p) &=& \prod_{i=1}^N  \mathrm{Bin}(y_i | K, p) \\
   &=& \prod_{k=0}^K  \pi(k)^{n_k}
\end{eqnarray*}
where $\pi(k) = \mathrm{Bin}(k | K,p)$ and where $n_k = \sum_{i=1}^N
I(y_i = k)$ denotes the number of individuals captured $k$ times in
$K$ surveys. We emphasize that this is conditional on $N$, in which
case we get to observe the $y=0$ observations and the resulting data
are just $iid$ binomial counts. Because this is a binomial regression
model of the variety described in Chapt. \ref{glms}, fitting this model using
a {\bf BUGS} engine poses no difficulty.

The essential problem in capture-recapture, however, is that $N$ is
not known because the number of uncaptured/missing individuals (i.e.,
those in the zero cell that occur with probability $\pi(0)$) is
unknown.  Consequently, the observed capture frequencies $n_k$ are no
longer independent. Instead, their joint distribution is multinomial
(e.g., see \citet[][p. xyz]{illian_etal:2008}):
\begin{equation}
    n_1, n_2, \ldots, n_K \sim \mathrm{Multin}(N, \pi(1), \pi(2), \ldots, \pi(K))
\label{closed.eq.multinomial4m0}
\end{equation}
Note that in our notation the number of uncaptured/missing individuals is
denoted by $n_0 = N - n$, where $n = \sum_{k=1}^K n_k$ denotes the total
number of distinct individuals seen in the $K$ samples.
XXX ANDY; MAYBE IT MIGHT BE WORTH MENTIONING WHY THE n0 DOESNT SHOW UP IN THE MULTINOMIAL XXXXX

To fit the model in which $N$ is {\it unknown}, we can regard $N$ as a
parameter and maximize the multinomial likelihood directly.  While
direct likelihood analysis of the multinomial model is
straightforward, that does not prove to be too useful in practice
because we seldom are concerned with models for the aggregated
encounter history frequencies, XXX which entail that capture probabilities are the same for all individuals XXX. In many instances, including for
spatial capture-recapture (SCR) models, we require a formulation of
the model that can accommodate individual level covariates XXX to account for differences in detection among individuals XXX which we
address subsequently in this chapter.


\subsection{The Spatial Context of Capture-Recapture}

XXX I WOULD CHANGE THE SECTION HEADING TO SOMETHING LIKE 'POPULATION CLOSURE AND THE SPATIAL CONTEXT OF CAPTURE-RECAPTURE XXX
A common assumption made is that of population ``closure'' which is
really just a colloquial way of saying (in part) the Bernoulli
assumptions stated explicitly above. In the biological context,
closure means, strictly, no additions or subtractions from the
population during study. This is manifest by the statement that the
encounters are independent and identically distributed (iid) Bernoulli
trials.  In practice, closure is usually interpreted by the manner in
which potential violations of that assumption arise. In particular,
two important elements of the closure assumption are ``demographic''
and ``geographic'' closure. If an individual dies then subsequent
values of $y_{ik}$ are clearly no longer Bernoulli trials with the
same parameter $p$; XXX since the probability of capturing that individual becomes 0 XXX. If there is no mortality or recruitment in the
population, then we say that demographic closure is
satisfied. Similarly, animals may emigrate or immigrate. If they do
not, then geographic closure is satisfied. Sometimes a distinction is
made between temporary and permanent emigration or immigration. That
is a relevant distinction in spatial capture-recapture models, because
SCR models explicitly accommodate ``temporary emigration'' of a
certain type, due to individuals moving about their home range. XXX In contrast, ordinary capture-recapture models cannot explicitly deal with the fact that, unless we're sampling a fenced enclosure or an island, individuals are bound to move off the trapping grid. XXX The
demographic closure assumption can also be relaxed using SCR models,
but we will save that discussion for Chapt. \ref{chapt.scr0}.
XXXX I FEEL LIKE THIS SECTION STILL NEEDS A SENTENCE THAT MAKES THE POINT - SPATIAL CONTEXT; POP CLOSURE AND SCR; BUT I AM HAVING TROUBLE PUTTING THAT INTO A FEW WORDS RIGHT NOW XXXX

\subsection{Conditional likelihood}

We saw that a basic closed population model is a simple logistic
regression model if $N$ is known and, when $N$ is unknown, the model
is multinomial with index or sample size parameter $N$. This
multinomial model, being conditional on $N$, is sometimes referred to
as the ``joint likelihood'' the ``full likelihood'' or the
``unconditional likelihood'' (or model in place of likelihood). This
formulation differs from the so-called ``conditional likelihood''
approach in which the likelihood of the observed encounter histories
is devised conditional on the event that an individual is captured at
least once.  To construct this likelihood, we have to recognize that
individuals appear or not in the sample based on the value of the
random variable $y_{i}$, that is, we capture them if and only if
$y_{i}>0$.  The observation model is therefore based on $\Pr(y|y>0)$.
For the simple case of Model $M_0$, the resulting conditional
distribution is a ``zero truncated'' binomial distribution which
accounts for the fact that we cannot observe the value $y=0$ in the
data set \citep[see][sec. 5.1]{royle_dorazio:2008}.  Both the
conditional and unconditional models are legitimate modes of analysis
in all capture-recapture types of studies, and they provide equally
valid descriptions of the data and for many practical purposes provide
equivalent inferences, at least in large sample sizes
\citep{sanathanan:1972}.

In this book we emphasize Bayesian analysis of capture-recapture
models using data augmentation (discussed subsequently), which
produces yet a third distinct formulation of capture recapture-models
based on the zero-{\it inflated} binomial distribution that we
describe in the next section.  Thus, there are 3 distinct formulations
of the model -- or modes of analysis -- for analyzing all
capture-recapture models based on the (1) binomial model for the joint
or unconditional specification; (2) zero-truncated binomial that
arises ``conditional on $n$''; and (3) the zero-inflated binomial that
arises under data augmentation.  Each formulation has a distinct
complement of model parameters (shown in Table \ref{tab.3.modes} for
Model $M_0$).


\begin{table}
\centering
\caption{Modes of analysis of capture-recapture models. Closed
  population models can be analyzed using the joint or ``full
  likelihood'' which contains $N$ as an explicit parameter, the
  conditional likeilhood which does not involve $N$, or by data
  augmentation which replaces $N$ with $\psi$. Each approach yields a
  distinct likelihood.}
\begin{tabular}{ccc}
Mode of analysis & parameters in model & statistical model \\ \hline
Joint likelihood                &	$p$, $N$	&	multinomial with index $N$\\
Conditional likelihood 		&	$p$	&	zero-truncated binomial \\
Data augmentation		&	$p$, $\psi$	&	zero-inflated binomial\\
\end{tabular}
\label{tab.3.modes}
\end{table}



\section{ Data Augmentation }
\label{closed.sec.da}

We consider a method of analyzing closed population models using data
augmentation (DA) which is useful for Bayesian analysis and, in
particular, analysis of models using the various BUGS engines and
other software.  Data augmentation is a general statistical concept
that is widely used in statistics in many different settings. The
classical reference is \citet{tanner_wong:1987} but see also
\citet{liu_wu:1999}.  Data augmentation can be adapted to provide a
very generic framework for Bayesian analysis of capture-recapture
models with unknown $N$. This idea was introduced for closed
populations by \citet{royle_etal:2007}, and has subsequently been
applied to a number of different contexts including individual
covariate models \citep{royle:2009}, open population models
\citep{royle_dorazio:2008,royle_dorazio:2010, gardner_etal:2010ecol},
spatial capture-recapture models \citep{royle_young:2008,
  royle_etal:2010, gardner_etal:2009}, and many
others. \citet[][Chapt. 6]{kery_schaub:2011} provides a good introduction to data
augmentation in the context of closed population models. 


Conceptually, data augmentation is a reparameterization of the
``complete data'' model -- that which is conditional on $N$. The
reparameterization is achieved by embedding this data set into a
larger data set having $M> N$ ``rows'' (individuals) and reexpressing
the model conditional on $M$ instead of $N$. XXX The great thing about data augmentation is that we do not need to know $N$ for this reparameterization. XXX Although this has a whiff of
arbitrariness or even ad hockery to it in the choice of $M$, 
it is always possible, in practice, to choose $M$ pretty easily for
a given problem and context and results will be insensitive to choice
of $M$\footnote{Unless the data set is sufficiently small that parameters are
weakly
identified}.
Then, under data augmentation, analysis
 is focused on the ``augmented data set.'' That is, we analyze the bigger
 data set - the one having $M$ rows - with an appropriate model that
 accounts for the augmentation. Inference is focused directly on
 estimating the proportion $\psi = E[N]/M$, instead of directly on $N$,
 where $\psi$ is the ``data augmentation parameter.''


\subsection{DA links occupancy models and closed population models}

%We provide a heuristic description of data augmentation based on the
XXX There is a XXX close correspondence between so-called ``occupancy'' models and closed
population models following \citet[][sec. 5.6]{royle_dorazio:2008}.

In occupancy models \citep{mackenzie_etal:2002, tyre_etal:2003} the
sampling situation is that $M$ sites, or patches, are sampled multiple
times to assess whether a species occurs at each site.  This yields
encounter data such as that illustrated in the left panel of Table
\ref{closed.tab.occ}. The important problem is that a species may occur at
a site, but go undetected, yielding the ``all-zero'' encounter
histories which are observed. However, some of the all-zeros may well
correspond to sites where the species in fact {\it does}
occur. Thus, while the zeros are observed, there are too many of them
and, in a sense, the inference problem is to allocate the zeros into
``structural'' (fixed) and ``sampling'' (or stochastic) zeros. More
formally, inference is focused on the parameter $\psi$, the
probability that a site is occupied.  In contrast, in classical closed
population studies, we observe a data set as in the middle panel of
Table \ref{closed.tab.occ} where {\it no} zeros are observed. The inference
problem is, essentially, to estimate how many sampling zeros there are
- or should be - in a ``complete'' data set. This objective
(how many sampling zeros?) is precisely the same for both types of
problems if an upper limit $M$ is specified for the closed population
model. The only distinction being that, in occupancy models, $M$ is
set by design (i.e., the number of sites to visit) whereas a natural
choice of $M$ for capture-recapture models may not be
obvious. However, by assuming a uniform prior for $N$ on the integers
$[0,M]$, this upper bound is induced \citep{royle_etal:2007}. Then,
one can analyze capture-recapture models by adding $M-n$ all-zero
encounter histories to the data set and regarding the augmented data
set, essentially, as a site-occupancy data set.

Thus, the heuristic motivation of data augmentation is to fix the size
of the data set by adding {\it too many} all-zero encounter histories
to create the data set shown in the right panel of Table
\ref{closed.tab.occ} - and then analyze the augmented data set using an
occupancy type model which includes both ``unoccupied sites'' as well
as ``occupied sites'' at which detections did not occur. We call these
$M-n$ all-zero histories ``potential individuals'' because they exist
to be recruited (in a non-biological sense) into the population, for
example during an analysis by MCMC.

To analyze the augmented data set, we recognize that it is a
zero-inflated version of the known-$N$ data set. That is, some of the
augmented all-zeros are sampling zeros (corresponding to actual
individuals that were missed) and some are ``structural'' zeros, which
do not correspond to individuals in the population. For a basic
closed-population model, the resulting likelihood under data
augmentation - that is, for the data set of size $M$ -- is a simple
zero-inflated binomial likelihood.  The zero-inflated binomial model
can be described ``hierarchically'', by introducing a set of binary
latent variables, $z_{1},z_{2},\ldots, z_{M}$, to indicate whether
each individual $i$ is ($z_i=1$) or is not ($z_i=0$) a member of the
population of $N$ individuals exposed to sampling. We assume that
$z_{i} \sim \mbox{Bern}(\psi)$ where $\psi$ is the probability that an
individual in the data set of size $M$ is a member of the sampled
population - in the sense that $1-\psi$ is the probability of
realizing a ``structural zero'' in the augmented data set.  The
zero-inflated binomial model which arises under data augmentation can
be formally expressed by the following set of assumptions:

\begin{eqnarray*}
 y_{i}|{z_{i}=1} & \sim  &\mbox{Bin}(K, p) \\
 y_{i}|{z_{i}=0} & \sim &  \delta(0)  \\
 z_{i} & \stackrel{iid}{\sim} & \mbox{Bern}(\psi) \\
 \psi & \sim & \mathrm{Unif}(0,1) \\
 p & \sim & \mathrm{Unif}(0,1)
\end{eqnarray*}
for $i=1, \ldots, M$, where $\delta(0)$ is a point mass at $y=0$.

Note that, under data augmentation, 
$N$ is no longer an explicit parameter of this
model. Instead, we estimate $\psi$ and functions of the latent
variables. In particular, under the assumptions of the zero-inflated
model, $z_{i} \stackrel{iid}{\sim} \mbox{Bern}(\psi)$; therefore, $N$
is a function of these latent variables:
 \[
 N = \sum_{i=1}^{M} z_{i}.
\]
Further, we note that the latent $z_i$ parameters can be removed from
the model by integration, in which case the joint probability of the
data is
\begin{equation}
  \Pr(y_1, \ldots, y_M | p, \psi) = \prod_{i=1}^M  \psi \mathrm{Bin}(y_i | K, p) +  I(y_i=0) (1-\psi)
\end{equation}
Which can be maximized directly to obtain the MLEs of the structural
parameters $\psi$ and $p$ or those of other more complex models
\citep[e.g., see][]{royle:2006}. We could estimate these parameters
and then use them to obtain an estimator of $N$ using the so-called
``Best unbiased predictor'' \citep[see][]{royle_dorazio:2011}.

\begin{table}
\centering
\caption{Hypothetical occupancy data set (left), capture-recapture data
 in standard form (center), and capture-recapture data augmented with
 all-zero capture histories (right). }
\begin{tabular}{cccc|cccc|cccc}
\hline
\multicolumn{4}{c}{Occupancy data}    &
\multicolumn{4}{c}{Capture-recapture} &
\multicolumn{4}{c}{Augmented C-R}     \\ \hline
site    & k=1 & k=2 & k=3 & ind & k=1 &k=2  & k=3 & ind & k=1 & k=2 & k=3           \\ \hline
1  & 0   & 1   & 0   & 1   & 0   & 1  & 0   & 1   & 0   & 1   & 0                   \\
2  & 1   & 0   & 1   & 2   & 1   & 0 & 1    & 2 & 1 & 0 & 1 \\
3  & 0   & 1   & 0   & .   & 0   & 1 & 0    & 3 & 1 & 0 & 1 \\
4  & 1   & 0   & 1   & .   & 1   & 0 & 1    & 4 & 1 & 0 & 1 \\
5  & 0   & 1   & 1   & .   & 0   & 1 & 1    & 5 & 1 & 0 & 1 \\
.  & 0   & 1   & 1   & .   & 0   & 1 & 1    & . & 0 & 1 & 1 \\
.  & 1   & 1   & 1   & .   & 1   & 1 & 1    & . & 0 & 1 & 1 \\
.  & 1   & 1   & 1   & .   & 1   & 1 & 1    & . & 1 & 1 & 1 \\
   & 1   & 1   & 1   & .   & 1   & 1 & 1    & . & 1 & 1 & 1 \\
n  & 1   & 1   & 1   & n   & 1   & 1 & 1    & n & 1 & 1 & 1 \\
.  & 0   & 0   & 0   &     &     &   &      & . & 0 & 0 & 0 \\
.  & 0   & 0   & 0   &     &     &   &      & . & 0 & 0 & 0 \\
   & 0   & 0   & 0   &     &     &   &      &   & 0 & 0 & 0 \\
   & 0   & 0   & 0   &     &     &   &      &   & 0 & 0 & 0 \\
   & 0   & 0   & 0   &     &     &   &      &   & 0 & 0 & 0 \\
   & 0   & 0   & 0   &     &     &   &      & N & 0 & 0 & 0 \\
.  & 0   & 0   & 0   &     &     &   &      & . & 0 & 0 & 0 \\
.  & 0   & 0   & 0   &     &     &   &      &   & 0 & 0 & 0 \\
M  & 0   & 0   & 0   &     &     &   &      & . & 0 & 0 & 0 \\
   &     &     &     &     &     &   &      & . & . & . & . \\
   &     &     &     &     &     &   &      & . & . & . & . \\
   &     &     &     &     &     &   &      & . & . & . & . \\
   &     &     &     &     &     &   &      & M & 0 & 0 & 0 \\
\end{tabular}
\label{closed.tab.occ}
\end{table}


\subsection{Model $M_0$ in BUGS}

For model $M_0$ in which we can aggregate the encounter data to
individual-specific encounter frequencies, the augmented data are
given by the vector of frequencies $(y_{1}, \ldots, y_{n}, 0, 0,
\ldots, 0)$. The zero-inflated model of the augmented data combines
the model of the latent variables, $z_{i} \sim \mbox{Bern}(\psi)$ with
the conditional-on-$z$ binomial model:
\begin{eqnarray*}
y_{i}|z_{i} = 1   &\sim& \mbox{Bin}(K,p) \\
y_{i} | z_{i} = 0 &\sim& \delta(0) 
\end{eqnarray*}
It is convenient to express the conditional-on-$z$ observation model concisely as:
\[
 y_{i}|z_{i} \sim \mbox{Bin}(K, p z_{i})
\]
Thus, if $z_{i}=0$ then the success probability of the binomial
distribution is identically 0 whereas, if $z_{i}=1$, then the success
probability is $p$. This is useful in describing the model in the {\bf
  BUGS}
language, as shown in Panel \ref{closed.panel.da4m0}.
 Note the last line of the model
specification  provides the expression for computing $N$ from the
data augmentation variables $z_{i}$.

\begin{panel}[htp]
\centering
\rule[0.15in]{\textwidth}{.03in}
%\begin{minipage}{5in}
{\small
\begin{verbatim}
model{
p  ~ dunif(0,1)
psi~dunif(0,1)

# nind = number of individuals captured at least once
#   nz = number of uncaptured individuals added for PX-DA
for(i in 1:(nind+nz)) {
    z[i]~dbern(psi)
   mu[i]<-z[i]*p
    y[i]~dbin(mu[i],K)
 }

N<-sum(z[1:(nind+nz)])
}
\end{verbatim}
}
%\end{minipage}
\rule[-0.15in]{\textwidth}{.03in}
\caption{Model $M_{0}$ under data augmentation.}
\label{closed.panel.da4m0}
\end{panel}




Specification of a more general model in terms of the individual
encounter observations $y_{ik}$ is not much more difficult than for
the individual encounter frequencies.  We define the
observation model by a double loop and change the indexing of things
accordingly, i.e.,
\begin{verbatim}
for(i in 1:(nind+nz)) {
    z[i]~dbern(psi)
  for(k in 1:K){
      mu[i,k]<-z[i]*p
      y[i,k]~dbin(mu[i,k],1)
  }
}
\end{verbatim}
In this manner, it is straightforward to incorporate covariates on $p$ XXX for both individuals and sampling occasions XXX
(see discussion of this below and also Chapt. \ref{chapt.covariates} 
and consider other extensions.

\subsection{Formal development of data augmentation}

Use of DA for solving inference problems with unknown $N$ can be
justified as originating from the choice of uniform prior on $N$.  The
$\mathrm{Unif}(0,M)$ prior for $N$ is innocuous in the sense that the
posterior associated with this prior is equal to the likelihood for
sufficiently large $M$.  One way of inducing the $\mathrm{Unif}(0,M)$
prior on $N$ is by assuming the following hierarchical prior:
\begin{eqnarray}
\label{closed.eq.NgivenM}
  N &\sim& \mathrm{Bin}(M, \psi) \\ \nonumber
  \psi &\sim& \mathrm{Unif}(0,1)
\end{eqnarray}
which includes a new model parameter $\psi$ XXX (note that we have seen $\psi$ in the previous section as the proportion $E[N]/M$).XXX This parameter denotes
the probability that an individual in the super-population of size $M$
is a member of the population of $N$ individuals exposed to sampling.
The model assumptions, specifically the multinomial model 
(Eq. \ref{closed.eq.multinomial4m0})
and Eq. \ref{closed.eq.NgivenM}, may be combined to yield a
reparameterization of the conventional model that is appropriate for
the augmented data set of known size $M$:
\begin{equation}
\label{closed.eq.multinomial4DA}
    (n_1, n_2, \ldots, n_K) \sim \mathrm{Multin}(M, \psi  \pi(1), \psi \pi(2), \ldots, \psi \pi(K))
\end{equation}
This arises by removing $N$ from Eq. \ref{closed.eq.multinomial4m0} by 
integrating
over the binomial prior distribution for $N$. Thus, the models we
analyze under data augmentation arise formally by removing the
parameter $N$ from the ordinary model - the model conditional on $N$ -
by integrating over a binomial prior distribution for $N$.

Note that the $M-n$ unobserved individuals in the augmented data set
have probability $\psi \pi(0) + (1-\psi)$, indicating that these
unobserved individuals are a mixture of individuals that are sampling
zeros ($\psi \pi_0$, and belong to the population of size $N$) and
others that are ``structural zeros'' (occurring in the augmented data
set with probability $1 - \psi$). In Eq.~\ref{closed.eq.multinomial4DA} $N$
has been eliminated as a formal parameter of the model by
marginalization (integration) and replaced with the new parameter
$\psi$, the data augmentation parameter.
However, the full likelihood containing both $N$ and $\psi$ can also be
analyzed \citep[see][]{royle_etal:2007}.


\subsection{Remarks on Data Augmentation}

Data augmentation may seem like a strange and mysterious black-box,
and likely it is unfamiliar to most people, even those with substantial
experience with capture-recapture models. However, it really is a
formal reparameterization of capture-recapture models in which $N$ is
removed from the ordinary (conditional-on-$N$) model by integration.
In the case of Model $M_0$, data augmentation produces the zero-inflated
binomial which is distinct from the original observation model, but
only in the sense that it embodies, explicitly, the $\mbox{Unif}(0,M)$
prior for $N$.  Choice of $M$ might be cause for some concern related
to potential sensitivity to choice of $M$. The guiding principle is
that it should be chosen large enough so that the posterior for $N$ is
not truncated, but no larger because large values entail more
computational burden. It seems likely that the properties of the
Markov chains should be affected by $M$ and so some optimality might
exist \citep{gopalaswamy_etal:2012}, as in occupancy models
\citep{mackenzie_royle:2005}. Formal analysis of this is needed.


We emphasize the motivation for data augmentation being that it
produces a data set of fixed size, so that the parameter dimension in
any capture-recapture model is also fixed.  As a result, MCMC is a
relatively simple proposition using standard Gibbs Sampling.  Consider
the simplest context - analyzing Model $M_0$ using the occupancy type
model. In this case, DA converts Model $M_0$ to a basic occupancy model
and the parameters $p$ and $\psi$ have known full-conditional
distributions (in fact, beta distributions) that can be sampled from
directly.  Furthermore, the data augmentation variables - i.e., the 
data augmentation variables $z$, can be sampled from Bernoulli full
conditionals. MCMC is not too much more difficult for complicated
models - sometimes the hyperparameters need to be sampled using a
Metropolis-Hastings step, but nothing more sophisticated than that is
required.

There are other approaches to analyzing models with unknown $N$, using
reversible jump MCMC (RJMCMC) or other so-called ``trans-dimensional''
(TD) algorithms
 \citep{durban_elston:2005, king_brooks:2001, king_etal:2008,
schofield_barker:2008, wright_etal:2009}. What distinguishes DA from RJMCMC and
related TD methods is that DA is used to create a distinctly new model
that is unconditional on $N$ and we (usually) analyze the
unconditional model. The various TD/RJMCMC approaches seek to analyze
the conditional-on-$N$ model in which the dimensional of the parameter
space is a variable function of $N$. TD/RJMCMC approaches might appear
to have the advantage that one can model $N$ explicitly or consider
alternative priors for $N$. However, despite that $N$ is removed as an
explicit parameter in DA, it is possible to develop hierarchical
models that involve structure on $N$ \citep{converse_royle:2010,
  royle_etal:2011ms} which we consider in Chapt. \ref{chapt.hscr}.

\subsection{Example: Black Bear Study on Fort Drum}

To illustrate the analysis of Model $M_0$ using data augmentation, we use
a data set collected at Fort Drum Military Installation in upstate New
York by the Department of Defense, Cornell University and
colleagues. These data have been analyzed in various forms by
\citet{wegan:2008,gardner_etal:2009} and \citet{gardner_etal:2010jwm}.
The specific data used here are encounter histories on 47 individuals
obtained from an array of 38 baited ``hair snares''
(Fig. \ref{fig.3.bears1}) during June and July 2006.  Barbed wire
traps were baited and checked for hair samples each week for eight
weeks, thus we have $K=8$ sample intervals. The data are provided 
in the {\bf R} package \mbox{\tt scrbook} 
and the analysis can be set up and run as
follows. Here, the data were augmented with $M-n = 128$ ($M=175$)
all-zero encounter histories.

\begin{figure}
\centering
\includegraphics[height=2.5in,width=1.9in]{Ch3/figs/hairsnares.png}
\caption{Fort Drum study area and hair snare locations.}
\label{fig.3.bears1}
\end{figure}

{\small
\begin{verbatim}
library("scrbook")
data("beardata")
trapmat<-beardata$trapmat
nind<-dim(beardata$bearArray)[1]
K<-dim(beardata$bearArray)[3]
ntraps<-dim(beardata$bearArray)[2]

M=175
nz<-M-nind
Yaug <- array(0, dim=c(M,ntraps,K))

Yaug[1:nind,,]<-beardata$bearArray
y<- apply(Yaug,c(1,3),sum) # summarize by ind x rep
y[y>1]<- 1             # toss out duplicate obs
ytot<-apply(y,1,sum)   # total encounters out of K
\end{verbatim}
}

The raw data object, \mbox{\tt beardata\$bearArray} is a 3-dimensional
array $\mbox{\tt nind} \times \mbox{\tt ntraps} \times K$ of
individual encounter events (i.e., $y_{ijk} = 1$ if individual $i$ was
encountered in trap $j$ during occasion $k$, and 0 otherwise).  For
fitting model $M_{0}$ or $M_{h}$ (see below), it is sufficient to
reduce the data to individual encounter frequencies which we have
labeled \mbox{\tt ytot} above.  The {\bf BUGS} model file along with
commands to fit the model are as follows:

{\small
\begin{verbatim}
set.seed(2013)               # to obtain the same results each time
library("R2WinBUGS")
data0<-list(y=y,M=M,K=K)
params0<-list('psi','p','N')
zst=c(rep(1,nind),rbinom(M-nind, 1, .5))
inits =  function() {list(z=zst, psi=runif(1), p=runif(1)) }

cat("
model {

psi~dunif(0, 1)
p~dunif(0,1)

for (i in 1:M){
   z[i]~dbern(psi)
   for(k in 1:K){
      tmp[i,k]<-p*z[i]
      y[i,k]~dbin(tmp[i,k],1)
       }
       }
N<-sum(z[1:M])
}
",file="modelM0.txt")

fit0 = bugs(data0, inits, params0, model.file="modelM0.txt",
       n.chains=3, n.iter=2000, n.burnin=1000, n.thin=1,
       debug=TRUE,working.directory=getwd())
\end{verbatim}
}
This produces the follow posterior
 summary statistics:
{\small
\begin{verbatim}
> print(fit0,digits=2)
Inference for Bugs model at "modelM0.txt", fit using WinBUGS,
 3 chains, each with 2000 iterations (first 1000 discarded)
 n.sims = 3000 iterations saved
           mean    sd   2.5%    25%    50%    75%  97.5% Rhat n.eff
psi        0.29  0.04   0.22   0.26   0.29   0.31   0.36    1  3000
p          0.30  0.03   0.25   0.28   0.30   0.32   0.35    1  3000
N         49.94  1.99  47.00  48.00  50.00  51.00  54.00    1  3000
deviance 489.05 11.28 471.00 480.45 488.80 495.40 513.70    1  3000

[.. some output deleted ...]
\end{verbatim}
}
{\bf WinBUGS} did well in choosing an MCMC algorithm for this model --
we have $\hat{R} = 1$ for each parameter, and an effective sample size
of 3000, equal to the total number of posterior samples.
We see that the posterior mean of $N$ under this
model is $49.94$ and a 95\% posterior interval is $(48,54)$.  We
revisit these data later in the context of more complex models.



In order to obtain an estimate of density, $D$, we need an area to
associate with the estimate of $N$, XXXX and in Chapt. \ref{chapt.intro} we already went through a number of commonly used procedures to
conjure up such an area, including buffering the trap array by the home
range radius, often estimated by the mean maximum distance moved
(MMDM) \citep{parmenter_etal:2003},
$1/2$ MMDM \citep{dice:1938} or
directly from telemetry data (REF XXX NEED REF HERE, WALLACE ET AL 2003 DO THIS; I HAVE SEEN 2 PAPERS CITING OTIS ET AL 1978 IN THIS CONTEXT BUT I ONLY FOUND THE SECITON WHERE THEY SUGGEST USING INFORMATION ON ANIMAL HOME RANGE AS OBTAIN FROM TRAPPING DATA; I GUESS THIS DICE GUY SAID TO USE THE HOME RANGE RADIUS AND PEOPLE JUST TRY TO GET AT THIS WHICHEVER WAY THEY CAN; BE IT RECAPTURES OR OTHER HOME RANGE INFORMATION XXXXX).
Typically, the trap
array is defined by the convex hull around the trap locations, and
this is what we applied a buffer to. We computed the buffer by using
an estimate of the mean female home range radius (2.19 km) estimated from
telemetry studies \citep{bales_etal:2005} instead of using an estimate
based on our relatively more sparse recapture data.
 For the Fort Drum study, the convex hull has area
$157.135$ $km^2$, and the buffered convex hull has area $277.011$
$km^2$.
To create this we used functions contained in the {\bf R} package
\mbox{\tt rgeos} and created a utility function \mbox{\tt bcharea}
which is in our {\bf R} package \mbox{\tt scrbook}. The commands are
as follows:
\begin{verbatim}
library("rgeos")

bcharea<-function(buff,traplocs){
p1<-Polygon(rbind(traplocs,traplocs[1,]))
p2<-Polygons(list(p1=p1),ID=1)
p3<-SpatialPolygons(list(p2=p2))
p1ch<-gConvexHull(p3)
 bp1<-(gBuffer(p1ch, width=buff))
 plot(bp1, col='gray')
 plot(p1ch, border='black', lwd=2, add=TRUE)
 gArea(bp1)
}

bcharea(2.19,traplocs=trapmat)
\end{verbatim}
The resulting buffered convex hull is shown in Fig. \ref{closed.fig.bch}.
\begin{figure}
\begin{center}
\includegraphics[height=3in,width=3in]{Ch3/figs/bufferedCH}
\end{center}
\caption{buffered convex hull of the bear hair snare array}
\label{closed.fig.bch}
\end{figure}

To conjure up a
density estimate under model $M_0$, we compute the appropriate
posterior summary of $N$ and the prescribed area ($277.011$ $km^2$):
\begin{verbatim}
> summary(fit0$sims.list$N/277.011)
   Min. 1st Qu.  Median    Mean 3rd Qu.    Max.
 0.1697  0.1733  0.1805  0.1803  0.1841  0.2130

> quantile(fit0$sims.list$N/277.011,c(0.025,0.975))
     2.5%     97.5%
0.1696684 0.1949381
\end{verbatim}
which yields a density estimate of about $0.18$ ind/km$^2$, and a $95\%$ Bayesian
confidence interval of $(0.170, 0.195)$.

In summary, we have an estimate of density if we have faith in our
stated value of the ``sample area''. Clearly though this is largely
subjective, and not something we can formally evaluate from the data.
How certain are we of this area? Can
we quantify our uncertainty about this quantity? 
 More important, what exactly is
the meaning of this area and, in this context, how do we gauge bias
and/or variance of ``estimators'' of it? (i.e., what is it
estimating?).  
XXX I DON'T KNOW IF IT'S WORTH MENTIONING THE DELTA APPROXIMATION KARANTH AND NICHOLS (1998) USE XXX
There is no theory to guide us in trying to answer these important questions.


\section{Temporally varying and behavioral effects}

The purpose of this chapter is mainly to emphasize the central
importance of the binomial model in capture-recapture and so we have
considered models for individual encounter frequencies - the number of
times individuals are captured out of $K$ samples.  Sometimes it is
not acceptable to aggregate the encounter data for each individual --
such as when encounter probability varies over time among samples. 
Time-varying responses that are relevant in many
capture-recapture studies are ``effort'' such as amount of search time,
number of observers, or trap nights, or when encounter probability
varies over time or as a function of date or season due to species behavior
\citep{kery_etal:2010}.
  A common situation in a large number of carnivore studies is that in
which there exists a ``behavioral response'' to trapping (even if the
animal is not physically trapped).
XXXX IS THERE ANY PARTICULAR REASON WHY YOU ONLY REFER TO CARNIVORES HERE? XXXX

Behavioral response is an important concept in carnivore studies
because individuals might learn to come to baited traps or avoid traps
due to trauma related to being encountered.  There are a number of
ways to parameterize a behavioral response to encounter. The
distinction between persistent and ephemeral was made by
\citet{yang_chao:2005} who considered a general behavioral response
model of the form:
\[
\mbox{logit}(p_{ik}) = \alpha_{0} + \alpha_{1}*y_{i,k-1} + \alpha_{2} x_{ik}
\]
where $x_{ik}$ is a covariate indicator variable of previous capture
(i.e., $x_{ik} = 1$ if captured in any previous period). Therefore,
encounter probability changes depending on whether an individual was
captured in the immediate previous period (ephemeral behavioral
response XXX described by the term $\alpha_{1}*y_{i,k-1}$) or in any previous period (persistent behavioral
response). The former probably models a behavioral response due to
individuals moving around their territory relatively slowly over time
and the latter probably accommodates trap happiness due to baiting or
shyness due to trauma.   XXX Spatial capture-recapture models allow us to include trap-specific covariates, XXX and in such models it makes
sense to consider a local behavioral response that is trap-specific
\citep{royle_etal:2011jwm} - that is, the encounter probability is
modified for an individual trap depending on previous capture in
that trap.

Models with temporal effects are easy to describe in the {\bf BUGS} language
and analyze and we provide a number of examples in
Chapt. \ref{chapt.covariates} and elsewhere. 


\section{ Models with individual heterogeneity}
\label{closed.sec.modelmh}

Here we consider models with individual-specific encounter probability
parameters, say $p_{i}$, which we model according to some probability
distribution, $g(\theta)$. We denote this basic model assumption as
$p_{i} \sim g(\theta)$. This type of model is similar in concept to
extending a GLM to a GLMM but in the capture-recapture context $N$ is
unknown.  The basic class of models is often referred to as ``Model
$M_h$'' but really this is a broad class of models, each being
distinguished by the specific distribution assumed for $p_{i}$.  There
are many different varieties of Model $M_{h}$ including parametric and
various putatively non-parametric approaches
\citep{burnham_overton:1978, norris_pollock:1996, pledger:2000}. One
important practical matter is that estimates of $N$ can be extremely
sensitive to the choice of heterogeneity model
\citep{fienberg_etal:1999, dorazio_royle:2003, link:2003}. Indeed,
\citet{link:2003} showed that in some cases it's possible to find
models that yield precisely the same expected data, yet produce wildly
different estimates of $N$. In that sense, $N$ for most practical
purposes is not identifiable across classes of mixture models, and
this should be understood before fitting any such model. One solution
to this problem is to seek to model explicit factors that contribute
to heterogeneity, e.g., using individual covariate models (See
\ref{closed.sec.indcov} below). Indeed, spatial capture-recapture
models seek to do just that, by modeling heterogeneity due to the
spatial organization of individuals in relation to traps or other
encounter mechanism.  For additional background and applications of
Model $M_{h}$ see \citet[][chapt. 6]{royle_dorazio:2008} and
\citet[][chapt. 6]{kery_schaub:2011}.

Model $M_{h}$ has important historical relevance to spatial
capture-recapture situations \citep{karanth:1995} because
investigators recognized that the juxtaposition of individuals with
the array of trap locations should yield heterogeneity in encounter
probability, and thus it became common to use some version of Model $M_h$
in spatial trapping arrays to estimate $N$.  While this doesn't
resolve the problem of not knowing the area relevant to $N$, it does
yield an estimator that accommodates the heterogeneity in $p$ induced
by the spatial aspect of capture-recapture studies.

To see how this juxtaposition induces heterogeneity, we have to
understand the relevance of movement in capture-recapture models.
Imagine a quadrat that can be uniformly searched by a crew of
biologists for some species of reptile (see
\citet{royle_young:2008}).  Figure \ref{closed.fig.quadrat} shows a
sample quadrat searched repeatedly over a period of time. Further,
suppose that species exhibits some sense of spatial fidelity in the
form of a home range or territory, and individuals move about their
home range (home range centroids are given by the blue dots) in some
kind of random fashion.  
%It is natural to think about it in terms of a
%movement process and sometimes that movement process can be modeled
%explicitly using hierarchical models \citep{royle_young:2008,
%  royle_etal:2011mee}.  
Heuristically, we imagine that each individual in
the vicinity of the study area is liable to experience variable
exposure to encounter due to the overlap of its home range with the
sampled area - essentially the long-run proportion of times the
individual is within the sample plot boundaries, say $\phi$. We
might model the exposure of an individual to capture by supposing that
$z_{i} = 1$ if individual $i$ is available to be captured (i.e.,
within the survey plot) during any sample, and $0$ otherwise. Then,
$\Pr(z_{i}=1) = \phi$.  In the context of spatial studies, it is
natural that $\phi$ should depend on {\it where} an individual lives,
i.e., it should be individual-specific $\phi_{i}$
\citep{chandler_etal:2011}. This system describes, precisely, that of
``random temporary emigration'' \citep{kendall_etal:1997} where $\phi_{i}$
is the individual-specific probability of being ``available'' for
capture.

Conceptually, SCR models aim to deal with
this problem of variable exposure to sampling due to movement in the
proximity of the trapping array explicitly and formally with auxiliary
spatial information.  If individuals are detected with probability
$p_{0}$, {\it conditional} on $z_{i} = 1$, then the marginal
probability of detecting  individual $i$ is
\[
 p_{i} = p_{0}\phi_{i}
\]
so we see clearly that individual heterogeneity in encounter
probability is induced as a result of the juxtaposition of individuals
(i.e., their home ranges) with the sample apparatus and the movement
of individuals about their home range.

\begin{figure}
\begin{center}
\includegraphics[height=3in]{Ch3/figs/quadrat}
\end{center}
\caption{A quadrat searched for lizards and the locations of each
  lizard over some period of time.}
\label{closed.fig.quadrat}
\end{figure}

We will work with a specific type of Model $M_{h}$ here, that in which
we extend the basic binomial observation model of Model $M_{0}$ so
that
\[
\mbox{logit}(p_{i}) = \mu + \eta_{i}
\]
where
\[
\eta_{i} \sim \mbox{Normal}(0, \sigma_{p}^2)
\]
We could as well write
\[
\mbox{logit}(p_{i}) \sim \mbox{Normal}(\mu,\sigma_{p}^2)
\]
This ``logit-normal mixture'' was analyzed by
\citet{coull_agresti:1999} and elsewhere. It is a natural extension of
the basic model with constant $p$, as a mixed GLMM, and similar models
occur throughout statistics. It is also natural to consider a beta
prior distribution for $p_{i}$ \citep{dorazio_royle:2003} and
so-called ``finite-mixture'' models XXX (models in which individuals are assumed to belong to a finite number of latent classes, each of which has its own capture probability) XXX are also popular
\citep{norris_pollock:1996, pledger:2000}.

\subsection{Analysis of Model $M_h$}

If $N$ is known, it is worth taking note of the essential simplicity
of model $M_h$ as a binomial GLMM.  This is a type of model that is
widely applied in just about every scientific discipline and using
standard methods of inference based either on integrated likelihood
\citep{laird_ware:1982, berger_etal:1999} which we discuss in
Chapt. \ref{chapt.mle} or standard Bayesian
methods. However, because $N$ is not known, inference is somewhat more
challenging. We address that here using Bayesian analysis based on
data augmentation (DA). Although we use data augmentation in the context of
Bayesian methods here, we note that
heterogeneity models formulated under DA are easily analyzed by
conventional likelihood methods as zero-inflated binomial mixtures
\citep{royle:2006} and more traditional analysis of model $M_h$ based on
integrated likelihood, without using data augmentation, has been
considered by \citet{coull_agresti:1999}, \citet{dorazio_royle:2003},
and others.

As with model $M_{0}$, we have the Bernoulli model for the
zero-inflation variables: $z_{i} \sim \mbox{Bern}(\psi)$ and the model
of the observations expressed conditional on the latent variables
$z_{i}$. For $z_{i}=1$, we have a binomial model with
individual-specific $p_{i}$:
\[
y_{i}|{z_{i} \! = \! 1} \sim \mbox{Bin}(K,p_{i})
\]
and otherwise $y_{i} |{ z_{i} \! = \! 0} \sim \delta(0)$. Further, we
prescribe a distribution for $p_{i}$. Here we assume
\[
\mathrm{logit}(p_{i}) \sim \mbox{Normal}(\mu,\sigma^2)
\]
The basic {\bf BUGS} description for this model, assuming a
$\mbox{Unif}(0,1)$ prior for $p_{0} = \mbox{logit}^{-1}(\mu)$, is given
as follows:
{\small
\begin{verbatim}
model{

p0 ~ dunif(0,1)       # prior distributions
mup<- log(p0/(1-p0))
taup~dgamma(.1,.1)
psi~dunif(0,1)

for(i in 1:(nind+nz)){
  z[i]~dbern(psi)     # zero inflation variables
  lp[i] ~ dnorm(mup,taup) # individual effect
  logit(p[i])<-lp[i]
  mu[i]<-z[i]*p[i]
  y[i]~dbin(mu[i],J)  #  observation model
 }

N<-sum(z[1:(nind+nz)])  # N is a derived parameter
}
\end{verbatim}
}


\subsection{Analysis of the Fort Drum data}

The logit-normal heterogeneity model was fitted to the bear data from
the Fort Drum study, and we used data augmentation to produce a data
set of $M=500$ individuals.  We ran the model using {\bf JAGS} with
the instructions given as follows:
{\small
\begin{verbatim}
[... get data as before ....]

set.seed(2013)

cat("
model{
p0 ~ dunif(0,1)       # prior distributions
mup<- log(p0/(1-p0))
sigmap ~ dunif(0,10)
taup<- 1/(sigmap*sigmap)
psi~dunif(0,1)

for(i in 1:(nind+nz)){
  z[i]~dbern(psi)     # zero inflation variables
  lp[i] ~ dnorm(mup,taup) # individual effect
  logit(p[i])<-lp[i]
  mu[i]<-z[i]*p[i]
  y[i]~dbin(mu[i],K)  #  observation model
 }

N<-sum(z[1:(nind+nz)])
}
",file="modelMh.txt")

data1<-list(y=ytot, nz=nz, nind=nind,K=K) 
params1= c('p0','sigmap','psi','N')
inits =  function() {list(z=as.numeric(ytot>=1), psi=.6, p0=runif(1),
          sigmap=runif(1,.7,1.2),lp=rnorm(M,-2)) }

library("rjags")
jm<- jags.model("modelMh.txt", data=data1, inits=inits, n.chains=4,
                 n.adapt=1000)
jout<- coda.samples(jm, params1, n.iter=200000, thin=1)
\end{verbatim}
}
This produces the posterior distribution for $N$ shown
in Fig. \ref{closed.fig.bearMh}. Posterior summaries of parameters are
given as follows:
{\small
\begin{verbatim}
> summary(jout)

Iterations = 2001:202000
Thinning interval = 1 
Number of chains = 4 
Sample size per chain = 2e+05 

1. Empirical mean and standard deviation for each variable,
   plus standard error of the mean:

           Mean       SD  Naive SE Time-series SE
N      117.7740 56.31633 6.296e-02       1.960115
p0       0.0728  0.05522 6.174e-05       0.001655
psi      0.2366  0.11362 1.270e-04       0.003909
sigmap   2.0795  0.53096 5.936e-04       0.016789

2. Quantiles for each variable:

            2.5%      25%       50%      75%    97.5%
N      62.000000 82.00000 102.00000 134.0000 277.0000
p0      0.003143  0.02842   0.06077   0.1066   0.2036
psi     0.117269  0.16377   0.20522   0.2712   0.5560
sigmap  1.211900  1.69434   2.02113   2.4028   3.2694
\end{verbatim}
}


We used $M=500$ for this analysis and we
note that  while the posterior mass of $N$ is concentrated away from this
upper bound (Fig. \ref{closed.fig.bearMh}), the posterior has an
extremely long right tail, with some posterior values at the upper
bound $N=500$. Maybe or
maybe not sufficient data augmentation.\footnote{
{\bf to do: } insert final results. longer run. more data
augmentation. compare with winbugs.
}
The model runs effectively in {\bf WinBUGS} but sometimes with apparently
inefficient mixing for reasons that may be related to bad starting
values. In some cases this was resolved if we supplied starting values
for the $logit(p_{i})$ parameters and $\tau$.


Because of the skewed posterior we see that the posterior mean ($N=117$)
is
considerably higher than the posterior mode ($N=102$). Moreover, 
posterior summaries are estimated with a relatively high error
(``Time-series SE'' of around 2.0)\footnote{need to define this somewhere XXX THIS COMES UP IN CH2 XXX}.
Further, it may be surprising that the posterior mode does not compare
well with the MLE. To compute the posterior mode we could easily find
the posterior value of $N$ with the highest mass because $N$ is
discrete. But we want to smooth out some of the Monte Carlo error a
bit so we used a smoothing spline to the posterior frequencies of $N$
as follows:
\begin{verbatim}
  tt<-table(jout[[1]][,"N"])[1:80]
  xg<-as.numeric(names(tt))
  plot(xg,tt)
  sp<- smooth.spline(xg,tt,df=9)
  sp$x[sp$y==max(sp$y)]
[1] 80
\end{verbatim}
The \mbox{\tt df} argument controls the degree of smoothing and we
find in this case that the modal value (i.e., 80) is not too sensitive
to the smoothing parameter but this should be checked in any specific
instance\footnote{we need to give examples of using \mbox{\tt
    density()} to obtain modes}.

To compute the MLE, we used 
the {\bf R} code contained in Panel 6.1 of \citet{royle_dorazio:2008}.  The
MLE of $log(n_{0})$, the logarithm of the number of uncaptured
individuals, is $\widehat{log(n0)} = 3.86$ and therefore $\hat{N} =
exp(3.86)+47 = 94.47$ which is not at all consistent with the apparent
mode in 
Fig. \ref{closed.fig.bearMh}.
\footnote{We note that the result is inconsistent with Gardner et
  al. (2009) who reported an MLE of 104.1 ($density = 0.437
  inds/km^2$) although we do not know the reason for this at the
  present time.}  
%To convert this to density we use the buffered area
%as computed above (255.3 $km^2$)\footnote{WRONG \#} and perform the
%required summary analysis on the posterior samples of $N$, which
%results in about $0.37$ individuals/$km^2$. The reader should carry
%out this analysis to confirm the estimates, and also obtain the $95\%$
%confidence interval.

{\bf Remarks:} First of all the posterior for this model and data set is
very sensitive to prior distributions. While MLEs are invariant to
transformation of the parameters, the posterior distribution
definitely is {\it not} invariant. In the present case, the use of a
$\mbox{Unif}(0,1)$ prior for $p_{0} = \mbox{expit}(\mu)$ is somewhat
informative -- in particular, it is not at all ``flat'' on the scale
of $\mu$ -- and this affects the posterior.  We generally always
recommend use of a $\mbox{Unif}(0,1)$ prior for $\mbox{expit}(\mu)$ in such
models. That said, we were surprised at this result, and we
experimented with other prior configurations including putting a flat
prior on $\mu$ directly. That specific prior suggests the possibility
that the posterior distribution may be improper for that prior
specification. This kind of small sample instability has been widely
noted in model $M_h$ \citep{fienberg_etal:1999, dorazio_royle:2003} and
is not unrelated to sensitivity to
model XXX WORD MISSING? XXX which has also been identified as an important issue in model
$M_{h}$ \citep{dorazio_royle:2003,link:2003}.
Conclusion: The mode is well-defined but the data set is sparse and
hence inferences are poor and sensitive to model choices. Get over it.


\begin{figure}
\centering
\includegraphics[height=4.5in,width=4.5in]{Ch3/figs/bear-modelMh-post}
\caption{Posterior of $N$ for Fort Drum bear study data under the
logit-normal version of model $M_h$. 
}
\label{closed.fig.bearMh}
\end{figure}


\subsection{Building your own MCMC algorithm}

For fun, we construct our own MCMC algorithm using a Metropolized
Gibbs sampler for model $M_{h}$ in Chapt. \ref{chapt.mcmc}, where we
also develop the MCMC 
algorithms for spatial capture-recapture models.
XXX MAYBE PUT THIS IN A FOOTNOTE? XXX

\begin{comment}

To begin, we first collect all of our model components
which are as follows: $[y_{i}| p_{i},z_{i}]$,
$[p_{i}|\mu_{p},\sigma_{p}]$, and $[z_{i}|\psi]$
for {\it each} $i=1,2,\ldots,M$ and then prior distributions
$[\mu_{p}]$, $[\sigma_{p}]$ and $[\psi]$.
The joint posterior distribution of all unknown quantities in the model
is proportional to the joint distribution of all elements
$y_{i},p_{i},z_{i}$ and also the prior distributions of the prior parameters:
\[
\left\{ \prod_{i=1}^{M} [y_{i}|p_{i},z_{i}][p_{i}|\mu_{p},\sigma_{p}]
[z_{i}|\psi] \right\} [\mu_{p},\sigma_{p},\psi]
\]
For prior distributions, we assume that $\mu_{p},\sigma_{p}, \psi$ are
mutually independent and for $\mu_{p}$ and $\sigma_{p}$ we use
improper uniform priors, and $\psi \sim \mbox{Unif}(0,1)$.  Note that
the likelihood contribution for each individual, when conditioned on
$p_{i}$ and $z_{i}$, does not depend on $\psi$, $\mu_{p}$, or
$\sigma_{p}$.  As such, the full-conditionals for the structural
parameters $\psi$ only depends on the collection of data augmentation
variables $z_{i}$, and that for $\mu_{p}$ and $\sigma_{p}$ will only
depends on the collection of latent variables $p_{i}; i=1,2,\ldots,M$.
The full conditionals for all the unknowns are as follows:

{\bf (1)} For $p_{i}$:
\begin{eqnarray*}
[p_{i}|y_{i}, \mu_p, \sigma_{p},z_{i}=1] &\propto  &
[y_{i}|p_{i}][p_{i}|\mu_p,\sigma_{p}^{2}] \mbox{ if $z_{i}=1$ }  \\
                 &  &  [p_{i}|\mu_p,\sigma_{p}] \mbox{if $z_{i}=0$ }
\end{eqnarray*}

{\bf (2)} for $z_{i}$:
\[
z_{i} | \cdot \propto [y_{i}|z_{i}*p_{i}] \mbox{Bern}(z_{i}|\psi)
\]

{\bf (3)} For $\mu_{p}$:
\[
[\mu_{p} | \cdot ] \sim \prod_{i} [p_{i}| \cdot] *\mbox{const}
\]


{\bf (4)} For $\sigma_{p}$:
\[
[ \sigma_{p}|\cdot ] \sim\prod_{i}[p_{i}| \cdot ]*\mbox{const}
\]

{\bf (5)} For $\psi$:
\[
\psi|\cdot\sim \mbox{Beta}(1 + \sum z_{i}, 1 + M - \sum z_{i})
\]


We've  identified each of the full conditional
distributions in sufficient detail to apply the
Metropolis-Hastings algorithm. With the exception of $\psi$ which has
a convenient analytic solution -- it is a beta distribution which we
can easily sample directly. In truth, we could also sample $\mu_{p}$
and $\sigma_{p}^{2}$ directly with certain choices of prior
distributions. For example, if $\mu_{p} \sim \mbox{Normal}(0, 1000)$
then the full conditional for $\mu_{p}$ is also normal, etc..
We implement an MCMC algorithm for this model in the following block
of {\bf R} code.  The basic structure is: initialize the parameters
and create any required output or intermediate data holders, and then
begin the main MCMC loop which, in this case, generates 100000
samples.\footnote{This data grabbing function is not implemented yet}

{\small
\begin{verbatim}
## obtain the bear data by executing the previous data grabbing
## function

temp<-getdata()
M<-temp$M
K<-temp$K
ytot<-temp$ytot

###
### MCMC algorithm for Model Mh

out<-matrix(NA,nrow=100000,ncol=4)
dimnames(out)<-list(NULL,c("mu","sigma","psi","N"))
lp<- rnorm(M,-1,1)
p<-expit(lp)
mu<- -1
p0<-exp(mu)/(1+exp(mu))
sigma<- 1
psi<- .5
z<-rbinom(M,1,psi)
z[ytot>0]<-1

for(i in 1:100000){

### update the logit(p) parameters
lpc<- rnorm(M,lp,1)  # 0.5 is a tuning parameter
pc<-expit(lpc)
lik.curr<-log(dbinom(ytot,K,z*p)*dnorm(lp,mu,sigma))
lik.cand<-log(dbinom(ytot,K,z*pc)*dnorm(lpc,mu,sigma))
kp<- runif(M) < exp(lik.cand-lik.curr)
p[kp]<-pc[kp]
lp[kp]<-lpc[kp]

p0c<- rnorm(1,p0,.05)
if(p0c>0 & p0c<1){
muc<-log(p0c/(1-p0c))
lik.curr<-sum(dnorm(lp,mu,sigma,log=TRUE))
lik.cand<-sum(dnorm(lp,muc,sigma,log=TRUE))
if(runif(1)<exp(lik.cand-lik.curr)) {
 mu<-muc
 p0<-p0c
}
}

sigmac<-rnorm(1,sigma,.5)
if(sigmac>0){
lik.curr<-sum(dnorm(lp,mu,sigma,log=TRUE))
lik.cand<-sum(dnorm(lp,mu,sigmac,log=TRUE))
if(runif(1)<exp(lik.cand-lik.curr))
 sigma<-sigmac
}

### update the z[i] variables
zc<-  ifelse(z==1,0,1)  # candidate is 0 if current = 1, etc..
lik.curr<- dbinom(ytot,K,z*p)*dbinom(z,1,psi)
lik.cand<- dbinom(ytot,K,zc*p)*dbinom(zc,1,psi)
kp<- runif(M) <  (lik.cand/lik.curr)
z[kp]<- zc[kp]

psi<-rbeta(1, sum(z) + 1, M-sum(z) + 1)

out[i,]<- c(mu,sigma,psi,sum(z))
}
\end{verbatim}
}


{\bf Remarks}: (1) for parameters with bounded support, i.e.,
$\sigma_{p}$ and $p_{0}$, we are using a random walk candidate
generator but rejecting draws outside of the parameter space.  (2) We
mostly use Metropolis-Hastings except for the data augmentation
parameter $\psi$ which we sample directly from its full-conditional
distribution which is a beta distribution.  (3) Even the latent data
augmentation variables $z_{i}$ are updated using Metropolis-Hastings
although they too can be updated directly from their full-conditional.
\end{comment}


\begin{comment}

\subsection{Exercises related to model Mh}

\begin{itemize}
\item[(1)] Enclose the MCMC algorithm in an R function and provide
  arguments for some of the parameters of the function that a user
  might wish to modify.
\item[(2)] Execute the function and compare the results to those
  generated from WinBUGS in the previous section
\item[(3)] Note that the prior distribution for the ``mean'' parameter
  is given on $p_0=exp(\mu)/(1+exp(\mu))$.  Reformulate the algorithm
  with a flat prior on $\mu$ and see what happens. Contemplate this.
\item[(4)] Using Bayes rule, figure out the full conditional for
  $z_{i}$ so that you don't have to use MH for that one. It might be
  more efficient. Is it?
\item[(5)] Modify the MCMC algorithm so that the prior for $\mu_{p}$
  is an improper flat prior. i.e., $[\mu_{p}] \propto 1$. Describe the
  posterior distribution of $N$. 
\end{itemize}

\end{comment}



\section{Individual Covariate Models: Toward Spatial Capture-Recapture}
\label{closed.sec.indcov}


A standard situation in capture-recapture models is when an individual
covariate is measured, and this covariate is thought to influence
encounter probability.  As with other closed population models, we
begin with the basic binomial observation model:
\[
y_{i} \sim \mbox{Bin}(K, p_{i})
\]
and we assume also  a model for encounter probability according to:
\begin{equation}
 \mbox{logit}(p_{i}) = \alpha + \beta x_{i}
\label{closed.eq.ha}
\end{equation}
Classical examples of covariates influencing detection probability are
type of animal (juvenile/adult or male/female), a continuous covariate
such as body mass \citep[][ch. 6]{royle_dorazio:2008}, or a
discrete covariate such as group or cluster size. For example, in
models of aerial survey data, it is natural to model detection
probabilities as a function of the observation-level individual
covariate, ``group size'' \citep{royle:2008, royle:2009,
  langtimm_etal:2011}.

Such ``individual covariate models'' are similar in structure to Model
$M_{h}$, except that the individual effects are {\it observed} for the
$n$ individuals that appear in the sample. These models are important
here because spatial capture-recapture models are precisely a form of
individual covariate model, an idea that we will develop here and
elsewhere. Specifically, they are such models, but where the
individual covariate is a partially observed latent variable for 
captured individuals. As such, it is a type of measurement error.
That is, unlike model $M_h$, we do have some direct information about the
latent variable, which comes from the spatial locations/distribution
of individual recaptures.

Traditionally, estimation of $N$ in individual covariate models is
achieved using methods based on ideas of unequal probability sampling
(i.e., Horwitz-Thompson estimation; see \citet{huggins:1989} and
\citet{alho:1990}). An estimator of $N$ is
\[
\hat{N} = \sum_{i}^{n} \frac{1}{\tilde{p}_{i}}
\]
where $\tilde{p}_{i}$ is the probability that individual $i$ appeared
in the sample.  That is, $\tilde{p}_{i} = \Pr(y_{i}>0)$
where, in closed population capture-recapture models, 
\[
\Pr(y_{i}>0) = (1- (1-p_{i})^K)
\]
where $p_{i}$ is a function of parameters $\alpha$ and $\beta$
according to
Eq. \ref{closed.eq.ha}.
In practice, parameters are estimated 
from the conditional-likelihood of the observed encounter histories
which is, for observation $y_{i}$, 
\[
{\cal L}_{c}(\alpha, \beta | y_{i}) = \frac{ \mbox{Bin}(y_{i}|\alpha,\beta) } { \tilde{p}_{i}}.
\]

Here we take a formal model-based approach to Bayesian analysis of
such models based on the joint likelihood
using data augmentation \citep{royle:2009}. Classical
likelihood analysis of the so-called ``full likelihood'' is covered 
 by \citet{borchers_etal:2002}.  For Bayesian analysis of
individual covariate models, because the individual covariate is
unobserved for the $N-n$ uncaptured individuals, we require a model to
describe variation among individuals, essentially allowing the sample
to be extrapolated to the population\footnote{weak argument}.  For our present purposes, we
consider a continuous covariate and we assume that it has a normal
distribution:
\[
x_{i} \sim \mbox{Normal}(\mu,\sigma^{2})
\]

Data augmentation can be applied directly to this class of models. In
particular, reformulation of the model under DA yields a basic
zero-inflated binomial model of the form:
\begin{eqnarray*}
z_{i} &\sim& \mbox{Bern}(\psi) \; \; \; i=1,2,\ldots,M\\
y_{i}|{z_{i}\! =\! 1} &\sim& \mbox{Bin}(K,p_{i}(x_{i})) \\
y_{i} |{ z_{i}\! =\! 0} &\sim& \delta(0)  \\
x_{i} & \sim & \mbox{Normal}(\mu,\sigma^{2})
\end{eqnarray*}
Fully spatial capture-recapture models use this
formulation with a latent covariate that is directly related to the
individual detection probability (see next section). As with the
previous models, implementation is trivial in the {\bf BUGS} language. The
{\bf BUGS} specification is very similar to that for model $M_h$, but we
require the distribution of the covariate to be specified, along with
priors for the parameters of that distribution.


\subsection{Example: Location of capture as a covariate.}

If we had a regular grid of traps over some closed geographic system
then we imagine that the average location of capture would be a decent
estimate (heuristically) of an individual's home range center.
Intuitively some measure of typical distance from home range center to
traps for an individual should be a decent covariate to explain
heterogeneity in encounter probability, i.e., individuals with more
exposure to traps should have higher encounter probabilities and vice
versa.  A version of this idea was put forth by
\citet{boulanger_mclellan:2001} (see also \citet{ivan:2012}), but
using the Huggins-Alho estimator and with covariate ``distance to
edge'' of the trapping array. A limitation of this  approach is
that it does not provide a solution to the problem that the trap area
is fundamentally ill-defined, nor does it readily accommodate the
inherent and heterogeneous variation in this measured covariate.

Here, we provide an example of this type of heuristically motivated
approach using the fully model-based individual covariate model
described above analyzed by data augmentation. We take a slightly
different approach than that adopted by
\citet{boulanger_mclellan:2001}. By analyzing the full likelihood and
placing a prior distribution on the individual covariate, we resolve
the problem of having an ill-defined area over which the population
size is distributed. After you read later chapters of this book, it
will be apparent that SCR models represent a formalization of this
heuristic procedure.

For our purposes here, we define $x_{i} = ||{\bf s}_{i} - {\bf
  x}_{0}||$ where ${\bf s}_{i}$
is the average encounter location of individual $i$ and ${\bf x}_{0}$ is the
centroid of the trap array.  Conceptually, individuals in the middle
of the array should have higher probability of encounter and, as
$x_{i}$ increases, $p_{i}$ should therefore decrease. We note that we
have defined ${\bf s}_{i}$ in terms of a sample quantity - the observed mean
- which is ad hoc but consistent with existing applications in the literature.
For an expansive, dense trapping grid then we might expect the sample mean
encounter location to be a good estimate of home range center but,
clearly this is biased for individuals that live around the edge (or
off) the trapping array. Regardless, it should be good enough for our
present purposes of demonstrating this heuristically appealing
application of an individual covariate model. A key point is that
${\bf s}_{i}$ is missing for each individual that is not encountered and
thus so is $x_{i}$. Thus,
it is a latent variable, or random effect, and we need therefore to
specify a probability distribution for it.
As a measurement of distance we know it must be
positive-valued. Thinking about this like a distance sampling problem
lets first try to make $x_{i}$ uniform from $0$ to some large number,
say $D_{max}$, beyond which it would be difficult to imagine an
individual being captured. For example, $D_{max}$ should be at a home
range diameter past the furthest trap from the center.
As such, we use this distribution for the individual covariate
``distance from center of the trap array''
\[
 x_{i} \sim \mbox{Unif}(0,D_{max})
\]
where $D_{max}$ is a specified constant, which we may choose to be
arbitrarily large.  In practice, people have
used distance from edge of the trap array but that is less easy to
make sense of.


\subsubsection{Fort Drum Bear Study}


\begin{figure}
\centering
\includegraphics[height=3.5in,width=3.5in]{Ch3/figs/bear_spiderplot.png}
\caption{Spider plot of the Fort Drum study data.}
\label{closed.fig.spiderplot}
\end{figure}


We have to do a little bit of data processing to fit this individual
covariate model to the Fort Drum data. 
We need to compute the individual covariate ${\bf x}_{i}$ (distance from the centroid of the trapping
array) using the {\bf R} function
\mbox{\tt spiderplot}
provided in \mbox{\tt scrbook}. This function also produces a keen plot shown in
Fig. \ref{closed.fig.spiderplot} which we call a ``spider plot''.
The {\bf R} commands for obtaining the individual covariate ``distance from trap centroid''
are as follows:
\begin{verbatim}
library("scrbook")
data("beardata")
toad<- spiderplot(beardata$bearArray,beardata$trapmat)
xcent<-toad$xcent
\end{verbatim}
We picked $D_{max} = 11.5$ $km^2$ which is about the distance from the
array center to the furthest trap. 
Once we specific $D_{max}$ then the implication is that the population
size parameter applies to the area 
within 11.5 units of the trap centroid\footnote{To be convincing
  this might  need a little bit of hand-holding}. The {\bf BUGS} model
specification and {\bf R} commands to package the data and fit the model are
as follows:

{\small
\begin{verbatim}
cat("
model{
p0 ~ dunif(0,1)       # prior distributions
mup<- log(p0/(1-p0))
psi~dunif(0,1)
beta~dnorm(0,.01)

for(i in 1:(nind+nz)){
  xcent[i]~dunif(0,maxD)
  z[i]~dbern(psi)     # DA variables
  lp[i] <- mup + beta*xcent[i] # individual effect
  logit(p[i])<-lp[i]
  mu[i]<-z[i]*p[i]
  y[i]~dbin(mu[i],K)  #  observation model
 }
N<-sum(z[1:(nind+nz)])
}
",file="modelMcov.txt")

data2<-list(y=ytot,nz=nz,nind=nind,K=K,xcent=xcent,Dmax=maxD)
params2<-list('p0','psi','N','beta')
inits =  function() {list(z=zst, psi=psi, p0=runif(1),beta=rnorm(1) ) }
fit2 = bugs(data2, inits, params2, model.file="modelMcov.txt",working.directory=getwd(),    
       debug=T, n.chains=3, n.iter=11000, n.burnin=1000, n.thin=1)
\end{verbatim}
}

This produces the following posterior summaries:
{\small
\begin{verbatim}
Inference for Bugs model at "modelMcov.txt", fit using WinBUGS,
 3 chains, each with 11000 iterations (first 1000 discarded)
 n.sims = 30000 iterations saved
           mean    sd   2.5%    25%    50%    75%  97.5% Rhat n.eff
p0         0.54  0.07   0.40   0.50   0.54   0.59   0.67    1  1100
psi        0.34  0.05   0.25   0.31   0.34   0.37   0.44    1  3500
N         58.92  5.49  50.00  55.00  58.00  62.00  71.00    1  1900
beta      -0.25  0.06  -0.36  -0.29  -0.25  -0.21  -0.12    1   780
deviance 459.51 13.21 435.80 450.20 458.80 467.90 487.40    1  2600
\end{verbatim}
}


It might be 
perplexing that the estimated $N$ is much lower than obtained by model
$M_h$ but there is a good explanation for this, discussed
subsequently. That issue notwithstanding, it is worth pondering how
this model could be an improvement (conceptually or technically) over
some other model/estimator including $M_0$ and $M_h$ considered
previously. Well, for one, we have accounted formally for
heterogeneity due to spatial location of individuals relative to
exposure to the trap array, characterized by the centroid of the
array. Moreover, we have done so using a model that is based on an
explicit mechanism, as opposed to a phenomenological one such as Model
$M_h$. Moreover, importantly, using our new model, {\it the estimated N
  applies to an explicit area which is defined by our prescribed value
  of maxD}. That is, this area is a fixed component of the model and
the parameter $N$ therefore has explicit spatial context, as the number
of individuals with home range centers less than $D_{max}$ from the
centroid of the trap array. As such, the implied ``effective trap
area''\footnote{This is a bad use of this term. We have never defined
  ETA or ESA. What is it, exactly? XXX IT IS SOMEWHAT DEFINED IN CH1; IN THE QUOTE FROM OBRIEN; ALTHOUGH HE NAMES IT EFFECTIVE AREA XXX} for a given $D_{max}$ is that of a circle
with radius $D_{max}$.



\begin{figure}
\begin{center}
\includegraphics[width=3.5in]{Ch3/figs/Nchains}
\end{center}
\caption{Needs a caption}
\label{closed.fig.ha}
\end{figure}

\subsection{Extension of the Model}

This model is actually not a very good model for one important reason:
Imposing a uniform prior distribution on $x$
implies that density is {\it not constant} over space. In
particular, this model implies that it {\it decreases} as we move away
from the centroid of the trap array. 
That is, $x_{i} \sim \mbox{Unif}(0,D_{max})$ implies constant $N$ in
each distance band from the centroid but obviously the {\it area} of
each distance band is increasing.  
This is one reason we have a
lower estimate of density than that obtained previously from model $M_0$ and also why,
if we were to increase $D_{max}$, we would see density continue to
decrease.

Fortunately, the use of an individual covariate model is {\it not} restricted to
use of this specific distribution for the individual
covariate. Clearly, it is a bad choice and, therefore, we should think
about whether we can choose a better distribution for $D_{max}$ - one that
doesn't imply a decreasing density as distance from the centroid
increases.  Conceptually, what we want to do is impose a prior on
distance from the centroid, $x$, such that density is proportional to
the amount of area in each successive distance band as you move
farther away from the centroid.  In fact, there is theory that exists
which tells us what the correct distribution of $x$ is
$2x/D_{max}^2$. This can be derived by noting that $F(x) = \Pr(X<x) =
\pi*x*x/\pi*D_{max}^{2}$ . Then, $f(x) = dF/dx =
2*x/(D_{max}^{2})$. This is a sort of triangular distribution in
density
induced because the incremental area in each additional distance band
increases linearly with radius (i.e., distance from centroid). It is
sometimes comforting to verify things empirically:
{\small
\begin{verbatim}
 u<-runif(10000,-1,1)
 v<-runif(10000,-1,1)
 d<- sqrt(u*u+v*v)
 hist(d[d<1])
 hist(d[d<1],100)
 hist(d[d<1],100,probability=TRUE)
 abline(0,2)
\end{verbatim}
}

It would be useful if we could describe this distribution in {\bf BUGS} but
there is not a built-in way to do this that we are aware of.  One possibility is to use a
discrete version of the pdf. We might also be able to use what is
referred to in {\bf WinBUGS} jargon as the ``zeros trick'' (see {\it Advanced
BUGS tricks} in the manual) although we haven't pursued this approach. Instead, we
consider using a discrete version and break $D_{max}$ into $L$ distance
classes of width $\delta$, with probabilities proportional to
$2*x$. In particular, if we denote the cut-points by $xg_{1}=0,xg_{2}, \ldots,
xg_{L+1}=D_{max}$ and the interval midpoints are $xm_{i} = 
xg_{i+1}-\delta$ then the interval probabilities are $p_{i} = 
2*xm_{i}*\delta/(D_{max}^{2})$, which we can compute once and then pass
them to {\bf WinBUGS} as data.

The {\bf R} commands for doing all of this (noting that we have already loaded and processed
the Fort Drum bear data) are given  as follows. In the model description the
variable $x$ (observed distance from centroid of the trap array) has been rounded so that the
discrete version of the $f(x)$ can be used as described
previously. The new variable labeled \mbox{\tt xround} is actually
then the integer category label in units of $\delta$ from 0. Thus, to
convert back to distance in the expression for $lp[i]$, \mbox{\tt
  xround[i]} has to be multiplied by $\delta$. Here is the {\bf BUGS} model 
  specification:
{\small
\begin{verbatim}
delta<-.2
xround<-xcent%/%delta  + 1
Dgrid<- seq(delta,maxD,delta)
xprobs<- delta*(2*Dgrid/(maxD*maxD))
xprobs<-xprobs/sum(xprobs)

cat("
model{
p0 ~ dunif(0,1)       # prior distributions
mup<- log(p0/(1-p0))
psi~dunif(0,1)
beta~dnorm(0,.01)

for(i in 1:(nind+nz)){
  xround[i]~dcat(xprobs[])
  z[i]~dbern(psi)                     # zero inflation variables
  lp[i] <- mup + beta*xround[i]*delta # individual effect
  logit(p[i])<-lp[i]
  mu[i]<-z[i]*p[i]
  y[i]~dbin(mu[i],K)  #  observation model
 }

N<-sum(z[1:(nind+nz)])
}
",file="modelMcov.txt")
\end{verbatim}
}

To fit the model we do this - keeping in mind that the data objects
required below have been defined in previous analyses of this chapter:
{\small
\begin{verbatim}
data2<-list(y=ytot,nz=nz,nind=nind,K=K,xround=xround,xprobs=xprobs,delta=delta)
params2<-list('p0','psi','N','beta')
inits =  function() {list(z=z, psi=psi, p0=runif(1),beta=rnorm(1) ) }
fit = bugs(data2, inits, params2, model.file="modelMcov.txt",
          working.directory=getwd(), debug=FALSE, n.chains=3, n.iter=11000, 
          n.burnin=1000, n.thin=2)
\end{verbatim}
}

This is a useful model because it induces a clear definition of area
in which the population of $N$ individuals reside. Under this model,
that area is defined by specification of $D_{max}$. We can apply the model
for different values of $D_{max}$ and observe that the estimated $N$ varies
with $D_{max}$. Fortunately, we see empirically, that while $N$ seems
highly sensitive to the prescribed value of $D_{max}$, density seems to
be invariant to $D_{max}$ as long as it is chosen to be sufficiently
large. We fit the model for a random of values of $D_{max}$ from $D_{max}=12$ (restricting
values of $x$ to be in close proximity to
the trap array) on up to 20. The results are given in Table
\ref{closed.tab.Dmax}.


\begin{table}[htp]
\centering
\caption{Analysis of Fort Drum bear hair snare data using the individual covariate model, for different values of Dmax, the upper limit of the uniform distribution of `distance from centroid of the trap array' }
\begin{tabular}{ccc}
\hline \hline
 Dmax & mean & SD \\ \hline
  12& 0.230 & 0.038 \\
  15& 0.244 &0.041 \\
  17& 0.249 &0.044 \\
  18& 0.249 &0.043\\
  19& 0.250 &0.043\\
  20& 0.250 &0.044
\end{tabular}
\label{closed.tab.Dmax}
\end{table}


We see that the posterior mean and SD of density (individuals per
square km) appear insensitive to choice of $D_{max}$ once we get a 
ways away from the maximum observed value of about 11.5. The estimated
density of 0.25 per km$^2$ is actually quite a bit lower than we 
reported using model $M_h$ 
for which  no relevant ``area'' quantity is explicit in the model.
Using MLEs of $N$ in conjunction with buffer strips
(see Table \ref{intro.tab.fdtests}) our estimates were in the range of $0.32-0.43$ and
the Bayesian estimates were XXXX (posterior mode of N = 102) or XXX (posterior mean of N = 117)
(see sec.
\ref{closed.sec.modelmh} above). 
On the other hand our estimate of $\hat{D} = 0.25$ here (based on the posterior mean) is 
higher than that reported from model $M_0$ using the buffered area
(0.18). There is no basis really for comparing or contrasting these
various estimates and it would be a useful philosophical exercise for
the reader to discuss this matter. In particular, application of models
$M_0$ and $M_h$ are distinctly {\it not} spatially explicit models -- the
area within which the population\footnote{We need to look back at
  Chapter 1 and make sure we quit calling this ``sample area'' - it
  really isn't that at al, but rather the area within which $N$
  resides.} resides is not defined under either model. There is
therefore no reason at all to think that the estimates produced under
either either closed population model, based on a buffered ``trap area'', 
are justifiable by any
theory. In fact, we would get exactly the same estimate of $N$ no
matter what we declare the area to be. On the other hand, the
individual covariate model explicitly describes a distribution for
``distance from centroid'' that is a reasonable and standard null
model - it posits, in the absence of direct information, that
individual home range centers are randomly distributed in space and
that probability of detection depends on the distance between home
range center and the centroid of the trap array. Under this definition
of the system, we see that density is invariant to the choice of
sample area which seems like a desirable feature. 

The individual
covariate model is not ideal, however, because it does not make full
use of the spatial information in the data set, i.e., the trap
locations and the locations of each individual encounter, and there is hope
to extend this model in order to resolve remaining deficiencies.


\subsection{Invariance of density to $D_{max}$}

Under the model above, and also under models that we consider in later
chapters, a general property of the estimators is that while $N$
increases with the prescribed trap area (equivalent to $D_{max}$ in this
case), we expect that density estimators should be invariant to this
area. In the model used above, we note that $Area(D_{max}) = 
\pi*D_{max}^{2}$ and $E[N(D_{max})] = \lambda*Area(D_{max})$ and thus
$E[Density(D_{max})] = \lambda$, i.e., constant. This should be 
interpreted as the {\it prior} density. Absent data, then realizations
under the model will have density $\lambda$ regardless of what $D_{max}$
is prescribed to be.  As we verified empirically above, the posterior
density is also invariant Of $D_{max}$ as long as the implied area
is large enough so that the data no longer provide
information about density (i.e., ``far away'').

\subsection{Toward Fully Spatial Capture-recapture Models}

We developed this model for the average observed location and equated
it to home range center ${\bf s}_{i}$. Intuitively, taking the average
encounter location as an estimate of home range center makes sense but
more so when the trapping grid is dense and expansive relative to
typical home range sizes.  However, our approach also ignored the
variable precision with which each ${\bf s}_{i}$ is estimated and also, as
noted previously, estimates of ${\bf s}_{i}$ around the ``edge'' (however we
define that) are biased because the observations are truncated (we can
only observe locations within the trap array).  In the next chapter we
provide a further extension of this individual covariate model that
definitively resolves the ad hoc nature of the individual covariate
approach we took here. In that chapter we build a model in which ${\bf s}_{i}$
are regarded as latent variables and the observation locations (i.e.,
trap specific encounters) are linked to those latent variables with an
explicit model. We note that the model fitted previously could be
adapted easily to deal with ${\bf s}_{i}$ as a latent variable, simply by
adding a prior distribution for ${\bf s}_{i}$. The reader should contemplate
how to do this in {\bf BUGS}.


\section{DISTANCE SAMPLING: A primative Spatial Capture-Recapture Model}

Distance sampling is one of the most popular methods for estimating
animal abundance. One of the great benefits of distance sampling is
that it provides explicit estimates of {\it density}. The distance
sampling model is a special case of a closed population model with a
covariate. The covariate in this case, $x_{i}$, is the distance
between an individual's location ``$u$'' and the observation location
or transect. In fact, the model underlying distance sampling is
precisely the same model as that which applies to the
individual-covariate models, except that observations are made at only
$K=1$ sampling occasion. In a sense, distance sampling is a spatial
capture-recapture model, but without the ``recapture.''  This first
and most basic spatial capture-recapture model has been used routinely
for decades and, formally, it is a spatially-explicit model in the
sense that it describes, explicitly, the spatial organization of
individual locations (although this is not always stated explicitly)
and, as a result, somewhat general models of how individuals are
distributed in space can be specified \citep{royle_etal:2004,
  johnson_etal:2010, sillett_etal:2011}.

As before, the distance sampling model, under data augmentation,
includes a set of $M$ zero-inflation variables $z_{i}$ and the
binomial model expressed conditional on $z$ (binomial for $z=1$, and
fixed zeros for $z=0$).  In distance sampling we pay for having only a
single sample (i.e., $K=1$) by requiring constraints on the model of
detection probability. A standard model is
\[
\log(p_{i}) = \beta x_{i}^{2}
\]
for $\beta < 0$, where $x_i$ denotes the distance at which the $i$th
individual is detected relative to some reference location where
perfect detectability ($p=1$) is assumed. This function corresponds to
the ``half-normal'' detection function (i.e., with $\beta =
1/\sigma^{2}$).  If $K>1$ then an intercept in this model is
identifiable and
such models are usually called ``capture-recapture distance
sampling''\citep{alpizar_pollock:1996,borchers_etal:1998}.

As with previous examples, we require a distribution for the individual covariate $x_{i}$. The customary choice is
\[
x_{i} \sim \mbox{Unif}(0,B)
\]
wherein $B>0$ is a known constant, being the upper limit of data
recording by the observer (i.e., the point count radius, or transect
half-width). In practice, this is sometimes asserted to be infinity,
but in such cases the distance data are usually truncated.
Specification of this distance sampling model in the {\bf BUGS} language is
shown in Panel \ref{closed.panel.distance} from \citet{royle_dorazio:2008}.


\begin{panel}[htp]
\centering
\rule[0.15in]{\textwidth}{.03in}
\begin{minipage}{5in}
\begin{verbatim}
beta~dunif(0,10)
psi~dunif(0,1)

for(i in 1:(nind+nz)){
   z[i]~dbern(psi)    # DA Variables
   x[i]~dunif(0,B)    # B=strip width
   p[i]<-exp(logp[i])   # DETECTION MODEL
   logp[i]<-   - beta*(x[i]*x[i])
   mu[i]<-z[i]*p[i]
   y[i]~dbern(mu[i])  # OBSERVATION MODEL
 }
N<-sum(z[1:(nind+nz)])
D<- N/striparea  # area of transects
\end{verbatim}
\end{minipage}
\rule[-0.15in]{\textwidth}{.03in}
\caption{Distance sampling model in {\bf BUGS}, using a half-normal
detection function.}
\label{closed.panel.distance}
\end{panel}

As with the individual covariate model in the previous section, the
distance sampling model can be equivalently specified by putting a
prior distribution on individual {\it location} instead of distance
between individual and observation point (or transect).  Thus we can
write the general distance sampling model as
\[
p_{i} = f(\beta,||{\bf u}_{i} - {\bf x}_0||)
\]
along with
\[
 {\bf u}_{i} \sim \mbox{Unif}({\cal S})
\]
where ${\bf x}_{0}$ is a fixed point (or line) and ${\bf u}_{i}$ is
the individual's location which is observable for $n$ individuals. In
practice it is easier to record distance instead of location.  Basic
math can be used to argue that if individuals have a uniform
distribution in space, then the distribution of Euclidean distance is
also uniform. In particular, if a transect of length $L$ is used and $x$
is distance to the transect then $F(x) = \Pr(X\le x) = L*x/L*B = x/B$ and
$f(x) = dF/dx = (1/B)$. For measurements of radial distance, see the
previous section.

In the context of our general characterization of SCR models 
(Chapt. \ref{modeling.sec.characterization}),
we suggested that every SCR model can be described,
conceptually, by a hierarchical model of the form:
\[
 [y|u][u|s][s].
\]
Distance sampling ignores the part of the model pertaining to ${\bf
  s}$, and deals only with the model components for the observed
data  ${\bf u}$\footnote{Equivalently, we could also say that $[u]$ in
  the distance sampling model is $[u] = \int [u|{\bf s}][{\bf s}]
  d{\bf s}$}. Thus, we are left with a hierarchical model of the form
\[
[y|{\bf u}][{\bf u}].
\]
In contrast, as we will see in the next chapters, basic SCR models
(Chapt. \ref{chapt.scr0}) ignore ${\bf u}$ and condition on ${\bf s}$,
which is not observed:
\[
[y|{\bf s}][{\bf s}]
\]
Since $[{\bf u}]$ and $[{\bf s}]$ are both assumed to be uniformly
distributed, these are structurally equivalent models! The main
differences have to do with interpretation of model components and
whether or not the latent variables are observable (in distance
sampling they are).

So why bother with SCR models when distance sampling yields density
estimates and accounts for spatial heterogeneity in detection? For
one, imagine trying to collect distance sampling data on tigers!
Clearly, distance sampling requires that one can collect large
quantities of distance data, which is not always possible. For tigers,
it is much easier, efficient, and safer to employ camera traps or
tracking plates and then apply SCR models. Furthermore, as we will see
in Chapts.
\ref{chapt.searchencounter} and \ref{chapt.scrds}, SCR models can use distance data to estimate all the
parameters of our enchilada, allowing us to study distribution,
movement, and density. Thus, SCR models are much more general and
versatile than distance sampling models (which clearly are a special
case), and can accommodate data from virtually all animal survey
designs.


\subsection{Example: Muntjac deer survey from Nagarahole, India }

Here we fit distance sampling models to distance sampling data on the
muntjac deer (Muntiakus muntjak) collected in the year 2004 from
Nagarahole National Park in southern India
(Kumar et al. unpublished data). The muntjac is
a solitary species and distance measurements were made on 57 groups
that were largely singletons with 4 pairs of individuals.  Commands
for reading in and organizing the data for {\bf WinBUGS}, followed by
writing the model to a text file, are given below. Note that the total sampled area of
the transects is fed in as ``striparea'' which is $708$ (km of transect walked)
multiplied by the strip width ($B=120 = 0.12$ km) multiplied by 2.
{\small 
\begin{verbatim}
library("R2WinBUGS")
data<- read.csv("Muntjac.csv")
hist(data[,3],30)
nind<-nrow(data)
y<-rep(1,nind)
nz<-400
y<-c(y,rep(0,nz))
x<-data[,3]
x<-c(x,rep(NA,nz))
z<-y

cat("
model{
beta~dunif(0,10)
psi~dunif(0,1)

for(i in 1:(nind+nz)){
   z[i]~dbern(psi)    # DA Variables
   x[i]~dunif(0,B)    # B=strip width
   p[i]<-exp(logp[i])   # DETECTION MODEL
   logp[i]<-   -beta*(x[i]*x[i])
   mu[i]<-z[i]*p[i]
   y[i]~dbern(mu[i])  # OBSERVATION MODEL
 }
N<-sum(z[1:(nind+nz)])
D<- N/striparea  # area of transects
}
",file="dsamp.txt")
\end{verbatim}
}
Next, we provide inits, indicate which parameters to monitor, and then
pass those things to {\bf WinBUGS}:
{\small
\begin{verbatim}
data<-list(y=y,x=x,nz=nz,nind=nind,B=120,striparea=(708*2*.120))
params<-list('beta','N','D','psi')
inits =  function() {list(z=z, psi=runif(1), beta=runif(1,0,.02) )}
fit = bugs(data, inits, params, model.file="dsamp.txt",working.directory=getwd(),    
       debug=T, n.chains=3, n.iter=11000, n.burnin=1000, n.thin=2)
\end{verbatim}
}
Posterior summaries are provided in the following table. Estimated
density is pretty low, 1.1 individuals per sq. km.\footnote{ This is much
  lower than Samba's estimate produced from WinBUGS accounting for group
  size. Reason unknown. }
{\small
\begin{verbatim}
Inference for Bugs model at "dsamp.txt", fit using WinBUGS,
 3 chains, each with 11000 iterations (first 1000 discarded), n.thin = 2
 n.sims = 15000 iterations saved
           mean    sd   2.5%    25%    50%    75%  97.5% Rhat n.eff
beta       0.00  0.00   0.00   0.00   0.00   0.00   0.00    1  1100
N        185.73 26.53 138.00 167.00 184.00 203.00 242.00    1   570
D          1.09  0.16   0.81   0.98   1.08   1.20   1.42    1   570
psi        0.41  0.06   0.30   0.36   0.40   0.45   0.54    1   670
deviance 655.74 16.26 626.00 644.50 655.10 666.40 689.80    1  1300

[.... some output deleted .... ]
\end{verbatim}
}

\section{Summary and Outlook}

Traditional closed population capture-recapture models are closely
related to binomial generalized linear models.  Indeed, the only real
distinction is that in capture-recapture models, the population size
parameter $N$ (corresponding also to the size of a hypothetical
``complete'' data set) is unknown.  This requires special
consideration in the analysis of capture-recapture models. The
classical approach to inference recognizes that the observations don't
have a standard binomial distribution but, rather, a truncated
binomial (from which which the so-called ``conditional likelihood''
derives) since we only have encounter frequency data on observed
individuals. If instead we analyze the models using data augmentation,
the observations can be modeled using a zero-inflated binomial
distribution. In short, when we deal with the unknown-$N$ problem using
data augmentation then we are left with zero-inflated GLM and GLMMs
instead of ordinary GLM or GLMMs. The analysis of such zero-inflated
models is practically convenient, especially using the various
Bayesian analysis packages that use the {\bf BUGS} language.

Spatial capture-recapture models that we will consider in the rest of
the chapters of this book are closely related to what have been called
individual covariate models. Heuristically, spatial capture-recapture
models arise by defining individual covariates based on observed
locations of individuals -- we can think of using some function of
mean encounter location as an individual covariate. We did this in a
novel way, by using distance to the centroid of the trapping array as
a covariate. We analyzed the ``full likelihood'' using data
augmentation, and placed a prior distribution on the individual
covariate which was derived from an assumption that individual
locations are, a priori, uniformly distributed in space. This
assumption provides for invariance of the density estimator to the
choice of population size area (induced by maximum distance from the
centroid of the trap array). The model addressed some important problems in the
use of closed population models: it allows for heterogeneity in
encounter probability due to the spatial context of the problem and it
also provides a direct estimate of density because area is a feature
of the model (via the prior on the individual covariate). The model is
still not completely general because it does not make use of
the fully spatial encounter histories, which provide direct
information about the locations and density of individuals.  A
specific individual covariate model that is in widespread use is
classical ``distance sampling.'' The model underlying distance
sampling is precisely a special kind of SCR model - but one without
replicate samples. Understanding distance sampling and individual
covariate models more broadly provides a solid basis for understanding
and analyzing spatial capture-recapture models.



\chapter{Fully Spatial Capture-Recapture Models}
\label{chapt.scr0}

%%% TO DO  as of 12/29/11

 %%% Spell check document

 %%% Change "beta" to "theta"

 %%% Fix up R scripts and consolidate for R package
 %%% R commands to process wolverine data need included in that section

 %%% Run Wolverine 2k 4k and 8k grids in JAGS compare to WinBUGS
 %%%     insert those results in text

 %%%  For discrete state-space stuff, convert BUGS output to JAGS and
 %%%  figure out MC errors
 %%% Finish Table that has those results in it

 %% pick up all hard references to chapters and make float


\chapter{Fully Spatial Capture-Recapture Models}
\markboth{Chapter 4 }{}
\label{chapt.scr0}

\vspace{.3in}

In previous sections we discussed some classes of models that could be
viewed as primitive spatial capture-recapture models. We looked at a
basic distance sampling model and we also considered a classical
individual covariate modeling approach in which we defined a covariate
to be the distance from (estimated) home range center to the center of
the trap array. These were spatial in the sense that they included
some characterization of where individuals live but, on the other
hand, only a primitive or no characterization of trap location.  That
said, very little distinguishes these two models from spatial
capture-recapture models that we consider in this chapter which fully
recognize the spatial attribution of both individual animals {\it and}
the locations of encounter devices.

Fully spatial capture-recapture models must accommodate the spatial
organization of individuals and the encounter devices because the
encounter process occurs at the level of individual traps.  Failure to
consider the trap-specific collection of data is the key deficiency
with classical ad-hoc approaches which aggregate encounter information
to the resolution of the entire trap array. We have  previously
addressed some problems that this induces including induced
heterogeneity in encounter probability, imprecise notation of ``sample
area'' and not being able to accommodate trap-specific
effects.
In this chapter we resolve these issues by developing 
our first fully spatial capture-recapture
model which turns out to be precisely the model considered in sec. \ref{closed.sec.indcov}
 but instead of defining the individual covariate to be distance
to centroid of the array we define $J$ individual covariates - the
distance to {\it each} trap. And, instead of using estimates of
individual locations ${\bf s}$, we consider a fully hierarchical model in
which we regard ${\bf s}$ as a latent variable and impose a prior
distribution on it.  We can think of having $J$ independent
capture-recapture studies generating one data set for each trap, and
applying the individual covariate model with random activity centers,
and that is all the basic SCR model is.

In the following sections of this chapter we investigate the basic
spatial capture-recapture model, which we refer to as ``model SCR0'',  and address some important
considerations related to its analysis in {\bf WinBUGS}. We also demonstrate
how to summarize posterior output for the purposes of producing
density maps or spatial predictions of density.

\section{Sampling Design and Data Structure}

In our development here, we will assume a standard sampling design in
which an array of $J$ traps is operated for $K$ time periods (say,
nights) producing encounters of $n$ individuals.  Because sampling
occurs by traps and also over time, the most general data structure
yields encounter histories for {\it each individual} that are
temporally {\it and} spatially indexed. Thus a typical data set will
include an encounter history {\it matrix} for each individual.  For
the most basic model, there are no time-varying covariates that
influence encounter, there are no explicit individual-specific
covariates, and there are no covariates that influence density we will
develop models in this chapter for encounter data that are aggregated
over the temporal replicates. For example, suppose we observe 6
individuals in sampling at 4 traps over 3 nights of sampling then a
plausible data set is the $6 \times 4$ matrix of encounters, out of 3,
of the form:
\begin{verbatim}
      trap1 trap2 trap3 trap4
 [1,]     1     0     0     0
 [2,]     0     2     0     0
 [3,]     0     0     0     1
 [4,]     0     1     0     0
 [5,]     0     0     1     1
 [6,]     1     0     1     0
\end{verbatim}

We develop models in this chapter for devices such as ``hair snares''
or other DNA sampling methods \citep{kery_etal:2010,
  gardner_etal:2010jwm} and related types of sampling devices in which
(i) effective ``traps'' may capture any number of individuals (i.e.,
they don't fill up; This is referred to as a ``multi-catch'' type of
sampling \citep{efford_etal:2009ecol}); (ii) an individual may be
captured in any number of traps during each occasion but (iii)
individuals can be encountered at most 1 time in a trap during any
occasion.  The statistical assumptions are that individual encounters
within and among traps are independent, and this allows us to regard
individual- and trap-specific encounters as $iid$ Bernoulli trials
(see next section).  These basic (but admittedly at this point
somewhat imprecise) assumptions define the basic spatial
capture-recapture model, which we will refer to as ``SCR0'' 
so that we may use that model as a point of reference without having
to provide a long-winded enumeration of assumptions and sampling
design each time we do. We will make things more precise as we develop
a formal statistical definition of the model shortly.

While the model is mostly directly relevant
to hair snares and other DNA sampling methods for which multiple
detections of an individual are not distinguishable,
we will also make use of the model for data that arise from
camera-trapping studies. In practice, with camera trapping,
individuals might be photographed several times in a night but we will
typically distill such data into a single binary encounter event for
reasons discussed later in Chapt. \ref{chapt.poisson-mn}.


\section{The binomial observation model }

We assume that the individual and trap-specific encounters, $y_{ij}$,
are mutually independent outcomes of a binomial random variable:
\begin{equation}
	y_{ij} \sim \mbox{Bin}(K, p_{ij})
\label{scr0.eq.bin}
\end{equation}
This is the basic model underlying ``logistic regression'' (Chapt. \ref{chapt.glms})
as well as standard closed population models
(Chapt. \ref{chapt.closed}). The key
element of the model is that the encounter probability $p_{ij}$ is
indexed by (i.e., depends on) both individual and trap. In a sense,
then, we can think of each {\it trap} as producing individual level
encounter history data of the classical variety - an $\mbox{\tt nind}
\times \mbox{\tt nreps}$
matrix of 0's and 1's (this is the ``encountered at most 1 time''
assumption).


As we did in sec. \ref{closed.sec.indcov}, we will make explicit the notion that
$p_{ij}$ is defined conditional on ``where'' individual $i$
lives. Naturally, we think about defining an individual home range and
then relating $p_{ij}$ explicitly to the centroid of the individuals
home range, or its center of activity \citep{efford:2004,
  borchers_efford:2008, royle_young:2008}.  Therefore, define ${\bf
  s}_{i}$, a two-dimensional spatial coordinate, to be the activity
center for individual $i$. Then, the SCR model postulates that
encounter probability, $p_{ij}$, is a decreasing function
of distance between ${\bf s}_{i}$ and the location of trap $j$, ${\bf x}_{j}$.
 Naturally, if we think of modeling binomial counts using
logistic regression, we might specify the model according to:
\begin{equation}
	\mbox{logit}(p_{ij}) = \alpha_{0} + \alpha_1 ||{\bf s}_{i}-{\bf x}_{j} ||
\label{scr0.eq.logit}
\end{equation}
where, here, $||{\bf s}_{i}-{\bf x}_{j}||$ is the distance between
${\bf s}_{i}$ and ${\bf x}_{j}$. We sometimes write $||{\bf
  s}_{i}-{\bf x}_{j}|| = dist({\bf s}_{i},{\bf x}_{j}) =
d_{ij}$. Alternatively, if we think about distance sampling then we
might use the ``half-normal'' model of the form:
\[
p_{ij} = p_{0}*\exp(-\alpha_{1} *||{\bf s}_{i}-{\bf x}_{j}||^2)
\]
Or any of a large number of standard detection models that are
commonly used (we consider more in Chapt. \ref{chapt.covariates}). The half-normal model implies
\begin{equation}
\log(p_{ij})  = \log(p_{0}) - \alpha_{1} *||{\bf s}_{i}-{\bf x}_{j}||^2
\label{scr0.eq.norm}
\end{equation}
%We would always like to be clear that encounter probability depends on individual activity
%centers {\it and} trap locations {\it and} parameter(s) $\theta$, and
%so it would be ideal to write $p({\bf s}_{i},{\bf x}_{j}; \theta)$ or
%something similar. However, this can be extremely unwieldy and
%clutter up what are otherwise extremely simple mathematical
%expressions and formulae. As such, we will usually abbreviate these
%various dependencies by writing $p_{ij}$ or sometimes $p_{\theta,ij}$,
%understanding that $p_{ij}$ is actually a function of the various important
%quantities.
We probably expect that the parameter $\alpha_{1}$ in
Eq. \ref{scr0.eq.logit} or \ref{scr0.eq.norm} should be negative, so
that the probability of encounter decreases with distance between the
trap and individual home range center.  
Whatever model encounter probability we choose, we should always keep
in mind that the model is described conditional on ${\bf s}_{i}$,
which is an unobserved random variable.  Thus, to be precise about
this, we should write the observation model as
\[
y_{ij}|{\bf s}_{i} \sim \mbox{Bin}(K, p({\bf s}_{ij};\alpha_{1}))
\]


The joint likelihood for the
data, conditional on the collection of individual activity centers,
can therefore be expressed as
\[
{\cal L}(\alpha_{1} | \{ {\bf y}_{i},{\bf s}_{i} \}_{i=1}^{N})
 =  \prod_{i} \prod_{j} \mbox{Bin}(y_{ij}|p_{ij}(\alpha_{1}))
\]
Which, if we switch the indices on the product operators, this shows
the SCR likelihood (conditional on ${\bf s}$) to be the product of $J$
{\it independent} capture-recapture likelihoods - one for each trap.
However, the data have a distinct ``repeated measures'' type of structure, with
each of the $j$ likelihood contributions for each individual being
grouped by individual. Thus, we cannot analyze the model
meaningfully by $J$ trap-specific models. In classical repeated measures
types of models, we accommodate the group structure of the data using
random effects (random individual or group level variables). For SCR
models we take the same basic approach, which we develop subsequently.

\subsection{Distance as a latent variable}

If we knew precisely every ${\bf s}_{i}$ in the population (and how
many, $N$), then the model specified by eqs. \ref{scr0.eq.bin} and
\ref{scr0.eq.logit} is just an ordinary logistic
regression type of a model which we learned how to fit using {\bf
  WinBUGS} previously (Chapt. \ref{chapt.glms}), with a covariate $d_{ij}$. However,
the activity centers are unobservable even in the best possible
circumstances. In that case, $d_{ij}$ is an unobserved variable,
analogous to classical random effects models. We need to therefore
extend the model to accommodate these random variables with an
additional model component. A standard, and perhaps not unreasonable,
assumption is the so-called ``uniformity assumption'' which is to say
that the ${\bf s}_{i}$ are uniformly distributed over space (the
obvious next question ``which space?'' is addressed below).  This
uniformity assumption amounts to a uniform prior distribution on ${\bf
  s}_{i}$, i.e., the pdf of ${\bf s}_{i}$ is constant, which we may
express
\begin{equation}
	\Pr({\bf s}_{i}) \propto \mbox{\tt const}
\label{scr0.eq.sprior}
\end{equation}
 As it turns out, this assumption is usually not precise
enough to fit SCR models in practice for reasons we discuss in the
following section.  We will give another way to represent this prior
distribution that is more concrete, but it depends on specifying the
``state-space'' of the random variable ${\bf s}_{i}$. The term
state-space is a technical way of saying ``possible outcomes''.

To summarize the preceeding model developing, a basic SCR model is
defined by 3 essential components:
\begin{itemize}
\item[(1)] Observation model: $y_{ij}|{\bf s}_{i} \sim \mbox{Bin}(K, p_{ij})$
\item[(2)] Encounter probability: $\mbox{logit}(p_{ij}) = \alpha_{0} +
  \alpha_{1}*||{\bf s}_{i}-{\bf x}_{j}||$
\item[(3)] Point process model: $\Pr({\bf s}_{i} ) \propto \mbox{\tt const}$
\end{itemize}
Therefore, the SCR model is little more than an ordinary
capture-recapture model for closed populations. It is such a model,
but augmented with a set of ``individual effects'', ${\bf s}_{i}$,
which relate some sense of individual location to encounter
probability. 

\section{ The Binomial Point-process Model}

The collection of individual activity centers ${\bf s}_{1},\ldots,
{\bf s}_{N}$ represent a realization of a {\it binomial point process}
\citep[][p. xyz]{illian_etal:2008}.  The binomial point process (BPP)
is analogous to a Poisson point process in the sense that it
represents a ``random scatter'' of points in space - except that the
total number of points is {\it fixed}, whereas, in a Poisson point
process it is random (having a Poisson distribution).  As an example,
we show in Fig. \ref{scr0.fig.bpp} locations of 20 individual activity
centers (black dots) in relation to a grid of 25 traps. For a Poisson
point process the number of such points in the prescribed state-space
would be random whereas often we will simulate fixed numbers of
points, e.g., for evaluating the performance of procedures such as how
well does our estimator perform of $N=50$?
\begin{figure}
\begin{center}
\includegraphics[height=2.5in]{Ch4/figs/binomialpoint}
\end{center}
\caption{Realization (small circles) of a binomial point process with $N=20$. The
  large circles represent trap locations.}
\label{scr0.fig.bpp}
\end{figure}

It is natural to consider a binomial point process in the context of
capture-recapture models because it preserves $N$ in the model and thus
preserves the linkage directly with closed population models. In fact,
under the binomial point process model then model $M_0$ and other closed
models are simple limiting cases of SCR models, i.e., as the
coefficient on distance tends to 0.
In addition, use of
the BPP model allows us to use data augmentation for Bayesian analysis
of the models as in Chapt. \ref{chapt.closed}, thus yielding a methodologically
coherent approach to analyzing the different classes of
models. Despite this, making explicit assumptions about $N$, such as
Poisson, is convenient in some cases (see Chapt. \ref{chapt.hscr}).

One consequence of having fixed $N$, in the BPP model, is that the
model is not strictly a model of ``complete spatial randomness''. This
is because if one forms counts $n(A_{1}),\ldots, n(A_{k})$ in any set
of disjoint regions say $A_{1}, \ldots, A_{k}$, then these counts are
{\it not} independent.  In fact, they have a multinomial distribution
\citep[see][p. XYZ]{illian_etal:2008}. Thus, the BPP model introduces
a slight bit of dependence in the distribution of points. However, in
most situations this will have no practical effect on any inference or
analysis and, as a practical matter, we will usually regard the BPP
model as one of spatial independence among individual activity centers
because each activity center is distributed independently of each
other activity center. Despite this implicit independence we see in
Fig. \ref{scr0.fig.bpp} that {\it realizations} of randomly distributed
points will typically exhibit distinct non-uniformity. Thus,
independent, uniformly distributed points will almost never appear
regularly, uniformly or systematically distributed. For this reason,
the basic binomial (or Poisson) point process models are enormously
useful in practical settings.  More relevant for SCR models is that we
actually have a little bit of data for some individuals and thus the
resulting posterior point pattern can deviate strongly from
uniformity, a point we come back to repeatedly in this book.
The uniformity hypothesis is only
a {\it prior} distribution which is directly affected by the quantity
and quality of observations, to produce a posterior distribution which
may appear distinctly non-uniform.


\subsection{Definition of home range center}

Some will be offended by our use of the concept of ``home range
center'' and thus will have difficulty in believing that the resulting
model is really useful for anything.  Indeed, the idea of a home range
or activity center is a vague concept anyway, a purely
phenomenological construct.  Despite this, it doesn't really matter
whether or not a home range makes sense for a particular species -
individuals of any species inhabit {\it some} region of space and we
can define the ``home range center'' to be the center of the space
that individual was occupying (or using) during the period in which
traps were active. Thinking about it in that way, it could even be
observable (almost) as the centroid of a very large number of radio
fixes over the course of a survey period or a season.  Thus, this
practical version of a home range center in terms of space usage is a well-defined construct
regardless of whether one thinks the home range concept is meaningful,
even if individuals are not particularly territorial.  This is why we
usually use the term ``activity center'' or maybe even ``centroid of
space usage'' and we recognize that this construct is a transient
thing which applies only to a well-defined period of study.



\subsection{The state-space of the point process}

Shortly we will focus on Bayesian analysis of this model with $N$
known so that we can directly apply what we learned in
Chapt. \ref{chapt.glms} to 
this situation. To do this, we note that the individual effects ${\bf
  s}_{i},\ldots, {\bf s}_{N}$ are unknown quantities and we will need
to be able to simulate each ${\bf s}_{i}$ in the population from the
posterior distribution.  It should be self-evident that we cannot
simulate the ${\bf s}_{i}$ unless we describe precisely the region
over which they are uniformly distributed. This is
the quantity referred to above as the state-space, denoted henceforth
by ${\cal S}$, which is a region or a set of points comprising the
potential values of ${\bf s}_{i}$. Thus, an equivalent explicit
statement of the ``uniformity assumption'' is
\[
{\bf s}_{i} \sim \mbox{Unif}({\cal S})
\]
where ${\cal S}$ is a precisely defined region. e.g., in Fig. 
\ref{scr0.fig.bpp}, ${\cal S}$ is the square defined by $[-1,7] \times
[-1, 7]$. Thus each of the $N=20$ points were generated by randomly
selecting each coordinate on the line $[-1, 7]$. 


\subsubsection{Prescribing the state-space}

Evidently, we need to define the state-space, ${\cal S}$. How can we
possibly do this objectively? Prescribing any particular ${\cal S}$
seems like the equivalent of specifying a ``buffer'' which we
criticized previously as being ad hoc. How is it, then, is choosing a
state-space is {\it not} ad hoc? As a practical matter, it turns out
that estimates of density are insensitive to choice of the
state-space. As we observed in Chapt. \ref{chapt.closed}, it is true that $N$ increases
with ${\cal S}$, but only at the same rate as the area of ${\cal S}$
increases under the
prior assumption of constant density. As a result, we say that density
is invariant to ${\cal S}$ as long as ${\cal S}$ is sufficiently
large. Thus, while choice of ${\cal S}$ is (or can be) essentially
arbitrary, once ${\cal S}$ is chosen, it defines the population being
exposed to sampling, which scales appropriately with the size of the
state-space.

For our simulated system developed previously in this chapter, we
defined the state space to be a square within which our trap array was
centered. For many practical situations this might be an
acceptable approach to defining the state-space. We provide an example
of this in sec. \ref{scr0.sec.wolverine} below in which the trap array is
irregular and also situated within a realistic landscape that is
distinctly irregular.  In general, it is most practical to define the
state-space as a regular polygon (e.g., rectangle) containing the trap
array without differentiating unsuitable habitat. Although defining
the state-space to be a regular polygon has computational advantages
(e.g., we can implement this more efficiently in {\bf WinBUGS} and
cannot for irregular polygons), a regular polygon induces an apparent
problem of admitting into the state-space regions that are distinctly
non-habitat (e.g., oceans, large lakes, ice fields, etc.).  It is
difficult to describe complex sets in mathematical terms that can be
used in {\bf BUGS}. As an alternative, we can provide a
representation of the state-space as a discrete set of points (sec.
\ref{scr0.sec.discrete}) that will allow specific points to be deleted
or not depending on whether they represent habitat, or we can define
the state-space as an arbitrary  collection of polygons stored as a GIS
shapefile
which can be analyzed easily using MCMC
(see sec. \ref{mcmc.sec.state-space}), but not so easily in the {\bf
  BUGS} variants.  In what follows below we provide an
analysis of the camera data defining the state-space to be a regular
continuous polygon (a rectangle).


\subsection{Invariance and the State-space as a model assumption}
\label{scr0.sec.invariance}

We will assert for all models we consider in this book that density is
invariant to the size and extent of ${\cal S}$, if ${\cal S}$ is
sufficiently large as long 
as our model relating $p_{ij}$ to ${\bf  s}_{i}$ is a decreasing
function of distance.  
We can prove this easily by drawing an analogy with a 1-d case such as
in distance sampling.  Let $y_{j}$ be the number of individuals
captured in some interval $[d_{j-1},d_{j})$, and define $d_{J} = B$
for some large value of $B$.  By choosing $B$ large enough we
guarantee that $E[y_{J+1}] = 0$ and therefore this ``last cell'' 
contributes nothing to
the likelihood
in regular situations in which the detection function decays
monotonically with distance and prior density is constant.  


Sometimes
our estimate of density can be influenced if we make ${\cal S}$ too small but
this might be sensible if ${\cal S}$ is naturally well-defined. As we discussed
in chapter 1, {\bf choice of ${\cal S}$ is part of the model and thus it makes
  sense that estimates of density might be sensitive to its definition
  in problems where it is natural to restrict ${\cal S}$}.
One could imagine
however that in specific cases where you're studying a small
population with well-defined habitat preferences that a problem could
arise because changing the state-space around based on differing
opinions and GIS layers really changes the estimate of total
population size. But this is a real biological problem and a natural
consequence of the spatial formalization of capture-recapture models -
a feature, not a bug or some statistical artifact - and it should be
resolved with better information, research, and thinking.
 For situations where there is not a natural
choice of ${\cal S}$, we should default to choosing ${\cal S}$ to be very large in order
to achieve invariance or otherwise evaluate sensitivity of density
estimates by trying a couple of different values of ${\cal S}$. This is a
standard ``sensitivity to prior'' argument that Bayesians always have
to be conscious of.  We demonstrate this in our analysis of section
\ref{scr0.sec.wolverine}
below. Note that $area({\cal S})$ affects data augmentation. If you
increase $area({\cal S})$ then there are more individuals to account for and
therefore the size of the augmented data set $M$ must increase.

We have been told that one can carry-out non-Bayesian analyses of SCR
models without having to specify the state-space of the point process
or perhaps while only specifying it imprecisely.  This assertion is
incorrect. We assume people are thinking this because {\it they} don't
have to specify it explicitly because someone else has done it for
them in a package that does integrated likelihood. Even to do
integrated likelihood (see Chapt. \ref{chapt.mle}) we have to integrate the
conditional-on-${\bf s}$ likelihood over some 2-dimensional space.  It might
work that the integration can be done from $-\infty$ to $+\infty$ but
that is a mathematical artifact of specific detection functions, and
an implicit definition of a state-space that doesn't make biological
sense, even though it may in fact be innocuous;


\subsection{Connection to Model  $M_h$}  \label{scr0.sec.scrmh}

SCR models are closely related to heterogeneity models. In SCR models,
heterogeneity in encounter probability is induced by both the effect
of distance in the model for detection probability and also from
specification of the state-space. Hence, the state-space  is an
explicit element of the model. 
To understand this, suppose we have a random
effect with some prior distribution:
\[
{\bf s} \sim \mbox{Unif}({\cal S})
\]
And $p({\bf s}) = p(y=1|{\bf s})$ is some function of ${\bf
  s}$. Therefore, for any specific $g(p)$ and ${\cal S}$ we can work
out what the implied heterogeneity model is for example, the mean,
variance or other moments of the population distribution of $p$ can be
evaluated by integrating $p({\bf s})$ over the state-space of ${\bf
  s}$.  We
show an illustration in Fig. \ref{scr0.fig.buffereffect} which
shows a histogram of $p$ for a hypothetical population of 100000
individuals on a state-space enclosing our $5 \times 5$ trap array
above, under the logistic model for distance. {\bf R} code is
provided in the {\bf R} package \mbox{\tt scrbook} to produce this analysis for the
logistic and half-normal models. The histogram shows the encounter
probability under buffers of 0.2, 0.5 and 1.0. We see the mass shifts
to the left as the buffer increases, implying more individuals
 with lower encounter probabilies, as their home range
centers increase in distance from the trap array.


\begin{figure}
\begin{center}
\includegraphics[width=5in]{Ch4/figs/buffereffect}
\end{center}
\caption{Implied population distribution of $p_{i}$ for a population
  of individuals as a function of the size of the state-space buffer
  around a trap array. The trap array is fixed and centered within a
  square state-space.}
\label{scr0.fig.buffereffect}
\end{figure}

Another way to understand this is by representing ${\cal S}$ as a set
of discrete points on a grid. In the coarsest possible case where
${\cal S}$ is a single arbitrary point, then every individual has
exactly the same $p$. As we increase the number of points in ${\cal
  S}$ then more distinct values of $p$ are possible. As such, when
${\cal S}$ is characterized by discrete points then SCR models are
precisely a type of finite-mixture model \citep{norris_pollock:1996,
  pledger:2000}, except, in the case of SCR models, we have some information about which
group an individual belong (i.e., where their activity center is), as
a result of their captures in traps.

This context suggests the problem raised by \citet{link:2003}. He
showed that in most practical situations $N$ may not be identifiable
across classes of mixture distributions which in the context of SCR
models is the pair $(g, {\cal S})$.  The difference, however, is that
we do obtain some direct information about ${\bf s}$ in SCR models and
therefore it may be reasonable to expect that
$N$ is identifiable across models characterized by $(g,{\cal
  S})$.

\subsection{Connection to Distance Sampling}

It is worth emphasizing that the basic SCR model is a binomial
encounter model in which distance is a covariate. As such, it is
striking similarity to a classical distance sampling model. Both have
distance as a covariate but in classical distance sampling problems
the focus is on the distance between the observer and the animal at an
instant in time, not the distance between a trap and an animal's home
range center. As a practical matter, in distance sampling, ``distance'' is {\it
  observed} for those individuals that appear in the
sample. Conversely, in SCR problems, it is only imperfectly observed
(we have partial information in the form of trap observations).
Clearly, it is preferable to observe distance if possible, but 
distance sampling requires field methods that
are often not practical in many situations, e.g. when surveying
tigers. Furthermore, SCR models allow us to relax many of the
assumption made in classical distance sampling, and SCR models allow
for estimates of quantities other than density, such as home range
size, and space usage (see Chapt. \ref{chapt.ecoldist}).


\section{Simulating SCR Data}

It is always useful to simulate data because it allows you to
understand the system that you're modeling and also calibrate your
understanding with the parameter values of the model. That is, you can
simulate data using different parameter values until you obtain data
that ``looks right'' based on your knowledge of the specific situation
that you're interested in. Here we provide a simple script to
illustrate how to simulate spatial encounter history data. In this
exercise we simulate data for 100 individuals and a 25 trap array laid
out in a $5 \times 5$ grid of unit spacing.  The specific encounter model is
the half-normal model given above and we used this code to simulate
data used in subsequent analyses.  The 100 activity centers were
simulated on a state-space defined by a $8 \times 8$ square within which the
trap array was centered (thus the trap array is buffered by 2
units). Therefore, the density of individuals in this system is fixed
at $100/64$.

{\small
\begin{verbatim}
	set.seed(2013)
# create 5 x 5 grid of trap locations with unit spacing
traplocs<- cbind(sort(rep(1:5,5)),rep(1:5,5))
Dmat<-e2dist(traplocs,traplocs) # in cases where speed doesn't matter, it might be
                                # clearer to just show the slow for-loop.
                                # Plus, people will want to copy/paste this stuff
ntraps<-nrow(traplocs)

# define state-space of point process. (i.e., where animals live).
# "delta" just adds a fixed buffer to the outer extent of the traps.
delta<-2
Xl<-min(traplocs[,1] - delta)
Xu<-max(traplocs[,1] + delta)
Yl<-min(traplocs[,2] - delta)
Yu<-max(traplocs[,2] + delta)

N<-100   # population size
K<- 20    # number nights of effort

sx<-runif(N,Xl,Xu)    # simulate activity centers
sy<-runif(N,Yl,Yu)
S<-cbind(sx,sy)
D<- e2dist(S,traplocs)  # distance of each individual from each trap

alpha0<- -2.5      # define parameters of encounter probability
sigma<- 0.5        #
alpha1<- 1/(2*sigma*sigma)
probcap<- expit(-2.5)*exp( - alpha1*D*D)    # probability of encounter
# now generate the encounters of every individual in every trap
Y<-matrix(NA,nrow=N,ncol=ntraps)
for(i in 1:nrow(Y)){
   Y[i,]<-rbinom(ntraps,K,probcap[i,])
}
\end{verbatim}
}


Subsequently we will generate data using this code packaged in an {\bf
  R}
function called \mbox{\tt simSCR0.fn} in the package \mbox{\tt
  scrbook} which takes a number of
arguments including \mbox{\tt discard0} which, if \mbox{\tt TRUE}, will return
only the encounter histories for captured individuals.  A second
argument is \mbox{\tt array3d} which, if \mbox{\tt TRUE}, returns the 3-d
encounter history array instead of the aggregated \mbox{\tt nind}
$\times \mbox{\tt ntraps}$ encounter frequencies (see below). Finally
we provide a random number seed, \mbox{\tt sd = 2013} to ensure
repeatability of the analysis here. We obtain a data set as above using the
following command:
\begin{verbatim}
data<-simSCR0.fn(discard0=TRUE,array3d=FALSE,sd=2013)
\end{verbatim}
The {\bf R} object \mbox{\tt data} is a list, so let's take a look at
what's in the list and then harvest some of its elements for further
analysis below.
\begin{verbatim}
> names(data)
[1] "Y"        "traplocs" "xlim"     "ylim"     "N"        "alpha0"   "beta"
[8] "sigma"    "K"
> Y<-data$Y
> traplocs<-data$traplocs
\end{verbatim}


\subsection{Formatting and manipulating real data sets}
\label{scr0.sec.formats}

Conventional capture-recapture data are easily stored and manipulated
as a 2-dimensional array, an $\mbox{\tt nind} \times \mbox{\tt
  nperiod}$ matrix, which is maximally informative for any
conventional capture-recapture model, but not for spatial
capture-recapture models.  For SCR models we must preserve the spatial
information in the encounter history information. We will routinely
analyze data from 3 standard formats:
\begin{itemize}
\item[(1)] The basic 2-dimensional data format, which is an \mbox{\tt
    nind} $\times$ \mbox{\tt ntraps} encounter frequency matrix such
  as that simulated previously; These are the total encounters in each
  trap, summed over replicate samples.
\item[(2)] The maximally informative 3-dimensional array which we
  establish here the convention that it has dimensions \mbox{\tt nind}
  $\times$ \mbox{\tt nperiods} $\times$ \mbox{\tt ntraps} and
\item[(3)] We use a compact format - the ``SCR flat format'' - which
  we describe below in section \ref{scr0.sec.wolverine}.
\end{itemize}
To simulate data in the most informative format - the ``3-d array'' -
we can use the {\bf R} commands given previously but replace the last
4 lines with the following:
{\small
\begin{verbatim}
Y<-array(NA,dim=c(N,K,ntraps))
for(i in 1:nrow(Y)){
for(j in 1:ntraps){
 Y[i,1:K,j]<-rbinom(K,1,probcap[i,j])
}
}
\end{verbatim}
}
We see that a collection of $K$ binary encounter events are generated
for {\it each} individual and for {\it each} trap.  The probabilities
have those Bernoulli trials are computed based on the distance from
each individuals home range center and the trap (see calculation
above), and those are housed in the matrix \mbox{\tt probcap}. Our data simulator
function \mbox{\tt simSRC0.fn} will return the full 3-d array if
\mbox{\tt array3d=TRUE} is specified in the function call.  To recover
the 2-d matrix from the 3-d array, and subset the 3-d array to
individuals that were captured, we do this:
{\small
\begin{verbatim}
Y2d<- apply(Y,c(1,3),sum) # sum over the ``replicates'' dimension (2nd margin of the array)
ncaps<-apply(Y2d,1,sum)   # compute how many times each individual was captured
Y<-Y[ncaps>0,,]           # keep those individuals that were captured
\end{verbatim}
}

\section{Fitting an SCR Model in BUGS}
\label{scr0.sec.winbugs1}

Clearly if we somehow knew the value of $N$ then we could fit this
model directly because, in that case, it is a special kind of logistic
regression model - one with a random effect, but that enters into the
model in a peculiar fashion - and also with a distribution (uniform)
which we don't usually think of as standard for random effects models.
So our aim here is to analyze the known-$N$ problem, using our
simulated data, as an incremental step in our progress toward fitting
more generally useful models.

To begin, we use our simulator to grab a data set and then harvest the
elements of the resulting object for further analysis.
\begin{verbatim}
data<-simSCR0.fn(discard0=FALSE,sd=2013)
y<-data$Y
traplocs<-data$traplocs
nind<-nrow(y)
X<-data$traplocs
J<-nrow(X)
y<-rbind(y,matrix(0,nrow=(100-nrow(y)),ncol=J ) )
Xl<-data$xlim[1]
Yl<-data$ylim[1]
Xu<-data$xlim[2]
Yu<-data$ylim[2]
\end{verbatim}

Note that we specify \mbox{\tt discard0 = FALSE} so that we have a
"complete" data set, i.e., one with the all-zero encounter histories
corresponding to uncaptured individuals. Now, within an {\bf R} session, we
can create the {\bf BUGS} model file and fit the model using the following
commands. 
{\small
\begin{verbatim}
cat("
model {
alpha0~dnorm(0,.1)
logit(p0)<- alpha0
alpha1~dnorm(0,.1)
for(i in 1:N){
 s[i,1]~dunif(Xl,Xu)
 s[i,2]~dunif(Yl,Yu)
for(j in 1:J){
d[i,j]<- pow(pow(s[i,1]-X[j,1],2) + pow(s[i,2]-X[j,2],2),0.5)
y[i,j] ~ dbin(p[i,j],K)
p[i,j]<- p0*exp(- alpha1*d[i,j]*d[i,j])
}
}

}
",file = "SCR0a.txt")
\end{verbatim}
}
This model describes the half-normal detection model but it
would be trivial to modify that to various others including the
logistic described above. One consequence of using the half-normal is
that we have to constrain the encounter probability to be in $[0,1]$
which we do here by defining \mbox{\tt alpha0} to be the logit of the
intercept parameter \mbox{\tt p0}. Note that the distance covariate is
computed within the {\bf BUGS} model specification given the matrix of trap
locations, \mbox{\tt X}, which is provided to {\bf WinBUGS} as data.

Next we do a number of organizational activities including bundling
the data for {\bf WinBUGS}, defining some initial values, the parameters to
monitor and some basic MCMC settings.  We choose initial values for
the activity centers ${\bf s}$ by generating uniform random numbers in
the state-space but, for the observed individuals, we replace those
values by each individual's mean trap coordinate for all encounters
{\small
\begin{verbatim}
sst<-cbind(runif(nind,Xl,Xu),runif(nind,Yl,Yu))  # starting values for s
for(i in 1:nind){
if(sum(y[i,])==0) next
sst[i,1]<- mean( X[y[i,]>0,1] )
sst[i,2]<- mean( X[y[i,]>0,2] )
}

data <- list (y=y,X=X,K=K,N=nind,J=J,Xl=Xl,Yl=Yl,Xu=Xu,Yu=Yu)
inits <- function(){
  list (alpha0=rnorm(1,-4,.4),alpha1=runif(1,1,2),s=sst)
}

library("R2WinBUGS")
parameters <- c("alpha0","alpha1")
nthin<-1
nc<-3
nb<-1000
ni<-2000
out <- bugs (data, inits, parameters, "SCR0a.txt", n.thin=nthin,
n.chains=nc, n.burnin=nb,n.iter=ni,debug=TRUE,working.dir=getwd())
\end{verbatim}
}
There is little to say about the preceding basic operations other than
to suggest that the interested reader explore the output and
additional analyses by running the script provided in the {\bf R}
package \mbox{\tt scrbook}.
 We ran $1000$ burn-in and $1000$ after burn-in, 3 chains,
to obtain 3000 posterior samples.  Because we know $N$ for this
particular data set we only have 2 parameters of the detection model
to summarize (\mbox{\tt alpha0} and \mbox{\tt alpha1}).  When the
object \mbox{\tt out} is produced we print a summary of the results as
follows:
{\small
\begin{verbatim}
> print(out,digits=3)
Inference for Bugs model at "SCR0a.txt", fit using WinBUGS,
 3 chains, each with 2000 iterations (first 1000 discarded)
 n.sims = 3000 iterations saved
            mean     sd    2.5%     25%    50%     75%   97.5%  Rhat n.eff
alpha0    -2.496  0.224  -2.954  -2.648  -2.48  -2.340  -2.091 1.013   190
alpha1     2.442  0.419   1.638   2.145   2.44   2.721   3.303 1.005   530
deviance 292.803 21.155 255.597 277.500 291.90 306.000 339.302 1.006   380

For each parameter, n.eff is a crude measure of effective sample size,
and Rhat is the potential scale reduction factor (at convergence, Rhat=1).

DIC info (using the rule, pD = Dbar-Dhat)
pD = -138.8 and DIC = 154.0
DIC is an estimate of expected predictive error (lower deviance is better).
\end{verbatim}
}

We know the data were generated with \mbox{\tt alpha0} $= -2.5$ and
\mbox{\tt alpha1 = -2}. The estimates look reasonably close to those
data-generating values and we probably feel pretty good about the
performance of the Bayesian analysis and MCMC algorithm that {\bf WinBUGS}
cooked-up based on our sample size of 1 data set.  It is worth noting
that the Rhat statistics indicate reasonable convergence but, as a
practical matter, we might choose to run the MCMC algorithm for
additional time to bring these closer to 1.0 and to increase the
effective posterior sample size (\mbox{\tt n.eff}). Other summary output includes
``deviance'' and related things including the deviance information
criterion (DIC). We discuss these things in Chapts. \ref{chapt.mcmc}
and \ref{chapt.gof}.



\section{Unknown N}
\label{scr0.sec.unknownN}

In all real applications $N$ is unknown and that fact is kind of an
important feature of the capture-recapture problem!  We handled this
important issue in Chapt. \ref{chapt.closed} using the method of data augmentation
which we apply here to achieve a realistic analysis of model SCR0. As
with the basic closed population models considered previously, we
formulate the problem here by augmenting our observed data set with a
number of ``all zero'' encounter histories - what we referred to in
Chapt. \ref{chapt.closed} as potential individuals. If $n$ is the number of observed
individuals, then let $M-n$ be the number of potential individuals in
the data set. For the basic $y_{ij}$ data structure (individuals x
traps encounter frequencies) we simply add additional rows of ``all
0'' observations to that data set. This is because such
``individuals'' are unobserved, and therefore necessarily have
$y_{ij}=0$ for all $j$.  A data set, say with 4 traps and 6 individuals,
augmented with 4 pseudo-individuals therefore might look like this:
{\small
\begin{verbatim}
      trap1 trap2 trap3 trap4
 [1,]     1     0     0     0
 [2,]     0     2     0     0
 [3,]     0     0     0     1
 [4,]     0     1     0     0
 [5,]     0     0     1     1
 [6,]     1     0     1     0
 [7,]     0     0     0     0
 [8,]     0     0     0     0
 [9,]     0     0     0     0
[10,]     0     0     0     0
\end{verbatim}
}
We typically have more than 4 traps and, if we're fortunate, many more
individuals in our data set.

For the augmented data, we introduce a set of binary latent variables
(the data augmentation variables), $z_{i}$, and the model is extended
to describe $\Pr(z_{i} = 1)$ which is, in the context of this problem,
the probability that an individual in the augmented data set is a
member of the population that was sampled. In other words, if $z_{i}=1$
for one of the ``all zero'' encounter histories, this is implied to be
a sampling zero whereas observations for which $z_{i}=0$ are
``structural zeros'' under the model.

How big does the augmented data set have to be? We discussed this
issue in Chapt. \ref{chapt.closed} where we noted that the size of the data set is
equivalent to the upper limit of a uniform prior distribution on $N$.
Practically speaking, it should be sufficiently large so that the
posterior distribution for $N$ is not truncated. On the other hand, if
it is too large then unnecessary calculations are being done. An
approach to choosing $M$ by trial-and-error is indicated. You can take
a ballpark estimate of the probability that an individual is captured
at all during the study, say $\tilde{p}$, which is related to the
``per sample'' encounter probability, $p$, by $\tilde{p} = 1-(1-p)^{K}$, obtain $N$ as $n/\tilde{p}$, and then set $M =
2*N$, as a first guess. Do a short MCMC run and then consider whether
you need to increase $M$. See Chapt. \ref{chapt.mcmc} for an
example of this. \citet[][ch. 6]{kery_schaub:2011}
 provide an assessment of choosing $M$ in closed population models.

Analysis by data augmentation removes $N$ as an explicit parameter of
the model. Instead, $N$ is a derived parameter, computed by $N=
\sum_{i=1}^{M} z_{i}$. Similarly, {\it density}, $D$, is also a
derived parameter computed as $D=N/area({\cal S})$. For our
simulator, we're using an $8 \times 8$ state-space and thus we will
compute $D$ as $D=N/64$.

\subsection{Analysis using data augmentation in WinBUGS}

As before we begin by obtaining a data set using our \mbox{\tt
  simSCR0.fn} routine and then harvesting the required data objects
from the resulting data list.  Note that we use the \mbox{\tt
  discard0=TRUE} option this time so that we get a ``real'' data set
with no all-zero encounter histories. After harvesting the data we
produce the {\bf WinBUGS} model specification which now includes $M$
encounter histories including the augmented potential individuals, the
data augmentation parameters $z_{i}$, and the data augmentation
parameter $\psi$.
{\small
\begin{verbatim}
data<-simSCR0.fn(discard0=TRUE,sd=2013)
y<-data$Y
traplocs<-data$traplocs
nind<-nrow(y)
X<-data$traplocs
J<-nrow(X)
Xl<-data$xlim[1]
Yl<-data$ylim[1]
Xu<-data$xlim[2]
Yu<-data$ylim[2]

cat("
model {
alpha0~dnorm(0,.1)
logit(p0)<- alpha0
alpha1~dnorm(0,.1)
psi~dunif(0,1)

for(i in 1:M){
 z[i] ~ dbern(psi)
 s[i,1]~dunif(Xl,Xu)
 s[i,2]~dunif(Yl,Yu)
for(j in 1:J){
d[i,j]<- pow(pow(s[i,1]-X[j,1],2) + pow(s[i,2]-X[j,2],2),0.5)
y[i,j] ~ dbin(p[i,j],K)
p[i,j]<- z[i]*p0*exp(- alpha1*d[i,j]*d[i,j])
}
}
N<-sum(z[])
D<-N/64
}
",file = "SCR0a.txt")
\end{verbatim}
}

To prepare our data we have to augment the data matrix \mbox{\tt y}
with $M-n$ all-zero encounter histories, we have to create starting
values for the variables $z_{i}$ and also the activity centers ${\bf
  s}_{i}$ of which, for each, we require $M$ values. Otherwise the
remainder of the code for bundling the data, creating initial values
and executing {\bf WinBUGS} looks much the same as before except with more
or differently named arguments.
{\small
\begin{verbatim}
## Data augmentation stuff
M<-200
y<-rbind(y,matrix(0,nrow=M-nind,ncol=ncol(y)))
z<-c(rep(1,nind),rep(0,M-nind))

sst<-cbind(runif(M,Xl,Xu),runif(M,Yl,Yu))  # starting values for s
for(i in 1:nind){
if(sum(y[i,])==0) next
sst[i,1]<- mean( X[y[i,]>0,1] )
sst[i,2]<- mean( X[y[i,]>0,2] )
}
data <- list (y=y,X=X,K=K,M=M,J=J,Xl=Xl,Yl=Yl,Xu=Xu,Yu=Yu)
inits <- function(){
  list (alpha0=rnorm(1,-4,.4),alpha1=runif(1,1,2),s=sst,z=z)
}

library("R2WinBUGS")
parameters <- c("alpha0","alpha1","N")
nthin<-1
nc<-3
nb<-1000
ni<-2000
out <- bugs (data, inits, parameters, "SCR0a.txt", n.thin=nthin,n.chains=nc,
 n.burnin=nb,n.iter=ni,debug=TRUE,working.dir=getwd())
\end{verbatim}
}

{\bf Remarks}:  (1) Note the differences in this new {\bf WinBUGS} model
with that appearing in the known-$N$ version.  (2) Also the input data
has changed - the augmented data set has more rows of
all-zeros. Previously we knew that $N=100$ but in this analysis we
pretend not to know $N$, but think that $N=200$ is a good upper-bound;
(3) Population size $N({\cal S})$ is a derived parameter, being computed by
summing up all of the data augmentation variables $z_{i}$ (as we've
done previously in Chapt. \ref{chapt.closed}); (4) Density, $D\equiv D({\cal S})$, is also a derived
parameter. Summarizing the output from {\bf WinBUGS} produces:
{\small
\begin{verbatim}
> print(out1,digits=2)
Inference for Bugs model at "SCR0a.txt", fit using WinBUGS,
 3 chains, each with 2000 iterations (first 1000 discarded)
 n.sims = 3000 iterations saved
           mean    sd   2.5%    25%    50%    75%  97.5% Rhat n.eff
alpha0    -2.57  0.23  -3.04  -2.72  -2.56  -2.41  -2.15 1.01   320
alpha1     2.46  0.42   1.63   2.16   2.46   2.73   3.33 1.02   120
N        113.62 15.73  86.00 102.00 113.00 124.00 147.00 1.01   260
D          1.78  0.25   1.34   1.59   1.77   1.94   2.30 1.01   260
deviance 302.60 23.67 261.19 285.47 301.50 317.90 354.91 1.00  1400

For each parameter, n.eff is a crude measure of effective sample size,
and Rhat is the potential scale reduction factor (at convergence, Rhat=1).

DIC info (using the rule, pD = var(deviance)/2)
pD = 279.9 and DIC = 582.5
DIC is an estimate of expected predictive error (lower deviance is better).
\end{verbatim}
}

The column labeled ``MC error'' is the Monte Carlo error - the error
inherent in the attempt to compute these posterior summaries by
MCMC
(see secs.  for discussion of this quantity
\ref{glms.sec.convergence} \ref{mcmc.sec.mcmcsummary}).
It is desirable to run the Markov chain algorithm long enough so
as to reduce the MC error to a tolerable level. What constitutes
tolerable is up to the investigator. Certainly less than 1\% is called
for. As a general rule, Rhat gets closer to 1 and MC error decreases
toward 0 as the number of iterations increases.  We see that the
estimated parameters ($\alpha_0$ and $\alpha1$) are comparable to the
previous results obtained for the known-$N$ case, and also not too
different from the data-generating values. The posterior of $N$
overlaps the data-generating value substantially with a mean of
$113.62$.  To obtain these results we fitted the true data-generating
model, that based on the half-normal detection model, to a single
simulated data set. For fun and excitement we fit the {\it wrong}
model, one with a logistic-linear detection model
(Eq. \ref{scr0.eq.logit}),
to the same  
data set. This is easily achieved by modifying the {\bf WinBUGS} model
specification above, although we provide the {\bf R} script in the
{\bf R} package \mbox{\tt scrbook}.
Those results are given below. We see that the estimate of
$N$, the main parameter of interest, is very similar to that obtained
under the correct model, convergence is worse (as measured by Rhat)
which may not have anything to do with the model being wrong,
and the posterior deviance favors the correct model (it is smaller) while the DIC does not.
We consider 
 the effectiveness of DIC for carrying-out model selection in chapter
\ref{chapt.gof}.
{\small
\begin{verbatim}
> print(out2,digits=2)
Inference for Bugs model at "SCR0a.txt", fit using WinBUGS,
 3 chains, each with 2000 iterations (first 1000 discarded)
 n.sims = 3000 iterations saved
           mean    sd   2.5%    25%    50%    75%  97.5% Rhat n.eff
alpha0    -1.59  0.27  -2.16  -1.77  -1.58  -1.42  -1.07 1.05    60
beta       3.77  0.43   2.92   3.48   3.79   4.05   4.66 1.04    70
N        122.57 18.67  90.00 109.00 122.00 135.00 163.00 1.00  3000
D          1.92  0.29   1.41   1.70   1.91   2.11   2.55 1.00  3000
deviance 312.67 22.43 271.00 297.20 311.50 327.00 359.60 1.02   130

For each parameter, n.eff is a crude measure of effective sample size,
and Rhat is the potential scale reduction factor (at convergence, Rhat=1).

DIC info (using the rule, pD = var(deviance)/2)
pD = 247.5 and DIC = 560.1
DIC is an estimate of expected predictive error (lower deviance is better).
\end{verbatim}
}

\subsection{Use of other BUGS engines: JAGS}

There are two other popular {\bf BUGS} engines in widespread use: {\bf
  OpenBUGS} \citep{thomas_etal:2006} and {\bf JAGS}
\citep{plummer:2003}. Both of these are easily called from {\bf
  R}. {\bf OpenBUGS} can be used instead of {\bf WinBUGS} by changing
the package option in the \mbox{\tt bugs} call to \mbox{\tt
  package=OpenBUGS}.  {\bf JAGS} can be called using the function
\mbox{\tt jags()} in package \mbox{\tt R2JAGS} which has nearly the
same arguments as \mbox{\tt bugs()}.  We prefer to use the {\bf R}
library \mbox{\tt rjags} \citep{plummer:2009} which has a slightly
different implementation that we demonstrate here as we reanalyze the
simulated data set in the previous section (note: the same {\bf R}
commands are used to generate the data and package the data, inits and
parameters to monitor). The function \mbox{\tt jags.model} is used to
initialize the model and run the MCMC algorithm for an adaptive
burn-in period.  Then the Markov chains are updated using \mbox{\tt
  coda.samples()} to obtain posterior samples for analysis, as
follows:
\begin{verbatim}
jm<- jags.model("SCR0a.txt", data=data, inits=inits, n.chains=nc,
                 n.adapt=nb))
jm<- coda.samples(jm, parameters, n.iter=ni-nb, thin=nthin)
\end{verbatim}
We find that {\bf JAGS} seems to be 20-30\% faster for the basic SCR
model which the reader can evaluate using the function \mbox{\tt
  SCR0bayes} in the {\bf R} package \mbox{\tt scrbook}.



\section{Wolverine Camera Trapping Study}
\label{scr0.sec.wolverine}

We provide an analysis here of A. Magoun's wolverine data
\citep{magoun_etal:2011, royle_etal:2011jwm}. The study took place in SE
Alaska (Fig. \ref{scr0.fig.wolverinelocs}) where 37 cameras were
operational for variable periods of time (min = 5 days, max = 108
days, median = 45 days).  A consequence of this is that the binomial
sample size $K$ (see Eq. \ref{scr0.eq.bin})
 is variable for each camera. Thus, we
must provide a matrix of sample sizes as data to {\bf BUGS} and modify the
model specification in sec. \ref{scr0.sec.unknownN}
accordingly. Our treatment of the
data here is based on the analysis of  \citet{royle_etal:2011jwm}.

\begin{figure}
\begin{center}
\includegraphics[height=3in]{Ch4/figs/wolverinelocs}
\end{center}
\caption{Wolverine camera trap locations from \citet{magoun_etal:2011}.}
\label{scr0.fig.wolverinelocs}
\end{figure}

To carry-out an analysis of these data, we require the matrix of trap
coordinates and the encounter history data.  We store data in an the
``scr flat format'' (see sec.  \ref{scr0.sec.formats} above), an
efficient file format which is easily manipulated and also used as the
input file format in {\bf SPACECAP} \citep{gopalaswamy_etal:2012} and
in the {\bf R} package \mbox{\tt SCRbayes} \citep{russell_etal:2012}.
To illustrate this format, the wolverine data are available in the
package \mbox{\tt scrbook} by typing:
\begin{verbatim}
data(wolverine)
\end{verbatim}
which contains a list having elements \mbox{\tt wcaps} and
\mbox{\tt wtraps}.
The ``encounter data file''
\mbox{\tt wcaps}  has 3 columns and 115 rows, each representing a
unique encounter event including the trap identity, the individual
identity and the sample occasion index (\mbox{\tt sample}).
The first 10 rows of this matrix are as
follows:
{\small
\begin{verbatim}
> wolverine$wcaps[1:10,]
       trapid individual sample
  [1,]      1          2    127
  [2,]      1          2    128
  [3,]      1          2    129
  [4,]      1         18    130
  [5,]      2          3    106
  [6,]      2         18    104
  [7,]      5          5     73
  [8,]      5          5     89
  [9,]      6         18    117
 [10,]      6         18    118
\end{verbatim}
}
Each row is a unique 
individual/trap encounter, and the 3 variables (columns) are: 
\mbox{\tt trapid} -- an
integer that runs from \mbox{\tt 1:ntraps}, \mbox{\tt individual} runs from
\mbox{\tt 1:nind} and
\mbox{\tt sample} 
runs from \mbox{\tt 1:nperiods}. Often (as the case here) \mbox{\tt
  sample} 
will
correspond to daily sample intervals. The variable \mbox{\tt trapid} will have to
correspond to the row of a matrix containing the trap coordinates - in
this case the file \mbox{\tt wtraps} which we describe further below.

Note that the information provided in this encounter data file
\mbox{\tt wcaps}
does not represent a completely informative summary
of the data. For example, if no individuals were captured in a certain
trap or during a certain period, then this compact data format will
have no record. Thus we will need to know ntraps and nperiods when
reformatting this SCR data format into a 2-d encounter frequency
matrix or 3-d array. In addition, the encounter data file does not
provide information about which periods each trap was operated. This
additional information is also necessary as the trap-specific sample
sizes must be passed to {\bf BUGS} as data. We provide this information in a
2nd data file, along with the trap coordinates, in the 
 ``trap deployment'' file which is described
below.

For our purposes we
need to convert the \mbox{\tt wcaps} file
into the $n \times J$ array of
binomial encounter frequencies, although more general models might
require an encounter-history formulation of the model which requires a
full 3-d array.  To obtain our $n \times J$ encounter frequency
matrix, we do this the hard way by first converting the encounter data
file into a 3-d array and then summarize to trap totals. We have a
handy function \mbox{\tt SCR23darray.fn} which takes the compact
encounter data file with optional arguments ntraps and nperiods, and
converts it to a 3-d array, and then we use the {\bf R} function
\mbox{\tt apply} to summarize over the ``sample'' period dimension (by
convention here, this is the 2nd dimension). To apply this to the
wolverine
data in order to compuate the 3-d array we do this:
{\small
\begin{verbatim}
y3d <-SCR23darray.fn(wolverine$wcaps,wolverine$wtraps)
y <- apply(y3d,c(1,3),sum)
\end{verbatim}
}
See the help file for more information on \mbox{\tt SCR23darray.fn}.
The 3-d array is necessary to fit certain types
of models (e.g., behavioral response) and this is why we sometimes
will require this maximally informative 3-d data format but, here, we
analyze the summarized data.



The other important information needed to fit SCR models is the
``trap deployment'' file
which provides the additional information
not contained in the encounter data file. The traps file has \mbox{\tt
  nperiods} $+ 3$ columns. The first column is assumed to be a trap identifier,
columns 2 and 3 are the easting and northing coordinates (assumed to
be in a Euclidean coordinate system), and columns 4 to (\mbox{\tt nperiods} + 3)
are binary indicators of whether each trap was operational in each
time period. The first 10 rows (out of 37) and 10 columns (out of 167)
of the trap deployment file for the wolverine data are:
{\small
\begin{verbatim}
> wolverine$wtraps[1:10,1:10]

   Easting Northing 1 2 3 4 5 6 7 8 
1   632538  6316012 0 0 0 0 0 0 0 0
2   634822  6316568 1 1 1 1 1 1 1 1
3   638455  6309781 0 0 0 0 0 0 0 0
4   634649  6320016 0 0 0 0 0 0 0 0
5   637738  6313994 0 0 0 0 0 0 0 0
6   625278  6318386 0 0 0 0 0 0 0 0
7   631690  6325157 0 0 0 0 0 0 0 0
8   632631  6316609 0 0 0 0 0 0 0 0
9   631374  6331273 0 0 0 0 0 0 0 0
10  634068  6328575 0 0 0 0 0 0 0 0
\end{verbatim}
}
This tells us that trap 2 was operated in periods (days) 1-7 but the other
traps were not operational during those periods. It is extremely
important to recognize that each trap was operated for a variable
period of time and thus the binomial "sample size" is different for
each, and this needs to be accounted for in the {\bf BUGS} model specification.
To compute the vector of sample sizes $K$, and extract the trap
locations,  we do this:
\begin{verbatim}
traps<- wolverine$wtraps
traplocs<- traps[,1:2]
K<- apply(traps[,3:ncol(traps)],1,sum)
\end{verbatim}
This results in a matrix traplocs which contains the coordinates of
each trap and a vector $K$ containing the number of days that each trap
was operational. We now have all the information required to fit a
basic SCR model in {\bf BUGS}.

Summarizing these data files for the wolverine study, we see that 21
unique individuals were captured a total of 115 times. Most
individuals were captured 1-6 times, with 4, 1, 4, 3, 1, and 2
individuals captured 1-6 times, respectively.  In addition, 1
individual was captured each 8 and 14 times and 2 individuals each
were captured 10 and 13 times.  The number of unique traps that
captured a particular individual ranged from 1-6, with 5, 10, 3, 1, 1,
and 1 individual captured in each of 1-6 traps, respectively, for a
total of 50 unique wolverine-trap encounters.  These numbers might be
hard to get your mind around whereas some tabular summary is often
more convenient. For that it seems natural to tabulate individuals by
trap and total encounter frequencies. The spatial information in SCR
data is based on multi-trap captures\footnote{I will add more 
context here on revision about spatial recaptures, lost recaptures,
ordinary recaptures. Function \mbox{\tt SCRsmy} in \mbox{\tt
  scrbook}}, 
and so, it is informative to
understand how many unique traps each individual is captured in. At
the same, it is useful to understand how many total captures we have
of each individual because this is, in an intuitive sense, the
effective sample size.  So, we reproduce Table 1 from
\citet{royle_etal:2011jwm} which shows the trap and total encounter
frequencies:

\begin{table} [htp]
  \caption{Individual frequencies of capture for wolverines captured
    in camera traps in Southeast Alaska in 2008. Rows index unique
    trap frequencies and columns represent total number of captures
    (e.g., we captured 4 individuals 1 time, necessarily in only 1
    trap; we captured 3 individuals 3 times but in 2 different traps)}
\centering
\begin{tabular}{c c c c c c c c c c c}
\hline
 & & & & & & & &  No.&of&captures \\
\hline
No. of traps & 1 & 2 & 3 & 4 & 5 & 6 & 8 & 10 &13 &14 \\
\hline
1 & 4 & 1 & 0 & 0 & 0 & 0 & 0 & 0 & 0 & 0 \\
2 & 0 & 0 & 3 & 3 & 0 & 2 & 1 & 2 & 0 & 0 \\
3 & 0 & 0 & 1 & 1 & 0 & 0 & 0 & 0 & 0 & 1 \\
4 & 0 & 0 & 0 & 0 & 0 & 0 & 0 & 0 & 1 & 0 \\
5 & 0 & 0 & 0 & 0 & 1 & 0 & 0 & 0 & 0 & 0 \\
6 & 0 & 0 & 0 & 0 & 0 & 0 & 0 & 0 & 1 & 0 \\
\hline
\end{tabular}
\end{table}

\subsection{Fitting the model in WinBUGS}

For illustrative purposes here we fit the simplest SCR model with the
half-normal distance function although we revisit these data with more
complex models in later chapters. The model is summarized by the
following 3 components:
\begin{itemize}
\item[(1)] $y_{ij}|{\bf s}_{i} \sim \mbox{Bin}(K, z_{i}\; p_{ij})$
\item[(2)] $p_{ij} = p_{0} \exp(-\alpha1 \; ||{\bf s}_{i}-x_{j}||^2)$
\item[(3)] $ {\bf s}_{i} \sim \mbox{Unif}({\cal S})$
\item[(4)] $ z_{i} \sim \mbox{Bern}(\psi)$
\end{itemize}
We assume customary flat priors on the structural (hyper-) parameters
of the model, $\alpha_{0} = \mbox{logit}(p_{0})$, $\alpha1$ and $\psi$.  It remains to define the
state-space ${\cal S}$. For this, we nested the trap array (Fig.
\ref{scr0.fig.wolverinelocs}) in a
a rectangular state-space extending $20$ km beyond the traps in each cardinal
direction.  We also considered larger state-spaces up to 50 km to
evaluate that choice.  The buffer of the state space should be larger
enough so that individuals beyond the state-space boundary are not
likely to be encountered. Thus some knowledge of typical space usage
patterns of the species is useful.  For the analysis, 
we scaled the coordinate system 
so that a unit distance was equal to $10$ km, producing a rectangular
state-space of dimension $9.88 \times 10.5$ units ($area = 10374$ km$^2$)
within which the trap array was nested. As a general rule, we
recommend scaling the state-space so that it is defined near the
origin $(x,y)=(0,0)$. While the scaling of the coordinate system is
theoretically irrelevant, a poorly scaled coordinate system can
produce Markov chains that mix poorly.  For the scaled coordinate
system we fit models for various choices of a rectangular state-space
based on 
buffers from 1.0 to 5.0 units on the scaled coordinate system (10 km to
50 km). In the {\bf R} package \mbox{\tt scrbook} we provide a
function
\mbox{\tt wolvSCR0.fn} which will fit the basic SCR model. For
example, to fit the model in 
{\bf WinBUGS} using data augmentation with $M=300$ potential individuals,
using 3 Markov chains each of 12000 total iterations, discarding the
first 2000 as burn-in, we execute the following {\bf R} commands:
{\small
\begin{verbatim}
library("scrbook")
data(wolverine)
traps<-wolverine$wtraps
y3d <-SCR23darray.fn(wolverine$wcaps,wolverine$wtraps)
toad<-wolvSCR0.fn(y3d,traps,nb=12000,ni=2000,delta=1,M=300)
\end{verbatim}
}
The argument $\delta$ determines the buffer size of the state-space.
Note that this analysis takes 
between 1-2 hours on many machines so we recommend trying it out with
lower values of $M$ and fewer iterations.
The output
follows (note, we have a parameter ``sigma'' which we discuss
shortly)\footnote{Final as of 1/11/2012. 
output saved in \mbox{\tt wolv-buffer-study.txt}}:

{\small
\begin{verbatim}
All based on 3 chains, 12k iters, 2k burn, 30k total
Buffer = 10 km
           mean    sd   2.5%    25%    50%    75%  97.5% Rhat n.eff
psi        0.13  0.03   0.08   0.11   0.13   0.15   0.20    1 10000
sigma      0.65  0.06   0.55   0.61   0.64   0.68   0.76    1  1800
p0         0.06  0.01   0.04   0.05   0.06   0.06   0.08    1 20000
N         39.63  6.70  29.00  35.00  39.00  44.00  54.00    1  7100
D          5.92  1.00   4.33   5.22   5.82   6.57   8.06    1  7100
beta       1.23  0.21   0.85   1.08   1.22   1.36   1.66    1  1800
deviance 410.05 12.06 388.70 401.50 409.20 417.80 435.60    1 22000

Buffer = 15 km
 n.sims = 30000 iterations saved
           mean    sd   2.5%    25%    50%    75%  97.5% Rhat n.eff
psi        0.16  0.04   0.10   0.14   0.16   0.19   0.25    1  3800
sigma      0.64  0.06   0.54   0.60   0.64   0.67   0.76    1   510
p0         0.06  0.01   0.04   0.05   0.06   0.06   0.08    1 17000
N         48.77  9.19  34.00  42.00  48.00  54.00  69.00    1  3300
D          5.78  1.09   4.03   4.98   5.69   6.40   8.18    1  3300
beta       1.25  0.21   0.86   1.10   1.24   1.39   1.70    1   510
deviance 411.00 12.16 389.50 402.40 410.30 418.70 437.00    1  5400

Buffer = 20 km
           mean    sd   2.5%    25%    50%    75%  97.5% Rhat n.eff
psi        0.20  0.05   0.12   0.17   0.20   0.23   0.30    1 16000
sigma      0.64  0.06   0.54   0.60   0.63   0.67   0.76    1  1200
p0         0.06  0.01   0.04   0.05   0.06   0.06   0.08    1  1900
N         59.84 11.89  40.00  51.00  59.00  67.00  86.00    1 20000
D          5.77  1.15   3.86   4.92   5.69   6.46   8.29    1 20000
beta       1.26  0.21   0.87   1.11   1.25   1.40   1.71    1  1200
deviance 411.01 12.36 389.10 402.30 410.20 418.80 437.50    1  1500

Buffer = 25 km
           mean    sd   2.5%    25%    50%    75%  97.5% Rhat n.eff
psi        0.24  0.05   0.15   0.20   0.24   0.28   0.36    1  3400
sigma      0.64  0.05   0.54   0.60   0.63   0.67   0.75    1  3600
p0         0.06  0.01   0.04   0.05   0.06   0.06   0.08    1  5000
N         72.40 14.72  47.00  62.00  71.00  81.00 105.00    1  2700
D          5.79  1.18   3.76   4.96   5.67   6.47   8.39    1  2700
beta       1.26  0.21   0.88   1.12   1.25   1.40   1.71    1  3600
deviance 411.35 12.23 389.70 402.70 410.55 419.20 437.20    1 30000

Buffer = 30 km
           mean    sd   2.5%    25%    50%    75%  97.5% Rhat n.eff
psi        0.29  0.06   0.18   0.24   0.28   0.33   0.43    1  3100
sigma      0.63  0.05   0.54   0.60   0.63   0.67   0.75    1  5600
p0         0.06  0.01   0.04   0.05   0.06   0.06   0.08    1 11000
N         86.42 17.98  56.00  74.00  85.00  97.00 126.02    1  3900
D          5.82  1.21   3.77   4.98   5.72   6.53   8.49    1  3900
beta       1.27  0.21   0.88   1.12   1.26   1.41   1.71    1  5600
deviance 411.06 12.37 389.20 402.50 410.20 418.90 437.60    1 10000

Buffer = 35 km
           mean    sd   2.5%    25%    50%    75%  97.5% Rhat n.eff
psi        0.34  0.08   0.21   0.29   0.34   0.39   0.50    1 30000
sigma      0.63  0.05   0.54   0.60   0.63   0.67   0.75    1  4500
p0         0.06  0.01   0.04   0.05   0.06   0.06   0.08    1 24000
N        101.79 21.54  65.00  87.00 100.00 115.00 148.00    1 30000
D          5.85  1.24   3.74   5.00   5.75   6.61   8.51    1 30000
beta       1.27  0.21   0.89   1.12   1.25   1.40   1.70    1  4500
deviance 411.10 12.20 389.50 402.40 410.30 418.90 437.20    1 22000

Buffer = 40 km
           mean    sd   2.5%    25%    50%    75%  97.5% Rhat n.eff
psi        0.39  0.09   0.24   0.33   0.39   0.45   0.60 1.01   480
sigma      0.64  0.05   0.54   0.60   0.63   0.67   0.75 1.01   410
p0         0.06  0.01   0.04   0.05   0.06   0.06   0.08 1.00 21000
N        118.05 26.14  75.00 100.00 116.00 133.00 178.00 1.01   450
D          5.87  1.30   3.73   4.97   5.76   6.61   8.84 1.01   450
beta       1.27  0.21   0.89   1.12   1.25   1.40   1.72 1.01   410
deviance 411.37 12.35 389.30 402.60 410.60 419.30 437.50 1.00  9700

Buffer = 45 km
           mean    sd   2.5%    25%    50%    75%  97.5% Rhat n.eff
psi        0.45  0.10   0.28   0.38   0.44   0.51   0.66    1  3600
sigma      0.64  0.05   0.54   0.60   0.63   0.67   0.75    1 10000
p0         0.06  0.01   0.04   0.05   0.06   0.06   0.08    1  8100
N        134.43 28.68  85.00 114.00 132.00 153.00 196.00    1  3300
D          5.83  1.24   3.68   4.94   5.72   6.63   8.50    1  3300
beta       1.26  0.21   0.88   1.11   1.24   1.39   1.69    1 10000
deviance 411.36 12.19 389.60 402.70 410.60 419.10 437.30    1  9400

Buffer = 50 km
           mean    sd   2.5%    25%    50%    75%  97.5% Rhat n.eff
psi        0.51  0.11   0.31   0.43   0.50   0.57   0.74    1  3200
sigma      0.63  0.05   0.54   0.60   0.63   0.67   0.75    1  4700
p0         0.06  0.01   0.04   0.05   0.06   0.06   0.08    1  3300
N        151.61 31.65  96.00 129.00 149.00 172.00 221.00    1  3400
D          5.79  1.21   3.66   4.92   5.69   6.56   8.43    1  3400
beta       1.27  0.21   0.89   1.12   1.25   1.40   1.70    1  4700
deviance 410.81 12.18 389.20 402.30 410.10 418.50 436.70    1 30000

Buffer = 55 km 
           mean    sd   2.5%    25%    50%    75%  97.5% Rhat n.eff
psi        0.56  0.12   0.35   0.48   0.55   0.64   0.82 1.01   260
sigma      0.64  0.05   0.54   0.60   0.63   0.67   0.76 1.00  1600
p0         0.06  0.01   0.04   0.05   0.06   0.06   0.08 1.00 30000
N        169.28 35.81 108.00 143.00 166.00 192.00 247.00 1.01   260
D          5.73  1.21   3.66   4.84   5.62   6.50   8.36 1.01   260
beta       1.25  0.21   0.88   1.11   1.24   1.39   1.69 1.00  1600
deviance 411.28 12.38 389.40 402.60 410.50 419.10 437.50 1.00 26000
\end{verbatim}
}

We see that the estimated density is roughly consistent as we increase
the state-space buffer from $15$ to $50$ $km$. We do note that the data
augmentation parameter $\psi$ (and, correspondingly, $N$) increase with
the size of the state space in accordance with the deterministic
relationship $N= D*A$. However, density is constant more or less as we
increase the size of the state-space beyond a certain point.  For the
10 $km$ state-space buffer, we see a slight effect on the posterior
distribution of $D$. This is not a bug but rather a feature. As we noted
above, the state-space is part of the model.


\subsection{Thoughts on the Wolverine Analysis}

Our point estimate of wolverine density from this study, using the
posterior mean from the state-space based on the 20
$km$ buffer, is 
approximately $5.77$ individuals/1000 $km^2$ with  a 95\% posterior
interval of $[3.86, 8.29]$. Density is estimated imprecisely
which might not be surprising given the low sample size ($n=21$
individuals!). This seems to be a basic feature of carnivore studies
although it should not (in our view) preclude the study of their
populations nor attempts to estimate density or vital rates.

One thing we haven't talked about yet is that we can calibrate the
desired size of the state-space by looking at the estimated home range
radius of the species. For some models it is possible to convert the
parameter $\alpha1$ directly into the home range radius (sec. 
XXX MISSING XYZ). For the half-normal model we interpret the half-normal scale
parameter $\sigma$ which is related to $\alpha1$ by $\alpha1 =
1/(2\sigma^2)$ as the radius of a bivariate normal movement model. 
In this case $\sigma = 1.82$ standardized units = 18.2 $km$ which 
translates into a home range area of XXXX MISSING XXXXX. 

It is worth thinking about this model, and these estimates, computed
under a rectangular state space roughly centered over the trapping
array (Fig. \ref{scr0.fig.wolverinelocs}).
Does it make sense to define the state-space to
include, for example, ocean? What are the possible consequences of
this? What can we do about it?  There's no reason at all that the
state space has to be a regular polygon -- we defined it as such here
strictly for convenience and for ease of implementation in {\bf WinBUGS}
where it enables us to specify the prior for the activity centers as
uniform priors for each coordinate.  While it would be possible to
define a more realistic state-space using some general polygon GIS coverage, it
might take some effort to implement that in the {\bf BUGS} language
but it is not difficult to devise custom MCMC algorithms to do that
(see Chapt. \ref{chapt.mcmc}).
Alternatively, we recommend
using a discrete representation of the state-space -- i.e., approximate
${\cal S}$ by a grid of $G$ points. We discuss this in sec. 
\ref{scr0.sec.discrete}.


\section{Constructing Density Maps}
\label{scr0.sec.mapping}

One of the most useful aspects of SCR models is that they are
parameterized in terms of individual locations - i.e., {\it where}
each individual lives -- and, thus, we can compute many useful or
interesting summaries of the activity centers.  For example, we can
make a spatial density plot by tallying up the number of activity
centers ${\bf s}_{i}$ in boxes of arbitrary size and then producing a
nice multi-color spatial plot of those which, we find, increases the
acceptance probability of your manuscripts by a substantial amount.
We discussed in Chapt. \ref{chapt.glms} the idea of estimating derived
parameters from MCMC output. In SCR models, there are many derived
parameters that are functions of the latent point locations $({\bf
  s}_{1},\ldots, {\bf s}_{N})$. In the present context, the number of
individuals living in any well-defined polygon is a derived
parameter. Specifically, let $B({\bf x})$ indicate a box centered at
${\bf x}$ then
\[
N({\bf x})=\sum_{i} I({\bf s}_{i} \in B({\bf x}))
\]
is the population size of box $B({\bf x})$, and $D({\bf x}) = N({\bf
  x})/||B({\bf x})||$ is the local density. These are just ``derived
parameters'' (see Chapt.  \ref{chapt.glms}) which are estimated from
MCMC output using the appropriate Monte Carlo average. One thing to be
careful about, in the context of models in which $N$ is unknown, is
that, for each MCMC iteration $m$, we only tabulate those activity
centers which correspond to individuals in the sampled
population. i.e., for which the data augmentation variable $z_{i} =
1$.  In this case, we take all of the output for MCMC iterations
$m=1,2,\ldots,\mbox{\tt niter}$ and compute this summary:
\[
   N({\bf x},m) = \sum_{z_{i,m}=1} I(s_{i,m} \in B({\bf x}))
\]
Thus, $N({\bf x},1),N({\bf x},2),\dots,$ is the Markov chain for
parameter $N({\bf x})$.  In what follows we will provide a set of {\bf
  R} commands for doing this calculations and making a basic image
plot from the MCMC output.

{\flushleft \bf Step 1:} Define the center points of each box, $B({\bf
  x})$, or point at which local density will be estimated:
\begin{verbatim}
xg<-seq(Xl,Xu,,50)
yg<-seq(Yl,Yu,,50)
\end{verbatim}

{\flushleft \bf Step 2:} Extract the MCMC histories for the activity
centers and the data augmentation variables.  Note that these are each
$N \times \mbox{\tt niter}$ matrices:
\begin{verbatim}
Sxout<-out$sims.list$s[,,1]
Syout<-out$sims.list$s[,,2]
z<-out$sims.list$z
\end{verbatim}

{\flushleft \bf Step 3:} We associate each coordinate with the proper
box using the {\bf R} command \mbox{\tt cut()}. Note that we keep only
the activity centers for which $z=1$ (i.e., individuals that belong to
the population of size $N$):
\begin{verbatim}
Sxout<-cut(Sxout[z==1],breaks=xg,include.lowest=TRUE)
Syout<-cut(Syout[z==1],breaks=yg,include.lowest=TRUE)
\end{verbatim}

{\flushleft \bf Step 4:} Use the \mbox{\tt table()} command to tally
up how many activity centers are in each $B(x)$:
\begin{verbatim}
Dn<-table(Sxout,Syout)
\end{verbatim}

{\flushleft \bf Step 5:} Use the \mbox{\tt image()} command to display
the resulting matrix.
\begin{verbatim}
image(xg,yg,Dn/nrow(z),col=terrain.colors(10))
\end{verbatim}
Praise the Lord! This map is somewhat useful or at least it looks
pretty and will facilitate the publication of your papers.

It is worth emphasizing here that density maps will not usually appear
uniform despite that we have assumed that activity centers are
uniformly distributed. This is because the observed encounters of
individuals provide direct information about the location of the
$i=1,2,\ldots,n$ activity centers and thus their ``estimated''
locations will be affected by the observations. In a limiting sense,
were we to sample space intensely enough, every individual would be
captured a number of times and we would have considerable information
about all $N$ point locations. Consequently, the uniform prior would
have almost no influence at all on the estimated density surface in
this limiting situation. Thus, in practice, the influence of the
uniformity assumption increases as the fraction of the population
encountered decreases.

{\bf On the non-intuitiveness of \mbox{\tt image()} } -- the {\bf R}
function \mbox{\tt image()} might
not be very intuitive to some -- it plots $M[1,1]$ in the lower left
corner. If you want $M[]$ to be plotted ``as
you look at it'' then $M[1,1]$ should be in the upper left corner.  We
have a function \mbox{\tt rot()} which does that. If you do \mbox{\tt image(rot(M))} then it
puts it on the monitor as if it was a map you were looking at.  You
can always specify the $x$ and $y-$ labels explicitly as we did above.

{\bf Spatial dot plots } -- Now here is a cruder version based on the
``spatial dot map'' function \mbox{\tt spatial.plot}, which uses
the function \mbox{\tt image.scale()}.
The \mbox{\tt spatial.plot} function requires arguments of point
locations and the resulting value to be displayed:
\begin{verbatim}
spatial.plot<- function(x,y){
 nc<-as.numeric(cut(y,20))
 plot(x,pch=" ")
 points(x,pch=20,col=topo.colors(20)[nc],cex=2)
 image.scale(y,col=topo.colors(20))
}
# To execute the function do this:
spatial.plot(cbind(xg,yg), Dn/nrow(z))
\end{verbatim}

\subsection{Example: Wolverine density map. }

The {\bf R} commands for producing density maps from MCMC output of
spatial capture-recapture models is provided in the {\bf R} function
\mbox{\tt SCRdensity} in the package \mbox{\tt scrbook}. 
We used the posterior output from the wolverine model fitted previous
to compute a relatively coarse version of a density map, using a $10 \times
10$ grid (Fig. \ref{scr0.fig.density10x10}) and using a $30 \times 30$
grid (Fig. \ref{scr0.fig.density20x20}). The {\bf R} commands for
producing such a plot (for short MCMC run) are as follows:
{\small
\begin{verbatim}
library("scrbook")
data(wolverine)
traps<-wolverine$wtraps
y3d <-SCR23darray.fn(wolverine$wcaps,wolverine$wtraps)
# this takes 341 seconds on a standard CPU circa 2011
unix.time(bln<-wolvSCR0.fn(y3d,traps,nb=1000,ni=2000,delta=1,M=100))
Sx<-bln$sims.list$s[,,1]
Sy<-bln$sims.list$s[,,2]
w<- bln$sims.list$w
obj<-list(Sx=Sx,Sy=Sy,w=w)
tmp<-SCRdensity(obj,scalein=100,scaleout=100)
\end{verbatim}
In these figures density is
expressed in units of individuals per $100$ $km^2$, while the area of
the pixels is about 103.7 $km^2$ and 11.5 $km^2$, respectively. That
calculation is based on:
\begin{verbatim}
> total.area<- (Yu-Yl)*(Xu-Xl)*100
> total.area/(10*10)
[1] 103.7427
> total.area/(30*30)
[1] 11.52697
\end{verbatim}

A couple of things are worth noting: First is that as we move away
from ``where the data live'' - away from the trap array - we see that
the density approaches the mean density. This is a property of the
estimator as long as the ``detection function'' decreases sufficiently
rapidly as a function of distance.
Relatedly, it is also a property of statistical smoothers
such as splines, kernel smoothers, and regression smoothers -
predictions tend toward the global mean as the influence of data
diminishes. Another way to think of it is that it is a consequence of
the prior - which imposes uniformity, and as you get far away from the
data, the predictions tend to the prior. The other thing to note about
this map is that density is not $0$ over water (although the coastline
is not shown). This might be perplexing
to some who are fairly certain that wolverines do not like
water. However, there is nothing about the model that recognizes water
from non-water and so the model predicts over water {\it as if} it
were habitat similar to that within which the array is nested. But,
all of this is ok as far as estimating density goes and, furthermore,
we can compute valid estimates of $N$ over any well-defined region which
presumably wouldn't include water if we so choose.

\begin{figure}
\begin{center}
\includegraphics[height=3in,width=3.375in]{Ch4/figs/density10x10}
\end{center}
\caption{Needs a caption}
\label{scr0.fig.density10x10}
\end{figure}

\begin{figure}
\begin{center}
\includegraphics[height=3in,width=3.375in]{Ch4/figs/density30x30}
\end{center}
\caption{Needs a caption}
\label{scr0.fig.density20x20}
\end{figure}

\section{Discrete State-Space}
\label{scr0.sec.discrete}

The SCR model developed previously in this chapter assumes that
individual activity centers are distributed uniformly over the
prescribed state-space. Clearly this will not always be a reasonable
assumption. In chapter \ref{chapt.state-space} we talk about developing models
that allow explicitly for non-uniformity of the activity centers by
modeling covariate effects on density. A simpler method of affecting
the distribution of activity centers, which we address here, is to
modify the shape of the state-space explicitly. For example, we might
be able to classify the state-space into distinct blocks of habitat
and non-habitat. In that case we can remove the non-habitat from the
state-space and assume uniformity of the activity centers over the
remaining portions judged to be suitable habitat.  There are two ways
to approach this: We can use a regular grid of points to represent the
state-space, i.e., by the set of coordinates ${\bf s}_1, \ldots, {\bf
  s}_{G}$, and assign a equal probabilities to each possible value, or
we can retain the continuous formulation of the state-space but use
basic polygon operations to induce constraints on the state-space We
focus here on the formulation of our basic SCR model in terms of a
discrete state-space but later on (chapter \ref{chapt.mcmc} and also
Appendix XYZ) we demonstrate the latter approach based on using
polygon operations to define an irregular state-space.

Use of a discrete state-space can be computationally expensive in {\bf
  WinBUGS}. That said, it isn't too difficult to do the MCMC
calculations in {\bf R} which we discuss briefly in chapter
\ref{chapt.mcmc}. The {\bf R} package {\tt SPACECAP}
\citep{gopalaswamy_etal:2011} arose from the {\bf R} implementation
developed for the application in \citet{royle_etal:2009}.  As we will
see in chapter \ref{chapt.mle}, we must prescribe the state-space by a
discrete mesh of points in order to do integrated likelihood and so if
we are using a discrete state-space this can be accommodated directly
in our code for obtaining MLEs.

While clipping out non-habitat seems like a good idea, its not obvious
that we accomplish any biologically reasonable objective by doing
so. We might prefer to do it when non-habitat represents a clear-cut
restriction on the state-space such as a reserve boundary or a lake,
ocean or river. It makes sense in those situations.  Unfortunately,
having the capability to do this also causes people to start defining
``habitat'' vs. ``non-habitat'' based on their understanding of the
system whereas it can't be known whether the animal being studied has
the same understanding. Moreover, differentiating of the landscape by
habitat or habitat quality probably affects the geometry and
morphology of home ranges much more than the plausible locations of
activity centers. That is, a home range centroid could, in actual
fact, occur in a walmart parking lot if there is pretty good habitat
around walmart, so there is probably no sense to cut out the walmart
lot and preclude it as the location for an activity center.  It would
generally be better to include some definition of habitat quality in
the model for the detection probability (see chapter XYZ).


\subsection{Evaluation of Coarseness of Discrete Approximation}

The coarseness of the state-space should not really have much of an
effect on estimates if the grain is sufficiently fine relative to
typical animal home range sizes.  Why is this?  We have two analogies
that can help us understand this. First is the relationship to Model
$M_{h}$.  As noted in section \ref{scr0.sec.scrmh} above, we can think
about SCR models as a type of finite mixture
\citep{norris_pollock:1996, pledger:2000} where we are fortunate to be
able to obtain direct information about which ``group'' individuals
belong to (group being location of activity center).  In the standard
finite mixture models we typically find that only 1 or a very small
number of groups (e.g., 2 or 3 at the most) can explain really high
levels of heterogeneity and are adequate for most data sets of small
to moderate sample sizes. We therefore expect a similar effect in SCR
models when we discretize the state-space.
We can also
think about discretizing the state-space as being related
to numerical integration where we find (see
chapter \ref{chapt.mle}) that we don't need a very fine
grid of support points to evaluate the integral to a reasonable
level of accuracy. We demonstrate this here by reanalyzing simulated
data using a state-space defined by a different numbers of support points.
We provide an R script called \mbox{\tt simSCR0discrete.fn} in the
{\bf R} package \mbox{\tt scrbook}.  We note that for this comparison
we generated the actual activity centers as a continuous random
variable and thus the discrete state-space is, strictly speaking, an
approximation to truth. That said, we regard all state-space
specifications as approximations to truth because they are all,
strictly speaking, models of some unknown truth. Thus the use of any
specific discrete state-space is not intrinsically more ``wrong'' than
any specific continuous representation.


We used {\bf JAGS} from the \mbox{\tt rjags} function to obtain the results
for $6 \times 6$, $9 \times 9$, $12 \times 12$, $15\times 15$,
$20\times 20$, $25 \times 25$ and $30 \times 30$ state-space grids.
We used 2000 burn, 12000 total iters with 3 chains, therefore a total
of 30000 posterior samples.
For {\bf WinBUGS} we used 3 chains of 5k total with 1k burnin means 12k
total posterior samples.
Summary results for these analyses are shown in
Table XYZ\footnote{Andy to finish later. }.

\begin{verbatim}
Table XYZ.
             Mean       SD    NaiveSE  Time-seriesSE  runtime
6    N     109.7717 15.98959 0.0923160    0.377737    1239
9    N     114.4621 16.72025 0.0965344    0.468659    1267
12   N     115.4309 17.12403 0.098866     0.464830    1576
15   N     114.7699 17.0242  0.0982894    0.425238    1638
20   N     116.0370 17.10686 0.0987665    0.486867    1647
25   N     116.3228 16.98323 0.0980527    0.465527    1661
30   N     116.4252 17.4078  0.100504     0.533735    1806
WinBUGS
             Mean       SD    NaiveSE  Time-seriesSE  runtime
6    N     111.67    16.61                             2274
9    N     114.23    17.99                             4300
12   N     115.98    17.38                             7100
15   N     115.38    17.94                            13010

Note: WinBUGS based on fewer samples too!

To get SE and time-series SE do this:
You can use as.mcmc.list() to convert to a coda object. Then use summary.�
\end{verbatim}

The results in terms of the posterior summaries are, as we
expect, very similar using {\bf WinBUGS}. However, it was interesting
to note that {\bf WinBUGS} runtime is much worse (note the number of
iterations is lower for {\bf WinBUGS} yet the runtime is much longer)
and, furthermore, it seems to scale with the size of the
discrete state-space grid. While that was expected, it was unexpected
that the runtime of {\bf JAGS} would seem relatively consistent
as we increase the grid size.
We suspect that {\bf WinBUGS} is evaluating the full-conditional for
each activity center at all $G$ possible values whereas it may be that
{\bf JAGS} is evaluating the full-conditional only at a subset of
values or perhaps using previous calculations more effectively.

While this might suggest that one should always use {\bf JAGS} for
this analysis, we found in our analysis of the wolverine (next
section) that {\bf JAGS} could be extremely sensitive to starting
values, producing MCMC algorithms that sometimes simply did not work.

\subsection{Analysis of the wolverine camera trapping data}

We reanalyzed the wolverine data using discrete state-space grids with points spaced by 2,
4 and 8 km (depicted in Fig. \ref{scr0.fig.wolvgrids}). These were
constructed from
the 40 km buffered state-space, and deleting the points over water \citep[see][]{royle_etal:2011jwm}.
 Our interest in doing this was
to evaluate the relative influence of grid resolution on estimated
density because the coarser grids will be more efficient from a
computational stand-point and so we would prefer to use them, but perhaps not
if there is a strong influence on estimated density.

{\bf Note}: Results from WinBUGS are given below -- these are updated
based on longer MCMC runs and replace prelim results as of Jan 1 2012
or so. 
To be done: density map.



\begin{figure}
\begin{center}
\includegraphics[height=2.5in,width=5in]{Ch4/figs/wolvgrids}
\end{center}
\caption{2 km 4 km and 8km wolverine state-space grids extending about
40 km from the vicinity of the trap array. }
\label{scr0.fig.wolvgrids}
\end{figure}

{\small
\begin{verbatim}
This took about 6 days in WinBUGS. Terrible mixing for the 2km and
8km. Why is this? We may never know!

> print(out.2km,digits=2)
Inference for Bugs model at "modelfile.txt", fit using WinBUGS,
 3 chains, each with 11000 iterations (first 1000 discarded)
 n.sims = 30000 iterations saved
       mean    sd  2.5%   25%   50%   75%  97.5% Rhat n.eff
psi    0.43  0.09  0.27  0.37  0.43  0.49   0.63 1.00   560
sigma  0.62  0.05  0.54  0.59  0.62  0.65   0.73 1.01   160
lam0   0.05  0.01  0.04  0.04  0.05  0.06   0.07 1.01   320
p0     0.05  0.01  0.03  0.04  0.05  0.05   0.06 1.01   320
N     86.56 16.94 57.00 75.00 85.00 97.00 124.00 1.00   510
D      8.78  1.72  5.78  7.60  8.62  9.83  12.57 1.00   510

For each parameter, n.eff is a crude measure of effective sample size,
and Rhat is the potential scale reduction factor (at convergence, Rhat=1).
> print(out.4km,digits=2)
Inference for Bugs model at "modelfile.txt", fit using WinBUGS,
 3 chains, each with 11000 iterations (first 1000 discarded)
 n.sims = 30000 iterations saved
       mean    sd  2.5%   25%   50%    75%  97.5% Rhat n.eff
psi    0.45  0.09  0.28  0.38  0.44   0.50   0.64    1  1300
sigma  0.61  0.04  0.53  0.58  0.61   0.64   0.71    1  1600
lam0   0.05  0.01  0.04  0.05  0.05   0.06   0.07    1  2500
p0     0.05  0.01  0.03  0.04  0.05   0.05   0.07    1  2500
N     89.25 17.44 59.00 77.00 88.00 100.00 127.00    1  1100
D      9.01  1.76  5.96  7.77  8.88  10.10  12.82    1  1100

For each parameter, n.eff is a crude measure of effective sample size,
and Rhat is the potential scale reduction factor (at convergence, Rhat=1).
> print(out.8km,digits=2)
Inference for Bugs model at "modelfile.txt", fit using WinBUGS,
 3 chains, each with 11000 iterations (first 1000 discarded)
 n.sims = 30000 iterations saved
       mean    sd  2.5%   25%   50%   75%  97.5% Rhat n.eff
psi    0.42  0.09  0.26  0.36  0.41  0.47   0.61 1.00   940
sigma  0.68  0.05  0.59  0.64  0.67  0.71   0.77 1.01   220
lam0   0.05  0.01  0.03  0.04  0.05  0.05   0.06 1.00   560
p0     0.05  0.01  0.03  0.04  0.04  0.05   0.06 1.00   560
N     83.18 16.14 56.00 72.00 82.00 93.00 119.00 1.00   700
D      8.28  1.61  5.57  7.17  8.16  9.26  11.84 1.00   700

For each parameter, n.eff is a crude measure of effective sample size,
and Rhat is the potential scale reduction factor (at convergence, Rhat=1).
\end{verbatim}
}

The density is a bit different depending on the grid size. Also the
effectiveness of the MCMC algorithsm is pretty remarkably different. 
We did the analysis in JAGS also. The results are shown below. {\bf Note}: I
am going to run these again but for longer to finalize the results.

{\small
\begin{verbatim}
 ### 01/10/2012 -- need to rerun these JAGS runs but use more
iterations and check results.


2km
Iterations = 7001:13000
Thinning interval = 1
Number of chains = 3
Sample size per chain = 6000

          Mean        SD  Naive SE Time-series SE
N     86.28522 16.950626 1.263e-01      0.4878973
lam0   0.04807  0.007512 5.599e-05      0.0002199
p0     0.04581  0.006820 5.083e-05      0.0001996
psi    0.28904  0.062117 4.630e-04      0.0017481
sigma  0.62769  0.043596 3.249e-04      0.0018724

4km
          Mean        SD  Naive SE Time-series SE
N     85.53139 16.998966 1.267e-01      0.5181297
lam0   0.04636  0.007542 5.621e-05      0.0002382
p0     0.04425  0.006867 5.118e-05      0.0002172
psi    0.28650  0.061922 4.615e-04      0.0018276
sigma  0.64281  0.048321 3.602e-04      0.0022911

8km
          Mean        SD  Naive SE Time-series SE
N     83.97039 16.508146 1.230e-01      0.4548782
lam0   0.04519  0.006919 5.157e-05      0.0001738
p0     0.04319  0.006319 4.710e-05      0.0001589
psi    0.28146  0.060653 4.521e-04      0.0016555
sigma  0.66956  0.040989 3.055e-04      0.0015070
\end{verbatim}
}

\subsection{SCR models as multi-state models}

While we invoke a discrete state-space artificially, by gridding the
underlying continuous state-space, sometimes the state-space is more
naturally discrete. Consider a situation in which discrete patches of
habitat are searched using some method and it might be convenient (or
occur inadvertently) to associate samples to the patch level instead
of recording observation locations. In this case we might use a model
${\bf s}_{i} \sim dcat(probs[])$  where $probs[]$ are the probabilities that
an individual inhabits a particular patch. We consider such a case
study in chapter XXPoissonXXX from \citet{mollet_etal:2012} who
obtained a population size estimate of a large grouse species known as
the capracaillie. Forest patches were searched for scat which was
identified to individual by DNA analysis.
Even when space is {\it not}
naturally discrete, measurements are often made at a fairly coarse
grain (e.g., meters or tens of meters along a stream), or associated
with spatial quadrats for scat searches and therefore the state-space
may be effectively discrete in many situations.

This discrete formulation of SCR models suggests that SCR models are
related to ordinary multi-state models \citep[][ch. 9]{kery_schaub:2011}
which are also parameterized in terms of a discrete state
variable which is often defined as a spatially-indexed state related
either to location of capture or breeding location. While many
multi-state models exist in which the state variable is not related to
space, multi-state models have been extremely useful in development
models of movements among geographic states and indeed this type of
problem motivated their early developments by \citet{arnason:1972,
  arnason:1973} and \citet{hestbeck_etal:1991}.  We pursue this
connection a little bit more in chapter XXX XYZ.




\section{ Summary and Outlook }

A point we tried to emphasize in this chapter is that the basic SCR
model is not much more than an ordinary capture-recapture model for
closed populations -- it is simply that model but augmented with a set
of ``individual effects'', ${\bf s}_{i}$, which relate encounter
probability to some sense of individual location. SCR models are
therefore a type of individual covariate model (as introduced in
chapter \ref{chapt.closed} -- but with imperfect information about the
individual covariate. In other words, they are GLMM type models when
$N$ is known or, when $N$ is unknown, they are zero-inflated GLMMs
(see \citet{royle:2006}).  Another class of capture-recapture models
that SCR models are closely related to is so-called ``Model $M_{h}$.''
The effect of introducing a spatial location for individuals is that
it induces heterogeneity in detection probability, as in Model
$M_{h}$. However, unlike Model $M_{h}$, we obtain some information
about the individual effect which is completely latent in Model
$M_{h}$. If the state-space of the random effect ${\bf s}$ is discrete
then the SCR model resembles more closely the finite-mixture class of
heterogeneity models \citep{norris_pollock:1996} which parameterizes
heterogeneity by assuming that individuals belong to discrete classes
or groups (e.g., high, medium, low). In the context of SCR models we
obtain some information about the ``group membership'' in the
locations where individuals are captured.  Given the direct
relationship of SCR models with so many standard classes of models, we
find that they are really quite easy to analyze using standard MCMC
methods encased in black boxes such as {\bf WinBUGS} or {\bf JAGS} and
possibly other packages. They are also easy to analyze using classical
likelihood methods, which we address in chapter \ref{chapt.mle}.

Formal consideration of the collection of individual locations $({\bf
  s}_{1}, \ldots, {\bf s}_{N})$ in the model is fundamental to all of
the models considered in this book. In statistical terminology, we
think of the collection of points $\{ {\bf s}_{i} \}$ as a realization of a
point process and part of the promise, and ongoing challenge, of SCR
models is to develop models that reflect interesting biological
processes, for example interactions among points or temporal dynamics
in point locations.  Here we considered the simplest possible point
process model - the points are independent and uniformly
(``randomly'') distributed over space. Despite the simplicity of this
assumption, it should suffice in many applications of SCR models
although we do address generalizations of this model in later
chapters. Moreover, even though the {\it prior} distribution on the
point locations is uniform, the realized pattern may deviate markedly
from uniformity as the observed encounter data provide information to
impart deviations from uniformity. Thus, the estimated density map
will typically appear distinctly non-uniform.  As a general rule,
information in the data will govern estimates of individual point
locations so even fairly complex patterns of non-independence or
non-uniformity will appear in the data. That is, we find in
applications of the basic SCR model that this simple {\it a priori}
model can effectively reflect or adapt to complex realizations of the
underlying point process.  For example, if individuals are highly
territorial then the data should indicate this in the form of
individuals not being encountered in the same trap - the resulting
posterior distribution of point locations should therefore reflect
non-independence.  Obviously the complexity of posterior estimates of
the point pattern will depend on the quantity of data, both number of
individuals and captures per individual.  Because the point process is
such an integral component of SCR models, the state-space of the point
process plays an important role in developing SCR models. As we tried
to emphasize in this chapter, the choice of the stat-espace is part of
the model. It can have an influence on parameter estimates and other
inferences such as model selection (see chapter \ref{chapt.gof}). We
emphasize however that this is not an arbitrary decision like
``buffering'' because the model induces an explicit interpretation of
parameters and statistical effect on estimators.

We showed how to conduct inference about the underlying point process
including calculation of density maps from posterior output. We can do
other things we normally do with spatial point processes such as
compute ``K-functions'' and test for ``complete spatial randomness''
(CSR) which we develop in chapter \ref{chapt.gof}.  Modifying and
applying point process methods to SCR problems seems to us to be a
fruitful area of research.

An obvious question that might be floating around in your mind is why
should we ever go through all of this trouble when we could just use
{\bf MARK} or {\bf CAPTURE} to get an estimate of $N$ and apply $1/2$
MMDM methods?  The main reason is that these conventional methods are
predicated on models that represent explicit misspecifications of both
the observation and ecological process - they are wrong!  Not just
wrong, because of course all models are wrong, but they're not even
{\it plausible} models! Thus while we might be able to show adequate
fit or whatever, we think as a conceptual and philosophical model one
should not be using models that are not even plausible data-generating
models -- even if the plausible ones don't fit!  Perhaps more
charitably, these ordinary non-spatial models are models of the wrong
system. They do not account for trap identity. They don't account for
spatial organization or ``clustering'' of individual encounters in
space. And, ``density'' is not a parameter of those models because
density has no meaning absent an explicit representation of space. If
we do define space explicitly, e.g., as a buffered minimum convex
hull, then the normal models ($M_{0}$, $M_{h}$, etc..) assume that
individual capture-probability is not related to space, no matter how
we define the buffer.  Conversely, the SCR model is a model for
trap-specific encounter data - how individuals are organized in space
and interact with traps. SCR models provide a coherent framework for
inference about density or population size and also, because of the
formality of their derivation, can be extended and generalized to a
large variety of different situations, as we demonstrate in subsequent
chapters.

In the next few chapters we continue to work with this basic SCR
design and model but consider some important extensions of the basic
model.  For example, we consider
extensions
to  include covariates that vary by individual, trap, or over time
(chapter \ref{chapt.covariates}), spatial covariates on density
(chapter \ref{chapt.state-space}),
 open populations (chapter \ref{chapt.open}), model assessment and
 selection (chapter \ref{chapt.gof}) and other topics.
We also consider technical details of Bayesian (chapter
\ref{chapt.mcmc}) and  maximum
likelihood (chapter \ref{chapt.mle}) estimation so that the interested
reader can develop or extend their own methods to suit their needs.


\chapter{Other observation models}
\label{chapt.poisson}

%\chapter{Alternative Models for the Encounter Process}
\label{chapt.poisson-mn}

In the previous chapter we considered a very specific although not
terribly limited observation model. The observation model consisted of
two main elements: First a description of the encounter process 
by which individuals are detected in traps. Specifically, we 
assumed individual trap-specific encounters were iid Bernoulli
trials. The consequence of this is that individuals function
independently of one another and can be captured in
any number of traps during a specific interval of trapping
effort. The type of device is typical of bear hair snares, which we
considered as an example in that section. The 2nd element of the
encounter process model was the specific model – functional form –
relating encounter probability to individual activity center
(``detection probability model'').  It is natural to consider
alternative functional forms of this detection probability model which
we do in Chapt. \ref{chapt.covariates} and elsewhere. 

In this chapter we consider alternative observation models which
accommodate Poisson or multinomial observation models. For example, if
sampling devices can detect an individual some arbitrary number of
times during an interval, then it is natural to consider observation
models for encounter frequencies, such as the Poisson model. Another
type of encounter device is the ``multi-catch'' device (REF XYZ) which
is a physical device that can capture and hold an arbitrary number of
individuals. A typical example is a mist-net for birds 
\citep{borchers_efford:2008}.

We talk about how SCR are multi-state kinds of models. 

We talk about single catch traps. 


\section{Poisson Observation Model}

The models we analyze in Chapt. \ref{chapt.scr0} assumed binary
observations -- i.e., standard encounter history data -- so
that individuals are captured at most one time in a trap.  This makes
sense for many types of DNA sampling (e.g., based on hair snares)
because distinct visits to sampled locations or devices cannot be
differentiated. However, many encounter methods or devices make it
possible to encounter an individual some arbitrary number of times
during any particular sampling episode. That is, we might observe
encounter frequencies $y_{ijk}>0$ for individual $i$, trap $j$ and
sampling interval $k$.  As an example, if a camera device is
functioning properly it may be programmed to take photos every few
seconds if triggered.  For a second example, suppose we are searching
a quadrat for scat, we may find multiple samples from the same
individual.

Therefore, we seek observation models that accommodate such encounter
frequency data.  Let $y_{ijk}$ be the frequency of encounter for
individual $i$, in trap $j$, during occasion $k$, then a plausible
model is:
\[
 y_{ijk} \sim \mbox{Poisson}(\lambda_{ij})
\]
where the expected encounter frequency $\lambda_{ij}$ depends on both
individual and trap. As we did in the binary model of chapter 4, we
now seek to model the expected value of the observation (which was
$p_{ij}$ in chapter 4) as a function of the individual activity center
${\bf s}_{i}$.
We propose 
\[
 \lambda_{ij} = \lambda_{0}  g({\bf x}_{j},{\bf s}_{i})
\]
Where $g({\bf x},{\bf s})$ is some positive valued function. 
Then $\lambda_{0}g({\bf x},{\bf s})$ is the encounter rate in trap
${\bf x}$ for an individual having activity center ${\bf s}$.  

What does this mean? This means that the encounter rate looks like a
bivariate normal distribution.  If we might interpret encounters as
resulting from the outcome of a movement model in the following
sense. Suppose that we telemeter an individual and take measurements
of location sufficiently far apart in time that locations are
independent. Let $x_{t}$ be the location at time $t$. Take a large
number of samples, make a grid and count up the number of observations
in each grid cell.
\[
 E[y(x)] = E[y(x)| moves to x]\Pr(moves to x|s) = \lambda_{0} g(x|s)
\]


For the simplest model in which we have covariates that vary across
the replicate samples $k$, we can aggregate the observed data by the
propery of compound additivity of the Poisson distribution (if $x$ and
$y$ are $iid$ Poisson with mean $\lambda$ then $x+y$ is Poisson with
mean $2\lambda$). Therefore,
\[
y_{ij} = (\sum_{k=1}^{K} y_{ijk}) =  \mbox{Poisson}(K  \lambda_{0} 
g({\bf x}_{j},{\bf s}_{i}) )
\]
We see that $K$ and $\lambda_{0}$ serve the same role as affecting the
base encounter rate. Since the observation model is the same,
probabilistically speaking, for all values of $K$, evidently we need
only $K=1$ ``survey'' from which to estimate model parameters. We know
this intuitively as sampling by multiple traps serves as replication
in SCR models.


\subsection{Poisson relationship to the Bernoulli model}

There is a sense in which the Poisson and Bernoulli models can
be viewed as consistent with one another. Note that under the Poisson
model we have:
\begin{equation}
 \Pr(y>0) = 1-exp(-\lambda_{0} g({\bf x},{\bf s}))
\label{eq.cloglog}
\end{equation}
Therefore, 
if we equate the event ``encountered'' with the event that the
individual was captured at least 1 time under the Poisson model, i.e., $y>0$, then it would be
natural to set $p_{ij} = \Pr(y>0)$ according to \ref{eq.cloglog}. 

In fact, as $\lambda_0$ gets small, the Poisson model is a close approximation
to the Bernoulli model in the sense that $y$ in that case is almost
always 0 or 1 and, in fact, $\Pr(y>0) \rightarrow \lambda$.  This is
convenient in some cases because the Poisson model might be more
tractable to fit (or even vice versa). For an example, see the models
described in Chapt. \ref{chapt.scr-unmarked}, and we also consider
another case in sec. \ref{XYZ} below.
A plot of that is in order. This near equivalence is shown in  Figure
XYZ. The left panel shows a plot of $\lambda_{ij}$ vs. distance and
superimposed on that is a plot of $p_{ij}$ vs. distance, for values
$\lambda_{0} = .1$ and $\sigma = 1$. The right panel shows a plot of
$\Pr(y>0)$ vs. $E[y]$ and we see therefore that the models are
practically equivalent. 

\begin{verbatim}
x<-seq(0.001,5,,200)
lam0<- .1
sigma<- 1
lam<- lam0*exp(-x*x/(2*sigma*sigma))

par(mfrow=c(1,2))
p1<- 1-exp(-lam)
plot(x,lam,ylab="E[y] or Pr(y>0)",xlab="distance",type="l",lwd=2)
lines(x,p1,lwd=2,col="red")
plot(lam,p1,xlab="E[y]",ylab="Pr(y>0)",type="l",lwd=2)
abline(0,1,col="red")
\end{verbatim}

So under the Poisson model we have
\[
\Pr(y>0) \approx E[y] = \lambda_{0} g(x,s)
\]
whereas in the binary model from chapter 4 we had precisely
\[
\Pr(y>0) \equiv E[y] = p_{0} g(x,s)
\]
and so the models are exactly the same for the {\it expected values}
and very similar for the probability of observing a positive response,
as long as $\lambda_{0}$ is small.


What all of this suggests it that
if we see very few observations $>1$ then we wont lose much
information by using the Bernoulli model. On the other hand, the
Poisson model is more easy to compute with in some cases. 


\begin{figure}
\centering
\includegraphics[width=5in,height=2.5in]{Ch5/figs/Poisson-Bern.png}
\label{fig:elevMap}
\end{figure}



Even if we're not in the range where the Bernoulli model provides a
good approximation, we might choose to truncate the counts to binary
observations anyhow (``quantize'').
We might do
this intentionally, but sometimes this truncation is a feature of the
sampling. For example, in the case of bear hair snares, the number of
encounters might be well approximated by a Poisson distribution but we
cannot determine unique visits and so only get to observe the binary
event ``$y>0$''. Similarly for scat sampling problems it will not
generally be possible to diagnose distinct ``independent'' scat
samples. Under this model the data are only binary encounters and we
might therefore choose a model of the form:
\[
 cloglog(p_{ij}) = log(\lambda0)  + log(g({\bf x},{\bf s}))
\]
\begin{comment} 
This example shows us that the choice of link function is typically
directly related to a specific encounter frequency model and,
furthermore, the choice of link function is equivalent to choice of
``detection function.''  As another example, what if the latent
encounter frequencies are actually geometric random variables where
the mean is a function of distance? For the case where the support of
y includes 0 – so that $y$ is the number of failures before the 1st
success, then the mean is $\mu = (1-p)/p$.  $Pr(y>0) =$ ??
\[
logit() = ….?
\]
\end{comment}

\subsection{A cautionary note on modeling encounter frequencies}

Other models for counts might be appropriate. For example, ecologists
are especially fond of negative binomial models for count data
\citep{verhoef_boveng:2007,
white_bennetts:1996,kery_etal:2005}
but other models for excess-Poisson variation are possible. For
example, we might add a normally distributed random effect to
the linear predictor.

As a general rule we favor the Bernoulli observation model even if
encounter frequencies are obtained by sampling.  The main reason is
that, with frequency data, we are forced to confront a model choice
problem (i.e., Poisson, negative binomial, log-normal mixture) that is
wholly unrelated to the fundamental space usage process that underlies
the genesis of SCR data. Repeated encounters over short time intervals
are not likely to be the result of independent encounter
processes. E.g., an individual moving back and forth in front of a
camera yields a cluster of observations that is not informative about
the spatial structure of the model. Similarly in scat surveys (e.g.,
Thompson et al. in review), dogs are used to locate scats which are
processed in the lab for individuality.  The process of local scat
deposition is not really the outcome of movement but rather the
outcome of complex behavioral considerations as well as dependence in
detection of scat by dogs. E.g., they find one and then more likely to
find a nearby one, or they get into a den area and find lots of scats.
This additional model assumption required to model variation in
observed frequencies (i.e., conditional on location) provides
relatively little information about density, and we feel that the
model selection issue should therefore be avoided.

To elaborate on this, it seems natural to construct models for
encounter data that is conditional on movement outcomes: We suppose
that an individual visits a particular location with some probability
$p_{ik}$ say $z_{ik}\sim  \mbox{Bern}(p_{ik})$ and then deposits a number of scat,
or visits a camera some number of times with frequency $y_{ik}$ which
is 
an integer $> 0$. Therefore, a sensible model might be
$[y|z][z|\phi({\bf x},{\bf s})$
where the encounter frequency $y$ is independent of ${\bf x}$ and
${\bf s}$ conditional
on the binary event ``$z$'' that the individual visited the vicinity of
the trap.

Moreover, consideration of encounter frequency data could lead to
important identifiability problems along the lines of Link (2003). The
basic Poisson model can be over-dispersed in a number of ways to
produce different models of over-dispersion.  i.e., gamma noise,
normal noise, exponential noise, etc..  Thus we have different models
of heterogeneity analogous to the class of models considered by \citet{link:2003}.


\section{Analysis of a Poisson SCR model in BUGS}

We consider the simplest possible model here in which we have no
covariates that vary over replicate samples $k$ so that we work with
the aggregated individual- and trap-specific encounters:
\[
y_{ij} = (\sum_{k=1}^{K} y_{ijk}) =  \mbox{Poisson}(K  \lambda_{ij})
\]
We consider a bivariate normal form of $g({\bf x}_{j},{\bf s}_{i})$ so
that
\[
g({\bf x}_{j},{\bf s}_{i}) = exp( -||{\bf x}_{j} - {\bf
  s}_{i}||^{2} /(2\sigma^{2}))
\]
In this case, note that 
\[
log( \lambda_{ij})  =\alpha_{0} - \beta ||{\bf x}_{j} - {\bf s}_{i}||^2
\]
where $\alpha_{0} = log(\lambda_{0})$ and $\beta = 1/(2\sigma^2)$.


As usual, we approach Bayesian analysis of these
models using data augmentation (section \ref{closed.sec.da}). 
It is interesting in this case that DA
gives us a sort of zero-inflated Poisson model which is amazingly easy
to analyze by likelihood methods which maybe we will do in Chapter
XYZ.

So the model specified conditional on $z_{i}$ is
\[
y_{ij} \sim  Poisson(z_{i} K  \lambda_{ij})
\]
which evaluates to a point mass at $y=0$ if $z=0$. 


\subsection{Simulating Data}

Simulating a sample SCR data set under the Poisson model requires only
a couple minor modifications to the procedure we used in chapter 4. In
particular, we modify the block of code which defines the model to be
that of $E[y]$ and not $\Pr(y=1)$, and we change the random variable
generator from \mbox{\tt rbinom} to \mbox{\tt rpois}:
\begin{verbatim}
D<- e2dist(S,traplocs)

alpha0<- -2.5
sigma<- 0.5
beta<- 1/(2*sigma*sigma)

muy <-  exp(alpha0)*exp(-beta*D*D)
# now generate the encounters of every individual in every trap
Y<-matrix(NA,nrow=N,ncol=ntraps)
for(i in 1:nrow(Y)){
 Y[i,]<-rpois(ntraps,K*muy[i,])
}
\end{verbatim}

We modified our code from SCR0 in chapter 4 to simulate Poisson
encounter frequencies for each trap and then we analyze an ideal data
set using WinBUGS. The new function, available in the R package, is called
{\tt simPoissonSCR.fn}. 
The simulator can produce 3-d encounter history data too although we
don't do that here. 
Here is an example of simulating a data set and harvesting the
required data objects:

\begin{verbatim}
data<-simPoissonSCR.fn(discard0=TRUE,sd=2013)
y<-data$Y
traplocs<-data$traplocs
nind<-nrow(y)
X<-data$traplocs
K<-data$K
J<-nrow(X)
Xl<-data$xlim[1]
Yl<-data$ylim[1]
Xu<-data$xlim[2]
Yu<-data$ylim[2]

## Data augmentation stuff
M<-200
y<-rbind(y,matrix(0,nrow=M-nind,ncol=ncol(y)))
z<-c(rep(1,nind),rep(0,M-nind))
\end{verbatim}

To execute WinBUGS the process is identical to what we've done
previously..............................................
here..................
.................................

The results are given below. We note about the same answer as before.

{\small
\begin{verbatim}
> print(out1,digits=2)
Inference for Bugs model at "SCR-Poisson.txt", fit using WinBUGS,
 3 chains, each with 2000 iterations (first 1000 discarded)
 n.sims = 3000 iterations saved
           mean    sd   2.5%    25%    50%    75%  97.5% Rhat n.eff
alpha0    -2.57  0.19  -2.95  -2.69  -2.57  -2.44  -2.19 1.00  2600
beta       2.34  0.36   1.69   2.08   2.32   2.57   3.12 1.00  3000
N        114.13 15.25  87.97 103.00 113.00 124.00 147.00 1.01   370
D          1.78  0.24   1.37   1.61   1.77   1.94   2.30 1.01   370
deviance 329.95 21.92 290.00 314.20 329.50 344.40 375.80 1.00  1700

For each parameter, n.eff is a crude measure of effective sample size,
and Rhat is the potential scale reduction factor (at convergence, Rhat=1).

DIC info (using the rule, pD = var(deviance)/2)
pD = 240.2 and DIC = 570.2
DIC is an estimate of expected predictive error (lower deviance is better).
\end{verbatim}


At the end of this chaptter we provide an example of a Poisson SCR model fitted to 
real data. This example has some other features which we encounter before
arriving there. 

\subsection{Exercise}

Use the Bernoulli model simulator from Chapt. \ref{chapt.scr0} (\mbox{\tt
  simSCR0.fn}) to simulate a Bernoulli data set and then fit the
Poisson model. Compare the results of fitting the correct
data-generating model with those of fitting the misspecified Poisson
model. 



\begin{comment}
\section{Likelihood analysis of the Poisson model}

Counts are Poisson with a random effect so this is stupidly easy to
implement. 
We do the normal ``full likelihood'' approach in which we retain $N$
as a real parameter in the model. We adapt \mbox{\tt intlik3} from
chapter 5 here..... behold:

Poisson(lambda(s,x))

data augmentation = ZIP
\begin{verbatim}
Pr(yi) =   ( prod_{j} dpois(y) ) *psi + I(y=0)*(1-psi)

Actually if y(i,j) = Poisson( lambda(i,j) ) then we can just add up
sum_{j} y(i,j) =  Poisson( sum_{j} lambda(i,j)) right?

 int_{s} thatthing

Zero-inflate the result
\end{verbatim}
\end{comment}

\section{No real example}

In chapt. \ref{chapt.searchencounter} we analyze the cap crap data.


\section{Independent Multinomial Observations}

Several types of encounter devices yield multinomial observations in
which an individual can be caught in a single trap during a particular
encounter occasion.  Mist nettting is a major example -- these are
``multi-catch'' traps (Efford XYZ NEED REF HERE XXXX). Also some kinds of
mammal traps hold multiples of animals and can be thought of
similarly. Another one is area-searches of reptiles where we think of
a small polygon as the ``trap'' -- we could get multiple individuals
(turtles, lizards) in the same plot but not, in the same sample
session, at different plots.  The key feature is that capture of an
individual in a trap is {\it not} independent of capture in other
traps, because they can't be captured once they are captured. On the
other hand individuals behave independently of one another, or so it
might be reasonable to assume, so whether a trap captures some other
individual doesn't have bearning on whether it captures another.  This
last assumption is violated in an extreme case in classical ``single
catch'' traps which we address in section \ref{poisson-mn.sec.singlecatch}
below. In general we could imagine non-independence being important in
any multi-catch situation but to the best of our knowledge a general
model that encompasses complete dependence (single catch) and complete
independence (multi-catch) of individuals has not been proposed.  So
we treat the cases individually and, in this section , we address the
multi-catch situation wherein individuals behave independently.


In this case we regard the observation ${\bf y}_{ik}$ for
individual $i$ during sample occasion $k$ as a multinomial observation
which consists of a sequence of 0's and at most a single 1 indiciating
the trap of capture. For example, if we capture an individual in trap
2 during a 6 sample period study then ${\bf y}_{i} = (0,1,0,0,0,0)$.
If we sample for 5 periods in all and the individual is also caught
in trap 4 during sample 3, then the 5 encounter observations for that
individual are as follows:
\begin{verbatim}
sample |---- trap ---------|
       1   2   3   4   5   6
 1     0   1   0   0   0   0
 2     0   0   0   0   0   0
 3     0   0   0   1   0   0
 4     0   0   0   0   0   0
 5     0   0   0   0   0   0
\end{verbatim}
Statistically we regard the {\it rows} of this data matrix as {\it
  independent} multinomial trials.

Analogous to our previous Bernoulli and Poisson models, we seek to
construct the multinomial cell probabilities for each individual, as a
function of {\it where} that individual lives, through its center of
activity ${\bf s}$. Thus we suppose that
\[
 {\bf y}_{ik} \sim \mbox{Multinom}(1, {\bm \pi}({\bf s}_{i}) )
\]
where ${\bm \pi}({\bf s}_{i})$ is a vector of length $J+1$, which, by
convention here, we define $\pi_{i,J+1}$, the last cell, or the ``zero
cell'', to correspond to the event ``not captured''.  Now we have to
construct these cell probabilities in some meaningful way that depends
on each individuals' ${\bf s}$, which we do shortly.

A statistically equivalent distribution is the {\it categorical} distribution.
If ${\bf y}$ is a multinomial trial with probabilities
${\bm \pi}$ than the {\it position} of the non-zero
elemment of ${\bf y}$ is a categorical random variable with probabilities
${\bm \pi}$.
We express this as
\[
{\bf y} \sim \mbox{Categorical}( {\bm \pi} )
\]
In the context of SCR models the categorical version of the
multinomial trial corresponds to the {\it trap of capture}.  Using our
example above with 6 traps then ${\bf y}_{i1} = (0,1,0,0,0,0)$ then we
could as well say $y_{ik}$ is a categorical random variable with
possible outcomes $(1,2,3,4,5,6,7)$ where outcome $y=7$ corresponds to
``not captured'' (obviously how this is organized or labeled is
completely irrelevant, although it is convenient to use the integers
$1:(J+1)$).  Therefore, $y_{i1} = 2$, $y_{i2} = 7$, $y_{i3} = 4$ and
so on.

For simulating and fitting data in the {\bf BUGS} engines we will typically use
the categorical representation of the model because it is somewhat
more convenient.  We have found that fitting multinomial models in
{\bf WinBUGS} can be extremely inefficient whereas {\bf JAGS}
typically performs much better. In the examples here, we use {\bf
  JAGS} exclusively.

\subsection{Multinomial Relationship to Poisson}

The multinomial is related directly to the Poisson encounter rate
model in the following sense. Let $y_{ij}$ be the total number of
encounters for individual $i$ in trap $j$. Then, the trap frequencies
(expluding the last cell now), if we condition on the {\it total}
number of captures, $y_{i} = \sum_{j} y_{ij}$, are multinomial with
probabilities
\[
 \pi_{ij} =  \frac{ \lambda_{ij} } { \sum_{j} \lambda_{ij} } 
\]
for $j=1,2,\ldots,J$.
Or equivalently the {\it trap of
  capture} is categorical with probabilities
\[
 \pi_{ij} =  \frac{ \lambda_{ij} } { \sum_{j} \lambda_{ij} } 
\]
which is precsely, under the half normal model, 
\[
 \pi_{ij} =  \frac{ \exp( - \beta \! dist({\bf x},{\bf s})^2 ) }  {
   \sum_{j} \exp(-\beta \! dist({\bf x},{\bf s})^2)}
\]
This expression looks like a multinomial inverse-logit transform of a model having
quadratic distance term, and also ``maximum entropy'' from MAXENT
species distribution modeling, or resource utilzation distribution
from telemetry studies.
So we can think of this multinomial model as arising naturally 
by having Poiosson encounters and then conditioning on the total. 
It is a sensible model to have anyhow, as it just allocates captures
to traps in proportion to the square of distance.  We could try other
models here too (Note: What do Borchers and Efford 2008 do?).

%People might think this multinomioal model is somehow more general
%than assuming Poisson encounter frequencies since we might cook up the
%multinomioal without having to specify a distribution for
%$y_{i}$. That said, we note that it arises under 
%If we now uncondition on the total..... 
%$y_{ij}$ is Poisson with mean $\sum_{j}$ stuff... we have a product of
%Poissons, i.e., the model we started with. 

The interpretation of this model merits some discussion. That is, 
{\it given that an individual is captured}, the probabilities given by
eq. XYZ determine 
the distribution among traps. To fully specify the model, we have to
model the probability that an individual is captured, say $p$.

We deduced the multinomioal by assuming a Poisson distribution
..... so
where did this $p$ come from?

So lets not worry about the distribution of the total count
but instead estimated this excess parameter p (this is what Royle et al.
and Gardner et al. tried to do).  In this case the multinomial gets
another cell probability , the J+1 cell, 
\[
 \pi\_{ij} =  \frac{ p exp( - beta d^2 ) }  { \sum\_{j} exp(-beta d^2)}
\]
and the last cell
\[
 \pi\_{i,J+1} =  1-p 
\]

What i like about this particular multiomial model is that whether or not
an individual is encounter in trap $j$ is just a Bernoulli trial with
probability
\[ 
(p/stuff)*exp(-beta*d^2)
\]
and if we just label (p/stuff) = p0 then this is precisely our
Bernoulli model with a half-normal detection model.  Thus we ``condition
on $y_{ij}$ and we dont have to fess up to a model for this encounter
rate, which is most of the time just reflecting behavioral stuff of the
species under, study and we wind up with a basic default Bernoulli model
which doesn't require any assumptions about the encounter rate of 
individuals.  So not having to model encounter rate seems like a good
benefit of the Bernoulli model -- which is why we said what we did above.


\subsection{Simulating data and fitting in WinBUGS}

We're going to show the nugget of a simulation function which is
used in the function \mbox{\tt sim.mnSCR} found in the {\bf R} package
\mbox{\tt scrbook}.  The first lines of the following {\bf R} code
make use of some things that should be defined but we omit them here:
{\small
\begin{verbatim}
S<-cbind(runif(N,Xl,Xu),runif(N,Yl,Yu))
# how far is each individual from each trap?
D<- e2dist(S,traplocs)

# paramter values
sigma<- 0.5
alpha0<- -1
theta<- 1/(2*sigma*sigma)

# make an empty data matrix and fill it up
Ycat<-matrix(NA,nrow=N,ncol=K)
for(i in 1:N){
for(k in 1:K){
lp<- alpha0 - theta*D[i,]*D[i,]
cp<- exp(c(lp,0))
cp<- cp/sum(cp)
Ycat[i,k]<- sample(1:(ntraps+1),1,prob=cp)
}
}
\end{verbatim}
}
The resulting data matrix in this case has the maximal dimension $N$
and so, for analysis, to mimic a real situation, we would have to discard the uncaptured
individuals. 
\mbox{\tt sim.mnSCR} will also simulate data that includes a
behavioral response, which will be the typical situation in
small-mammal trapping problems, which we first developed this code to
deal with \citep[see][for details]{converse_royle:2012}.

Here we use our function \mbox{\tt sim.mnSCR} to simulate a data set
with $K=7$ periods, etc.. We'll run the model using {\bf JAGS} which we
have found is much more effective for this class of models.
We get the data set-up for analysis by augmenting the size of the data
set to $M=200$. In addition we choose starting values for ${\bf s}$ and the
data augmentation variables $z$.  For ${\bf s}$ here we cheat a little bit
and use the true values for the obseved individuals and then augment
the matrix ${\bf S}$ with $M-n$ randomly selected activity centers.

{\small 
\begin{verbatim}
set.seed(2013)
parms<-list(N=100,alpha0= -.40, alpha1= 0,sigma=0.5)
data<-sim.mnSCR(parms,K=7,ssbuff=2)
nind<-nrow(data$Ycat)

M<-200
Ycat<-rbind(data$Ycat,matrix(nrow(data$X)+1,nrow=(M-nind),ncol=data$K))
Sst <-rbind(data$S,cbind(runif(M-nind,data$xlim[1],data$xlim[2]),
                         runif(M-nind,data$ylim[1],data$ylim[2])))
zst<-c(rep(1,160),rep(0,40))
\end{verbatim}
}

The model specification is not much more complicated than the binomial
or Poisson models given previously. The main consideration is that we
define the cell probabilities for each trap $j=1,2,\dots,J$ and then
define the last cell probability, $J+1$, for ``not captured'', to be
the complement of the sum of the others. The code is shown in Panel
\ref{poisson-mn.panel.mn}.
In the last lines of code here we
specify $N$ and density, $D$, as  derived parameters.


\begin{panel}[htp]
\centering
\rule[0.15in]{\textwidth}{.03in}
%\begin{minipage}{2.5in}
{\small
\begin{verbatim}
cat("
model {
psi ~ dunif(0,1)
alpha0 ~ dnorm(0,10)
sigma ~dunif(0,10)
theta<- 1/(2*sigma*sigma)

for(i in 1:M){
  z[i] ~ dbern(psi)
  S[i,1] ~ dunif(xlim[1],xlim[2])
  S[i,2] ~ dunif(ylim[1],ylim[2])
  for(j in 1:ntraps){
    #distance from capture to the center of the home range
    d[i,j] <- pow(pow(S[i,1]-X[j,1],2) + pow(S[i,2]-X[j,2],2),1)
  }
  for(k in 1:K){
    for(j in 1:ntraps){
      lp[i,k,j] <- exp(alpha0 - theta*d[i,j])*z[i]            
      cp[i,k,j] <- lp[i,k,j]/(1+sum(lp[i,k,]))
    }
    cp[i,k,ntraps+1] <- 1-sum(cp[i,k,1:ntraps])  # last cell = not captured
    Ycat[i,k] ~ dcat(cp[i,k,])
  }  
}   

N <- sum(z[1:M]) 
A <- ((xlim[2]-xlim[1])*trap.space)*((ylim[2]-ylim[1])*trap.space)
D <- N.tot/A
}
",file="model.txt")

\end{verbatim}
}
%\end{minipage}
\rule[-0.15in]{\textwidth}{.03in}
\caption{
WinBUGS model specification for the multinomial observation model. 
}
\label{poisson-mn.panel.mn}
\end{panel}

Finally we need to package everything up (inits, parameters, data) and send
it off to {\bf JAGS} to build a MCMC simulator for us:

{\small
\begin{verbatim}
library("rjags")

inits <- function(){list (z=zst,sigma=runif(1,.5,1) ,S=Sst) }              
parameters <- c("psi","alpha0","theta","sigma","N","D")
data <- list (X=data$X,K=data$K,trap.space=1,Ycat=Ycat,M=M,
              ntraps=nrow(data$X),ylim=data$ylim,xlim=data$xlim)         

out1 <- jags.model("model.txt", data, inits, n.chains=3, n.adapt=500)
out2 <- coda.samples(out1,parameters,n.iter=1000)
\end{verbatim}
}


Summary of analysis for the simulated data set here.....  



\section{ Mist-netting example}

Here we do an analysis of a real data set using the multinomial model.
the data are for 
adult Arctic Warblers ({\it Phylloscopus borealis}) banded 
 along the Colville River near Umiat, Alaska in 2006. The data are from 
 5 MAPS (REF) stations located in close proximity of one another, as
 well as 
 birds target (netids starting with "UMIA") or passive (netids starting 
 with "PASS") netted in the general area (a couple of nets, 
 netid == 'PASS01' and 'UMIAB3' are pretty far away though...). In total, 
 there are 258 captures of 179 individual birds. This is is really a 
 large number of birds of a single species for MAPS stations. 
 
Each of these MAPS stations has 12-15 nets.

We used XYZ....
 
A few issues:
 data is manipulated into multinomial trials and we have to convert.
 multiple captures somehow.....
 lots of space.
 transient individuals?  affect is N = number of guys ``ever available''
 


\section{SCR Models are Multi-State Models}

\begin{comment}
SCR models are multi-state models where stat-especific encounter
probabilities are a function of distance -- or something like that. 
\end{comment}

This multinomial observation model and also the discrete formulation
of the state-model given in section XYZ both allude to the fact that
SCR models are a variation of 
ordinary multi-state models \citep[][Chapt. 9]{kery_schaub:2011}
but where the state variable is static and represents a
geographic location. Multi-state models are extremely useful for
modeling movements among geographic states and indeed this type of
problem motivated their early developments by
\citet{arnason:1972,arnason:1973} and 
\citet{hestbeck_etal:1991} albeit in the context of a dynamic state
variable.  

Sometimes the state-space is naturally discrete. Consider a situation
in which discrete patches of habitat are searched using some method
and it might be convenient (or occur inadvertently) to associate
samples to the patch level instead of recording observation locations,
as in the capracillie example given in section XYZ above.  In this
case we use the discrete analog of the ``uniformity assumption'' in
which ${\bf s}_{i} \sim dcat(probs[])$ where $probs[]$ are the
probabilities that an individual inhabits a particular patch which
should be proportional to area of each patch.  Even when space is {\it
  not} naturally discrete, measurements are often made at a fairly
coarse grain (e.g., meters or tens of meters along a stream), or
associated with spatial quadrats for scat searches. And, of course, we
could approximate any continuous space with a discrete state-space,
and therefore apply multi-state models directly to any SCR problem.

\subsection{Modeling ‘manders on a stream network}

Here is a cool example: We catch salamander’s or fish along a stream
and only record stream segment instead of actual location – this is
motivated by Evan Grant’s work and also Lowe xyz??

each stream segment is individuals current state and its easy to use
either a Markov model or a home range model. ....

This is also a good open population example

\subsection{SCR as a Dynamic multi-state model}

Having a static state variable is not that interesting in the grand
scheme of multi-state models which most of the time consider a dynamic
state variable. Such models will arise frequently in spatial
capture-recapture settings. Let s denote the individual activity
center and suppose its state-space is discrete.  Now let $u[i,t]$ be
the patch in which individual $i$ was observed during sample $t$. Then
a simplistic movement model is that the successive movement outcomes
are $iid$
\[
u[i,t] \sim  dcat[ psi[s[i],] ]
\]

We can reformulate the basic SCR0 model as a dynamic multi-state model
as follows.  First lets grid up the state-space into “survey strata”
which we might define here has .5 unit squares so that the whole
state-space has 16*16 such squares. [actually do this so they are
centered on traps].We retain our assumption
\[
 s_{i} \sim Uniform(S)
\]
Secondly we define a movement model in which
\[
u[I,t] \sim dcat(pi)
\]
Where
\[
 pi_{k} = exp(-dist(x,s)/sigma2)/sum[that]
\]
This is the MAXENT distribution but also corresponds to Poisson with
mean $lam0*exp(-dist^2/sigma)$.  THIS IS CRUCIAL – THIS IS IMPT!!
 Makes it clear that encounter is the same as movement.

Now define
\[
 p|u[i,t] = p0*if(u[i,t] \in trap grid cell)
\]

Multi-state model with a “random movement” process.


We could easily extend this to a kind of Markovian movement model
where the probabilities depend on the previous state $u_{i,t-1}$ but
the simpler model of ``random'' movement satisfies our immediate needs.
 
So we see that SCR models are exactly a type of multi-state model when
the states are naturally discrete.  Another naturally discrete
state-space is ``nest sites''. Goncalo’s study and Florent’s
study. Schaub’s study on woopoos.


\section{Single-catch traps}
\label{poisson-mn.sec.singlecatch}

The classical animal trapping experiment is based on a physical trap
which captures a single animal and holds that individual until
subsequent molestation by a biologist. 
This type of observation model -- the ``single catch'' trap -- 
was the original situation considered by \citet{efford:2004}.

The single-catch model is basically a multinomial model but one in
which the number of available traps is reduced as each individual is
captured. As such, the constraints on the likelihood for each
individual are latent and shit is complicated beyond belief.
As a result, at the time of this writing, there has not been a formal
development of either likelihood  or Bayesian analysis of this model.

Nevertheless, it is not too difficult to describe the basic model
formally. In particular, there is a nice conditional structure resulting from a ``removal
process'' operating on the traps.  The first guy captured has the 
basic multinomial observation model:
\[
{\bf y}_{i} \sim Multinom({\bm \pi}_{i})
\]
whereas the 2nd guy captured has one cell removed:
\[
{\bf y}_{i} \sim Multinom({\bm \pi}_{i}(1-{\bf y}_{i})    )
\]
and so on.
So the {\bf order of capture} is relevant to the construction of these
multinomial cell probabilities. 
Thus the observations each have a multinomial model, but the
multinomial observations have a unique kind of conditional dependence
structure among them.

\subsection{Approximate Analysis}

To analyze the model here we consider using a misspecified model based
on either the Poisson or independent multinomial


How good of an approximation is the multi-catch model?

What about the Poisson model with a really low lambda?

Can we solve the big kahuna?

Use Sarah's data here.


\section{Trapping Webs}


\section{Acoustic Arrays}


\section{Summary and Outlook}

There are other types of encounter models.......

Efford adapts SCR models to acoustic detection devices.... a few words
on that here.....

There are models for which
only specific summary statistics are observable (Chandler and Royle
2011, etc..) which we cover in chapter XYZ.  We consider other models
for detection probability in some prior or later chapter. 







\chapter{Maximum likelihood estimation}
\label{chapt.mle}

\chapter{Alternative Models for the Encounter Process}
\label{chapt.poisson-mn}

In the previous chapter we considered a very specific although not
terribly limited observation model. The observation model consisted of
two main elements: First a description of the encounter process 
by which individuals are detected in traps. Specifically, we 
assumed individual trap-specific encounters were iid Bernoulli
trials. The consequence of this is that individuals function
independently of one another and can be captured in
any number of traps during a specific interval of trapping
effort. The type of device is typical of bear hair snares, which we
considered as an example in that section. The 2nd element of the
encounter process model was the specific model – functional form –
relating encounter probability to individual activity center
(``detection probability model'').  It is natural to consider
alternative functional forms of this detection probability model which
we do in Chapt. \ref{chapt.covariates} and elsewhere. 

In this chapter we consider alternative observation models which
accommodate Poisson or multinomial observation models. For example, if
sampling devices can detect an individual some arbitrary number of
times during an interval, then it is natural to consider observation
models for encounter frequencies, such as the Poisson model. Another
type of encounter device is the ``multi-catch'' device (REF XYZ) which
is a physical device that can capture and hold an arbitrary number of
individuals. A typical example is a mist-net for birds 
\citep{borchers_efford:2008}.

We talk about how SCR are multi-state kinds of models. 

We talk about single catch traps. 


\section{Poisson Observation Model}

The models we analyze in Chapt. \ref{chapt.scr0} assumed binary
observations -- i.e., standard encounter history data -- so
that individuals are captured at most one time in a trap.  This makes
sense for many types of DNA sampling (e.g., based on hair snares)
because distinct visits to sampled locations or devices cannot be
differentiated. However, many encounter methods or devices make it
possible to encounter an individual some arbitrary number of times
during any particular sampling episode. That is, we might observe
encounter frequencies $y_{ijk}>0$ for individual $i$, trap $j$ and
sampling interval $k$.  As an example, if a camera device is
functioning properly it may be programmed to take photos every few
seconds if triggered.  For a second example, suppose we are searching
a quadrat for scat, we may find multiple samples from the same
individual.

Therefore, we seek observation models that accommodate such encounter
frequency data.  Let $y_{ijk}$ be the frequency of encounter for
individual $i$, in trap $j$, during occasion $k$, then a plausible
model is:
\[
 y_{ijk} \sim \mbox{Poisson}(\lambda_{ij})
\]
where the expected encounter frequency $\lambda_{ij}$ depends on both
individual and trap. As we did in the binary model of chapter 4, we
now seek to model the expected value of the observation (which was
$p_{ij}$ in chapter 4) as a function of the individual activity center
${\bf s}_{i}$.
We propose 
\[
 \lambda_{ij} = \lambda_{0}  g({\bf x}_{j},{\bf s}_{i})
\]
Where $g({\bf x},{\bf s})$ is some positive valued function. 
Then $\lambda_{0}g({\bf x},{\bf s})$ is the encounter rate in trap
${\bf x}$ for an individual having activity center ${\bf s}$.  

What does this mean? This means that the encounter rate looks like a
bivariate normal distribution.  If we might interpret encounters as
resulting from the outcome of a movement model in the following
sense. Suppose that we telemeter an individual and take measurements
of location sufficiently far apart in time that locations are
independent. Let $x_{t}$ be the location at time $t$. Take a large
number of samples, make a grid and count up the number of observations
in each grid cell.
\[
 E[y(x)] = E[y(x)| moves to x]\Pr(moves to x|s) = \lambda_{0} g(x|s)
\]


For the simplest model in which we have covariates that vary across
the replicate samples $k$, we can aggregate the observed data by the
propery of compound additivity of the Poisson distribution (if $x$ and
$y$ are $iid$ Poisson with mean $\lambda$ then $x+y$ is Poisson with
mean $2\lambda$). Therefore,
\[
y_{ij} = (\sum_{k=1}^{K} y_{ijk}) =  \mbox{Poisson}(K  \lambda_{0} 
g({\bf x}_{j},{\bf s}_{i}) )
\]
We see that $K$ and $\lambda_{0}$ serve the same role as affecting the
base encounter rate. Since the observation model is the same,
probabilistically speaking, for all values of $K$, evidently we need
only $K=1$ ``survey'' from which to estimate model parameters. We know
this intuitively as sampling by multiple traps serves as replication
in SCR models.


\subsection{Poisson relationship to the Bernoulli model}

There is a sense in which the Poisson and Bernoulli models can
be viewed as consistent with one another. Note that under the Poisson
model we have:
\begin{equation}
 \Pr(y>0) = 1-exp(-\lambda_{0} g({\bf x},{\bf s}))
\label{eq.cloglog}
\end{equation}
Therefore, 
if we equate the event ``encountered'' with the event that the
individual was captured at least 1 time under the Poisson model, i.e., $y>0$, then it would be
natural to set $p_{ij} = \Pr(y>0)$ according to \ref{eq.cloglog}. 

In fact, as $\lambda_0$ gets small, the Poisson model is a close approximation
to the Bernoulli model in the sense that $y$ in that case is almost
always 0 or 1 and, in fact, $\Pr(y>0) \rightarrow \lambda$.  This is
convenient in some cases because the Poisson model might be more
tractable to fit (or even vice versa). For an example, see the models
described in Chapt. \ref{chapt.scr-unmarked}, and we also consider
another case in sec. \ref{XYZ} below.
A plot of that is in order. This near equivalence is shown in  Figure
XYZ. The left panel shows a plot of $\lambda_{ij}$ vs. distance and
superimposed on that is a plot of $p_{ij}$ vs. distance, for values
$\lambda_{0} = .1$ and $\sigma = 1$. The right panel shows a plot of
$\Pr(y>0)$ vs. $E[y]$ and we see therefore that the models are
practically equivalent. 

\begin{verbatim}
x<-seq(0.001,5,,200)
lam0<- .1
sigma<- 1
lam<- lam0*exp(-x*x/(2*sigma*sigma))

par(mfrow=c(1,2))
p1<- 1-exp(-lam)
plot(x,lam,ylab="E[y] or Pr(y>0)",xlab="distance",type="l",lwd=2)
lines(x,p1,lwd=2,col="red")
plot(lam,p1,xlab="E[y]",ylab="Pr(y>0)",type="l",lwd=2)
abline(0,1,col="red")
\end{verbatim}

So under the Poisson model we have
\[
\Pr(y>0) \approx E[y] = \lambda_{0} g(x,s)
\]
whereas in the binary model from chapter 4 we had precisely
\[
\Pr(y>0) \equiv E[y] = p_{0} g(x,s)
\]
and so the models are exactly the same for the {\it expected values}
and very similar for the probability of observing a positive response,
as long as $\lambda_{0}$ is small.


What all of this suggests it that
if we see very few observations $>1$ then we wont lose much
information by using the Bernoulli model. On the other hand, the
Poisson model is more easy to compute with in some cases. 


\begin{figure}
\centering
\includegraphics[width=5in,height=2.5in]{Ch5/figs/Poisson-Bern.png}
\label{fig:elevMap}
\end{figure}



Even if we're not in the range where the Bernoulli model provides a
good approximation, we might choose to truncate the counts to binary
observations anyhow (``quantize'').
We might do
this intentionally, but sometimes this truncation is a feature of the
sampling. For example, in the case of bear hair snares, the number of
encounters might be well approximated by a Poisson distribution but we
cannot determine unique visits and so only get to observe the binary
event ``$y>0$''. Similarly for scat sampling problems it will not
generally be possible to diagnose distinct ``independent'' scat
samples. Under this model the data are only binary encounters and we
might therefore choose a model of the form:
\[
 cloglog(p_{ij}) = log(\lambda0)  + log(g({\bf x},{\bf s}))
\]
\begin{comment} 
This example shows us that the choice of link function is typically
directly related to a specific encounter frequency model and,
furthermore, the choice of link function is equivalent to choice of
``detection function.''  As another example, what if the latent
encounter frequencies are actually geometric random variables where
the mean is a function of distance? For the case where the support of
y includes 0 – so that $y$ is the number of failures before the 1st
success, then the mean is $\mu = (1-p)/p$.  $Pr(y>0) =$ ??
\[
logit() = ….?
\]
\end{comment}

\subsection{A cautionary note on modeling encounter frequencies}

Other models for counts might be appropriate. For example, ecologists
are especially fond of negative binomial models for count data
\citep{verhoef_boveng:2007,
white_bennetts:1996,kery_etal:2005}
but other models for excess-Poisson variation are possible. For
example, we might add a normally distributed random effect to
the linear predictor.

As a general rule we favor the Bernoulli observation model even if
encounter frequencies are obtained by sampling.  The main reason is
that, with frequency data, we are forced to confront a model choice
problem (i.e., Poisson, negative binomial, log-normal mixture) that is
wholly unrelated to the fundamental space usage process that underlies
the genesis of SCR data. Repeated encounters over short time intervals
are not likely to be the result of independent encounter
processes. E.g., an individual moving back and forth in front of a
camera yields a cluster of observations that is not informative about
the spatial structure of the model. Similarly in scat surveys (e.g.,
Thompson et al. in review), dogs are used to locate scats which are
processed in the lab for individuality.  The process of local scat
deposition is not really the outcome of movement but rather the
outcome of complex behavioral considerations as well as dependence in
detection of scat by dogs. E.g., they find one and then more likely to
find a nearby one, or they get into a den area and find lots of scats.
This additional model assumption required to model variation in
observed frequencies (i.e., conditional on location) provides
relatively little information about density, and we feel that the
model selection issue should therefore be avoided.

To elaborate on this, it seems natural to construct models for
encounter data that is conditional on movement outcomes: We suppose
that an individual visits a particular location with some probability
$p_{ik}$ say $z_{ik}\sim  \mbox{Bern}(p_{ik})$ and then deposits a number of scat,
or visits a camera some number of times with frequency $y_{ik}$ which
is 
an integer $> 0$. Therefore, a sensible model might be
$[y|z][z|\phi({\bf x},{\bf s})$
where the encounter frequency $y$ is independent of ${\bf x}$ and
${\bf s}$ conditional
on the binary event ``$z$'' that the individual visited the vicinity of
the trap.

Moreover, consideration of encounter frequency data could lead to
important identifiability problems along the lines of Link (2003). The
basic Poisson model can be over-dispersed in a number of ways to
produce different models of over-dispersion.  i.e., gamma noise,
normal noise, exponential noise, etc..  Thus we have different models
of heterogeneity analogous to the class of models considered by \citet{link:2003}.


\section{Analysis of a Poisson SCR model in BUGS}

We consider the simplest possible model here in which we have no
covariates that vary over replicate samples $k$ so that we work with
the aggregated individual- and trap-specific encounters:
\[
y_{ij} = (\sum_{k=1}^{K} y_{ijk}) =  \mbox{Poisson}(K  \lambda_{ij})
\]
We consider a bivariate normal form of $g({\bf x}_{j},{\bf s}_{i})$ so
that
\[
g({\bf x}_{j},{\bf s}_{i}) = exp( -||{\bf x}_{j} - {\bf
  s}_{i}||^{2} /(2\sigma^{2}))
\]
In this case, note that 
\[
log( \lambda_{ij})  =\alpha_{0} - \beta ||{\bf x}_{j} - {\bf s}_{i}||^2
\]
where $\alpha_{0} = log(\lambda_{0})$ and $\beta = 1/(2\sigma^2)$.


As usual, we approach Bayesian analysis of these
models using data augmentation (section \ref{closed.sec.da}). 
It is interesting in this case that DA
gives us a sort of zero-inflated Poisson model which is amazingly easy
to analyze by likelihood methods which maybe we will do in Chapter
XYZ.

So the model specified conditional on $z_{i}$ is
\[
y_{ij} \sim  Poisson(z_{i} K  \lambda_{ij})
\]
which evaluates to a point mass at $y=0$ if $z=0$. 


\subsection{Simulating Data}

Simulating a sample SCR data set under the Poisson model requires only
a couple minor modifications to the procedure we used in chapter 4. In
particular, we modify the block of code which defines the model to be
that of $E[y]$ and not $\Pr(y=1)$, and we change the random variable
generator from \mbox{\tt rbinom} to \mbox{\tt rpois}:
\begin{verbatim}
D<- e2dist(S,traplocs)

alpha0<- -2.5
sigma<- 0.5
beta<- 1/(2*sigma*sigma)

muy <-  exp(alpha0)*exp(-beta*D*D)
# now generate the encounters of every individual in every trap
Y<-matrix(NA,nrow=N,ncol=ntraps)
for(i in 1:nrow(Y)){
 Y[i,]<-rpois(ntraps,K*muy[i,])
}
\end{verbatim}

We modified our code from SCR0 in chapter 4 to simulate Poisson
encounter frequencies for each trap and then we analyze an ideal data
set using WinBUGS. The new function, available in the R package, is called
{\tt simPoissonSCR.fn}. 
The simulator can produce 3-d encounter history data too although we
don't do that here. 
Here is an example of simulating a data set and harvesting the
required data objects:

\begin{verbatim}
data<-simPoissonSCR.fn(discard0=TRUE,sd=2013)
y<-data$Y
traplocs<-data$traplocs
nind<-nrow(y)
X<-data$traplocs
K<-data$K
J<-nrow(X)
Xl<-data$xlim[1]
Yl<-data$ylim[1]
Xu<-data$xlim[2]
Yu<-data$ylim[2]

## Data augmentation stuff
M<-200
y<-rbind(y,matrix(0,nrow=M-nind,ncol=ncol(y)))
z<-c(rep(1,nind),rep(0,M-nind))
\end{verbatim}

To execute WinBUGS the process is identical to what we've done
previously..............................................
here..................
.................................

The results are given below. We note about the same answer as before.

{\small
\begin{verbatim}
> print(out1,digits=2)
Inference for Bugs model at "SCR-Poisson.txt", fit using WinBUGS,
 3 chains, each with 2000 iterations (first 1000 discarded)
 n.sims = 3000 iterations saved
           mean    sd   2.5%    25%    50%    75%  97.5% Rhat n.eff
alpha0    -2.57  0.19  -2.95  -2.69  -2.57  -2.44  -2.19 1.00  2600
beta       2.34  0.36   1.69   2.08   2.32   2.57   3.12 1.00  3000
N        114.13 15.25  87.97 103.00 113.00 124.00 147.00 1.01   370
D          1.78  0.24   1.37   1.61   1.77   1.94   2.30 1.01   370
deviance 329.95 21.92 290.00 314.20 329.50 344.40 375.80 1.00  1700

For each parameter, n.eff is a crude measure of effective sample size,
and Rhat is the potential scale reduction factor (at convergence, Rhat=1).

DIC info (using the rule, pD = var(deviance)/2)
pD = 240.2 and DIC = 570.2
DIC is an estimate of expected predictive error (lower deviance is better).
\end{verbatim}


At the end of this chaptter we provide an example of a Poisson SCR model fitted to 
real data. This example has some other features which we encounter before
arriving there. 

\subsection{Exercise}

Use the Bernoulli model simulator from Chapt. \ref{chapt.scr0} (\mbox{\tt
  simSCR0.fn}) to simulate a Bernoulli data set and then fit the
Poisson model. Compare the results of fitting the correct
data-generating model with those of fitting the misspecified Poisson
model. 



\begin{comment}
\section{Likelihood analysis of the Poisson model}

Counts are Poisson with a random effect so this is stupidly easy to
implement. 
We do the normal ``full likelihood'' approach in which we retain $N$
as a real parameter in the model. We adapt \mbox{\tt intlik3} from
chapter 5 here..... behold:

Poisson(lambda(s,x))

data augmentation = ZIP
\begin{verbatim}
Pr(yi) =   ( prod_{j} dpois(y) ) *psi + I(y=0)*(1-psi)

Actually if y(i,j) = Poisson( lambda(i,j) ) then we can just add up
sum_{j} y(i,j) =  Poisson( sum_{j} lambda(i,j)) right?

 int_{s} thatthing

Zero-inflate the result
\end{verbatim}
\end{comment}

\section{No real example}

In chapt. \ref{chapt.searchencounter} we analyze the cap crap data.


\section{Independent Multinomial Observations}

Several types of encounter devices yield multinomial observations in
which an individual can be caught in a single trap during a particular
encounter occasion.  Mist nettting is a major example -- these are
``multi-catch'' traps (Efford XYZ NEED REF HERE XXXX). Also some kinds of
mammal traps hold multiples of animals and can be thought of
similarly. Another one is area-searches of reptiles where we think of
a small polygon as the ``trap'' -- we could get multiple individuals
(turtles, lizards) in the same plot but not, in the same sample
session, at different plots.  The key feature is that capture of an
individual in a trap is {\it not} independent of capture in other
traps, because they can't be captured once they are captured. On the
other hand individuals behave independently of one another, or so it
might be reasonable to assume, so whether a trap captures some other
individual doesn't have bearning on whether it captures another.  This
last assumption is violated in an extreme case in classical ``single
catch'' traps which we address in section \ref{poisson-mn.sec.singlecatch}
below. In general we could imagine non-independence being important in
any multi-catch situation but to the best of our knowledge a general
model that encompasses complete dependence (single catch) and complete
independence (multi-catch) of individuals has not been proposed.  So
we treat the cases individually and, in this section , we address the
multi-catch situation wherein individuals behave independently.


In this case we regard the observation ${\bf y}_{ik}$ for
individual $i$ during sample occasion $k$ as a multinomial observation
which consists of a sequence of 0's and at most a single 1 indiciating
the trap of capture. For example, if we capture an individual in trap
2 during a 6 sample period study then ${\bf y}_{i} = (0,1,0,0,0,0)$.
If we sample for 5 periods in all and the individual is also caught
in trap 4 during sample 3, then the 5 encounter observations for that
individual are as follows:
\begin{verbatim}
sample |---- trap ---------|
       1   2   3   4   5   6
 1     0   1   0   0   0   0
 2     0   0   0   0   0   0
 3     0   0   0   1   0   0
 4     0   0   0   0   0   0
 5     0   0   0   0   0   0
\end{verbatim}
Statistically we regard the {\it rows} of this data matrix as {\it
  independent} multinomial trials.

Analogous to our previous Bernoulli and Poisson models, we seek to
construct the multinomial cell probabilities for each individual, as a
function of {\it where} that individual lives, through its center of
activity ${\bf s}$. Thus we suppose that
\[
 {\bf y}_{ik} \sim \mbox{Multinom}(1, {\bm \pi}({\bf s}_{i}) )
\]
where ${\bm \pi}({\bf s}_{i})$ is a vector of length $J+1$, which, by
convention here, we define $\pi_{i,J+1}$, the last cell, or the ``zero
cell'', to correspond to the event ``not captured''.  Now we have to
construct these cell probabilities in some meaningful way that depends
on each individuals' ${\bf s}$, which we do shortly.

A statistically equivalent distribution is the {\it categorical} distribution.
If ${\bf y}$ is a multinomial trial with probabilities
${\bm \pi}$ than the {\it position} of the non-zero
elemment of ${\bf y}$ is a categorical random variable with probabilities
${\bm \pi}$.
We express this as
\[
{\bf y} \sim \mbox{Categorical}( {\bm \pi} )
\]
In the context of SCR models the categorical version of the
multinomial trial corresponds to the {\it trap of capture}.  Using our
example above with 6 traps then ${\bf y}_{i1} = (0,1,0,0,0,0)$ then we
could as well say $y_{ik}$ is a categorical random variable with
possible outcomes $(1,2,3,4,5,6,7)$ where outcome $y=7$ corresponds to
``not captured'' (obviously how this is organized or labeled is
completely irrelevant, although it is convenient to use the integers
$1:(J+1)$).  Therefore, $y_{i1} = 2$, $y_{i2} = 7$, $y_{i3} = 4$ and
so on.

For simulating and fitting data in the {\bf BUGS} engines we will typically use
the categorical representation of the model because it is somewhat
more convenient.  We have found that fitting multinomial models in
{\bf WinBUGS} can be extremely inefficient whereas {\bf JAGS}
typically performs much better. In the examples here, we use {\bf
  JAGS} exclusively.

\subsection{Multinomial Relationship to Poisson}

The multinomial is related directly to the Poisson encounter rate
model in the following sense. Let $y_{ij}$ be the total number of
encounters for individual $i$ in trap $j$. Then, the trap frequencies
(expluding the last cell now), if we condition on the {\it total}
number of captures, $y_{i} = \sum_{j} y_{ij}$, are multinomial with
probabilities
\[
 \pi_{ij} =  \frac{ \lambda_{ij} } { \sum_{j} \lambda_{ij} } 
\]
for $j=1,2,\ldots,J$.
Or equivalently the {\it trap of
  capture} is categorical with probabilities
\[
 \pi_{ij} =  \frac{ \lambda_{ij} } { \sum_{j} \lambda_{ij} } 
\]
which is precsely, under the half normal model, 
\[
 \pi_{ij} =  \frac{ \exp( - \beta \! dist({\bf x},{\bf s})^2 ) }  {
   \sum_{j} \exp(-\beta \! dist({\bf x},{\bf s})^2)}
\]
This expression looks like a multinomial inverse-logit transform of a model having
quadratic distance term, and also ``maximum entropy'' from MAXENT
species distribution modeling, or resource utilzation distribution
from telemetry studies.
So we can think of this multinomial model as arising naturally 
by having Poiosson encounters and then conditioning on the total. 
It is a sensible model to have anyhow, as it just allocates captures
to traps in proportion to the square of distance.  We could try other
models here too (Note: What do Borchers and Efford 2008 do?).

%People might think this multinomioal model is somehow more general
%than assuming Poisson encounter frequencies since we might cook up the
%multinomioal without having to specify a distribution for
%$y_{i}$. That said, we note that it arises under 
%If we now uncondition on the total..... 
%$y_{ij}$ is Poisson with mean $\sum_{j}$ stuff... we have a product of
%Poissons, i.e., the model we started with. 

The interpretation of this model merits some discussion. That is, 
{\it given that an individual is captured}, the probabilities given by
eq. XYZ determine 
the distribution among traps. To fully specify the model, we have to
model the probability that an individual is captured, say $p$.

We deduced the multinomioal by assuming a Poisson distribution
..... so
where did this $p$ come from?

So lets not worry about the distribution of the total count
but instead estimated this excess parameter p (this is what Royle et al.
and Gardner et al. tried to do).  In this case the multinomial gets
another cell probability , the J+1 cell, 
\[
 \pi\_{ij} =  \frac{ p exp( - beta d^2 ) }  { \sum\_{j} exp(-beta d^2)}
\]
and the last cell
\[
 \pi\_{i,J+1} =  1-p 
\]

What i like about this particular multiomial model is that whether or not
an individual is encounter in trap $j$ is just a Bernoulli trial with
probability
\[ 
(p/stuff)*exp(-beta*d^2)
\]
and if we just label (p/stuff) = p0 then this is precisely our
Bernoulli model with a half-normal detection model.  Thus we ``condition
on $y_{ij}$ and we dont have to fess up to a model for this encounter
rate, which is most of the time just reflecting behavioral stuff of the
species under, study and we wind up with a basic default Bernoulli model
which doesn't require any assumptions about the encounter rate of 
individuals.  So not having to model encounter rate seems like a good
benefit of the Bernoulli model -- which is why we said what we did above.


\subsection{Simulating data and fitting in WinBUGS}

We're going to show the nugget of a simulation function which is
used in the function \mbox{\tt sim.mnSCR} found in the {\bf R} package
\mbox{\tt scrbook}.  The first lines of the following {\bf R} code
make use of some things that should be defined but we omit them here:
{\small
\begin{verbatim}
S<-cbind(runif(N,Xl,Xu),runif(N,Yl,Yu))
# how far is each individual from each trap?
D<- e2dist(S,traplocs)

# paramter values
sigma<- 0.5
alpha0<- -1
theta<- 1/(2*sigma*sigma)

# make an empty data matrix and fill it up
Ycat<-matrix(NA,nrow=N,ncol=K)
for(i in 1:N){
for(k in 1:K){
lp<- alpha0 - theta*D[i,]*D[i,]
cp<- exp(c(lp,0))
cp<- cp/sum(cp)
Ycat[i,k]<- sample(1:(ntraps+1),1,prob=cp)
}
}
\end{verbatim}
}
The resulting data matrix in this case has the maximal dimension $N$
and so, for analysis, to mimic a real situation, we would have to discard the uncaptured
individuals. 
\mbox{\tt sim.mnSCR} will also simulate data that includes a
behavioral response, which will be the typical situation in
small-mammal trapping problems, which we first developed this code to
deal with \citep[see][for details]{converse_royle:2012}.

Here we use our function \mbox{\tt sim.mnSCR} to simulate a data set
with $K=7$ periods, etc.. We'll run the model using {\bf JAGS} which we
have found is much more effective for this class of models.
We get the data set-up for analysis by augmenting the size of the data
set to $M=200$. In addition we choose starting values for ${\bf s}$ and the
data augmentation variables $z$.  For ${\bf s}$ here we cheat a little bit
and use the true values for the obseved individuals and then augment
the matrix ${\bf S}$ with $M-n$ randomly selected activity centers.

{\small 
\begin{verbatim}
set.seed(2013)
parms<-list(N=100,alpha0= -.40, alpha1= 0,sigma=0.5)
data<-sim.mnSCR(parms,K=7,ssbuff=2)
nind<-nrow(data$Ycat)

M<-200
Ycat<-rbind(data$Ycat,matrix(nrow(data$X)+1,nrow=(M-nind),ncol=data$K))
Sst <-rbind(data$S,cbind(runif(M-nind,data$xlim[1],data$xlim[2]),
                         runif(M-nind,data$ylim[1],data$ylim[2])))
zst<-c(rep(1,160),rep(0,40))
\end{verbatim}
}

The model specification is not much more complicated than the binomial
or Poisson models given previously. The main consideration is that we
define the cell probabilities for each trap $j=1,2,\dots,J$ and then
define the last cell probability, $J+1$, for ``not captured'', to be
the complement of the sum of the others. The code is shown in Panel
\ref{poisson-mn.panel.mn}.
In the last lines of code here we
specify $N$ and density, $D$, as  derived parameters.


\begin{panel}[htp]
\centering
\rule[0.15in]{\textwidth}{.03in}
%\begin{minipage}{2.5in}
{\small
\begin{verbatim}
cat("
model {
psi ~ dunif(0,1)
alpha0 ~ dnorm(0,10)
sigma ~dunif(0,10)
theta<- 1/(2*sigma*sigma)

for(i in 1:M){
  z[i] ~ dbern(psi)
  S[i,1] ~ dunif(xlim[1],xlim[2])
  S[i,2] ~ dunif(ylim[1],ylim[2])
  for(j in 1:ntraps){
    #distance from capture to the center of the home range
    d[i,j] <- pow(pow(S[i,1]-X[j,1],2) + pow(S[i,2]-X[j,2],2),1)
  }
  for(k in 1:K){
    for(j in 1:ntraps){
      lp[i,k,j] <- exp(alpha0 - theta*d[i,j])*z[i]            
      cp[i,k,j] <- lp[i,k,j]/(1+sum(lp[i,k,]))
    }
    cp[i,k,ntraps+1] <- 1-sum(cp[i,k,1:ntraps])  # last cell = not captured
    Ycat[i,k] ~ dcat(cp[i,k,])
  }  
}   

N <- sum(z[1:M]) 
A <- ((xlim[2]-xlim[1])*trap.space)*((ylim[2]-ylim[1])*trap.space)
D <- N.tot/A
}
",file="model.txt")

\end{verbatim}
}
%\end{minipage}
\rule[-0.15in]{\textwidth}{.03in}
\caption{
WinBUGS model specification for the multinomial observation model. 
}
\label{poisson-mn.panel.mn}
\end{panel}

Finally we need to package everything up (inits, parameters, data) and send
it off to {\bf JAGS} to build a MCMC simulator for us:

{\small
\begin{verbatim}
library("rjags")

inits <- function(){list (z=zst,sigma=runif(1,.5,1) ,S=Sst) }              
parameters <- c("psi","alpha0","theta","sigma","N","D")
data <- list (X=data$X,K=data$K,trap.space=1,Ycat=Ycat,M=M,
              ntraps=nrow(data$X),ylim=data$ylim,xlim=data$xlim)         

out1 <- jags.model("model.txt", data, inits, n.chains=3, n.adapt=500)
out2 <- coda.samples(out1,parameters,n.iter=1000)
\end{verbatim}
}


Summary of analysis for the simulated data set here.....  



\section{ Mist-netting example}

Here we do an analysis of a real data set using the multinomial model.
the data are for 
adult Arctic Warblers ({\it Phylloscopus borealis}) banded 
 along the Colville River near Umiat, Alaska in 2006. The data are from 
 5 MAPS (REF) stations located in close proximity of one another, as
 well as 
 birds target (netids starting with "UMIA") or passive (netids starting 
 with "PASS") netted in the general area (a couple of nets, 
 netid == 'PASS01' and 'UMIAB3' are pretty far away though...). In total, 
 there are 258 captures of 179 individual birds. This is is really a 
 large number of birds of a single species for MAPS stations. 
 
Each of these MAPS stations has 12-15 nets.

We used XYZ....
 
A few issues:
 data is manipulated into multinomial trials and we have to convert.
 multiple captures somehow.....
 lots of space.
 transient individuals?  affect is N = number of guys ``ever available''
 


\section{SCR Models are Multi-State Models}

\begin{comment}
SCR models are multi-state models where stat-especific encounter
probabilities are a function of distance -- or something like that. 
\end{comment}

This multinomial observation model and also the discrete formulation
of the state-model given in section XYZ both allude to the fact that
SCR models are a variation of 
ordinary multi-state models \citep[][Chapt. 9]{kery_schaub:2011}
but where the state variable is static and represents a
geographic location. Multi-state models are extremely useful for
modeling movements among geographic states and indeed this type of
problem motivated their early developments by
\citet{arnason:1972,arnason:1973} and 
\citet{hestbeck_etal:1991} albeit in the context of a dynamic state
variable.  

Sometimes the state-space is naturally discrete. Consider a situation
in which discrete patches of habitat are searched using some method
and it might be convenient (or occur inadvertently) to associate
samples to the patch level instead of recording observation locations,
as in the capracillie example given in section XYZ above.  In this
case we use the discrete analog of the ``uniformity assumption'' in
which ${\bf s}_{i} \sim dcat(probs[])$ where $probs[]$ are the
probabilities that an individual inhabits a particular patch which
should be proportional to area of each patch.  Even when space is {\it
  not} naturally discrete, measurements are often made at a fairly
coarse grain (e.g., meters or tens of meters along a stream), or
associated with spatial quadrats for scat searches. And, of course, we
could approximate any continuous space with a discrete state-space,
and therefore apply multi-state models directly to any SCR problem.

\subsection{Modeling ‘manders on a stream network}

Here is a cool example: We catch salamander’s or fish along a stream
and only record stream segment instead of actual location – this is
motivated by Evan Grant’s work and also Lowe xyz??

each stream segment is individuals current state and its easy to use
either a Markov model or a home range model. ....

This is also a good open population example

\subsection{SCR as a Dynamic multi-state model}

Having a static state variable is not that interesting in the grand
scheme of multi-state models which most of the time consider a dynamic
state variable. Such models will arise frequently in spatial
capture-recapture settings. Let s denote the individual activity
center and suppose its state-space is discrete.  Now let $u[i,t]$ be
the patch in which individual $i$ was observed during sample $t$. Then
a simplistic movement model is that the successive movement outcomes
are $iid$
\[
u[i,t] \sim  dcat[ psi[s[i],] ]
\]

We can reformulate the basic SCR0 model as a dynamic multi-state model
as follows.  First lets grid up the state-space into “survey strata”
which we might define here has .5 unit squares so that the whole
state-space has 16*16 such squares. [actually do this so they are
centered on traps].We retain our assumption
\[
 s_{i} \sim Uniform(S)
\]
Secondly we define a movement model in which
\[
u[I,t] \sim dcat(pi)
\]
Where
\[
 pi_{k} = exp(-dist(x,s)/sigma2)/sum[that]
\]
This is the MAXENT distribution but also corresponds to Poisson with
mean $lam0*exp(-dist^2/sigma)$.  THIS IS CRUCIAL – THIS IS IMPT!!
 Makes it clear that encounter is the same as movement.

Now define
\[
 p|u[i,t] = p0*if(u[i,t] \in trap grid cell)
\]

Multi-state model with a “random movement” process.


We could easily extend this to a kind of Markovian movement model
where the probabilities depend on the previous state $u_{i,t-1}$ but
the simpler model of ``random'' movement satisfies our immediate needs.
 
So we see that SCR models are exactly a type of multi-state model when
the states are naturally discrete.  Another naturally discrete
state-space is ``nest sites''. Goncalo’s study and Florent’s
study. Schaub’s study on woopoos.


\section{Single-catch traps}
\label{poisson-mn.sec.singlecatch}

The classical animal trapping experiment is based on a physical trap
which captures a single animal and holds that individual until
subsequent molestation by a biologist. 
This type of observation model -- the ``single catch'' trap -- 
was the original situation considered by \citet{efford:2004}.

The single-catch model is basically a multinomial model but one in
which the number of available traps is reduced as each individual is
captured. As such, the constraints on the likelihood for each
individual are latent and shit is complicated beyond belief.
As a result, at the time of this writing, there has not been a formal
development of either likelihood  or Bayesian analysis of this model.

Nevertheless, it is not too difficult to describe the basic model
formally. In particular, there is a nice conditional structure resulting from a ``removal
process'' operating on the traps.  The first guy captured has the 
basic multinomial observation model:
\[
{\bf y}_{i} \sim Multinom({\bm \pi}_{i})
\]
whereas the 2nd guy captured has one cell removed:
\[
{\bf y}_{i} \sim Multinom({\bm \pi}_{i}(1-{\bf y}_{i})    )
\]
and so on.
So the {\bf order of capture} is relevant to the construction of these
multinomial cell probabilities. 
Thus the observations each have a multinomial model, but the
multinomial observations have a unique kind of conditional dependence
structure among them.

\subsection{Approximate Analysis}

To analyze the model here we consider using a misspecified model based
on either the Poisson or independent multinomial


How good of an approximation is the multi-catch model?

What about the Poisson model with a really low lambda?

Can we solve the big kahuna?

Use Sarah's data here.


\section{Trapping Webs}


\section{Acoustic Arrays}


\section{Summary and Outlook}

There are other types of encounter models.......

Efford adapts SCR models to acoustic detection devices.... a few words
on that here.....

There are models for which
only specific summary statistics are observable (Chandler and Royle
2011, etc..) which we cover in chapter XYZ.  We consider other models
for detection probability in some prior or later chapter. 







\chapter{MCMC details}
\label{chapt.mcmc}

\input{Ch6/Chapter6.tex}

\chapter{Goodness of Fit and stuff}
\label{chapt.gof}



\chapter{Covariate models}
\label{chapt.covariates}




\chapter{%State-space Covariates
%Modeling Spatial Variation in Density Using State-Space Covariates
Modeling Spatial Variation in Density
}
\markboth{Spatial Variation in Density}{}
\label{chapt.state-space}

\vspace{0.3cm}

\begin{comment} ok this is a minor tech mpoint for now: but this is introduced as a ``point process''
but what is being decribed here is a REALiZATION of a point process. Lets clarify this in the final
draft
\end{comment}
Underlying all spatial capture-recapture models is a point process
model that describes the distribution of individual activity
centers (${\bf s}$) within the state space ($\cal{S}$).
%, which is
%typically a two-dimensional polygon defining the study area.
Point process models are charcterized by $\mathcal{S}$ and by an
intensity parameter defined at each point in $\mathcal{S}$. If this
intensity is constant, the point process is said to be homogeneous,
and thus far we have focused our
attention on the homogeneous binomial point process whose realized
values are:
${\bf s}_i \sim \mbox{Unif}({\cal S}), i=1,2,\dots,N$, where $N$ is the
size of the population. This is a model of
``spatial-randomness''\footnote{The phrase ``complete
  spatial-randomness'' is reserved for the homogeneous Poisson point
  process}
because the intensity of the
activity centers is constant across the study area.
% and the activity
%centers are distributed independently of each other.

The spatial-randomness assumption is often viewed as restrictive
because ecological processes such as
territoriality and habitat selection can result in non-uniform
distributions of organisms. We have argued, however, that this
assumption is less restrictive than may be recognized because the
homogeneous point process actually allows for infinite
possible configurations of activity centers. Furthermore, given enough data,
the uniform prior will have very little influence on the estimated
locations of activity centers. Nonetheless, the homogeneous point
process model does not allow one to model population density using
covariates, which is a central objective of much ecological research.
For example, a homogeneous point process model
may result in a density surface map indicating that individuals were
more abundant in one habitat than another, but it does not do so
explicitly and so cannot be used to make predictions about
habitat-specific abundance in other regions. A more direct approach would be to replace
the homogeneous model with an inhomogeneous model in which the point process
intensity varies spatially.
%density using covariates as is done in generalized linear models (GLMs)
%\citep{mccullagh_nelder:1989}. % where a
%link function is used to connect the intensity parameter to the linear
%predictor.

In this chapter we present a method
for fitting inhomogeneous binomial point process models by modeling
the intensity parameter as a function of
covariates in much the same way as is done with generalized linear
models. The covariates we consider differ
from those covered in previous chapters, which were typically
attributes of the animal ({\it e.g.} sex or age) or the trap ({\it
  e.g.} baited or not) and were used to model movement or encounter
rate. In contrast, here we wish to
model covariates that are defined for all points in
$\cal{S}$, which we will refer to as
state-space covariates or density covariates. These may
include continuous covariates such as elevation, or discrete
covariates such as habitat type.

Inhomogeneous Poisson point process models were discussed in the original
formulation of SCR models \citep{efford:2004} and were described in
detail by \citet{borchers_efford:2008}. Our approach is
similar to that of \citet{borchers_efford:2008}, except that we use a binomial
rather than a Poisson model because the binomial model is
easily integrated into our MCMC algorithm.  %data augmentation scheme
%and is consistent
%with the objective of determining how a {\it fixed} number of activity
%centers are distributed with respect to covariates.
The method we use to accommodate inhomogeneous binomial point process
models %within our MCMC algorithm
is simple---we
replace the uniform prior with a prior describing the
distribution of the $N$ activity centers conditional on the
covariates. Development of this prior, which does not have a
standard form, is a central component of this chapter. First we
begin with a review of homogeneous point process models.


\section{Homogeneous point process revisited}

The homogeneous Poisson point process is \textit{the} model of ``complete
spatial randomness'' and is often used in ecology as a null model
to test for departures from randomness
\citep{diggle:2003,illian_etal:2008}. Given its central role in the
analysis of point processes, it is helpful to compare it with
the binomial model that we use in our SCR models. The sole parameter
of the homogeneous Poisson point process model is the
intensity parameter $\mu$ which describes the expected number
% start with \mu(s)???? Or, wait for inhomogeneous case?
of points in an infinitesimally small area. %Note that this intensity
%parameter is a single value, i.e. it does not vary spatially.
The intensity parameter can also be used to compute the expected number of points
in any region $B$ of the state-space $\cal{S}$. Specifically,
$\mathbb{E}[n(B)] = A(B)\mu$ where $A(B)$ is the area of region $B$.
This just says that
the expected number of points is the area of $B$
multiplied by the intensity parameter.
%This is one
%of the distinctions between the Poisson model and the binomial model,
%for which the counts $\{n(B_k)\}$ are not i.i.d., as we will explain
%shortly.

An important distinction between the Poisson point process and the
binomial point process is that $N$ is a random variable in the former
model but not in
the latter. In other words, the binomial point process conditions on $N$.
Here is some simple \R~code to illustrate this point:
\begin{verbatim}
mu <- 4                            # intensity
Np <- rpois(1, mu)                 # Np is random
PPP <- cbind(runif(Np), runif(Np)) # Poisson point process

Nb <- 4                            # Nb is fixed
BPP <- cbind(runif(Nb), runif(Nb)) # Binomial point process
\end{verbatim}
which generates realizations from Poisson and binomial point
processes in the unit square ($\mathcal{S} = [0,1]\times[0,1]$).
For both models, the $N$ points are
%independent of one another and
distributed uniformly
in $\mathcal{S}$, and they have the same intensity parameter,
$\mu=4$. However, in the binomial case the intensity parameter is
defined different, being a function of $N$ and the area of the
state-space, $\mu = N/A(\mathcal{S})$.

Another distinction between the two models is that if we divide the
state-space into $K$ disjunct regions, the number of points in each
region $\{ n(B_k): k=1,\dots,K \}$ are
independent and identically distributed (i.i.d.) under the Poisson model,
but some dependence exists under the binomial model.
%In the Poisson case
%we have $\n(B_k) \sim \text{Pois}(A(B_k)\mu)$, and if the points were independent
%$n(B_k) \sim Bin(N, p(B_k))$ for the binomial,
%where $p(B_k)$ is simply the proportion of the state-space in region
%$B_k$.
Fig.~\ref{state-space.fig.homo} illustrates this point.
The depicted state-space is the unit square, and thus the probability of a
point falling in each of the 25 disjunct regions is $p(B_k) = 1/25$ and
the expected counts are $\mathbb{E}(n(B_k)) = Np_k$.
%In
%the figure $N=50$, and consequently we would expect 2 points per pixel, which
%happens to be the empirical mean in this instance.
However, these counts are not
independent realizations from a binomial distribution since $\sum_{k=1}^K
n(B_k) = N$. Instead, the model for the entire vector
is ${n(B_1), n(B_2), \dots, n(B_k)} \sim \mbox{Multin}(N, \{p(B_1), p(B_2), \dots,
p(B_K) \})$ \citep{illian_etal:2008}.
%\begin{verbatim}
%n.Bk <- rmultinom(1, size=50, prob=rep(1/25, 25))
%matrix(n.Bk, 5, 5)
%\end{verbatim}
The dependence among counts has virtually
no practical consequence when the number of pixels is large. For
example, if there are 100 pixels, the number of points in one pixels
carries very little information about the expected number of points in another
pixel. However, if there are only 2 pixels, then clearly the number of
points in one pixel allows one to determine how many points will occur in the
remaining pixel.
%To gain familiarity with the multinomial distribution
%and the discrete representation of space, use the \verb+rmultinom+
%function in \R~to simulate counts similar to those shown in
%Fig.~\ref{state-space.fig.homo}, for example using commands
%such as:
%\begin{verbatim}
%n.Bk <- rmultinom(1, size=50, prob=rep(1/25, 25))
%matrix(n.Bk, 5, 5)
%\end{verbatim}


\begin{figure}[ht!]
\centering
\includegraphics[width=5in,height=2.5in]{Ch11/figs/homoPlots}
\label{state-space.fig.homo}
\caption{Homogeneous binomial point process with $N$=50 points
  represented in continuous and discrete space.}
\end{figure}


The discrete space representation of the binomial point process is of
practical importance when fitting SCR models because spatial covariates
are almost always represented in a discrete-space format called
``rasters'' in GIS-speak. In such cases, we often need to change our
definition of the prior for an activity center from ${\bf s}_i \sim
\mbox{Unif}(\cal{S})$ to ${\bf s}_i \sim \mbox{Multin}(1, \mathbf{\pi})$. In the
latter case, the activity center is simply defined as an integer
representing pixel ``id''.
%Note also that the multinomial distribution
%with an index of 1 (\emph{i.e.} \verb+size=1+ in \verb+rmultinom+)
%is referred to as the categorical distribution,
%which we will frequently use in the \verb+BUGS+ language.



\section{Inhomogeneous binomial point process}

\hl{Check for x instead of s throughout}

As with the homogeneous model, the inhomogeneous binomial point process
model is developed conditional on $N$. The primary distinction is that
the uniform distribution is replaced with another distribution
allowing for the intensity parameter to vary spatially. To arrive at
this new distribution, replace the scalar intensity parameter $\mu$
with the function $\mu(s, {\bm \beta})$, where $\bm \beta$ is a
vector of coefficients describing the effects of
spatially-referenced covariates on the point process intensity. In
what follows, we will often abbreviate the intensity function as $\mu(s)$,
dropping the vector of coefficients for readability. Since an intensity must be strictly
positive, and because the logarithm is the canonical link function of the Poisson
generalized linear model, it is natural to model $\mu(s, \beta)$ as
\[
\log(\mu(s, \beta)) = \beta_0 + \sum_{j=1}^J \beta_j v_j(s), \quad  x \in \cal{S}
\]
where $\beta_j$ is the regression coefficient for covariate
$v_j(s)$. To be clear, $v(s)$ is the value of any covariate, such as
habitat type or elevation, at location $x$ and it assumed to be
defined at all locations in the state-space.
This equation should look
familiar because it is the standard linear predictor used in log-linear
GLMs. One caveat is that the intercept $\beta_0$ is not a
unique parameter to be estimated.
%Note, however, that we have not included
%an intercept. The reason for this is that it would be confounded with
%$N$ (see Chapt. \ref{chapt.hscr}).
The reason for this is that %This should be intuitive since
$\beta_0$ represents population density at the location $x$ when
%, the expected value of $N$ in some infinitesimally
%small area when
the other $\beta$'s equal 0. However, we already
have a parameter in the model for expected abundance, namely $\mathbb{E}[D] =
N/A(\mathcal{S}) =  \psi M / A(\mathcal{S})$\footnote{Remember, $M$ is the size of the augmented population, and
$\psi$ is the probability that a member of $M$ is an actual
constituent of the population (Chapt. ~\ref{chapt.scr0}).}. Thus, in
practice, we can either remove $\beta0$ and model a value proportional
to the intensity, or we can define $\beta_0=\log(N/A(\mathcal{S}))$ and model the
intensity directly.
%an intercept would be
%redundant, and without it we are still able to achieve our goal of
%describing the distribution of $N$ activity centers as a function of
%spatial covariates. One caveat is that if we wish to make predictions
%to unsampled regions, it is useful to include the intercept $\beta_0 =
%\log(N)$.

Now that we have a model of the intensity parameter $\mu(s)$,
we need to develop the associated probability density function
$[\bf s]$ to use
in place of the uniform prior. Remembering that
the integral of a pdf must be unity, we can create the pdf
$[\bf s]$ by dividing
$\mu(s)$ by a normalizing constant, which in this case is the integral
of $\mu(s)$ evaluated over the entire state-space.
The probability density function is therefore
\begin{equation}
[\mathbf{s}] = \frac{\mu(s, \beta)}{\int_{x \in \mathcal{S}} \mu(s, \beta)\, \mathrm{d}s}
\label{eq.pdf.ipp}
\end{equation}
Substituting this distribution for the
uniform prior allows us to fit inhomogeneous binomial point process
models to spatial capture-recapture data. We can also use this
distribution to obtain the expected number of individuals in any given
region. Specifically, the proportion of $N$ expected to occur in any
region $B$ %when heterogeneity in density is present
is $p(B) = \int_B
f(s, \beta)\, \mathrm{d}x$. These are
also the multinomial cell probabilities if the regions are
disjoint and compose the entire state-space. We provide an example in
the next section, and in Fig.\ref{state-space.fig.hetero}.

As a practical matter, note that the integral in the
denominator of $f(s, \beta)$ is evaluated over space, and since we always regard
space as two-dimensional, this is a two-dimensional integral that can
be approximated using the methods discussed in
Chapter~\ref{chapt.poisson-mn}, which include
Monte Carlo integration and Gaussian quadrature. Alternatively, if
our state-space covariates are in raster format, \emph{i.e} they are
in discrete space, the integral can be replaced with a sum over
all pixels,
\begin{equation}
f(s, \beta) = \frac{\mu(s, \beta)}{\sum_{x \in \mathcal{S}} \mu(s, \beta)\, \mathrm{d}x}
\label{eq.pdf.dipp.d}
\end{equation}
which is much more efficient computationally.

Although the discrete space approach is standard practice, it is
technically unjustified because covariate values must be known for all
points in space. This same problem is present anytime that we have a
sample of the spatial covariates, rather than a function defining
their value for all points in space. In such cases, it may be necessary to
interpolate the values of the covariates for points in space where
they were not measured. One option would be to use a Kriging
interpolator, as demonstrated by \citet{rathbun:1996}. Another option
is to sample the spatial covariates using probabalistic sampling
methods, which allow for design-based estimators of their values for
the entire study area \citep{rathbun_etal:2007}. Either option could
be implemented as part of the MCMC algorithm, but even though such
approaches are technically necessary, we do not demonstrate them here
because it seems likely that they will be inconsequential in most
cases where the raster data are of high resolution, such that the loss
of information is negligible when going from continuous space to
discrete space.

We now have all the tools needed to fit inhomogeneous point process
(IPP) models. If we refer to the distribution for the
inhomogeneous point process as ``IPP'', we can write a
hierarchical description of a SCR model with a Poisson encounter process and
a half-normal detection function as
\begin{gather*}
w_i \sim \mbox{Bern}(\psi) \\
{\bf s_i} \sim \mbox{IPP}(\mu(s,\beta)) \\
\lambda_{ij} = \lambda_0 \exp(-\|{\bf s_i} - {\bf x_{j}}\|^2/(2\sigma^2)) \\
y_{ij} \sim \mbox{Poisson}(\lambda_{ij} w_i)
\end{gather*}
The use of $\mbox{IPP}(\mu(s, \beta))$ instead of
$\mbox{Unif}(\cal{S})$ is the only difference between a homogeneous
point process model and an inhomogeneous point process model, and the
two are equivalent when $\beta=0$.

\begin{comment}
The IPP for the activity centers
results in another IPP for the observation process, $\lambda(s)$, the
expected number of captures for a trap
at point. As was true for the homogeneous model, this
intensity function is a product of the point process intensity
and the encounter rate function, $\lambda(s) = \mu(s, {\bm \beta})
\lambda_{ij}$.
\end{comment}

In the next sections we walk through a few examples, building up from
the simplest case where we actually observe the activity centers as
though they were data. In the second example, we fit our new model to simulated
data in which density is a function of a single continuous
covariate. \hl{To build upon the developments in the previous chapter, we
further consider the plausible case where a state-space covariate is also a
covariate of ecological distance.} A small simulation study indicates
that both effects can be estimated. A fourth example shows an analysis in discrete space using
both \secr~\citep{efford:2011} and \jags~\citep{plummer:2003}. In the
fifth and final example, we model the intensity of
activity centers for a real dataset collected on jaguars
(\emph{Panthera onca}) in Argentina.

\section{Observed Point Processes}

In SCR models, the point process is not directly observed, but in
other contexts it is. Examples include the locations of disease
outbreaks, the locations of trees in a forest, or the locations of
radio-tracked animals. Indeed Eq.~\ref{eq.pdf.ipp} has been used
extensively in the radio-telemetry literature to model so-called
``resource selection functions'' \citep{manly_etal:2002,lele_keim:2006}.
When the point locations are directly observed,
estimating the parameters $\bf \beta$ is straight-forward as
demonstrated in the following example. This example also illustrates
the fundamental process that we will later embed in our MCMC algorithm
used to fit SCR models with IPP.

Suppose we knew the locations of 100 animals' activity
centers, perhaps as the result of an extensive telemetry study. To
estimate the intensity surface $\mu(s, \beta)$ underlying these
points, we need to derive the likelihood for our data under this
model. Given the probability density function $f(s, \beta)$
(Eq.~\ref{eq.pdf.ipp}) and assuming that the points are
mutually independent of one another,
the likelihood is given by the product
of $R$ such terms, where $R=100$ is the sample size in our
hypothetical example,
\emph{i.e.} the observed number of activity centers.
\[
\mathcal{L}({\bf \beta} | {\bf x}_i, \beta) = \prod_{i=1}^R f(s_i)
\]
Having defined the likelihood we could choose a prior distribution for
$\beta$ and obtain the posterior distribution of
$\bf \beta$ using Bayesian methods, or we can find the maximum likelihood
estimates (MLEs) using standard numerical methods as is demonstrated
below.

First, we simulate some data. Simulating data under an inhomogeneous point process model is often
accomplished using indirect methods such as rejection
sampling. Rejection sampling proceeds by
simulating data from a standard distribution and then accepting or
rejecting each sample using probabilities defined by the distribution
of interest. For more information, readers should consult an
accessible text such as \citet{robert_casella:2010}. In our example, we
simulate from a uniform distribution and then accept or reject using
the (scaled) probability density function $f(s, \beta)$. Note that we first define a
spatial covariate (elevation) that is a simple function of the spatial
coordinates increasing from the southwest to the northeast of our
state-space.\footnote{Such functional forms of
covariates are rarely available. Instead,  continuous spatial
covariates are more often measured on a discrete grid.}

The following \R~commands demonstrate the use of rejection sampling to
simulate an inhomogeneous point process for the covariate depicted in
Fig.~\ref{state-space.fig.hetero}. The code uses the \verb+cuhre+ function in
the {\tt R2Cuba} package to integrate the intensity function over
space \citep{hahn_etal:2011}. An alternative would be to evaluate the
integral on a fine grid of points as we have done in previous
chapters, but it is useful to gain familiarity with more efficient
integration functions in \R.

\begin{small}
\begin{verbatim}
# spatial covariate (with mean 0)
elev.fn <- function(s) x[1]+x[2]-1
# intensity function
mu <- function(s, beta) exp(beta*elev.fn(s=x))

# Simulate IPP using rejection sampling
set.seed(300225)
N <- 100
count <- 1
s <- matrix(NA, N, 2)
beta <- 2 # parameter of interest
elev.fn <- function(s) x[1]+x[2]-1
# Intensity function, mu(s,beta)
mu <- function(s, beta) exp(beta*elev.fn(x=x))
# 2-dimensional integration over space
int.mu <- R2Cuba:::cuhre(2, 1, mu, beta=beta)$value
elev.min <- elev.fn(c(0,0)) #elev.fn(cbind(0,0))
elev.max <- elev.fn(c(1,1)) #elev.fn(cbind(1,1))
Q <- max(c(exp(beta*elev.min) / int.mu,   #2d(beta),
           exp(beta*elev.max) / int.mu))   #2d(beta)))
while(count <= 100) {
  x.c <- runif(1, 0, 1); y.c <- runif(1, 0, 1)
  s.cand <- c(x.c,y.c)
  pr <- exp(beta*elev.fn(s.cand)) / int.mu #2d(beta)
  if(runif(1) < pr/Q) {
    s[count,] <- s.cand
    count <- count+1
    }
  }
\end{verbatim}
\end{small}


\begin{figure}[ht]
\centering
\includegraphics[width=5in,height=2.5in]{Ch11/figs/heteroPlots}
\label{state-space.fig.hetero}
\caption{An example of a spatial covariate, say elevation, and a
  realization of a inhomogeneous binomial point process with $N$=100
  and $\mu(s) = exp(\beta \mbox{elev}(s))$ where $\beta=2$.}
\end{figure}

The simulated data are shown in Fig~\ref{state-space.fig.hetero}. High elevations
are represented by light green and low elevations by dark green. The
activity centers of 100 animals are shown as
points, and it is clear that these simulated animals prefer the high
elevations.  %Perhaps they are mountain goats.
The underlying model describing this preference is
$\log(\mu(s)) = \exp(\beta \times elev(s))$
where $\beta=2$ is the parameter to be estimated.

Given these points, we will now estimate $\beta$ by minimizing the
negative-log-likelihood using \verb+R+'s \verb+optim+ function.

\begin{small}
\begin{verbatim}
# Negative log-likelihood
nll <- function(beta) {
    int.mu <- R2Cuba:::cuhre(2, 1, mu, beta=beta)$value
    -sum(beta*elev.fn(s) - log(int.mu))
}
starting.value <- 0
fm <- optim(starting.value, nll, method="Brent",
            lower=-5, upper=5, hessian=TRUE)
c(Est=fm$par, SE=sqrt(1/fm$hessian)) # estimates and SEs
\end{verbatim}
\end{small}


Maximizing the likelihood took a small fraction of a second, and we
obtained an estimate of $\hat{\beta}=1.99$. We could plug
this estimate into our linear model at each point in the state-space to
obtain the MLE for the intensity surface.

This example demonstrates
that if we had the data we wish we had, {\it i.e.} if we knew the
coordinates of the activity centers $\bf s$, we could easily estimate the
parameters governing the underlying point process. Unfortunately, in
SCR models, the activity centers cannot be directly observed, but
spatial re-captures provide us with the information needed to
estimate these latent parameters.

\section{Fitting inhomogeneous point process SCR models}

\subsection{Continuous space}

One of the nice things about hierarchical models is that they allow us
to break a problem up into a series of simple conditional
sub-models. Thus,
we can simply add the methods described above into our existing MCMC
algorithm to simulate the posterior distributions of $\beta$ conditional on the
simulated values of $\mathbf{s}$. To demonstrate, we will continue with
the previous example. Specifically, we will overlay a grid of
traps on the map shown in Fig.~\ref{state-space.fig.hetero}. We will then
simulate capture histories conditional upon the activity
centers. Then, we will attempt to estimate the activity center
locations as though we did not know where they were, as is the case in
real applications.

The following \R~code simulates encounter histories under a
Poisson observation model (see Chapt. \ref{chapt.poisson-mn}), which could be appropriate in camera
trapping studies or when using other methods in which animals could
be detected multiple times at a trap during a single occasion.

\begin{small}
\begin{verbatim}
# Create trap locations
xsp <- seq(-0.8, 0.8, by=0.2)
len <- length(xsp)
X <- cbind(rep(xsp, each=len), rep(xsp, times=len))

# Simulate capture histories, and augment the data
ntraps <- nrow(X)
T <- 5
y <- array(NA, c(N, ntraps, T))

nz <- 50 # augmentation
M <- nz+nrow(y)
yz <- array(0, c(M, ntraps, T))

sigma <- 0.1  # half-normal scale parameter
lam0 <- 0.5   # basal encounter rate
lam <- matrix(NA, N, ntraps)

set.seed(5588)
for(i in 1:N) {
    for(j in 1:ntraps) {
        distSq <- (s[i,1]-X[j,1])^2 + (s[i,2] - X[j,2])^2
        lam[i,j] <- exp(-distSq/(2*sigma^2)) * lam0
        y[i,j,] <- rpois(T, lam[i,j])
    }
}
yz[1:nrow(y),,] <- y # Fill
\end{verbatim}
\end{small}

Now that we have a simulated capture-recapture dataset $y$, and we have
augmented it to create the new data object $yz$, we are ready to
begin sampling from the posteriors. A commented Gibbs sampler written
in \R~is available in the accompanying \R~package \scrbook~(see
?scrIPP).
\begin{comment} see Ch 7 MCMC for SCR and cite some section of that \end{comment}
% There are two small parts of the
% \R~code that distinguish it from previous code we have shown to
% fit homogeneous point processes. First, we need to update the parameter
% ${\bf \beta}$ conditional on all other parameters in the model. The code to
% do so is: %\begin{comment} need cite to Ch 2 or 7 on MCMC for this \end{comment}
% \begin{small}
% \begin{verbatim}
% # Denominator of f(x, beta). Integral of mu(x, beta) over space
% D1 <- cuhre(2, 1, mu, lower=c(xlims[1], ylims[1]),
%             upper=c(xlims[2], ylims[2]), beta=beta1)$value
% # Compute the denominator again using a proposed beta1
% beta1.cand <- rnorm(1, beta1, tune[3])
% D1.cand <- cuhre(2, 1, mu, lower=c(xlims[1], ylims[1]),
%                  upper=c(xlims[2], ylims[2]), beta=beta1.cand)$value
% # Compute log(f(x))
% ll.beta1 <- sum(  beta1*elev.fn.v(S) - log(D1) )
% ll.beta1.cand <- sum( beta1.cand*elev.fn.v(S) - log(D1.cand) )
% if(runif(1) < exp(ll.beta1.cand - ll.beta1) )  {
%      beta1<-beta1.cand
% }
% \end{verbatim}
% \end{small}
% Next, we need to put the new prior on the activity centers:
% \begin{small}
% \begin{verbatim}
% # Compute the prior for s_i and a candidate. denominator is constant
% prior.S <- beta1*elev(S[i,1], S[i,2]) # - log(D1)
% prior.S.cand <- beta1*elev(Scand[1] + Scand[2]) # - log(D1)
% if(runif(1)< exp((ll.S.cand+prior.S.cand) - (ll.S+prior.S))) {
%     S[i,] <- Scand
%     lam <- lam.cand
%     D[i,] <- dtmp
%     }
% \end{verbatim}
% \end{small}
We can apply this modified sampler to our data using the
following \R~commands:
\begin{small}
\begin{verbatim}
set.seed(3434)
fm1 <- scrIPP(yz, X, M, 6000, xlims=c(0,1), ylims=c(0,1),
            tune=c(0.003, 0.08, 0.3, 0.07) )
plot(mcmc(fm1$out))
rejectionRate(mcmc(fm1$out))
\end{verbatim}
\end{small}
We obtain posterior distributions that are summarized in
Table~\ref{ch9.tab.simIPP}.
%Mixing is good, and as usual,
%life is very nice when we are working with simulated data.

\begin{table}[b]
\centering
\caption{Posterior summaries from inhomogeneous point process model
  fitted to simulated data. Space was treated as continuous.}
\begin{tabular}{lrrrrr}
\hline
& Mean & SD & 2.5\% & 50\% & 97.5\% \\
\hline
 $\sigma =0.10$ &   0.1026 &   0.0048 &   0.0935 &   0.1025 &   0.1123 \\
 $\lambda_0=0.50$ &   0.4419 &   0.0493 &   0.3496 &   0.4400 &   0.5390 \\
 $\psi =0.66$ &   0.6826 &   0.0554 &   0.5762 &   0.6820 &   0.7923 \\
 $\beta =2.00$ &   2.1601 &   0.3390 &   1.5193 &   2.1583 &   2.8043 \\
 $N =100$ & 102.7696 &   6.2689 &  92.0000 & 102.0000 & 117.0000 \\
\hline
\end{tabular}
\label{ch9.tab.simIPP}
\end{table}


Fitting continuous space IPP models is somewhat
difficult in \bugs~because our prior ``IPP'' is not one of the
available distributions that come with the software. It is
possible to add new distributions in \bugs, but it is somewhat
cumbersome.  \secr~allows
users to fit continuous space IPPs using polynomials of the x- and y-
coordinates, but it does not accept truly continuous covariates that
are functions of space. However, these
are not really important limitations because discrete
space versions of the IPP model are straight-forward, and virtually all spatial
covariates are, or can be, defined as such.


\subsection{Discrete space}

To fit IPPs using covariates in discrete space, \emph{i.e.} in raster
format, we follow the same steps
as outlined in Chapter~\ref{chapt.poisson-mn}---we define ${\bf s}_i$ as
pixel ID, and we use the categorical distribution as a prior. A good
example is found in \citep{mollet_etal:2012}. Here we present
an analysis of the simulated data shown in the %right panel of
Fig.~\ref{state-space.fig.hetero}. The spatial covariate, let's call it
elevation again, was simulated
using using the code shown on the help page
\verb+ch9simData+ in \scrbook. The points are the number of
activity centers in each pixel, generated from a single realization of
the inhomogeneous point process model with intensity
$\mu(x) = 2 \times \mbox{elev}(s)$.
\begin{figure}[ht]
\centering
\includegraphics[width=3in,height=3in]{Ch11/figs/discrete}
\label{ch9.fig.discrete}
\caption{Simulated activity centers in discrete space. The spatial
  covariate, elevation, is highest in the lighter areas. Density of
  activity centers (circles) increases with elevation. A single
  activity center is shown as a small circle, and larger circles
  represent two activity centers in a pixel. Trap locations
  are shown as crosses.}
\end{figure}

The \bugs~code to fit an IPP model to these data is shown in
panel~\ref{ch9.panel1}.The vector \verb+probs[]+ is the prior
probability defined
by~\ref{eq.pdf.ipp.d}, which is the probability that an individual's
activity center is located at pixel $x$. \verb+Sgrid+ is the
matrix of coordinates for each pixel.

%\begin{panel}[h!]
%\centering
%\rule[0.15in]{\textwidth}{.03in}
\begin{small}
\begin{verbatim}
model{
sigma ~ dunif(0, 1)
lam0 ~ dunif(0, 5)
beta ~ dnorm(0,0.1)
psi ~ dbeta(1,1)
for(x in 1:nPix) {
  theta[x] <- exp(beta*elevation[j])
  probs[x] <- theta[j]/sum(theta[])
}
for(i in 1:M) {
  w[i] ~ dbern(psi)
  s[i] ~ dcat(probs[])
  x0g[i] <- Sgrid[s[i],1]
  y0g[i] <- Sgrid[s[i],2]
  for(j in 1:ntraps) {
    dist[i,j] <- sqrt(pow(x0g[i]-grid[j,1],2) +
                      pow(y0g[i]-grid[j,2],2))
    lambda[i,j] <- lam0*exp(-dist[i,j]*dist[i,j] /
                            (2*sigma*sigma)) * w[i]
    y[i,j] ~ dpois(lambda[i,j])
    }
  }
N <- sum(w[])
Density <- N/1 # unit square
}
\end{verbatim}
\end{small}
%\rule[0.15in]{\textwidth}{.03in}
%\caption{\bugs~code for fitting inhomogeneous point process model in
%  discrete space.}
%\label{ch9.panel1}
%\end{panel}

This model can also be fit in \secr, which refers
to the raster data as a ``habitat mask''. \R~code to
fit the models using \secr~and \jags~is available in \scrbook---see
\verb#help(ch9secrYjags)#. Results of the
comparison are shown in Table \ref{ch9:tab:secrYjags} and are
very similar as expected.
\begin{comment}
\hl{ANDY, is there any point in discussing
  the slight differences?}
  YES: If we can explain it. Could it be MC error alone?
Otherwise I guess attributing it to differences between MLE and BAyes is ok. That seems like
a reasonable thing.
\end{comment}
\begin{table}[h!]
\centering
\caption{Comparison of \secr~and \jags~results. Point estimates from
  the Bayesian analysis are posterior means. Intervals are lower and
  upper 95\% CIs.}
\begin{tabular}{llrrrr}
\hline
Software & Parameter & Estimate & SD & lower & upper \\
\hline
 secr & $N=50$ & 49.2803 & 5.7535 & 41.0087 & 64.3879 \\
      & $\beta=2$ &  2.1772 & 0.5628 &  1.0741 &  3.2804 \\
      & $\lambda_0=0.8$ &  0.9203 & 0.0764 &  0.7824 &  1.0825 \\
      & $\sigma=0.1$ &  0.0990 & 0.0038 &  0.0918 &  0.1068 \\
\hline
 JAGS & $N=50$ & 48.2072 & 5.4053 & 39.0000 & 60.0000 \\
      & $\beta=2$ &  2.1026 & 0.5323 &  1.0889 &  3.1506 \\
      & $\lambda_0=0.8$ &  0.9328 & 0.0766 &  0.7898 &  1.0921 \\
      & $\sigma=0.1$ &  0.1004 & 0.0041 &  0.0929 &  0.1089 \\
\hline
\end{tabular}
\label{ch9:tab:secrYjags}
\end{table}


\section{Ecological distance and state-space covariates}

Habitat characteristics that affect population
density could also affect home range size and movement behavior. For
example, a
species that occurs in high density in a forest may be reluctant to
venture from a forest patch into an adjacent field. Thus, even if a
trap placed in a field is located very close to an animal's activity
center, the probability of capture may be very low. In this case
forest cover is a covariate of both density and encounter probability,
and we could model it as such by combining the methods described in
this chapter and in Chapter~\ref{chapt.ecoldist}. To demonstrate, we
continue with our analysis of the data shown in
Fig~\ref{state-space.fig.hetero}. Once again, we suppose that density
increases with elevation, but this time, we also make the
assumption that home range size decreases as density increases. This
commonly-observed phenomenon can be explained by numerous factors such
as intra-specific competition \citep{sillett_etal:2004} or optimal
foraging behavior \citep{tufto_etal:1996,said_servanty:2005}. To model
this effect, we
introduce the parameter $\theta$, which determines the ``cost'' of
moving between pixels. If $\theta=0$, then the animal perceives
distance as Euclidean. If $\theta>0$, then least-cost distance (LCD)
is greater than than Euclidean distance (ED). In most cases, we would
not expect,
or should not even consider the possibility of $\theta<0$ because this
implies that LCD$<$ED, which would mean that an animal could view
1000km as 1m. In addition to the fact that this is not biologically
justifiable, it also suggests that the area of the state-space could
be infinitely large. Thus, one may want to enforce the constraint that
$\theta$ is strictly $\geq 0$. See Chapter~\ref{chapt.ecoldist} for
more details.

One may wonder if it is possible to estimate both $\beta$
and $\theta$ using standard SCR data. Currently, it is not possible to
model least-cost distance using \jags~or \secr, so we wrote our own
function, \verb+scrDED+, to fit the model using maximum likelihood. An
example analysis is provided on the help page for the function in our
\R~package \scrbook. We briefly note here that the function requires
the capture history data, the trap locations, and the raster data
formatted using the {\tt raster} package
\citep{hijmans_vanetten:2012}. The linear model for the
intensity parameter $\mu(s, \beta)$ and the least-cost distance
function $lcd(\theta)$ are specified using \R's formula interface. A
simple function call is
\begin{verbatim}
fm <- scrDED(y, traplocs=X, den.formula=~elev, dist.formula=~elev,
             rasters=elev.raster)
\end{verbatim}
To assess the possibility of estimating both $\beta$ and $\theta$, we
conducted a small simulation study, generating 500 datasets from the
model with both parameters set to 1, which corresponds to the
conditions described above. Rather incredibly, we see that it is
possible to estimate both parameters with high accuracy
(Fig~\ref{ch9.fig.sim}).

\begin{figure}[ht]
\centering
\includegraphics[width=4in,height=2in]{Ch11/figs/scrDEDsim}
\caption{Histograms of parameter estimates from 500 simulations under
  the model in which both density and ecological distance are affected
by the same covariate, elevation. The vertical lines indicate the
data-generating value.}
\label{ch9.fig.sim}
\end{figure}



\section{The jaguar data}

Estimating density of large felines has been a priority for many
conservation organizations, but no robust methodologies existed before
the advent of SCR. Distance sampling is not feasible for such rare and
cryptic species, and traditional capture-recapture methods yield
estimates that are highly sensitive to the subjective choice of the
effective survey area. In this example, we
demonstrate how readily density can be estimated for a
globally imperiled species using SCR. Furthermore, we show how
inhomogeneous point process models can be used to test important
hypotheses regarding the ecological factors affecting density.

In this example, we make use of a single year of data from an 8-year
camera-trapping study of jaguars in Argentina,
along the borders with Brazil and Paraguay. The data come from 46
camera stations, each consisting of a pair of cameras placed along
roads or trails. Forty-five detections of 16 jaguars (8 males and 8
females) were made over a 95-day sampling period. The mean number of
sampling days at each camera station was 48.2.

Estimating density is a central objective of this study because
ultimately, an estimate of the total population size for the entire
study area is needed, which can only be obtained by extrapolation of
density estimates. A second, and related, objective was to assess
the influence of poaching on jaguar density. Although jaguars
themselves are occasionally killed by poachers, the larger concern is
the influence of poaching on prey species. To protect jaguars and
related species, protected areas have
been established and three levels of protection are
recognized in the study region as depicted in Fig.~\ref{ch9.fig.jaguarCts}.

\begin{figure}[ht]
\centering
\includegraphics[width=3in,height=3in]{Ch11/figs/jaguarCountMap}
\label{ch9.fig.jaguarCts}
\caption{Jaguar detections at 46 camera trap stations. The three levels of
  protection status are no protection (beige), some protection (light
  green), and national park (dark green). Non-habitat is shown in gray
  and represents large soybean monocultures. }
\end{figure}

To assess the influence of poaching on jaguar density, we treated
protection status as an ordinal variable with 3 levels: no protection,
some protection, and high protection (national parks). Clearly these
are ordered, and our
hypothesis is that poaching pressure should decrease and jaguar
density should increase with the level of
protection. Thus, $\beta$ in this example is a ``slope''
parameter describing the degree to which protection status affects
jaguar density. We also hypothesized that males and females could have
different home range sizes and that the sex ratio may not be
1:1. Furthermore, we restricted the state-space to exclude the large
soybean monocultures surrounding the study area, and we only
considered
area south of the Iguazu River, which runs along the northern border
of the park shown in dark green in
Fig.~\ref{ch9.fig.jaguarCts}. Rather than restricting the
state-space, we could have modeled the permeability of the river using
the methods described in the previous section and in
Chapter~\ref{chapt.ecoldist}; however, no sampling was conducted on
the northern side of the river, and ancillary data indicates that
jaguars very rarely forge the waterway. \R~code to fit the model is
available in \scrbook  on the help page \verb+jaguarDataCh9+. Parameter
estimates are shown in Table\ref{ch9.tab.jagposts}.
\begin{table}
\centering
\caption{Summaries of posterior distributions from the model of jaguar
  density. $\sigma_f$ and $\sigma_m$ are the scale parameters of
  the half-normal detection function for females and males
  respectively. $\rho$ is the
  sex-ratio. $\lambda_0$ is base-line encounter rate. $\beta$ is the
  effect of protection on jaguar density. D is the overall density
  estimate. D1, D2, and D3 are the density estimates
  (jaguars/100km$^2$) for the three levels of protection. }
\begin{tabular}{lrrrrr}
\hline
& Mean & SD & 2.5\% & 50\% & 97.5\% \\
\hline
 $\sigma_f$ &  7361.731 &  1907.566 &  4899.740 &  7002.770 & 12083.110 \\
 $\sigma_m$ &  8177.068 &  1545.717 &  5916.151 &  7955.788 & 11842.486 \\
 $\rho$ &     0.516 &     0.118 &     0.286 &     0.516 &     0.741 \\
 $\lambda_0$ &     0.007 &     0.002 &     0.003 &     0.007 &     0.012 \\
 $\beta$ &     4.405 &     1.443 &     2.553 &     4.143 &     7.775 \\
 D &     0.533 &     0.708 &     0.000 &     0.000 &     0.072 \\
 D1 &     0.132 &     0.010 &     0.095 &     0.095 &     0.616 \\
 D2 &     1.415 &     0.050 &     0.214 &     0.531 &     1.503 \\
 D3 &     3.516 &     0.000 &     0.292 &     3.105 &     4.220 \\
\hline
\end{tabular}
\label{ch9.tab.jagposts}
\end{table}

Our results
indicate that efforts to protect jaguars by reducing poaching are
working. Density was $>$26 times higher in the national park than in the
unprotected area. Fig.~\ref{ch9:fig:Dsurface} shows the estimated
density surface.

\begin{figure}[ht]
\centering
\includegraphics[width=3in,height=3in]{Ch11/figs/Dsurface34}
\label{ch9:fig:Dsurface}
\caption{Estimated density surface for the jaguar dataset}
\end{figure}


We note that there is room for improvement in our analysis. The
political boundaries used to demarcate protected areas are not as
concrete as we might like. In reality poaching pressure is likely to
be higher near remote park boundaries than in well-guarded park
interiors. One option
for addressing this would be to use a continuous measure of poaching
pressure such as distance from the nearest town, or some other
accessibility metric. It would also be interesting to model density
separately for each sex. Many of the detections outside of the park
were of males, and thus it is possible that the sexes use habitat
differently. Developing models for these two hypotheses could be
readily accomplished using slight modifications of the code found in
the \R~package \scrbook.



\section{Summary}

When state-space covariates are available,
density can be modeled by replacing the uniform prior on the activity
centers with a
prior based on a normalized log-linear function of covariates. This
distribution has been widely used in ecology to model point processes
as well as resource selection probability functions
\citep{manly_etal:2002,lele_keim:2006}. In the SCR
context, use of this new prior results in
a model for the inhomogeneous point process describing the
location of activity centers, which can be used to test hypotheses
about spatial variation in density. In
rare cases, these covariates are truly continuous in the sense that
they are defined as a function of space. More often, covariates are
represented as rasters, which simplifies the analysis. Fitting these
models can be accomplished using \bugs, \secr, or the custom \R~code
presented in this chapter and found in the package \scrbook.
%However,
%at the time this book was written, \scrbook is only software available
%for fitting models with covariates of both density and ecological
%distance.

All the examples in this section included a single state-space
covariate, but this was for simplicity only. Including multiple
covariates poses no additional challenges. Similarly, additional model
structure such sex-specific encounter rate parameters or behavioral
responses can be accommodated. Even more remarkable is the ability to
consider covariates that affect both density and ecological
distance. The ramifications of this are enormous for applied
ecological research and conservation efforts because, for instance,
researchers can use capture-recapture data to identify areas where
density is high, and to model important quantities such as landscape
connectivity \citep{royle_etal:2012ecol}. Addressing such questions
is simply not possible using standard, non-spatial capture-recapture
methods. Accomplishing these goals will of course require more data
than is needed to estimate the parameters of a basic SCR model.



\chapter{Inhomogeneous Point Process}
\label{chapt.ipp}

\chapter{Open models}
\label{chapt.open}


\bibliography{AndyRefs_alphabetized}


\markboth{Index}{Index}

\printindex

\documentclass{book}

\usepackage{elsst-book}
\usepackage{float}
\usepackage{amsmath}
\usepackage{amsfonts}
\usepackage{graphicx}
\usepackage{lineno}
\usepackage{natbib}
\usepackage{hyperref}
\usepackage{verbatim}
\usepackage{soul}
\usepackage{color}

\bibliographystyle{asa}

\usepackage{makeidx,bm,amsmath,url}
\makeindex

\floatstyle{plain}
\floatname{panel}{Panel}
\newfloat{algorithm}{h}{txt}[chapter]
\newfloat{panel}{h}{txt}[chapter]


\newcommand{\R}{\textbf{R}}
\newcommand{\bugs}{\textbf{BUGS}}
\newcommand{\jags}{\textbf{JAGS}}
\newcommand{\secr}{\mbox{\tt secr}}
\newcommand{\scrbook}{\mbox{\tt scrbook}}


\linenumbers

\begin{document}

\title{ Spatial Capture-Recapture  }
\subtitle{
%Hierarchical modeling of capture-recapture data with auxiliary spatial information
}
\author{The Four Horsemen (and women) }

\affiliation{First Author Short Address\\ Second Author Short Address}
\address{
USGS Patuxent Wildlife Research Center \\
North Carolina State University
}

\maketitle

\newpage

\setcounter{tocdepth}{2}
\tableofcontents

%\chapter{Introduction}
%\label{chapt.intro}

\chapter{
Introduction to Spatial Capture-Recapture
}
\markboth{Introduction}{}
\label{chapt.intro}


\vspace{.3in}

Information about abundance or density of populations, and their vital
rates, is fundamental to applied ecology and conservation biology.  To
that end, a huge variety of statistical methods have been devised, and
among these, the most well-developed are collectively known as
capture-recapture (or capture-mark-recapture) methods. For example,
the volumes by \citet{seber:1982}, \citet{borchers_etal:2002},
\citet{williams_etal:2002}, and \citet{amstrup_etal:2005} are largely
synthetic treatments of such methods, and contributions on modeling
and estimation using capture-recapture are plentiful in the
peer-reviewed ecology literature.  Capture-recapture techniques make
use of individual encounter history data, by which we mean sequences
of 0's and 1's denoting if an individual was encountered at a
particular trap during a certain time period. For example, the
encounter history ``010'' indicates that this individual was
encountered only during the second of three trapping occasions. As we
will see, these data contain
information about encounter probability, abundance, and other
parameters of interest in the study of population dynamics.

A diverse and growing number of methods exist for obtaining encounter
history data. Such methods are, naturally, taxa-specific. They include
classical ``traps'' which capture and retain animals until visited by
a biologist who removes the individual, marks it, or otherwise molests
it in some scientific fashion.  Small-mammal traps and mist nets for
birds are standard examples. Traps that physically capture and
restrain individuals are common, but capture-recapture methods no
longer require ``capture'' or even physical marking of individuals.
Recent technological advances have produced a
large number of passive detection devices that produce individual
encounter history data. These include camera traps
\citep{karanth_nichols:1998, oconnell_etal:2010}, acoustic recording
devices \citep{dawson_efford:2009}, and methods that obtain DNA
samples such as hair snares for bears \citep{gardner_etal:2010jwm}, scent
posts for many carnivores \citep{kery_etal:2010}, and related methods which allow DNA
to be extracted from scat, urine or animal tissue in order to identify
individuals.  This book is concerned with how such data can be used to
carry out inference about animal abundance or density, and other
demographic parameters such as survival, recruitment, and movement
using new classes of capture-recapture models which utilize auxiliary
spatial information related to the encounter process.  We refer to
such methods as spatial capture-recapture (SCR) models\footnote{In
the literature the term spatially explicit capture-recapture (SECR) is
also used}.

As the name implies, the primary feature of SCR models that
distinguishes them from traditional CR methods is that they make use
of the spatial information inherent to capture-recapture studies. That
is, the encounter histories are associated with spatial coordinates,
and these coordinates are informative about home range
characteristics, movement and space usage.
As we will see, this allows us to overcome three critical
deficiencies of non-spatial methods, namely,
traditional CR methods cannot be used to formally estimate density,
include of trap-level covariates of density or capture probability, or
account for heterogeneity in encounter probability that
results from the spatial organization of animals and traps.
Thus, spatial modeling is not just
a fun academic exercise; it provides a solution to basic problems in
the study of animal populations that have been acknowledged for more
than 70 years \citep{dice:1938}.


\section{Scope of this Book}

In this book, we try to achieve a broad methodological scope from
basic closed population models %using a number of distinct observation
%models
for inference about population density on up to open population models
for inference about vital rates such as survival and recruitment. %---spatial versions of
%conventional Jolly-Seber models. %A number of conceptual and
%methodological themes unify the main topical coverage of this book, and
%those are:
Much of the material is a synthesis of recent research but we also expand SCR models in a 
number of useful directions, including to accomodate unmarked individuals
(Chapt. \ref{chapt.xxxx}), use of telemetry information (Chapt. XXXX), and developing 
explicit models of individual space usage (Chapt. \ref{chapt.ecoldist}), and many other
new topics that have yet to appear in the literature. 
Our intent is to
provide a comprehensive resource for ecologists interested in
understanding and applying SCR models to solve common problems
faced in the study of population dynamics. To do so, we make use of
hierarchical models, which allow extrodinary
flexibility in accomodating virtually any type of capture-recapture
data. We present many example analyses, of real and simulated data
using likelihood-based and Bayesian methods---examples that readers
can replicate using the code presented in the text and
the resources made available on-line and in our accompanying {\bf R} package
{\tt scrbook}.

Although we aim to reach a
broad audience, at times we go into details that may only be of
interest to advanced practitioners who need to extend these models to
unique situations.  We hope that these advanced topics will not
discourage those new to these methods, but instead we believe this
material will allow readers to advance their understanding and become
less reliant on restrictive tools and software. Before disucssing the
specifics of SCR models, we begin with an overview of the methods
used to collect capture-recapture data, and a brief summary of
traditional non-spatial capture-recapture models.

\section{Lions and Tigers and Bears, oh my:  Genesis of
Spatial capture-recapture data}

A diverse number of methods and devices exist for producing individual
encounter history data with auxiliary spatial information about
individual locations. Historically, physical ``traps'' have been widely
used to sample animal populations. These include live traps, leg-hold
traps, mist nets, pitfall traps and many other types of
devices. Although these are still widely used, huge advances have been
made in developing new methodologies for obtaining encounter history
data non-invasively. We briefly review some of these here, which we
will consider more explicitly in later chapters of this book.

\subsection{Camera trapping}

Considerable recent work has gone into the development of
camera-trapping methodologies. For a historical overview of this
method see \citet{kays_etal:2008, kucera_barrett:2011}.  Several
recent synthetic works have been published including
\citet{nichols_karanth:2002}, and an edited volume by
\citet{oconnell_etal:2010} devoted solely to camera trapping concepts
and methods. As a method for estimating abundance some of the earliest
work that relates to the use of camera trapping data in
capture-recapture models originates from Karanth and colleagues
\citep{karanth:1995, karanth_nichols:1998, karanth_nichols:2000}. In
studies that use camera trapping, cameras are situated along trails or
at baited stations and individual animals are photographed and
subsequently identified either manually by a person sitting behind a
computer,  or sometimes now using computational
methods. Camera trapping methods are widely used for species that have
unique stripe or spotting patterns such as tigers \citep{karanth:1995,
  karanth_nichols:1998}, ocelots
\citep{trolle_kery:2003,trolle_kery:2005}, leopards
\citep{balme_etal:2010}, and many other cat species.
% Scientific names
Camera traps are
also used for other species such as wolverines
\citep{magoun_etal:2011}, and even species that are less easy to
identify uniquely such as mountain lions and coyotes
(e.g. \citet{kelly_etal:2008}.  We note that even for species that are
not readily identified by pelage patterns, it might be efficient to use 
camera traps in conjunction with spatial capture-recapture models to
estimate density (see Chapt.~\ref{chapt.scr-unmarked}).
%, if an initial sample of individuals can be collared
%or tagged in some way so that subsequent encounter by camera-traps can
%yield individual information. In this way, the probability of
%encounter can be estimated from the camera traps based on the
%pre-marked individuals, and this is applied to the frequencies of
%unmarked individuals to estimate density.


\begin{figure}
\begin{center}
\includegraphics[width=5in]{Ch1/figs/wolverinetiger}
\end{center}
\caption{Wolverine in camera trap from A. Magoun (left). Picture of Tiger in
  camera trap from U. Karanth (right)}
\label{fig.wolverinetiger}
\end{figure}

\subsection{DNA Sampling}

Recent technological advances in the extraction and analysis of
genetic information have made a huge positive impact on the study of
animal populations. DNA obtained from hair, blood or scat is now
routinely used to obtain individual identity and encounter history
information about individuals \citep{taberlet_bouvent:1992,
  woods_etal:1999, mills_etal:2000, schwartz_monfort:2008}.  A common
method is based on the use of ``hair snares'' (Fig. \ref{fig.bearcat})
which are widely used to study bear populations
\citep{woods_etal:1999, gardner_etal:2010jwm, garshelis_etal:2006,
  kendall_etal:2009}.  A sample of hair is obtained as individuals
pass under or around barbed-wire (or other physical mechanism) to take
bait. Hair snares have also been used to sample felid populations
\citep{garciaalaniz_etal:2010} and other species. DNA information can
also be extracted from urine and as a result DNA can be used to study
feline populations which are attracted to scent-sticks and deposit
urine which is subsequently analyzed in the lab
\citep{valiere_taberlet:2000, kery_etal:2010}.


\begin{figure}
\begin{center}
\includegraphics[width=5in]{Ch1/figs/bearcat}
\end{center}
\caption{Picture of hair snare. Bear (left). European wildcat
  (right). Pictures from??}
\label{fig.bearcat}
\end{figure}

\begin{figure}
\begin{center}
\includegraphics[width=5in]{Ch1/figs/beardog}
\end{center}
\caption{Guy holding fisher (left). Scat dog team working the ground
  (right). Pictures from Craig Thompson.}
\label{fig.fisherscatdog}
\end{figure}


\subsection{Acoustic surveys}

Many studies of birds \citep{dawson_efford:2009}, bats, and whales \citep{marques_etal:2009}  now collect data using
devices that record vocalizations. When vocalizations can be identified by individual from multiple
recording devices, then spatial encounter histories are produced that are amenable to 
the application of SCR models \citep{dawson_efford:2009, efford_etal:2009ecol}.

\subsection{Search-Encounter Methods}

There are other methods which don't fall into a nice clean taxonomy of
``devices''. Spatial encounter histories\footnote{defined? probably
  not! need to do that} are commomnly obtained by conducting manual
searches of geographic sample units such as quadrats, transects or
road or trail networks.
For example,
DNA-based encounter histories can be obtained from scat
samples located along roads or trails or by specially trained dogs
\citep{mackay_etal:2008} searching space
(Fig. \ref{fig.fisherscatdog}). This method has been used in studies
of martens, fishers \citep{thompson_etal:inpress}, lynx, coyotes,
birds \citet{kery_etal:2010}, and many other species. We might search
space on foot and pick up individuals and physically mark them
somehow. This is pretty common in surveys that involve reptiles and
amphibians, e.g., we might walk transects through a forest and pick-up
box turtles \citep{hall_etal:1999} or search space for lizards
\citep{royle_young:2008} and also surveys designed to obtain animal
scat. These methods don't seem like normal capture-recapture in the
sense that the encounter of individuals is not associated with
specific trap location, but SCR models are equally relevant for
analysis of such data (see Chapt. \ref{chapt.searchencounter}).


\section{ Historical Context: A Brief Synopsis of the Literature}

Spatial capture-recapture is a relatively new methodological
development, at least with regard to formal estimation and
inference. However, the basic problems that motivate the need for
formal spatially-explicit models have been recognized for decades and
quite a large number of ideas have been proposed to deal with these
problems. We review some of these ideas here.


\subsection{Buffering}

 The standard approach to estimating density even now is to estimate $N$ using
conventional closed population models \citep{otis_etal:1978} and then
try to associate with this estimate some specific sampled area, say $A$,
the area which is contributing individuals to the population for which
$N$ is being estimated. The strategy is to define $A$ by placing a buffer
of say $W$ around the trap array or some polygon which encloses the trap
array. The historical context is succintly put by \citep{obrien:2011}
from which we draw this description:

\begin{quote}
  ``At its most simplistic, $A$ may be described by a concave polygon
  defined by connecting the outermost trap locations ($A_{tp}$; \citet{mohr:1947}).
 This assumes that animals do not move from outside the
  bounded area to inside the area or vice versa. Unless the study is
  conducted on a small island or a physical barrier is erected in the
  study area to limit movement of animals, this assumption is unlikely
  to be true. More often, a boundary area of width $W$ ($A_{w}$) is added to
  the area defined by the polygon $A_{tp}$ to reflect the area beyond the
  limit of the traps that potentially is contributing animals to the
  abundance estimate \citep{otis_etal:1978}. The sampled area, also known
  as the effective area, is then $A(W) = A_{tp} + A_{w}$. Calculation of the
  buffer strip width ($W$) is critical to the estimation of density and
  is problematic because there is no agreed upon method of estimating
  $W$. Solutions to this problem all involve ad hoc methods that date
  back to early attempts to estimate abundance and home ranges based
  on trapping grids
  \citep[see][]{hayne:1949}. \citet{dice:1938} first drew attention
  to this problem in small mammal studies and recommended using
  one-half the diameter of an average home range. Other solutions have
  included use of inter-trap distances \citep{blair:1940,burt:1943}, mean
  movements among traps, maximum movements among traps \citep{holdenried:1940, hayne:1949},
 nested grids \citep{otis_etal:1978}, and assessment
  lines \citep{smith_etal:1971}.''
\end{quote}

The idea of using 1/2 mean maximum distance moved
\citep{wilson_anderson:1985a} seems to be the standard approach even
today, presumably justified by Dice''s suggestion to use 1/2 the home
range diameter. Alternatively, some studies have used the full
MMDM\footnote{Do they really say that?}
(e.g. \citet{parmenter_etal:2003}). And, sometimes home range size is
estimated by telemetry \citep{karanth:1995}\footnote{Is this correct
  cite for this?}. This is usually combined
with an AIC-based selection from among the closed-population models in
\citet{otis_etal:1978} which most often suggests heterogeneity (Model
Mh).  Almost all of these early methods were motivated by studies of
small mammals using classical ``trapping grids'' but, more recently,
their popularity has increased with the advent of new technologies and
especially related to non-invasive sampling methods such as camera
trapping. In particular, the series of papers by Karanth and Nichols
\citep{karanth:1995, karanth_nichols:1998, karanth_nichols:2002}
has led to fairly widespread adoption of these ideas.

Some of the heuristic ideas based on buffer strips do have some
technical justification in the sense of estimating parameters of an
underlying movement model from observed movements. For example, if we
let $x$ be a random variable indicating movement outcomes of an
individual about its  home range center, and suppose that $x$ has pdf
$g(x)$ then we can understand properties of MMDM by studying the
properties of the sample order statistics, as the maximum distance
moved is the sample range based on a sample of observations of
individual locations. 



%As an illustration, imagine a 1-dimensional
%system where individuals have a home range that amounts to a line
%segment. Then suppose that individual movements are $\mbox{uniform}(0,A)$. It
%can be shown that the sampling distribution of the sample range, R,
%scaled by $A$, say $R/A$ has a beta distribution, $\mbox{beta}(n-1,2)$
%\citep[][p. 235]{casella_berger:2002}
%and thus the diameter of the home range, i.e. $A$, is
%estimated (biasedly) by$ R/( (n-1)/(n+1) )$. For large $n$ we could then
%say that the sample range, i.e., ''maximum distance moved'' seems like a good estimator of home range diameter and, therefore, $R/2$ is an estimator of home-range radius.

%There are a number of technical issues that arise in attempting to use
%such heuristics to justify the application in practice. For one, the
%moments of the sample order statistics are strongly affected by sample
%size, which is typically quite small (per individual encountered) and
%thus, in general, are biased and estimated with variable precision
%depending on sample size. For example, the expected value of MMDM is
%$k(n)*A$ , i.e., the true home range diameter is related to observed
%MMDM by some function of sample size, $k(n)$, that increases to 1. In
%the case where the underlying movement model is uniform, $k(n) =
%(n-1)/(n+1)$ (from above) which motivates a formula for ``adjusting''
%observed MMDM for small sample size. We suspect that many such
%formulae are obtainable depending on the assumed movement distribution
%\citep[e.g., formula 6.16 in][]{obrien:2011}. We might also think about taking
%the {\it maximum} (over individuals) of the maximum distance moved
%because under the specific model considered here (iid uniform) then
%all individuals have the same home range radius. This increases our
%sample size ($n$) and thus the observed sample range should be more
%accurate.


%%Another issue of somewhat more importance (and less easy to
%rectify) is that the {\it observation} of movement outcomes is biased
%by the locations of traps. We cannot observe movements ``off the
%trapping grid'' (or between traps) and thus our observed movements
%will generally be smaller than expected under any particular model
%(the uniform in this case). Moreover, the trap spacing also induces a
%discreteness to the movements that causes a further level of
%approximation based on hypothetical movement
%distributions. Nevertheless, formal analysis of `` buffering''
%strategies based on sample order statistics under specific models for
%movement does at least provide some heuristic support for specific
%choices.  The interested reader should ponder the distribution of the
%sample minimum, maximum and range under other distributions such as a
%normal (and bivariate normal), exponential distribution and perhaps
%others. In addition, contemplate the effect of censoring of movements
%to some arbitrary limit ($B<A$) to mimic bias in observed movement
%outcomes due to a finite trap grid.

\subsection{Trapping webs}

The use of buffer strips is conventional and widespread due to the
heuristic appeal of that idea and its easy implementation, but other
conceptual approaches exist to address specific problems motivated by
the spatial context of capture-recapture data. D.R. Anderson came up
with the idea of the ``trapping web'' \citep{anderson_etal:1983} which
does not seem to have been widely adopted in practice.
% although there
%is a clear mathematical formalization to the trapping web design
%\citep{link_barker:1994}.
One reason for this is
the design is somewhat restrictive in the sense that it requires
a large number of traps be organized in close proximity to one
another.

\subsection{Temporary Emigration}

Another intuitively appealing idea is that by \citet{white_shenk:2000}
who discuss ``correcting bias of grid trapping estimates'' by
recognizing that the basic problem is like random temporary emigration
\citep{kendall_etal:1997} where individuals flip a coin with
probability $\phi$ to determine if they are ``available'' to be
sampled or not.  White and Shenk's idea was to estimate $\phi$ from
radio telemetry, as the proportion of time an individual spends in the
study area. They obtain the estimated super-population size by using
standard closed population models and then obtain density by $\hat{D}
= \hat{N}\hat{\phi}/A$ where $A$ is the nominal area of the trapping
array (e.g., minimum convex hull).  A problem with this approach is
that individuals that were radio collared represent a biased sample
i.e., you fundamentally have to sample individuals randomly from the
population {\it in proportion to their exposure to sampling} and that
seems practically impossible to accomplish. In other words, ``in the study area'' has no
precise meaning itself and is impossible to characterize in almost all capture-recapture studies.
Deciding what is ``in the study area'' is effectively the same as choosing an arbitrary buffer which defines
who is in the study area who who isn't.
That said, the temporary
emigration analogy is a good heuristic for understanding SCR models
and has a precise technical relevance to certain models.

Another very interesting idea is that of using some summary of
``average location'' as an individual covariate in standard
capture-recapture models. \citet{boulanger_mclellan:2001} use
distance-to-edge (DTE) as a covariate in the Huggins-Alho type of
model. \citet{ivan:2012} uses this approach in conjunction with an
adjustment to the estimated $N$ obtained by estimating the proportion of
time individuals are ``on the area formally covered by the grid''
using radio telemetry.  We do not dwell too much on these different
variations but we do note that the use of DTE as an individual
covariate amounts to some kind of intermediate model between simple
closed population models and fully spatial capture-recapture models,
which we address directly in Chapt. \ref{chapt.closed}.
%We note that no adjustment
%based on telemetry information is necessary if one were simply to
%place a prior distribution on the individual covariate (which is not
%to say that telemetry data isn't useful, just that the same objective
%can be achieved without telemetry data).

While these procedures are all heuristically appealing, they are also
essentially ad hoc in the sense that the underlying model remains
unspecified or at least imprecisely characterized and so there is
little or no basis for modifying, extending or generalizing the
methods. These methods are distinctly {\it not} model-based procedures
even though they might well be heuristically appealing under specific
movement models. Despite this, there seems to be an enormous amount of
literature developing, evaluating and ``validating'' these literally
dozens of heuristic ideas that solve specific problems, as well as
various related tweeks and tunings of them and really it hasn't led to
any substantive breakthroughs that are sufficiently general or
theoretically rigorous.



%A classical argument in favor of the HA model is
%that it ``doesn't require assumptions about the covariate'' but the
%assumption is explicit in capture-recapture models and thus it is
%natural to attack inference based on the ``joint likelihood''
%\citep{borchers_etal:2002}. This has proven necessary in certain other
%classes of individual covariate models in which natural models arise
%for the individual covariate, such as time-varying individual
%covariates \citep{bonner_schwarz:2006}, or covariates with measurement
%error (e.g., distance sampling; see
%\citet[][ch. 7]{royle_dorazio:2008}).
%The model-based formulation is easily adapted to standard
%individual covariate models as well \citep{royle:2008}. Throughout
%this book we rely heavily on Bayesian inference of the joint
%likelihood, using the formulation based on data-augmentation
%\citep{royle_etal:2007, royle_young:2008, royle:2009} though we also
%discuss the development of likelihood-based inference in chapter 5 and
%apply those methods in some cases.


\section{The Failure of Classical Capture-Recapture}

We briefly introduced and reviewed a number of classical techniques for applying non-spatial capture-recapture
models to studies of animal populations. These techniques, such as buffering, are based on many heuristically appealing
ideas. 
But these are just heuristics and do not resolve the essential, basic problem with conventional
(''non-spatial'') capture-recapture models which is that there is no linkage {\it in the model} between 
the quantity being informed by the data (i.e., $N$) and any stated or prescribed ``area'', $A$.
For capture-recapture models to provide a coherent framework for inference about population density,
$N$ has to scale, as part of the model, with $A$ so that the model imposes biological context
on $A$ (i.e., as the area over which the $N$ individuals reside). SCR models achieve this.

Put another way, 
ordinary capture-recapture methods are 
distinctly non-spatial. They don't admit spatial indexing of sampling
(observation) or
of individuals (process). This leads immediately to 3 main deficiencies:
 (1) there is no coherent basis for estimating density
 (2) non-spatial models {\it induce} a form of heterogeneity that can
 only at best be approximate by classical models of latent heterogeneity
 (3) ordinary models do not accommodate trap-level covariates which
 exist in a preponderance of studies.


We confront some of the issues that motivate the need for spatial
capture-recapture models by considering analysis of data from a study
design to estimate black bear abundance on the Fort Drum Military
Installation in upstate New York (see Ch. 3 for more details). The
specific data used here are encounter histories on 47 individuals
obtained from an array of 38 baited ``hair snares'' during June and
July 2006. The study area and locations of the 38 hair snares are
shown in Fig. \ref{fig.hairsnares}.  Barbed wire traps (see
Fig. \ref{fig.bearcat}) were baited and checked for hair samples each
week for eight weeks.  Analysis of these data appears in
\citet{gardner_etal:2010jwm} and we use the data in a number of analyses
in later chapters.

\begin{figure}
\begin{center}
\includegraphics[height=3in]{Ch1/figs/hairsnares}
\end{center}
\caption{Locations of black bear hair snares on Fort Drum.}
\label{fig.hairsnares}
\end{figure}

We regarded this data set as a standard capture-recapture data set -
an encounter history matrix with 47 rows and 8 columns with entries
$y_{ik}$, where $y_{ik}=1$ if individual $i$ was captured in sample
$k$ and $y_{ik}=0$ otherwise. There is a standard closed population
model, colloquially referred to as ``Model M0'' (see Ch. 3), which
assumes that encounter probability $p$ is constant for all individuals
and sample periods.  We fitted Model M0 to the Fort Drum data using
traditional likelihood methods, yielding the maximum likelihood
estimate (MLE) of $\hat{N} = 49.19$ with an asymptotic standard error
(SE) of $1.9$.

The key issue in using closed population models with such data is how
on earth do we interpret this estimate of $N=49.19$ bears? Does it
represent the entire population of Fort Drum? Certainly not -- we
merely sampled half of the Fort! So to get at the total bear
population size of Fort Drum , we'd have to convert our $\hat{N}$ to
an estimate of density and extrapolate. To get at density, then,
should we
assert that $N$ applies to the southern half of Fort Drum below some
arbitrary line? Surely bears move on and off of Fort Drum without
regard to hypothetical boundaries. Without additional information
there is simply no way of converting this estimate of $N$ to density,
and hence it is really not meaningful biologically. To resolve this
problem, we will adopt the customary approach of converting $N$ to $D$
by buffering the convex hull around the trap array. The convex hull
has area $157.135$ $km^2$. We follow \citet{bales_etal:2005} in
buffering the convex hull of the trap array by the radius of the mean
female home range size\footnote{Did Bales et al. actually do this?}.
The mean female home range radius was
estimated \citep{wegan:2008} for our study region to be $2.19$
km\footnote{Is this number right out of Wegan's disseration?}, and
the area of the convex hull buffered by $2.19$ km is $277.01$
km$^2$. ({\bf R}
commands to compute the convex hull, buffer it, and compute the area
are given in the {\bf R} package \mbox{scrbook} which accompanies the
book).  Hence, the estimated density
here is approximately $0.178$ bears/km$^2$ for an estimated population
size obtained using Model $M_0$.  We could assert that the problem has
been solved, go home, and have a beer.  But then, on the other hand,
maybe we should question this estimated home range radius from
\citep{wegan:2008} -- after all, home ranges can change for many reasons. Instead, we may decide to rely on a buffer width based on
one-half MMDM estimated from the actual hair snare data as is more customary
\citep{dice:1938}. In that case the buffer width is $1.19$ km, and the
resulting estimated density is increased to $0.225$ bears/ha$^2$ about
27 \% larger.  But wait - some studies actually found the full MMDM
\citep{parmenter_etal:2003} to be a more appropriate measure of
movement (e.g \citet{soisalo_cavalcanti:2006}). So maybe we should use
the full MMDM
which is $2.37$ km, pretty close to the telemetry-based estimate
and therefore providing a similar estimate of density ($0.171$
bears/ha$^2$). So in trying to decide how to buffer our trap array we
have already generated 3 density estimates. The crux of the matter is
obvious: Although it is intuitive that $N$ should scale with area --
the number of bears should go up as area increases and go down as area
decreases -- in this ad hoc approach of accounting for animal movement
$N$ remains the same, no matter what area we decide we sampled. The
number of bears and the area they live in are not formally tied
together within the model, because estimating $N$ and estimating the
area $N$ refers to are two completely independent analytical steps.

Unfortunately, our problems don't end here. In thinking about the use of model M0, we might naturally question
some of the basic assumptions that go into that model. The obvious one
to question is that which declares that $p$ is constant. One obvious
source of variation in $p$ is variation {\it among individuals}. We
expect that individuals may have more or less exposure to trapping due
to their location relative to traps.
%Maybe we could add a table of how many traps each bear was caught in
% #traps: 1   2  3  4  5  6  7  8 10
% #bears: 19 15  5  2  2  1  1  1  1

This has led many to consider
capture-recapture models that allow for individual heterogeneity in
$p$. Such models have the colloquial name of ``Model Mh.''
We fitted this model (see ch. 3 for details) to the Fort Drum data
using each of the 3 buffer widths previously described (telemetry, 1/2
MMDM and MMDM), producing the estimates reported in Table
\ref{intro.tab.fdests}. While we can tell by the models' AIC that Mh is
clearly favored by more than 30 units, we might still not be entirely
happy with our results. Clearly there is information in our data that
could tell us something about the exposure of individual bears to the
trap array -- where they were captured, and how many times -- but
since space has no representation in our model, we can't make use of
this information. Model Mh thus merely accounts for what we observe in
our data (some bears were more frequently captured than others) rather
than explicitly accounting for the processes that generated the data.

So what are we left with?  Our density estimates span  a range
from $0.17$ to $0.43$ bears/km$^2$ depending on which estimator of $N$ we use and
what buffer strip we apply. Should we feel strongly about one or the other?
AIC favors model Mh, but did it adequately account for the differences in exposure of individuals to the trap array? If so, which buffer should we
prefer? \footnote{Give AIC of models in table. Andy to finish. }
%Moreover, we could find more variations of
%model Mh to choose among, but see \citep{link:2003}.
And if we choose one type of buffer, how do we compare our density estimates to those from other studies that may opt for a different kind of buffer?
Clearly, there is no compelling solution to deriving density from our
estimate of $N$, and we are left not much wiser about bear density at
Fort Drum than we were before we conducted this analysis.

%%%% We could just finish this part off with a paragraph about these additional open questions - the whipped cream of problems on the capture-recapture sundae - including the trap-level covariates; or we could come up with some trap-level covariate example for the bears (different baits used, blabla).
Some of the open questions at this point:
How do we characterize uncertainty of the buffer ``estimate''?  And,
in what sense is the
buffer even an estimate of something? What is it an estimate of?
The summary here should be that there's not a compelling solution to be derived from
this ``estimate $N$ conjure up a buffer'' approach.
{\bf The main point that N doesn't scale with A is not made
  clearly here.}

\begin{table}[ht]
\centering
\caption{Table on estimates of D for the Fort Drum data
using M0 and Mh and different buffers.}
\begin{tabular}{ll|cc}
\hline
model & buffer &  $\hat{D}$ & SE \\ \hline
M0   & telemetry &  0.178 & 0.178 \\
M0    & MMDM     &  0.171 & 0.171\\
M0   & 1/2 MMDM  &  0.225 & 0.225\\
Mh(ln) & telemetry &0.341 & 0.144\\
Mh(ln) & MMDM    &  0.327 & 0.138\\
Mh(ln) & 1/2 MMDM & 0.432 & 0.183\\
\end{tabular}
\label{intro.tab.fdests}
\end{table}



\section{Extension of Closed Population Models}

The deficiency with classical closed population models is that they
have no spatial context. $N$ is just an integer parameter that applies
equally well to some population in a computer, estimating the number
of unique words in a book, or a bucket full of goldfish.  The question
of {\it where} the $N$ items belong is central both to interpretation
of data and estimates from all capture-recapture studies and, in fact,
to the construction of spatial capture-recapture models considered in
this book.  Surely it must matter whether the $N$ items exist as words
in a book, or goldfish in a bowl, or birds in a forest patch! That
classical closed population models have no spatial context leads to a
number of conceptual and methodological problems or limitations as we
have discussed and even encountered in our analyses so far.

Thus, the essential problem is that classical closed population models
are too simple - they ignore the spatial attribution of traps and
encounter events, movement and variability in exposure of individuals
to trap proximity, and they do not yield estimates of {\it density}.
These are not problems per se but rather just features
of this simple class of models, and they
should be addressed formally by the development of
more general models.



\subsection{The modern age}

%Spatial capture-recapture models are
%statistical and mathematical models that extend non-spatial
%``ordinary'' capture-recapture models to accommodate the spatial
%structure inherent in sampling animal populations - i.e., trap
%locations, individual locations, and individual use of space.

The solution to the various issues that arise in the application of
ordinary capture-recapture models is to extend the closed population
model so that $N$ becomes spatially explicit.  \
%A natural way is to
%define a point process \citep{efford:2004} that describes how
%individuals are organized in space and that, when points are
%aggregated over space, the value $N$ is derived in a meaningful way.
%Thus, in this book, we adopt the view that the locations of the $N$
%individuals in the population are a {\it realization of a spatial
%  point process}.
\citet{efford:2004} was the first to formalize an explicit model for
spatial capture-recapture problems in the context of trapping arrays.
He adopted a Poisson point process model to describe the distribution
of individuals and then what is essentially a distance sampling
formulation of the observation model which describes the probability
of detection as a function of individual location, regarded as a
latent variable governed by the point process model. While earlier
(and contemporary) methods of estimating density from trap arrays have
been ad hoc in the sense of lacking a formal description of the
spatial model, Efford achieved a formalization of the model,
describing explicit mechanisms governing the spatial distribution of
individuals and how they are encountered by traps, but
adopted a more or less ad hoc framework for inference under that
spatial model using a simulation based method known as inverse
prediction \citep{gopalaswamy:2012}.

Recently, there has been a flurry of effort devoted to formalizing
inference under this model-based framework for the analysis of spatial
capture-recapture data \citep{gopalaswamy:2012}. There are two distinct lines of work which
adopt the model-based formulation in terms of the underlying point
process but differ primarily by the manner in which inference is
achieved. One approach \citep{borchers_efford:2008} is a classical inference approach based on
likelihood, and the other \citep{royle_young:2008} adopts a
Bayesian framework for inference.

To motivate the origins and relevance of these approaches, we note
that, fundamentally, spatial capture-recapture models are related to
classical ``individual covariate'' models (colloquially referred to as
Huggins-Alho models) in capture-recapture \citep{huggins:1989,
  alho:1990}.  In particular, the individual covariate\footnote{have
  we mentioned what the individual covariate is, yet?} is observed in
these classical individual covariate models whereas it is not directly
observed in SCR models.  To accommodate that, a prior distribution for
the individual covariate is required. In essence then, SCR models are
similar to a fully model-based formulation of classical Huggins-Alho
models (see \citet{royle:2009}). Likelihood analysis
\citep{borchers_efford:2008} proceeds by removing the random effect
from the likelihood by integration whereas Bayesian analysis
\citep{royle_young:2008} proceeds by analyzing the conditional model
directly, usually by methods of Markov chain Monte Carlo (MCMC).




\subsection{Abundance as the Aggregation of a Point Process}

Spatial point process models represent a major methodological theme in
spatial statistics \citep[][ch. xyz]{cressie:1992} and they are
widely applied as models for many ecological phenomena
\citep{stoyan_penttinen:2000,illian_etal:2008}. Point process models apply to
situations in which the random variable in question represents the
locations of events or objects: trees in a forest, weeds in a field,
bird nests, etc.  As such, it seems natural to describe the
organization of individuals in space using point process models.

One
of the key features of SCR models is that the point locations are
latent, or unobserved, and we only obtain imperfect information about
the point locations by observing individuals at trap or observation
locations.  Thus, the realized locations of individuals represent a
type of ``thinned'' point process, where the thinning mechanism is not
random but, rather, biased by the observation mechanism.  It is
natural to think about the observed point process as some kind of a
compound or aggregate point process with a set of ``parent'' nodes
being the locations of individual home ranges or their centroids,
and the observed locations as
``offspring'' - i.e., a Poisson cluster process (PCP). In that
context, density estimation in SCR models is analogous to estimating the number of
parents in the PCP \citep{chandler_royle:2012}.
% Other types of point
% process models for the realized locations have direct relevance to SCR
% models (See \citet{chandler_royle:2012}, discussed in chapter XYZ).

In the context of SCR models, we suppose there is a point on the
landscape that we'll think of as a home range center or, if this is
unappealing, we can think of it as the centroid of an individual's
activities during the time of sampling. In general, this point is
unknown for any individual but if we could track an individual over
time and take many observations then we could perhaps get a good idea
of where that point is.  We'll think of the collection of these points
as defining the spatial distribution of individuals in the
population. Most of the recent developments in modeling and inference
from spatial encounter history data, including most methods discussed
in this book, are predicated on the view that individuals are
organized in space according to a relatively simple point process
model. More specifically, we assume that the collection of individual
activity centers are ``$iid$'' random variables distributed uniformly
over some region. This is consistent with the assumption that the
activity centers represent the realization of a Poisson point process
or, if the total number of activity centers if fixed, then this is
usually referred to as a binomial point process.

%%I think we could shorten the home range paragraph; I like the definition
%%'the centroid of an individual's
%%% activities during the time of sampling'. I think the definition of
%% home range is something like the colleciton of points/sites/areas
%% an animal uses over the course of its lifetime so it's vague anyway
%% and what that definition means for the different forms of home
%%ranges - territory, migratory species etc - is pretty much left open.
We use the terms home range or activity center interchangeably. The
term ``home range center'' suggests that models are only relevant to
animals that exhibit such behavior of establishing home ranges or
territories and since not all species do that, perhaps the
construction of SCR models based on this idea is flawed. However,
 the notion of a home range center is just a conceptual
device and we don't view this concept as being strictly consistent
with classical notions of animal territories. Rather our view is
that a home range or territory is inherently dynamic, temporally, and thus it is a
transient quantity - where the animal lived during the period of
study.  Therefore, whether or not individuals of a species establish home ranges
is irrelevant because, once a precise time period is defined, this defines a distinct region of space
that an individual must have occupied. In other
words, the definition of ``home range center'' is predicated, in a
sense, on the specification of a time period over which individuals
are studied. A term that might be less offensive than ``home range
center'' is ``centroid of space usage (CSU)'' which should not
conflict directly with preconceived understandings and interpretations
of home range\footnote{Utilization distribution is the same thing
  I guess?}.


\subsection{The state-space}

If we let ${\bf s}_{i}; i=1,2,\ldots,N$ be the locations of individual
activity centers, then the question ``what are the possible values of
${\bf s}$?'' needs to be addressed because the individual ${\bf
  s}_{i}$ are {\it unknown}. As a technical matter, we will regard
them as random effects and in order to apply standard methods of
statistical inference we need to provide a distribution for these
random effects.  In the context of the point process model, the
possible values of the point locations referred to as the
``state-space'' of the point process and this is some region or set of
points which we will denote by ${\cal S}$.
${\cal S}$ is a region within which points are located - essentially a
prior distribution for ${\bf s}_{i}$ (or, equivalently, the random effects
distribution).
%%Don't think prior has come up yet; maybe not that important here?
In animal studies as a description of
where individuals that could be captured are located it encloses our
study area -- the region within which we might have located traps or
detection devices.  The state-space of the point process should
accommodate all individuals that could have been captured in the study
area.

In the practical application of SCR models, in most cases estimates of
density will be relatively insensitive to choice of state-space (see
Section XYZ)
%%% I also think the rest of this paragraph could be postponed to a later chapter
unless there are meaningful features to the state-space
which should be accommodated. For example, if the region within which
traps are located contains a coastline or a huge body of water then
clipping that out of the state-space will typically have a large
effect on density. This should be expected because, insofar as the
state-space serves as a prior distribution on the latent variables
${\bf s}_{i}$ then, {\it the state-space is very much a
  component of the model. } We discuss choosing the state-space in
Chapter 4.

When the underlying point process is well-defined, including a precise
definition of the state-space, this in turn induces a precise
definition of the parameter $N$ ``population size'' as the number of
individual activity centers located within the prescribed state-space.
A deficiency with some classical methods of ``adjustment'' is they
attempted to prescribe something like a state-space - a ``sampled
area'' - except absent any precise linkage of individuals with the
state-space. SCR models formalize the linkage between individuals and
space and, in doing so, provide an explicit definition of $N$
associated with
a well-defined spatial region, and hence
density. In a sense, the whole idea of SCR models is that by defining
this point process and its state-space ${\cal S}$, this gives context and
meaning to $N$ which can be estimated directly for that specific
state-space. Thus, it is fixing ${\cal S}$ that resolves the problem of
``unknown area'' that was addressed previously (Section XXXX).
%% I find the next two sentences a little confusing
But the
existence of an explicit state-space ${\cal S}$ is kind of beside the
point -- ${\cal S}$ is really not always terribly important
itself. Instead, as soon as you give the latent variables ${\bf s}$ a
place to live, and this is recognized explicitly in the model upon which inference is based,
 then you achieve spatial explicitness of the model.










\subsection{Other elements of SCR models}

Broadly speaking we differentiate
between two situations: Sampling based on fixed arrays or sampling
based on ``search encounter'' methods. The former includes things like
camera traps, hair snares, mist nets and conventional traps. Fixed
arrays limit the observation location to pre-defined points, where
traps are located. Using such methods the model is a little simpler
because the ``movement process'' of individuals is confounded with the
``observation process''.
The 2nd type of model -- search encounter models -- typically
will allow locations in continuous space, possibly only restricted by
polygon boundaries \citep{royle_young:2008}.
Search-encounter data
usually allow for the separate modeling and estimation of movement
model parameters from encounter model parameters but not always,
depending on whether replication of the sampling is done.  The
classical distance sampling model with no replication (i.e., $t=1$) is a basic model
which confounds the two processes.


Depending on the type of device being considered, certain restrictions
on the observable variable are induced which suggest specific
probability models for the observable random variable, suggesting
either binomial, Poisson or multinomial (and possibly other)
observation models.
One type of a
device is what we think of as the classical ``camera trap'' and which
\citet{efford:2011} refers to as a ``proximity detector''. We can take
pictures of or detect any number of individuals and an individual can
be caught in any number of traps, and an arbitrary number of
times. Iid Bernoulli model is convenient but if you think the
re-encounters are valuable then you can have a frequency model.  Bear
hair snares are slightly different because you cannot differentiate
re-encounters.
The standard observation model that applies for ``single-catch''
\citep{efford_etal:2004} traps posits that individuals are encountered
in at most one trap per sample occasion and traps only hold one
individual.  Unfortunately we're really screwed in the single-catch
situation.
A ``multi-catch'' is like a mist-net or other things - individual is
captured and restrained but traps hold > 1 individual. In this case,
the observation model is a multinomial. There are
many variations on all of these models and new models.







\begin{comment}

\subsection{Why is density so important? }

Knowledge of population size is a fundamental piece of information in
conservation. Since the risk of a species/population going extinct is
a function of how many individuals of that species there are, much of
conservation-related research revolves around abundance. Consider, for
example, the concept of minimum viable population size � to assess
whether a population has a good chance of persistence over some time
frame we need to know how big it is to begin with. The idea of a
minimum viable population is reflected in many applied conservation
efforts. For example, in a range-wide assessment of the jaguar�s
population status, researchers were asked to delineate Jaguar
Conservation Units (JCU�s), of which one criterion was ``holding at
least 50 jaguars'' � a number considered a substantial population
\citep{sanderson_etal:2002}.

While the importance of abundance is indisputable, there are some
major issues associated with this measure. First, you cannot compare
mere values of abundance unless they refer to a specific area. If you
look at the IUCN Red List of Endangered Species entry for the
population status of the tiger, it will tell you that there are an
estimated 1700 tigers in India but only about 20 in Cambodia
\citep{chundawat_etal:2011}. Now, this will not automatically make you
lament the state of tiger conservation in Cambodia as compared to
India (although seeing these numbers you might well lament the state
of the tiger in general), because you know these numbers refer to
countries that are extremely different in size. Rather, if you wanted
to know something about where tigers are currently doing better,
you�d probably divide the number of tigers by the countries�
areas and compare tiger densities (turns out India�s tigers are
still doing better, not by a factor of 85, as mere abundances suggest,
but by a factor of 5). Although abundance and density are obviously
directly related to each other, they are different in their
applicability. Particularly, density as a scaled measure lets us
compare results across sites (as we just demonstrated for the tiger
example). In addition, some concepts incorporated in conservation
biology explicitly deal with density. For example, population growth
rate, home ranges or the probability of epidemics/disease spread are
density-dependent; the Allee effect links individual reproductive
success to population density in low-density populations.

Second, going back to the tiger example once more, we may wonder how
researchers even came up with these numbers for total population
size. Tiger abundance can be estimated using camera-traps, because
individuals have distinct stripe patterns so that photographic data
can be analyzed with capture-recapture models. But surely, no-one ever
camera-trapped the whole of India. This is a typical situation, even
on a much smaller scale. Ecologists generally sample only a small
fraction of the area used by a species or population, but want to
estimate total population size, i.e. the number of individuals
occurring in sampled {\it and unsampled} areas. If we can use the data
from sampled area to obtain a density estimate, explicit predictions
of abundance can be made to regions of any size (assuming that density
is constant across the region we are inferring to and equal to density
in the sampled area)\footnote{Note that the way total tiger abundance
  estimates are derived for India is much more complex than just
  looking at tiger density somewhere in India and then extrapolating
  it to the entire country (for details, see \citep{jhala_etal:2011});
  we merely use these numbers here to illustrate the general
  problem.}.

To summarize, density not only influences several ecological
processes, but also allows us to compare population status among
different sites; even where total abundance is of primary interest,
density can help us arrive at a total population estimate even when
we�re unable to survey the total population. Capture-recapture
models were designed to estimate abundance, but they generally cannot
be used to formally estimate density. This limitation of non-spatial
CR models has long been recognized (REF) and several ad hoc approaches
to overcome this problem have been devised. We will discuss those and
their shortcomings in XXX. The great advantage of SCR models over
non-spatial capture-recapture models is that they formally link
abundance and area so that they actually estimate density.


\end{comment}









\section{Summary: The Promise of Spatial Capture-Recapture}

Spatial capture-recapture models are an extension of ordinary
capture-recapture models to accommodate the spatial organization of
both individuals in a population and the observation mechanism (e.g.,
locations of traps).  They resolve problems which have been recognized
historically and for which various ad hoc solutions have been
suggested: heterogeneity in encounter probability due to the spatial
organization of individuals relative to traps, the ability to model
trap-level effects on encounter, and that a
well-defined sample area does not exist in most studies, and thus
estimates of $N$ using ordinary capture-recapture models cannot be
related directly to density.

However, SCR models are not merely an extension of technique but
rather they represent an extention in a much more
profound way in that they make ecological processes explicit in the
model -- processes of spatial organization of individuals, movement
and space-usage of individuals. While capture-recapture models have
existed for decades this is a completely new element of
closed capture-recapture models.
This is so profoundly important because
ecological scientists study elements of ecological theory using
observational data that exhibits various biases relating to the
observation mechanisms employed. In the context of capture-recapture,
we observe individual encounter history data from which we can use SCR
models to infer where individual live, how they organize themselves in
space and move around in space and how they interact with other
individuals.  Moreover, SCR models show great promise in their ability
to integrate explicit ecological theories directly into the models so
that we can directly test hypotheses about either space usage (e.g.,
Chapter XYZ) or movement (Chapter. XYZ) or the distribution of
individuals in space (Chapter XYZ). We imagine that in the near future
SCR models will include point process models that allow for
interactions among individuals such as inhibition or  clustering.

Thus, SCR models are capture-recapture models that enable ecologists
to explicitly integrate biological context and theory with encounter
history data, which is something that has always been the focus of
``open population'' models but never, until very recently, has been
considered formally in closed population models. We therefore believe
that SCR models will enable ecologists to test theories of space usage
and environmental effects, social behavior and other important
theories.


In chapter 2 we provide the basic analysis tools to understand and
analyze SCR models - namely GLMs with random effects, and their
analysis in R and WinBUGS.  Because SCR models represent extensions of
basic closed population models, we cover ordinary closed population
models in chapter 3 wherein, along with chapter 4, we will see that
SCR models are a type of individual covariate model, which are
conceptual and technical intermediates between Model Mh and classical
individual covariate models.  In subsequent chapters we will cover a
bunch of different types of SCR models related to the type of
encounter process - e.g., type of trap - and also different
embellishments of the basic model structure as alluded to in section
XYZ above.  We will consider many different extensions of SCR models
to accommodate covariates on encounter probability, and density. We
also consider important practical extensions such as SCR for open
populations (Chapter xyz), combining SCR data with auxiliary
information from telemetry (chapter XYZ) and multiple encounter
methods (chapter XYZ).


\chapter{
Bayesian Analysis of GL(M)Ms Using R/WinBUGS
}
\markboth{Bayesian Analysis of GLMMS}{}
\label{chapt.glms}

\vspace{.3in}

%%%% STUFF TO DO
%%% 1. Prior lack of invariance to transformation stuff: Reference and Figure
%%% 2. Full conditional example from ch. 6 copy notation
%%% 3. Check out algorithm environment
%%% 4. reference for sampling from f() with bounded support
%%% 5. need refs on choosing prior disributions
%%% 6. Check Bayesian p-value definition
%%% 7. FIX parameter notation! I have beta0 beta1 , alpha beta, and a,
%%%     b in the same chapter!   Use alpha beta probably?
%%% 8. spell check this document

A major theme of this book is that spatial capture-recapture models
are, for the most part, just generalized linear models (GLMs) wherein
the covariate, distance between trap and home range center, is
partially or fully unobserved  -- and therefore regarded as
a random effect. Such models
are usually referred to as Generalized Linear Mixed Models (GLMMs)
and, therefore, SCR models can be thought of as a specialized type of
GLMM. Naturally then, we should consider analysis of these slightly
simpler models in order to gain some experience and, hopefully,
develop a better understanding of spatial capture-recapture models.

In this chapter, we consider classes of GLM models - Poisson and
binomial (i.e., logistic regression) GLMs - that will prove to be
enormously useful in the analysis of capture-recapture models of all
kinds. Many readers are probably familiar with these models because
they represent probably
the most generally useful models in all of Ecology and, as
such, have received considerable attention in many introductory and
advanced texts. We focus on them here in order to introduce the
readers to the analysis of such models in {\bf R} and {\bf WinBUGS},
which we will
translate directly to the analysis of SCR models in subsequent
chapters.

Bayesian analysis is convenient for analyzing GLMMs because it allows
us to work directly with the conditional model -- i.e., the model that
is conditional on the random effects, using computational methods
known as Markov chain Monte Carlo (MCMC). Learning how to do Bayesian
analysis of GLMs and GLMMs in {\bf WinBUGS} is, in part, the purpose
of this chapter.  While we use {\bf WinBUGS} to do the Bayesian
computations, we organize and summarize our data and execute {\bf
  WinBUGS} from within {\bf R} using the useful package \mbox{\tt
  R2WinBUGS} \citep{sturtz_etal:2005}.  \citet{kery:2010}, and
\citet{kery_schaub:2011} provide excellent introductions to the basics
of Bayesian analysis and GLMs at an accessible level. We don't want to
be too redundant with those books and so we avoid a detailed
treatement of Bayesian methodology - instead just providing a cursory
overview so that we can move on and attack the problems we're most
interested in related to spatial capture-recapture.  In addition,
there are a number of texts that provide general introductions to
Bayesian analysis, MCMC, and their applications in Ecology including
\citet{mccarthy:2007}, \citet{kery:2010}, \citet{link_barker:2009},and
\citet{king_etal:2009}.


While this chapter is about Bayesian analysis of GLMMs, such models
are routinely analyzed using likelihood methods too, as discussed by
\citet{royle_dorazio:2008}, and \citet{kery:2010}. Indeed, likelihood
analysis of such models is the primary focus of many applied
statistics texts, a good one being \citet{zuur_etal:2009}. Later in
this book, we will use likelihood methods to analyze SCR models but,
for now, we concentrate on providing a basic introduction to Bayesian
analysis because that is the approach we will use in a majority of
cases in later chapters.


\section{ Notation}

We will sometimes use conventional ``bracket notation'' \index{bracket
  notation} to refer to
probability distributions. If $y$ is a random variable the $[y]$
indicates its distribution or its probability density/mass function
(pdf, pmf) depending on context. If $x$ is another random variable
then $[y|x]$ is the conditional distribution of $y$ given $x$, and
$[y,x]$ is the joint distribution of $y$ and $x$. To differentiate
specific distributions in some contexts we might label them $g(y)$,
$g(y|\theta)$, $f(x)$, or similar. We will also write $y \sim
\mbox{Normal}(\mu,\sigma^{2})$ to indicate that $y$ ``is distributed as'' a normal
random variable with parameters $\mu$ and $\sigma^{2}$. The expected value
or mean of a random variable is $E[y] = \mu$ ,and $Var[y] = \sigma^{2}$ is
the variance of $y$.  To indicate specific observations we'll use an
index such as ``$i$''. So, $y_{i}$ for $i=1,2,\ldots,n$ indicates
observations for $n$ individuals. Finally, we write $\Pr(y)$ to indicate specific probabilities, i.e., of events ``$y$'' or similar.


To illustrate these concepts and notation, suppose $z$ is a binary
outcome (e.g., species occurrence) and we might assume the model: $z
\sim \mbox{Bern}(p)$ for observations.  Under this model $\Pr(z=1) =
\psi$, which is also the expected value $E[z] = \psi$. The variance is
$Var[z] = \psi*(1-\psi)$ and the probability mass function (pmf) is $[z]
= \psi^{z} (1-\psi)^{1-z}$. Sometimes we write $[z|\psi]$ when it is
important to emphasize the conditional dependence of $z$ on $\psi$. As
another example, suppose $y$ is a random variable denoting whether or
not a species is detected if an occupied site is surveyed. In this
case it might be natural to express the pmf of the observations $y$
{\it conditional} on $z$. That is, $[y|z]$. In this case, $[y|z=1]$ is
the conditional pmf of $y$ given that a site is occupied, and it is
natural to assume that $[y|z=1] = \mbox{Bern}(p)$ where $p$ is the
``detection probability'' - the probability that we detect the
species, given that it is present. The model for the observations $y$
is completely specified once we describe the other conditional pmf
$[y|z=0]$. For this conditional distribution it is sometimes
reasonable to assume $\Pr(y=1|z=0) = 0$ (\citet{mackenzie_etal:2002};
see also \citet{royle_link:2006}). That is, if the species is absent,
the probability of detection is 0. This implies that
$\Pr(y=0|z=0)=1$. To allow for situations in which the true state $z$
is unobserved, we  assume that $[z]$ is Bernoulli with parameter
$\psi$.  In this case, the marginal distribution of $y$ is
\[
 [y] = [y|z=1]Pr(z=1) + [y|z=0]Pr(z=0)
\]
because $[y|z=0]$ is a point mass at $y=0$, by assumption, then
\[
\Pr(y=1) = p \psi
\]
And
\[
\Pr(y=0) = (1-p)*\psi + (1-\psi)
\]


\section{
GLMs and GLMMs}
We have asserted already that SCR models work out most of the time to
be variations of GLMs and GLMMs. Some of you might therefore ask: What
are GLMs and GLMMs, anyhow?   These models are covered extensively in
many very good applied statistics books and we refer the reader
elsewhere for a detailed introduction. We think \citet{kery:2010},
\citet{kery_schaub:2011}, and \citet{zuur_etal:2009} are all
accessible treatments of considerable merit. Here, we'll give the 1
minute
treatment of GLMMs, not trying to be complete but rather only
to preserve a coherent organization to the book.


The generalized linear model (GLM) is an extension of standard linear
models by allowing the response
variable to have some distribution from the exponential family of
distributions (i.e., not just normal). This includes the normal
distribution but also dozens of others such as the Poisson, binomial,
gamma, exponential, and many more. In addition, GLMS allow the
response variable to be related to the predictor variables (i.e.,
covariates) using a
link function, which is usually nonlinear.  Finally, GLMs typically
accommodate a relationship between the mean and variance. The
classical reference for GLMs is \citet{nelder_wedderburn:1972} and
also \citet{mccullagh_nelder:1989}.
The GLM consists of three components:
\begin{itemize}
\item[1.] A probability distribution for the dependent variable $y$,
from a class of probability distributions known as the exponential family.
\item[2.] A ``linear predictor'' $\eta = {\bf X}{\bm \beta}$  .
\item[3.] A link function $g$ that relates $E[y]$ to the linear predictor, $E[y] = \mu = g^{-1}(\eta)$. Therefore $g(E[y]) = \eta$.
\end{itemize}

The dependent variable $y$ is assumed to be an outcome from a
distribution of the exponential family which includes many common
distributions including the normal, gamma, Poisson, binomial, and many
others. The mean of the distribution of $y$ is assumed to depend on predictor variables $x$ according to
\[
 g(E[y]) = {\bf x}'{\bm \beta}
\]
where $E[y]$ is the expected value of $y$, and ${\bf x}'{\bm \beta}$
is termed the {\it linear predictor}, i.e., a linear function of the
predictor variables with unknown parameters ${\bm \beta}$ to be
estimated.  The function $g$ is the link function. In standard GLMs,
the variance of $y$ is a function $V$ of the mean of $y$: $Var(y) =
V(\mu)$ (see below for examples).

A Poisson GLM posits that $y \sim \mbox{Poisson}(\lambda)$ with $E[y]
=\lambda$ and usually the model for the mean is specified using the
{\it log link function} by
\[
log(\lambda_{i}) = \beta_0 + \beta_{1}*x_{i}
\]
The variance function is $\mbox{V}(y_{i}) = \lambda_{i}$.  The
binomial GLM posits that $y_{i} \sim \mbox{Binomial}(K,p)$ where $K$
is the fixed sample size parameter and $E[y_{i}] = K*p_{i}$. Usually
the model for the mean is specified using the {\it logit link
  function} according to
\[
 logit(p_{i}) = \beta_{0} + \beta_{1}*x_{i}
\]
Where $logit(u) = log(u/(1-u))$.  The inverse-logit function, $g^{-1}$ ,
is a function we will refer to as ``expit'', so that $expit(u) =
exp(u)/(1+exp(u))$.

A GLMM is the extension of GLMs to accommodate ``random
effects''. Often this involves adding a normal random effect to the
linear predictor, and so a simple example is:
\[
 \log(\lambda_{i}) = \alpha_{i} + \beta_{1}*x_{i}
\]
where
\[
 \alpha_{i} \sim \mbox{Normal}(\mu,\sigma^{2})
\]
%Many other probability distributions and formulations of the linear
%predictor might be considered.  It is not widely appreicated that
%the link function and
%distribution of the random effect interact directly to affect the
%implied probability distribution of the linear predictor. For the
%Poisson case just considered, $\lambda_{i}$ has a log-normal
%distribution. However, if we set $\lambda_{i} = \alpha_{i}exp(\beta*x_{i})$
%where $\alpha_{i}$ has a Gamma distribution, then $\lambda_{i}$ has
%similarly a gamma distribution with modified scale parameter.  These
%different model assumptions are seldom evaluated formally in practice
%although in many practical situations (in ecology), they imply
%specific things about the ecological process being studied
%(e.g., see \citet{royle_dorazio:2008} section XYZ on occupancy
%logit/cloglog etc..).



\section{Bayesian Analysis}

Bayesian analysis is unfamiliar to many ecological researchers because
older cohorts of ecologists were largely educated in the classical
statistical paradigm of frequentist inference. But advances in
technology and increasing exposure to benefits of Bayesian analysis
are fast making Bayesians out of people or at least making Bayesian
analysis an acceptable, general, alternative to classical, frequentist
inference.

Conceptually, the main thing about Bayesian inference is that it uses
probability directly to characterize uncertainty about things we don't
know.  ``Things'', in this case, are parameters of models and, just as
it is natural to characterize uncertain outcomes of stochastic
processes using probability, it seems natural also to characterize
information about unknown ``parameters'' using probability. At least
this seems natural to us and, we think, most ecologists either
explicitly adopt that view or tend to fall into that point of view
naturally.  Conversely, frequentists use probability in many different
ways, but never to characterize uncertainty about
parameters\footnote{To hear this will be shocking to some readers
  perhaps.} Instead, frequentists use probability to characterize the
behavior of {\it procedures} such as estimators or confidence
intervals (see below), which can lead to some inelegant or unnatural
interpretations of things.  It is paradoxical that people readily
adopt a philosophy of statistical inference in which the things you
don't know (i.e., parameters) should {\it not} be regarded as random
variables, so that, as a consequence, one cannot use probability to
characterize ones state of knowledge about them.


\subsection{Bayes Rule}

As its name suggests, Bayesian analysis makes use of Bayes' rule in
order to make direct probability statements about model
parameters. Given two random variables $z$ and $y$, Bayes rule relates
the two conditional probability distributions $[z|y]$ and $[y|z]$ by
the relationship:
\[
[z|y] = [y|z][z]/[y]
\]
Bayes' rule itself is a mathematical fact and there is no debate in
the statistical community as to its validity and relevance to many
problems. Generally speaking, these distributions are characterized as
follows: $[y|z]$ is the conditional probability distribution of $y$
{\it given} $z$, $[z]$ is the marginal distribution of $z$ and $[y]$
is the marginal distribution of $y$. In the context of Bayesian
inference we usually associate specific meanings in which $[y|z]$ is
thought of as ``the likelihood'', $[z]$ as the ``prior'' and so on. We
leave this for later because here the focus is on this expression of
Bayes rule as a basic fact of probability.

As an example of a simple application of Bayes rule,
consider the problem of determining species presence at a sample
location based on imperfect survey information. Let $z$ be a binary
random variable that denotes species presence $(z=1)$ or absence
$(z=0)$, let $\Pr(z=1) = \psi$ where $\psi$ is usually called
occurrence probability, ``occupancy'' \citep{mackenzie_etal:2002} or ``prevalence''.
Let $y$ be the {\it observed} presence
($y=1$) or absence ($y=0$), and let $p$ be the probability that a
species is detected in a single survey at a site given that it is
present. Thus, $\Pr(y=1|z=1)=p$. The interpretation of this is that,
if the species is present, we will only observe presence with
probability $p$. In addition, we assume here that $\Pr(y=1|z=0) =
0$. That is, the species cannot be detected if it is not present which
is a conventional view adopted in most biological sampling problems (but
see \citet{royle_link:2006}).
If we survey a site $T$ times but never detect the species,
then this clearly does not imply that the species is not present
($z=0$) at this site. Rather, our degree of belief in $z=0$ should be
made with a probabilistic statement
$\Pr(z=1|y_1=0,\ldots,y_{T}=0)$. If the $T$ surveys are independent so
that we might regard $y_{t}$ as $iid$ Bernoulli trials, then the total
number of detections, say $y$, is Binomial with probability $p$ then
we can use Bayes rule to compute the probability that it is present
given that it is not detected in $T$ samples. In words, the expression
we seek is:
\[
\Pr(\mbox{present} | \mbox{not detected}) = \frac{\Pr(\mbox{not detected} |
  \mbox{present})\Pr(\mbox{present})}{\Pr(\mbox{detected})}
\]
Mathematically, this is
\begin{eqnarray*}
\Pr(z=1|y=0) &= &\Pr(y=0|z=1)\Pr(z=1)/\Pr(y=0)  \\
             &= & [(1-p)^{T} \psi]/[ (1-p)^T \psi + (1-\psi) ].
\end{eqnarray*}
To apply this,
suppose that $T=2$ surveys are done at a wetland for a species of
frog, and the species is not detected there. Suppose further that $\psi
= .8$ and $p = .5$ are obtained from a prior study.  Then the
probability that the species is present at this site is
$.25*.8/(.25*.8 + .2) = 0.50$. That is, there seems to be about a
50/50 chance that the site is occupied despite the fact that the
species wasn't observed there.

In summary, Bayes' rule provides a simple linkage between the
conditional probabilities $[y|z]$ and $[z|y]$ which is useful whenever
one needs to deduce one from the other.
Bayes' rule as a basic fact of probability is not disputed.


\subsection{Bayesian Inference}


What is controversial to some is the scope and manner in which Bayes
rule is applied by Bayesian analysts. Bayesian analysts assert that
Bayes rule is relevant, in general, to all statistical problems by
regarding all unknown quantities of a model as realizations of random
variables - this includes ``data'', latent variables, and also
``parameters''. Classical (non-Bayesian) analysts sometimes object to
regarding ``parameters'' as outcomes of random variables. Classically,
parameters are thought of as ``fixed but unknown'' (using the
terminology of classical statistics). Of course, in Bayesian analysis
they are also unknown and, in fact, there is a single data-generating
value and so they are also fixed. The difference is that this fixed
but unknown value is regarded as having been generated from some
probability distribution. Specification of that probability
distribution is necessary to carryout Bayesian analysis, but it is not
required in classical frequentist inference.


To see the general relevance of Bayes rule in the context of
statistical inference, let $y$ denote observations - i.e., ``data'' -
and let $[y|\theta]$ be the observation model (often colloquially
referred to as the ``likelihood'').  Suppose theta is a parameter of
interest having (prior) probability distribution $[\theta]$. These are
combined to obtain the posterior distribution using Bayes' rule, which
is:
\[
 [\theta|y]= [y|\theta][\theta]/[y]
\]
Asserting the general relevance of Bayes rule to all statistical
problems, we can conclude that the two main features of Bayesian
inference are that: (1) ``parameters'' $\theta$ are regarded as realizations of
a random variable and, as a result, (2) inference is based on the
probability distribution of the parameters given the data,
$[\theta|y]$,
which is
called the posterior distribution. This is the result of using Bayes
rule to combine ``the likelihood'' and the prior distribution.  The
key concept is regarding parameters as realizations of a random
variable because, once you admit this conceptual view, this leads
directly to the posterior distribution, a very natural quantity upon
which to base inference about things we don't know -  including
parameters of statistical models.  In particular, $[\theta|y]$ is a
probability distribution for $\theta$ and therefore we can make direct
probability statements to characterize uncertainty about
$\theta$.

The denominator of our invocation of Bayes rule, $[y]$,
is the marginal distribution of the data $y$.  We note without further
remark right now that, in many practical problems, this can be an
enormous pain to compute. The main reason that the Bayesian paradigm
has become so popular in the last 20 years or so is because methods
exist for characterizing the posterior distribution that do not
require that we possess a mathematical understanding of $[y]$, i.e.,
we never have to compute it or know what it looks like, or know
anything specific about it.

A common misunderstanding on the distinction between Bayesian and
frequentist inference goes something like this ``in frequentist
inference parameters are fixed but unknown but in a Bayesian analysis
parameters are random.'' At best this is a sad caricature of the
distinction and at worst it is downright wrong. What is true is that,
to a Bayesian, parameters are random variables. However, a Bayesian
assumes, just like a frequentist, that there was a single
data-generating value of that parameter - a fixed, and unknown value
that produced the given data set.
The distinction between Bayesian and frequentist approaches is that
Bayesians regard the parameter as a random variable, and its value as
the outcome of a random value, on par with the observations. This
allows Bayesians to use probability to make direct probability
statements about parameters. Frequentist inference procedures do not
permit direct probability statements to be made about parameter
values -- because parameters are not random variables!

While we can understand the conceptual basis of Bayesian inference
merely by understanding Bayes rule -- that's really all there is to it
-- it is not so easy to understand the basis of classical
``frequentist'' inference which is mostly
like\footnote{Characterization from Sims REF XYZ} a ``basket of
methods'' with little coherent organization. What is mostly coherent
in frequentist inference is the manner in which items in this basket
of methods are evaluated -- the performance of a given procedure is
evaluated by ``averaging over'' hypothetical realizations of $y$,
regarding the {\it estimator} as a random variable. For example, if
$\hat{\theta}$ is an estimator of $\theta$ then the frequentist is
interested in $E_{y}[\hat{\theta}|y]$ which is used to characterize
bias. If the expected value of $\hat{\theta}$, when averaged over
realizations of $y$, is equal to $\theta$, then $\hat{\theta}$ is
unbiased.

The view of parameters as fixed constants and estimators as random variables
leads to interpretations that are not so straightforward. For
example confidence intervals having the interpretation ``95\%
probability that the interval contains the true value" and p-values
being "the probability of observing an outcome as extreme or more than
the one observed.'' These are far from intuitive interpretations to
most people.  Moreover, this is conceptually probblematic to some
because the hypothetical realizations that characterize the
performance of our procedure we will never get to observe.

While we do tend to favor Bayesian inference for the conceptual
simplicity (parameters are random, posterior inference), we mostly
advocate for a pragamatic non-partisian approach to inference because,
frankly, some of these ``bucket of methods'' are actually very
convenient in certain situations as we will see in later chapters.


\subsection{Prior distributions}


The prior distribution $[\theta]$ is an important feature of Bayesian
inference. As a conceptual matter,
the prior distribution characterizes ``prior beliefs'' or ``prior
information'' about a parameter. Indeed,
an oft-touted benefit of Bayesian analysis is the ease with which
prior information can be included in an analysis.
However, more commonly, the prior is chosen to
express a lack of prior information, even if previous studies have
been done and even if the investigator does in fact know quite a bit
about a parameter.
This is because
the manner in which prior information is embodied in a prior (and the
amount of information) is
usually very subjective and thus the result can wind up being very
contentious, e.g., different investigators might report different
results based on subjective assessments of things. Thus it is usually
better to ``let the data speak'' and use priors that reflect absence
of information beyond the data set being analyzed.

But still the need occasionally arises to embody prior information or
beliefs about a parameter formally into the estimation scheme.
 In SCR models we often have a parameter that is closely linked
to ``home range radius'' and thus auxiliary information on the home
range size of a species can be used as prior information (e.g., see
\citet{chandler_royle:2012} ; also chapter XYZ).

XXXXXXXX
you gonna add something about priors and their potential to truncate posteriors here?
XXXXXXXX

XXXXXXXX

noninformative prior on one scale is informative on another scale.
e.g., flat prior on logit(p) is very different from uniform(0,1) on
p...
show graphic......

reference to non-invariance of prior distributions to transformation......

XXXXXXXX

\subsection{Posterior Inference}

In Bayesian inference, we are not focusing on estimating a single
point or interval but rather on characterizing a whole distribution --
the posterior distribution -- from which one can report any summary of
interest. A point estimate might be the posterior mean, median, mode,
etc..  In many applications in this book, we will compute 95\%
Bayesian intervals using the 2.5\% and 97.5\% quantiles of the
posterior distribution. For such intervals, it is correct to say
$\Pr(L < \theta < U) = 0.95$. That is, "the probability that $\theta$
is between $L$ and $U$ is $0.95$". 

As an
example, suppose we conducted a Bayesian analysis to estimate
detection probability of some species at a study site (p), and we
obtained a posterior distribution of beta(20,10) for the parameter
p. The following R commands demonstrate how we make inferences based
upon summaries of the posterior distribution. Fig. \ref{densityvsdetection.fig} shows the
posterior along with the summary statistics.

\begin{verbatim}
> (post.median <- qbeta(0.5, 20, 10))
[1] 0.6704151
> (post.95ci <- qbeta(c(0.025, 0.975), 20, 10))
[1] 0.4916766 0.8206164
\end{verbatim}

Thus, we can state that there is a 95\% probability that $\theta$ lies
between 0.49 and 0.82.

\begin{figure}
\begin{center}
\includegraphics[height=2.5in]{Ch2/figs/densityvsdetection}
%get figure file from Ch7 folder
\end{center}
\caption{Probability density plot of a hypothetical posterior distribution of beta(20,10); dashed lines indicate mean and upper and lower 95\% interval}
\label{densityvsdetection.fig}
\end{figure}

It is not a subtle thing that this
cannot be said using frequentist methods - although people tend to say
it anyway and not really understand why it is wrong or even that it is
wrong. This is actually a failing of frequentist ideas and the
inability of frequentists to get people to overcome their natural
tendency to use probability - which is something that, as a
frequentist, you simply cannot do in the manner that you would like
to.



Posterior inference is the main practical element of Bayesian
analysis. We get to make an inference conditional on the data that we
actually observed - i.e., what we actually know.  To us, this seems
logical - to condition on what we know. Conversely, frequentist
inference is based on considering average performance over
hypothetical unobserved data sets (i.e., the ``relative frequency''
interpretation of probability).  Frequentists know that their
procedures work well when averaged over all hypothetical, unobserved,
data sets but no one ever really knows how well they work for the
specific data set analyzed. That seems like a relevant question to
biologists who oftentimes only have their one, extremely valuable,
data set.  This distinction comes into play a lot in exposing
philosophical biases in the peer review of statistical analyses in
ecology in the sense that, despite these opposing conceptual views to
inference (i.e. conditional on the data you have, or averaged over
hypothetical realizations), those who conduct a Bayesian analysis are
often (in ecology, almost always) required to provide a frequentist
evaluation of their Bayesian procedure.

\subsection{Small sample inference}

Using Bayesian inference, we obtain an estimate of the posterior
distribution which is an exhaustive summary of the state-of-knowledge
about an unknown quantity. It is the posterior distribution - not an
estimate of that thing. It is also not, usually, an approximation
except to within Monte Carlo error (in cases where we use simulation
to calculate it).  One of the great virtues of Bayesian analysis which
is not really appreciated is that it is completely valid for any
particular sample size. i.e., it is $[\theta|y]$, as precise as we
claim it to be based on our ability to do calculations, for the
particular sample size and observations that we have even if we have
only a single datum $y$.  The same cannot be said for almost all
frequentist procedures in which estimates or variances are very often
(almost always in practice) based on ``asymptotic approximations'' to
the procedure which is actually being employed.

There seems to be a prevailing view in statistical ecology that
classical likelihood-based procedures are virtuous because of the
availability of simple formulas and procedures for carrying out
inference, such as calculating standard errors, doing model selection
by AIC, and assessing goodness-of-fit.  In large samples, this may be
an important practical benefit, but the theoretical validity of these
procedures cannot be asserted in most situations involving small
samples.  This is not a minor issue because it is typical in many
wildlife sampling problems - especially in surveys of carnivores or
rare/endangered species - to wind up with a small, sometimes extremely
small, data set. For example, a recent paper on the fossa
(Cryptoprocta ferox), an endangered carnivore in Madagascar, estimated
an adult density of 0.18 adults / km sq based on 20 animals captured
over 3 years \citep{hawkins_racey:2005}. A similar paper on the
endangered southern river otter (Lontra provocax) estimated a density
of 0.25 animals per river km based on 12 individuals captured over 3
years \citep{sepulveda_etal:2007}. \citet{gardner_etal:2010} analyzed
data from a study of the Pampas cat, a species for which very little
is known, wherein only 22 individual cats were captured .during the
two year period.  \citet{trolle_kery:2005} reported only 9 individual
ocelots captured and \citet{jackson_etal:2006} captured 6 individual
snow leopards using camera trapping. Thus, studies of rare and/or
secretive carnivores necessarily and flagrantly violate one of Le
Cam's Basic Principles, that of ``If you need to use asymptotic
arguments, do not forget to let your number of observations tend to
infinity.''\citep{lecam:1990}.

The biologist thus faces a dilemma with such data. On one hand, these
datasets, and the resulting inference, are often criticized as being
poor and unreliable. Or, even worse\footnote{Actual quote from a
  referee}, ``the data set is so small, this is a poor analysis.''  On
the other hand, such data may be all that is available for species
that are extraordinarily important for conservation and management.
The Bayesian framework for inference provides a valid, rigorous, and
flexible framework that is theoretically justifiable in arbitrary
sample sizes. This is not to say that one will obtain precise
estimates of density or other parameters, just that your inference is
coherent and justifiable from a conceptual and technical statistical
point of view. That is, we report the posterior probability
$\Pr(D|data)$ which is easily interpretable and just what it is
advertised to be and we don't need to do a simulation study to
evaluate how well some approximate $\Pr(D|data)$ deviates from the
actual $\Pr(D|data)$ because they are precisely the same quantity.



\section{Characterizing posterior distributions by MCMC simulation}

In practice, it is not really feasible to ever compute the marginal
probability distribution $\Pr(y)$, the denominator resulting from
application of Bayes' rule. For decades this impeded the adoption of
Bayesian methods by practitioners. Or, the few Bayesian analyses done
were based on asymptotic normal approximations to the posterior
distribution. While this was useful stuff from a theoretical and
technical standpoint and, practically, it allowed people to make the
probability statements that they naturally would like to make, it was
kind of a bad joke around the Bayesian water-cooler to, on one hand,
criticize classical statistics for being, essentially, completely ad
hoc in their approach to things but then, on the other hand, have to
devise various approximations to what they were trying to
characterize. The advent of Markov chain Monte Carlo (MCMC) methods
has made it easier to calculate posterior distributions for just about
any problem to arbitrary levels of precision.

Broadly speaking, MCMC is a class of methods for drawing random
numbers (sampling or simulating) from the target posterior
distribution.  Thus, even though we might not recognize the posterior
as a named distribution or be able to analyze its features
analytically, e.g., devise mathematical expressions for the mean and
variance, we can use these MCMC methods to obtain a large sample from
the posterior and then use that sample to characterize features of the
posterior. What we do with the sample depends on our intentions --
typically we obtain the mean or median for use as a point estimate,
and take a confidence interval based on Monte Carlo estimates of the
quantiles.  These are estimates, but not like frequentist
estimates. Rather, they are Monte Carlo estimates with an associated
Monte Carlo error which is largely determined arbitrarily by the
analyst. They are not estimates qualified by a sampling distribution
as in classical statistics. If we run our MCMC long enough then our
reported value of $E[\theta|y]$ or any feature of the posterior
distribution is precisely what we say it is. There is no ``sampling
variation'' in the frequentist sense of the word.  In summary, the
MCMC samples provide a Monte Carlo characterization of {\it the}
posterior distribution.


\section{What Goes on Under the MCMC Hood}

We will develop and apply MCMC methods in some detail for spatial
capture-recapture models in chapter \ref{chapt.mcmc}. Here we provide
a simple illustration of some basic ideas related to the practice of MCMC.

A type of MCMC method relevant to most problems is Gibbs sampling (REF
XYZ XYZ),
which is based on the idea of iterative simulation from the ``full
conditional'' distributions (also called conditional posterior
distributions). The full conditional distribution for an unknown
quantity is the conditional distribution of that quantity given every
other random variable in the model - the data and all other
parameters. For example, for a normal regression model with $y \sim
\mbox{Normal}(\alpha + \beta x , 1)$ then the two full conditionals are, in
symbolic terms,
\[
[\alpha|y,\beta]
\]
 and
\[
[\beta|y,\alpha].
\]
We might use our knowledge of probability to identify these
mathematically. In particular, by Bayes' Rule, $[\alpha|y,\beta] =
[y|\alpha,\beta][\alpha|\beta]/[y|\beta]$ and similarly for
$[\beta|y,\alpha]$. For example, if we have priors for $[\alpha]$ and $[\beta]$
which are also normal distributions, some algebra reveals that
XXXX COPY NOTATION FFROM CH. 6 XXXXX
\[
[\alpha|y,\beta] = Normal(ybar,...weighted variance here...).
\]
Similarly,
\[
 [\beta|y,\alpha] is normal(........)
\]

The MCMC algorithm for this model has us simulate in succession,
repeatedly, from those two distributions. See \citet{gelman_etal:2004}
for more examples of Gibbs sampling for the normal model. A
conceptual representation of the MCMC algorithm for this simple model
is therefore:
XXXX Check out ALGORITHM environment XXXXX
\begin{verbatim}
 Algorithm

       0. Initialize $\alpha$ and $\beta$

       Repeat{
           1. Draw a new value of $\alpha$ from Eq. \ref{xyz}

           2. Draw a new value of $\beta$ from Eq. \ref{xyz}
       }
\end{verbatim}

As we just saw for this simple ``normal-normal'' model it is sometimes
possible to specify the full conditional distributions
analytically. In general, when certain so-called conjugate prior
distributions are chosen, the form of full conditional distributions
is similar to that of the observation model. In this normal-normal
case, the normal distribution for the mean parameters is the conjugate
prior under the normal model, and thus the full-conditional
distributions are also normal. This is convenient because, in such
cases, we can simulate directly from them using standard methods (or
{\bf R}
functions).  But, in practice, we don't really ever need to know such
things because most of the time we can get by using a simple
algorithm, called the Metropolis-Hastings (henceforth ``MH'')
algorithm, to obtain samples from these full conditional distributions
without having to recognize them as specific, named, distributions.
This gives us enormous freedom in developing models
and analyzing them without having to resolve them mathematically
because to implement the MH algorithm we need only identify the full
conditional distribution up to a constant of proportionality, that
being the marginal distribution in the denominator (e.g., $[y|\beta]$
above).

We will talk about the Metropolis-Hastings algorithm shortly, and we
will use it extensively in the analysis of SCR models (e.g., chapter
\ref{chapt.mcmc}).

\subsection{Rules for constructing full conditional distributions}
\label{glms.sec.rules}

The basic strategy for constructing full-conditional distributions for
devising MCMC algorithms can be reduced conceptually to a couple of
basic steps summarized as follows:
\begin{itemize}
\item[(step 1)] Collect all stochastic components of the model;
\item[(step 2)] Recognize and express the full conditional in question
  as proportional to the product of all components;
\item[(step 3)] Remove the ones that don't have the focal parameter in them.
\item[(step 4)] Do some algebra on the result in order to identify the resulting pdf or pmf.
\end{itemize}
Of the 4 steps, the last of those is the main step that requires quite
a bit of statistical experience and intuition because various
algebraic tricks can be used to reshape the mess into something
noticeable - i.e., a standard, named distribution. But step 4 is not
necessary if we decide instead to use the Metrpolis-Hastings algorithm
as described below.

To illustrate for computing $[\alpha|y,\beta]$ we first apply step 1
and identify the model components as: $[y|\alpha, \beta]$, $[\alpha]$
and $[\beta]$. Step 2 has us write $[\alpha|y,\beta] \propto
[y|\alpha,\beta][\alpha][\beta]$.  Step 3: We note that $[\beta]$ is not a
function of alpha and therefore we remove it to obtain $[\alpha|y,\beta]
\propto [y|\alpha,\beta][\alpha]$. Similarly we obtain $[\beta|y,\alpha]
\propto [y|\alpha,\beta][\beta]$. We apply step 4 and manipulate
these algebraically to arrive at the result or, alternatively, we can
sample them indirectly using the Metropolis-Hastings algorithm (see
below).


\subsection{Metropolis-Hastings algorithm}

The Metropolis-Hastings (MH) algorithm is a completely generic method for
sampling from any distribution, say $f(\theta)$. In our applications,
$f(\theta)$ will typically be the full conditional distribution of
$\theta$.
While we sometimes use Gibbs sampling, we seldom
use ``pure'' Gibbs sampling because we might use MH to sample from one
or more of the full conditional distributions.
When the MH algorithm is used to sample from  full
conditional distributions of a Gibbs sampler the resulting hybrid algorithm is
called
 {\it Metrpolized Gibbs sampling} or
more commonly {\it Metropolis-within-Gibbs}.
Shortly we will
actually construct such an algorithm for a simple class of models.

The MH algorithm generates candidates from some
proposal or candidate-generating distribution, that may be conditional
on the current value of the parameter, denoted by
$h(\theta^{*}|\theta^{t})$. Here, $\theta^{*}$ is the {\it candidate}
or proposed
value and $\theta^{t}$ is the current value, i.e., at iteration $t$ of
the MCMC algorithm.  The proposed value
is accepted with probability

\[
r = \frac{ f(\theta^{*}) h(\theta^{t}|\theta^{*})}
    {f(\theta^{t}) h(\theta^{*}|\theta^{t}) }
\]
which we call the MH acceptance probability.
This ratio can sometimes be $>1$ in which case we set it equal to
1. It is useful to note that $h()$ can be anything at all. No matter
the choice of $h()$, we can evaluate this ratio numerically because
the marginal $f(y)$ cancels from both the numerator and
denominator, which is the magic of the MH algorithm.


\section{Practical Bayesian Analysis and MCMC}

There are a number of really important practical issues to be
considered in any Bayesian analysis and we cover some of these briefly
here.

\subsection{Choice of prior distributions}

{\bf XXX integrate this material with previous section on prior
distributions XXXXXX}

Bayesian analysis requires that we choose prior
distributions for all of the structural parameters of the model (we
use the term structural parameter to mean all parameters that aren't
customary thought of as latent variables). We will strive to use
priors that are meant to express little or no prior information -
default or customary ``non-informative'' or diffuse priors. This will
be $\mbox{Unif}(a,b)$ priors for parameters that have a natural
bounded support and, for parameters that live on the real line we use
either (1) diffuse normal priors; (2) ``improper'' uniform priors or
(3) sometimes even a bounded $\mbox{Unif}(a,b)$ prior if that greatly
improves the performance of {\bf WinBUGS} or other software doing the MCMC
for us.  In {\bf WinBUGS} a prior with low ``precision'', $\tau$, where
$\tau = 1/\sigma^2$, such as $\mbox{Norm}(0,.01)$ will typically be
used. Of course $\tau = 0.01$ ($\sigma^{2} = 100$) might be very
informative for a regression parameter that has a high
variance. Therefore, we recommend that predictor variables {\it
  always} be standardized. Clearly there are a lot of choices for
ostensibly non-informative priors, and the degree of
non-informativeness depends on the parameterization. For example, a
natural non-informative prior for the intercept of a logistic
regression
\[
\mbox{logit}(p_{i}) = \alpha + \beta x_{i}
\]
Would be $[\alpha] = \mbox{const}$ which is the same as saying $a \sim
Unif(\infty,infty)$, the customary improper uniform prior.
However, we might also use a prior on the parameter $p0
= logit^{-1}(a)$, which is $Pr(y=1)$ for the value $x=0$. Since $p0$ is a
probability a natural choice is $p0 \sim Unif(0,1)$. These two priors can
affect results (see Chapter 3.XYZ), yet they are both sensible
non-informative priors. Choice of priors and parameterization is
very much problem-specific and often largely subjective. Moreover, it
also affects the behavior of MCMC algorithms and therefore the analyst
needs to pay some attention to this issue and possibly try different
things out.
XXX REFS on prior distributions XXXXXX

\subsection{Convergence and so-forth}

Once we have carried-out an analysis by MCMC, there are many other
practical issues that we have to confront.  One of the most important
is ``have the chains converged?'' Since we do not know what the stationary posterior distribution of our Markov chain should look like (this is the whole point of doing an MCMC approximation), we effectively have no means to assess whether it has truly converged to this desired distribution or not. Most MCMC algorithms only guarantee
that, eventually, the samples being generated will be from the target
posterior distribution, but no-one can tell us how long this will take. Also, you only now the part of your posterior distribution that the Markov chain has explored so far– for all you know the chain could be stuck in a local maximum, while other maxima remain completely undiscovered.  Acknowledging that there is truly nothing we can do to ever proof convergence of our MCMC chains, there are several things we can do to increase the degree of confidence we have about the convergence of our chains. Some problems are easily detected using simple plots.  Typically a period of transience is
observed in the early part of the MCMC algorithm, and this is usually
discarded as the ``burn-in'' period. The quick diagnostic to whether convergence has been achieved is that
your Markov chains look ``grassy'' -- see Fig.  \ref{glms.fig.grassy}
below.  
Another way to check
convergence is to update the parameters some more and see if the
posterior changes. Yet another option, and one generally implemented in WinBUGS, is to run several Markov chains and to start them off at different initial values that are overdispersed relative to the posterior distribution. Such initial values help to explore different areas of the parameter space simultaneously; if after a while all chains oscillate around the same average value, chances are good that they indeed converged to the posterior distribution. \footnote{Running several parallel chains is computationally expensive. But extra computational demands are not the only and by no means the major concern some people voice when it comes to running several parallel MCMC chains to assess convergence. Again, consider the fact that we do not know anything about the true form of the posterior distribution we are trying to approximate. How do we, then, know how to pick overdispersed initial values? We don’t – all we can do is pick overdispersed values relative to our expectations of what the posterior should look like. To use a quote from the home page of Charlie Geyer, a Bayesian statistician from the University of Minnesota, ``If you don't know any good starting points [...], then restarting the sampler at many bad starting points is [...] part of the problem, not part of the solution.'' (http://users.stat.umn.edu/~charlie/mcmc/diag.html). His suggestion is that your only chance to discover a potential problem with your MCMC sampler is to run it for a very long time. But again, there is no way of knowing how long is long enough.
It is up to you to decide, which school of thoughts appeals more to you – one long versus several parallel Markov chains. Irrespectively, part of developing an MCMC sampler should be to make sure (within reasonable limits) that you are not missing regions of high posterior density because of the way you specify your starting values. Once you have explored the behavior of your chain under a – reasonable – range of starting values, you may feel comfortable enough to run only one long chain.} Gelman and Rubin came up with the so-called``R-hat'' statistic ($\hat{R}$) or Brooks-Gelman-Rubin statistic
 that essentially compares within-chain and between-chain variance to check for convergence of multiple chains (\citep{gelman_etal:1996}). $\hat{R}$ should be close to 1 if the Markov
chains have converged and sufficient posterior samples have been
obtained. In practice, $\hat{R} = 1.2$ is probably good enough for
some problems.  For some models you can't actually realize a low
$\hat{R}$. E.g., if the posterior is a discrete mixture of distributions
then you can be misled into thinking that your Markov chains
have not converged when in fact the chains are just jumping back and
forth in the posterior state-space. 
So, for example, model
selection methods (section XYZ) sometimes suggests non-convergence.
Another situation is when one of the parameters is on the boundary of
the parameter space which might appear to be very poor mixing, but all
within some extreme region of the parameter space.\footnote{it would
  be nice if we could compile examples of this later in the book and
  reference back to this point}.
This
kind of stuff is normally ok and you need to think really hard about
the context of the model and the problem before you conclude that your
MCMC algorithm is ill-behaved.

Some models exhibit ``poor mixing'' of the Markov chains or what
people might also say ``have not coverged'' (or ``slow convergence'')
which is a term we would disagree with because the samples might well
be from the posterior (i.e., the Markov chains have converged to the
proper stationary distribution) but simply mix around the posterior
rather slowly. Anyway, poor mixing can happen for a huge number of
reasons -- when parameters are highly correlated (even confounded), or
barely identified from the data, or the algorithms are very terrible
and probably many other reasons.  Slow mixing equates to high
autocorrelation in the Markov chain - the successive draws are highly
correlated, and thus we need to run the MCMC algorithm much longer to
get an effective sample size that is sufficient for estimation - or to
reduce the MC error to a tolerable level.  A strategy often used to
reduce autocorrelation is ``thinning'' - i.e., keep every $m^{th}$
value of the Markov chain output. However, thinning is necessarily
inefficient from the stand point of inference - you can always get
more precise posterior estimates by using all of the MCMC output
regardless of the level of autocorrelation
\citep{maceachern_berliner:1994}. Practical considerations might
necessitate thinning, even though it is statistically inefficient. For
example, in models with many parameters or other unknowns being
tabulated, the output files might be enormous and unwieldy to work
with. In such cases, thinning is perfectly reasonable. In many cases,
how well the Markov chains mix is strongly influenced by
parameterization, standardization of covariates, and the prior
distributions being used. Some things work better than others, and the
investigator should experiment with different settings and
remain calm when things don't work out perfectly. MCMC is an
art, and a science.


{\bf Is the posterior sample large enough?}  The subsequent samples generated from a Markov chain are not iid samples from the posterior distribution, due to the correlation amongst samples introduced by the Markov process and the sample size has to be adjusted to account for the autocorrelation in subsequent samples (see Chapter 8 in \citet{robert_casella:2010} for more details). This adjusted sample size is referred to as the effective sample size. Checking the degree of autocorrelation in your Markov chains and estimating the effective sample size your chain has generated should be part of evaluating your model output. WinBUGS will automatically return the effective sample size for all monitored parameters. If you find that your supposedly long Markov chain has only generated a very short effective sample, you should consider a longer run. What exactly constitutes a reasonable effective sample size is hard to say. A more palpable measure of whether you've run your chain for enough iterations is the time-series or Monte Carlo error – the 'noise' introduced into your samples by the stochastic MCMC process. The MC error is printed by default in
summaries of BUGS output. You want that to be smallish relative to the
magnitude of the parameter and this might depend on the purpose of the
analysis. For a preliminary analysis you might settle for a few
percent whereas for a final analysis then certainly less than 1\% is
called for, but you can run
your MCMC
algorithm as long as it takes. A consequence of the MC error is that even for the exact same model, results will always be different. Thus, as a good rule
of thumb you should never report
MCMC results to more than 2 decimal places.
Note that MC error in summaries of the
posterior is not the same as having an ``approximate'' solution in a
standard likelihood analysis or similar.  The approximate SE in
likelihood inference is actually wrong in its actual value.... XYZ.


\subsection{Bayesian confidence intervals}

The 95\% Bayesian interval based on percentiles of the posterior
is not a unique interval - there are many of them - and the so-called
``highest posterior density'' (HPD) interval is the narrowest
interval. We might compute that frequently because it is easy to do
with an integer parameter which $N$ is (See the next chapter). The
95 \% HPD is not often exactly 95\% but usually slightly more
conservative than nominal because it is the narrowest interval that
contains at least 95\%  of the posterior mass.

\subsection{Estimating functions of parameters}

A benefit of analysis by MCMC is that we can seamlessly estimate
functions of parameters by simply tabulating the desired function of
the simulated posterior draws. For example, if $\theta$ is the
parameter of interest and let $\theta^{(i)}$ for $i=1,2,\ldots,M$ be
the posterior samples of $\theta$. Let $\eta = exp(\theta)$, then a
posterior sample of $\eta$ can be obtained simply by computing
$exp(\theta^{(i)})$ for $i=1,2,\ldots,M$. We give another example in
section
\ref{glms.sec.xopt}
below and throughout this book.
Almost all SCR models in this book involve at least 1 derived
parameter. For example, density $D$ is a derived parameter, being a
function of population size $N$ and the area $A$ of the underlying
state-space of the point process (see chapter \ref{chapt.scr0}).

\section{Bayesian Analysis using WinBUGS}

We won't be too concerned with devising our own MCMC algorithms for
every analysis
although we will do that a few times for fun.  More often, we
will rely on the freely available software package {\bf WinBUGS} or
{\bf JAGS}
for doing this.  We will always execute these {\bf BUGS} engines from
within {\bf R} using the \mbox{\tt R2WinBUGS} (REF XYZ XYZ) or
\mbox{\tt rjags} pacages. {\bf WinBUGS} and {\bf JAGS} are  MCMC black boxes
that takes a pseudo-code description (i.e., written in the {\bf BUGS}
language) of all of the relevant stochastic
and deterministic elements of a model and generates an MCMC algorithm
for that model. But you never get to see the algorithm. Instead,
{\bf WinBUGS}/{\bf JAGS} will run the algorithm and just return the Markov chain output
- the posterior samples of model parameters.

The great thing about using the {\bf BUGS} language is that it forces
you to become intimate with your statistical model - you have to write
each element of the model down, admit (explicitly) all of the various
assumptions, understand what the actual probability assumptions are
and how data relate to latent variables and data and latent variables
relate to parameters, and how parameters relate to one another.

While we normally use
{\bf WinBUGS} or {\bf JAGS} in this book, we note that {\bf
 OpenBUGS} is the current active development tree of the {\bf BUGS}
language. See \citet[][ch.xyz]{kery:2010} and
\citet[][appendix xyz]{kery_schaub:2011} for more on practial analysis
in {\bf WinBUGS}.
That book should also be consulted
for a more comprehensive introduction to using {\bf WinBUGS}. In this
example, we're going to accelerate pretty fast.

\subsection{Linear Regression in WinBUGS}

We provide a brief introductory example of a normal regression model
using a small simulated data set. The following commands are executed
from within your R workspace, the command line being indicated by
\mbox{\tt ``>''}. First, simulate a covariate $x$ and observations $y$ having
prescribed intercept, slope and variance:
\begin{verbatim}
> x<-rnorm(10)
> mu<- -3.2+ 1.5*x
> y<-rnorm(10,mu,sd=4)
\end{verbatim}
The {\bf BUGS} model specification for a normal regression model is
written within {\bf R} as a character string input to the command
\mbox{\tt cat()} and
then dumped to a text file named \mbox{\tt normal.txt}:
\begin{verbatim}
> cat("
model {
   for (i in 1:10){
      y[i]~dnorm(mu[i],tau)        # the "likelihood"
      mu[i]<- beta0 + beta1*x[i]   # the linear predictor
     }
   beta0~dnorm(0,.01)              # prior distributions
   beta1~dnorm(0,.01)
   sigma~dunif(0,100)
   tau<-1/(sigma*sigma)            # tau is a derived parameter
}
",file="normal.txt")
\end{verbatim}
Alternatively, you
can write the model specifications directly within a text file and
save it in your current working directory, but we do not usually take
that approach in this book.

{\bf Remarks:} {\bf 1.} {\bf WinBUGS} parameterizes the normal in
terms of the mean and inverse-variance, called the precision. Thus,
\mbox{\tt dnorm(0,.01)} implies a variance of 100;
{\bf 2.} We typically use diffuse normal priors for mean parameters, $\beta_0$ and $\beta_1$ in this case, but sometimes we might use uniform priors with suitable bounds -B and +B.
{\bf 3.} We typically use a $\mbox{Unif}(0,B)$ prior on standard
deviation parameters
(Gelman XXX 2006 XXXX). But sometimes we might use a gamma prior on the precision parameter $\tau$.
{\bf 4.} In a {\bf WinBUGS} model file, every variable referenced in
the model description has to be
either data, which will be input (see below), a random variable which
must have a probability distribution associated with it using the
``\verb#~#'', or it has to be a derived parameter connected to variables and
data using ``\mbox{\tt <-}''.


To fit the model, we need to describe various data objects to {\bf
  WinBUGS}. In particular,
we create an {\bf R} list object called \mbox{\tt data} which
are the data objects identified in the BUGS model file.
 In the example, the
data consist of two objects which exist as $y$ and $x$ in the {\bf R}
workspace and also in the {\bf WinBUGS} model definition.
 We also have to create an {\bf R} function
that produces a list of starting values \mbox{\tt inits} that get sent to
{\bf WinBUGS}.
 Finally, we identify
the names of the parameters (labeled correspondingly in the {\bf WinBUGS}
model specification) that we want {\bf WinBUGS} to save the MCMC output
for. In this example, we will ``monitor'' the parameters
$\beta_0$, $\beta_1$, $\sigma$ and $\tau$.
{\bf WinBUGS} is executed using the {\bf R} command
\mbox{\tt bugs()}.
We set the option \mbox{\tt debug=TRUE} if we want the {\bf WinBUGS}
GUI to stay open (useful for analyzing MCMC output and looking at the
{\bf WinBUGS} error log). Also, we set \mbox{\tt working.dir=getwd()}
so that {\bf WinBUGS} output files and the log file are saved in the
current {\bf R} working directory.
  All of these activities look like this:
{\small
\begin{verbatim}
 library("R2WinBUGS")    # "attach" the R2WinBUGS library
 data <- list ( "y","x")
 inits <- function()
  list ( beta1=rnorm(1),beta0=rnorm(1),sigma=runif(1,0,2) )
 parameters <- c("beta0","beta1","sigma","tau")
 out<-bugs (data, inits, parameters, "normal.txt", n.thin=2, n.chains=2,
             n.burnin=2000, n.iter=6000, debug=TRUE,working.dir=getwd())
\end{verbatim}
}

{\bf Remarks:} A common question is ``how should my data be
formatted?'' That depends on how you describe the model in the {\bf
  BUGS} language, how your data are input into {\bf R} and
subsequently formatted.  There is no unique way to describe any
particular model and so you have some flexibility. We talk about data
format further in the context of capture-recapture models and SCR
models in chapter \ref{chapt.scr0} and elsewhere.  In general,
starting values are optional but we recommend to always provide
reasonable starting values for structural parameters, but are not
always necessary for random effects.  Note that the previously created
objects defining data, initial values and parameters to monitor are
passed to the function \mbox{\tt bugs()}.  In addition, various other
things are declared: The number of Markov chains (\mbox{\tt
  n.chains}), the thinning rate (\mbox{\tt n.thin}),
the number of burn-in iterations (\mbox{\tt n.burnin}) and the total
number of iterations
(\mbox{\tt n.iter}).
To develop a detailed understanding of the various parameters and
settings used for MCMC, consult a basic reference such as
\citet{kery:2010}.



You should execute all of the commands given above and then look at
the resulting output. Kill the {\bf WinBUGS} GUI and the data will be
read back into {\bf R} (or specify \mbox{\tt debug=FALSE}).  We don't
want to give instructions on how to navigate and use the GUI - see XYZ
REF (XYZ) for that.
The object \mbox{\tt out} prints important
summaries by default (this is slightly edited):

{\small
\begin{verbatim}
> print(out,digits=2)
Inference for Bugs model at "normal.txt", fit using WinBUGS,
 2 chains, each with 6000 iterations (first 2000 discarded), n.thin = 2
 n.sims = 4000 iterations saved
          mean   sd  2.5%   25%   50%   75% 97.5% Rhat n.eff
beta0    -2.43 1.84 -6.21 -3.50 -2.42 -1.34  1.27    1  4000
beta1     2.62 1.54 -0.42  1.68  2.62  3.57  5.67    1  4000
sigma     5.29 1.66  3.11  4.14  4.95  6.05  9.39    1  4000
tau       0.05 0.02  0.01  0.03  0.04  0.06  0.10    1  4000
deviance 59.85 3.24 56.18 57.47 59.00 61.37 68.32    1   840

For each parameter, n.eff is a crude measure of effective sample size,
and Rhat is the potential scale reduction factor (at convergence, Rhat=1).

DIC info (using the rule, pD = Dbar-Dhat)
pD = 2.6 and DIC = 62.4
\end{verbatim}
}

{\bf Remarks:} (1) convergence is assessed using the $\hat{R}$
statistic -- which we might sometimes write ``$Rhat$''. A value of $Rhat$ near 1
indicates convergence; (2) DIC is the
``deviance information criterion'' \citep{spiegelhalter_etal:2002}
(see section \ref{glms.sec.modsel})
 which
some people use in a manner similar to AIC although it is recognized
to have some problems in hierarchical models \citep{millar:2009}. We
evaluate this in the context of SCR models in chapter XYZ XYZ.

\subsection{Inference about functions of model parameters}
\label{glms.sec.xopt}

Using the MCMC draws for a given model we can easily obtain the
posterior distribution of any function of model parameters.  We showed
this in the above example by providing the posterior of $\tau$ when
the model was parameterized in terms of standar deviation $\sigma$.
 As another example, suppose that the
normal regression model above had a quadratic response function of the
form
\[
	E(y_i) = \beta_0 + \beta_1 x_i + \beta_2 x_{i}^{2}
\]
Then the optimum value of $x$, i.e., that corresponding to the optimal
expected response, can be found by setting the derivative of
this function to 0 and solving for $x$. We find that
\[
df/dx = \beta_1 +
2*\beta_2 x = 0
\]
yields that $x_{opt} = -\beta_1/(2*\beta_2)$.  We can just
take our posterior draws for $beta_1$ and $beta_2$ and obtain a
posterior sample of $x_{opt}$ by this simple calculation. As an exercise, take
the normal model above and simulate a quadratic response and then
describe the posterior distribution of $x_{opt}$.


\section{Model Checking and Selection}
\label{glms.sec.modsel}

In general terms model checking - or assessing the adequacy of the
model - and model selection are quite thorny issues and, despite
contrary and, sometimes, strongly held belief among practitioners, there are not
really definitive, general solutions to either problem. We're against
dogma on these issues and think people need to be open-minded about
such things and recognize that models can be useful whether or not
they pass certain statistical tests. Some models are intrinsically
better than others because they make more biological sense or foster
understanding or achieve some objective that some  bootstrap
or other goodness-of-fit test can't decide for you. That said, it
gives you some confidence if your model seems adequate and we try to
provide some fit assessment in most real applications of SCR models
We provide a very brief overview of concepts here, but provide more
detailed coverage in chapter \ref{chapt.gof}.
See also
\citet[][ch. xyz]{kery:2010} and
\citet[][ch. xyz]{link_barker:2009}
for specific context related to Bayesian
model checking and selection.

\subsection{Goodness-of-fit}

Goodness-of-fit testing is an important element of any analysis
because  our model represents a general set of hypotheses
about the ecological and observation processes that generated our
data. Thus, if our model ``fits'' in some statistical or scientific
sense, then we believe it to be consistent with the hypotheses that
went into the model. More formally, we would conclude that the data
are {\it not inconsistent} with the hypotheses, or that the model
appears adequate. If we have enough
data, then of course we will reject any set of statistical hypotheses.
Conversely, we can always come up with a model that fits by making the
model extremely complex. Despite this paradox, it seems to us that
simple models that you can understand should usually be preferred even
if they don't fit, for example if they embody essential mechanisms
central to our understanding of things, or
if we think that some contributing factors to lack-of-fit are minor or
irrelevant to the scientific context and intended use of the model.
In other words, models can be useful irrespective of whether they fit
according to some formal statistical test of fit.  Yet
the tension is there to obtain fitting models, and this comes naturally at
the expense of models that can be easily interpreted and studied and
effectively used.
Moreover, conducting goodness-of-fit tests is
not always so easy to do. Moreover, it is never really easy (or
especially convenient) to decide if your goodness-of-fit test is worth
anything. It might have 0 power!
Despite this,
we recommend attempting to assess model fit in real applications,
as a general rule, and we provide some basic guidance here and some more
specific to SCR models in
chapter \ref{chapt.gof}.

To evaluate goodness-of-fit in Bayesian analyses, we will most often
use the Bayesian p-value \citep{gelman_etal:1996}.  The basic idea is to define
a fit statistic or ``discrepancy measure'' and compare the posterior distribution of that
statistic to the posterior predictive distribution of that statistic
for hypothetical perfect data sets for which the model is known to be correct. For
example, with count frequency data, a standard measure of fit is the
sum of squares of the ``Pearson residuals'',
\[
D(y_i,\theta) = \frac{(y_i - E(y_i))^{2}}{Var( y_{i} )}
\]
The fit statistic based on the squared residuals is
\[
FIT = \sum_{i} D(y_{i},\theta)^{2}
\]
which can be computed at each iteration of a MCMC algorithm given the
current values of parameters that determine the
 response distribution.  At the same time (i.e., at each MCMC
 iteration),
the equivalent statistic is computed for a
``new'' data set, simulated using the current parameter values. The
Bayesian p-value is simply the posterior probability $\Pr(\mbox{Fit} >
\mbox{Fit}_{new})$\footnote{Check this definition!}
 which should be close to $0.50$ for a good model -- one that
 ``fits'' in the sense that the observed data set is
 consistent with realizations simulated under the model being fitted
 to the observed data. In practice
we judge ``close to 0.50'' as being ``not too close to 0 or 1'' and,
as always, closeness is somewhat subjective. We're happy with anything
$>.1$ and $<.9$ but might settle for $>.05$ and $<0.95$. In summary,
the Bayesian p-value seems like a bootstrap idea, is easy to compute,
and widely used as a result.

Another useful fit statistic is the Freeman-Tukey
statistic\footnote{Ref for this?}, in which
\[
D({\bf y},\theta) = \sum_{i} ( \sqrt(y_{i}) - \sqrt(e_{i}) )^2
\]
\citep{brooks_etal:2000}, where $y_{i}$ is the observed value of
observation $i$ and $e_{i}$ its expected value. In contrast to a
chi-square discrepancy, the Freeman-Tukey statistic removes the need
to pool cells with small expected values.


\subsection{Model Selection }

For model selection we typically use three different methods: First
is, let's say, common sense. If a parameter has posterior mass
concentrated away from 0 then it seems like it should be regarded as
important - that is, it is ``significant.''  This approach seems to
have fallen out of favor with all of the interest over the last 10 or
15 years on model selection in ecology. It seems reasonable to us.


For regression problems we sometimes use the factor weighting idea
which is to introduce a set of binary variables $w_{k}$ for variable
$k$, and express the model as, e.g., for a single covariate model:
 \[
 E(y_i) = \alpha + w \beta x_{i}
\]
where $w$ is given a Bernoulli prior distribution with some prescribed
probability. E.g., $w \sim Bern(0.50)$ to provide a prior probability
of 0.50 that variable $x$ should be an element of the linear
predictor. The posterior probability of the event $w=1$ is a gauge of
the importance of the variable $x$. i.e., high values of $\Pr(w=1)$
indicate stronger evidence to support that ``$x$ is in the model''
whereas values of $\Pr(w=1)$ close to 0 suggest that $x$ is less
important.



This idea seems to be due to \citet{kuo_mallick:1998}\footnote{ Is
  this also what people call Zellner's G-priors?} and see
\citet[][ch. XXXX]{royle_dorazio:2008} for an example in the context
of logistic regression. This approach seems to even work sometimes
with fairly complex hierarchical models of a certain form. E.g.,
\citet{royle:2008} applied it to a random effects model to evaluate
the importance of the random effect component of the model.  The main
problem with this approach is that its effectiveness and results will
typically be highly sensitive to the prior distribution on the
structural parameters (e.g., see \citet[][table xyz]{royle_dorazio:2008}).
The reason for this is obvious: If $w = 0$ for the current
iteration of the MCMC algorithm, so that $\beta$ is sampled from the
prior distribution, and the prior distribution is very diffuse, then
extreme values of $\beta$ are likely. Consequently, when the current value of
$\beta$ is
far away from the mass of the posterior when $w=1$, then the Markov
chain may only jump from $w=0$ to $w=1$ infrequently.  One seemingly
reasonable solution to this problem (Aitken XYZ FIND THIS
XXXXX\footnote{see Royle 2008 paper for reference}) is to fit the full
model to obtain posterior distributions for all parameters, and then
use those as prior distributions in a ``model selection'' run of the
MCMC algorithm.  This seems preferable to more-or-less arbitrary restriction of
the prior support to improve the performance of the MCMC algorithm.

A third method that that we advocate is subject-matter
context. It seems that there are some situations -- some models -- where one should not
have to do model selection because it is necessitated by the specific
context of the problem, thus rendering a formal hypotesis test pointless
\citep{johnson:1999}.
SCR models are such an example. In SCR models, we will see that
``spatial location'' of individuals is an element of the model. The
simpler, reduced, model is an ordinary capture-recapture model which
is not spatially explicit (i.e., chapter \ref{chapt.closed}),
but it seems silly and pointless to think about actually using the
reduced model even if we could concoct some statistical test to refute
the more complex model.  The simpler model is manifestly wrong but,
more importantly, not even a plausible data-generating model!
Other examples are when effort, area or
sample rate is used as a covariate. One might prefer to have such things in
models regardless of whether or not they pass some statistical litmus
test (although one can always find referees to argue for pedantic procedure
over thinking).


Many problems can be approached using one of these methods but there
are also broad classes of problems that can't and, for those, you're
on your own. In later chapters we will address model selection in
specific contexts and we hope those will prove useful for a majority
of the situations you encounter.


\section{Poisson GLMs}
\label{glms.sec.poisson}

The Poisson GLM (also known as ``Poisson regression'') is probably the
most relevant and important class of models in all of ecology. The
basic model assumes observations $y_{i}; i=1,2,...,n$ follow a Poisson
distribution with mean $\lambda$ which we write
\[
 	y_{i} \sim \mbox{Poisson}(\lambda)
\]
Commonly $y_{i}$ is a count of animals or plants at some point in
space and lambda might depend on i. For example, $i$ might index point
count locations in a forest, BBS route centers, or sample quadrats, or
similar.  If covariates are available it is typical to model them as
linear effects on the log mean. If $x(i)$ is some measured covariate
associated with observation $i$. Then,
\[
 	log(x(i)) = \alpha  + \beta*x(i)
\]

While we only specify the mean of the Poisson model directly, the
Poisson model (and all GLMs) has a ``built-in'' variance which is
directly related to the mean. In this case, $\mbox{Var}(y) = \mbox{E}(y) =
\lambda$. Thus the model accommodates a linear increase in variance
with the mean.

\subsection{Important properties of the Poisson distribution}
\label{glms.sec.properties}

There are two properties of the Poisson distribution
that make it extremely useful in ecology. First
is the property of {\it compound additivity}. If $y_1$ and $y_2$ are
Poisson random variables with means $\lambda_1$ and $\lambda_2$,
then their sum $N=y_1+y_2$ is Poisson with mean $\lambda_1+\lambda_2$. Thus,
if the observations can be viewed as an aggregate of counts over some
finer unit of measurement, then the mean aggregates in a corresponding
manner.
Secondly, the Poisson distribution has a direct relationship to the multinomial.
If $y_1$ and $y_2$ are $iid$ Poisson then,
conditional on their sum $N = y_1 + y_2$, their joint distribution is multinomial
 with sample size $N$ and cell probabilities
$\lambda_1/(\lambda_1+\lambda_2)$ and
$\lambda_2/(\lambda_1+\lambda_2)$.  As a result of this, most
multinomial models can be analyzed as a Poisson GLM and {\it vice versa}.

\subsection{Example: Breeding Bird Survey Data}

As an example we consider a classical situation in ecology where
counts of an organism are made at a collection of spatial
locations. In this particular example, we have mourning dove counts
made along North American Breeding Bird Survey (BBS) routes in
Pennsylvania, USA. A route consists of 50 stops separated by 0.5
mile. For the purposes here we are defining $y_i$ = route total count
and he sample location will be marked by the center point of the BBS
route.  The survey is run annually and the data set we have is
1966-1998. BBS data can be obtained online at \mbox{\tt http:\//\//www.pwrc.usgs.gov\//bbs\//}.
We will make use of the whole data set shortly but for now we're going
to focus on a specific year of counts -- 1990 -- for the sake of
building a simple model.
 For 1990 there were 77 active routes. We have the data stored
in a \mbox{\tt .csv} file\footnote{check this data format} where rows index the unique route, column 1 is the
route ID, columns 2-3 are the route coordinates (longitude/latitude),
column 4 is a habitat covariate ``forest cover'' (standardized, see
below) and the remaining columns are the yearly counts. Years for
which a route was not run are coded as ``\mbox{\tt NA}'' in the data matrix. We
imagine that this will be a typical format for many ecological
studies, perhaps with more columns representing covariates.  To read
in the data and display the first few elements of this matrix, do
this:
{\small
\begin{verbatim}
> a<-read.csv("pa-bbsdovedata-all.csv")
> data[1:2,1:6]
      X     lon    lat    habitat X66 X67
1 72002 -80.445 41.501 -0.3871372  NA  24
2 72003 -80.347 41.214 -1.0171629  NA  NA
\end{verbatim}
}

It is useful to display the spatial pattern in the observed counts. For that we use a
spatial dot plot - where we plot the coordinates of the observations
and mark the color of the plotting symbol based on the magnitude of
the count.  We have a special plotting function for that which is
called \mbox{\tt spatial.plot()} and it is available with the
supplemental {\bf R} package.
Actually, what we want to do here is plot the
log-count (+1 of course) which (Fig. \ref{glms.fig.padovecounts}) displays a notable pattern that could
be related to something. The {\bf R} commands for obtaining this figure are:
{\small
\begin{verbatim}
data<-read.csv("pa-bbsdovedata-all.csv")
y<-data[,29]  # pick out 1990
notna<-!is.na(y)
y<-y[notna]
spatial.plot(data[notna,2:3],y)
\end{verbatim}
}
 We can ponder the potential effects that
might lead to dove counts being high....corn fields, telephone wires,
barn roofs along with misidentification of pigeons, these could all
correlated reasonably well with the observed count of mourning doves.
Unfortunately we don't have any of that information.

\begin{figure}
\begin{center}
\includegraphics[height=2.75in]{Ch2/figs/PA1}
\end{center}
\caption{Needs a caption}
\label{glms.fig.padovecounts}
\end{figure}

We do have a measure of forest cover in the vicinity of each point
which is contained in the data set (variable ``habitat''). This was derived
from a larger GIS coverage of the state (provided in the data file
``\mbox{\tt pahabdata.csv}'') which can be plotted using the \mbox{\tt spatial.plot} function
using the following commands
{\small
\begin{verbatim}
> map('state',regions="penn",lwd=2)
> spatial.plot(pahabdata[,2:3],pahabdata[,"dfor"],cx=2)
> map('state',regions="penn",lwd=2,add=TRUE)
\end{verbatim}
}
where the result appears in Fig. \ref{glms.fig.paforest}.
We see a prominent pattern that indicates high forest coverage in the
central part of the state and low forest cover in the SE.  Inspecting
the previous figure of log-counts suggests a relationship between
counts and forest cover which is perhaps not surprising.
\begin{figure}
\begin{center}
\includegraphics[height=2.75in]{Ch2/figs/PA2}
\end{center}
\caption{Needs a caption}
\label{glms.fig.paforest}
\end{figure}

\subsection{Doing it in WinBUGS}

Here we demonstrate how to fit a Poisson GLM in {\bf WinBUGS} using the
covariate $x_{i} =$ forest cover. It is advisable that $x_i$ be
standardized in most cases as this will improve mixing of the Markov
chains. Recall that the data we have stored include a standardized
covariate (forest cover) and so we don't have to worry about that
here.  To read the BBS data into {\bf R} and get things set up for
{\bf WinBUGS}
we issue the following commands:
{\small
\begin{verbatim}
data<-read.csv("pa-bbsdovedata-all.csv")
y<-data[,29]                        # pick out 1990
notna<-!is.na(y)
y<-y[notna]                         # discard missing
habitat<-data[notna,4]              # get habitat data
library("R2WinBUGS")                # load R2WinBUGS
data <- list ( "y","M","habitat")   # bundle data for WinBUGS
\end{verbatim}
}
Now we write out the Poisson model specification in {\bf WinBUGS}
pseudo-code, provide initial values, identify parameters to be
monitored and then execute {\bf WinBUGS}:
{\small
\begin{verbatim}
cat("
model {
    for (i in 1:M){
      y[i]~dpois(lam[i])
      log(lam[i])<- beta0+beta1*habitat[i]
     }
 beta0~dunif(-5,5)
 beta1~dunif(-5,5)
}
",file="PoissonGLM.txt")

inits <- function()  list ( beta0=rnorm(1),beta1=rnorm(1))
parameters <- c("beta0","beta1")
out<-bugs (data, inits, parameters, "PoissonGLM.txt", n.thin=2,n.chains=2,
                n.burnin=2000,n.iter=6000,debug=TRUE,working.dir=getwd())
\end{verbatim}
}
{\bf Remarks:} (1) Note the close correspondence in how the model is
specified here compared with the normal regression model
previously. As an exercise you should discuss the specific differences
between the {\bf BUGS} model specifications for the normal and Poisson
models.
{\small
\begin{verbatim}
> print(out,digits=3)
Inference for Bugs model at
``PoissonGLM.txt'', fit using WinBUGS,
 2 chains, each with 4000 iterations (first 1000 discarded), n.thin = 2
 n.sims = 3000 iterations saved
             mean     sd     2.5%      25%      50%      75%    97.5%  Rhat n.eff
beta0       3.151  0.025    3.102    3.135    3.151    3.168    3.199 1.001  2300
beta1      -0.498  0.021   -0.539   -0.512   -0.498   -0.484   -0.457 1.001  3000
fit       869.930 19.856  835.500  855.700  868.600  881.900  913.602 1.002  1600
fitnew     76.709 12.519   54.098   68.107   76.215   84.510  102.602 1.001  3000
deviance 1116.605  2.014 1115.000 1115.000 1116.000 1117.000 1122.000 1.001  3000
\end{verbatim}
}

We might wonder whether this model provides an adequate fit to our
data.  To evaluate that, we used a Bayesian p-value analysis with fit
statistic based on the Freeman-Tukey residual by replacing the model
specification above with this:
{\small
\begin{verbatim}
cat("
model {
    for (i in 1:M){
      y[i]~dpois(lam[i])
      log(lam[i])<- beta0+beta1*habitat[i]
      d[i]<-  pow(pow(y[i],0.5)-pow(lam[i],0.5),2)   #

      ynew[i]~dpois(lam[i])
      dnew[i]<-pow( pow(ynew[i],0.5)-pow(lam[i],0.5),2)

     }
 fit<-sum(d[])
 fitnew<-sum(dnew[])
 beta0~dunif(-5,5)
 beta1~dunif(-5,5)
}
",file="PoissonGLM.txt")
\end{verbatim}
}
The Bayesian p-value is the proportion of times $fitnew > fit$ which,
for this data set, is 0, which was 1.0 in this case (calculation
omitted). This suggests that the basic Poisson model does not fit
well.

\subsection{ Constructing your own MCMC algorithm}

At this point it might be helpful to suffer through an example
building a custom MCMC algorithm. Here, we develop an MCMC algorithm
for
the Poisson regression model, using a Metropolis-within-Gibbs sampling framework. Building MCMC algorithms is covered in more detail in Chapt. \ref{chapt.mcmc} where you can also find step-by-step instructions for Metropolis-within-Gibbs samplers, should the following section move through all this stuff too quickly.  

We will assume that the two parameters have diffuse
normal priors, say $[\alpha] = \mbox{Norm}(0,100)$ and
$[\beta]=\mbox{Norm}(0,100)$ where each has {\it standard deviation}
100 (recall that {\bf WinBUGS} parameterizes the normal in terms of $1/\sigma^{2}$).
We need to assemble the relevant elements of the model which are these
two prior distributions and the
likelihood $[{\bf y}|\alpha,\beta] = \prod_{i} [y_i|\alpha \beta] $ which is,
mathematically, the product of the Poisson pmf evaluated at each $y_i$,
given particular values of $\alpha$ and $\beta$.
Next, we need to identify the full conditionals
$[\alpha|\beta, {\bf y}]$ and $[\beta|\alpha,{\bf y}]$.  We use the all-purpose
rule for constructing full conditionals
(section \ref{glms.sec.rules})
 to discover that:
\[
 [\alpha|\beta,{\bf y}] \propto \left\{ \prod_{i} [y_{i}|\alpha,\beta]\right\}[\alpha]
\]
and
\[
 [\beta|\alpha,{\bf y}] \propto \left\{ \prod_{i}
   [y_{i}|\alpha,\beta]\right\} [\beta]
\]
Remember, we could replace the ``$\propto$'' with ``$=$'' if we
put $[y|\beta]$ or $[y|\alpha]$ in the denominator. But, in general,
$[y|\alpha]$ or $[y|\beta]$ will be quite a pain to compute and, more
importantly, it is a constant as far as the operative parameters
($\alpha$ or $\beta$,
respectively) are concerned. Therefore,
the MH acceptance probability will be the ratio of the
ful-conditional evaluated at a candidate draw to that evaluated at the
current draw, and so the denominator required to change $\propto$ to $=$
winds up canceling from the MH acceptance probability.
Here we will
use the so-called random walk candidate generator, which is a Normal proposal distribution, so that, for example,
 $\alpha^{*} \sim \mbox{Normal}(\alpha^{t},\delta)$ where $\delta$ is
 the standard-deviation of the proposal distribution, which is just a
 tuning parameter that is set by the user and adjusted to achieve efficient mixing of chains (see Section XX in Chapt. \ref{chapt.mcmc}) \footnote{
It would help
lots of people out to see a non-symmetric proposal distribution, and
the extra step needed to account for it. RS: We can include this in the MCMC chapter
}.
We remark also that calculations are often done on the log-scale to
preserve numerical integrity of things when quantities evaluate to
small or large numbers, so keep in mind, for example,
$a*b = exp(log(a) + log(b))$.
 The ``Metropolis within
Gibbs'' algorithm for a Poisson regression turns out to be  remarkably simple:
{\small
\begin{verbatim}
set.seed(2013)

out<-matrix(NA,nrow=1000,ncol=2)   # matrix to store the output
alpha<- -1                         # starting values
beta <- -.8

# begin the MCMC loop ; do 1000 iterations
for(i in 1:1000){

# update the alpha parameter
lambda<- exp(alpha+beta*habitat)
lik.curr<- sum(log(dpois(y,lambda)))
prior.curr<- log(dnorm(alpha,0,100))
alpha.cand<-rnorm(1,alpha,.05)         # generate candidate
lambda.cand<- exp(alpha.cand + beta*habitat)
lik.cand<- sum(log(dpois(y,lambda.cand)))
prior.cand<- log(dnorm(alpha.cand,0,100))
mhratio<- exp(lik.cand +prior.cand - lik.curr-prior.curr)
if(runif(1)< mhratio)
     alpha<-alpha.cand

# update the beta parameter
lik.curr<- sum(log(dpois(y,exp(alpha+beta*habitat))))
prior.curr<- log(dnorm(beta,0,100))
beta.cand<-rnorm(1,beta,.25)
lambda.cand<- exp(alpha+beta.cand*habitat)
lik.cand<- sum(log(dpois(y,lambda.cand)))
prior.cand<- log(dnorm(beta.cand,0,100))
mhratio<- exp(lik.cand + prior.cand - lik.curr - prior.curr)
if(runif(1)< mhratio)
     beta<-beta.cand

out[i,]<-c(alpha,beta)             # save the current values
}


plot(out[,1],ylim=c(-1.5,3.3),type="l",lwd=2,ylab="parameter value",
     xlab="MCMC iteration")
lines(out[,2],lwd=2,col="red")
\end{verbatim}
}
{\bf XXX Andy I removed the bad tuning example and the respective exercise here and added it in Ch7 XXXX}
The first 300 iterations of the MCMC history of each parameter
are shown in Fig. \ref{glms.fig.poissonmcmc2}. These chains are
not very appealing but a couple of things are evident: 
We see
that the burn-in takes about 250 iterations and that after that chains seem to mix reasonably well, although this is not so clear given the scale of the y-axis.
We generated 10,000 posterior samples,
discarding the first 500 as burn-in, and the result is shown in
Fig. \ref{glms.fig.grassy}, this time seperate panels for each
parameter.
The ``grassy''
look of the MCMC history is diagnostic of Markov chains that are
well-mixing and we would generally be very satisfied with results that
look like this.

\begin{figure}
\begin{center}
\includegraphics[height=3in,width=4in]{Ch2/figs/poissonmcmc2}
\end{center}
\caption{Same as previous fig but with $\delta = 0.05$.}
\label{glms.fig.poissonmcmc2}
\end{figure}

\begin{figure}
\begin{center}
\includegraphics[height=4in,width=5in]{Ch2/figs/poissonmcmc3}
\end{center}
\caption{nice grassy mcmc output, longer run of previous with $\delta
  = 0.05$.}
\label{glms.fig.grassy}
\end{figure}

{\bf Remarks:} (1) We used a specific set of starting values for these
simulations. It should be clear that starting values closer to the
mass of the posterior distribution might cause burn-in to occur
faster. As an exercise, evaluate that.  
(2) For the flat normal prior distributions
here we could leave the prior contribution out of the full conditional
evaluation since it is locally constant, i.e., constant in the vicinity of
the posterior mass, and thus has no practical effect. Removing the
prior contribution from the MH acceptance probability is equivalent to
saying that the parameters have an improper uniform prior, i.e.,
$\alpha \sim \mbox{const}$, which is commonly used for mean parameters
in practice.
Note also that we have
used a different prior than in our {\bf WinBUGS} model specification
given previously. As an
exercise, evaluate whether this seems to affect the result.

\section{Poisson GLM with Random Effects}

What we will be doing in most of this book is dealing with random
effects in GLM-like models - similar to what
are usually referred to as generalized
linear mixed models (GLMMs). We provide a brief introduction by way of
example, extending our Poisson regression model to include a random effect.

ANDY STOPPED HERE

{\bf The Log-Normal mixture:} The classical situation involves a GLM
with a normally distributed random effect that is additive on the
linear predictor. For the Poisson case, we have:
\[
 	log(\lambda_{i}) = \alpha  + \beta x_{i} + \eta_{i}
\]
where $\eta_{i} \sim \mbox{Normal}(0,\sigma^{2})$.  A natural
alternative is to have multiplicative gamma-distributed noise,
$exp(\eta_{i}) \sim  \mbox{Gamma}(a,b)$ which would correspond to a
negative binomial kind of over-dispersion, implying a different
mean/variance relationship to the log-normal mixture (the interested
reader should work that out).   Choosing between such possibilities is
not a topic we will get into here because it doesn't seem possible to
provide general guidance on it.
For this model we carried-out a goodness-of-fit evaluation using the
Bayesian p-value based on a Pearson residual statistic. See also
\citep[][ch. 18]{kery:2010}
for an example involving a binomial mixed model\footnote{Kery has
  noticed that such tests probably have 0 power. Should use the
  marginal frequency of the data}.
 Anyhow, it is really amazingly simple
to express this model in {\bf WinBUGS} and have {\bf WinBUGS}  draw samples from the posterior distribution using the following code for the BBS dove counts:
{\small
\begin{verbatim}
data<-read.csv("pa-bbsdovedata-all.csv")
locs<-data[,2:3]
habitat<-data[,4]
y<-data[,29]     # grab year 1990
M<-length(y)

set.seed(2013)

cat("
model {
  for (i in 1:M){
     y[i]~dpois(lam[i])
     log(lam[i])<- alpha+ beta*habitat[i] + eta[i]
     frog[i]<-beta*habitat[i] + eta[i]
     eta[i] ~ dnorm(0,tau)
     d[i]<-  pow(pow(y[i],0.5)-pow(lam[i],0.5),2)

     ynew[i]~dpois(lam[i])
     dnew[i]<- pow(pow(ynew[i],0.5)-pow(lam[i],0.5),2)
   }
 fit<-sum(d[])
 fitnew<-sum(dnew[])

 alpha~dunif(-5,5)
 beta~dunif(-5,5)
 sigma~dunif(0,10)
 tau<-1/(sigma*sigma)
}

",file="model.txt")
data <- list ( "y","M","habitat")
inits <- function()
  list ( alpha=rnorm(1),beta=rnorm(1),sigma=runif(1,0,4))
parameters <- c("alpha","beta","sigma","tau","fit","fitnew")
library("R2WinBUGS")

out<-bugs (data, inits, parameters, "model.txt", n.thin=2,n.chains=2,
 n.burnin=1000,n.iter=5000,debug=TRUE)
\end{verbatim}
}
This produces the following posterior summary statistics:
{\small
\begin{verbatim}
> print(out,digits=2)
Inference for Bugs model at "model.txt", fit using WinBUGS,
 2 chains, each with 5000 iterations (first 1000 discarded), n.thin = 2
 n.sims = 4000 iterations saved
           mean    sd   2.5%    25%    50%    75%  97.5% Rhat n.eff
alpha      2.98  0.08   2.82   2.93   2.98   3.03   3.12 1.00  1400
beta      -0.53  0.07  -0.68  -0.58  -0.53  -0.49  -0.38 1.01   350
sigma      0.60  0.06   0.49   0.56   0.59   0.64   0.73 1.00  2000
tau        2.88  0.57   1.88   2.47   2.86   3.24   4.12 1.00  2000
fit       26.58  3.72  19.87  23.96  26.37  29.01  34.46 1.00  4000
fitnew    26.83  3.90  19.60  24.12  26.68  29.36  35.04 1.00  4000
deviance 445.94 12.18 424.00 437.40 445.20 453.90 471.50 1.00  4000

[... some output deleted ...]
\end{verbatim}
}
The Bayesian p-value for this model is
\begin{verbatim}
> mean(out$sims.list$fit>out$sims.list$fitnew)
[1] 0.4815
\end{verbatim}
indicating a pretty good fit. Given the site-level random effect, it
would be surprising for this model to not fit! One thing we notice is
that the posterior standard deviations of the regression parameters
are much higher, a result of the excess variation. Wwe would also
notice much less precise predictions of hypothetical new
observations.


ANDY STOPPED HERE.




\section{Binomial GLMs}

Another extremely important class of models in ecology are
binomial models. We use binomial models for count data whenever the
observations are counts or frequencies and it is natural to condition
on a ``sample size'', say $K$, the maximum frequency possible in a sample.
 The random variable, $y \le K$, is then the
frequency of occurrences out of $K$ ``trials''. The parameter of the binomial
models is $p$, often called ``success probability'' which is related
to the expected value of $y$ by $E(y) = pK$. Usually we are interested
in modeling covariates that affect the parameter $p$, and such models
are called binomial GLMs , binomial
regression models or logistic regression, although logistic regression
 really only applies when the logistic link is used to model
the relationship between $p$ and covariates (see below).

One of the most typical binomial GLMs occurs when the sample size
equals 1 and the outcome, $y$, is ``presence'' ($y=1$) or ``absence''
($y=0$) of a species. This is a classical ``species distribution''
modeling situation. A special situation occurs when presence/absence
is observed with error \citep{mackenzie_etal:2002,tyre_etal:2003}.
In that case, $K>1$ samples
are usually needed for effective estimation of model parameters.

 In standard binomial regression problems the sample size
is fixed by design but interesting models also arise when the sample
size is itself a random variable. These are the $N$-mixture models
\citep{royle:2004, kery_etal:2005, royle_dorazio:2008, kery:2010}
and related models (in this case, $N$ being the sample size,
which we labeled $K$ above)\footnote{Some of the jargon is actually a little
bit confusing here
because the binomial index is customarily referred to as ``sample size''
but in the context of $N$-mixture models $N$ is actually the
``population size''}.
Another
situation in which the binomial sample size is ``fixed'' is closed
population capture-recapture models in which a population of
individuals is sampled $K$ times.  The number of times each individual
is encountered is a binomial outcome with parameter - encounter
probability -- $p$, based on a sample of size $K$.  In addition, the
total number of unique individuals observed, $n$, is also a binomial
random variable based onpopulation size $N$.  We consider such
models in the chapter \ref{chapt.closed}.


\subsection{Binomial regression}

In binomial models, covariates are modeled on a suitable
transformation (the link function) of the binomial success
probability, $p$.  Let $x_{i}$ denote some measured covariate for
sample unit $i$ and let $p_{i}$ be the success probability for unit $i$.
The standard choice is the ``logit'' link function which is:
\[
log(p_i/(1-p_i)) = \alpha + \beta*x_{i}.
\]
The inverse-logit (or ``expit'') is
\[
p_{i} = \mbox{expit}(\alpha + \beta*x_{i}) =
 \frac{ \exp(\alpha + \beta*x_{i})}
      {1 + \exp(\alpha + \beta*x_{i} ) }
\]
There are many other possible link functions. However, ecologists seem
to adopt the logit link function without question in most
applications\footnote{a notable exception is distance sampling, which
  is all about choosing among link functions}.  We sometimes use the
``complementary log-log'' (= ``cloglog'') link function in ecological
applications because it arises naturally in many situations
\citep[][p. 150]{royle_dorazio:2008}. For example, consider the
``probability of observing a count greater than 0'' under a Poisson
model: $\Pr(y>0) = 1-exp(- \lambda)$. In that case,
\[
cloglog(p) =log(- log(1-p)) = log(\lambda)
\]
So that if you have covariates in your linear predictor for $E(y)$
under a Poisson model then they are linear on the complementary
log-log link of $p$.
In models of species occurrence it seems natural to view occupancy as
being derived from local abundance $N$
\citep{royle_nichols:2003,royle_dorazio:2006,dorazio:2007}.
Therefore,
models of local abundance in which $N \sim \mbox{Poisson}(A \lambda)$
for a habitat patch of area $A$ implies a model for occupancy $\psi$
of the form
\[
 cloglog(\psi) = log(A) + log(\lambda).
\]
We will use the cloglog link in some analyses of
SCR models in chapter \ref{chapt.scr0} and elsewhere.


\subsection{ Example: Waterfowl Banding Data}

It would be easy to consider a standard ``distribution modeling''
application where $K=1$ and the outcome is occurrence ($y=1$) or not
($y=0$) of some species. Such examples abound in books (e.g.,
\citet[][ch. 3]{royle_dorazio:2008}; \citet[][ch. 21]{kery:2010};
\citet[][ch. 13]{kery_schaub:2011}) and in the literature.
Instead, we will
consider an example involving band returns of waterfowl which were
analyzed by \citet{royle_dubovsky:2001}\footnote{I hate this example.
  Anyone got a better one thats not distribution modeling?}.

For these data, $y_i$ is the number of waterfowl bands recovered out
of $B_i$ birds banded at some location ${\bf s}_{i}$. In this case $B_{i}$ is
fixed. Thinking about recovery rate as being proportional to harvest
rate, we use these data to explore geographic gradients in recovery rate
resulting from variability in harvest pressure experienced by
populations depending on their migration ecology. As such, we fit a
basic binomial GLM with a linear response to geographic coordinates
(including an interaction term). The data are provided with the {\bf
  R} package \mbox{\tt scrbook}. Here we
 provide the part of the script for creating the model and fitting the
 model in
{\bf WinBUGS} using the \mbox{\tt bugs} function.
There are few structural differences between this model and the
Poisson GLM fitted previously. The main things are due to the data
structure (we have a matrix here instead of a vector) and otherwise we
change the main distributional assumption to binomial (specified with
\mbox{\tt dbin}) and then use the \mbox{\tt logit} function to relate
the parameter $p_{it}$ to the covariates.  Here is the script:

{\small
\begin{verbatim}
load("mallarddata")  # not sure how this will look

sink("model.txt")
cat("
model {
 for(t in 1:5){
    for (i in 1:nobs){
       y[i,t] ~ dbin(p[i,t], B[i,t])
       logit(p[i,t]) <- alpha0[t] + alpha1*X[i,1] + alpha2*X[i,2] + alpha3*X[i,1]*X[i,2]
     }
}
	alpha1~dnorm(0,.001)
	alpha2~dnorm(0,.001)
	alpha3~dnorm(0,.001)
	for(t in 1:5){
 	alpha0[t] ~ dnorm(0,.001)
 }
}
",fill=TRUE)
sink()

data  <- list(B=mallard.bandings, y=mallard.recoveries,
             nobs=nrow(banding.locs),X=banding.locs)
inits <- function(){
      list(alpha0=rnorm(5),alpha1=0,alpha2=0,alpha3=0) }
parms <- list('alpha0','alpha1','alpha2','alpha3')
out   <- bugs(data,inits, parms,"model.txt",n.chains=3,
 	n.iter=2000,n.burnin=1000, n.thin=2,debug=TRUE)
\end{verbatim}
}


Posterior summaries of model parameters are as follows:
{\small
\begin{verbatim}
> print(out,digits=3)
Inference for Bugs model at "model.txt", fit using WinBUGS,
 3 chains, each with 2000 iterations (first 1000 discarded), n.thin = 2
 n.sims = 1500 iterations saved
              mean    sd     2.5%      25%      50%      75%    97.5%  Rhat n.eff
alpha0[1]   -2.346 0.036   -2.417   -2.370   -2.346   -2.323   -2.277 1.001  1500
alpha0[2]   -2.356 0.032   -2.420   -2.379   -2.356   -2.335   -2.292 1.001  1500
alpha0[3]   -2.220 0.035   -2.291   -2.244   -2.219   -2.197   -2.153 1.001  1500
alpha0[4]   -2.144 0.039   -2.225   -2.169   -2.143   -2.116   -2.068 1.000  1500
alpha0[5]   -1.925 0.034   -1.990   -1.949   -1.924   -1.901   -1.856 1.004   570
alpha1      -0.023 0.003   -0.028   -0.025   -0.023   -0.022   -0.018 1.001  1500
alpha2       0.020 0.006    0.009    0.016    0.020    0.024    0.031 1.001  1500
alpha3       0.000 0.001   -0.002   -0.001    0.000    0.000    0.002 1.001  1500
deviance  1716.001 4.091 1710.000 1713.000 1715.000 1718.000 1726.000 1.001  1500

[... some output deleted ...]
\end{verbatim}
}

The basic result suggests a negative east-west gradient and a positive
south to north gradient but no interaction. A map of the response
surface is shown in Fig. \ref{glms.fig.bandrecovery}.
 We did an additional MCMC run where we saved the binomial
parameter $p$ and computed the Bayesian p-value (double use of ``p''
here is confusing, but I guess that happens sometimes!)
using a fit statistic based on the Freeman-Tukey
statistic (see Section XXX above). The result indicates that the
linear response surface model does not provide an adequate fit of the
data. The reader should contemplate whether this invalidates the basic
interpretation of the result.


\begin{figure}
\begin{center}
\includegraphics[height=2.75in]{Ch2/figs/responsesurface}
\end{center}
\caption{Predicted recovery rate of bands.}
\label{glms.fig.bandrecovery}
\end{figure}

\section{ Summary and Outlook}

GLMs and GLMMs are the most useful statistical methods in all of
ecology. The principles and procedures underlying these methods are
relevant to nearly all modeling and analysis problems in every branch
of ecology. Moreover, understanding how to analyze these models is
crucial in a huge number of diverse problems. If you understand and
can conduct classical likelihood and Bayesian analysis of Poisson and
binomial GLM(M)s, then you will be successful analyzing and
understanding more complex classes of models that arise. We will see
shortly that spatial capture-recapture models are a type of GLMM
and thus having a basic
understanding of the conceptual origins and formulation of GLM(M)s and
their analysis is extremely useful.

We note that GLM(M)s are routinely
analyzed by likelihood methods but we have focused on Bayesian
analysis here in order to develop the tools that are less familiar to
most ecologists.  In particular, Bayesian analysis of models with random
effects is relatively straightforward because the models
are easy to analyze conditional on the random effect, using methods of
MCMC.  Thus, we will often analyze SCR models in later chapters by
MCMC, explicitly adopting a Bayesian inference framework.
In that regard, the various {\bf BUGS} engines ({\bf WinBUGS}, {\bf
  OpenBUGS}, {\bf JAGS}) are enormously useful because they
provide an accessible platform for
carrying out  analyses by MCMC by just
describing the model, and not having to worry about how to actually
build MCMC algorithms.  That said, the {\bf BUGS} language is more important
than just to the extent that it enables one to do MCMC - it is useful
as a modeling tool because it fosters understanding, in the sense that
it forces you to become intimate with your model. You have to write
down all of the probability assumptions, the relationships between
observations and latent variables and parameters. This is really a
great learning paradigm that you can grow with.

While we have emphasized Bayesian analysis in this chapter, and make
primary use of it through the book, we
we will provide an introduction to likelihood analysis in chapter
\ref{chapt.mle} and use those  methods also from time to time.
 Before getting to that, however, it will be useful to
talk about more basic, conventional closed population
capture-recapture models and these are the topic of the next chapter.


\chapter{
 Closed Population Models
}
\markboth{Chapter 3}{}
\label{chapt.closed}

\vspace{.3in}
%%Andy, I really like connecting a new chapter to the previous ones with a few words, so I added this half sentence
Having covered the basics of hierarchical models and their implementation, in this chapter we will consider ordinary capture-recapture (CR)
models for estimating population size in closed populations. We will
see that such models are closely related to binomial (or logistic)
regression type models. In fact, when $N$ is known, they are precisely
such models.  We consider some important extensions of ordinary closed
population models that accommodate various types of ``individual
effects'' --- either in the form of explicit covariates (sex, age,
body mass) or unstructured ``heterogeneity'' in the form of an
individual random effect. In general, these models are variations of
generalized linear or generalized linear mixed models (GLMMs).
Because of the paramount importance of this concept, we focus mainly
on fairly simple models in which the observations are individual
encounter frequencies, $y_{i}$ = the number of encounters of
individual $i$ out of $K$ replicate samples of the population which,
for the models we consider here, is the outcome of a binomial random
variable.  Along the way, we consider the spatial context of
capture-recapture data and models and demonstrate that density cannot
be formally estimated when spatial information is ignored. We also
review some of the informal methods of estimating density using CR
methods, and consider some of their limitations.  We will be exposed
to our first primitive spatial capture-recapture models which arise as
relatively minor variations of so-called ``individual covariate
models'' (of the \citet{huggins:1989} and \citet{alho:1990}
variety). In a sense, the point of this chapter is to establish that
linkage XX between non-spatial and spatial capture-recapture models XXX in a direct and concise manner beginning with the basic
``Model $M_0$'' and extensions of that model to include individual
heterogeneity and also individual covariates. A special type of
individual covariate models is distance sampling, which could be
thought of as the most primitive spatial capture-recapture model.  In
later chapters we further develop and extend ideas introduced in this
chapter.

We emphasize Bayesian analysis of capture-recapture models and we
accomplish this using a method related to classical ``data
augmentation'' from the statistics literature XXX SOMETHING WRONG WITH BRACKETS IN REF XXX
\citet[e.g.,][]{tanner_wong:1987}).  This is a general concept in
statistics but, in the context of capture-recapture models where $N$
is unknown, it has a consistent implementation across classes of
capture-recapture models and one that is really convenient from the
standpoint of doing MCMC \citep{royle_etal:2007}. We use data
augmentation throughout this book and thus emphasize its conceptual
and technical origins and demonstrate applications to closed
population models.  We refer the reader to
\citet[][ch. 6]{kery_schaub:2011} for an accessible and complimentary
development of ordinary closed population models.


\section{The Simplest Closed Population Model: Model $M_0$}

To start looking at the simples capture-recapture model, let's suppose that there exists a population of $N$ individuals which we
subject to repeated sampling, say over $K$ nights, where individuals
are captured, marked, and subsequently recaptured.  We suppose that
individual encounter histories are obtained, and these are of the form
of a sequence of 0's and 1's indicating capture $(y=1)$ or not $(y=0)$
during any sampling occasion (``sample'').  As an example, suppose
$K=5$ sampling occasions, then an individual captured during sample 2
and 3 but not otherwise would have an encounter history of the form
${\bf y}=(0,1,1,0,0)$. Thus, the observation ${\bf y}_{i}$ for each
individual $(i)$ is a vector having elements denoted by $y_{ik}$ for
$k=1,2,..,K$. Usually this is organized as a row of a matrix with
elements $y_{ik}$, see Table \ref{tab.3.1}.  Except where noted
explicitly, we suppose that observations are independent within
individuals and among individuals.  Formally, this allows us to say
that $y_{ik}$ are $iid$ Bernoulli random variables and we may write $y_{ik}
\sim \mbox{Bern}(p)$.  Consequently, for this very simple model in
which $p$ is in fact constant, then we can declare that the individual
encounter frequencies (total captures), $y_{i} = \sum_{k} y_{ik}$,
have a binomial distribution based on a sample of size $K$. That is
\[
y_{i}  = \sum_{k} y_{ik} \sim \mbox{Bin}(p,K)
\]
for every individual in the population. This is a remarkably simple
model that forms the cornerstone of almost all of classical
capture-recapture models, including most spatial capture-recapture
models discussed throughout this book.  

Evidently, the basic
capture-recapture model structure is precisely a simplistic version of
a logistic-regression model with only an intercept term
($\mbox{logit}(p) = \mbox{constant}$).  To say that all
capture-recapture models are just logistic regressions is only
slightly inaccurate. In fact, we are proceeding here ``conditional on
$N$'', i.e., as if we knew $N$. In practice we don't, of course, and
that is kind of the point of capture-recapture models as estimating
$N$ is the central objective. But, by proceeding conditional on $N$,
we can specify a simple model and then deal with the fact that $N$ is
unknown using standard methods that you are already familiar with
(i.e., GLMs - see chapter 2).
\begin{table}
\centering
\caption{a capture-recapture data set with $n=6$ observed individuals
and $K=5$ samples.}
\begin{tabular}{r|ccccc|c}
&  \multicolumn{5}{c}{Sample occasion} &  \\ \hline
 indiv $i$ &  1 & 2 & 3 & 4 & 5 & $y_{i}$ \\ \hline
  1 &     1 & 0 & 0 & 1 & 0  & 2   \\
  2 &     0 & 1 & 0 & 0 & 1  & 2   \\
  3 &     1 & 0 & 0 & 1 & 0  & 2   \\
  4 &     1 & 0 & 1 & 0 & 1  & 3   \\
  5 &     0 & 1 & 0 & 0 & 0  & 1   \\
  $n=6$ & 1 & 0 & 0 & 0 & 0  & 1   \\ \hline
\end{tabular}
\label{tab.3.1}
\end{table}

Assuming individuals of the population are observed independently, the
joint probability distribution of the observations is the product of
$N$ binomials
\begin{eqnarray*}
  \Pr(y_1, \ldots, y_N | p) &=& \prod_{i=1}^N  \mathrm{Bin}(y_i | K, p) \\
   &=& \prod_{k=0}^K  \pi(k)^{n_k}
\end{eqnarray*}
where $\pi(k) = \mathrm{Bin}(k | K,p)$ and where $n_k = \sum_{i=1}^N
I(y_i = k)$ denotes the number of individuals captured $k$ times in
$K$ surveys. We emphasize that this is conditional on $N$, in which
case we get to observe the $y=0$ observations and the resulting data
are just $iid$ binomial counts. Because this is a binomial regression
model of the variety described in Chapt. \ref{glms}, fitting this model using
a {\bf BUGS} engine poses no difficulty.

The essential problem in capture-recapture, however, is that $N$ is
not known because the number of uncaptured/missing individuals (i.e.,
those in the zero cell that occur with probability $\pi(0)$) is
unknown.  Consequently, the observed capture frequencies $n_k$ are no
longer independent. Instead, their joint distribution is multinomial
(e.g., see \citet[][p. xyz]{illian_etal:2008}):
\begin{equation}
    n_1, n_2, \ldots, n_K \sim \mathrm{Multin}(N, \pi(1), \pi(2), \ldots, \pi(K))
\label{closed.eq.multinomial4m0}
\end{equation}
Note that in our notation the number of uncaptured/missing individuals is
denoted by $n_0 = N - n$, where $n = \sum_{k=1}^K n_k$ denotes the total
number of distinct individuals seen in the $K$ samples.
XXX ANDY; MAYBE IT MIGHT BE WORTH MENTIONING WHY THE n0 DOESNT SHOW UP IN THE MULTINOMIAL XXXXX

To fit the model in which $N$ is {\it unknown}, we can regard $N$ as a
parameter and maximize the multinomial likelihood directly.  While
direct likelihood analysis of the multinomial model is
straightforward, that does not prove to be too useful in practice
because we seldom are concerned with models for the aggregated
encounter history frequencies, XXX which entail that capture probabilities are the same for all individuals XXX. In many instances, including for
spatial capture-recapture (SCR) models, we require a formulation of
the model that can accommodate individual level covariates XXX to account for differences in detection among individuals XXX which we
address subsequently in this chapter.


\subsection{The Spatial Context of Capture-Recapture}

XXX I WOULD CHANGE THE SECTION HEADING TO SOMETHING LIKE 'POPULATION CLOSURE AND THE SPATIAL CONTEXT OF CAPTURE-RECAPTURE XXX
A common assumption made is that of population ``closure'' which is
really just a colloquial way of saying (in part) the Bernoulli
assumptions stated explicitly above. In the biological context,
closure means, strictly, no additions or subtractions from the
population during study. This is manifest by the statement that the
encounters are independent and identically distributed (iid) Bernoulli
trials.  In practice, closure is usually interpreted by the manner in
which potential violations of that assumption arise. In particular,
two important elements of the closure assumption are ``demographic''
and ``geographic'' closure. If an individual dies then subsequent
values of $y_{ik}$ are clearly no longer Bernoulli trials with the
same parameter $p$; XXX since the probability of capturing that individual becomes 0 XXX. If there is no mortality or recruitment in the
population, then we say that demographic closure is
satisfied. Similarly, animals may emigrate or immigrate. If they do
not, then geographic closure is satisfied. Sometimes a distinction is
made between temporary and permanent emigration or immigration. That
is a relevant distinction in spatial capture-recapture models, because
SCR models explicitly accommodate ``temporary emigration'' of a
certain type, due to individuals moving about their home range. XXX In contrast, ordinary capture-recapture models cannot explicitly deal with the fact that, unless we're sampling a fenced enclosure or an island, individuals are bound to move off the trapping grid. XXX The
demographic closure assumption can also be relaxed using SCR models,
but we will save that discussion for Chapt. \ref{chapt.scr0}.
XXXX I FEEL LIKE THIS SECTION STILL NEEDS A SENTENCE THAT MAKES THE POINT - SPATIAL CONTEXT; POP CLOSURE AND SCR; BUT I AM HAVING TROUBLE PUTTING THAT INTO A FEW WORDS RIGHT NOW XXXX

\subsection{Conditional likelihood}

We saw that a basic closed population model is a simple logistic
regression model if $N$ is known and, when $N$ is unknown, the model
is multinomial with index or sample size parameter $N$. This
multinomial model, being conditional on $N$, is sometimes referred to
as the ``joint likelihood'' the ``full likelihood'' or the
``unconditional likelihood'' (or model in place of likelihood). This
formulation differs from the so-called ``conditional likelihood''
approach in which the likelihood of the observed encounter histories
is devised conditional on the event that an individual is captured at
least once.  To construct this likelihood, we have to recognize that
individuals appear or not in the sample based on the value of the
random variable $y_{i}$, that is, we capture them if and only if
$y_{i}>0$.  The observation model is therefore based on $\Pr(y|y>0)$.
For the simple case of Model $M_0$, the resulting conditional
distribution is a ``zero truncated'' binomial distribution which
accounts for the fact that we cannot observe the value $y=0$ in the
data set \citep[see][sec. 5.1]{royle_dorazio:2008}.  Both the
conditional and unconditional models are legitimate modes of analysis
in all capture-recapture types of studies, and they provide equally
valid descriptions of the data and for many practical purposes provide
equivalent inferences, at least in large sample sizes
\citep{sanathanan:1972}.

In this book we emphasize Bayesian analysis of capture-recapture
models using data augmentation (discussed subsequently), which
produces yet a third distinct formulation of capture recapture-models
based on the zero-{\it inflated} binomial distribution that we
describe in the next section.  Thus, there are 3 distinct formulations
of the model -- or modes of analysis -- for analyzing all
capture-recapture models based on the (1) binomial model for the joint
or unconditional specification; (2) zero-truncated binomial that
arises ``conditional on $n$''; and (3) the zero-inflated binomial that
arises under data augmentation.  Each formulation has a distinct
complement of model parameters (shown in Table \ref{tab.3.modes} for
Model $M_0$).


\begin{table}
\centering
\caption{Modes of analysis of capture-recapture models. Closed
  population models can be analyzed using the joint or ``full
  likelihood'' which contains $N$ as an explicit parameter, the
  conditional likeilhood which does not involve $N$, or by data
  augmentation which replaces $N$ with $\psi$. Each approach yields a
  distinct likelihood.}
\begin{tabular}{ccc}
Mode of analysis & parameters in model & statistical model \\ \hline
Joint likelihood                &	$p$, $N$	&	multinomial with index $N$\\
Conditional likelihood 		&	$p$	&	zero-truncated binomial \\
Data augmentation		&	$p$, $\psi$	&	zero-inflated binomial\\
\end{tabular}
\label{tab.3.modes}
\end{table}



\section{ Data Augmentation }
\label{closed.sec.da}

We consider a method of analyzing closed population models using data
augmentation (DA) which is useful for Bayesian analysis and, in
particular, analysis of models using the various BUGS engines and
other software.  Data augmentation is a general statistical concept
that is widely used in statistics in many different settings. The
classical reference is \citet{tanner_wong:1987} but see also
\citet{liu_wu:1999}.  Data augmentation can be adapted to provide a
very generic framework for Bayesian analysis of capture-recapture
models with unknown $N$. This idea was introduced for closed
populations by \citet{royle_etal:2007}, and has subsequently been
applied to a number of different contexts including individual
covariate models \citep{royle:2009}, open population models
\citep{royle_dorazio:2008,royle_dorazio:2010, gardner_etal:2010ecol},
spatial capture-recapture models \citep{royle_young:2008,
  royle_etal:2010, gardner_etal:2009}, and many
others. \citet[][Chapt. 6]{kery_schaub:2011} provides a good introduction to data
augmentation in the context of closed population models. 


Conceptually, data augmentation is a reparameterization of the
``complete data'' model -- that which is conditional on $N$. The
reparameterization is achieved by embedding this data set into a
larger data set having $M> N$ ``rows'' (individuals) and reexpressing
the model conditional on $M$ instead of $N$. XXX The great thing about data augmentation is that we do not need to know $N$ for this reparameterization. XXX Although this has a whiff of
arbitrariness or even ad hockery to it in the choice of $M$, 
it is always possible, in practice, to choose $M$ pretty easily for
a given problem and context and results will be insensitive to choice
of $M$\footnote{Unless the data set is sufficiently small that parameters are
weakly
identified}.
Then, under data augmentation, analysis
 is focused on the ``augmented data set.'' That is, we analyze the bigger
 data set - the one having $M$ rows - with an appropriate model that
 accounts for the augmentation. Inference is focused directly on
 estimating the proportion $\psi = E[N]/M$, instead of directly on $N$,
 where $\psi$ is the ``data augmentation parameter.''


\subsection{DA links occupancy models and closed population models}

%We provide a heuristic description of data augmentation based on the
XXX There is a XXX close correspondence between so-called ``occupancy'' models and closed
population models following \citet[][sec. 5.6]{royle_dorazio:2008}.

In occupancy models \citep{mackenzie_etal:2002, tyre_etal:2003} the
sampling situation is that $M$ sites, or patches, are sampled multiple
times to assess whether a species occurs at each site.  This yields
encounter data such as that illustrated in the left panel of Table
\ref{closed.tab.occ}. The important problem is that a species may occur at
a site, but go undetected, yielding the ``all-zero'' encounter
histories which are observed. However, some of the all-zeros may well
correspond to sites where the species in fact {\it does}
occur. Thus, while the zeros are observed, there are too many of them
and, in a sense, the inference problem is to allocate the zeros into
``structural'' (fixed) and ``sampling'' (or stochastic) zeros. More
formally, inference is focused on the parameter $\psi$, the
probability that a site is occupied.  In contrast, in classical closed
population studies, we observe a data set as in the middle panel of
Table \ref{closed.tab.occ} where {\it no} zeros are observed. The inference
problem is, essentially, to estimate how many sampling zeros there are
- or should be - in a ``complete'' data set. This objective
(how many sampling zeros?) is precisely the same for both types of
problems if an upper limit $M$ is specified for the closed population
model. The only distinction being that, in occupancy models, $M$ is
set by design (i.e., the number of sites to visit) whereas a natural
choice of $M$ for capture-recapture models may not be
obvious. However, by assuming a uniform prior for $N$ on the integers
$[0,M]$, this upper bound is induced \citep{royle_etal:2007}. Then,
one can analyze capture-recapture models by adding $M-n$ all-zero
encounter histories to the data set and regarding the augmented data
set, essentially, as a site-occupancy data set.

Thus, the heuristic motivation of data augmentation is to fix the size
of the data set by adding {\it too many} all-zero encounter histories
to create the data set shown in the right panel of Table
\ref{closed.tab.occ} - and then analyze the augmented data set using an
occupancy type model which includes both ``unoccupied sites'' as well
as ``occupied sites'' at which detections did not occur. We call these
$M-n$ all-zero histories ``potential individuals'' because they exist
to be recruited (in a non-biological sense) into the population, for
example during an analysis by MCMC.

To analyze the augmented data set, we recognize that it is a
zero-inflated version of the known-$N$ data set. That is, some of the
augmented all-zeros are sampling zeros (corresponding to actual
individuals that were missed) and some are ``structural'' zeros, which
do not correspond to individuals in the population. For a basic
closed-population model, the resulting likelihood under data
augmentation - that is, for the data set of size $M$ -- is a simple
zero-inflated binomial likelihood.  The zero-inflated binomial model
can be described ``hierarchically'', by introducing a set of binary
latent variables, $z_{1},z_{2},\ldots, z_{M}$, to indicate whether
each individual $i$ is ($z_i=1$) or is not ($z_i=0$) a member of the
population of $N$ individuals exposed to sampling. We assume that
$z_{i} \sim \mbox{Bern}(\psi)$ where $\psi$ is the probability that an
individual in the data set of size $M$ is a member of the sampled
population - in the sense that $1-\psi$ is the probability of
realizing a ``structural zero'' in the augmented data set.  The
zero-inflated binomial model which arises under data augmentation can
be formally expressed by the following set of assumptions:

\begin{eqnarray*}
 y_{i}|{z_{i}=1} & \sim  &\mbox{Bin}(K, p) \\
 y_{i}|{z_{i}=0} & \sim &  \delta(0)  \\
 z_{i} & \stackrel{iid}{\sim} & \mbox{Bern}(\psi) \\
 \psi & \sim & \mathrm{Unif}(0,1) \\
 p & \sim & \mathrm{Unif}(0,1)
\end{eqnarray*}
for $i=1, \ldots, M$, where $\delta(0)$ is a point mass at $y=0$.

Note that, under data augmentation, 
$N$ is no longer an explicit parameter of this
model. Instead, we estimate $\psi$ and functions of the latent
variables. In particular, under the assumptions of the zero-inflated
model, $z_{i} \stackrel{iid}{\sim} \mbox{Bern}(\psi)$; therefore, $N$
is a function of these latent variables:
 \[
 N = \sum_{i=1}^{M} z_{i}.
\]
Further, we note that the latent $z_i$ parameters can be removed from
the model by integration, in which case the joint probability of the
data is
\begin{equation}
  \Pr(y_1, \ldots, y_M | p, \psi) = \prod_{i=1}^M  \psi \mathrm{Bin}(y_i | K, p) +  I(y_i=0) (1-\psi)
\end{equation}
Which can be maximized directly to obtain the MLEs of the structural
parameters $\psi$ and $p$ or those of other more complex models
\citep[e.g., see][]{royle:2006}. We could estimate these parameters
and then use them to obtain an estimator of $N$ using the so-called
``Best unbiased predictor'' \citep[see][]{royle_dorazio:2011}.

\begin{table}
\centering
\caption{Hypothetical occupancy data set (left), capture-recapture data
 in standard form (center), and capture-recapture data augmented with
 all-zero capture histories (right). }
\begin{tabular}{cccc|cccc|cccc}
\hline
\multicolumn{4}{c}{Occupancy data}    &
\multicolumn{4}{c}{Capture-recapture} &
\multicolumn{4}{c}{Augmented C-R}     \\ \hline
site    & k=1 & k=2 & k=3 & ind & k=1 &k=2  & k=3 & ind & k=1 & k=2 & k=3           \\ \hline
1  & 0   & 1   & 0   & 1   & 0   & 1  & 0   & 1   & 0   & 1   & 0                   \\
2  & 1   & 0   & 1   & 2   & 1   & 0 & 1    & 2 & 1 & 0 & 1 \\
3  & 0   & 1   & 0   & .   & 0   & 1 & 0    & 3 & 1 & 0 & 1 \\
4  & 1   & 0   & 1   & .   & 1   & 0 & 1    & 4 & 1 & 0 & 1 \\
5  & 0   & 1   & 1   & .   & 0   & 1 & 1    & 5 & 1 & 0 & 1 \\
.  & 0   & 1   & 1   & .   & 0   & 1 & 1    & . & 0 & 1 & 1 \\
.  & 1   & 1   & 1   & .   & 1   & 1 & 1    & . & 0 & 1 & 1 \\
.  & 1   & 1   & 1   & .   & 1   & 1 & 1    & . & 1 & 1 & 1 \\
   & 1   & 1   & 1   & .   & 1   & 1 & 1    & . & 1 & 1 & 1 \\
n  & 1   & 1   & 1   & n   & 1   & 1 & 1    & n & 1 & 1 & 1 \\
.  & 0   & 0   & 0   &     &     &   &      & . & 0 & 0 & 0 \\
.  & 0   & 0   & 0   &     &     &   &      & . & 0 & 0 & 0 \\
   & 0   & 0   & 0   &     &     &   &      &   & 0 & 0 & 0 \\
   & 0   & 0   & 0   &     &     &   &      &   & 0 & 0 & 0 \\
   & 0   & 0   & 0   &     &     &   &      &   & 0 & 0 & 0 \\
   & 0   & 0   & 0   &     &     &   &      & N & 0 & 0 & 0 \\
.  & 0   & 0   & 0   &     &     &   &      & . & 0 & 0 & 0 \\
.  & 0   & 0   & 0   &     &     &   &      &   & 0 & 0 & 0 \\
M  & 0   & 0   & 0   &     &     &   &      & . & 0 & 0 & 0 \\
   &     &     &     &     &     &   &      & . & . & . & . \\
   &     &     &     &     &     &   &      & . & . & . & . \\
   &     &     &     &     &     &   &      & . & . & . & . \\
   &     &     &     &     &     &   &      & M & 0 & 0 & 0 \\
\end{tabular}
\label{closed.tab.occ}
\end{table}


\subsection{Model $M_0$ in BUGS}

For model $M_0$ in which we can aggregate the encounter data to
individual-specific encounter frequencies, the augmented data are
given by the vector of frequencies $(y_{1}, \ldots, y_{n}, 0, 0,
\ldots, 0)$. The zero-inflated model of the augmented data combines
the model of the latent variables, $z_{i} \sim \mbox{Bern}(\psi)$ with
the conditional-on-$z$ binomial model:
\begin{eqnarray*}
y_{i}|z_{i} = 1   &\sim& \mbox{Bin}(K,p) \\
y_{i} | z_{i} = 0 &\sim& \delta(0) 
\end{eqnarray*}
It is convenient to express the conditional-on-$z$ observation model concisely as:
\[
 y_{i}|z_{i} \sim \mbox{Bin}(K, p z_{i})
\]
Thus, if $z_{i}=0$ then the success probability of the binomial
distribution is identically 0 whereas, if $z_{i}=1$, then the success
probability is $p$. This is useful in describing the model in the {\bf
  BUGS}
language, as shown in Panel \ref{closed.panel.da4m0}.
 Note the last line of the model
specification  provides the expression for computing $N$ from the
data augmentation variables $z_{i}$.

\begin{panel}[htp]
\centering
\rule[0.15in]{\textwidth}{.03in}
%\begin{minipage}{5in}
{\small
\begin{verbatim}
model{
p  ~ dunif(0,1)
psi~dunif(0,1)

# nind = number of individuals captured at least once
#   nz = number of uncaptured individuals added for PX-DA
for(i in 1:(nind+nz)) {
    z[i]~dbern(psi)
   mu[i]<-z[i]*p
    y[i]~dbin(mu[i],K)
 }

N<-sum(z[1:(nind+nz)])
}
\end{verbatim}
}
%\end{minipage}
\rule[-0.15in]{\textwidth}{.03in}
\caption{Model $M_{0}$ under data augmentation.}
\label{closed.panel.da4m0}
\end{panel}




Specification of a more general model in terms of the individual
encounter observations $y_{ik}$ is not much more difficult than for
the individual encounter frequencies.  We define the
observation model by a double loop and change the indexing of things
accordingly, i.e.,
\begin{verbatim}
for(i in 1:(nind+nz)) {
    z[i]~dbern(psi)
  for(k in 1:K){
      mu[i,k]<-z[i]*p
      y[i,k]~dbin(mu[i,k],1)
  }
}
\end{verbatim}
In this manner, it is straightforward to incorporate covariates on $p$ XXX for both individuals and sampling occasions XXX
(see discussion of this below and also Chapt. \ref{chapt.covariates} 
and consider other extensions.

\subsection{Formal development of data augmentation}

Use of DA for solving inference problems with unknown $N$ can be
justified as originating from the choice of uniform prior on $N$.  The
$\mathrm{Unif}(0,M)$ prior for $N$ is innocuous in the sense that the
posterior associated with this prior is equal to the likelihood for
sufficiently large $M$.  One way of inducing the $\mathrm{Unif}(0,M)$
prior on $N$ is by assuming the following hierarchical prior:
\begin{eqnarray}
\label{closed.eq.NgivenM}
  N &\sim& \mathrm{Bin}(M, \psi) \\ \nonumber
  \psi &\sim& \mathrm{Unif}(0,1)
\end{eqnarray}
which includes a new model parameter $\psi$ XXX (note that we have seen $\psi$ in the previous section as the proportion $E[N]/M$).XXX This parameter denotes
the probability that an individual in the super-population of size $M$
is a member of the population of $N$ individuals exposed to sampling.
The model assumptions, specifically the multinomial model 
(Eq. \ref{closed.eq.multinomial4m0})
and Eq. \ref{closed.eq.NgivenM}, may be combined to yield a
reparameterization of the conventional model that is appropriate for
the augmented data set of known size $M$:
\begin{equation}
\label{closed.eq.multinomial4DA}
    (n_1, n_2, \ldots, n_K) \sim \mathrm{Multin}(M, \psi  \pi(1), \psi \pi(2), \ldots, \psi \pi(K))
\end{equation}
This arises by removing $N$ from Eq. \ref{closed.eq.multinomial4m0} by 
integrating
over the binomial prior distribution for $N$. Thus, the models we
analyze under data augmentation arise formally by removing the
parameter $N$ from the ordinary model - the model conditional on $N$ -
by integrating over a binomial prior distribution for $N$.

Note that the $M-n$ unobserved individuals in the augmented data set
have probability $\psi \pi(0) + (1-\psi)$, indicating that these
unobserved individuals are a mixture of individuals that are sampling
zeros ($\psi \pi_0$, and belong to the population of size $N$) and
others that are ``structural zeros'' (occurring in the augmented data
set with probability $1 - \psi$). In Eq.~\ref{closed.eq.multinomial4DA} $N$
has been eliminated as a formal parameter of the model by
marginalization (integration) and replaced with the new parameter
$\psi$, the data augmentation parameter.
However, the full likelihood containing both $N$ and $\psi$ can also be
analyzed \citep[see][]{royle_etal:2007}.


\subsection{Remarks on Data Augmentation}

Data augmentation may seem like a strange and mysterious black-box,
and likely it is unfamiliar to most people, even those with substantial
experience with capture-recapture models. However, it really is a
formal reparameterization of capture-recapture models in which $N$ is
removed from the ordinary (conditional-on-$N$) model by integration.
In the case of Model $M_0$, data augmentation produces the zero-inflated
binomial which is distinct from the original observation model, but
only in the sense that it embodies, explicitly, the $\mbox{Unif}(0,M)$
prior for $N$.  Choice of $M$ might be cause for some concern related
to potential sensitivity to choice of $M$. The guiding principle is
that it should be chosen large enough so that the posterior for $N$ is
not truncated, but no larger because large values entail more
computational burden. It seems likely that the properties of the
Markov chains should be affected by $M$ and so some optimality might
exist \citep{gopalaswamy_etal:2012}, as in occupancy models
\citep{mackenzie_royle:2005}. Formal analysis of this is needed.


We emphasize the motivation for data augmentation being that it
produces a data set of fixed size, so that the parameter dimension in
any capture-recapture model is also fixed.  As a result, MCMC is a
relatively simple proposition using standard Gibbs Sampling.  Consider
the simplest context - analyzing Model $M_0$ using the occupancy type
model. In this case, DA converts Model $M_0$ to a basic occupancy model
and the parameters $p$ and $\psi$ have known full-conditional
distributions (in fact, beta distributions) that can be sampled from
directly.  Furthermore, the data augmentation variables - i.e., the 
data augmentation variables $z$, can be sampled from Bernoulli full
conditionals. MCMC is not too much more difficult for complicated
models - sometimes the hyperparameters need to be sampled using a
Metropolis-Hastings step, but nothing more sophisticated than that is
required.

There are other approaches to analyzing models with unknown $N$, using
reversible jump MCMC (RJMCMC) or other so-called ``trans-dimensional''
(TD) algorithms
 \citep{durban_elston:2005, king_brooks:2001, king_etal:2008,
schofield_barker:2008, wright_etal:2009}. What distinguishes DA from RJMCMC and
related TD methods is that DA is used to create a distinctly new model
that is unconditional on $N$ and we (usually) analyze the
unconditional model. The various TD/RJMCMC approaches seek to analyze
the conditional-on-$N$ model in which the dimensional of the parameter
space is a variable function of $N$. TD/RJMCMC approaches might appear
to have the advantage that one can model $N$ explicitly or consider
alternative priors for $N$. However, despite that $N$ is removed as an
explicit parameter in DA, it is possible to develop hierarchical
models that involve structure on $N$ \citep{converse_royle:2010,
  royle_etal:2011ms} which we consider in Chapt. \ref{chapt.hscr}.

\subsection{Example: Black Bear Study on Fort Drum}

To illustrate the analysis of Model $M_0$ using data augmentation, we use
a data set collected at Fort Drum Military Installation in upstate New
York by the Department of Defense, Cornell University and
colleagues. These data have been analyzed in various forms by
\citet{wegan:2008,gardner_etal:2009} and \citet{gardner_etal:2010jwm}.
The specific data used here are encounter histories on 47 individuals
obtained from an array of 38 baited ``hair snares''
(Fig. \ref{fig.3.bears1}) during June and July 2006.  Barbed wire
traps were baited and checked for hair samples each week for eight
weeks, thus we have $K=8$ sample intervals. The data are provided 
in the {\bf R} package \mbox{\tt scrbook} 
and the analysis can be set up and run as
follows. Here, the data were augmented with $M-n = 128$ ($M=175$)
all-zero encounter histories.

\begin{figure}
\centering
\includegraphics[height=2.5in,width=1.9in]{Ch3/figs/hairsnares.png}
\caption{Fort Drum study area and hair snare locations.}
\label{fig.3.bears1}
\end{figure}

{\small
\begin{verbatim}
library("scrbook")
data("beardata")
trapmat<-beardata$trapmat
nind<-dim(beardata$bearArray)[1]
K<-dim(beardata$bearArray)[3]
ntraps<-dim(beardata$bearArray)[2]

M=175
nz<-M-nind
Yaug <- array(0, dim=c(M,ntraps,K))

Yaug[1:nind,,]<-beardata$bearArray
y<- apply(Yaug,c(1,3),sum) # summarize by ind x rep
y[y>1]<- 1             # toss out duplicate obs
ytot<-apply(y,1,sum)   # total encounters out of K
\end{verbatim}
}

The raw data object, \mbox{\tt beardata\$bearArray} is a 3-dimensional
array $\mbox{\tt nind} \times \mbox{\tt ntraps} \times K$ of
individual encounter events (i.e., $y_{ijk} = 1$ if individual $i$ was
encountered in trap $j$ during occasion $k$, and 0 otherwise).  For
fitting model $M_{0}$ or $M_{h}$ (see below), it is sufficient to
reduce the data to individual encounter frequencies which we have
labeled \mbox{\tt ytot} above.  The {\bf BUGS} model file along with
commands to fit the model are as follows:

{\small
\begin{verbatim}
set.seed(2013)               # to obtain the same results each time
library("R2WinBUGS")
data0<-list(y=y,M=M,K=K)
params0<-list('psi','p','N')
zst=c(rep(1,nind),rbinom(M-nind, 1, .5))
inits =  function() {list(z=zst, psi=runif(1), p=runif(1)) }

cat("
model {

psi~dunif(0, 1)
p~dunif(0,1)

for (i in 1:M){
   z[i]~dbern(psi)
   for(k in 1:K){
      tmp[i,k]<-p*z[i]
      y[i,k]~dbin(tmp[i,k],1)
       }
       }
N<-sum(z[1:M])
}
",file="modelM0.txt")

fit0 = bugs(data0, inits, params0, model.file="modelM0.txt",
       n.chains=3, n.iter=2000, n.burnin=1000, n.thin=1,
       debug=TRUE,working.directory=getwd())
\end{verbatim}
}
This produces the follow posterior
 summary statistics:
{\small
\begin{verbatim}
> print(fit0,digits=2)
Inference for Bugs model at "modelM0.txt", fit using WinBUGS,
 3 chains, each with 2000 iterations (first 1000 discarded)
 n.sims = 3000 iterations saved
           mean    sd   2.5%    25%    50%    75%  97.5% Rhat n.eff
psi        0.29  0.04   0.22   0.26   0.29   0.31   0.36    1  3000
p          0.30  0.03   0.25   0.28   0.30   0.32   0.35    1  3000
N         49.94  1.99  47.00  48.00  50.00  51.00  54.00    1  3000
deviance 489.05 11.28 471.00 480.45 488.80 495.40 513.70    1  3000

[.. some output deleted ...]
\end{verbatim}
}
{\bf WinBUGS} did well in choosing an MCMC algorithm for this model --
we have $\hat{R} = 1$ for each parameter, and an effective sample size
of 3000, equal to the total number of posterior samples.
We see that the posterior mean of $N$ under this
model is $49.94$ and a 95\% posterior interval is $(48,54)$.  We
revisit these data later in the context of more complex models.



In order to obtain an estimate of density, $D$, we need an area to
associate with the estimate of $N$, XXXX and in Chapt. \ref{chapt.intro} we already went through a number of commonly used procedures to
conjure up such an area, including buffering the trap array by the home
range radius, often estimated by the mean maximum distance moved
(MMDM) \citep{parmenter_etal:2003},
$1/2$ MMDM \citep{dice:1938} or
directly from telemetry data (REF XXX NEED REF HERE, WALLACE ET AL 2003 DO THIS; I HAVE SEEN 2 PAPERS CITING OTIS ET AL 1978 IN THIS CONTEXT BUT I ONLY FOUND THE SECITON WHERE THEY SUGGEST USING INFORMATION ON ANIMAL HOME RANGE AS OBTAIN FROM TRAPPING DATA; I GUESS THIS DICE GUY SAID TO USE THE HOME RANGE RADIUS AND PEOPLE JUST TRY TO GET AT THIS WHICHEVER WAY THEY CAN; BE IT RECAPTURES OR OTHER HOME RANGE INFORMATION XXXXX).
Typically, the trap
array is defined by the convex hull around the trap locations, and
this is what we applied a buffer to. We computed the buffer by using
an estimate of the mean female home range radius (2.19 km) estimated from
telemetry studies \citep{bales_etal:2005} instead of using an estimate
based on our relatively more sparse recapture data.
 For the Fort Drum study, the convex hull has area
$157.135$ $km^2$, and the buffered convex hull has area $277.011$
$km^2$.
To create this we used functions contained in the {\bf R} package
\mbox{\tt rgeos} and created a utility function \mbox{\tt bcharea}
which is in our {\bf R} package \mbox{\tt scrbook}. The commands are
as follows:
\begin{verbatim}
library("rgeos")

bcharea<-function(buff,traplocs){
p1<-Polygon(rbind(traplocs,traplocs[1,]))
p2<-Polygons(list(p1=p1),ID=1)
p3<-SpatialPolygons(list(p2=p2))
p1ch<-gConvexHull(p3)
 bp1<-(gBuffer(p1ch, width=buff))
 plot(bp1, col='gray')
 plot(p1ch, border='black', lwd=2, add=TRUE)
 gArea(bp1)
}

bcharea(2.19,traplocs=trapmat)
\end{verbatim}
The resulting buffered convex hull is shown in Fig. \ref{closed.fig.bch}.
\begin{figure}
\begin{center}
\includegraphics[height=3in,width=3in]{Ch3/figs/bufferedCH}
\end{center}
\caption{buffered convex hull of the bear hair snare array}
\label{closed.fig.bch}
\end{figure}

To conjure up a
density estimate under model $M_0$, we compute the appropriate
posterior summary of $N$ and the prescribed area ($277.011$ $km^2$):
\begin{verbatim}
> summary(fit0$sims.list$N/277.011)
   Min. 1st Qu.  Median    Mean 3rd Qu.    Max.
 0.1697  0.1733  0.1805  0.1803  0.1841  0.2130

> quantile(fit0$sims.list$N/277.011,c(0.025,0.975))
     2.5%     97.5%
0.1696684 0.1949381
\end{verbatim}
which yields a density estimate of about $0.18$ ind/km$^2$, and a $95\%$ Bayesian
confidence interval of $(0.170, 0.195)$.

In summary, we have an estimate of density if we have faith in our
stated value of the ``sample area''. Clearly though this is largely
subjective, and not something we can formally evaluate from the data.
How certain are we of this area? Can
we quantify our uncertainty about this quantity? 
 More important, what exactly is
the meaning of this area and, in this context, how do we gauge bias
and/or variance of ``estimators'' of it? (i.e., what is it
estimating?).  
XXX I DON'T KNOW IF IT'S WORTH MENTIONING THE DELTA APPROXIMATION KARANTH AND NICHOLS (1998) USE XXX
There is no theory to guide us in trying to answer these important questions.


\section{Temporally varying and behavioral effects}

The purpose of this chapter is mainly to emphasize the central
importance of the binomial model in capture-recapture and so we have
considered models for individual encounter frequencies - the number of
times individuals are captured out of $K$ samples.  Sometimes it is
not acceptable to aggregate the encounter data for each individual --
such as when encounter probability varies over time among samples. 
Time-varying responses that are relevant in many
capture-recapture studies are ``effort'' such as amount of search time,
number of observers, or trap nights, or when encounter probability
varies over time or as a function of date or season due to species behavior
\citep{kery_etal:2010}.
  A common situation in a large number of carnivore studies is that in
which there exists a ``behavioral response'' to trapping (even if the
animal is not physically trapped).
XXXX IS THERE ANY PARTICULAR REASON WHY YOU ONLY REFER TO CARNIVORES HERE? XXXX

Behavioral response is an important concept in carnivore studies
because individuals might learn to come to baited traps or avoid traps
due to trauma related to being encountered.  There are a number of
ways to parameterize a behavioral response to encounter. The
distinction between persistent and ephemeral was made by
\citet{yang_chao:2005} who considered a general behavioral response
model of the form:
\[
\mbox{logit}(p_{ik}) = \alpha_{0} + \alpha_{1}*y_{i,k-1} + \alpha_{2} x_{ik}
\]
where $x_{ik}$ is a covariate indicator variable of previous capture
(i.e., $x_{ik} = 1$ if captured in any previous period). Therefore,
encounter probability changes depending on whether an individual was
captured in the immediate previous period (ephemeral behavioral
response XXX described by the term $\alpha_{1}*y_{i,k-1}$) or in any previous period (persistent behavioral
response). The former probably models a behavioral response due to
individuals moving around their territory relatively slowly over time
and the latter probably accommodates trap happiness due to baiting or
shyness due to trauma.   XXX Spatial capture-recapture models allow us to include trap-specific covariates, XXX and in such models it makes
sense to consider a local behavioral response that is trap-specific
\citep{royle_etal:2011jwm} - that is, the encounter probability is
modified for an individual trap depending on previous capture in
that trap.

Models with temporal effects are easy to describe in the {\bf BUGS} language
and analyze and we provide a number of examples in
Chapt. \ref{chapt.covariates} and elsewhere. 


\section{ Models with individual heterogeneity}
\label{closed.sec.modelmh}

Here we consider models with individual-specific encounter probability
parameters, say $p_{i}$, which we model according to some probability
distribution, $g(\theta)$. We denote this basic model assumption as
$p_{i} \sim g(\theta)$. This type of model is similar in concept to
extending a GLM to a GLMM but in the capture-recapture context $N$ is
unknown.  The basic class of models is often referred to as ``Model
$M_h$'' but really this is a broad class of models, each being
distinguished by the specific distribution assumed for $p_{i}$.  There
are many different varieties of Model $M_{h}$ including parametric and
various putatively non-parametric approaches
\citep{burnham_overton:1978, norris_pollock:1996, pledger:2000}. One
important practical matter is that estimates of $N$ can be extremely
sensitive to the choice of heterogeneity model
\citep{fienberg_etal:1999, dorazio_royle:2003, link:2003}. Indeed,
\citet{link:2003} showed that in some cases it's possible to find
models that yield precisely the same expected data, yet produce wildly
different estimates of $N$. In that sense, $N$ for most practical
purposes is not identifiable across classes of mixture models, and
this should be understood before fitting any such model. One solution
to this problem is to seek to model explicit factors that contribute
to heterogeneity, e.g., using individual covariate models (See
\ref{closed.sec.indcov} below). Indeed, spatial capture-recapture
models seek to do just that, by modeling heterogeneity due to the
spatial organization of individuals in relation to traps or other
encounter mechanism.  For additional background and applications of
Model $M_{h}$ see \citet[][chapt. 6]{royle_dorazio:2008} and
\citet[][chapt. 6]{kery_schaub:2011}.

Model $M_{h}$ has important historical relevance to spatial
capture-recapture situations \citep{karanth:1995} because
investigators recognized that the juxtaposition of individuals with
the array of trap locations should yield heterogeneity in encounter
probability, and thus it became common to use some version of Model $M_h$
in spatial trapping arrays to estimate $N$.  While this doesn't
resolve the problem of not knowing the area relevant to $N$, it does
yield an estimator that accommodates the heterogeneity in $p$ induced
by the spatial aspect of capture-recapture studies.

To see how this juxtaposition induces heterogeneity, we have to
understand the relevance of movement in capture-recapture models.
Imagine a quadrat that can be uniformly searched by a crew of
biologists for some species of reptile (see
\citet{royle_young:2008}).  Figure \ref{closed.fig.quadrat} shows a
sample quadrat searched repeatedly over a period of time. Further,
suppose that species exhibits some sense of spatial fidelity in the
form of a home range or territory, and individuals move about their
home range (home range centroids are given by the blue dots) in some
kind of random fashion.  
%It is natural to think about it in terms of a
%movement process and sometimes that movement process can be modeled
%explicitly using hierarchical models \citep{royle_young:2008,
%  royle_etal:2011mee}.  
Heuristically, we imagine that each individual in
the vicinity of the study area is liable to experience variable
exposure to encounter due to the overlap of its home range with the
sampled area - essentially the long-run proportion of times the
individual is within the sample plot boundaries, say $\phi$. We
might model the exposure of an individual to capture by supposing that
$z_{i} = 1$ if individual $i$ is available to be captured (i.e.,
within the survey plot) during any sample, and $0$ otherwise. Then,
$\Pr(z_{i}=1) = \phi$.  In the context of spatial studies, it is
natural that $\phi$ should depend on {\it where} an individual lives,
i.e., it should be individual-specific $\phi_{i}$
\citep{chandler_etal:2011}. This system describes, precisely, that of
``random temporary emigration'' \citep{kendall_etal:1997} where $\phi_{i}$
is the individual-specific probability of being ``available'' for
capture.

Conceptually, SCR models aim to deal with
this problem of variable exposure to sampling due to movement in the
proximity of the trapping array explicitly and formally with auxiliary
spatial information.  If individuals are detected with probability
$p_{0}$, {\it conditional} on $z_{i} = 1$, then the marginal
probability of detecting  individual $i$ is
\[
 p_{i} = p_{0}\phi_{i}
\]
so we see clearly that individual heterogeneity in encounter
probability is induced as a result of the juxtaposition of individuals
(i.e., their home ranges) with the sample apparatus and the movement
of individuals about their home range.

\begin{figure}
\begin{center}
\includegraphics[height=3in]{Ch3/figs/quadrat}
\end{center}
\caption{A quadrat searched for lizards and the locations of each
  lizard over some period of time.}
\label{closed.fig.quadrat}
\end{figure}

We will work with a specific type of Model $M_{h}$ here, that in which
we extend the basic binomial observation model of Model $M_{0}$ so
that
\[
\mbox{logit}(p_{i}) = \mu + \eta_{i}
\]
where
\[
\eta_{i} \sim \mbox{Normal}(0, \sigma_{p}^2)
\]
We could as well write
\[
\mbox{logit}(p_{i}) \sim \mbox{Normal}(\mu,\sigma_{p}^2)
\]
This ``logit-normal mixture'' was analyzed by
\citet{coull_agresti:1999} and elsewhere. It is a natural extension of
the basic model with constant $p$, as a mixed GLMM, and similar models
occur throughout statistics. It is also natural to consider a beta
prior distribution for $p_{i}$ \citep{dorazio_royle:2003} and
so-called ``finite-mixture'' models XXX (models in which individuals are assumed to belong to a finite number of latent classes, each of which has its own capture probability) XXX are also popular
\citep{norris_pollock:1996, pledger:2000}.

\subsection{Analysis of Model $M_h$}

If $N$ is known, it is worth taking note of the essential simplicity
of model $M_h$ as a binomial GLMM.  This is a type of model that is
widely applied in just about every scientific discipline and using
standard methods of inference based either on integrated likelihood
\citep{laird_ware:1982, berger_etal:1999} which we discuss in
Chapt. \ref{chapt.mle} or standard Bayesian
methods. However, because $N$ is not known, inference is somewhat more
challenging. We address that here using Bayesian analysis based on
data augmentation (DA). Although we use data augmentation in the context of
Bayesian methods here, we note that
heterogeneity models formulated under DA are easily analyzed by
conventional likelihood methods as zero-inflated binomial mixtures
\citep{royle:2006} and more traditional analysis of model $M_h$ based on
integrated likelihood, without using data augmentation, has been
considered by \citet{coull_agresti:1999}, \citet{dorazio_royle:2003},
and others.

As with model $M_{0}$, we have the Bernoulli model for the
zero-inflation variables: $z_{i} \sim \mbox{Bern}(\psi)$ and the model
of the observations expressed conditional on the latent variables
$z_{i}$. For $z_{i}=1$, we have a binomial model with
individual-specific $p_{i}$:
\[
y_{i}|{z_{i} \! = \! 1} \sim \mbox{Bin}(K,p_{i})
\]
and otherwise $y_{i} |{ z_{i} \! = \! 0} \sim \delta(0)$. Further, we
prescribe a distribution for $p_{i}$. Here we assume
\[
\mathrm{logit}(p_{i}) \sim \mbox{Normal}(\mu,\sigma^2)
\]
The basic {\bf BUGS} description for this model, assuming a
$\mbox{Unif}(0,1)$ prior for $p_{0} = \mbox{logit}^{-1}(\mu)$, is given
as follows:
{\small
\begin{verbatim}
model{

p0 ~ dunif(0,1)       # prior distributions
mup<- log(p0/(1-p0))
taup~dgamma(.1,.1)
psi~dunif(0,1)

for(i in 1:(nind+nz)){
  z[i]~dbern(psi)     # zero inflation variables
  lp[i] ~ dnorm(mup,taup) # individual effect
  logit(p[i])<-lp[i]
  mu[i]<-z[i]*p[i]
  y[i]~dbin(mu[i],J)  #  observation model
 }

N<-sum(z[1:(nind+nz)])  # N is a derived parameter
}
\end{verbatim}
}


\subsection{Analysis of the Fort Drum data}

The logit-normal heterogeneity model was fitted to the bear data from
the Fort Drum study, and we used data augmentation to produce a data
set of $M=500$ individuals.  We ran the model using {\bf JAGS} with
the instructions given as follows:
{\small
\begin{verbatim}
[... get data as before ....]

set.seed(2013)

cat("
model{
p0 ~ dunif(0,1)       # prior distributions
mup<- log(p0/(1-p0))
sigmap ~ dunif(0,10)
taup<- 1/(sigmap*sigmap)
psi~dunif(0,1)

for(i in 1:(nind+nz)){
  z[i]~dbern(psi)     # zero inflation variables
  lp[i] ~ dnorm(mup,taup) # individual effect
  logit(p[i])<-lp[i]
  mu[i]<-z[i]*p[i]
  y[i]~dbin(mu[i],K)  #  observation model
 }

N<-sum(z[1:(nind+nz)])
}
",file="modelMh.txt")

data1<-list(y=ytot, nz=nz, nind=nind,K=K) 
params1= c('p0','sigmap','psi','N')
inits =  function() {list(z=as.numeric(ytot>=1), psi=.6, p0=runif(1),
          sigmap=runif(1,.7,1.2),lp=rnorm(M,-2)) }

library("rjags")
jm<- jags.model("modelMh.txt", data=data1, inits=inits, n.chains=4,
                 n.adapt=1000)
jout<- coda.samples(jm, params1, n.iter=200000, thin=1)
\end{verbatim}
}
This produces the posterior distribution for $N$ shown
in Fig. \ref{closed.fig.bearMh}. Posterior summaries of parameters are
given as follows:
{\small
\begin{verbatim}
> summary(jout)

Iterations = 2001:202000
Thinning interval = 1 
Number of chains = 4 
Sample size per chain = 2e+05 

1. Empirical mean and standard deviation for each variable,
   plus standard error of the mean:

           Mean       SD  Naive SE Time-series SE
N      117.7740 56.31633 6.296e-02       1.960115
p0       0.0728  0.05522 6.174e-05       0.001655
psi      0.2366  0.11362 1.270e-04       0.003909
sigmap   2.0795  0.53096 5.936e-04       0.016789

2. Quantiles for each variable:

            2.5%      25%       50%      75%    97.5%
N      62.000000 82.00000 102.00000 134.0000 277.0000
p0      0.003143  0.02842   0.06077   0.1066   0.2036
psi     0.117269  0.16377   0.20522   0.2712   0.5560
sigmap  1.211900  1.69434   2.02113   2.4028   3.2694
\end{verbatim}
}


We used $M=500$ for this analysis and we
note that  while the posterior mass of $N$ is concentrated away from this
upper bound (Fig. \ref{closed.fig.bearMh}), the posterior has an
extremely long right tail, with some posterior values at the upper
bound $N=500$. Maybe or
maybe not sufficient data augmentation.\footnote{
{\bf to do: } insert final results. longer run. more data
augmentation. compare with winbugs.
}
The model runs effectively in {\bf WinBUGS} but sometimes with apparently
inefficient mixing for reasons that may be related to bad starting
values. In some cases this was resolved if we supplied starting values
for the $logit(p_{i})$ parameters and $\tau$.


Because of the skewed posterior we see that the posterior mean ($N=117$)
is
considerably higher than the posterior mode ($N=102$). Moreover, 
posterior summaries are estimated with a relatively high error
(``Time-series SE'' of around 2.0)\footnote{need to define this somewhere XXX THIS COMES UP IN CH2 XXX}.
Further, it may be surprising that the posterior mode does not compare
well with the MLE. To compute the posterior mode we could easily find
the posterior value of $N$ with the highest mass because $N$ is
discrete. But we want to smooth out some of the Monte Carlo error a
bit so we used a smoothing spline to the posterior frequencies of $N$
as follows:
\begin{verbatim}
  tt<-table(jout[[1]][,"N"])[1:80]
  xg<-as.numeric(names(tt))
  plot(xg,tt)
  sp<- smooth.spline(xg,tt,df=9)
  sp$x[sp$y==max(sp$y)]
[1] 80
\end{verbatim}
The \mbox{\tt df} argument controls the degree of smoothing and we
find in this case that the modal value (i.e., 80) is not too sensitive
to the smoothing parameter but this should be checked in any specific
instance\footnote{we need to give examples of using \mbox{\tt
    density()} to obtain modes}.

To compute the MLE, we used 
the {\bf R} code contained in Panel 6.1 of \citet{royle_dorazio:2008}.  The
MLE of $log(n_{0})$, the logarithm of the number of uncaptured
individuals, is $\widehat{log(n0)} = 3.86$ and therefore $\hat{N} =
exp(3.86)+47 = 94.47$ which is not at all consistent with the apparent
mode in 
Fig. \ref{closed.fig.bearMh}.
\footnote{We note that the result is inconsistent with Gardner et
  al. (2009) who reported an MLE of 104.1 ($density = 0.437
  inds/km^2$) although we do not know the reason for this at the
  present time.}  
%To convert this to density we use the buffered area
%as computed above (255.3 $km^2$)\footnote{WRONG \#} and perform the
%required summary analysis on the posterior samples of $N$, which
%results in about $0.37$ individuals/$km^2$. The reader should carry
%out this analysis to confirm the estimates, and also obtain the $95\%$
%confidence interval.

{\bf Remarks:} First of all the posterior for this model and data set is
very sensitive to prior distributions. While MLEs are invariant to
transformation of the parameters, the posterior distribution
definitely is {\it not} invariant. In the present case, the use of a
$\mbox{Unif}(0,1)$ prior for $p_{0} = \mbox{expit}(\mu)$ is somewhat
informative -- in particular, it is not at all ``flat'' on the scale
of $\mu$ -- and this affects the posterior.  We generally always
recommend use of a $\mbox{Unif}(0,1)$ prior for $\mbox{expit}(\mu)$ in such
models. That said, we were surprised at this result, and we
experimented with other prior configurations including putting a flat
prior on $\mu$ directly. That specific prior suggests the possibility
that the posterior distribution may be improper for that prior
specification. This kind of small sample instability has been widely
noted in model $M_h$ \citep{fienberg_etal:1999, dorazio_royle:2003} and
is not unrelated to sensitivity to
model XXX WORD MISSING? XXX which has also been identified as an important issue in model
$M_{h}$ \citep{dorazio_royle:2003,link:2003}.
Conclusion: The mode is well-defined but the data set is sparse and
hence inferences are poor and sensitive to model choices. Get over it.


\begin{figure}
\centering
\includegraphics[height=4.5in,width=4.5in]{Ch3/figs/bear-modelMh-post}
\caption{Posterior of $N$ for Fort Drum bear study data under the
logit-normal version of model $M_h$. 
}
\label{closed.fig.bearMh}
\end{figure}


\subsection{Building your own MCMC algorithm}

For fun, we construct our own MCMC algorithm using a Metropolized
Gibbs sampler for model $M_{h}$ in Chapt. \ref{chapt.mcmc}, where we
also develop the MCMC 
algorithms for spatial capture-recapture models.
XXX MAYBE PUT THIS IN A FOOTNOTE? XXX

\begin{comment}

To begin, we first collect all of our model components
which are as follows: $[y_{i}| p_{i},z_{i}]$,
$[p_{i}|\mu_{p},\sigma_{p}]$, and $[z_{i}|\psi]$
for {\it each} $i=1,2,\ldots,M$ and then prior distributions
$[\mu_{p}]$, $[\sigma_{p}]$ and $[\psi]$.
The joint posterior distribution of all unknown quantities in the model
is proportional to the joint distribution of all elements
$y_{i},p_{i},z_{i}$ and also the prior distributions of the prior parameters:
\[
\left\{ \prod_{i=1}^{M} [y_{i}|p_{i},z_{i}][p_{i}|\mu_{p},\sigma_{p}]
[z_{i}|\psi] \right\} [\mu_{p},\sigma_{p},\psi]
\]
For prior distributions, we assume that $\mu_{p},\sigma_{p}, \psi$ are
mutually independent and for $\mu_{p}$ and $\sigma_{p}$ we use
improper uniform priors, and $\psi \sim \mbox{Unif}(0,1)$.  Note that
the likelihood contribution for each individual, when conditioned on
$p_{i}$ and $z_{i}$, does not depend on $\psi$, $\mu_{p}$, or
$\sigma_{p}$.  As such, the full-conditionals for the structural
parameters $\psi$ only depends on the collection of data augmentation
variables $z_{i}$, and that for $\mu_{p}$ and $\sigma_{p}$ will only
depends on the collection of latent variables $p_{i}; i=1,2,\ldots,M$.
The full conditionals for all the unknowns are as follows:

{\bf (1)} For $p_{i}$:
\begin{eqnarray*}
[p_{i}|y_{i}, \mu_p, \sigma_{p},z_{i}=1] &\propto  &
[y_{i}|p_{i}][p_{i}|\mu_p,\sigma_{p}^{2}] \mbox{ if $z_{i}=1$ }  \\
                 &  &  [p_{i}|\mu_p,\sigma_{p}] \mbox{if $z_{i}=0$ }
\end{eqnarray*}

{\bf (2)} for $z_{i}$:
\[
z_{i} | \cdot \propto [y_{i}|z_{i}*p_{i}] \mbox{Bern}(z_{i}|\psi)
\]

{\bf (3)} For $\mu_{p}$:
\[
[\mu_{p} | \cdot ] \sim \prod_{i} [p_{i}| \cdot] *\mbox{const}
\]


{\bf (4)} For $\sigma_{p}$:
\[
[ \sigma_{p}|\cdot ] \sim\prod_{i}[p_{i}| \cdot ]*\mbox{const}
\]

{\bf (5)} For $\psi$:
\[
\psi|\cdot\sim \mbox{Beta}(1 + \sum z_{i}, 1 + M - \sum z_{i})
\]


We've  identified each of the full conditional
distributions in sufficient detail to apply the
Metropolis-Hastings algorithm. With the exception of $\psi$ which has
a convenient analytic solution -- it is a beta distribution which we
can easily sample directly. In truth, we could also sample $\mu_{p}$
and $\sigma_{p}^{2}$ directly with certain choices of prior
distributions. For example, if $\mu_{p} \sim \mbox{Normal}(0, 1000)$
then the full conditional for $\mu_{p}$ is also normal, etc..
We implement an MCMC algorithm for this model in the following block
of {\bf R} code.  The basic structure is: initialize the parameters
and create any required output or intermediate data holders, and then
begin the main MCMC loop which, in this case, generates 100000
samples.\footnote{This data grabbing function is not implemented yet}

{\small
\begin{verbatim}
## obtain the bear data by executing the previous data grabbing
## function

temp<-getdata()
M<-temp$M
K<-temp$K
ytot<-temp$ytot

###
### MCMC algorithm for Model Mh

out<-matrix(NA,nrow=100000,ncol=4)
dimnames(out)<-list(NULL,c("mu","sigma","psi","N"))
lp<- rnorm(M,-1,1)
p<-expit(lp)
mu<- -1
p0<-exp(mu)/(1+exp(mu))
sigma<- 1
psi<- .5
z<-rbinom(M,1,psi)
z[ytot>0]<-1

for(i in 1:100000){

### update the logit(p) parameters
lpc<- rnorm(M,lp,1)  # 0.5 is a tuning parameter
pc<-expit(lpc)
lik.curr<-log(dbinom(ytot,K,z*p)*dnorm(lp,mu,sigma))
lik.cand<-log(dbinom(ytot,K,z*pc)*dnorm(lpc,mu,sigma))
kp<- runif(M) < exp(lik.cand-lik.curr)
p[kp]<-pc[kp]
lp[kp]<-lpc[kp]

p0c<- rnorm(1,p0,.05)
if(p0c>0 & p0c<1){
muc<-log(p0c/(1-p0c))
lik.curr<-sum(dnorm(lp,mu,sigma,log=TRUE))
lik.cand<-sum(dnorm(lp,muc,sigma,log=TRUE))
if(runif(1)<exp(lik.cand-lik.curr)) {
 mu<-muc
 p0<-p0c
}
}

sigmac<-rnorm(1,sigma,.5)
if(sigmac>0){
lik.curr<-sum(dnorm(lp,mu,sigma,log=TRUE))
lik.cand<-sum(dnorm(lp,mu,sigmac,log=TRUE))
if(runif(1)<exp(lik.cand-lik.curr))
 sigma<-sigmac
}

### update the z[i] variables
zc<-  ifelse(z==1,0,1)  # candidate is 0 if current = 1, etc..
lik.curr<- dbinom(ytot,K,z*p)*dbinom(z,1,psi)
lik.cand<- dbinom(ytot,K,zc*p)*dbinom(zc,1,psi)
kp<- runif(M) <  (lik.cand/lik.curr)
z[kp]<- zc[kp]

psi<-rbeta(1, sum(z) + 1, M-sum(z) + 1)

out[i,]<- c(mu,sigma,psi,sum(z))
}
\end{verbatim}
}


{\bf Remarks}: (1) for parameters with bounded support, i.e.,
$\sigma_{p}$ and $p_{0}$, we are using a random walk candidate
generator but rejecting draws outside of the parameter space.  (2) We
mostly use Metropolis-Hastings except for the data augmentation
parameter $\psi$ which we sample directly from its full-conditional
distribution which is a beta distribution.  (3) Even the latent data
augmentation variables $z_{i}$ are updated using Metropolis-Hastings
although they too can be updated directly from their full-conditional.
\end{comment}


\begin{comment}

\subsection{Exercises related to model Mh}

\begin{itemize}
\item[(1)] Enclose the MCMC algorithm in an R function and provide
  arguments for some of the parameters of the function that a user
  might wish to modify.
\item[(2)] Execute the function and compare the results to those
  generated from WinBUGS in the previous section
\item[(3)] Note that the prior distribution for the ``mean'' parameter
  is given on $p_0=exp(\mu)/(1+exp(\mu))$.  Reformulate the algorithm
  with a flat prior on $\mu$ and see what happens. Contemplate this.
\item[(4)] Using Bayes rule, figure out the full conditional for
  $z_{i}$ so that you don't have to use MH for that one. It might be
  more efficient. Is it?
\item[(5)] Modify the MCMC algorithm so that the prior for $\mu_{p}$
  is an improper flat prior. i.e., $[\mu_{p}] \propto 1$. Describe the
  posterior distribution of $N$. 
\end{itemize}

\end{comment}



\section{Individual Covariate Models: Toward Spatial Capture-Recapture}
\label{closed.sec.indcov}


A standard situation in capture-recapture models is when an individual
covariate is measured, and this covariate is thought to influence
encounter probability.  As with other closed population models, we
begin with the basic binomial observation model:
\[
y_{i} \sim \mbox{Bin}(K, p_{i})
\]
and we assume also  a model for encounter probability according to:
\begin{equation}
 \mbox{logit}(p_{i}) = \alpha + \beta x_{i}
\label{closed.eq.ha}
\end{equation}
Classical examples of covariates influencing detection probability are
type of animal (juvenile/adult or male/female), a continuous covariate
such as body mass \citep[][ch. 6]{royle_dorazio:2008}, or a
discrete covariate such as group or cluster size. For example, in
models of aerial survey data, it is natural to model detection
probabilities as a function of the observation-level individual
covariate, ``group size'' \citep{royle:2008, royle:2009,
  langtimm_etal:2011}.

Such ``individual covariate models'' are similar in structure to Model
$M_{h}$, except that the individual effects are {\it observed} for the
$n$ individuals that appear in the sample. These models are important
here because spatial capture-recapture models are precisely a form of
individual covariate model, an idea that we will develop here and
elsewhere. Specifically, they are such models, but where the
individual covariate is a partially observed latent variable for 
captured individuals. As such, it is a type of measurement error.
That is, unlike model $M_h$, we do have some direct information about the
latent variable, which comes from the spatial locations/distribution
of individual recaptures.

Traditionally, estimation of $N$ in individual covariate models is
achieved using methods based on ideas of unequal probability sampling
(i.e., Horwitz-Thompson estimation; see \citet{huggins:1989} and
\citet{alho:1990}). An estimator of $N$ is
\[
\hat{N} = \sum_{i}^{n} \frac{1}{\tilde{p}_{i}}
\]
where $\tilde{p}_{i}$ is the probability that individual $i$ appeared
in the sample.  That is, $\tilde{p}_{i} = \Pr(y_{i}>0)$
where, in closed population capture-recapture models, 
\[
\Pr(y_{i}>0) = (1- (1-p_{i})^K)
\]
where $p_{i}$ is a function of parameters $\alpha$ and $\beta$
according to
Eq. \ref{closed.eq.ha}.
In practice, parameters are estimated 
from the conditional-likelihood of the observed encounter histories
which is, for observation $y_{i}$, 
\[
{\cal L}_{c}(\alpha, \beta | y_{i}) = \frac{ \mbox{Bin}(y_{i}|\alpha,\beta) } { \tilde{p}_{i}}.
\]

Here we take a formal model-based approach to Bayesian analysis of
such models based on the joint likelihood
using data augmentation \citep{royle:2009}. Classical
likelihood analysis of the so-called ``full likelihood'' is covered 
 by \citet{borchers_etal:2002}.  For Bayesian analysis of
individual covariate models, because the individual covariate is
unobserved for the $N-n$ uncaptured individuals, we require a model to
describe variation among individuals, essentially allowing the sample
to be extrapolated to the population\footnote{weak argument}.  For our present purposes, we
consider a continuous covariate and we assume that it has a normal
distribution:
\[
x_{i} \sim \mbox{Normal}(\mu,\sigma^{2})
\]

Data augmentation can be applied directly to this class of models. In
particular, reformulation of the model under DA yields a basic
zero-inflated binomial model of the form:
\begin{eqnarray*}
z_{i} &\sim& \mbox{Bern}(\psi) \; \; \; i=1,2,\ldots,M\\
y_{i}|{z_{i}\! =\! 1} &\sim& \mbox{Bin}(K,p_{i}(x_{i})) \\
y_{i} |{ z_{i}\! =\! 0} &\sim& \delta(0)  \\
x_{i} & \sim & \mbox{Normal}(\mu,\sigma^{2})
\end{eqnarray*}
Fully spatial capture-recapture models use this
formulation with a latent covariate that is directly related to the
individual detection probability (see next section). As with the
previous models, implementation is trivial in the {\bf BUGS} language. The
{\bf BUGS} specification is very similar to that for model $M_h$, but we
require the distribution of the covariate to be specified, along with
priors for the parameters of that distribution.


\subsection{Example: Location of capture as a covariate.}

If we had a regular grid of traps over some closed geographic system
then we imagine that the average location of capture would be a decent
estimate (heuristically) of an individual's home range center.
Intuitively some measure of typical distance from home range center to
traps for an individual should be a decent covariate to explain
heterogeneity in encounter probability, i.e., individuals with more
exposure to traps should have higher encounter probabilities and vice
versa.  A version of this idea was put forth by
\citet{boulanger_mclellan:2001} (see also \citet{ivan:2012}), but
using the Huggins-Alho estimator and with covariate ``distance to
edge'' of the trapping array. A limitation of this  approach is
that it does not provide a solution to the problem that the trap area
is fundamentally ill-defined, nor does it readily accommodate the
inherent and heterogeneous variation in this measured covariate.

Here, we provide an example of this type of heuristically motivated
approach using the fully model-based individual covariate model
described above analyzed by data augmentation. We take a slightly
different approach than that adopted by
\citet{boulanger_mclellan:2001}. By analyzing the full likelihood and
placing a prior distribution on the individual covariate, we resolve
the problem of having an ill-defined area over which the population
size is distributed. After you read later chapters of this book, it
will be apparent that SCR models represent a formalization of this
heuristic procedure.

For our purposes here, we define $x_{i} = ||{\bf s}_{i} - {\bf
  x}_{0}||$ where ${\bf s}_{i}$
is the average encounter location of individual $i$ and ${\bf x}_{0}$ is the
centroid of the trap array.  Conceptually, individuals in the middle
of the array should have higher probability of encounter and, as
$x_{i}$ increases, $p_{i}$ should therefore decrease. We note that we
have defined ${\bf s}_{i}$ in terms of a sample quantity - the observed mean
- which is ad hoc but consistent with existing applications in the literature.
For an expansive, dense trapping grid then we might expect the sample mean
encounter location to be a good estimate of home range center but,
clearly this is biased for individuals that live around the edge (or
off) the trapping array. Regardless, it should be good enough for our
present purposes of demonstrating this heuristically appealing
application of an individual covariate model. A key point is that
${\bf s}_{i}$ is missing for each individual that is not encountered and
thus so is $x_{i}$. Thus,
it is a latent variable, or random effect, and we need therefore to
specify a probability distribution for it.
As a measurement of distance we know it must be
positive-valued. Thinking about this like a distance sampling problem
lets first try to make $x_{i}$ uniform from $0$ to some large number,
say $D_{max}$, beyond which it would be difficult to imagine an
individual being captured. For example, $D_{max}$ should be at a home
range diameter past the furthest trap from the center.
As such, we use this distribution for the individual covariate
``distance from center of the trap array''
\[
 x_{i} \sim \mbox{Unif}(0,D_{max})
\]
where $D_{max}$ is a specified constant, which we may choose to be
arbitrarily large.  In practice, people have
used distance from edge of the trap array but that is less easy to
make sense of.


\subsubsection{Fort Drum Bear Study}


\begin{figure}
\centering
\includegraphics[height=3.5in,width=3.5in]{Ch3/figs/bear_spiderplot.png}
\caption{Spider plot of the Fort Drum study data.}
\label{closed.fig.spiderplot}
\end{figure}


We have to do a little bit of data processing to fit this individual
covariate model to the Fort Drum data. 
We need to compute the individual covariate ${\bf x}_{i}$ (distance from the centroid of the trapping
array) using the {\bf R} function
\mbox{\tt spiderplot}
provided in \mbox{\tt scrbook}. This function also produces a keen plot shown in
Fig. \ref{closed.fig.spiderplot} which we call a ``spider plot''.
The {\bf R} commands for obtaining the individual covariate ``distance from trap centroid''
are as follows:
\begin{verbatim}
library("scrbook")
data("beardata")
toad<- spiderplot(beardata$bearArray,beardata$trapmat)
xcent<-toad$xcent
\end{verbatim}
We picked $D_{max} = 11.5$ $km^2$ which is about the distance from the
array center to the furthest trap. 
Once we specific $D_{max}$ then the implication is that the population
size parameter applies to the area 
within 11.5 units of the trap centroid\footnote{To be convincing
  this might  need a little bit of hand-holding}. The {\bf BUGS} model
specification and {\bf R} commands to package the data and fit the model are
as follows:

{\small
\begin{verbatim}
cat("
model{
p0 ~ dunif(0,1)       # prior distributions
mup<- log(p0/(1-p0))
psi~dunif(0,1)
beta~dnorm(0,.01)

for(i in 1:(nind+nz)){
  xcent[i]~dunif(0,maxD)
  z[i]~dbern(psi)     # DA variables
  lp[i] <- mup + beta*xcent[i] # individual effect
  logit(p[i])<-lp[i]
  mu[i]<-z[i]*p[i]
  y[i]~dbin(mu[i],K)  #  observation model
 }
N<-sum(z[1:(nind+nz)])
}
",file="modelMcov.txt")

data2<-list(y=ytot,nz=nz,nind=nind,K=K,xcent=xcent,Dmax=maxD)
params2<-list('p0','psi','N','beta')
inits =  function() {list(z=zst, psi=psi, p0=runif(1),beta=rnorm(1) ) }
fit2 = bugs(data2, inits, params2, model.file="modelMcov.txt",working.directory=getwd(),    
       debug=T, n.chains=3, n.iter=11000, n.burnin=1000, n.thin=1)
\end{verbatim}
}

This produces the following posterior summaries:
{\small
\begin{verbatim}
Inference for Bugs model at "modelMcov.txt", fit using WinBUGS,
 3 chains, each with 11000 iterations (first 1000 discarded)
 n.sims = 30000 iterations saved
           mean    sd   2.5%    25%    50%    75%  97.5% Rhat n.eff
p0         0.54  0.07   0.40   0.50   0.54   0.59   0.67    1  1100
psi        0.34  0.05   0.25   0.31   0.34   0.37   0.44    1  3500
N         58.92  5.49  50.00  55.00  58.00  62.00  71.00    1  1900
beta      -0.25  0.06  -0.36  -0.29  -0.25  -0.21  -0.12    1   780
deviance 459.51 13.21 435.80 450.20 458.80 467.90 487.40    1  2600
\end{verbatim}
}


It might be 
perplexing that the estimated $N$ is much lower than obtained by model
$M_h$ but there is a good explanation for this, discussed
subsequently. That issue notwithstanding, it is worth pondering how
this model could be an improvement (conceptually or technically) over
some other model/estimator including $M_0$ and $M_h$ considered
previously. Well, for one, we have accounted formally for
heterogeneity due to spatial location of individuals relative to
exposure to the trap array, characterized by the centroid of the
array. Moreover, we have done so using a model that is based on an
explicit mechanism, as opposed to a phenomenological one such as Model
$M_h$. Moreover, importantly, using our new model, {\it the estimated N
  applies to an explicit area which is defined by our prescribed value
  of maxD}. That is, this area is a fixed component of the model and
the parameter $N$ therefore has explicit spatial context, as the number
of individuals with home range centers less than $D_{max}$ from the
centroid of the trap array. As such, the implied ``effective trap
area''\footnote{This is a bad use of this term. We have never defined
  ETA or ESA. What is it, exactly? XXX IT IS SOMEWHAT DEFINED IN CH1; IN THE QUOTE FROM OBRIEN; ALTHOUGH HE NAMES IT EFFECTIVE AREA XXX} for a given $D_{max}$ is that of a circle
with radius $D_{max}$.



\begin{figure}
\begin{center}
\includegraphics[width=3.5in]{Ch3/figs/Nchains}
\end{center}
\caption{Needs a caption}
\label{closed.fig.ha}
\end{figure}

\subsection{Extension of the Model}

This model is actually not a very good model for one important reason:
Imposing a uniform prior distribution on $x$
implies that density is {\it not constant} over space. In
particular, this model implies that it {\it decreases} as we move away
from the centroid of the trap array. 
That is, $x_{i} \sim \mbox{Unif}(0,D_{max})$ implies constant $N$ in
each distance band from the centroid but obviously the {\it area} of
each distance band is increasing.  
This is one reason we have a
lower estimate of density than that obtained previously from model $M_0$ and also why,
if we were to increase $D_{max}$, we would see density continue to
decrease.

Fortunately, the use of an individual covariate model is {\it not} restricted to
use of this specific distribution for the individual
covariate. Clearly, it is a bad choice and, therefore, we should think
about whether we can choose a better distribution for $D_{max}$ - one that
doesn't imply a decreasing density as distance from the centroid
increases.  Conceptually, what we want to do is impose a prior on
distance from the centroid, $x$, such that density is proportional to
the amount of area in each successive distance band as you move
farther away from the centroid.  In fact, there is theory that exists
which tells us what the correct distribution of $x$ is
$2x/D_{max}^2$. This can be derived by noting that $F(x) = \Pr(X<x) =
\pi*x*x/\pi*D_{max}^{2}$ . Then, $f(x) = dF/dx =
2*x/(D_{max}^{2})$. This is a sort of triangular distribution in
density
induced because the incremental area in each additional distance band
increases linearly with radius (i.e., distance from centroid). It is
sometimes comforting to verify things empirically:
{\small
\begin{verbatim}
 u<-runif(10000,-1,1)
 v<-runif(10000,-1,1)
 d<- sqrt(u*u+v*v)
 hist(d[d<1])
 hist(d[d<1],100)
 hist(d[d<1],100,probability=TRUE)
 abline(0,2)
\end{verbatim}
}

It would be useful if we could describe this distribution in {\bf BUGS} but
there is not a built-in way to do this that we are aware of.  One possibility is to use a
discrete version of the pdf. We might also be able to use what is
referred to in {\bf WinBUGS} jargon as the ``zeros trick'' (see {\it Advanced
BUGS tricks} in the manual) although we haven't pursued this approach. Instead, we
consider using a discrete version and break $D_{max}$ into $L$ distance
classes of width $\delta$, with probabilities proportional to
$2*x$. In particular, if we denote the cut-points by $xg_{1}=0,xg_{2}, \ldots,
xg_{L+1}=D_{max}$ and the interval midpoints are $xm_{i} = 
xg_{i+1}-\delta$ then the interval probabilities are $p_{i} = 
2*xm_{i}*\delta/(D_{max}^{2})$, which we can compute once and then pass
them to {\bf WinBUGS} as data.

The {\bf R} commands for doing all of this (noting that we have already loaded and processed
the Fort Drum bear data) are given  as follows. In the model description the
variable $x$ (observed distance from centroid of the trap array) has been rounded so that the
discrete version of the $f(x)$ can be used as described
previously. The new variable labeled \mbox{\tt xround} is actually
then the integer category label in units of $\delta$ from 0. Thus, to
convert back to distance in the expression for $lp[i]$, \mbox{\tt
  xround[i]} has to be multiplied by $\delta$. Here is the {\bf BUGS} model 
  specification:
{\small
\begin{verbatim}
delta<-.2
xround<-xcent%/%delta  + 1
Dgrid<- seq(delta,maxD,delta)
xprobs<- delta*(2*Dgrid/(maxD*maxD))
xprobs<-xprobs/sum(xprobs)

cat("
model{
p0 ~ dunif(0,1)       # prior distributions
mup<- log(p0/(1-p0))
psi~dunif(0,1)
beta~dnorm(0,.01)

for(i in 1:(nind+nz)){
  xround[i]~dcat(xprobs[])
  z[i]~dbern(psi)                     # zero inflation variables
  lp[i] <- mup + beta*xround[i]*delta # individual effect
  logit(p[i])<-lp[i]
  mu[i]<-z[i]*p[i]
  y[i]~dbin(mu[i],K)  #  observation model
 }

N<-sum(z[1:(nind+nz)])
}
",file="modelMcov.txt")
\end{verbatim}
}

To fit the model we do this - keeping in mind that the data objects
required below have been defined in previous analyses of this chapter:
{\small
\begin{verbatim}
data2<-list(y=ytot,nz=nz,nind=nind,K=K,xround=xround,xprobs=xprobs,delta=delta)
params2<-list('p0','psi','N','beta')
inits =  function() {list(z=z, psi=psi, p0=runif(1),beta=rnorm(1) ) }
fit = bugs(data2, inits, params2, model.file="modelMcov.txt",
          working.directory=getwd(), debug=FALSE, n.chains=3, n.iter=11000, 
          n.burnin=1000, n.thin=2)
\end{verbatim}
}

This is a useful model because it induces a clear definition of area
in which the population of $N$ individuals reside. Under this model,
that area is defined by specification of $D_{max}$. We can apply the model
for different values of $D_{max}$ and observe that the estimated $N$ varies
with $D_{max}$. Fortunately, we see empirically, that while $N$ seems
highly sensitive to the prescribed value of $D_{max}$, density seems to
be invariant to $D_{max}$ as long as it is chosen to be sufficiently
large. We fit the model for a random of values of $D_{max}$ from $D_{max}=12$ (restricting
values of $x$ to be in close proximity to
the trap array) on up to 20. The results are given in Table
\ref{closed.tab.Dmax}.


\begin{table}[htp]
\centering
\caption{Analysis of Fort Drum bear hair snare data using the individual covariate model, for different values of Dmax, the upper limit of the uniform distribution of `distance from centroid of the trap array' }
\begin{tabular}{ccc}
\hline \hline
 Dmax & mean & SD \\ \hline
  12& 0.230 & 0.038 \\
  15& 0.244 &0.041 \\
  17& 0.249 &0.044 \\
  18& 0.249 &0.043\\
  19& 0.250 &0.043\\
  20& 0.250 &0.044
\end{tabular}
\label{closed.tab.Dmax}
\end{table}


We see that the posterior mean and SD of density (individuals per
square km) appear insensitive to choice of $D_{max}$ once we get a 
ways away from the maximum observed value of about 11.5. The estimated
density of 0.25 per km$^2$ is actually quite a bit lower than we 
reported using model $M_h$ 
for which  no relevant ``area'' quantity is explicit in the model.
Using MLEs of $N$ in conjunction with buffer strips
(see Table \ref{intro.tab.fdtests}) our estimates were in the range of $0.32-0.43$ and
the Bayesian estimates were XXXX (posterior mode of N = 102) or XXX (posterior mean of N = 117)
(see sec.
\ref{closed.sec.modelmh} above). 
On the other hand our estimate of $\hat{D} = 0.25$ here (based on the posterior mean) is 
higher than that reported from model $M_0$ using the buffered area
(0.18). There is no basis really for comparing or contrasting these
various estimates and it would be a useful philosophical exercise for
the reader to discuss this matter. In particular, application of models
$M_0$ and $M_h$ are distinctly {\it not} spatially explicit models -- the
area within which the population\footnote{We need to look back at
  Chapter 1 and make sure we quit calling this ``sample area'' - it
  really isn't that at al, but rather the area within which $N$
  resides.} resides is not defined under either model. There is
therefore no reason at all to think that the estimates produced under
either either closed population model, based on a buffered ``trap area'', 
are justifiable by any
theory. In fact, we would get exactly the same estimate of $N$ no
matter what we declare the area to be. On the other hand, the
individual covariate model explicitly describes a distribution for
``distance from centroid'' that is a reasonable and standard null
model - it posits, in the absence of direct information, that
individual home range centers are randomly distributed in space and
that probability of detection depends on the distance between home
range center and the centroid of the trap array. Under this definition
of the system, we see that density is invariant to the choice of
sample area which seems like a desirable feature. 

The individual
covariate model is not ideal, however, because it does not make full
use of the spatial information in the data set, i.e., the trap
locations and the locations of each individual encounter, and there is hope
to extend this model in order to resolve remaining deficiencies.


\subsection{Invariance of density to $D_{max}$}

Under the model above, and also under models that we consider in later
chapters, a general property of the estimators is that while $N$
increases with the prescribed trap area (equivalent to $D_{max}$ in this
case), we expect that density estimators should be invariant to this
area. In the model used above, we note that $Area(D_{max}) = 
\pi*D_{max}^{2}$ and $E[N(D_{max})] = \lambda*Area(D_{max})$ and thus
$E[Density(D_{max})] = \lambda$, i.e., constant. This should be 
interpreted as the {\it prior} density. Absent data, then realizations
under the model will have density $\lambda$ regardless of what $D_{max}$
is prescribed to be.  As we verified empirically above, the posterior
density is also invariant Of $D_{max}$ as long as the implied area
is large enough so that the data no longer provide
information about density (i.e., ``far away'').

\subsection{Toward Fully Spatial Capture-recapture Models}

We developed this model for the average observed location and equated
it to home range center ${\bf s}_{i}$. Intuitively, taking the average
encounter location as an estimate of home range center makes sense but
more so when the trapping grid is dense and expansive relative to
typical home range sizes.  However, our approach also ignored the
variable precision with which each ${\bf s}_{i}$ is estimated and also, as
noted previously, estimates of ${\bf s}_{i}$ around the ``edge'' (however we
define that) are biased because the observations are truncated (we can
only observe locations within the trap array).  In the next chapter we
provide a further extension of this individual covariate model that
definitively resolves the ad hoc nature of the individual covariate
approach we took here. In that chapter we build a model in which ${\bf s}_{i}$
are regarded as latent variables and the observation locations (i.e.,
trap specific encounters) are linked to those latent variables with an
explicit model. We note that the model fitted previously could be
adapted easily to deal with ${\bf s}_{i}$ as a latent variable, simply by
adding a prior distribution for ${\bf s}_{i}$. The reader should contemplate
how to do this in {\bf BUGS}.


\section{DISTANCE SAMPLING: A primative Spatial Capture-Recapture Model}

Distance sampling is one of the most popular methods for estimating
animal abundance. One of the great benefits of distance sampling is
that it provides explicit estimates of {\it density}. The distance
sampling model is a special case of a closed population model with a
covariate. The covariate in this case, $x_{i}$, is the distance
between an individual's location ``$u$'' and the observation location
or transect. In fact, the model underlying distance sampling is
precisely the same model as that which applies to the
individual-covariate models, except that observations are made at only
$K=1$ sampling occasion. In a sense, distance sampling is a spatial
capture-recapture model, but without the ``recapture.''  This first
and most basic spatial capture-recapture model has been used routinely
for decades and, formally, it is a spatially-explicit model in the
sense that it describes, explicitly, the spatial organization of
individual locations (although this is not always stated explicitly)
and, as a result, somewhat general models of how individuals are
distributed in space can be specified \citep{royle_etal:2004,
  johnson_etal:2010, sillett_etal:2011}.

As before, the distance sampling model, under data augmentation,
includes a set of $M$ zero-inflation variables $z_{i}$ and the
binomial model expressed conditional on $z$ (binomial for $z=1$, and
fixed zeros for $z=0$).  In distance sampling we pay for having only a
single sample (i.e., $K=1$) by requiring constraints on the model of
detection probability. A standard model is
\[
\log(p_{i}) = \beta x_{i}^{2}
\]
for $\beta < 0$, where $x_i$ denotes the distance at which the $i$th
individual is detected relative to some reference location where
perfect detectability ($p=1$) is assumed. This function corresponds to
the ``half-normal'' detection function (i.e., with $\beta =
1/\sigma^{2}$).  If $K>1$ then an intercept in this model is
identifiable and
such models are usually called ``capture-recapture distance
sampling''\citep{alpizar_pollock:1996,borchers_etal:1998}.

As with previous examples, we require a distribution for the individual covariate $x_{i}$. The customary choice is
\[
x_{i} \sim \mbox{Unif}(0,B)
\]
wherein $B>0$ is a known constant, being the upper limit of data
recording by the observer (i.e., the point count radius, or transect
half-width). In practice, this is sometimes asserted to be infinity,
but in such cases the distance data are usually truncated.
Specification of this distance sampling model in the {\bf BUGS} language is
shown in Panel \ref{closed.panel.distance} from \citet{royle_dorazio:2008}.


\begin{panel}[htp]
\centering
\rule[0.15in]{\textwidth}{.03in}
\begin{minipage}{5in}
\begin{verbatim}
beta~dunif(0,10)
psi~dunif(0,1)

for(i in 1:(nind+nz)){
   z[i]~dbern(psi)    # DA Variables
   x[i]~dunif(0,B)    # B=strip width
   p[i]<-exp(logp[i])   # DETECTION MODEL
   logp[i]<-   - beta*(x[i]*x[i])
   mu[i]<-z[i]*p[i]
   y[i]~dbern(mu[i])  # OBSERVATION MODEL
 }
N<-sum(z[1:(nind+nz)])
D<- N/striparea  # area of transects
\end{verbatim}
\end{minipage}
\rule[-0.15in]{\textwidth}{.03in}
\caption{Distance sampling model in {\bf BUGS}, using a half-normal
detection function.}
\label{closed.panel.distance}
\end{panel}

As with the individual covariate model in the previous section, the
distance sampling model can be equivalently specified by putting a
prior distribution on individual {\it location} instead of distance
between individual and observation point (or transect).  Thus we can
write the general distance sampling model as
\[
p_{i} = f(\beta,||{\bf u}_{i} - {\bf x}_0||)
\]
along with
\[
 {\bf u}_{i} \sim \mbox{Unif}({\cal S})
\]
where ${\bf x}_{0}$ is a fixed point (or line) and ${\bf u}_{i}$ is
the individual's location which is observable for $n$ individuals. In
practice it is easier to record distance instead of location.  Basic
math can be used to argue that if individuals have a uniform
distribution in space, then the distribution of Euclidean distance is
also uniform. In particular, if a transect of length $L$ is used and $x$
is distance to the transect then $F(x) = \Pr(X\le x) = L*x/L*B = x/B$ and
$f(x) = dF/dx = (1/B)$. For measurements of radial distance, see the
previous section.

In the context of our general characterization of SCR models 
(Chapt. \ref{modeling.sec.characterization}),
we suggested that every SCR model can be described,
conceptually, by a hierarchical model of the form:
\[
 [y|u][u|s][s].
\]
Distance sampling ignores the part of the model pertaining to ${\bf
  s}$, and deals only with the model components for the observed
data  ${\bf u}$\footnote{Equivalently, we could also say that $[u]$ in
  the distance sampling model is $[u] = \int [u|{\bf s}][{\bf s}]
  d{\bf s}$}. Thus, we are left with a hierarchical model of the form
\[
[y|{\bf u}][{\bf u}].
\]
In contrast, as we will see in the next chapters, basic SCR models
(Chapt. \ref{chapt.scr0}) ignore ${\bf u}$ and condition on ${\bf s}$,
which is not observed:
\[
[y|{\bf s}][{\bf s}]
\]
Since $[{\bf u}]$ and $[{\bf s}]$ are both assumed to be uniformly
distributed, these are structurally equivalent models! The main
differences have to do with interpretation of model components and
whether or not the latent variables are observable (in distance
sampling they are).

So why bother with SCR models when distance sampling yields density
estimates and accounts for spatial heterogeneity in detection? For
one, imagine trying to collect distance sampling data on tigers!
Clearly, distance sampling requires that one can collect large
quantities of distance data, which is not always possible. For tigers,
it is much easier, efficient, and safer to employ camera traps or
tracking plates and then apply SCR models. Furthermore, as we will see
in Chapts.
\ref{chapt.searchencounter} and \ref{chapt.scrds}, SCR models can use distance data to estimate all the
parameters of our enchilada, allowing us to study distribution,
movement, and density. Thus, SCR models are much more general and
versatile than distance sampling models (which clearly are a special
case), and can accommodate data from virtually all animal survey
designs.


\subsection{Example: Muntjac deer survey from Nagarahole, India }

Here we fit distance sampling models to distance sampling data on the
muntjac deer (Muntiakus muntjak) collected in the year 2004 from
Nagarahole National Park in southern India
(Kumar et al. unpublished data). The muntjac is
a solitary species and distance measurements were made on 57 groups
that were largely singletons with 4 pairs of individuals.  Commands
for reading in and organizing the data for {\bf WinBUGS}, followed by
writing the model to a text file, are given below. Note that the total sampled area of
the transects is fed in as ``striparea'' which is $708$ (km of transect walked)
multiplied by the strip width ($B=120 = 0.12$ km) multiplied by 2.
{\small 
\begin{verbatim}
library("R2WinBUGS")
data<- read.csv("Muntjac.csv")
hist(data[,3],30)
nind<-nrow(data)
y<-rep(1,nind)
nz<-400
y<-c(y,rep(0,nz))
x<-data[,3]
x<-c(x,rep(NA,nz))
z<-y

cat("
model{
beta~dunif(0,10)
psi~dunif(0,1)

for(i in 1:(nind+nz)){
   z[i]~dbern(psi)    # DA Variables
   x[i]~dunif(0,B)    # B=strip width
   p[i]<-exp(logp[i])   # DETECTION MODEL
   logp[i]<-   -beta*(x[i]*x[i])
   mu[i]<-z[i]*p[i]
   y[i]~dbern(mu[i])  # OBSERVATION MODEL
 }
N<-sum(z[1:(nind+nz)])
D<- N/striparea  # area of transects
}
",file="dsamp.txt")
\end{verbatim}
}
Next, we provide inits, indicate which parameters to monitor, and then
pass those things to {\bf WinBUGS}:
{\small
\begin{verbatim}
data<-list(y=y,x=x,nz=nz,nind=nind,B=120,striparea=(708*2*.120))
params<-list('beta','N','D','psi')
inits =  function() {list(z=z, psi=runif(1), beta=runif(1,0,.02) )}
fit = bugs(data, inits, params, model.file="dsamp.txt",working.directory=getwd(),    
       debug=T, n.chains=3, n.iter=11000, n.burnin=1000, n.thin=2)
\end{verbatim}
}
Posterior summaries are provided in the following table. Estimated
density is pretty low, 1.1 individuals per sq. km.\footnote{ This is much
  lower than Samba's estimate produced from WinBUGS accounting for group
  size. Reason unknown. }
{\small
\begin{verbatim}
Inference for Bugs model at "dsamp.txt", fit using WinBUGS,
 3 chains, each with 11000 iterations (first 1000 discarded), n.thin = 2
 n.sims = 15000 iterations saved
           mean    sd   2.5%    25%    50%    75%  97.5% Rhat n.eff
beta       0.00  0.00   0.00   0.00   0.00   0.00   0.00    1  1100
N        185.73 26.53 138.00 167.00 184.00 203.00 242.00    1   570
D          1.09  0.16   0.81   0.98   1.08   1.20   1.42    1   570
psi        0.41  0.06   0.30   0.36   0.40   0.45   0.54    1   670
deviance 655.74 16.26 626.00 644.50 655.10 666.40 689.80    1  1300

[.... some output deleted .... ]
\end{verbatim}
}

\section{Summary and Outlook}

Traditional closed population capture-recapture models are closely
related to binomial generalized linear models.  Indeed, the only real
distinction is that in capture-recapture models, the population size
parameter $N$ (corresponding also to the size of a hypothetical
``complete'' data set) is unknown.  This requires special
consideration in the analysis of capture-recapture models. The
classical approach to inference recognizes that the observations don't
have a standard binomial distribution but, rather, a truncated
binomial (from which which the so-called ``conditional likelihood''
derives) since we only have encounter frequency data on observed
individuals. If instead we analyze the models using data augmentation,
the observations can be modeled using a zero-inflated binomial
distribution. In short, when we deal with the unknown-$N$ problem using
data augmentation then we are left with zero-inflated GLM and GLMMs
instead of ordinary GLM or GLMMs. The analysis of such zero-inflated
models is practically convenient, especially using the various
Bayesian analysis packages that use the {\bf BUGS} language.

Spatial capture-recapture models that we will consider in the rest of
the chapters of this book are closely related to what have been called
individual covariate models. Heuristically, spatial capture-recapture
models arise by defining individual covariates based on observed
locations of individuals -- we can think of using some function of
mean encounter location as an individual covariate. We did this in a
novel way, by using distance to the centroid of the trapping array as
a covariate. We analyzed the ``full likelihood'' using data
augmentation, and placed a prior distribution on the individual
covariate which was derived from an assumption that individual
locations are, a priori, uniformly distributed in space. This
assumption provides for invariance of the density estimator to the
choice of population size area (induced by maximum distance from the
centroid of the trap array). The model addressed some important problems in the
use of closed population models: it allows for heterogeneity in
encounter probability due to the spatial context of the problem and it
also provides a direct estimate of density because area is a feature
of the model (via the prior on the individual covariate). The model is
still not completely general because it does not make use of
the fully spatial encounter histories, which provide direct
information about the locations and density of individuals.  A
specific individual covariate model that is in widespread use is
classical ``distance sampling.'' The model underlying distance
sampling is precisely a special kind of SCR model - but one without
replicate samples. Understanding distance sampling and individual
covariate models more broadly provides a solid basis for understanding
and analyzing spatial capture-recapture models.



%%% TO DO  as of 12/29/11

 %%% Spell check document

 %%% Change "beta" to "theta"

 %%% Fix up R scripts and consolidate for R package
 %%% R commands to process wolverine data need included in that section

 %%% Run Wolverine 2k 4k and 8k grids in JAGS compare to WinBUGS
 %%%     insert those results in text

 %%%  For discrete state-space stuff, convert BUGS output to JAGS and
 %%%  figure out MC errors
 %%% Finish Table that has those results in it

 %% pick up all hard references to chapters and make float


\chapter{Fully Spatial Capture-Recapture Models}
\markboth{Chapter 4 }{}
\label{chapt.scr0}

\vspace{.3in}

In previous sections we discussed some classes of models that could be
viewed as primitive spatial capture-recapture models. We looked at a
basic distance sampling model and we also considered a classical
individual covariate modeling approach in which we defined a covariate
to be the distance from (estimated) home range center to the center of
the trap array. These were spatial in the sense that they included
some characterization of where individuals live but, on the other
hand, only a primitive or no characterization of trap location.  That
said, very little distinguishes these two models from spatial
capture-recapture models that we consider in this chapter which fully
recognize the spatial attribution of both individual animals {\it and}
the locations of encounter devices.

Fully spatial capture-recapture models must accommodate the spatial
organization of individuals and the encounter devices because the
encounter process occurs at the level of individual traps.  Failure to
consider the trap-specific collection of data is the key deficiency
with classical ad-hoc approaches which aggregate encounter information
to the resolution of the entire trap array. We have  previously
addressed some problems that this induces including induced
heterogeneity in encounter probability, imprecise notation of ``sample
area'' and not being able to accommodate trap-specific
effects.
In this chapter we resolve these issues by developing 
our first fully spatial capture-recapture
model which turns out to be precisely the model considered in sec. \ref{closed.sec.indcov}
 but instead of defining the individual covariate to be distance
to centroid of the array we define $J$ individual covariates - the
distance to {\it each} trap. And, instead of using estimates of
individual locations ${\bf s}$, we consider a fully hierarchical model in
which we regard ${\bf s}$ as a latent variable and impose a prior
distribution on it.  We can think of having $J$ independent
capture-recapture studies generating one data set for each trap, and
applying the individual covariate model with random activity centers,
and that is all the basic SCR model is.

In the following sections of this chapter we investigate the basic
spatial capture-recapture model, which we refer to as ``model SCR0'',  and address some important
considerations related to its analysis in {\bf WinBUGS}. We also demonstrate
how to summarize posterior output for the purposes of producing
density maps or spatial predictions of density.

\section{Sampling Design and Data Structure}

In our development here, we will assume a standard sampling design in
which an array of $J$ traps is operated for $K$ time periods (say,
nights) producing encounters of $n$ individuals.  Because sampling
occurs by traps and also over time, the most general data structure
yields encounter histories for {\it each individual} that are
temporally {\it and} spatially indexed. Thus a typical data set will
include an encounter history {\it matrix} for each individual.  For
the most basic model, there are no time-varying covariates that
influence encounter, there are no explicit individual-specific
covariates, and there are no covariates that influence density we will
develop models in this chapter for encounter data that are aggregated
over the temporal replicates. For example, suppose we observe 6
individuals in sampling at 4 traps over 3 nights of sampling then a
plausible data set is the $6 \times 4$ matrix of encounters, out of 3,
of the form:
\begin{verbatim}
      trap1 trap2 trap3 trap4
 [1,]     1     0     0     0
 [2,]     0     2     0     0
 [3,]     0     0     0     1
 [4,]     0     1     0     0
 [5,]     0     0     1     1
 [6,]     1     0     1     0
\end{verbatim}

We develop models in this chapter for devices such as ``hair snares''
or other DNA sampling methods \citep{kery_etal:2010,
  gardner_etal:2010jwm} and related types of sampling devices in which
(i) effective ``traps'' may capture any number of individuals (i.e.,
they don't fill up; This is referred to as a ``multi-catch'' type of
sampling \citep{efford_etal:2009ecol}); (ii) an individual may be
captured in any number of traps during each occasion but (iii)
individuals can be encountered at most 1 time in a trap during any
occasion.  The statistical assumptions are that individual encounters
within and among traps are independent, and this allows us to regard
individual- and trap-specific encounters as $iid$ Bernoulli trials
(see next section).  These basic (but admittedly at this point
somewhat imprecise) assumptions define the basic spatial
capture-recapture model, which we will refer to as ``SCR0'' 
so that we may use that model as a point of reference without having
to provide a long-winded enumeration of assumptions and sampling
design each time we do. We will make things more precise as we develop
a formal statistical definition of the model shortly.

While the model is mostly directly relevant
to hair snares and other DNA sampling methods for which multiple
detections of an individual are not distinguishable,
we will also make use of the model for data that arise from
camera-trapping studies. In practice, with camera trapping,
individuals might be photographed several times in a night but we will
typically distill such data into a single binary encounter event for
reasons discussed later in Chapt. \ref{chapt.poisson-mn}.


\section{The binomial observation model }

We assume that the individual and trap-specific encounters, $y_{ij}$,
are mutually independent outcomes of a binomial random variable:
\begin{equation}
	y_{ij} \sim \mbox{Bin}(K, p_{ij})
\label{scr0.eq.bin}
\end{equation}
This is the basic model underlying ``logistic regression'' (Chapt. \ref{chapt.glms})
as well as standard closed population models
(Chapt. \ref{chapt.closed}). The key
element of the model is that the encounter probability $p_{ij}$ is
indexed by (i.e., depends on) both individual and trap. In a sense,
then, we can think of each {\it trap} as producing individual level
encounter history data of the classical variety - an $\mbox{\tt nind}
\times \mbox{\tt nreps}$
matrix of 0's and 1's (this is the ``encountered at most 1 time''
assumption).


As we did in sec. \ref{closed.sec.indcov}, we will make explicit the notion that
$p_{ij}$ is defined conditional on ``where'' individual $i$
lives. Naturally, we think about defining an individual home range and
then relating $p_{ij}$ explicitly to the centroid of the individuals
home range, or its center of activity \citep{efford:2004,
  borchers_efford:2008, royle_young:2008}.  Therefore, define ${\bf
  s}_{i}$, a two-dimensional spatial coordinate, to be the activity
center for individual $i$. Then, the SCR model postulates that
encounter probability, $p_{ij}$, is a decreasing function
of distance between ${\bf s}_{i}$ and the location of trap $j$, ${\bf x}_{j}$.
 Naturally, if we think of modeling binomial counts using
logistic regression, we might specify the model according to:
\begin{equation}
	\mbox{logit}(p_{ij}) = \alpha_{0} + \alpha_1 ||{\bf s}_{i}-{\bf x}_{j} ||
\label{scr0.eq.logit}
\end{equation}
where, here, $||{\bf s}_{i}-{\bf x}_{j}||$ is the distance between
${\bf s}_{i}$ and ${\bf x}_{j}$. We sometimes write $||{\bf
  s}_{i}-{\bf x}_{j}|| = dist({\bf s}_{i},{\bf x}_{j}) =
d_{ij}$. Alternatively, if we think about distance sampling then we
might use the ``half-normal'' model of the form:
\[
p_{ij} = p_{0}*\exp(-\alpha_{1} *||{\bf s}_{i}-{\bf x}_{j}||^2)
\]
Or any of a large number of standard detection models that are
commonly used (we consider more in Chapt. \ref{chapt.covariates}). The half-normal model implies
\begin{equation}
\log(p_{ij})  = \log(p_{0}) - \alpha_{1} *||{\bf s}_{i}-{\bf x}_{j}||^2
\label{scr0.eq.norm}
\end{equation}
%We would always like to be clear that encounter probability depends on individual activity
%centers {\it and} trap locations {\it and} parameter(s) $\theta$, and
%so it would be ideal to write $p({\bf s}_{i},{\bf x}_{j}; \theta)$ or
%something similar. However, this can be extremely unwieldy and
%clutter up what are otherwise extremely simple mathematical
%expressions and formulae. As such, we will usually abbreviate these
%various dependencies by writing $p_{ij}$ or sometimes $p_{\theta,ij}$,
%understanding that $p_{ij}$ is actually a function of the various important
%quantities.
We probably expect that the parameter $\alpha_{1}$ in
Eq. \ref{scr0.eq.logit} or \ref{scr0.eq.norm} should be negative, so
that the probability of encounter decreases with distance between the
trap and individual home range center.  
Whatever model encounter probability we choose, we should always keep
in mind that the model is described conditional on ${\bf s}_{i}$,
which is an unobserved random variable.  Thus, to be precise about
this, we should write the observation model as
\[
y_{ij}|{\bf s}_{i} \sim \mbox{Bin}(K, p({\bf s}_{ij};\alpha_{1}))
\]


The joint likelihood for the
data, conditional on the collection of individual activity centers,
can therefore be expressed as
\[
{\cal L}(\alpha_{1} | \{ {\bf y}_{i},{\bf s}_{i} \}_{i=1}^{N})
 =  \prod_{i} \prod_{j} \mbox{Bin}(y_{ij}|p_{ij}(\alpha_{1}))
\]
Which, if we switch the indices on the product operators, this shows
the SCR likelihood (conditional on ${\bf s}$) to be the product of $J$
{\it independent} capture-recapture likelihoods - one for each trap.
However, the data have a distinct ``repeated measures'' type of structure, with
each of the $j$ likelihood contributions for each individual being
grouped by individual. Thus, we cannot analyze the model
meaningfully by $J$ trap-specific models. In classical repeated measures
types of models, we accommodate the group structure of the data using
random effects (random individual or group level variables). For SCR
models we take the same basic approach, which we develop subsequently.

\subsection{Distance as a latent variable}

If we knew precisely every ${\bf s}_{i}$ in the population (and how
many, $N$), then the model specified by eqs. \ref{scr0.eq.bin} and
\ref{scr0.eq.logit} is just an ordinary logistic
regression type of a model which we learned how to fit using {\bf
  WinBUGS} previously (Chapt. \ref{chapt.glms}), with a covariate $d_{ij}$. However,
the activity centers are unobservable even in the best possible
circumstances. In that case, $d_{ij}$ is an unobserved variable,
analogous to classical random effects models. We need to therefore
extend the model to accommodate these random variables with an
additional model component. A standard, and perhaps not unreasonable,
assumption is the so-called ``uniformity assumption'' which is to say
that the ${\bf s}_{i}$ are uniformly distributed over space (the
obvious next question ``which space?'' is addressed below).  This
uniformity assumption amounts to a uniform prior distribution on ${\bf
  s}_{i}$, i.e., the pdf of ${\bf s}_{i}$ is constant, which we may
express
\begin{equation}
	\Pr({\bf s}_{i}) \propto \mbox{\tt const}
\label{scr0.eq.sprior}
\end{equation}
 As it turns out, this assumption is usually not precise
enough to fit SCR models in practice for reasons we discuss in the
following section.  We will give another way to represent this prior
distribution that is more concrete, but it depends on specifying the
``state-space'' of the random variable ${\bf s}_{i}$. The term
state-space is a technical way of saying ``possible outcomes''.

To summarize the preceeding model developing, a basic SCR model is
defined by 3 essential components:
\begin{itemize}
\item[(1)] Observation model: $y_{ij}|{\bf s}_{i} \sim \mbox{Bin}(K, p_{ij})$
\item[(2)] Encounter probability: $\mbox{logit}(p_{ij}) = \alpha_{0} +
  \alpha_{1}*||{\bf s}_{i}-{\bf x}_{j}||$
\item[(3)] Point process model: $\Pr({\bf s}_{i} ) \propto \mbox{\tt const}$
\end{itemize}
Therefore, the SCR model is little more than an ordinary
capture-recapture model for closed populations. It is such a model,
but augmented with a set of ``individual effects'', ${\bf s}_{i}$,
which relate some sense of individual location to encounter
probability. 

\section{ The Binomial Point-process Model}

The collection of individual activity centers ${\bf s}_{1},\ldots,
{\bf s}_{N}$ represent a realization of a {\it binomial point process}
\citep[][p. xyz]{illian_etal:2008}.  The binomial point process (BPP)
is analogous to a Poisson point process in the sense that it
represents a ``random scatter'' of points in space - except that the
total number of points is {\it fixed}, whereas, in a Poisson point
process it is random (having a Poisson distribution).  As an example,
we show in Fig. \ref{scr0.fig.bpp} locations of 20 individual activity
centers (black dots) in relation to a grid of 25 traps. For a Poisson
point process the number of such points in the prescribed state-space
would be random whereas often we will simulate fixed numbers of
points, e.g., for evaluating the performance of procedures such as how
well does our estimator perform of $N=50$?
\begin{figure}
\begin{center}
\includegraphics[height=2.5in]{Ch4/figs/binomialpoint}
\end{center}
\caption{Realization (small circles) of a binomial point process with $N=20$. The
  large circles represent trap locations.}
\label{scr0.fig.bpp}
\end{figure}

It is natural to consider a binomial point process in the context of
capture-recapture models because it preserves $N$ in the model and thus
preserves the linkage directly with closed population models. In fact,
under the binomial point process model then model $M_0$ and other closed
models are simple limiting cases of SCR models, i.e., as the
coefficient on distance tends to 0.
In addition, use of
the BPP model allows us to use data augmentation for Bayesian analysis
of the models as in Chapt. \ref{chapt.closed}, thus yielding a methodologically
coherent approach to analyzing the different classes of
models. Despite this, making explicit assumptions about $N$, such as
Poisson, is convenient in some cases (see Chapt. \ref{chapt.hscr}).

One consequence of having fixed $N$, in the BPP model, is that the
model is not strictly a model of ``complete spatial randomness''. This
is because if one forms counts $n(A_{1}),\ldots, n(A_{k})$ in any set
of disjoint regions say $A_{1}, \ldots, A_{k}$, then these counts are
{\it not} independent.  In fact, they have a multinomial distribution
\citep[see][p. XYZ]{illian_etal:2008}. Thus, the BPP model introduces
a slight bit of dependence in the distribution of points. However, in
most situations this will have no practical effect on any inference or
analysis and, as a practical matter, we will usually regard the BPP
model as one of spatial independence among individual activity centers
because each activity center is distributed independently of each
other activity center. Despite this implicit independence we see in
Fig. \ref{scr0.fig.bpp} that {\it realizations} of randomly distributed
points will typically exhibit distinct non-uniformity. Thus,
independent, uniformly distributed points will almost never appear
regularly, uniformly or systematically distributed. For this reason,
the basic binomial (or Poisson) point process models are enormously
useful in practical settings.  More relevant for SCR models is that we
actually have a little bit of data for some individuals and thus the
resulting posterior point pattern can deviate strongly from
uniformity, a point we come back to repeatedly in this book.
The uniformity hypothesis is only
a {\it prior} distribution which is directly affected by the quantity
and quality of observations, to produce a posterior distribution which
may appear distinctly non-uniform.


\subsection{Definition of home range center}

Some will be offended by our use of the concept of ``home range
center'' and thus will have difficulty in believing that the resulting
model is really useful for anything.  Indeed, the idea of a home range
or activity center is a vague concept anyway, a purely
phenomenological construct.  Despite this, it doesn't really matter
whether or not a home range makes sense for a particular species -
individuals of any species inhabit {\it some} region of space and we
can define the ``home range center'' to be the center of the space
that individual was occupying (or using) during the period in which
traps were active. Thinking about it in that way, it could even be
observable (almost) as the centroid of a very large number of radio
fixes over the course of a survey period or a season.  Thus, this
practical version of a home range center in terms of space usage is a well-defined construct
regardless of whether one thinks the home range concept is meaningful,
even if individuals are not particularly territorial.  This is why we
usually use the term ``activity center'' or maybe even ``centroid of
space usage'' and we recognize that this construct is a transient
thing which applies only to a well-defined period of study.



\subsection{The state-space of the point process}

Shortly we will focus on Bayesian analysis of this model with $N$
known so that we can directly apply what we learned in
Chapt. \ref{chapt.glms} to 
this situation. To do this, we note that the individual effects ${\bf
  s}_{i},\ldots, {\bf s}_{N}$ are unknown quantities and we will need
to be able to simulate each ${\bf s}_{i}$ in the population from the
posterior distribution.  It should be self-evident that we cannot
simulate the ${\bf s}_{i}$ unless we describe precisely the region
over which they are uniformly distributed. This is
the quantity referred to above as the state-space, denoted henceforth
by ${\cal S}$, which is a region or a set of points comprising the
potential values of ${\bf s}_{i}$. Thus, an equivalent explicit
statement of the ``uniformity assumption'' is
\[
{\bf s}_{i} \sim \mbox{Unif}({\cal S})
\]
where ${\cal S}$ is a precisely defined region. e.g., in Fig. 
\ref{scr0.fig.bpp}, ${\cal S}$ is the square defined by $[-1,7] \times
[-1, 7]$. Thus each of the $N=20$ points were generated by randomly
selecting each coordinate on the line $[-1, 7]$. 


\subsubsection{Prescribing the state-space}

Evidently, we need to define the state-space, ${\cal S}$. How can we
possibly do this objectively? Prescribing any particular ${\cal S}$
seems like the equivalent of specifying a ``buffer'' which we
criticized previously as being ad hoc. How is it, then, is choosing a
state-space is {\it not} ad hoc? As a practical matter, it turns out
that estimates of density are insensitive to choice of the
state-space. As we observed in Chapt. \ref{chapt.closed}, it is true that $N$ increases
with ${\cal S}$, but only at the same rate as the area of ${\cal S}$
increases under the
prior assumption of constant density. As a result, we say that density
is invariant to ${\cal S}$ as long as ${\cal S}$ is sufficiently
large. Thus, while choice of ${\cal S}$ is (or can be) essentially
arbitrary, once ${\cal S}$ is chosen, it defines the population being
exposed to sampling, which scales appropriately with the size of the
state-space.

For our simulated system developed previously in this chapter, we
defined the state space to be a square within which our trap array was
centered. For many practical situations this might be an
acceptable approach to defining the state-space. We provide an example
of this in sec. \ref{scr0.sec.wolverine} below in which the trap array is
irregular and also situated within a realistic landscape that is
distinctly irregular.  In general, it is most practical to define the
state-space as a regular polygon (e.g., rectangle) containing the trap
array without differentiating unsuitable habitat. Although defining
the state-space to be a regular polygon has computational advantages
(e.g., we can implement this more efficiently in {\bf WinBUGS} and
cannot for irregular polygons), a regular polygon induces an apparent
problem of admitting into the state-space regions that are distinctly
non-habitat (e.g., oceans, large lakes, ice fields, etc.).  It is
difficult to describe complex sets in mathematical terms that can be
used in {\bf BUGS}. As an alternative, we can provide a
representation of the state-space as a discrete set of points (sec.
\ref{scr0.sec.discrete}) that will allow specific points to be deleted
or not depending on whether they represent habitat, or we can define
the state-space as an arbitrary  collection of polygons stored as a GIS
shapefile
which can be analyzed easily using MCMC
(see sec. \ref{mcmc.sec.state-space}), but not so easily in the {\bf
  BUGS} variants.  In what follows below we provide an
analysis of the camera data defining the state-space to be a regular
continuous polygon (a rectangle).


\subsection{Invariance and the State-space as a model assumption}
\label{scr0.sec.invariance}

We will assert for all models we consider in this book that density is
invariant to the size and extent of ${\cal S}$, if ${\cal S}$ is
sufficiently large as long 
as our model relating $p_{ij}$ to ${\bf  s}_{i}$ is a decreasing
function of distance.  
We can prove this easily by drawing an analogy with a 1-d case such as
in distance sampling.  Let $y_{j}$ be the number of individuals
captured in some interval $[d_{j-1},d_{j})$, and define $d_{J} = B$
for some large value of $B$.  By choosing $B$ large enough we
guarantee that $E[y_{J+1}] = 0$ and therefore this ``last cell'' 
contributes nothing to
the likelihood
in regular situations in which the detection function decays
monotonically with distance and prior density is constant.  


Sometimes
our estimate of density can be influenced if we make ${\cal S}$ too small but
this might be sensible if ${\cal S}$ is naturally well-defined. As we discussed
in chapter 1, {\bf choice of ${\cal S}$ is part of the model and thus it makes
  sense that estimates of density might be sensitive to its definition
  in problems where it is natural to restrict ${\cal S}$}.
One could imagine
however that in specific cases where you're studying a small
population with well-defined habitat preferences that a problem could
arise because changing the state-space around based on differing
opinions and GIS layers really changes the estimate of total
population size. But this is a real biological problem and a natural
consequence of the spatial formalization of capture-recapture models -
a feature, not a bug or some statistical artifact - and it should be
resolved with better information, research, and thinking.
 For situations where there is not a natural
choice of ${\cal S}$, we should default to choosing ${\cal S}$ to be very large in order
to achieve invariance or otherwise evaluate sensitivity of density
estimates by trying a couple of different values of ${\cal S}$. This is a
standard ``sensitivity to prior'' argument that Bayesians always have
to be conscious of.  We demonstrate this in our analysis of section
\ref{scr0.sec.wolverine}
below. Note that $area({\cal S})$ affects data augmentation. If you
increase $area({\cal S})$ then there are more individuals to account for and
therefore the size of the augmented data set $M$ must increase.

We have been told that one can carry-out non-Bayesian analyses of SCR
models without having to specify the state-space of the point process
or perhaps while only specifying it imprecisely.  This assertion is
incorrect. We assume people are thinking this because {\it they} don't
have to specify it explicitly because someone else has done it for
them in a package that does integrated likelihood. Even to do
integrated likelihood (see Chapt. \ref{chapt.mle}) we have to integrate the
conditional-on-${\bf s}$ likelihood over some 2-dimensional space.  It might
work that the integration can be done from $-\infty$ to $+\infty$ but
that is a mathematical artifact of specific detection functions, and
an implicit definition of a state-space that doesn't make biological
sense, even though it may in fact be innocuous;


\subsection{Connection to Model  $M_h$}  \label{scr0.sec.scrmh}

SCR models are closely related to heterogeneity models. In SCR models,
heterogeneity in encounter probability is induced by both the effect
of distance in the model for detection probability and also from
specification of the state-space. Hence, the state-space  is an
explicit element of the model. 
To understand this, suppose we have a random
effect with some prior distribution:
\[
{\bf s} \sim \mbox{Unif}({\cal S})
\]
And $p({\bf s}) = p(y=1|{\bf s})$ is some function of ${\bf
  s}$. Therefore, for any specific $g(p)$ and ${\cal S}$ we can work
out what the implied heterogeneity model is for example, the mean,
variance or other moments of the population distribution of $p$ can be
evaluated by integrating $p({\bf s})$ over the state-space of ${\bf
  s}$.  We
show an illustration in Fig. \ref{scr0.fig.buffereffect} which
shows a histogram of $p$ for a hypothetical population of 100000
individuals on a state-space enclosing our $5 \times 5$ trap array
above, under the logistic model for distance. {\bf R} code is
provided in the {\bf R} package \mbox{\tt scrbook} to produce this analysis for the
logistic and half-normal models. The histogram shows the encounter
probability under buffers of 0.2, 0.5 and 1.0. We see the mass shifts
to the left as the buffer increases, implying more individuals
 with lower encounter probabilies, as their home range
centers increase in distance from the trap array.


\begin{figure}
\begin{center}
\includegraphics[width=5in]{Ch4/figs/buffereffect}
\end{center}
\caption{Implied population distribution of $p_{i}$ for a population
  of individuals as a function of the size of the state-space buffer
  around a trap array. The trap array is fixed and centered within a
  square state-space.}
\label{scr0.fig.buffereffect}
\end{figure}

Another way to understand this is by representing ${\cal S}$ as a set
of discrete points on a grid. In the coarsest possible case where
${\cal S}$ is a single arbitrary point, then every individual has
exactly the same $p$. As we increase the number of points in ${\cal
  S}$ then more distinct values of $p$ are possible. As such, when
${\cal S}$ is characterized by discrete points then SCR models are
precisely a type of finite-mixture model \citep{norris_pollock:1996,
  pledger:2000}, except, in the case of SCR models, we have some information about which
group an individual belong (i.e., where their activity center is), as
a result of their captures in traps.

This context suggests the problem raised by \citet{link:2003}. He
showed that in most practical situations $N$ may not be identifiable
across classes of mixture distributions which in the context of SCR
models is the pair $(g, {\cal S})$.  The difference, however, is that
we do obtain some direct information about ${\bf s}$ in SCR models and
therefore it may be reasonable to expect that
$N$ is identifiable across models characterized by $(g,{\cal
  S})$.

\subsection{Connection to Distance Sampling}

It is worth emphasizing that the basic SCR model is a binomial
encounter model in which distance is a covariate. As such, it is
striking similarity to a classical distance sampling model. Both have
distance as a covariate but in classical distance sampling problems
the focus is on the distance between the observer and the animal at an
instant in time, not the distance between a trap and an animal's home
range center. As a practical matter, in distance sampling, ``distance'' is {\it
  observed} for those individuals that appear in the
sample. Conversely, in SCR problems, it is only imperfectly observed
(we have partial information in the form of trap observations).
Clearly, it is preferable to observe distance if possible, but 
distance sampling requires field methods that
are often not practical in many situations, e.g. when surveying
tigers. Furthermore, SCR models allow us to relax many of the
assumption made in classical distance sampling, and SCR models allow
for estimates of quantities other than density, such as home range
size, and space usage (see Chapt. \ref{chapt.ecoldist}).


\section{Simulating SCR Data}

It is always useful to simulate data because it allows you to
understand the system that you're modeling and also calibrate your
understanding with the parameter values of the model. That is, you can
simulate data using different parameter values until you obtain data
that ``looks right'' based on your knowledge of the specific situation
that you're interested in. Here we provide a simple script to
illustrate how to simulate spatial encounter history data. In this
exercise we simulate data for 100 individuals and a 25 trap array laid
out in a $5 \times 5$ grid of unit spacing.  The specific encounter model is
the half-normal model given above and we used this code to simulate
data used in subsequent analyses.  The 100 activity centers were
simulated on a state-space defined by a $8 \times 8$ square within which the
trap array was centered (thus the trap array is buffered by 2
units). Therefore, the density of individuals in this system is fixed
at $100/64$.

{\small
\begin{verbatim}
	set.seed(2013)
# create 5 x 5 grid of trap locations with unit spacing
traplocs<- cbind(sort(rep(1:5,5)),rep(1:5,5))
Dmat<-e2dist(traplocs,traplocs) # in cases where speed doesn't matter, it might be
                                # clearer to just show the slow for-loop.
                                # Plus, people will want to copy/paste this stuff
ntraps<-nrow(traplocs)

# define state-space of point process. (i.e., where animals live).
# "delta" just adds a fixed buffer to the outer extent of the traps.
delta<-2
Xl<-min(traplocs[,1] - delta)
Xu<-max(traplocs[,1] + delta)
Yl<-min(traplocs[,2] - delta)
Yu<-max(traplocs[,2] + delta)

N<-100   # population size
K<- 20    # number nights of effort

sx<-runif(N,Xl,Xu)    # simulate activity centers
sy<-runif(N,Yl,Yu)
S<-cbind(sx,sy)
D<- e2dist(S,traplocs)  # distance of each individual from each trap

alpha0<- -2.5      # define parameters of encounter probability
sigma<- 0.5        #
alpha1<- 1/(2*sigma*sigma)
probcap<- expit(-2.5)*exp( - alpha1*D*D)    # probability of encounter
# now generate the encounters of every individual in every trap
Y<-matrix(NA,nrow=N,ncol=ntraps)
for(i in 1:nrow(Y)){
   Y[i,]<-rbinom(ntraps,K,probcap[i,])
}
\end{verbatim}
}


Subsequently we will generate data using this code packaged in an {\bf
  R}
function called \mbox{\tt simSCR0.fn} in the package \mbox{\tt
  scrbook} which takes a number of
arguments including \mbox{\tt discard0} which, if \mbox{\tt TRUE}, will return
only the encounter histories for captured individuals.  A second
argument is \mbox{\tt array3d} which, if \mbox{\tt TRUE}, returns the 3-d
encounter history array instead of the aggregated \mbox{\tt nind}
$\times \mbox{\tt ntraps}$ encounter frequencies (see below). Finally
we provide a random number seed, \mbox{\tt sd = 2013} to ensure
repeatability of the analysis here. We obtain a data set as above using the
following command:
\begin{verbatim}
data<-simSCR0.fn(discard0=TRUE,array3d=FALSE,sd=2013)
\end{verbatim}
The {\bf R} object \mbox{\tt data} is a list, so let's take a look at
what's in the list and then harvest some of its elements for further
analysis below.
\begin{verbatim}
> names(data)
[1] "Y"        "traplocs" "xlim"     "ylim"     "N"        "alpha0"   "beta"
[8] "sigma"    "K"
> Y<-data$Y
> traplocs<-data$traplocs
\end{verbatim}


\subsection{Formatting and manipulating real data sets}
\label{scr0.sec.formats}

Conventional capture-recapture data are easily stored and manipulated
as a 2-dimensional array, an $\mbox{\tt nind} \times \mbox{\tt
  nperiod}$ matrix, which is maximally informative for any
conventional capture-recapture model, but not for spatial
capture-recapture models.  For SCR models we must preserve the spatial
information in the encounter history information. We will routinely
analyze data from 3 standard formats:
\begin{itemize}
\item[(1)] The basic 2-dimensional data format, which is an \mbox{\tt
    nind} $\times$ \mbox{\tt ntraps} encounter frequency matrix such
  as that simulated previously; These are the total encounters in each
  trap, summed over replicate samples.
\item[(2)] The maximally informative 3-dimensional array which we
  establish here the convention that it has dimensions \mbox{\tt nind}
  $\times$ \mbox{\tt nperiods} $\times$ \mbox{\tt ntraps} and
\item[(3)] We use a compact format - the ``SCR flat format'' - which
  we describe below in section \ref{scr0.sec.wolverine}.
\end{itemize}
To simulate data in the most informative format - the ``3-d array'' -
we can use the {\bf R} commands given previously but replace the last
4 lines with the following:
{\small
\begin{verbatim}
Y<-array(NA,dim=c(N,K,ntraps))
for(i in 1:nrow(Y)){
for(j in 1:ntraps){
 Y[i,1:K,j]<-rbinom(K,1,probcap[i,j])
}
}
\end{verbatim}
}
We see that a collection of $K$ binary encounter events are generated
for {\it each} individual and for {\it each} trap.  The probabilities
have those Bernoulli trials are computed based on the distance from
each individuals home range center and the trap (see calculation
above), and those are housed in the matrix \mbox{\tt probcap}. Our data simulator
function \mbox{\tt simSRC0.fn} will return the full 3-d array if
\mbox{\tt array3d=TRUE} is specified in the function call.  To recover
the 2-d matrix from the 3-d array, and subset the 3-d array to
individuals that were captured, we do this:
{\small
\begin{verbatim}
Y2d<- apply(Y,c(1,3),sum) # sum over the ``replicates'' dimension (2nd margin of the array)
ncaps<-apply(Y2d,1,sum)   # compute how many times each individual was captured
Y<-Y[ncaps>0,,]           # keep those individuals that were captured
\end{verbatim}
}

\section{Fitting an SCR Model in BUGS}
\label{scr0.sec.winbugs1}

Clearly if we somehow knew the value of $N$ then we could fit this
model directly because, in that case, it is a special kind of logistic
regression model - one with a random effect, but that enters into the
model in a peculiar fashion - and also with a distribution (uniform)
which we don't usually think of as standard for random effects models.
So our aim here is to analyze the known-$N$ problem, using our
simulated data, as an incremental step in our progress toward fitting
more generally useful models.

To begin, we use our simulator to grab a data set and then harvest the
elements of the resulting object for further analysis.
\begin{verbatim}
data<-simSCR0.fn(discard0=FALSE,sd=2013)
y<-data$Y
traplocs<-data$traplocs
nind<-nrow(y)
X<-data$traplocs
J<-nrow(X)
y<-rbind(y,matrix(0,nrow=(100-nrow(y)),ncol=J ) )
Xl<-data$xlim[1]
Yl<-data$ylim[1]
Xu<-data$xlim[2]
Yu<-data$ylim[2]
\end{verbatim}

Note that we specify \mbox{\tt discard0 = FALSE} so that we have a
"complete" data set, i.e., one with the all-zero encounter histories
corresponding to uncaptured individuals. Now, within an {\bf R} session, we
can create the {\bf BUGS} model file and fit the model using the following
commands. 
{\small
\begin{verbatim}
cat("
model {
alpha0~dnorm(0,.1)
logit(p0)<- alpha0
alpha1~dnorm(0,.1)
for(i in 1:N){
 s[i,1]~dunif(Xl,Xu)
 s[i,2]~dunif(Yl,Yu)
for(j in 1:J){
d[i,j]<- pow(pow(s[i,1]-X[j,1],2) + pow(s[i,2]-X[j,2],2),0.5)
y[i,j] ~ dbin(p[i,j],K)
p[i,j]<- p0*exp(- alpha1*d[i,j]*d[i,j])
}
}

}
",file = "SCR0a.txt")
\end{verbatim}
}
This model describes the half-normal detection model but it
would be trivial to modify that to various others including the
logistic described above. One consequence of using the half-normal is
that we have to constrain the encounter probability to be in $[0,1]$
which we do here by defining \mbox{\tt alpha0} to be the logit of the
intercept parameter \mbox{\tt p0}. Note that the distance covariate is
computed within the {\bf BUGS} model specification given the matrix of trap
locations, \mbox{\tt X}, which is provided to {\bf WinBUGS} as data.

Next we do a number of organizational activities including bundling
the data for {\bf WinBUGS}, defining some initial values, the parameters to
monitor and some basic MCMC settings.  We choose initial values for
the activity centers ${\bf s}$ by generating uniform random numbers in
the state-space but, for the observed individuals, we replace those
values by each individual's mean trap coordinate for all encounters
{\small
\begin{verbatim}
sst<-cbind(runif(nind,Xl,Xu),runif(nind,Yl,Yu))  # starting values for s
for(i in 1:nind){
if(sum(y[i,])==0) next
sst[i,1]<- mean( X[y[i,]>0,1] )
sst[i,2]<- mean( X[y[i,]>0,2] )
}

data <- list (y=y,X=X,K=K,N=nind,J=J,Xl=Xl,Yl=Yl,Xu=Xu,Yu=Yu)
inits <- function(){
  list (alpha0=rnorm(1,-4,.4),alpha1=runif(1,1,2),s=sst)
}

library("R2WinBUGS")
parameters <- c("alpha0","alpha1")
nthin<-1
nc<-3
nb<-1000
ni<-2000
out <- bugs (data, inits, parameters, "SCR0a.txt", n.thin=nthin,
n.chains=nc, n.burnin=nb,n.iter=ni,debug=TRUE,working.dir=getwd())
\end{verbatim}
}
There is little to say about the preceding basic operations other than
to suggest that the interested reader explore the output and
additional analyses by running the script provided in the {\bf R}
package \mbox{\tt scrbook}.
 We ran $1000$ burn-in and $1000$ after burn-in, 3 chains,
to obtain 3000 posterior samples.  Because we know $N$ for this
particular data set we only have 2 parameters of the detection model
to summarize (\mbox{\tt alpha0} and \mbox{\tt alpha1}).  When the
object \mbox{\tt out} is produced we print a summary of the results as
follows:
{\small
\begin{verbatim}
> print(out,digits=3)
Inference for Bugs model at "SCR0a.txt", fit using WinBUGS,
 3 chains, each with 2000 iterations (first 1000 discarded)
 n.sims = 3000 iterations saved
            mean     sd    2.5%     25%    50%     75%   97.5%  Rhat n.eff
alpha0    -2.496  0.224  -2.954  -2.648  -2.48  -2.340  -2.091 1.013   190
alpha1     2.442  0.419   1.638   2.145   2.44   2.721   3.303 1.005   530
deviance 292.803 21.155 255.597 277.500 291.90 306.000 339.302 1.006   380

For each parameter, n.eff is a crude measure of effective sample size,
and Rhat is the potential scale reduction factor (at convergence, Rhat=1).

DIC info (using the rule, pD = Dbar-Dhat)
pD = -138.8 and DIC = 154.0
DIC is an estimate of expected predictive error (lower deviance is better).
\end{verbatim}
}

We know the data were generated with \mbox{\tt alpha0} $= -2.5$ and
\mbox{\tt alpha1 = -2}. The estimates look reasonably close to those
data-generating values and we probably feel pretty good about the
performance of the Bayesian analysis and MCMC algorithm that {\bf WinBUGS}
cooked-up based on our sample size of 1 data set.  It is worth noting
that the Rhat statistics indicate reasonable convergence but, as a
practical matter, we might choose to run the MCMC algorithm for
additional time to bring these closer to 1.0 and to increase the
effective posterior sample size (\mbox{\tt n.eff}). Other summary output includes
``deviance'' and related things including the deviance information
criterion (DIC). We discuss these things in Chapts. \ref{chapt.mcmc}
and \ref{chapt.gof}.



\section{Unknown N}
\label{scr0.sec.unknownN}

In all real applications $N$ is unknown and that fact is kind of an
important feature of the capture-recapture problem!  We handled this
important issue in Chapt. \ref{chapt.closed} using the method of data augmentation
which we apply here to achieve a realistic analysis of model SCR0. As
with the basic closed population models considered previously, we
formulate the problem here by augmenting our observed data set with a
number of ``all zero'' encounter histories - what we referred to in
Chapt. \ref{chapt.closed} as potential individuals. If $n$ is the number of observed
individuals, then let $M-n$ be the number of potential individuals in
the data set. For the basic $y_{ij}$ data structure (individuals x
traps encounter frequencies) we simply add additional rows of ``all
0'' observations to that data set. This is because such
``individuals'' are unobserved, and therefore necessarily have
$y_{ij}=0$ for all $j$.  A data set, say with 4 traps and 6 individuals,
augmented with 4 pseudo-individuals therefore might look like this:
{\small
\begin{verbatim}
      trap1 trap2 trap3 trap4
 [1,]     1     0     0     0
 [2,]     0     2     0     0
 [3,]     0     0     0     1
 [4,]     0     1     0     0
 [5,]     0     0     1     1
 [6,]     1     0     1     0
 [7,]     0     0     0     0
 [8,]     0     0     0     0
 [9,]     0     0     0     0
[10,]     0     0     0     0
\end{verbatim}
}
We typically have more than 4 traps and, if we're fortunate, many more
individuals in our data set.

For the augmented data, we introduce a set of binary latent variables
(the data augmentation variables), $z_{i}$, and the model is extended
to describe $\Pr(z_{i} = 1)$ which is, in the context of this problem,
the probability that an individual in the augmented data set is a
member of the population that was sampled. In other words, if $z_{i}=1$
for one of the ``all zero'' encounter histories, this is implied to be
a sampling zero whereas observations for which $z_{i}=0$ are
``structural zeros'' under the model.

How big does the augmented data set have to be? We discussed this
issue in Chapt. \ref{chapt.closed} where we noted that the size of the data set is
equivalent to the upper limit of a uniform prior distribution on $N$.
Practically speaking, it should be sufficiently large so that the
posterior distribution for $N$ is not truncated. On the other hand, if
it is too large then unnecessary calculations are being done. An
approach to choosing $M$ by trial-and-error is indicated. You can take
a ballpark estimate of the probability that an individual is captured
at all during the study, say $\tilde{p}$, which is related to the
``per sample'' encounter probability, $p$, by $\tilde{p} = 1-(1-p)^{K}$, obtain $N$ as $n/\tilde{p}$, and then set $M =
2*N$, as a first guess. Do a short MCMC run and then consider whether
you need to increase $M$. See Chapt. \ref{chapt.mcmc} for an
example of this. \citet[][ch. 6]{kery_schaub:2011}
 provide an assessment of choosing $M$ in closed population models.

Analysis by data augmentation removes $N$ as an explicit parameter of
the model. Instead, $N$ is a derived parameter, computed by $N=
\sum_{i=1}^{M} z_{i}$. Similarly, {\it density}, $D$, is also a
derived parameter computed as $D=N/area({\cal S})$. For our
simulator, we're using an $8 \times 8$ state-space and thus we will
compute $D$ as $D=N/64$.

\subsection{Analysis using data augmentation in WinBUGS}

As before we begin by obtaining a data set using our \mbox{\tt
  simSCR0.fn} routine and then harvesting the required data objects
from the resulting data list.  Note that we use the \mbox{\tt
  discard0=TRUE} option this time so that we get a ``real'' data set
with no all-zero encounter histories. After harvesting the data we
produce the {\bf WinBUGS} model specification which now includes $M$
encounter histories including the augmented potential individuals, the
data augmentation parameters $z_{i}$, and the data augmentation
parameter $\psi$.
{\small
\begin{verbatim}
data<-simSCR0.fn(discard0=TRUE,sd=2013)
y<-data$Y
traplocs<-data$traplocs
nind<-nrow(y)
X<-data$traplocs
J<-nrow(X)
Xl<-data$xlim[1]
Yl<-data$ylim[1]
Xu<-data$xlim[2]
Yu<-data$ylim[2]

cat("
model {
alpha0~dnorm(0,.1)
logit(p0)<- alpha0
alpha1~dnorm(0,.1)
psi~dunif(0,1)

for(i in 1:M){
 z[i] ~ dbern(psi)
 s[i,1]~dunif(Xl,Xu)
 s[i,2]~dunif(Yl,Yu)
for(j in 1:J){
d[i,j]<- pow(pow(s[i,1]-X[j,1],2) + pow(s[i,2]-X[j,2],2),0.5)
y[i,j] ~ dbin(p[i,j],K)
p[i,j]<- z[i]*p0*exp(- alpha1*d[i,j]*d[i,j])
}
}
N<-sum(z[])
D<-N/64
}
",file = "SCR0a.txt")
\end{verbatim}
}

To prepare our data we have to augment the data matrix \mbox{\tt y}
with $M-n$ all-zero encounter histories, we have to create starting
values for the variables $z_{i}$ and also the activity centers ${\bf
  s}_{i}$ of which, for each, we require $M$ values. Otherwise the
remainder of the code for bundling the data, creating initial values
and executing {\bf WinBUGS} looks much the same as before except with more
or differently named arguments.
{\small
\begin{verbatim}
## Data augmentation stuff
M<-200
y<-rbind(y,matrix(0,nrow=M-nind,ncol=ncol(y)))
z<-c(rep(1,nind),rep(0,M-nind))

sst<-cbind(runif(M,Xl,Xu),runif(M,Yl,Yu))  # starting values for s
for(i in 1:nind){
if(sum(y[i,])==0) next
sst[i,1]<- mean( X[y[i,]>0,1] )
sst[i,2]<- mean( X[y[i,]>0,2] )
}
data <- list (y=y,X=X,K=K,M=M,J=J,Xl=Xl,Yl=Yl,Xu=Xu,Yu=Yu)
inits <- function(){
  list (alpha0=rnorm(1,-4,.4),alpha1=runif(1,1,2),s=sst,z=z)
}

library("R2WinBUGS")
parameters <- c("alpha0","alpha1","N")
nthin<-1
nc<-3
nb<-1000
ni<-2000
out <- bugs (data, inits, parameters, "SCR0a.txt", n.thin=nthin,n.chains=nc,
 n.burnin=nb,n.iter=ni,debug=TRUE,working.dir=getwd())
\end{verbatim}
}

{\bf Remarks}:  (1) Note the differences in this new {\bf WinBUGS} model
with that appearing in the known-$N$ version.  (2) Also the input data
has changed - the augmented data set has more rows of
all-zeros. Previously we knew that $N=100$ but in this analysis we
pretend not to know $N$, but think that $N=200$ is a good upper-bound;
(3) Population size $N({\cal S})$ is a derived parameter, being computed by
summing up all of the data augmentation variables $z_{i}$ (as we've
done previously in Chapt. \ref{chapt.closed}); (4) Density, $D\equiv D({\cal S})$, is also a derived
parameter. Summarizing the output from {\bf WinBUGS} produces:
{\small
\begin{verbatim}
> print(out1,digits=2)
Inference for Bugs model at "SCR0a.txt", fit using WinBUGS,
 3 chains, each with 2000 iterations (first 1000 discarded)
 n.sims = 3000 iterations saved
           mean    sd   2.5%    25%    50%    75%  97.5% Rhat n.eff
alpha0    -2.57  0.23  -3.04  -2.72  -2.56  -2.41  -2.15 1.01   320
alpha1     2.46  0.42   1.63   2.16   2.46   2.73   3.33 1.02   120
N        113.62 15.73  86.00 102.00 113.00 124.00 147.00 1.01   260
D          1.78  0.25   1.34   1.59   1.77   1.94   2.30 1.01   260
deviance 302.60 23.67 261.19 285.47 301.50 317.90 354.91 1.00  1400

For each parameter, n.eff is a crude measure of effective sample size,
and Rhat is the potential scale reduction factor (at convergence, Rhat=1).

DIC info (using the rule, pD = var(deviance)/2)
pD = 279.9 and DIC = 582.5
DIC is an estimate of expected predictive error (lower deviance is better).
\end{verbatim}
}

The column labeled ``MC error'' is the Monte Carlo error - the error
inherent in the attempt to compute these posterior summaries by
MCMC
(see secs.  for discussion of this quantity
\ref{glms.sec.convergence} \ref{mcmc.sec.mcmcsummary}).
It is desirable to run the Markov chain algorithm long enough so
as to reduce the MC error to a tolerable level. What constitutes
tolerable is up to the investigator. Certainly less than 1\% is called
for. As a general rule, Rhat gets closer to 1 and MC error decreases
toward 0 as the number of iterations increases.  We see that the
estimated parameters ($\alpha_0$ and $\alpha1$) are comparable to the
previous results obtained for the known-$N$ case, and also not too
different from the data-generating values. The posterior of $N$
overlaps the data-generating value substantially with a mean of
$113.62$.  To obtain these results we fitted the true data-generating
model, that based on the half-normal detection model, to a single
simulated data set. For fun and excitement we fit the {\it wrong}
model, one with a logistic-linear detection model
(Eq. \ref{scr0.eq.logit}),
to the same  
data set. This is easily achieved by modifying the {\bf WinBUGS} model
specification above, although we provide the {\bf R} script in the
{\bf R} package \mbox{\tt scrbook}.
Those results are given below. We see that the estimate of
$N$, the main parameter of interest, is very similar to that obtained
under the correct model, convergence is worse (as measured by Rhat)
which may not have anything to do with the model being wrong,
and the posterior deviance favors the correct model (it is smaller) while the DIC does not.
We consider 
 the effectiveness of DIC for carrying-out model selection in chapter
\ref{chapt.gof}.
{\small
\begin{verbatim}
> print(out2,digits=2)
Inference for Bugs model at "SCR0a.txt", fit using WinBUGS,
 3 chains, each with 2000 iterations (first 1000 discarded)
 n.sims = 3000 iterations saved
           mean    sd   2.5%    25%    50%    75%  97.5% Rhat n.eff
alpha0    -1.59  0.27  -2.16  -1.77  -1.58  -1.42  -1.07 1.05    60
beta       3.77  0.43   2.92   3.48   3.79   4.05   4.66 1.04    70
N        122.57 18.67  90.00 109.00 122.00 135.00 163.00 1.00  3000
D          1.92  0.29   1.41   1.70   1.91   2.11   2.55 1.00  3000
deviance 312.67 22.43 271.00 297.20 311.50 327.00 359.60 1.02   130

For each parameter, n.eff is a crude measure of effective sample size,
and Rhat is the potential scale reduction factor (at convergence, Rhat=1).

DIC info (using the rule, pD = var(deviance)/2)
pD = 247.5 and DIC = 560.1
DIC is an estimate of expected predictive error (lower deviance is better).
\end{verbatim}
}

\subsection{Use of other BUGS engines: JAGS}

There are two other popular {\bf BUGS} engines in widespread use: {\bf
  OpenBUGS} \citep{thomas_etal:2006} and {\bf JAGS}
\citep{plummer:2003}. Both of these are easily called from {\bf
  R}. {\bf OpenBUGS} can be used instead of {\bf WinBUGS} by changing
the package option in the \mbox{\tt bugs} call to \mbox{\tt
  package=OpenBUGS}.  {\bf JAGS} can be called using the function
\mbox{\tt jags()} in package \mbox{\tt R2JAGS} which has nearly the
same arguments as \mbox{\tt bugs()}.  We prefer to use the {\bf R}
library \mbox{\tt rjags} \citep{plummer:2009} which has a slightly
different implementation that we demonstrate here as we reanalyze the
simulated data set in the previous section (note: the same {\bf R}
commands are used to generate the data and package the data, inits and
parameters to monitor). The function \mbox{\tt jags.model} is used to
initialize the model and run the MCMC algorithm for an adaptive
burn-in period.  Then the Markov chains are updated using \mbox{\tt
  coda.samples()} to obtain posterior samples for analysis, as
follows:
\begin{verbatim}
jm<- jags.model("SCR0a.txt", data=data, inits=inits, n.chains=nc,
                 n.adapt=nb))
jm<- coda.samples(jm, parameters, n.iter=ni-nb, thin=nthin)
\end{verbatim}
We find that {\bf JAGS} seems to be 20-30\% faster for the basic SCR
model which the reader can evaluate using the function \mbox{\tt
  SCR0bayes} in the {\bf R} package \mbox{\tt scrbook}.



\section{Wolverine Camera Trapping Study}
\label{scr0.sec.wolverine}

We provide an analysis here of A. Magoun's wolverine data
\citep{magoun_etal:2011, royle_etal:2011jwm}. The study took place in SE
Alaska (Fig. \ref{scr0.fig.wolverinelocs}) where 37 cameras were
operational for variable periods of time (min = 5 days, max = 108
days, median = 45 days).  A consequence of this is that the binomial
sample size $K$ (see Eq. \ref{scr0.eq.bin})
 is variable for each camera. Thus, we
must provide a matrix of sample sizes as data to {\bf BUGS} and modify the
model specification in sec. \ref{scr0.sec.unknownN}
accordingly. Our treatment of the
data here is based on the analysis of  \citet{royle_etal:2011jwm}.

\begin{figure}
\begin{center}
\includegraphics[height=3in]{Ch4/figs/wolverinelocs}
\end{center}
\caption{Wolverine camera trap locations from \citet{magoun_etal:2011}.}
\label{scr0.fig.wolverinelocs}
\end{figure}

To carry-out an analysis of these data, we require the matrix of trap
coordinates and the encounter history data.  We store data in an the
``scr flat format'' (see sec.  \ref{scr0.sec.formats} above), an
efficient file format which is easily manipulated and also used as the
input file format in {\bf SPACECAP} \citep{gopalaswamy_etal:2012} and
in the {\bf R} package \mbox{\tt SCRbayes} \citep{russell_etal:2012}.
To illustrate this format, the wolverine data are available in the
package \mbox{\tt scrbook} by typing:
\begin{verbatim}
data(wolverine)
\end{verbatim}
which contains a list having elements \mbox{\tt wcaps} and
\mbox{\tt wtraps}.
The ``encounter data file''
\mbox{\tt wcaps}  has 3 columns and 115 rows, each representing a
unique encounter event including the trap identity, the individual
identity and the sample occasion index (\mbox{\tt sample}).
The first 10 rows of this matrix are as
follows:
{\small
\begin{verbatim}
> wolverine$wcaps[1:10,]
       trapid individual sample
  [1,]      1          2    127
  [2,]      1          2    128
  [3,]      1          2    129
  [4,]      1         18    130
  [5,]      2          3    106
  [6,]      2         18    104
  [7,]      5          5     73
  [8,]      5          5     89
  [9,]      6         18    117
 [10,]      6         18    118
\end{verbatim}
}
Each row is a unique 
individual/trap encounter, and the 3 variables (columns) are: 
\mbox{\tt trapid} -- an
integer that runs from \mbox{\tt 1:ntraps}, \mbox{\tt individual} runs from
\mbox{\tt 1:nind} and
\mbox{\tt sample} 
runs from \mbox{\tt 1:nperiods}. Often (as the case here) \mbox{\tt
  sample} 
will
correspond to daily sample intervals. The variable \mbox{\tt trapid} will have to
correspond to the row of a matrix containing the trap coordinates - in
this case the file \mbox{\tt wtraps} which we describe further below.

Note that the information provided in this encounter data file
\mbox{\tt wcaps}
does not represent a completely informative summary
of the data. For example, if no individuals were captured in a certain
trap or during a certain period, then this compact data format will
have no record. Thus we will need to know ntraps and nperiods when
reformatting this SCR data format into a 2-d encounter frequency
matrix or 3-d array. In addition, the encounter data file does not
provide information about which periods each trap was operated. This
additional information is also necessary as the trap-specific sample
sizes must be passed to {\bf BUGS} as data. We provide this information in a
2nd data file, along with the trap coordinates, in the 
 ``trap deployment'' file which is described
below.

For our purposes we
need to convert the \mbox{\tt wcaps} file
into the $n \times J$ array of
binomial encounter frequencies, although more general models might
require an encounter-history formulation of the model which requires a
full 3-d array.  To obtain our $n \times J$ encounter frequency
matrix, we do this the hard way by first converting the encounter data
file into a 3-d array and then summarize to trap totals. We have a
handy function \mbox{\tt SCR23darray.fn} which takes the compact
encounter data file with optional arguments ntraps and nperiods, and
converts it to a 3-d array, and then we use the {\bf R} function
\mbox{\tt apply} to summarize over the ``sample'' period dimension (by
convention here, this is the 2nd dimension). To apply this to the
wolverine
data in order to compuate the 3-d array we do this:
{\small
\begin{verbatim}
y3d <-SCR23darray.fn(wolverine$wcaps,wolverine$wtraps)
y <- apply(y3d,c(1,3),sum)
\end{verbatim}
}
See the help file for more information on \mbox{\tt SCR23darray.fn}.
The 3-d array is necessary to fit certain types
of models (e.g., behavioral response) and this is why we sometimes
will require this maximally informative 3-d data format but, here, we
analyze the summarized data.



The other important information needed to fit SCR models is the
``trap deployment'' file
which provides the additional information
not contained in the encounter data file. The traps file has \mbox{\tt
  nperiods} $+ 3$ columns. The first column is assumed to be a trap identifier,
columns 2 and 3 are the easting and northing coordinates (assumed to
be in a Euclidean coordinate system), and columns 4 to (\mbox{\tt nperiods} + 3)
are binary indicators of whether each trap was operational in each
time period. The first 10 rows (out of 37) and 10 columns (out of 167)
of the trap deployment file for the wolverine data are:
{\small
\begin{verbatim}
> wolverine$wtraps[1:10,1:10]

   Easting Northing 1 2 3 4 5 6 7 8 
1   632538  6316012 0 0 0 0 0 0 0 0
2   634822  6316568 1 1 1 1 1 1 1 1
3   638455  6309781 0 0 0 0 0 0 0 0
4   634649  6320016 0 0 0 0 0 0 0 0
5   637738  6313994 0 0 0 0 0 0 0 0
6   625278  6318386 0 0 0 0 0 0 0 0
7   631690  6325157 0 0 0 0 0 0 0 0
8   632631  6316609 0 0 0 0 0 0 0 0
9   631374  6331273 0 0 0 0 0 0 0 0
10  634068  6328575 0 0 0 0 0 0 0 0
\end{verbatim}
}
This tells us that trap 2 was operated in periods (days) 1-7 but the other
traps were not operational during those periods. It is extremely
important to recognize that each trap was operated for a variable
period of time and thus the binomial "sample size" is different for
each, and this needs to be accounted for in the {\bf BUGS} model specification.
To compute the vector of sample sizes $K$, and extract the trap
locations,  we do this:
\begin{verbatim}
traps<- wolverine$wtraps
traplocs<- traps[,1:2]
K<- apply(traps[,3:ncol(traps)],1,sum)
\end{verbatim}
This results in a matrix traplocs which contains the coordinates of
each trap and a vector $K$ containing the number of days that each trap
was operational. We now have all the information required to fit a
basic SCR model in {\bf BUGS}.

Summarizing these data files for the wolverine study, we see that 21
unique individuals were captured a total of 115 times. Most
individuals were captured 1-6 times, with 4, 1, 4, 3, 1, and 2
individuals captured 1-6 times, respectively.  In addition, 1
individual was captured each 8 and 14 times and 2 individuals each
were captured 10 and 13 times.  The number of unique traps that
captured a particular individual ranged from 1-6, with 5, 10, 3, 1, 1,
and 1 individual captured in each of 1-6 traps, respectively, for a
total of 50 unique wolverine-trap encounters.  These numbers might be
hard to get your mind around whereas some tabular summary is often
more convenient. For that it seems natural to tabulate individuals by
trap and total encounter frequencies. The spatial information in SCR
data is based on multi-trap captures\footnote{I will add more 
context here on revision about spatial recaptures, lost recaptures,
ordinary recaptures. Function \mbox{\tt SCRsmy} in \mbox{\tt
  scrbook}}, 
and so, it is informative to
understand how many unique traps each individual is captured in. At
the same, it is useful to understand how many total captures we have
of each individual because this is, in an intuitive sense, the
effective sample size.  So, we reproduce Table 1 from
\citet{royle_etal:2011jwm} which shows the trap and total encounter
frequencies:

\begin{table} [htp]
  \caption{Individual frequencies of capture for wolverines captured
    in camera traps in Southeast Alaska in 2008. Rows index unique
    trap frequencies and columns represent total number of captures
    (e.g., we captured 4 individuals 1 time, necessarily in only 1
    trap; we captured 3 individuals 3 times but in 2 different traps)}
\centering
\begin{tabular}{c c c c c c c c c c c}
\hline
 & & & & & & & &  No.&of&captures \\
\hline
No. of traps & 1 & 2 & 3 & 4 & 5 & 6 & 8 & 10 &13 &14 \\
\hline
1 & 4 & 1 & 0 & 0 & 0 & 0 & 0 & 0 & 0 & 0 \\
2 & 0 & 0 & 3 & 3 & 0 & 2 & 1 & 2 & 0 & 0 \\
3 & 0 & 0 & 1 & 1 & 0 & 0 & 0 & 0 & 0 & 1 \\
4 & 0 & 0 & 0 & 0 & 0 & 0 & 0 & 0 & 1 & 0 \\
5 & 0 & 0 & 0 & 0 & 1 & 0 & 0 & 0 & 0 & 0 \\
6 & 0 & 0 & 0 & 0 & 0 & 0 & 0 & 0 & 1 & 0 \\
\hline
\end{tabular}
\end{table}

\subsection{Fitting the model in WinBUGS}

For illustrative purposes here we fit the simplest SCR model with the
half-normal distance function although we revisit these data with more
complex models in later chapters. The model is summarized by the
following 3 components:
\begin{itemize}
\item[(1)] $y_{ij}|{\bf s}_{i} \sim \mbox{Bin}(K, z_{i}\; p_{ij})$
\item[(2)] $p_{ij} = p_{0} \exp(-\alpha1 \; ||{\bf s}_{i}-x_{j}||^2)$
\item[(3)] $ {\bf s}_{i} \sim \mbox{Unif}({\cal S})$
\item[(4)] $ z_{i} \sim \mbox{Bern}(\psi)$
\end{itemize}
We assume customary flat priors on the structural (hyper-) parameters
of the model, $\alpha_{0} = \mbox{logit}(p_{0})$, $\alpha1$ and $\psi$.  It remains to define the
state-space ${\cal S}$. For this, we nested the trap array (Fig.
\ref{scr0.fig.wolverinelocs}) in a
a rectangular state-space extending $20$ km beyond the traps in each cardinal
direction.  We also considered larger state-spaces up to 50 km to
evaluate that choice.  The buffer of the state space should be larger
enough so that individuals beyond the state-space boundary are not
likely to be encountered. Thus some knowledge of typical space usage
patterns of the species is useful.  For the analysis, 
we scaled the coordinate system 
so that a unit distance was equal to $10$ km, producing a rectangular
state-space of dimension $9.88 \times 10.5$ units ($area = 10374$ km$^2$)
within which the trap array was nested. As a general rule, we
recommend scaling the state-space so that it is defined near the
origin $(x,y)=(0,0)$. While the scaling of the coordinate system is
theoretically irrelevant, a poorly scaled coordinate system can
produce Markov chains that mix poorly.  For the scaled coordinate
system we fit models for various choices of a rectangular state-space
based on 
buffers from 1.0 to 5.0 units on the scaled coordinate system (10 km to
50 km). In the {\bf R} package \mbox{\tt scrbook} we provide a
function
\mbox{\tt wolvSCR0.fn} which will fit the basic SCR model. For
example, to fit the model in 
{\bf WinBUGS} using data augmentation with $M=300$ potential individuals,
using 3 Markov chains each of 12000 total iterations, discarding the
first 2000 as burn-in, we execute the following {\bf R} commands:
{\small
\begin{verbatim}
library("scrbook")
data(wolverine)
traps<-wolverine$wtraps
y3d <-SCR23darray.fn(wolverine$wcaps,wolverine$wtraps)
toad<-wolvSCR0.fn(y3d,traps,nb=12000,ni=2000,delta=1,M=300)
\end{verbatim}
}
The argument $\delta$ determines the buffer size of the state-space.
Note that this analysis takes 
between 1-2 hours on many machines so we recommend trying it out with
lower values of $M$ and fewer iterations.
The output
follows (note, we have a parameter ``sigma'' which we discuss
shortly)\footnote{Final as of 1/11/2012. 
output saved in \mbox{\tt wolv-buffer-study.txt}}:

{\small
\begin{verbatim}
All based on 3 chains, 12k iters, 2k burn, 30k total
Buffer = 10 km
           mean    sd   2.5%    25%    50%    75%  97.5% Rhat n.eff
psi        0.13  0.03   0.08   0.11   0.13   0.15   0.20    1 10000
sigma      0.65  0.06   0.55   0.61   0.64   0.68   0.76    1  1800
p0         0.06  0.01   0.04   0.05   0.06   0.06   0.08    1 20000
N         39.63  6.70  29.00  35.00  39.00  44.00  54.00    1  7100
D          5.92  1.00   4.33   5.22   5.82   6.57   8.06    1  7100
beta       1.23  0.21   0.85   1.08   1.22   1.36   1.66    1  1800
deviance 410.05 12.06 388.70 401.50 409.20 417.80 435.60    1 22000

Buffer = 15 km
 n.sims = 30000 iterations saved
           mean    sd   2.5%    25%    50%    75%  97.5% Rhat n.eff
psi        0.16  0.04   0.10   0.14   0.16   0.19   0.25    1  3800
sigma      0.64  0.06   0.54   0.60   0.64   0.67   0.76    1   510
p0         0.06  0.01   0.04   0.05   0.06   0.06   0.08    1 17000
N         48.77  9.19  34.00  42.00  48.00  54.00  69.00    1  3300
D          5.78  1.09   4.03   4.98   5.69   6.40   8.18    1  3300
beta       1.25  0.21   0.86   1.10   1.24   1.39   1.70    1   510
deviance 411.00 12.16 389.50 402.40 410.30 418.70 437.00    1  5400

Buffer = 20 km
           mean    sd   2.5%    25%    50%    75%  97.5% Rhat n.eff
psi        0.20  0.05   0.12   0.17   0.20   0.23   0.30    1 16000
sigma      0.64  0.06   0.54   0.60   0.63   0.67   0.76    1  1200
p0         0.06  0.01   0.04   0.05   0.06   0.06   0.08    1  1900
N         59.84 11.89  40.00  51.00  59.00  67.00  86.00    1 20000
D          5.77  1.15   3.86   4.92   5.69   6.46   8.29    1 20000
beta       1.26  0.21   0.87   1.11   1.25   1.40   1.71    1  1200
deviance 411.01 12.36 389.10 402.30 410.20 418.80 437.50    1  1500

Buffer = 25 km
           mean    sd   2.5%    25%    50%    75%  97.5% Rhat n.eff
psi        0.24  0.05   0.15   0.20   0.24   0.28   0.36    1  3400
sigma      0.64  0.05   0.54   0.60   0.63   0.67   0.75    1  3600
p0         0.06  0.01   0.04   0.05   0.06   0.06   0.08    1  5000
N         72.40 14.72  47.00  62.00  71.00  81.00 105.00    1  2700
D          5.79  1.18   3.76   4.96   5.67   6.47   8.39    1  2700
beta       1.26  0.21   0.88   1.12   1.25   1.40   1.71    1  3600
deviance 411.35 12.23 389.70 402.70 410.55 419.20 437.20    1 30000

Buffer = 30 km
           mean    sd   2.5%    25%    50%    75%  97.5% Rhat n.eff
psi        0.29  0.06   0.18   0.24   0.28   0.33   0.43    1  3100
sigma      0.63  0.05   0.54   0.60   0.63   0.67   0.75    1  5600
p0         0.06  0.01   0.04   0.05   0.06   0.06   0.08    1 11000
N         86.42 17.98  56.00  74.00  85.00  97.00 126.02    1  3900
D          5.82  1.21   3.77   4.98   5.72   6.53   8.49    1  3900
beta       1.27  0.21   0.88   1.12   1.26   1.41   1.71    1  5600
deviance 411.06 12.37 389.20 402.50 410.20 418.90 437.60    1 10000

Buffer = 35 km
           mean    sd   2.5%    25%    50%    75%  97.5% Rhat n.eff
psi        0.34  0.08   0.21   0.29   0.34   0.39   0.50    1 30000
sigma      0.63  0.05   0.54   0.60   0.63   0.67   0.75    1  4500
p0         0.06  0.01   0.04   0.05   0.06   0.06   0.08    1 24000
N        101.79 21.54  65.00  87.00 100.00 115.00 148.00    1 30000
D          5.85  1.24   3.74   5.00   5.75   6.61   8.51    1 30000
beta       1.27  0.21   0.89   1.12   1.25   1.40   1.70    1  4500
deviance 411.10 12.20 389.50 402.40 410.30 418.90 437.20    1 22000

Buffer = 40 km
           mean    sd   2.5%    25%    50%    75%  97.5% Rhat n.eff
psi        0.39  0.09   0.24   0.33   0.39   0.45   0.60 1.01   480
sigma      0.64  0.05   0.54   0.60   0.63   0.67   0.75 1.01   410
p0         0.06  0.01   0.04   0.05   0.06   0.06   0.08 1.00 21000
N        118.05 26.14  75.00 100.00 116.00 133.00 178.00 1.01   450
D          5.87  1.30   3.73   4.97   5.76   6.61   8.84 1.01   450
beta       1.27  0.21   0.89   1.12   1.25   1.40   1.72 1.01   410
deviance 411.37 12.35 389.30 402.60 410.60 419.30 437.50 1.00  9700

Buffer = 45 km
           mean    sd   2.5%    25%    50%    75%  97.5% Rhat n.eff
psi        0.45  0.10   0.28   0.38   0.44   0.51   0.66    1  3600
sigma      0.64  0.05   0.54   0.60   0.63   0.67   0.75    1 10000
p0         0.06  0.01   0.04   0.05   0.06   0.06   0.08    1  8100
N        134.43 28.68  85.00 114.00 132.00 153.00 196.00    1  3300
D          5.83  1.24   3.68   4.94   5.72   6.63   8.50    1  3300
beta       1.26  0.21   0.88   1.11   1.24   1.39   1.69    1 10000
deviance 411.36 12.19 389.60 402.70 410.60 419.10 437.30    1  9400

Buffer = 50 km
           mean    sd   2.5%    25%    50%    75%  97.5% Rhat n.eff
psi        0.51  0.11   0.31   0.43   0.50   0.57   0.74    1  3200
sigma      0.63  0.05   0.54   0.60   0.63   0.67   0.75    1  4700
p0         0.06  0.01   0.04   0.05   0.06   0.06   0.08    1  3300
N        151.61 31.65  96.00 129.00 149.00 172.00 221.00    1  3400
D          5.79  1.21   3.66   4.92   5.69   6.56   8.43    1  3400
beta       1.27  0.21   0.89   1.12   1.25   1.40   1.70    1  4700
deviance 410.81 12.18 389.20 402.30 410.10 418.50 436.70    1 30000

Buffer = 55 km 
           mean    sd   2.5%    25%    50%    75%  97.5% Rhat n.eff
psi        0.56  0.12   0.35   0.48   0.55   0.64   0.82 1.01   260
sigma      0.64  0.05   0.54   0.60   0.63   0.67   0.76 1.00  1600
p0         0.06  0.01   0.04   0.05   0.06   0.06   0.08 1.00 30000
N        169.28 35.81 108.00 143.00 166.00 192.00 247.00 1.01   260
D          5.73  1.21   3.66   4.84   5.62   6.50   8.36 1.01   260
beta       1.25  0.21   0.88   1.11   1.24   1.39   1.69 1.00  1600
deviance 411.28 12.38 389.40 402.60 410.50 419.10 437.50 1.00 26000
\end{verbatim}
}

We see that the estimated density is roughly consistent as we increase
the state-space buffer from $15$ to $50$ $km$. We do note that the data
augmentation parameter $\psi$ (and, correspondingly, $N$) increase with
the size of the state space in accordance with the deterministic
relationship $N= D*A$. However, density is constant more or less as we
increase the size of the state-space beyond a certain point.  For the
10 $km$ state-space buffer, we see a slight effect on the posterior
distribution of $D$. This is not a bug but rather a feature. As we noted
above, the state-space is part of the model.


\subsection{Thoughts on the Wolverine Analysis}

Our point estimate of wolverine density from this study, using the
posterior mean from the state-space based on the 20
$km$ buffer, is 
approximately $5.77$ individuals/1000 $km^2$ with  a 95\% posterior
interval of $[3.86, 8.29]$. Density is estimated imprecisely
which might not be surprising given the low sample size ($n=21$
individuals!). This seems to be a basic feature of carnivore studies
although it should not (in our view) preclude the study of their
populations nor attempts to estimate density or vital rates.

One thing we haven't talked about yet is that we can calibrate the
desired size of the state-space by looking at the estimated home range
radius of the species. For some models it is possible to convert the
parameter $\alpha1$ directly into the home range radius (sec. 
XXX MISSING XYZ). For the half-normal model we interpret the half-normal scale
parameter $\sigma$ which is related to $\alpha1$ by $\alpha1 =
1/(2\sigma^2)$ as the radius of a bivariate normal movement model. 
In this case $\sigma = 1.82$ standardized units = 18.2 $km$ which 
translates into a home range area of XXXX MISSING XXXXX. 

It is worth thinking about this model, and these estimates, computed
under a rectangular state space roughly centered over the trapping
array (Fig. \ref{scr0.fig.wolverinelocs}).
Does it make sense to define the state-space to
include, for example, ocean? What are the possible consequences of
this? What can we do about it?  There's no reason at all that the
state space has to be a regular polygon -- we defined it as such here
strictly for convenience and for ease of implementation in {\bf WinBUGS}
where it enables us to specify the prior for the activity centers as
uniform priors for each coordinate.  While it would be possible to
define a more realistic state-space using some general polygon GIS coverage, it
might take some effort to implement that in the {\bf BUGS} language
but it is not difficult to devise custom MCMC algorithms to do that
(see Chapt. \ref{chapt.mcmc}).
Alternatively, we recommend
using a discrete representation of the state-space -- i.e., approximate
${\cal S}$ by a grid of $G$ points. We discuss this in sec. 
\ref{scr0.sec.discrete}.


\section{Constructing Density Maps}
\label{scr0.sec.mapping}

One of the most useful aspects of SCR models is that they are
parameterized in terms of individual locations - i.e., {\it where}
each individual lives -- and, thus, we can compute many useful or
interesting summaries of the activity centers.  For example, we can
make a spatial density plot by tallying up the number of activity
centers ${\bf s}_{i}$ in boxes of arbitrary size and then producing a
nice multi-color spatial plot of those which, we find, increases the
acceptance probability of your manuscripts by a substantial amount.
We discussed in Chapt. \ref{chapt.glms} the idea of estimating derived
parameters from MCMC output. In SCR models, there are many derived
parameters that are functions of the latent point locations $({\bf
  s}_{1},\ldots, {\bf s}_{N})$. In the present context, the number of
individuals living in any well-defined polygon is a derived
parameter. Specifically, let $B({\bf x})$ indicate a box centered at
${\bf x}$ then
\[
N({\bf x})=\sum_{i} I({\bf s}_{i} \in B({\bf x}))
\]
is the population size of box $B({\bf x})$, and $D({\bf x}) = N({\bf
  x})/||B({\bf x})||$ is the local density. These are just ``derived
parameters'' (see Chapt.  \ref{chapt.glms}) which are estimated from
MCMC output using the appropriate Monte Carlo average. One thing to be
careful about, in the context of models in which $N$ is unknown, is
that, for each MCMC iteration $m$, we only tabulate those activity
centers which correspond to individuals in the sampled
population. i.e., for which the data augmentation variable $z_{i} =
1$.  In this case, we take all of the output for MCMC iterations
$m=1,2,\ldots,\mbox{\tt niter}$ and compute this summary:
\[
   N({\bf x},m) = \sum_{z_{i,m}=1} I(s_{i,m} \in B({\bf x}))
\]
Thus, $N({\bf x},1),N({\bf x},2),\dots,$ is the Markov chain for
parameter $N({\bf x})$.  In what follows we will provide a set of {\bf
  R} commands for doing this calculations and making a basic image
plot from the MCMC output.

{\flushleft \bf Step 1:} Define the center points of each box, $B({\bf
  x})$, or point at which local density will be estimated:
\begin{verbatim}
xg<-seq(Xl,Xu,,50)
yg<-seq(Yl,Yu,,50)
\end{verbatim}

{\flushleft \bf Step 2:} Extract the MCMC histories for the activity
centers and the data augmentation variables.  Note that these are each
$N \times \mbox{\tt niter}$ matrices:
\begin{verbatim}
Sxout<-out$sims.list$s[,,1]
Syout<-out$sims.list$s[,,2]
z<-out$sims.list$z
\end{verbatim}

{\flushleft \bf Step 3:} We associate each coordinate with the proper
box using the {\bf R} command \mbox{\tt cut()}. Note that we keep only
the activity centers for which $z=1$ (i.e., individuals that belong to
the population of size $N$):
\begin{verbatim}
Sxout<-cut(Sxout[z==1],breaks=xg,include.lowest=TRUE)
Syout<-cut(Syout[z==1],breaks=yg,include.lowest=TRUE)
\end{verbatim}

{\flushleft \bf Step 4:} Use the \mbox{\tt table()} command to tally
up how many activity centers are in each $B(x)$:
\begin{verbatim}
Dn<-table(Sxout,Syout)
\end{verbatim}

{\flushleft \bf Step 5:} Use the \mbox{\tt image()} command to display
the resulting matrix.
\begin{verbatim}
image(xg,yg,Dn/nrow(z),col=terrain.colors(10))
\end{verbatim}
Praise the Lord! This map is somewhat useful or at least it looks
pretty and will facilitate the publication of your papers.

It is worth emphasizing here that density maps will not usually appear
uniform despite that we have assumed that activity centers are
uniformly distributed. This is because the observed encounters of
individuals provide direct information about the location of the
$i=1,2,\ldots,n$ activity centers and thus their ``estimated''
locations will be affected by the observations. In a limiting sense,
were we to sample space intensely enough, every individual would be
captured a number of times and we would have considerable information
about all $N$ point locations. Consequently, the uniform prior would
have almost no influence at all on the estimated density surface in
this limiting situation. Thus, in practice, the influence of the
uniformity assumption increases as the fraction of the population
encountered decreases.

{\bf On the non-intuitiveness of \mbox{\tt image()} } -- the {\bf R}
function \mbox{\tt image()} might
not be very intuitive to some -- it plots $M[1,1]$ in the lower left
corner. If you want $M[]$ to be plotted ``as
you look at it'' then $M[1,1]$ should be in the upper left corner.  We
have a function \mbox{\tt rot()} which does that. If you do \mbox{\tt image(rot(M))} then it
puts it on the monitor as if it was a map you were looking at.  You
can always specify the $x$ and $y-$ labels explicitly as we did above.

{\bf Spatial dot plots } -- Now here is a cruder version based on the
``spatial dot map'' function \mbox{\tt spatial.plot}, which uses
the function \mbox{\tt image.scale()}.
The \mbox{\tt spatial.plot} function requires arguments of point
locations and the resulting value to be displayed:
\begin{verbatim}
spatial.plot<- function(x,y){
 nc<-as.numeric(cut(y,20))
 plot(x,pch=" ")
 points(x,pch=20,col=topo.colors(20)[nc],cex=2)
 image.scale(y,col=topo.colors(20))
}
# To execute the function do this:
spatial.plot(cbind(xg,yg), Dn/nrow(z))
\end{verbatim}

\subsection{Example: Wolverine density map. }

The {\bf R} commands for producing density maps from MCMC output of
spatial capture-recapture models is provided in the {\bf R} function
\mbox{\tt SCRdensity} in the package \mbox{\tt scrbook}. 
We used the posterior output from the wolverine model fitted previous
to compute a relatively coarse version of a density map, using a $10 \times
10$ grid (Fig. \ref{scr0.fig.density10x10}) and using a $30 \times 30$
grid (Fig. \ref{scr0.fig.density20x20}). The {\bf R} commands for
producing such a plot (for short MCMC run) are as follows:
{\small
\begin{verbatim}
library("scrbook")
data(wolverine)
traps<-wolverine$wtraps
y3d <-SCR23darray.fn(wolverine$wcaps,wolverine$wtraps)
# this takes 341 seconds on a standard CPU circa 2011
unix.time(bln<-wolvSCR0.fn(y3d,traps,nb=1000,ni=2000,delta=1,M=100))
Sx<-bln$sims.list$s[,,1]
Sy<-bln$sims.list$s[,,2]
w<- bln$sims.list$w
obj<-list(Sx=Sx,Sy=Sy,w=w)
tmp<-SCRdensity(obj,scalein=100,scaleout=100)
\end{verbatim}
In these figures density is
expressed in units of individuals per $100$ $km^2$, while the area of
the pixels is about 103.7 $km^2$ and 11.5 $km^2$, respectively. That
calculation is based on:
\begin{verbatim}
> total.area<- (Yu-Yl)*(Xu-Xl)*100
> total.area/(10*10)
[1] 103.7427
> total.area/(30*30)
[1] 11.52697
\end{verbatim}

A couple of things are worth noting: First is that as we move away
from ``where the data live'' - away from the trap array - we see that
the density approaches the mean density. This is a property of the
estimator as long as the ``detection function'' decreases sufficiently
rapidly as a function of distance.
Relatedly, it is also a property of statistical smoothers
such as splines, kernel smoothers, and regression smoothers -
predictions tend toward the global mean as the influence of data
diminishes. Another way to think of it is that it is a consequence of
the prior - which imposes uniformity, and as you get far away from the
data, the predictions tend to the prior. The other thing to note about
this map is that density is not $0$ over water (although the coastline
is not shown). This might be perplexing
to some who are fairly certain that wolverines do not like
water. However, there is nothing about the model that recognizes water
from non-water and so the model predicts over water {\it as if} it
were habitat similar to that within which the array is nested. But,
all of this is ok as far as estimating density goes and, furthermore,
we can compute valid estimates of $N$ over any well-defined region which
presumably wouldn't include water if we so choose.

\begin{figure}
\begin{center}
\includegraphics[height=3in,width=3.375in]{Ch4/figs/density10x10}
\end{center}
\caption{Needs a caption}
\label{scr0.fig.density10x10}
\end{figure}

\begin{figure}
\begin{center}
\includegraphics[height=3in,width=3.375in]{Ch4/figs/density30x30}
\end{center}
\caption{Needs a caption}
\label{scr0.fig.density20x20}
\end{figure}

\section{Discrete State-Space}
\label{scr0.sec.discrete}

The SCR model developed previously in this chapter assumes that
individual activity centers are distributed uniformly over the
prescribed state-space. Clearly this will not always be a reasonable
assumption. In chapter \ref{chapt.state-space} we talk about developing models
that allow explicitly for non-uniformity of the activity centers by
modeling covariate effects on density. A simpler method of affecting
the distribution of activity centers, which we address here, is to
modify the shape of the state-space explicitly. For example, we might
be able to classify the state-space into distinct blocks of habitat
and non-habitat. In that case we can remove the non-habitat from the
state-space and assume uniformity of the activity centers over the
remaining portions judged to be suitable habitat.  There are two ways
to approach this: We can use a regular grid of points to represent the
state-space, i.e., by the set of coordinates ${\bf s}_1, \ldots, {\bf
  s}_{G}$, and assign a equal probabilities to each possible value, or
we can retain the continuous formulation of the state-space but use
basic polygon operations to induce constraints on the state-space We
focus here on the formulation of our basic SCR model in terms of a
discrete state-space but later on (chapter \ref{chapt.mcmc} and also
Appendix XYZ) we demonstrate the latter approach based on using
polygon operations to define an irregular state-space.

Use of a discrete state-space can be computationally expensive in {\bf
  WinBUGS}. That said, it isn't too difficult to do the MCMC
calculations in {\bf R} which we discuss briefly in chapter
\ref{chapt.mcmc}. The {\bf R} package {\tt SPACECAP}
\citep{gopalaswamy_etal:2011} arose from the {\bf R} implementation
developed for the application in \citet{royle_etal:2009}.  As we will
see in chapter \ref{chapt.mle}, we must prescribe the state-space by a
discrete mesh of points in order to do integrated likelihood and so if
we are using a discrete state-space this can be accommodated directly
in our code for obtaining MLEs.

While clipping out non-habitat seems like a good idea, its not obvious
that we accomplish any biologically reasonable objective by doing
so. We might prefer to do it when non-habitat represents a clear-cut
restriction on the state-space such as a reserve boundary or a lake,
ocean or river. It makes sense in those situations.  Unfortunately,
having the capability to do this also causes people to start defining
``habitat'' vs. ``non-habitat'' based on their understanding of the
system whereas it can't be known whether the animal being studied has
the same understanding. Moreover, differentiating of the landscape by
habitat or habitat quality probably affects the geometry and
morphology of home ranges much more than the plausible locations of
activity centers. That is, a home range centroid could, in actual
fact, occur in a walmart parking lot if there is pretty good habitat
around walmart, so there is probably no sense to cut out the walmart
lot and preclude it as the location for an activity center.  It would
generally be better to include some definition of habitat quality in
the model for the detection probability (see chapter XYZ).


\subsection{Evaluation of Coarseness of Discrete Approximation}

The coarseness of the state-space should not really have much of an
effect on estimates if the grain is sufficiently fine relative to
typical animal home range sizes.  Why is this?  We have two analogies
that can help us understand this. First is the relationship to Model
$M_{h}$.  As noted in section \ref{scr0.sec.scrmh} above, we can think
about SCR models as a type of finite mixture
\citep{norris_pollock:1996, pledger:2000} where we are fortunate to be
able to obtain direct information about which ``group'' individuals
belong to (group being location of activity center).  In the standard
finite mixture models we typically find that only 1 or a very small
number of groups (e.g., 2 or 3 at the most) can explain really high
levels of heterogeneity and are adequate for most data sets of small
to moderate sample sizes. We therefore expect a similar effect in SCR
models when we discretize the state-space.
We can also
think about discretizing the state-space as being related
to numerical integration where we find (see
chapter \ref{chapt.mle}) that we don't need a very fine
grid of support points to evaluate the integral to a reasonable
level of accuracy. We demonstrate this here by reanalyzing simulated
data using a state-space defined by a different numbers of support points.
We provide an R script called \mbox{\tt simSCR0discrete.fn} in the
{\bf R} package \mbox{\tt scrbook}.  We note that for this comparison
we generated the actual activity centers as a continuous random
variable and thus the discrete state-space is, strictly speaking, an
approximation to truth. That said, we regard all state-space
specifications as approximations to truth because they are all,
strictly speaking, models of some unknown truth. Thus the use of any
specific discrete state-space is not intrinsically more ``wrong'' than
any specific continuous representation.


We used {\bf JAGS} from the \mbox{\tt rjags} function to obtain the results
for $6 \times 6$, $9 \times 9$, $12 \times 12$, $15\times 15$,
$20\times 20$, $25 \times 25$ and $30 \times 30$ state-space grids.
We used 2000 burn, 12000 total iters with 3 chains, therefore a total
of 30000 posterior samples.
For {\bf WinBUGS} we used 3 chains of 5k total with 1k burnin means 12k
total posterior samples.
Summary results for these analyses are shown in
Table XYZ\footnote{Andy to finish later. }.

\begin{verbatim}
Table XYZ.
             Mean       SD    NaiveSE  Time-seriesSE  runtime
6    N     109.7717 15.98959 0.0923160    0.377737    1239
9    N     114.4621 16.72025 0.0965344    0.468659    1267
12   N     115.4309 17.12403 0.098866     0.464830    1576
15   N     114.7699 17.0242  0.0982894    0.425238    1638
20   N     116.0370 17.10686 0.0987665    0.486867    1647
25   N     116.3228 16.98323 0.0980527    0.465527    1661
30   N     116.4252 17.4078  0.100504     0.533735    1806
WinBUGS
             Mean       SD    NaiveSE  Time-seriesSE  runtime
6    N     111.67    16.61                             2274
9    N     114.23    17.99                             4300
12   N     115.98    17.38                             7100
15   N     115.38    17.94                            13010

Note: WinBUGS based on fewer samples too!

To get SE and time-series SE do this:
You can use as.mcmc.list() to convert to a coda object. Then use summary.�
\end{verbatim}

The results in terms of the posterior summaries are, as we
expect, very similar using {\bf WinBUGS}. However, it was interesting
to note that {\bf WinBUGS} runtime is much worse (note the number of
iterations is lower for {\bf WinBUGS} yet the runtime is much longer)
and, furthermore, it seems to scale with the size of the
discrete state-space grid. While that was expected, it was unexpected
that the runtime of {\bf JAGS} would seem relatively consistent
as we increase the grid size.
We suspect that {\bf WinBUGS} is evaluating the full-conditional for
each activity center at all $G$ possible values whereas it may be that
{\bf JAGS} is evaluating the full-conditional only at a subset of
values or perhaps using previous calculations more effectively.

While this might suggest that one should always use {\bf JAGS} for
this analysis, we found in our analysis of the wolverine (next
section) that {\bf JAGS} could be extremely sensitive to starting
values, producing MCMC algorithms that sometimes simply did not work.

\subsection{Analysis of the wolverine camera trapping data}

We reanalyzed the wolverine data using discrete state-space grids with points spaced by 2,
4 and 8 km (depicted in Fig. \ref{scr0.fig.wolvgrids}). These were
constructed from
the 40 km buffered state-space, and deleting the points over water \citep[see][]{royle_etal:2011jwm}.
 Our interest in doing this was
to evaluate the relative influence of grid resolution on estimated
density because the coarser grids will be more efficient from a
computational stand-point and so we would prefer to use them, but perhaps not
if there is a strong influence on estimated density.

{\bf Note}: Results from WinBUGS are given below -- these are updated
based on longer MCMC runs and replace prelim results as of Jan 1 2012
or so. 
To be done: density map.



\begin{figure}
\begin{center}
\includegraphics[height=2.5in,width=5in]{Ch4/figs/wolvgrids}
\end{center}
\caption{2 km 4 km and 8km wolverine state-space grids extending about
40 km from the vicinity of the trap array. }
\label{scr0.fig.wolvgrids}
\end{figure}

{\small
\begin{verbatim}
This took about 6 days in WinBUGS. Terrible mixing for the 2km and
8km. Why is this? We may never know!

> print(out.2km,digits=2)
Inference for Bugs model at "modelfile.txt", fit using WinBUGS,
 3 chains, each with 11000 iterations (first 1000 discarded)
 n.sims = 30000 iterations saved
       mean    sd  2.5%   25%   50%   75%  97.5% Rhat n.eff
psi    0.43  0.09  0.27  0.37  0.43  0.49   0.63 1.00   560
sigma  0.62  0.05  0.54  0.59  0.62  0.65   0.73 1.01   160
lam0   0.05  0.01  0.04  0.04  0.05  0.06   0.07 1.01   320
p0     0.05  0.01  0.03  0.04  0.05  0.05   0.06 1.01   320
N     86.56 16.94 57.00 75.00 85.00 97.00 124.00 1.00   510
D      8.78  1.72  5.78  7.60  8.62  9.83  12.57 1.00   510

For each parameter, n.eff is a crude measure of effective sample size,
and Rhat is the potential scale reduction factor (at convergence, Rhat=1).
> print(out.4km,digits=2)
Inference for Bugs model at "modelfile.txt", fit using WinBUGS,
 3 chains, each with 11000 iterations (first 1000 discarded)
 n.sims = 30000 iterations saved
       mean    sd  2.5%   25%   50%    75%  97.5% Rhat n.eff
psi    0.45  0.09  0.28  0.38  0.44   0.50   0.64    1  1300
sigma  0.61  0.04  0.53  0.58  0.61   0.64   0.71    1  1600
lam0   0.05  0.01  0.04  0.05  0.05   0.06   0.07    1  2500
p0     0.05  0.01  0.03  0.04  0.05   0.05   0.07    1  2500
N     89.25 17.44 59.00 77.00 88.00 100.00 127.00    1  1100
D      9.01  1.76  5.96  7.77  8.88  10.10  12.82    1  1100

For each parameter, n.eff is a crude measure of effective sample size,
and Rhat is the potential scale reduction factor (at convergence, Rhat=1).
> print(out.8km,digits=2)
Inference for Bugs model at "modelfile.txt", fit using WinBUGS,
 3 chains, each with 11000 iterations (first 1000 discarded)
 n.sims = 30000 iterations saved
       mean    sd  2.5%   25%   50%   75%  97.5% Rhat n.eff
psi    0.42  0.09  0.26  0.36  0.41  0.47   0.61 1.00   940
sigma  0.68  0.05  0.59  0.64  0.67  0.71   0.77 1.01   220
lam0   0.05  0.01  0.03  0.04  0.05  0.05   0.06 1.00   560
p0     0.05  0.01  0.03  0.04  0.04  0.05   0.06 1.00   560
N     83.18 16.14 56.00 72.00 82.00 93.00 119.00 1.00   700
D      8.28  1.61  5.57  7.17  8.16  9.26  11.84 1.00   700

For each parameter, n.eff is a crude measure of effective sample size,
and Rhat is the potential scale reduction factor (at convergence, Rhat=1).
\end{verbatim}
}

The density is a bit different depending on the grid size. Also the
effectiveness of the MCMC algorithsm is pretty remarkably different. 
We did the analysis in JAGS also. The results are shown below. {\bf Note}: I
am going to run these again but for longer to finalize the results.

{\small
\begin{verbatim}
 ### 01/10/2012 -- need to rerun these JAGS runs but use more
iterations and check results.


2km
Iterations = 7001:13000
Thinning interval = 1
Number of chains = 3
Sample size per chain = 6000

          Mean        SD  Naive SE Time-series SE
N     86.28522 16.950626 1.263e-01      0.4878973
lam0   0.04807  0.007512 5.599e-05      0.0002199
p0     0.04581  0.006820 5.083e-05      0.0001996
psi    0.28904  0.062117 4.630e-04      0.0017481
sigma  0.62769  0.043596 3.249e-04      0.0018724

4km
          Mean        SD  Naive SE Time-series SE
N     85.53139 16.998966 1.267e-01      0.5181297
lam0   0.04636  0.007542 5.621e-05      0.0002382
p0     0.04425  0.006867 5.118e-05      0.0002172
psi    0.28650  0.061922 4.615e-04      0.0018276
sigma  0.64281  0.048321 3.602e-04      0.0022911

8km
          Mean        SD  Naive SE Time-series SE
N     83.97039 16.508146 1.230e-01      0.4548782
lam0   0.04519  0.006919 5.157e-05      0.0001738
p0     0.04319  0.006319 4.710e-05      0.0001589
psi    0.28146  0.060653 4.521e-04      0.0016555
sigma  0.66956  0.040989 3.055e-04      0.0015070
\end{verbatim}
}

\subsection{SCR models as multi-state models}

While we invoke a discrete state-space artificially, by gridding the
underlying continuous state-space, sometimes the state-space is more
naturally discrete. Consider a situation in which discrete patches of
habitat are searched using some method and it might be convenient (or
occur inadvertently) to associate samples to the patch level instead
of recording observation locations. In this case we might use a model
${\bf s}_{i} \sim dcat(probs[])$  where $probs[]$ are the probabilities that
an individual inhabits a particular patch. We consider such a case
study in chapter XXPoissonXXX from \citet{mollet_etal:2012} who
obtained a population size estimate of a large grouse species known as
the capracaillie. Forest patches were searched for scat which was
identified to individual by DNA analysis.
Even when space is {\it not}
naturally discrete, measurements are often made at a fairly coarse
grain (e.g., meters or tens of meters along a stream), or associated
with spatial quadrats for scat searches and therefore the state-space
may be effectively discrete in many situations.

This discrete formulation of SCR models suggests that SCR models are
related to ordinary multi-state models \citep[][ch. 9]{kery_schaub:2011}
which are also parameterized in terms of a discrete state
variable which is often defined as a spatially-indexed state related
either to location of capture or breeding location. While many
multi-state models exist in which the state variable is not related to
space, multi-state models have been extremely useful in development
models of movements among geographic states and indeed this type of
problem motivated their early developments by \citet{arnason:1972,
  arnason:1973} and \citet{hestbeck_etal:1991}.  We pursue this
connection a little bit more in chapter XXX XYZ.




\section{ Summary and Outlook }

A point we tried to emphasize in this chapter is that the basic SCR
model is not much more than an ordinary capture-recapture model for
closed populations -- it is simply that model but augmented with a set
of ``individual effects'', ${\bf s}_{i}$, which relate encounter
probability to some sense of individual location. SCR models are
therefore a type of individual covariate model (as introduced in
chapter \ref{chapt.closed} -- but with imperfect information about the
individual covariate. In other words, they are GLMM type models when
$N$ is known or, when $N$ is unknown, they are zero-inflated GLMMs
(see \citet{royle:2006}).  Another class of capture-recapture models
that SCR models are closely related to is so-called ``Model $M_{h}$.''
The effect of introducing a spatial location for individuals is that
it induces heterogeneity in detection probability, as in Model
$M_{h}$. However, unlike Model $M_{h}$, we obtain some information
about the individual effect which is completely latent in Model
$M_{h}$. If the state-space of the random effect ${\bf s}$ is discrete
then the SCR model resembles more closely the finite-mixture class of
heterogeneity models \citep{norris_pollock:1996} which parameterizes
heterogeneity by assuming that individuals belong to discrete classes
or groups (e.g., high, medium, low). In the context of SCR models we
obtain some information about the ``group membership'' in the
locations where individuals are captured.  Given the direct
relationship of SCR models with so many standard classes of models, we
find that they are really quite easy to analyze using standard MCMC
methods encased in black boxes such as {\bf WinBUGS} or {\bf JAGS} and
possibly other packages. They are also easy to analyze using classical
likelihood methods, which we address in chapter \ref{chapt.mle}.

Formal consideration of the collection of individual locations $({\bf
  s}_{1}, \ldots, {\bf s}_{N})$ in the model is fundamental to all of
the models considered in this book. In statistical terminology, we
think of the collection of points $\{ {\bf s}_{i} \}$ as a realization of a
point process and part of the promise, and ongoing challenge, of SCR
models is to develop models that reflect interesting biological
processes, for example interactions among points or temporal dynamics
in point locations.  Here we considered the simplest possible point
process model - the points are independent and uniformly
(``randomly'') distributed over space. Despite the simplicity of this
assumption, it should suffice in many applications of SCR models
although we do address generalizations of this model in later
chapters. Moreover, even though the {\it prior} distribution on the
point locations is uniform, the realized pattern may deviate markedly
from uniformity as the observed encounter data provide information to
impart deviations from uniformity. Thus, the estimated density map
will typically appear distinctly non-uniform.  As a general rule,
information in the data will govern estimates of individual point
locations so even fairly complex patterns of non-independence or
non-uniformity will appear in the data. That is, we find in
applications of the basic SCR model that this simple {\it a priori}
model can effectively reflect or adapt to complex realizations of the
underlying point process.  For example, if individuals are highly
territorial then the data should indicate this in the form of
individuals not being encountered in the same trap - the resulting
posterior distribution of point locations should therefore reflect
non-independence.  Obviously the complexity of posterior estimates of
the point pattern will depend on the quantity of data, both number of
individuals and captures per individual.  Because the point process is
such an integral component of SCR models, the state-space of the point
process plays an important role in developing SCR models. As we tried
to emphasize in this chapter, the choice of the stat-espace is part of
the model. It can have an influence on parameter estimates and other
inferences such as model selection (see chapter \ref{chapt.gof}). We
emphasize however that this is not an arbitrary decision like
``buffering'' because the model induces an explicit interpretation of
parameters and statistical effect on estimators.

We showed how to conduct inference about the underlying point process
including calculation of density maps from posterior output. We can do
other things we normally do with spatial point processes such as
compute ``K-functions'' and test for ``complete spatial randomness''
(CSR) which we develop in chapter \ref{chapt.gof}.  Modifying and
applying point process methods to SCR problems seems to us to be a
fruitful area of research.

An obvious question that might be floating around in your mind is why
should we ever go through all of this trouble when we could just use
{\bf MARK} or {\bf CAPTURE} to get an estimate of $N$ and apply $1/2$
MMDM methods?  The main reason is that these conventional methods are
predicated on models that represent explicit misspecifications of both
the observation and ecological process - they are wrong!  Not just
wrong, because of course all models are wrong, but they're not even
{\it plausible} models! Thus while we might be able to show adequate
fit or whatever, we think as a conceptual and philosophical model one
should not be using models that are not even plausible data-generating
models -- even if the plausible ones don't fit!  Perhaps more
charitably, these ordinary non-spatial models are models of the wrong
system. They do not account for trap identity. They don't account for
spatial organization or ``clustering'' of individual encounters in
space. And, ``density'' is not a parameter of those models because
density has no meaning absent an explicit representation of space. If
we do define space explicitly, e.g., as a buffered minimum convex
hull, then the normal models ($M_{0}$, $M_{h}$, etc..) assume that
individual capture-probability is not related to space, no matter how
we define the buffer.  Conversely, the SCR model is a model for
trap-specific encounter data - how individuals are organized in space
and interact with traps. SCR models provide a coherent framework for
inference about density or population size and also, because of the
formality of their derivation, can be extended and generalized to a
large variety of different situations, as we demonstrate in subsequent
chapters.

In the next few chapters we continue to work with this basic SCR
design and model but consider some important extensions of the basic
model.  For example, we consider
extensions
to  include covariates that vary by individual, trap, or over time
(chapter \ref{chapt.covariates}), spatial covariates on density
(chapter \ref{chapt.state-space}),
 open populations (chapter \ref{chapt.open}), model assessment and
 selection (chapter \ref{chapt.gof}) and other topics.
We also consider technical details of Bayesian (chapter
\ref{chapt.mcmc}) and  maximum
likelihood (chapter \ref{chapt.mle}) estimation so that the interested
reader can develop or extend their own methods to suit their needs.


\chapter{Other observation models}
\label{chapt.poisson}

%\chapter{Alternative Models for the Encounter Process}
\label{chapt.poisson-mn}

In the previous chapter we considered a very specific although not
terribly limited observation model. The observation model consisted of
two main elements: First a description of the encounter process 
by which individuals are detected in traps. Specifically, we 
assumed individual trap-specific encounters were iid Bernoulli
trials. The consequence of this is that individuals function
independently of one another and can be captured in
any number of traps during a specific interval of trapping
effort. The type of device is typical of bear hair snares, which we
considered as an example in that section. The 2nd element of the
encounter process model was the specific model – functional form –
relating encounter probability to individual activity center
(``detection probability model'').  It is natural to consider
alternative functional forms of this detection probability model which
we do in Chapt. \ref{chapt.covariates} and elsewhere. 

In this chapter we consider alternative observation models which
accommodate Poisson or multinomial observation models. For example, if
sampling devices can detect an individual some arbitrary number of
times during an interval, then it is natural to consider observation
models for encounter frequencies, such as the Poisson model. Another
type of encounter device is the ``multi-catch'' device (REF XYZ) which
is a physical device that can capture and hold an arbitrary number of
individuals. A typical example is a mist-net for birds 
\citep{borchers_efford:2008}.

We talk about how SCR are multi-state kinds of models. 

We talk about single catch traps. 


\section{Poisson Observation Model}

The models we analyze in Chapt. \ref{chapt.scr0} assumed binary
observations -- i.e., standard encounter history data -- so
that individuals are captured at most one time in a trap.  This makes
sense for many types of DNA sampling (e.g., based on hair snares)
because distinct visits to sampled locations or devices cannot be
differentiated. However, many encounter methods or devices make it
possible to encounter an individual some arbitrary number of times
during any particular sampling episode. That is, we might observe
encounter frequencies $y_{ijk}>0$ for individual $i$, trap $j$ and
sampling interval $k$.  As an example, if a camera device is
functioning properly it may be programmed to take photos every few
seconds if triggered.  For a second example, suppose we are searching
a quadrat for scat, we may find multiple samples from the same
individual.

Therefore, we seek observation models that accommodate such encounter
frequency data.  Let $y_{ijk}$ be the frequency of encounter for
individual $i$, in trap $j$, during occasion $k$, then a plausible
model is:
\[
 y_{ijk} \sim \mbox{Poisson}(\lambda_{ij})
\]
where the expected encounter frequency $\lambda_{ij}$ depends on both
individual and trap. As we did in the binary model of chapter 4, we
now seek to model the expected value of the observation (which was
$p_{ij}$ in chapter 4) as a function of the individual activity center
${\bf s}_{i}$.
We propose 
\[
 \lambda_{ij} = \lambda_{0}  g({\bf x}_{j},{\bf s}_{i})
\]
Where $g({\bf x},{\bf s})$ is some positive valued function. 
Then $\lambda_{0}g({\bf x},{\bf s})$ is the encounter rate in trap
${\bf x}$ for an individual having activity center ${\bf s}$.  

What does this mean? This means that the encounter rate looks like a
bivariate normal distribution.  If we might interpret encounters as
resulting from the outcome of a movement model in the following
sense. Suppose that we telemeter an individual and take measurements
of location sufficiently far apart in time that locations are
independent. Let $x_{t}$ be the location at time $t$. Take a large
number of samples, make a grid and count up the number of observations
in each grid cell.
\[
 E[y(x)] = E[y(x)| moves to x]\Pr(moves to x|s) = \lambda_{0} g(x|s)
\]


For the simplest model in which we have covariates that vary across
the replicate samples $k$, we can aggregate the observed data by the
propery of compound additivity of the Poisson distribution (if $x$ and
$y$ are $iid$ Poisson with mean $\lambda$ then $x+y$ is Poisson with
mean $2\lambda$). Therefore,
\[
y_{ij} = (\sum_{k=1}^{K} y_{ijk}) =  \mbox{Poisson}(K  \lambda_{0} 
g({\bf x}_{j},{\bf s}_{i}) )
\]
We see that $K$ and $\lambda_{0}$ serve the same role as affecting the
base encounter rate. Since the observation model is the same,
probabilistically speaking, for all values of $K$, evidently we need
only $K=1$ ``survey'' from which to estimate model parameters. We know
this intuitively as sampling by multiple traps serves as replication
in SCR models.


\subsection{Poisson relationship to the Bernoulli model}

There is a sense in which the Poisson and Bernoulli models can
be viewed as consistent with one another. Note that under the Poisson
model we have:
\begin{equation}
 \Pr(y>0) = 1-exp(-\lambda_{0} g({\bf x},{\bf s}))
\label{eq.cloglog}
\end{equation}
Therefore, 
if we equate the event ``encountered'' with the event that the
individual was captured at least 1 time under the Poisson model, i.e., $y>0$, then it would be
natural to set $p_{ij} = \Pr(y>0)$ according to \ref{eq.cloglog}. 

In fact, as $\lambda_0$ gets small, the Poisson model is a close approximation
to the Bernoulli model in the sense that $y$ in that case is almost
always 0 or 1 and, in fact, $\Pr(y>0) \rightarrow \lambda$.  This is
convenient in some cases because the Poisson model might be more
tractable to fit (or even vice versa). For an example, see the models
described in Chapt. \ref{chapt.scr-unmarked}, and we also consider
another case in sec. \ref{XYZ} below.
A plot of that is in order. This near equivalence is shown in  Figure
XYZ. The left panel shows a plot of $\lambda_{ij}$ vs. distance and
superimposed on that is a plot of $p_{ij}$ vs. distance, for values
$\lambda_{0} = .1$ and $\sigma = 1$. The right panel shows a plot of
$\Pr(y>0)$ vs. $E[y]$ and we see therefore that the models are
practically equivalent. 

\begin{verbatim}
x<-seq(0.001,5,,200)
lam0<- .1
sigma<- 1
lam<- lam0*exp(-x*x/(2*sigma*sigma))

par(mfrow=c(1,2))
p1<- 1-exp(-lam)
plot(x,lam,ylab="E[y] or Pr(y>0)",xlab="distance",type="l",lwd=2)
lines(x,p1,lwd=2,col="red")
plot(lam,p1,xlab="E[y]",ylab="Pr(y>0)",type="l",lwd=2)
abline(0,1,col="red")
\end{verbatim}

So under the Poisson model we have
\[
\Pr(y>0) \approx E[y] = \lambda_{0} g(x,s)
\]
whereas in the binary model from chapter 4 we had precisely
\[
\Pr(y>0) \equiv E[y] = p_{0} g(x,s)
\]
and so the models are exactly the same for the {\it expected values}
and very similar for the probability of observing a positive response,
as long as $\lambda_{0}$ is small.


What all of this suggests it that
if we see very few observations $>1$ then we wont lose much
information by using the Bernoulli model. On the other hand, the
Poisson model is more easy to compute with in some cases. 


\begin{figure}
\centering
\includegraphics[width=5in,height=2.5in]{Ch5/figs/Poisson-Bern.png}
\label{fig:elevMap}
\end{figure}



Even if we're not in the range where the Bernoulli model provides a
good approximation, we might choose to truncate the counts to binary
observations anyhow (``quantize'').
We might do
this intentionally, but sometimes this truncation is a feature of the
sampling. For example, in the case of bear hair snares, the number of
encounters might be well approximated by a Poisson distribution but we
cannot determine unique visits and so only get to observe the binary
event ``$y>0$''. Similarly for scat sampling problems it will not
generally be possible to diagnose distinct ``independent'' scat
samples. Under this model the data are only binary encounters and we
might therefore choose a model of the form:
\[
 cloglog(p_{ij}) = log(\lambda0)  + log(g({\bf x},{\bf s}))
\]
\begin{comment} 
This example shows us that the choice of link function is typically
directly related to a specific encounter frequency model and,
furthermore, the choice of link function is equivalent to choice of
``detection function.''  As another example, what if the latent
encounter frequencies are actually geometric random variables where
the mean is a function of distance? For the case where the support of
y includes 0 – so that $y$ is the number of failures before the 1st
success, then the mean is $\mu = (1-p)/p$.  $Pr(y>0) =$ ??
\[
logit() = ….?
\]
\end{comment}

\subsection{A cautionary note on modeling encounter frequencies}

Other models for counts might be appropriate. For example, ecologists
are especially fond of negative binomial models for count data
\citep{verhoef_boveng:2007,
white_bennetts:1996,kery_etal:2005}
but other models for excess-Poisson variation are possible. For
example, we might add a normally distributed random effect to
the linear predictor.

As a general rule we favor the Bernoulli observation model even if
encounter frequencies are obtained by sampling.  The main reason is
that, with frequency data, we are forced to confront a model choice
problem (i.e., Poisson, negative binomial, log-normal mixture) that is
wholly unrelated to the fundamental space usage process that underlies
the genesis of SCR data. Repeated encounters over short time intervals
are not likely to be the result of independent encounter
processes. E.g., an individual moving back and forth in front of a
camera yields a cluster of observations that is not informative about
the spatial structure of the model. Similarly in scat surveys (e.g.,
Thompson et al. in review), dogs are used to locate scats which are
processed in the lab for individuality.  The process of local scat
deposition is not really the outcome of movement but rather the
outcome of complex behavioral considerations as well as dependence in
detection of scat by dogs. E.g., they find one and then more likely to
find a nearby one, or they get into a den area and find lots of scats.
This additional model assumption required to model variation in
observed frequencies (i.e., conditional on location) provides
relatively little information about density, and we feel that the
model selection issue should therefore be avoided.

To elaborate on this, it seems natural to construct models for
encounter data that is conditional on movement outcomes: We suppose
that an individual visits a particular location with some probability
$p_{ik}$ say $z_{ik}\sim  \mbox{Bern}(p_{ik})$ and then deposits a number of scat,
or visits a camera some number of times with frequency $y_{ik}$ which
is 
an integer $> 0$. Therefore, a sensible model might be
$[y|z][z|\phi({\bf x},{\bf s})$
where the encounter frequency $y$ is independent of ${\bf x}$ and
${\bf s}$ conditional
on the binary event ``$z$'' that the individual visited the vicinity of
the trap.

Moreover, consideration of encounter frequency data could lead to
important identifiability problems along the lines of Link (2003). The
basic Poisson model can be over-dispersed in a number of ways to
produce different models of over-dispersion.  i.e., gamma noise,
normal noise, exponential noise, etc..  Thus we have different models
of heterogeneity analogous to the class of models considered by \citet{link:2003}.


\section{Analysis of a Poisson SCR model in BUGS}

We consider the simplest possible model here in which we have no
covariates that vary over replicate samples $k$ so that we work with
the aggregated individual- and trap-specific encounters:
\[
y_{ij} = (\sum_{k=1}^{K} y_{ijk}) =  \mbox{Poisson}(K  \lambda_{ij})
\]
We consider a bivariate normal form of $g({\bf x}_{j},{\bf s}_{i})$ so
that
\[
g({\bf x}_{j},{\bf s}_{i}) = exp( -||{\bf x}_{j} - {\bf
  s}_{i}||^{2} /(2\sigma^{2}))
\]
In this case, note that 
\[
log( \lambda_{ij})  =\alpha_{0} - \beta ||{\bf x}_{j} - {\bf s}_{i}||^2
\]
where $\alpha_{0} = log(\lambda_{0})$ and $\beta = 1/(2\sigma^2)$.


As usual, we approach Bayesian analysis of these
models using data augmentation (section \ref{closed.sec.da}). 
It is interesting in this case that DA
gives us a sort of zero-inflated Poisson model which is amazingly easy
to analyze by likelihood methods which maybe we will do in Chapter
XYZ.

So the model specified conditional on $z_{i}$ is
\[
y_{ij} \sim  Poisson(z_{i} K  \lambda_{ij})
\]
which evaluates to a point mass at $y=0$ if $z=0$. 


\subsection{Simulating Data}

Simulating a sample SCR data set under the Poisson model requires only
a couple minor modifications to the procedure we used in chapter 4. In
particular, we modify the block of code which defines the model to be
that of $E[y]$ and not $\Pr(y=1)$, and we change the random variable
generator from \mbox{\tt rbinom} to \mbox{\tt rpois}:
\begin{verbatim}
D<- e2dist(S,traplocs)

alpha0<- -2.5
sigma<- 0.5
beta<- 1/(2*sigma*sigma)

muy <-  exp(alpha0)*exp(-beta*D*D)
# now generate the encounters of every individual in every trap
Y<-matrix(NA,nrow=N,ncol=ntraps)
for(i in 1:nrow(Y)){
 Y[i,]<-rpois(ntraps,K*muy[i,])
}
\end{verbatim}

We modified our code from SCR0 in chapter 4 to simulate Poisson
encounter frequencies for each trap and then we analyze an ideal data
set using WinBUGS. The new function, available in the R package, is called
{\tt simPoissonSCR.fn}. 
The simulator can produce 3-d encounter history data too although we
don't do that here. 
Here is an example of simulating a data set and harvesting the
required data objects:

\begin{verbatim}
data<-simPoissonSCR.fn(discard0=TRUE,sd=2013)
y<-data$Y
traplocs<-data$traplocs
nind<-nrow(y)
X<-data$traplocs
K<-data$K
J<-nrow(X)
Xl<-data$xlim[1]
Yl<-data$ylim[1]
Xu<-data$xlim[2]
Yu<-data$ylim[2]

## Data augmentation stuff
M<-200
y<-rbind(y,matrix(0,nrow=M-nind,ncol=ncol(y)))
z<-c(rep(1,nind),rep(0,M-nind))
\end{verbatim}

To execute WinBUGS the process is identical to what we've done
previously..............................................
here..................
.................................

The results are given below. We note about the same answer as before.

{\small
\begin{verbatim}
> print(out1,digits=2)
Inference for Bugs model at "SCR-Poisson.txt", fit using WinBUGS,
 3 chains, each with 2000 iterations (first 1000 discarded)
 n.sims = 3000 iterations saved
           mean    sd   2.5%    25%    50%    75%  97.5% Rhat n.eff
alpha0    -2.57  0.19  -2.95  -2.69  -2.57  -2.44  -2.19 1.00  2600
beta       2.34  0.36   1.69   2.08   2.32   2.57   3.12 1.00  3000
N        114.13 15.25  87.97 103.00 113.00 124.00 147.00 1.01   370
D          1.78  0.24   1.37   1.61   1.77   1.94   2.30 1.01   370
deviance 329.95 21.92 290.00 314.20 329.50 344.40 375.80 1.00  1700

For each parameter, n.eff is a crude measure of effective sample size,
and Rhat is the potential scale reduction factor (at convergence, Rhat=1).

DIC info (using the rule, pD = var(deviance)/2)
pD = 240.2 and DIC = 570.2
DIC is an estimate of expected predictive error (lower deviance is better).
\end{verbatim}


At the end of this chaptter we provide an example of a Poisson SCR model fitted to 
real data. This example has some other features which we encounter before
arriving there. 

\subsection{Exercise}

Use the Bernoulli model simulator from Chapt. \ref{chapt.scr0} (\mbox{\tt
  simSCR0.fn}) to simulate a Bernoulli data set and then fit the
Poisson model. Compare the results of fitting the correct
data-generating model with those of fitting the misspecified Poisson
model. 



\begin{comment}
\section{Likelihood analysis of the Poisson model}

Counts are Poisson with a random effect so this is stupidly easy to
implement. 
We do the normal ``full likelihood'' approach in which we retain $N$
as a real parameter in the model. We adapt \mbox{\tt intlik3} from
chapter 5 here..... behold:

Poisson(lambda(s,x))

data augmentation = ZIP
\begin{verbatim}
Pr(yi) =   ( prod_{j} dpois(y) ) *psi + I(y=0)*(1-psi)

Actually if y(i,j) = Poisson( lambda(i,j) ) then we can just add up
sum_{j} y(i,j) =  Poisson( sum_{j} lambda(i,j)) right?

 int_{s} thatthing

Zero-inflate the result
\end{verbatim}
\end{comment}

\section{No real example}

In chapt. \ref{chapt.searchencounter} we analyze the cap crap data.


\section{Independent Multinomial Observations}

Several types of encounter devices yield multinomial observations in
which an individual can be caught in a single trap during a particular
encounter occasion.  Mist nettting is a major example -- these are
``multi-catch'' traps (Efford XYZ NEED REF HERE XXXX). Also some kinds of
mammal traps hold multiples of animals and can be thought of
similarly. Another one is area-searches of reptiles where we think of
a small polygon as the ``trap'' -- we could get multiple individuals
(turtles, lizards) in the same plot but not, in the same sample
session, at different plots.  The key feature is that capture of an
individual in a trap is {\it not} independent of capture in other
traps, because they can't be captured once they are captured. On the
other hand individuals behave independently of one another, or so it
might be reasonable to assume, so whether a trap captures some other
individual doesn't have bearning on whether it captures another.  This
last assumption is violated in an extreme case in classical ``single
catch'' traps which we address in section \ref{poisson-mn.sec.singlecatch}
below. In general we could imagine non-independence being important in
any multi-catch situation but to the best of our knowledge a general
model that encompasses complete dependence (single catch) and complete
independence (multi-catch) of individuals has not been proposed.  So
we treat the cases individually and, in this section , we address the
multi-catch situation wherein individuals behave independently.


In this case we regard the observation ${\bf y}_{ik}$ for
individual $i$ during sample occasion $k$ as a multinomial observation
which consists of a sequence of 0's and at most a single 1 indiciating
the trap of capture. For example, if we capture an individual in trap
2 during a 6 sample period study then ${\bf y}_{i} = (0,1,0,0,0,0)$.
If we sample for 5 periods in all and the individual is also caught
in trap 4 during sample 3, then the 5 encounter observations for that
individual are as follows:
\begin{verbatim}
sample |---- trap ---------|
       1   2   3   4   5   6
 1     0   1   0   0   0   0
 2     0   0   0   0   0   0
 3     0   0   0   1   0   0
 4     0   0   0   0   0   0
 5     0   0   0   0   0   0
\end{verbatim}
Statistically we regard the {\it rows} of this data matrix as {\it
  independent} multinomial trials.

Analogous to our previous Bernoulli and Poisson models, we seek to
construct the multinomial cell probabilities for each individual, as a
function of {\it where} that individual lives, through its center of
activity ${\bf s}$. Thus we suppose that
\[
 {\bf y}_{ik} \sim \mbox{Multinom}(1, {\bm \pi}({\bf s}_{i}) )
\]
where ${\bm \pi}({\bf s}_{i})$ is a vector of length $J+1$, which, by
convention here, we define $\pi_{i,J+1}$, the last cell, or the ``zero
cell'', to correspond to the event ``not captured''.  Now we have to
construct these cell probabilities in some meaningful way that depends
on each individuals' ${\bf s}$, which we do shortly.

A statistically equivalent distribution is the {\it categorical} distribution.
If ${\bf y}$ is a multinomial trial with probabilities
${\bm \pi}$ than the {\it position} of the non-zero
elemment of ${\bf y}$ is a categorical random variable with probabilities
${\bm \pi}$.
We express this as
\[
{\bf y} \sim \mbox{Categorical}( {\bm \pi} )
\]
In the context of SCR models the categorical version of the
multinomial trial corresponds to the {\it trap of capture}.  Using our
example above with 6 traps then ${\bf y}_{i1} = (0,1,0,0,0,0)$ then we
could as well say $y_{ik}$ is a categorical random variable with
possible outcomes $(1,2,3,4,5,6,7)$ where outcome $y=7$ corresponds to
``not captured'' (obviously how this is organized or labeled is
completely irrelevant, although it is convenient to use the integers
$1:(J+1)$).  Therefore, $y_{i1} = 2$, $y_{i2} = 7$, $y_{i3} = 4$ and
so on.

For simulating and fitting data in the {\bf BUGS} engines we will typically use
the categorical representation of the model because it is somewhat
more convenient.  We have found that fitting multinomial models in
{\bf WinBUGS} can be extremely inefficient whereas {\bf JAGS}
typically performs much better. In the examples here, we use {\bf
  JAGS} exclusively.

\subsection{Multinomial Relationship to Poisson}

The multinomial is related directly to the Poisson encounter rate
model in the following sense. Let $y_{ij}$ be the total number of
encounters for individual $i$ in trap $j$. Then, the trap frequencies
(expluding the last cell now), if we condition on the {\it total}
number of captures, $y_{i} = \sum_{j} y_{ij}$, are multinomial with
probabilities
\[
 \pi_{ij} =  \frac{ \lambda_{ij} } { \sum_{j} \lambda_{ij} } 
\]
for $j=1,2,\ldots,J$.
Or equivalently the {\it trap of
  capture} is categorical with probabilities
\[
 \pi_{ij} =  \frac{ \lambda_{ij} } { \sum_{j} \lambda_{ij} } 
\]
which is precsely, under the half normal model, 
\[
 \pi_{ij} =  \frac{ \exp( - \beta \! dist({\bf x},{\bf s})^2 ) }  {
   \sum_{j} \exp(-\beta \! dist({\bf x},{\bf s})^2)}
\]
This expression looks like a multinomial inverse-logit transform of a model having
quadratic distance term, and also ``maximum entropy'' from MAXENT
species distribution modeling, or resource utilzation distribution
from telemetry studies.
So we can think of this multinomial model as arising naturally 
by having Poiosson encounters and then conditioning on the total. 
It is a sensible model to have anyhow, as it just allocates captures
to traps in proportion to the square of distance.  We could try other
models here too (Note: What do Borchers and Efford 2008 do?).

%People might think this multinomioal model is somehow more general
%than assuming Poisson encounter frequencies since we might cook up the
%multinomioal without having to specify a distribution for
%$y_{i}$. That said, we note that it arises under 
%If we now uncondition on the total..... 
%$y_{ij}$ is Poisson with mean $\sum_{j}$ stuff... we have a product of
%Poissons, i.e., the model we started with. 

The interpretation of this model merits some discussion. That is, 
{\it given that an individual is captured}, the probabilities given by
eq. XYZ determine 
the distribution among traps. To fully specify the model, we have to
model the probability that an individual is captured, say $p$.

We deduced the multinomioal by assuming a Poisson distribution
..... so
where did this $p$ come from?

So lets not worry about the distribution of the total count
but instead estimated this excess parameter p (this is what Royle et al.
and Gardner et al. tried to do).  In this case the multinomial gets
another cell probability , the J+1 cell, 
\[
 \pi\_{ij} =  \frac{ p exp( - beta d^2 ) }  { \sum\_{j} exp(-beta d^2)}
\]
and the last cell
\[
 \pi\_{i,J+1} =  1-p 
\]

What i like about this particular multiomial model is that whether or not
an individual is encounter in trap $j$ is just a Bernoulli trial with
probability
\[ 
(p/stuff)*exp(-beta*d^2)
\]
and if we just label (p/stuff) = p0 then this is precisely our
Bernoulli model with a half-normal detection model.  Thus we ``condition
on $y_{ij}$ and we dont have to fess up to a model for this encounter
rate, which is most of the time just reflecting behavioral stuff of the
species under, study and we wind up with a basic default Bernoulli model
which doesn't require any assumptions about the encounter rate of 
individuals.  So not having to model encounter rate seems like a good
benefit of the Bernoulli model -- which is why we said what we did above.


\subsection{Simulating data and fitting in WinBUGS}

We're going to show the nugget of a simulation function which is
used in the function \mbox{\tt sim.mnSCR} found in the {\bf R} package
\mbox{\tt scrbook}.  The first lines of the following {\bf R} code
make use of some things that should be defined but we omit them here:
{\small
\begin{verbatim}
S<-cbind(runif(N,Xl,Xu),runif(N,Yl,Yu))
# how far is each individual from each trap?
D<- e2dist(S,traplocs)

# paramter values
sigma<- 0.5
alpha0<- -1
theta<- 1/(2*sigma*sigma)

# make an empty data matrix and fill it up
Ycat<-matrix(NA,nrow=N,ncol=K)
for(i in 1:N){
for(k in 1:K){
lp<- alpha0 - theta*D[i,]*D[i,]
cp<- exp(c(lp,0))
cp<- cp/sum(cp)
Ycat[i,k]<- sample(1:(ntraps+1),1,prob=cp)
}
}
\end{verbatim}
}
The resulting data matrix in this case has the maximal dimension $N$
and so, for analysis, to mimic a real situation, we would have to discard the uncaptured
individuals. 
\mbox{\tt sim.mnSCR} will also simulate data that includes a
behavioral response, which will be the typical situation in
small-mammal trapping problems, which we first developed this code to
deal with \citep[see][for details]{converse_royle:2012}.

Here we use our function \mbox{\tt sim.mnSCR} to simulate a data set
with $K=7$ periods, etc.. We'll run the model using {\bf JAGS} which we
have found is much more effective for this class of models.
We get the data set-up for analysis by augmenting the size of the data
set to $M=200$. In addition we choose starting values for ${\bf s}$ and the
data augmentation variables $z$.  For ${\bf s}$ here we cheat a little bit
and use the true values for the obseved individuals and then augment
the matrix ${\bf S}$ with $M-n$ randomly selected activity centers.

{\small 
\begin{verbatim}
set.seed(2013)
parms<-list(N=100,alpha0= -.40, alpha1= 0,sigma=0.5)
data<-sim.mnSCR(parms,K=7,ssbuff=2)
nind<-nrow(data$Ycat)

M<-200
Ycat<-rbind(data$Ycat,matrix(nrow(data$X)+1,nrow=(M-nind),ncol=data$K))
Sst <-rbind(data$S,cbind(runif(M-nind,data$xlim[1],data$xlim[2]),
                         runif(M-nind,data$ylim[1],data$ylim[2])))
zst<-c(rep(1,160),rep(0,40))
\end{verbatim}
}

The model specification is not much more complicated than the binomial
or Poisson models given previously. The main consideration is that we
define the cell probabilities for each trap $j=1,2,\dots,J$ and then
define the last cell probability, $J+1$, for ``not captured'', to be
the complement of the sum of the others. The code is shown in Panel
\ref{poisson-mn.panel.mn}.
In the last lines of code here we
specify $N$ and density, $D$, as  derived parameters.


\begin{panel}[htp]
\centering
\rule[0.15in]{\textwidth}{.03in}
%\begin{minipage}{2.5in}
{\small
\begin{verbatim}
cat("
model {
psi ~ dunif(0,1)
alpha0 ~ dnorm(0,10)
sigma ~dunif(0,10)
theta<- 1/(2*sigma*sigma)

for(i in 1:M){
  z[i] ~ dbern(psi)
  S[i,1] ~ dunif(xlim[1],xlim[2])
  S[i,2] ~ dunif(ylim[1],ylim[2])
  for(j in 1:ntraps){
    #distance from capture to the center of the home range
    d[i,j] <- pow(pow(S[i,1]-X[j,1],2) + pow(S[i,2]-X[j,2],2),1)
  }
  for(k in 1:K){
    for(j in 1:ntraps){
      lp[i,k,j] <- exp(alpha0 - theta*d[i,j])*z[i]            
      cp[i,k,j] <- lp[i,k,j]/(1+sum(lp[i,k,]))
    }
    cp[i,k,ntraps+1] <- 1-sum(cp[i,k,1:ntraps])  # last cell = not captured
    Ycat[i,k] ~ dcat(cp[i,k,])
  }  
}   

N <- sum(z[1:M]) 
A <- ((xlim[2]-xlim[1])*trap.space)*((ylim[2]-ylim[1])*trap.space)
D <- N.tot/A
}
",file="model.txt")

\end{verbatim}
}
%\end{minipage}
\rule[-0.15in]{\textwidth}{.03in}
\caption{
WinBUGS model specification for the multinomial observation model. 
}
\label{poisson-mn.panel.mn}
\end{panel}

Finally we need to package everything up (inits, parameters, data) and send
it off to {\bf JAGS} to build a MCMC simulator for us:

{\small
\begin{verbatim}
library("rjags")

inits <- function(){list (z=zst,sigma=runif(1,.5,1) ,S=Sst) }              
parameters <- c("psi","alpha0","theta","sigma","N","D")
data <- list (X=data$X,K=data$K,trap.space=1,Ycat=Ycat,M=M,
              ntraps=nrow(data$X),ylim=data$ylim,xlim=data$xlim)         

out1 <- jags.model("model.txt", data, inits, n.chains=3, n.adapt=500)
out2 <- coda.samples(out1,parameters,n.iter=1000)
\end{verbatim}
}


Summary of analysis for the simulated data set here.....  



\section{ Mist-netting example}

Here we do an analysis of a real data set using the multinomial model.
the data are for 
adult Arctic Warblers ({\it Phylloscopus borealis}) banded 
 along the Colville River near Umiat, Alaska in 2006. The data are from 
 5 MAPS (REF) stations located in close proximity of one another, as
 well as 
 birds target (netids starting with "UMIA") or passive (netids starting 
 with "PASS") netted in the general area (a couple of nets, 
 netid == 'PASS01' and 'UMIAB3' are pretty far away though...). In total, 
 there are 258 captures of 179 individual birds. This is is really a 
 large number of birds of a single species for MAPS stations. 
 
Each of these MAPS stations has 12-15 nets.

We used XYZ....
 
A few issues:
 data is manipulated into multinomial trials and we have to convert.
 multiple captures somehow.....
 lots of space.
 transient individuals?  affect is N = number of guys ``ever available''
 


\section{SCR Models are Multi-State Models}

\begin{comment}
SCR models are multi-state models where stat-especific encounter
probabilities are a function of distance -- or something like that. 
\end{comment}

This multinomial observation model and also the discrete formulation
of the state-model given in section XYZ both allude to the fact that
SCR models are a variation of 
ordinary multi-state models \citep[][Chapt. 9]{kery_schaub:2011}
but where the state variable is static and represents a
geographic location. Multi-state models are extremely useful for
modeling movements among geographic states and indeed this type of
problem motivated their early developments by
\citet{arnason:1972,arnason:1973} and 
\citet{hestbeck_etal:1991} albeit in the context of a dynamic state
variable.  

Sometimes the state-space is naturally discrete. Consider a situation
in which discrete patches of habitat are searched using some method
and it might be convenient (or occur inadvertently) to associate
samples to the patch level instead of recording observation locations,
as in the capracillie example given in section XYZ above.  In this
case we use the discrete analog of the ``uniformity assumption'' in
which ${\bf s}_{i} \sim dcat(probs[])$ where $probs[]$ are the
probabilities that an individual inhabits a particular patch which
should be proportional to area of each patch.  Even when space is {\it
  not} naturally discrete, measurements are often made at a fairly
coarse grain (e.g., meters or tens of meters along a stream), or
associated with spatial quadrats for scat searches. And, of course, we
could approximate any continuous space with a discrete state-space,
and therefore apply multi-state models directly to any SCR problem.

\subsection{Modeling ‘manders on a stream network}

Here is a cool example: We catch salamander’s or fish along a stream
and only record stream segment instead of actual location – this is
motivated by Evan Grant’s work and also Lowe xyz??

each stream segment is individuals current state and its easy to use
either a Markov model or a home range model. ....

This is also a good open population example

\subsection{SCR as a Dynamic multi-state model}

Having a static state variable is not that interesting in the grand
scheme of multi-state models which most of the time consider a dynamic
state variable. Such models will arise frequently in spatial
capture-recapture settings. Let s denote the individual activity
center and suppose its state-space is discrete.  Now let $u[i,t]$ be
the patch in which individual $i$ was observed during sample $t$. Then
a simplistic movement model is that the successive movement outcomes
are $iid$
\[
u[i,t] \sim  dcat[ psi[s[i],] ]
\]

We can reformulate the basic SCR0 model as a dynamic multi-state model
as follows.  First lets grid up the state-space into “survey strata”
which we might define here has .5 unit squares so that the whole
state-space has 16*16 such squares. [actually do this so they are
centered on traps].We retain our assumption
\[
 s_{i} \sim Uniform(S)
\]
Secondly we define a movement model in which
\[
u[I,t] \sim dcat(pi)
\]
Where
\[
 pi_{k} = exp(-dist(x,s)/sigma2)/sum[that]
\]
This is the MAXENT distribution but also corresponds to Poisson with
mean $lam0*exp(-dist^2/sigma)$.  THIS IS CRUCIAL – THIS IS IMPT!!
 Makes it clear that encounter is the same as movement.

Now define
\[
 p|u[i,t] = p0*if(u[i,t] \in trap grid cell)
\]

Multi-state model with a “random movement” process.


We could easily extend this to a kind of Markovian movement model
where the probabilities depend on the previous state $u_{i,t-1}$ but
the simpler model of ``random'' movement satisfies our immediate needs.
 
So we see that SCR models are exactly a type of multi-state model when
the states are naturally discrete.  Another naturally discrete
state-space is ``nest sites''. Goncalo’s study and Florent’s
study. Schaub’s study on woopoos.


\section{Single-catch traps}
\label{poisson-mn.sec.singlecatch}

The classical animal trapping experiment is based on a physical trap
which captures a single animal and holds that individual until
subsequent molestation by a biologist. 
This type of observation model -- the ``single catch'' trap -- 
was the original situation considered by \citet{efford:2004}.

The single-catch model is basically a multinomial model but one in
which the number of available traps is reduced as each individual is
captured. As such, the constraints on the likelihood for each
individual are latent and shit is complicated beyond belief.
As a result, at the time of this writing, there has not been a formal
development of either likelihood  or Bayesian analysis of this model.

Nevertheless, it is not too difficult to describe the basic model
formally. In particular, there is a nice conditional structure resulting from a ``removal
process'' operating on the traps.  The first guy captured has the 
basic multinomial observation model:
\[
{\bf y}_{i} \sim Multinom({\bm \pi}_{i})
\]
whereas the 2nd guy captured has one cell removed:
\[
{\bf y}_{i} \sim Multinom({\bm \pi}_{i}(1-{\bf y}_{i})    )
\]
and so on.
So the {\bf order of capture} is relevant to the construction of these
multinomial cell probabilities. 
Thus the observations each have a multinomial model, but the
multinomial observations have a unique kind of conditional dependence
structure among them.

\subsection{Approximate Analysis}

To analyze the model here we consider using a misspecified model based
on either the Poisson or independent multinomial


How good of an approximation is the multi-catch model?

What about the Poisson model with a really low lambda?

Can we solve the big kahuna?

Use Sarah's data here.


\section{Trapping Webs}


\section{Acoustic Arrays}


\section{Summary and Outlook}

There are other types of encounter models.......

Efford adapts SCR models to acoustic detection devices.... a few words
on that here.....

There are models for which
only specific summary statistics are observable (Chandler and Royle
2011, etc..) which we cover in chapter XYZ.  We consider other models
for detection probability in some prior or later chapter. 







\chapter{
Likelihood Analysis of Spatial Capture-Recapture Models
}
\markboth{Chapter 5}{}
\label{chapt.mle}

%%%% TO-DO LIST
% 1. comparison of Bayes with MLE for wolverine data (need to rerun WinBUGS)
% 2. Beth clean up a couple things in SECR analysis.
% 3. Need to finish MLE for restricted state-space
%%   requires code from Rahel
% 4. draft up intlik3 wrapper "scr()"

\vspace{.3in}



In this book we mainly focus on Bayesian analysis of spatial
capture-recapture models. And, in the previous chapters we learned how
to fit some basic spatial capture-recapture models using a Bayesian
formulation of the models analyzed in BUGS engines including {\bf
  WinBUGS} and {\bf JAGS}.  Despite our focus on Bayesian analysis, it
is instructive to develop the basic conceptual and methodological
ideas behind classical analysis based on likelihood methods and
frequentist inference.  
This has been the approach taken by
\citet{borchers_efford:2008, dawson_efford:2009} and related papers.
Simple SCR models can be analyzed
fairly easily using such methods and, including even some classes of
models that we have not been able to fit using Bayesian methods. One
such class of models are those
that account for ecological distance in the detection model,
which we cover in  Chapt. \ref{chapt.ecoldist}).


This chapter provides some conceptual and technical footing for
likelihood-based analysis of spatial capture-recapture models. We
recognized earlier (Chapt. 4) that SCR models are versions of
binomial (or other) GLMs, but with random effects – i.e., GLMMs. These
models are 
routinely analyzed by likelihood methods. In particular, likelihood
analysis is based on the integrated likelihood in which the random
effects are removed by integration from the likelihood. In SCR models,
the random effect, ${\bf s}$, i.e., the 2-dimensional coordinate, is a
bivariate random effect. 

In this chapter, we show that it is
straightforward to compute the maximum likelihood estimates (MLE) for
SCR models by integrated likelihood. We develop the MLE framework
using {\bf R}, and we also provide a basic introduction to an {\bf R} package
\mbox{\tt secr} \citep{efford:2011} which is based on the stand-alone
package 
{\bf DENSITY} \citep{efford_etal:2004}.
 To set the context we analyze the SCR model
here when $N$ is known because, in that case, it is precisely a GLMM and
does not pose any difficulty at all. We generalize the model to allow
for unknown $N$ using both conventional ideas based on the ``joint
likelihood'' \citep[e.g.,][]{borchers_etal:2002}
and also using a formulation
based on data augmentation.  We obtain the MLEs for 
the SCR model from the wolverine camera trapping study \citep{magoun_etal:2011}
 analyzed in previous chapters to compare/contrast the
results.

\section{Likelihood analysis }

We noted in chapter 4 that, with $N$ known, the basic SCR model is a
type of binomial regression with a random effect. For such models we
can easily obtain maximum likelihood estimators of model parameters
based on integrated likelihood. The integrated likelihood is based on
the marginal distribution of the data $y$ in which the random effects
are removed by integration. Conceptually, our model is a specification
of the conditional-on-${\bf s}$ model $[y|{\bf s},\alpha]$ and we have
a ``prior distribution'' for ${\bf s}$, say $[{\bf s}]$, and the
marginal distribution of the data $y$ is
\[
[y|\alpha] =  \int_{\bf s} [y|{\bf s},\alpha][{\bf s}] d{\bf s}.
\]
When viewed as a function of $\alpha$ for purposes of estimation, the
marginal distribution $[y|\alpha]$ is often referred to as the {\it
  integrated likelihood}.

It is worth analyzing 
the simplest SCR model with known-$N$ in order to understand the
underlying mechanics and basic concepts. These are directly relevant to
the manner in which many capture-recapture models are classically
analyzed, such as model $M_h$, and individual covariate models (see
Chapt. \ref{chapt.closed} and  \citet[][chapt. 6]{royle_dorazio:2008}). To develop integrated
likelihood for SCR models, we first identify the conditional
likelhiood. 

The observation model for each encounter observation $y_{ij}$,
specified conditional on ${\bf s}_{i}$, is 
\begin{equation}
	y_{ij}| {\bf s}_{i} \sim \mbox{Bin}(K, p_{\alpha}({\bf x}_{j},{\bf s}_{i}))
\label{mle.eq.cond-on-s}
\end{equation}
where we have indicated the dependence of $p_{ij}$ on ${\bf s}$ and
parameters $\alpha$
explicitly.
For the random effect we have ${\bf s}_{i} \sim  \mbox{Unif}({\cal
  S})$.
The joint distribution of the data for individual $i$ is the product
of $J$ such terms (i.e., contributions from each of $J$ traps).
\[
  [{\bf y}_{i} | {\bf s}_{i} , \alpha] = 
  \prod_{j=1}^{J} \mbox{Bin}(K, p_{\alpha}({\bf x}_{j},{\bf s}_{i}) )
\]
We note this assumes that encounter of individual $i$ in each
trap is independent of encounter in every other trap, conditional on
${\bf s}_{i}$, this is the fundamental property of the basic model SCR0.


 The so-called marginal likelihood is computed by removing
${\bf s}_{i}$, by integration (hence also {\it integrated} likelihood), from the conditional-on-${\bf s}$
likelihood and regarding the {\it marginal} distribution of the data
as 
the likelihood. That
is, we compute:
\[
  [y|\alpha] = 
\int_{{\cal S}}  [ {\bf y}_{i} |{\bf s}_{i}, \alpha] g({\bf s}_{i}) d{\bf s}_{i}
\]
In most SCR models, $g({\bf s}) = 1/||{\cal S}||$ (but see Chapt. \ref{chapt.state-space}).

The joint likelihood for all $N$ individuals, assuming independence of
encounters among individuals, is the product of $N$ such terms:
\[
{\cal L}(\alpha | {\bf y}_{1},{\bf y}_{2},\ldots, {\bf y}_{N}) =     \prod_{i=1}^{N}
[{\bf y}_{i}|\alpha]
\]
We emphasize that two independence assumptions are explicit in this
development: independence of trap-specific encounters within
individuals and also independence among individuals. In particular,
this would only be valid when individuals are not physically
restrained or removed upon capture, and when traps do not ``fill up''.

The key operation for computing the likelihood is solving a
2-dimensional integration problem. There are some general purpose {\bf
  R} packages that implement a number of 
 multi-dimensional integration routines
including \mbox{\tt adapt} \citep{genz_etal:2007} and \mbox{\tt R2cuba}
\citep{hahn_etal:2011}.  In practice, we won't rely
on these extraneous {\bf R} packages (except see
chapt. \ref{chapt.state-space} for an application of \mbox{\tt Rcuba})
but instead will use perhaps less
efficient methods in which we replace the integral with a summation
over an equal area mesh of points on the state-space ${\cal S}$ and explicitly
evaluate the integrand at each point. We invoke the rectangular rule
for integration here\footnote{e.g., 
\url{http://en.wikipedia.org/wiki/Rectangle_method}
} in which we
evaluate the
integrand on a regular grid of points of equal area and compute the
average of
the integrand over that grid of points. 
Let $u=1,2,\ldots,nG$ index a grid of
$nG$ points, ${\bf s}_{u}$,  where the area of grid cell $u$ is
constant, say $A$.
In this case, the integrand, i.e., the marginal pmf of 
${\bf y}_{i}$, is approximated by  
\begin{equation}
         [{\bf y}_{i}|\alpha] = \frac{1}{nG} \sum_{u=1}^{nG}  [ {\bf
            y}_{i} |{\bf s}_u, \alpha]
\label{mle.eq.intlik}
\end{equation}

This is a specific case of the general expression that could be used
for approximating the integral for any arbitrary (bivariate or otherwise)
distribution $g({\bf s})$. The general case is
\[
[y]  = \frac{A}{nG} \sum_{u} [y|{\bf s}_{u}] [{\bf s}_{u}]
\]
 In the present context it happens that  $[{\bf s}] = (1/A)$
and thus the grid-cell area cancels in the above
expression to yield eq. \ref{mle.eq.intlik}.
The rectangular rule for integration can be seen as an application of
the Law of Total Probability for a discrete random variable ${\bf
  s}$, having $nG$ 
unique values with equal probabilities $1/nG$.



\subsection{ Implementation (simulated data)}

Here we will illustrate how to carryout this integration and
optimization based on the integrated likelihood using simulated data
 (i.e., following that from Chapter 4). Using \mbox{\tt simSCR0.fn}
 we simulate data for 100 individuals and a 25 trap array
laid out in a $5 \times 5$ grid of unit spacing.  The specific encounter
model is the half-normal model. The 100 activity centers were
simulated on a state-space defined by a $8 \times 8$ square 
within which the
trap array was centered (thus the trap array is buffered by 2
units). Therefore, the density of individuals in this system is fixed
at $100/64$.

In the following set of {\bf R} commands we generate the data and 
then harvest the required data objects:
{\small
\begin{verbatim}
data<-simSCR0.fn(discard0=FALSE,sd=2013)
y<-data$Y
traplocs<-data$traplocs
nind<-nrow(y)
X<-data$traplocs
J<-nrow(X)
K<-data$K
Xl<-data$xlim[1]
Yl<-data$ylim[1]
Xu<-data$xlim[2]
Yu<-data$ylim[2]
\end{verbatim}
}
Now we need to define the integration grid, say ${\bf G}$, which we do with
the following set of {\bf R} commands (here, \mbox{\tt delta} is the grid spacing):
{\small
\begin{verbatim}
delta<- .2
xg<-seq(Xl+delta/2,Xu-delta/2,by=delta) 
yg<-seq(Yl+delta/2,Yu-delta/2,by=delta) 
npix<-length(xg)          # assumes xg and yg same dimension here
area<- (Xu-Xl)*(Yu-Yl)/((npix)*(npix)) # don’t need area for anything
G<-cbind(rep(xg,npix),sort(rep(yg,npix)))
nG<-nrow(G)
\end{verbatim}
}
In this case, the integration grid is set up as a grid with spacing
$\delta = 0.2$ which produces a $40 \times 40$ grid of points for evaluating the
integrand if the state-space buffer is set at 2.

We next create an {\bf R} function that defines the likelihood as a function
of the data objects $y$ and $X$ which were created above but, in general,
you would read these files into {\bf R}, e.g., from a .csv file.
In addition to these data
objects, we need to have defined the  quantities $G$ and $nG$ associated
with the integration grid.
However, instead of worrying about making all of these objects and
keeping track of them we just put that code above into the likelihood
function, say \mbox{\tt intlik1}, and pass $\delta$ as an additional (optional) argument and a
few other things that we need such as the boundary of the state-space
over which the integration (summation) is being done. This function is
available in the package \mbox{\tt scrbook}, and it is reproduced here:

{\small 
\begin{verbatim}
intlik1<-function(parm,y=y,delta=.2,X=traplocs,ssbuffer=2){

Xl<-min(X[,1]) - ssbuffer 
Xu<-max(X[,1]) + ssbuffer
Yu<-max(X[,2]) + ssbuffer
Yl<-min(X[,2]) - ssbuffer

xg<-seq(Xl+delta/2,Xu-delta/2,,length=npix) 
yg<-seq(Yl+delta/2,Yu-delta/2,,length=npix) 
npix<-length(xg)

G<-cbind(rep(xg,npix),sort(rep(yg,npix)))
nG<-nrow(G)
D<- e2dist(X,G)  

alpha0<-parm[1]
alpha1<-parm[2]
probcap<- plogis(alpha0)*exp(-alpha1*D*D)
Pm<-matrix(NA,nrow=nrow(probcap),ncol=ncol(probcap))
                    # all zero encounter histories
n0<-sum(apply(y,1,sum)==0) 
                    # encounter histories with at least 1 detection
ymat<-y[apply(y,1,sum)>0,] 
ymat<-rbind(ymat,rep(0,ncol(ymat)))
lik.marg<-rep(NA,nrow(ymat))
for(i in 1:nrow(ymat)){
Pm[1:length(Pm)]<- (dbinom(rep(ymat[i,],nG),K,probcap[1:length(Pm)],log=TRUE))
lik.cond<- exp(colSums(Pm))
lik.marg[i]<- sum( lik.cond*(1/nG))  
}
nv<-c(rep(1,length(lik.marg)-1),n0)
-1*( sum(nv*log(lik.marg)) )
}
\end{verbatim}
}


The function accepts as
input the encounter history matrix, $y$, the trap locations, $X$, and the
state-space buffer. This allows us to vary the state-space buffer and
easily evaluate the sensitivity of the MLE to the size of the
state-space. 
Note that we have a peculiar handling of the encounter history
matrix $y$. In particular, we remove the all-zero encounter histories
from the matrix and tack-on a single all-zero encounter history as the
last row which then gets weighted by the number of such encounter
histories (\mbox{\tt n0}). This is a bit long-winded and strictly unnecessary
when $N$ is known, but we did it this way because the extension to the
unknown-$N$ case is now transparent (as we demonstrate in the following
section). 
 The matrix \mbox{\tt Pm} holds the log-likelihood contributions of
each encounter frequency for each possible state-space location of the
individual. 
The log contributions are summed up and the result
exponentiated on the next line, producing lik.cond, the
conditional-on-${\bf s}$ likelihood (Eq. \ref{mle.eq.cond-on-s}
above). The marginal
likelihood (\mbox{\tt lik.marg}) sums up the conditional elements weighted by
$\Pr({\bf s})$ (Eq. \ref{mle.eq.intlik} above).
This is a fairly primitive function which doesn't allow much
flexibility in the data structure. For example, it assumes that $K$,
the number 
of replicates, is constant for each trap. Further, it assumes that the
state-space is a square. We generalize this to some extent later in
this chapter. 

Here is the {\bf R} command for maximizing the likelihood and saving the
results into an object called \mbox{\tt frog}.  The output is a list of the
following structure and these specific estimates are produced using
the simulated data set:

{\small 
\begin{verbatim}
# should take 15-30 seconds

starts<-c(-2,2)
frog<-nlm(intlik1,starts,y=y,delta=.1,X=traplocs,ssbuffer=2,hessian=TRUE)
frog

$minimum
[1] 297.1896

$estimate
[1] -2.504824  2.373343

$gradient
[1] -2.069654e-05  1.968754e-05

$hessian
          [,1]      [,2]
[1,]  48.67898 -19.25750
[2,] -19.25750  13.34114

$code
[1] 1

$iterations
[1] 11
\end{verbatim}
} 
Details about this output can be found on the help page for
\mbox{\tt nlm}. We note briefly that \mbox{\tt frog\$minimum} is the
negative log-likelihood value at the MLEs, which are stored in the
\mbox{\tt frog\$estimate} component of the list. The Hessian is the
observed Fisher information matrix, which can be inverted to obtain
the variance-covariance matrix using the commands:
\begin{verbatim}
> solve(frog$hessian)
\end{verbatim}

It is worth drawing attention to the fact that the estimates are
different than the Bayesian estimates reported previously in Chapt. \ref{chapt.scr0}.
How can that be?!  There are several reasons for
this.  First Bayesian inference is based on the posterior distribution
and it is not generally the case that the MLE should correspond to any
particular value of the posterior distribution. If the prior
distributions in a Bayesian analysis are uniform, then the
(multivariate) mode of the
posterior is the MLE, but note that Bayesians almost always report
posterior {\it means} and so there will typically be a discrepancy
there. Secondly, we have implemented an approximation to the integral
here and there might be a slight bit of error induced by that. We will
evaluate that shortly. Third, the Bayesian analysis by MCMC is subject
to some amount of Monte Carlo error which the analyst should always be
aware of in practical situations.  All of these different explanations
are likely responsible for some of the discrepancy. Accounting for
these, we see general consistency between the
two estimates.

To compute the integrated likelihood we used a discrete representation
of the state-space so that the integral could be approximated as a
summation over possible values of ${\bf s}$ with each value being
weighted by its probability of occurring, which is $1/nG$ under the
assumption that ${\bf s}$ is uniform on the state-space ${\cal
  S}$. Recall
in Chapt. \ref{chapt.scr0} we 
used a discrete state-space in developing a Bayesian analysis of the
model in order to be able to modify the state-space in a flexible
manner. In that case, we could use the discretized state-space as the
integration grid and just feed it into our integrated likelihood
routine. 

In summary, we note that, for the basic SCR model, integrated
likelihood is a really easy calculation when $N$ is known. Even for $N$
unknown it is not too difficult, and we will do that shortly.
However, if you can solve the known-$N$ problem then you should be able
to do a real analysis, for example by considering different values of
$N$ and computing the results for each value and then making a plot of
the log-likelihood or AIC and choosing the value of $N$ that produces
the best log-likelihood or AIC. As a homework problem we suggest that
the reader take the code given above and try to estimate $N$ without
modifying the code – by just repeatedly calling that code for
different values of $N$ and trying to deduce the best value.
We will formalize the unknown-$N$ problem shortly.

The
software package {\bf DENSITY} \citep{efford_etal:2004} implements
certain types of SCR models using integrated likelihood methods, and
\mbox{\tt secr} \citep{efford:2011} is an {\bf R} package with similar functionality.
We provide an analysis of some data using \mbox{\tt secr} shortly along
with a discussion of its capabilities, and we use \mbox{\tt secr} in
later chapters for likelihood analysis of other SCR models.


\section{MLE when N is Unknown} 

Here we build on the previous introduction to integrated likelihood
but we consider now the case in which $N$ is unknown. We will see that
adapting the analysis based on the known-$N$ model is really
straightforward for the more general problem. The main distinction is
that we don’t observe the all-zero encounter history so we have to
make sure we compute the probability for that encounter history which
we do by tacking a row of zeros onto the encounter history matrix. In
addition, we include the number of such all-zero encounter histories
as an unknown parameter of the model. Call that unknown quantity
$n_{0}$, and we have to 
be sure to include a combinatorial term to
account for the fact that of the $n$ observed individuals there are
${N \choose n}$
 ways to realize a sample of size $n$. The combinatorial term
involves the unknown $n_{0}$ and thus it must be included in the likelihood.

Operationally then, things proceed much as before: 
We compute the marginal probability of each observed ${\bf y}_{i}$,
i.e., by removing the latent ${\bf s}_{i}$ by integration. In
addition, we 
 compute the marginal probability of the ``all-zero'' encounter
history ${\bf y}_{n+1}$, and make sure to weight it $n_{0}$ times. We
accomplish this by ``padding'' the data set with a single encounter
history having $y_{n+1,j}=0$ for all traps $j=1,2,\ldots,J$. Then we
be sure to include the combinatorial term in the likelihood or
log-likelihood computation. We demonstrate this shortly.

To analyze a specific case, we’ll read in our fake data set (simulated
using the parameters given above). To set some things up in our
workspace we do this:
\begin{verbatim}
data<-simSCR0.fn(discard0=TRUE,sd=2013)
y<-data$Y
nind<-nrow(y)
X<-data$traplocs
J<-nrow(X)
K<-data$K
\end{verbatim}
Recall that these data were generated with $N=100$, on an $8 \times 8$ unit
state-space representing the trap locations (${\bf X}$) buffered by 2 units.

As before, the likelihood is defined in the {\bf R} workspace as an
{\bf R}
function, \mbox{\tt intlik2} (contained in the package \mbox{\tt
  scrbook}),
 which takes an argument being the unknown parameters of the
model and additional arguments as prescribed. In particular, 
 we provide the encounter history matrix ${\bf y}$, the trap locations
\mbox{\tt traplocs}, the spacing of the integration grid (argument
\mbox{\tt delta}) and the
state-space buffer. Here is the new likelihood function:
{\small
\begin{verbatim}
intlik2<-function(parm,y=y,delta=.3,X=traplocs,ssbuffer=2){

Xl<-min(X[,1]) -ssbuffer
Xu<-max(X[,1])+ ssbuffer
Yu<-max(X[,2])+ ssbuffer
Yl<-min(X[,2])- ssbuffer

xg<-seq(Xl+delta/2,Xu-delta/2,delta) 
yg<-seq(Yl+delta/2,Yu-delta/2,delta) 
npix.x<-length(xg)
npix.y<-length(yg)
area<- (Xu-Xl)*(Yu-Yl)/((npix.x)*(npix.y))
G<-cbind(rep(xg,npix.y),sort(rep(yg,npix.x)))
nG<-nrow(G)
D<- e2dist(X,G) 

alpha0<-parm[1]
alpha1<-parm[2]
n0<-exp(parm[3])
probcap<- plogis(alpha0)*exp(-alpha1*D*D)
Pm<-matrix(NA,nrow=nrow(probcap),ncol=ncol(probcap))
ymat<-rbind(y,rep(0,ncol(y)))

lik.marg<-rep(NA,nrow(ymat))
for(i in 1:nrow(ymat)){
Pm[1:length(Pm)]<- (dbinom(rep(ymat[i,],nG),K,probcap[1:length(Pm)],log=TRUE))
lik.cond<- exp(colSums(Pm))
lik.marg[i]<- sum( lik.cond*(1/nG) )  
}                                                 
nv<-c(rep(1,length(lik.marg)-1),n0)
part1<- lgamma(nrow(y)+n0+1) - lgamma(n0+1)
part2<- sum(nv*log(lik.marg))
 -1*(part1+ part2)
}
\end{verbatim}
}
To execute this function for the data that we created with \mbox{\tt simSCR0.fn},
 we execute the following command (saving the result in our
friend \mbox{\tt frog}).
This results in the usual output, including the parameter estimates,
the gradient, and the numerical Hessian which is useful for obtaining
asymptotic standard errors (see below):
\begin{verbatim}
starts<-c(-2.5,2,log(4))
frog<-nlm(intlik2,starts,hessian=TRUE,y=y,X=X,delta=.2,ssbuffer=2)

There were 50 or more warnings (use warnings() to see the first 50)

frog
$minimum
[1] 113.5004

$estimate
[1] -2.538334  2.466515  4.232810

[... Additional output deleted ...]
\end{verbatim}
While this produces some {\bf R} warnings, these happen to be harmless
in this case, and we will see from the \mbox{\tt nlm} output that the
algorithm performed satisfactory in minimizing the objective function.
The estimate of population size for the state-space (using the default 
state-space buffer) is
\begin{verbatim}
nrow(y)+exp(4.2328)
[1] 110.9099
\end{verbatim}
Which differs from the data-generating value ($N=100$) as we might
expect for a single realization. We usually will present an estimate of uncertainty associated
with this MLE which we can obtain by inverting the Hessian. Note that
$Var(\hat{N}) = n + \mbox{Var}(\hat{n}_{0})$.
Since we
have parameterized the model in terms of $log(n_{0})$ we use a delta
approximation to obtain the variance on the scale of $n_{0}$ as
follows:
\begin{verbatim}
(exp(4.2328)^2)*solve(frog$hessian)[3,3]
[1] 260.2033
> sqrt(260)
[1] 16.12452
\end{verbatim}
Therefore, the asymptotic ``Wald-type'' confidence interval for $N$ is
$110.91 \pm 1.96 \times 16.125 = (79.305, 142.515)$. To report this in
terms of density, we scale appropriately by the area of the prescribed
state-space which is $64$ units of area (i.e., an $8 \times 8$ square).


\begin{comment}

\subsection{Exercises}

{\flushleft 
{\bf 1.}	
Run the analysis with different state-space buffers and comment on the result. 
}


{\flushleft 
{\bf 2.} Conduct a brief simulation study using this code by
  simulating 100 data sets and obtain the MLEs for each data set. Do
  things seem to be working as you expect?  }

{\flushleft 
{\bf 3.} 
Further extensions: It should be straightforward to
  generalize the integrated likelihood function to accommodate many
  different situations. For examples, if we want to include more
  covariates in the model we can just add stuff to the object \mbox{\tt probcap},
 and add the relevant parameters to the argument that gets
  passed to the main  function.  For the simulated data, make up a
  covariate by generating a Bernoulli covariate (``trap type'' – perhaps
  baited or not baited) randomly and try to modify the likelihood to
  accommodate that.  }

{\flushleft {\bf 4.}  We would probably be interested in devising the
  integrated likelihood for the full 3-d encounter history array so
  that we could include temporally varying covariates. This is not
  difficult but naturally will slow down the execution
  substantially. The interested reader should try to expand the
  capabilities of this basic {\bf R} function.  }
\end{comment}




\subsection{Integrated Likelihood using the model under data augmentation } 

Note that this likelihood analysis is based on the standard likelihood
in which $N$ (or $n_{0}$) is an explicit parameter. This is usually called
the ``joint likelihood'' or ``unconditional likelihood''.  We could also
express the joint likelihood using data augmentation, replacing the
parameter $N$ with $\psi$ \citep[e.g., see Sec. 7.1.6][for an example]{royle_dorazio:2008}.
We don't go into detail here, but we note that the
likelihood under data augmentation is a zero-inflated binomial
mixture – precisely an occupancy type model \citep{royle:2006}.
Thus, while it is possible to carryout likelihood analysis of
models under data augmentation, we primarily advocate data
augmentation for Bayesian analysis.


\subsection{ Extensions}

We have only considered basic SCR models with no additional
covariates. However, in practice, we are interested in other types of
covariate effects including ``behavioral response'', 
sex-specificity of parameters, and potentially other effects. Some of
these  can be added directly to the likelihood – if the covariate is fixed
and known for all individuals captured or not. An example is a
behavioral response, which amounts to having a covariate $x_{ik}=1$ if
individual $i$ was captured prior to occasion $k$ and $x_{ik}=0$
otherwise. For uncaptured individuals, $x_{ik}=0$ for all $k$.
 \citet{royle_etal:2011jwm} called this a global behavioral
response because the covariate is defined for all traps, no matter the
trap in which an individual was captured. We could also define a {\it
  local} behavioral response which occurs at the level of the trap,
i.e., $x_{ijk}=1$ if individual $i$ was captured in trap $j$ prior to
occasion $k$, etc.. 
Trap-specific covariates such as trap type or status, or
time-specific covariates such as date, are easily accommodated as
well. As an example, \citet{kery_etal:2010} develop a model for the
European wildcat in which traps are either baited or not (a
trap-specific covariate with only 2 values), and also encounter
probability varies over time in the form of a quadratic seasonal response.
We consider models with behavioral response or fixed covariates in
Chapt. \ref{chapt.covariates}.
Although the integrated likelihood routines we provided above can be
modified directly for such cases, which we leave to the interested
reader to investigate. 

Sex-specificity is more difficult to deal with since sex is not known
for uncaptured individuals (and sometimes not even for all captured
individuals).  To analyze such models, we do Bayesian analysis of the
joint likelihood facilitated by the use of data augmentation
\citep{gardner_etal:2010jwm,russell_etal:2012}. For covariates that are
not fixed and known for all individuals, it is somewhat more
challenging to do MLE for these based on the joint likelihood as we
have developed above. Instead it is more conventional to use what is
colloquially referred to as the ``Huggins-Alho'' type model which is
one of the approaches taken in the software package \mbox{\tt secr}
\citep[][see Sec. \ref{mle.sec.secr}]{efford:2011}. This idea is
motivated by thinking about unequal probability sampling methods known
as Horvitz-Thompson sampling \citep[e.g.,
see][]{overton_stehman:1995}.  We don't use that method anywhere in
this book because it represents a paradigm shift in the inference
framework which is done historically only for convenience (i.e., ease
of constructing an estimator) and not for philosophical or theoretical
reasons.






\section{Classical model selection and assessment}

In most analyses, one is interested in choosing from among various
potential models, or ranking models, or something else to do with
assessing the relative merits of a set of models. A good thing about
classical analysis based on likelihood is we can apply AIC methods
\citep{burnham_anderson:2002} without difficulty. There are two
distinct contexts for model-selection that we think are relevant to
SCR models. First is, and AIC selecting among models that represent
distinct biological hypotheses (e.g., covariates affecting encounter
probability or density). AIC is convenient for assessing the relative
merits of these different models although if there are only a few
models it is not objectionable to use hypothesis tests or confidence
intervals to determine importance of effects. The second model
selection context has to do with choosing among various detection
functions although, as a general rule, we don't recommend this
application of model selection.  This is because there is hardly ever
(if at all) a rational subject-matter based reason motivating specific
distance functions. As a result, we believe that doing too much model
selection will invariably lead to over-fitting and thus over-statement
of precision. This is the main reason that we haven't loaded you down
with a basket of models for detection probability so far, although we
discuss many possibilities in Chapt. \ref{chapt.gof} where we also
discuss focus more attention on methods and applications of model selection.


{\bf Goodness-of-fit} -- For many standard capture-recapture models,
it is possible to identify goodness-of-fit statistics based on the
multinomial likelihood and evaluate model adequacy using formal
statistical tests. Similar strategies can be applied to SCR models
using expected cell-frequencies based on the marginal distribution of
the observations. Also, because computing MLEs is somewhat more
efficient in many cases compared to Bayesian analysis, it is also
sometimes easy to use bootstrap methods\footnote{I'm not sure if there
  are references in the context of SCR models for this stuff....???}.

Bayesian goodness-of-fit, which we take up in more detail in
Chapt. \ref{chapt.gof}, is almost always addressed with Bayesian
p-values or some other posterior predictive check
(sec. \ref{glms.sec.gof}, \citet[][sec. 2.6]{kery:2010}).
\citet{royle_etal:2011mee} suggested checking model fit for SCR models
by decomposing fit into two components: (1) That of the encounter
process model, evaluated by the expected encounter frequencies
computed {\it conditional} on ${\bf s}$; and, (2) That of the spatial
point process model (``spatial randomness'').


\section{Likelihood analysis of the wolverine camera trapping data}
\label{mle.sec.wolverine}


Here we compute the MLEs for the wolverine data using an expanded
version of the function we developed in the previous section. To
accommodate that each trap might be operational a variable number of
nights, we provided an additional argument to the likelihood function
(allowing for a vector $K$), which requires also a modification to the
construction of the likelihood.  In addition,
we accommodate  the state-space is a general rectangle, and
we included a line in the code to compute the state-space area which
we apply below for computing density.  The more general function
(\mbox{\tt intlik3}) is given in the {\bf R} package \mbox{\tt scrbook}. It has a general
purpose wrapper named \mbox{\tt scr}\footnote{Not written yet} which has other capabilities too. 
To use this function to obtain the MLEs for the wolverine camera trap
study, we execute the following commands (note: these are in the help
file and will execute if you type \mbox{\tt example(intlik3)}:
{\small
\begin{verbatim}
library("scrbook")
data("wolverine")
 
traps<-wolverine$wtraps
traplocs<-traps[,1:2]/10000
K.wolv<-apply(traps[,3:ncol(traps)],1,sum)
traps<-cbind(1:nrow(traps),traps)  # pad an ID variable
y3d<-SCR23darray.fn(wolverine$wcaps,traps)
y2d<-apply(y3d,c(1,3),sum)

starts<-c(-1.5,1.2,log(4))
frog<-nlm(intlik3,starts,hessian=TRUE,y=y2d,K=K.wolv,X=traplocs,delta=.2,ssbuffer=2)
There were 23 warnings (use warnings() to see them)

frog
$minimum
[1] 220.4313

$estimate
[1] -2.817610  1.254757  3.583690

$gradient
[1]  1.210460e-06 -5.255072e-06 -5.710212e-07

$hessian
           [,1]       [,2]      [,3]
[1,]  37.686164 -11.849561  4.686501
[2,] -11.849561  30.842624 -9.193201
[3,]   4.686501  -9.193201 12.973354

$code
[1] 1

$iterations
[1] 12
\end{verbatim}
}
Of course we're interested in obtaining an estimate of population size
for the prescribed state-space, or density, and associated measures of
uncertainty which we do using the delta method approximation
\citep[][Appendix F4]{williams_etal:2002}
\footnote{
We found a good set of notes on the delta approximation on Dr. David
Patterson's ST549 notes: 
\url{http://www.math.umt.edu/patterson/549/Delta.pdf}
}).
To do all of that we need to manipulate the output of
\mbox{\tt nlm} since we have our  estimate in terms of $\mbox{\tt
  log(n0)}$. We execute the following commands:
{\small 
\begin{verbatim}
area<-attr(intlik3(starts,y=y2d,K=K.wolv,X=traplocs,delta=.2,ssbuffer=2),"SSarea")
Nhat<-nrow(y2d)+exp(frog$estimate[3])
area<-attr(intlik3(starts,y=y2d,K=K.wolv,X=traplocs,delta=.2,ssbuffer=2),"SSarea")
Dhat<- Nhat/area

Dhat
[1] 0.5494956

SE<- (1/area)*exp(frog$estimate[3])*sqrt(solve(frog$hessian)[3,3])

SE
[1] 0.1087101
\end{verbatim}
} 
So our estimate of density is $0.55$ individuals per ``standardized
unit'' which is 100 $km^2$, because we divided UTM coordinates by
10000.  So this is about 5.5 individuals per 1000 $km^2$ (the units
reported by \citep{royle_etal:2011jwm}), with a SE of around 1.09
individuals.  This compares closely with $5.77$
reported\footnote{check this!!!} in
sec. \ref{scr0.sec.wolverine} based on Bayesian
analysis of the model.


To evaluate the effect of the integration grid density, 
we obtained the MLEs for a state-space buffer of 2 (standardized
units) and for integration grid with spacing $\delta = .3, .2, .1,
.05$. The MLEs for these 4 cases including the relative runtime are
given in Table \ref{mle.tab.integration}.
We see the results change only slightly as the fineness of the
integration grid increases. Conversely, the runtime on the platform of
the day for the 4 cases increases rapidly. 
As we have suggested previously these runtimes could be regarded in
relative terms,  across platforms, for gaging the decrease in
speed as the fineness of the integration grid increases. The effect of
this is that we anticipate some numerical error in approximating the
integral on a mesh of points, and that error increases as the
coarseness of the mesh increases. 


\begin{table}[ht]
\centering
\caption{Run time and MLEs for different integration grid resolutions.}
\begin{tabular}{l|rccc}
\hline \hline
$\delta$ &   & \multicolumn{3}{c}{Estimates} \\ \hline
         &  runtime        & $\alpha_0$ & $\alpha1$ & $log(n_0)$ \\ \hline
 0.30   &  9.9  &  -2.819786 & 1.258468 & 3.569731  \\
 0.20   & 32.3  &  -2.817610 & 1.254757 & 3.583690 \\
 0.10  & 115.1  &  -2.817570 & 1.255112 & 3.599040 \\
 0.05 &  407.3 &   -2.817559&  1.255281&  3.607158 \\
\end{tabular}
\label{mle.tab.integration}
\end{table}


We studied the effect of the state-space buffer on the MLEs,
using a fixed $\delta = .2$ for all analyses. We used state-space buffers
of 1 to 4 units stepped by .5. This produced the following results,
given here are the state-space buffer, area of the state-space, the
MLE of $N$ for the prescribed state-space and the corresponding MLE of
density:
{\small
\begin{verbatim}
     ssbuff       Ass      Nhat      Dhat
[1,]    1.0  66.98212  37.73338 0.5633352
[2,]    1.5  84.36242  46.21008 0.5477567
[3,]    2.0 103.74272  57.00617 0.5494956
[4,]    2.5 125.12302  69.03616 0.5517463
[5,]    3.0 148.50332  82.17550 0.5533580
[6,]    3.5 173.88362  96.44018 0.5546249
[7,]    4.0 201.26392 111.83524 0.5556646
\end{verbatim}
}
The estimates of $D$ stabilize rapidly and the incremental difference
is within the numerical error associated with approximating the
integral.  


\subsection{Restricted state-space}
\label{mle.sec.shapefile}

In sec. \ref{scr0.sec.discrete} 
 we used a discrete representation of
the state-space in order to have control over its extent and shape,
for example so that we could clip out ``non-habitat''. Clearly that
formulation of the model is relevant to the use of integrated
likelihood in the sense that such a representation of the state-space
underlies the computation of the integral. Thus, for example, we could
easily compute the MLE of parameters under some model with a
restricted state-space merely by creating the required state-space at
whatever grid resolution is desired, and then feed that state-space
into the likelihood evaluation above. We can easily create an explicit
state-space grid for integration from arbitrary polygons or GIS
shapefiles \index{shapefile} which we 
demonstrate here. Our approach here is to create the integration grid
(or state-space grid) outside of the likelihood evaluation, and then
determine which points of the grid lie in the polygon defined by the
shapefile using 
functions in the {\bf R} packages \mbox{\tt sp} \index{R
  package!sp} and
\mbox{\tt maptools} \index{R package!maptools} \index{maptools}.  Here
are the {\bf R} commands for doing this:  
{\small
\begin{verbatim}
library(maptools}
library(sp)
SSp<-readShapeSpatial('Sim_Polygon.shp')
Pcoord<-SpatialPoints(G)
PinPoly<-over(Pcoord,SSp)
Pin<-as.numeric(!is.na(PinPoly[,1]))
G<-G[Pin==1,]
\end{verbatim}
}
We created  the function \mbox{\tt intlik4} which accepts the integration
grid as an explicit argument, and this function is also available in
the package  \mbox{\tt scrbook}.

We apply this modification to the wolverine camera trapping
study. \citet{royle_etal:2011jwm} created 2, 4 and 8 km state-space
grids so as to remove ``non-habitat'' (mostly ocean, bayes, and large
lakes). We previously analyzed the model using {\bf JAGS} and {\bf WinBUGS} in
Chapt. \ref{chapt.scr0}.  To set up the wolverine data and fit the
model we execute the following commands
{\small 
\begin{verbatim}
library("scrbook")
data("wolverine")

traps<-wolverine$wtraps
traplocs<-traps[,1:2]/10000
K.wolv<-apply(traps[,3:ncol(traps)],1,sum)
traps<-cbind(1:nrow(traps),traps)  # pad with an ID variable
y3d<-SCR23darray.fn(wolverine$wcaps,traps)
y2d<-apply(y3d,c(1,3),sum)
G<-wolverine$grid2/10000

starts<-c(-1.5,1.2,log(4))
frog<-nlm(intlik4,starts,hessian=TRUE,y=y2d,K=K.wolv,X=traplocs,G=G)

frog
$minimum
[1] 225.8355

$estimate
[1] -2.995541  1.265021  4.110476

$gradient
[1]  3.808485e-05 -9.930579e-06  3.906668e-06

$hessian
           [,1]       [,2]      [,3]
[1,]  47.059393 -21.415124  4.406148
[2,] -21.415124  38.255192 -7.386245
[3,]   4.406148  -7.386245 15.406613

$code
[1] 1

$iterations
[1] 14
\end{verbatim}
}

Next we convert the parameter estimates to estimates of total
population size for the prescribed state-space, and then obtain an
estimate of density (per 1000
$km^2$) using the area computed as the number of pixels in the
state-space grid \mbox{\tt G} multiplied by the area per grid cell. In
the present case (the calculation above) we used a state-space grid
with $2 \times 2$ $km$ pixels.  Finally, we compute
a standard errors using the delta approximation: 
\begin{verbatim}
Nhat<- 21+exp(frog$estimate[3])
SE<-  exp(frog$estimate[3])*sqrt(solve(frog$hessian)[3,3])
D<- (Nhat/(nrow(G)*area))*1000
SE.D<- (SE/(nrow(G)*area))*1000
\end{verbatim}
We did this for each the 2 $km$, 4 $km$ and 8 $km$ state-space grids
which produced the estimates summarized in Tab. \ref{mle.tab.wolv}.
These estimates compare with the 8.6 (2 km grid) and 8.2 (8 km grid)
reported in 
\citet{royle_etal:2011jwm} based on a clipped state-space as described
in sec. \ref{scr0.sec.discrete}.

\begin{table}
\centering
\caption{MLEs for the wolverine camera trapping data using 2, 4 and 8 km state-space grids.}
\begin{tabular}{cccccccc}
\hline \hline
grid &  $\alpha_0$  &  $\alpha_1$ &   $log(n_0)$  & $N$   &  SE & D(1000) &  SE \\ \hline
2  &  -2.995541& 1.265021 &4.110476 &81.97574& 16.30904 &8.310598 &1.653391\\
4  &  -2.991268&1.344055  &4.157026 &84.88126& 16.76202 &8.570401& 1.692450\\
8   & -3.051705& 1.080083 &4.058542 &78.88983& 15.31392 &7.851296& 1.524077\\
\end{tabular}
\label{mle.tab.wolv}
\end{table}


\begin{comment}
\subsection{
Exercises
}

{\flushleft
1.	Compute the 95\% confidence interval for wolverine density,
somehow. Comment on the practical implication of this level of precision.
}

{\flushleft
2.	Compute the AIC of this model and modify \mbox{\tt intlik3}
 to consider alternative link functions (at least one additional) and
 compare the  AIC of the different models and the estimates. Comment. 
}
\end{comment}


\section{Program DENSITY and the R package \mbox{\tt secr} }
\label{mle.sec.secr}


{\bf DENSITY} is a software program developed by \citet{efford:2004}
for fitting spatial capture-recapture models based mostly on classical
maximum likelihood estimation and related inference methods.
\citet{efford:2011} has also released an {\bf R} package named
\mbox{\tt secr}, that contains much of the functionality of {\bf
  DENSITY} but also incorporates new models and features.  Here, we
will focus on \mbox{\tt secr} as it will continue to be developed,
contains more functionality and is based in {\bf R}.


 To install
and run models in \mbox{\tt secr}, you must download the package and
load it in
{\bf R}.
\begin{verbatim}
 install.packages(“secr”)
 library(secr)
\end{verbatim}
\mbox{\tt secr} allows the user to simulate data and fit a suite of models with
various detection functions and covariate responses.  \mbox{\tt secr}
uses the
standard {\bf R} model specification framework using tildes. E.g., the model
command is \mbox{\tt secr.fit} and is generally written as
\begin{verbatim}
> secr.fit(capturedata, model = list(D~1, g0~1, sigma~1), buffer = 20000)
\end{verbatim}
where we have \verb#g0~1# indicating the intercept model. 
 Possible predictors for detection probability include both
pre-defined variables (e.g., \mbox{\tt t} and \mbox{\tt b}
corresponding to ``time'' and 
``behavior''), and user-defined covariates of several kinds. 
For example, to include a behavioral response, this would be written
as \verb#g0~b#.
The discussion of covariates is developed more in Chapt. \ref{chapt.covariates}\footnote{Beth:
  does secr fit a local trap-specific response or just a global
  behavioral response?}

Before we can fit the models, the data must first be packaged properly
for 
\mbox{\tt secr}.  Two input files are required: trap layout (location and
identification information for each trap) and capture data (e.g.,
sampling session, animal identification, trap day, and trap location).
\mbox{\tt secr} requires that you specify the trap type, the two most common for
camera trapping/hair snares are ‘proximity’ detectors and ‘count’
detectors.  The `proximity' detector type allows, at most, one
detection of each individual at a particular detector on any occasion
(i.e., it is equivalent to the Bernoulli or binomial encounter process
model).
The ‘count’ detector designation allows repeat encounters of each
individual at a particular detector on any occasion.  There are other
detector types that one can select such as: `polygon' detector type
which allows for a trap to be a sampled polygon, e.g., scat surveys,
and 'signal' detector which allows for traps that have a strength
indicator, e.g., acoustic arrays.  The detector types ‘single’ and
‘multi’ can be confusing as ‘multi’ seems like it would appropriate
for something like a camera trap, but instead these two designations
refer to traps that retain individuals, thus precluding the ability
for animals to be captured in other traps during the sampling
occasion.  The ‘single’ type indicates trap that can only catch one
animal at a time, while ‘multi’ indicates traps that may catch more
than one animal at a time.  For a full review of the detector types,
one should look at the help manual, which can be accessed in {\bf R} after
installing the \mbox{\tt secr} package by using the command:
\begin{verbatim}
 RShowDoc("secr-manual", package = "secr")
\end{verbatim}
As with all of the SCR models, \mbox{\tt secr} fits a detection function relating
the probability of detection to the distance of a detector from an
individual activity center. \mbox{\tt secr} allows the user to specify one of a
variety of detection functions including the commonly used
half-normal, hazard rate, and exponential.  There are 12 different
functions, but some are only available for simulating data, and one
should take caution when using different detection functions as the
interpretation of the parameters, such as $\sigma$, may not be consistent
across formulations.  The different detection functions are defined in
the \mbox{\tt secr} manual and can be found by calling the help function for the
detection function:
\begin{verbatim}
 ?detectfn
\end{verbatim}
It is useful to note that \mbox{\tt secr} requires the buffer distance to be
defined in meters and density will be returned as number of animals
per hectare.  Thus to make comparisons between \mbox{\tt secr} and other models,
we will often have to convert the density to the same units.  Also,
note that $\sigma$ is returned in units of meters.

\footnote{One question: SECR only ever reports “sigma”. What exactly is sigma?  It is a scale parameter of a detection function and all detection functions have a scale parameter. But in what sense is this sigma parameter related to “home range diameter”?  Efford doesn’t explain this, does he?  In some sections in chapter 4 or possibly 6 we get into this issue. 
}

\subsection{ Analysis using the \mbox{\tt secr} package}

To demonstrate the use of the \mbox{\tt secr} package, we will show how to do the
same analysis on the wolverine study as shown in section 4.6.  To use
the \mbox{\tt secr} package, the data need to be formatted in a similar but
slightly different manner than we use in {\bf
  WinBUGS}\footnote{Elaborate on this point -- and how is this
  different than introduced in chapter 4?}.  After installing
the \mbox{\tt secr} package, we first have to read in the trap locations and
other related information, such as if the trap is operational during a
sampling occasion.  The \mbox{\tt secr} package reads in the trap data through a
command called ``\mbox{\tt read.traps}'', which requires the detector type as
input.  The detector type is important because it will determine the
likelihood that \mbox{\tt secr} will use to fit the model.  Here, we have
selected ``proximity'' which corresponds to the Bernoulli encounter
model in which individuals are captured at most once in
each trap during each sampling occasion:
{\small
\begin{verbatim}
library("secr")
library("scrbook")
data("wolverine")
traps<-wolverine$wtraps

traps<-as.matrix(traps[,1:3])
dimnames(traps)<-list(NULL,c("trapID","x","y"))
traps<-as.data.frame(traps)
trapfile<-read.traps(data=traps,detector="proximity")
\end{verbatim}
}
Here we note that trap coordinates are extracted from the wolverine
data but we do {\it not} standardize them here. This is because
\mbox{\tt secr} defaults to coordinate scaling of meters which is the
standard scaling for the wolverine data. 

After reading in the data, we now need to create the encounter matrix
or array.  The \mbox{\tt secr} package does this through the use of the
\mbox{\tt make.capthist} command, where we provide the capture histories in raw
data format (each line contains the session, identification number,
occasion, and trap id for only 1 individual).  This is the format that
was shown in the data input file ``\mbox{\tt wcaps}'', and we only need a line or
two to organize the data into the order that the make.capthist command
wants.  In creating the capture history, we provide also the trapfile
with the trap information, and the format (e.g., here \mbox{\tt fmt= ``trapID''})
so that \mbox{\tt secr} knows how to match the encounters to the trap, and
finally, we provide the number of occasions:
{\small 
\begin{verbatim}
wolv.dat<-wolverine$wcaps[,c(2,3,1)]  # different order than SCR!!!
wolv.dat<- cbind(rep(1,nrow(wolv.dat)),wolv.dat)
dimnames(wolv.dat)<-list(NULL,c("Session","ID","Occasion","trapID"))
wolv.dat<-as.data.frame(wolv.dat)
wolvcapt<-make.capthist(wolv.dat,trapfile,fmt="trapID",noccasions=165)
\end{verbatim}
}
The function  \mbox{\tt secr.fit} will fit the model. We are using the
basic model (SCR0), so we do not need to make any specifications in
the command line except for the providing the buffer size (in $m$).  To
specify different models, you can change the default
\verb#D~1, g0~1, sigma~1#, which the interested reader can do with
very little difficulty.

{\small
\begin{verbatim}
 wolv.secr<-secr.fit(wolvcapt,model=list(D~1, g0~1, sigma~1), buffer=20000)

 wolv.secr

secr.fit( capthist = wolvcapt, model = list(D ~ 1, g0 ~ 1, sigma ~ 1), buffer = 20000 )
secr 2.3.1, 14:20:38 05 Mar 2012

Detector type     proximity 
Detector number   37 
Average spacing   4415.693 m 
x-range           593498 652294 m 
y-range           6296796 6361803 m 
N animals       :  21  
N detections    :  115 
N occasions     :  165 
Mask area       :  1037069 ha 

Model           :  D~1 g0~1 sigma~1 
Fixed (real)    :  none 
Detection fn    :  halfnormal 
Distribution    :  poisson 
N parameters    :  3 
Log likelihood  :  -746.754 
AIC             :  1499.508 
AICc            :  1500.92 

Beta parameters (coefficients) 
           beta    SE.beta        lcl       ucl
D     -9.749576 0.23027860 -10.200913 -9.298238
g0    -4.275735 0.15846099  -4.586313 -3.965158
sigma  8.699202 0.07868942   8.544974  8.853430

Variance-covariance matrix of beta parameters 
                  D            g0        sigma
D      0.0530282320  0.0005468918 -0.005226919
g0     0.0005468918  0.0251098856 -0.005885208
sigma -0.0052269186 -0.0058852077  0.006192025

Fitted (real) parameters evaluated at base levels of covariates 
       link     estimate  SE.estimate          lcl          ucl
D       log 5.831941e-05 1.360973e-05 3.713638e-05 9.158548e-05
g0    logit 1.371121e-02 2.142902e-03 1.008756e-02 1.861207e-02
sigma   log 5.998123e+03 4.727205e+02 5.140849e+03 6.998355e+03
\end{verbatim}
}

Under the fitted (real) parameters, we find $D$, the density, given in
units of individuals/hectare (1 hectare = 10000 $m^2$).  To convert this
into individuals/1000 $km^2$, we multiply by 100000, thus our density
estimate is 5.83 individuals/1000 $km^2$.  $\sigma$ is given in units of
meters, to convert to kilometers, we divide by 1000, which puts sigma
at 5.99 $km$.  Both of these estimates are very similar to those
provided in sec. \ref{scr0.sec.wolverine} for the buffer size equal to
20 $km$\footnote{How come the MLES are different from what I computed
  above?  What is sigma from back there? That section is missing
  I think}.


\begin{comment}
As an
exercise, run this analysis for 30 and 40 km buffers and compare those
found in section 4.6 under {\bf WinBUGS}.  
NOTE: The function \mbox{\tt
  secr.fit} 
will return a
warning when the buffer size appears to be too small.  This is useful
particularly with the different units being used between programs and
packages.
\end{comment}

\subsection{Analysis of Efford's Possum Data}



Do a secr analysis of the possum data set to follow up on chapter 5 material.......

Maybe just insert the analysis from the ?secr page?

Use the mask he provides (show picture)
Use rectangular mask.

Cite above material on state-space grid. Use our likelihood function
with his state-space grid.

Secr + grid
secr + no grid
my likelihood + grid

Note: should not compare AIC across analysis platforms because the
likelihoods can be scaled arbitrarily -- depending on what to leave in
or leave out.



\section{Summary and Outlook}

In this chapter, we showed that classical analysis of SCR models based
on likelihood methods is a relatively simple proposition.  Analysis is
based on the so-called integrated likelihood in which the individual
activity centers (random effects) are removed from the
conditional-on-{\bf s} likelihood by integration. We showed how to construct
the integrated likelihood and fit some simple models in the {\bf R}
programming language.  In addition, likelihood analysis for some broad
classes of SCR models can be accomplished in the software package
{\bf DENSITY} 
or the {\bf R}
library \mbox{\tt secr} which we provided an illustration of here. In later
chapters we provide more detailed analyses of SCR data likelihood
methods and the
\mbox{\tt secr}
package.

To compute the marginal (integrated) likelihood we have to precisely describe the
state-space of the underlying point process. In practice, this leads
to a ``buffer'' around the trap array. We note that this is not really a
``buffer strip'' in the sense of \citet{wilson_anderson:1985a},  
but it is somewhat more general here. In particular,
it establishes the support of the integrand and, 
in SCR models, it is an element of the model that
provides an explicit
linkage between population size $N$ and density $D$.
As a practical 
matter, it will typically be the case that, while estimates of $N$
increase with the area of the state-space (as they should!), estimates of density
stabilize. This is not a feature of the classical methods based on
using model $M_0$ or model $M_h$ and buffering the trap array.

Why or why not use likelihood inference exclusively? For certain
specific models, it is probably more computationally efficient to
produce MLEs (e.g., see Chapt. \ref{chapt.ecoldist}). However, {\bf BUGS} is extremely flexible in terms of
describing models, although it sometimes can be quite slow. We can
devise models in the {\bf BUGS} language easily that we cannot fit in
\mbox{\tt secr}. E.g.,
random individual effects of various types
(Chapt. \ref{chapt.covariates}), we can 
handle missing covariates in complete generality and seamlessly, and
impose arbitrary distributions on random variables. Moreover, models
can easily be adapted to include auxiliary data types. For example, we
might have camera trapping and genetic data and we can describe the
models directly in {\bf BUGS} and fit a joint model. For the MLE we have
to write a custom new piece of code for each model or hope someone has
done it for us.  Later we consider open population models which are
straightforward to develop in {\bf BUGS} but, so far, there is no
available platform for doing MLE although we imagine one could develop
this.  On
the other hand, likelihood analysis makes it easy to do
model-selection by AIC and in some cases compute standard errors or
carry-out goodness-of-fit evaluations. 
\begin{comment}
Another thing that is more conceptual here is non-CSR point
processes (Chapt. \ref{chapt.state-space}) and generating predictions of how many
individuals have home range centers in any particular polygon.  Basic
benefits of Bayesian analysis have been discussed elsewhere (XXXXXXXX Chapter
2? BPA book? Link and Barker?) and we believe these are compelling.
\end{comment}

In summary, basic SCR models are easy to implement by either
likelihood or Bayesian methods but some users will 
realize much more flexibility in model development using existing
platforms for Bayesian analysis. While these tend to be slow
(sometimes excruciatingly slow), this will probably not be an
impediment in most problems, especially at some near point in the
future.  Since we spent a lot of time here talking about specific
technical details on how to implement likelihood analysis of SCR
models, we provided a corresponding treatment in the next chapter on
how to devise MCMC algorithms for SCR models. This is a bit more
tedious and requires more coding, but is not technically challenging
(accept perhaps to develop highly efficient algorithms which we don’t
excel at).





\chapter{
MCMC for Spatial Capture-Recapture
}
\markboth{MCMC}{}
\label{chapt.mcmc}

%%% NOTES
%%% Andy's working through this doing format edits mostly and math
%%% stuff , but not in order
%%% anytime you see a XXX or XYZ that is a marker to change some
%%% hard-wired reference to a float

\vspace{.3in}

\section{Introduction}
In this chapter we will dive a little deeper into Markov chain Monte
Carlo (MCMC) sampling. We will construct custom MCMC samplers in {\bf R},
starting with easy-to-code GLMs and GLMMs and moving on to simple CR and SCR
models. Finally, we will illustrate some alternative
ready-to-use software packages for MCMC sampling. We will NOT provide
exhaustive background information on the theory and justification of
MCMC sampling – there are entire books dedicated to that subject and
we refer you to \citet{robert_casella:2004} and
\citet{robert_casella:2010}. Rather we aim to provide you with enough
background and technical know-how to start building your own MCMC
samplers for SCR models in {\bf R}. You will find that quite a few topics that come up 
in this chapter have already been covered in previous chapters, particularly the introduction
into Byesian analysis in Chapt. \ref{chapt.glms}. To keep you from having to leaf back and forth
we will in some places briefly review aspects of Bayesian analysis, but we try to focus on the more 
technical issues of building MCMC samplers relevant to SCR models. 



\subsection{Why build your own MCMC algorithm?}

The standard programs we have used so far to run MCMC analyses are
{\bf WinBUGS} \citep{gilks_etal:1994} and {\bf JAGS}
\citep{plummer:2003}. The wonderful thing about these {\bf BUGS}
engines
is that they automatically use  appropriate and, most of the time,
efficient forms
of MCMC sampling for the model specified by the user.

The fact that we have such a Swiss Army knife type of MCMC machine
begs the question: Why would anyone want to build their own MCMC
algorithm? For one, there are a limited number of distributions and
functions implemented in {\bf BUGS}. While {\bf OpenBUGS} provides more
options, some more complex models may be impossible to build within
these programs. A very simple example from spatial capture-recapture
that can give you a headache in {\bf WinBUGS} is when your state-space is an
irregular-shaped polygon, rather than an ideal rectangle that can be
characterized by four pairs of coordinates. It is easy to restrict
activity centers to any arbitrary polygon in {\bf R} using an ESRI shapefile
(and we will show you an example in a little bit), but you cannot use
a shape file in a {\bf BUGS} model.  Similarly, models of space usage
that take into account ecological distance
(Chapt. \ref{chapt.ecoldist} cannot be implemented in the {\bf BUGS}
engines.  Moreover, there are classes of 
SCR models that we have not been able to implement effectively using
likelihood methods, and are inefficient to run in the {\bf BUGS}
engines. An example are those models covered in Chapts. 
\ref{chapt.scr-unmarked} and \ref{chapt.partialID}. 

Sometimes implementing an MCMC algorithm in R may be faster than in
{\bf WinBUGS} - especially if you want to run simulation studies where you
have hundreds or more simulated data sets, several years' worth of
data or other large models, this can be a big advantage.

Finally, building your own MCMC algorithm is a great exercise to
understand how MCMC sampling works. So while using the {\bf BUGS} language requires you to understand the structure of your model, building an MCMC algorithm requires you to think about the relationship between your data, priors and posteriors, and how these can be efficiently analyzed and characterized. Not to mention that, if you are an R junkie, it can actually be fun.
However, if you don't think you will ever sit down and write your own
MCMC sampler, consider skipping this chapter - apart from coding it
will not cover anything SCR-related that is not covered by other, more
model-oriented chapters as well.


\section{MCMC and posterior distributions}

MCMC is a class of simulation methods for
drawing (correlated) random numbers from a target distribution, which
in Bayesian inference is the posterior distribution.
As a reminder, the posterior distribution is a probability
distribution for an unknown parameter, say $\theta$, given a set of
observed data and its prior probability distribution (the probability
distribution we assign to a parameter before we observe data).  The
great benefit of computing the posterior distribution of $\theta$ is
that it can be used to make probability statements about $\theta$,
such as the probability that $\theta$ is equal to some value, or the
probability that $\theta$ falls within some range of values. 
The posterior distribution summarizes all we know about a parameter
and thus, is the central object of interest in Bayesian
analysis. Unfortunately, in many if not most practical applications,
it is nearly impossible to directly compute the posterior. Recall
Bayes’ theorem:
\begin{equation}
p(\theta|y) = p(y|\theta) * p(\theta) / p(y),
\label{mcmc.eq.bayes}
\end{equation}
where $\theta$ is the parameter of interest, $y$ is the observed data,
$p(\theta|y)$ is the posterior, $p(y|\theta)$ the likelihood of the
data conditional on $\theta$, $p(\theta)$ the prior probability of
$\theta$, and, finally, $p(y)$ is the marginal probability of the
data, defined as 
\[
p(y) = \int p(y|\theta) * p(\theta) d\theta
\]

This marginal probability is a normalizing constant that ensures that
the posterior integrates to 1. Often, the
integral is  hard or impossible to evaluate, unless you are
dealing with a really simple model.  For example, consider 
a Normal model, with a set of $n$ observations, $y_{i};
i=1,2,\ldots,n$: 
\[
 y_{i} \sim \mbox{Normal}(\mu, \sigma),
\]
where $\sigma$ is known and our objective is to obtain an estimate of
$\mu$ using Bayesian statistics. To fully specify the model in a Bayesian
framework, we first have to define a prior distribution for $\mu$. Recall
from Chapt. \ref{chapt.glms} 
that for certain data models, certain priors lead to
conjugacy – i.e. if you choose a certain prior for your parameter,
your posterior distribution will be of a known parametric form. The
conjugate prior for the mean of a normal model is also a Normal
distribution:
\[
\mu \sim \mbox{Normal}(\mu_0, \sigma_{0}^{2})
\]
If $\mu_{0}$ and $\sigma_{0}^{2}$ are fixed, the posterior for $\mu$
has the following form (for the algebraic proof, see XXX RED BOOK? XXXX):
\begin{equation}
\mu|y \sim \mbox{Normal}(\mu_{n}, \sigma_{n}^{2})
\label{mcmc.eq.mu-posterior}
\end{equation}
where
\[
\mu_{n} = \frac{ \sigma^{2}}  {\sigma^{2}   +n* \sigma_{0}^{2}}*  \mu_0 +      \frac{n * \sigma_{0}^{2}}  {\sigma^{2}   +n* \sigma_{0}^{2}} *\bar{y}
\]
And
\[
 \sigma_{n}^{2} = \frac{\sigma^{2}  * \sigma_{0}^{2}} {\sigma^{2} + n*\sigma_{0}^{2}}
\]
We can directly obtain estimates of interest from this Normal
posterior distribution, such as the mean $\hat{\mu}$ and its variance; we
do not need to apply MCMC, since we can recognize the posterior as a
parametric distribution, including the normalizing constant $p(y)$.
But generally we will be interested in more complex models with
several, say $n$, parameters. In this case, computing $p(y)$ from
Eq. \ref{mcmc.eq.bayes} requires $n$-dimensional integration, which is
can be difficult or impossible. Thus, the posterior distribution in
generally only known up to a constant of proportionality:
\[
p(\theta|y) \propto p(y|\theta) * p(\theta)
\]
The power of MCMC is that it allows us to approximate the posterior
using simulation without evaluating the high dimensional integrals and
to directly sample from the posterior, even when the posterior
distribution is unknown! The price is that MCMC is computationally
expensive. Although MCMC first appeared in the scientific literature
in 1949 \citep{metropolis_etal:1949}, widespread use did not occur
until the 1980s when computational power and speed increased
\citep{gelfand_smith:1990}. It is safe to say that the advent of
practical MCMC methods is the primary reason why Bayesian inference
has become so popular during the past three decades.
In a nutshell, MCMC lets us generate sequential draws of $\theta$ (the
parameter(s) of interest) from distributions approximating the unknown
posterior over $T$ iterations. The distribution of the draw at $t$ depends
on the value drawn at $t$-1; hence, the draws from a Markov
chain\footnote{In case you are not familiar with Markov chains, for
  $T$ random samples $\theta^ {(1)}$, ... $\theta^{(T)}$ from a Markov chain
  the distribution of $\theta^{(t)}$ depends only on the immediately preceding
  value, $\theta^{(t-1)}$.}. As $T$ goes to infinity, the Markov chain
converges to the desired distribution – in our case the posterior
distribution for $\theta|y$. Thus, once the Markov chain has reached
its stationary distribution, the generated samples can be used to
characterize the posterior distribution, $p(\theta|y)$, and point
estimates of $\theta$, its standard error and confidence bounds, can
be obtained directly from this approximation of the posterior. 



\section{Types of MCMC sampling}

There are several MCMC algorithms, the most popular being Gibbs
sampling and Metropolis-Hastings sampling, both of which were briefly introduced in Chapt. \ref{chapt.glms}. We will be dealing with
these two classes in more detail and use them to construct the MCMC
algorithms for SCR models. Also, we will briefly review alternative
techniques that are applicable in some situations.


\subsection{Gibbs sampling}
\label{mcmc.sec.gibbs}

Gibbs sampling was named after the physicist J.W. Gibbs by
\citet{geman_geman:1984}, who applied the algorithm to a Gibbs
distribution \footnote{a distribution from physics we are not going to
  worry about, since it has no immediate connection with Gibbs
  sampling other than giving its name}. The roots of Gibbs sampling
can be traced back to work of \citet{metropolis_ulam:1953}, and it is
actually closely related to Metropolis sampling (see Chapter 11.5 in
\citet{gelman_etal:2004}, for the link between the two samplers). We
will focus on the technical aspects of this algorithm, but if you find
yourself hungry for more background, \citet{casella_george:1992}
provide a more in-depth introduction to the Gibbs sampler.

Let's go back to our
simple example from above to understand the motivation and functioning
of Gibbs sampling. Recall that for a Normal model with known variance
and a Normal prior for $\mu$, the posterior distribution of $\mu|y$ is also
Normal. Conversely, with a fixed (known) $\mu$, but unknown variance, the
conjugate prior for $\sigma^2$ is an Inverse-Gamma distribution with shape and scale parameters $a$ and $b$:
\[
\sigma^2 \sim InvGamma(a,b),
\]
With fixed $a$ and $b$, the posterior $p(\sigma|\mu,y)$ is also an Inverse Gamma distribution, namely:
\begin{equation}
\sigma|\mu,y \sim InvGamma (a_n, b_n),
\label{eq. 3}
\end{equation}
 where  $a_n = n/2   + a$ and $b_n = (1/2) \sum (y-\mu)^2 + b$.
However, what if we know neither $\mu$ nor $\sigma$, which is probably the
more common case? The joint posterior distribution of $\mu$ and $\sigma$
now has the general structure
\[
p(\mu, \sigma|y) = \frac{p(y|\mu) p(\mu) p(\sigma)}{ \int p(y|\mu) p(\mu) p(\sigma) d\mu d\sigma }
\]
or
\[
p(\mu, \sigma|y) \propto p(y|\mu) p(\mu) p(\sigma)
\]
\begin{comment} Rahel : use of p() here might violate some convention of the
book -- I dunno. Lets think about it \end{comment}
This cannot easily be reduced to a distribution we recognize. However,
we can condition $\mu$ on $\sigma$ (i.e., we treat $\sigma$ as fixed) and remove
all terms from the joint posterior distribution that do not involve $\mu$
to construct the full conditional distribution,
\[
p(\mu|\sigma,y)  \propto p(y|\mu) p(\mu)
\]

The full conditional of $\mu$ again takes the form of the Normal
distribution shown in Eq. \ref{mcmc.eq.mu-posterior}; similarly, $p(\sigma|\mu,y)$ takes
the form of the Inverse Gamma distribution shown in
Eq. \ref{eq. 3}  – both distribution we can easily sample
from. And this is precisely what we do when using Gibbs sampling – we
break down high-dimensional problems into convenient one-dimensional
problems by constructing the full conditional distributions for each
model parameter separately; and we sample from these full
conditionals, which, if we choose conjugate priors, are known
parametric distributions.
Let's put the concept of Gibbs sampling into the MCMC framework of
generating successive samples, using our simple Normal model with
unknown $\mu$ and $\sigma$ and conjugate priors as an example. These are the
steps you need to build a Gibbs sampler:

{\flushleft {\bf Step 0:} Begin with some initial values for $\theta$, $\theta^{(0)}$.   }
In our example, we have to specify initial values for $\mu$ and $\sigma$, for
example by drawing a random number from some uniform distribution, or
by setting them close to what we think they might be. (Note: This step
is required in any MCMC sampling – chains have to start from
somewhere. We will get back to these technical details a little
later.)
{\flushleft {\bf Step 1:} Draw $\theta^{(1)}$ from the conditional distribution p($\theta_{1}^{(1)}|\theta_{2}^{(0)}$,\ldots, $\theta_{d}^{(0)}$). }
Here, $\theta_1$ is $\mu$, which we draw from the Normal distribution in Eq. \ref{mcmc.eq.mu-posterior}  using $\sigma^{(0)}$ as value for $\sigma$.
{\flushleft {\bf Step 2:} Draw $\theta_{2}^{(1)}$ from the conditional distribution p($\theta_{2}^{(1)}|\theta_{1}^{(1)}$, $\theta_{3}^{(0)}$,\ldots, $\theta_{d}^{(0)}$). }
Here, $\theta_2$ is $\sigma$, which we draw from the Inverse Gamma
distribution of Eq. \ref{eq. 3}, using $\mu^{(1)}$ as value for $\mu$.

{\flushleft {\bf Step 3:} Draw $\theta_{d}^{(1)}$ from the conditional distribution p($\theta_{d}^{(1)}|\theta_{1}^{(1)}$,\ldots, $\theta_{d-1}^{(1)}$). }

In our example we have no additional parameters, so we only need step 0 through to 2.
Repeat Steps 1 to d for $T$ = a large number of samples.
In terms of {\bf R} coding, this means we have to write Gibbs updaters for
$\mu$ and $\sigma$ and embed them into a loop over $T$ iterations. The final
code in the form of an {\bf R} function is shown 
in Panel \ref{mcmc.panel.gibbs1}.


\begin{panel}[htp]
\centering
\rule[0.15in]{\textwidth}{.03in}
%\begin{minipage}{2.5in}
\begin{verbatim}
Normal.Gibbs<-function(y=y,mu0=mu0, sig0=sig0, a=a,b=b,niter=niter) {

ybar<-mean(y)
n<-length(y)
mu<-runif(1) #mean initial value
sig<-runif(1) #sd initial value
an<-n/2 + a

out<-matrix(nrow=niter, ncol=2)
colnames(out)<-c('mu', 'sig')

for (i in 1:niter) {

#update mu
mun<- (sig/(sig+n*sig0))*mu0 + (n*sig0/(sig+n* sig0))*ybar
sign <- (sig*sig0)/ (sig+n*sig0)
mu<-rnorm(1,mun, sqrt(sign))

#update sig
bn<- 0.5 * (sum((y-mu)^2)) +b
sig<-1/rgamma(1,shape=an, rate=bn)
out[i,]<-c(mu,sqrt(sig))

}
return(out)
}
\end{verbatim}
%\end{minipage}
\rule[-0.15in]{\textwidth}{.03in}
\caption{
R-code for a Gibbs sampler for a Normal model with unknown mu
and sig and conjugate (Normal and Inverse Gamma, respectively) priors
for both parameters.
}
\label{mcmc.panel.m0}
\end{panel}













This is it! You can use the code \mbox{\tt NormalGibbs.R} in the {\bf
  R} package \mbox{\tt scrbook}
to simulate some data, $y \sim \mbox{Normal}(5, 0.5)$ and run your first
Gibbs sampler. Your output will be a table with two columns, one per
parameter, and $T$ rows, one per iteration. For this 2-parameter example
you can visualize the joint posterior by plotting samples of $\mu$
against samples of $\sigma$ (Fig. \ref{postdist.fig}):
\begin{verbatim}
plot(out[,1], out[,2])
\end{verbatim}
The marginal distribution of each parameter is approximated by just
examining the samples of this particular parameter – you can visualize
it by plotting a histogram of the samples (Fig. \ref{plotsofPD.fig} a and b):
\begin{verbatim}
par(mfrow=c(1,2))
hist(out[,1]); hist (out[,2])
\end{verbatim}

\begin{comment}
Rahel: for some analyses you might want to use a specified random
number seed so that the reader can obtain exactly these results (or
that Figures look exactly the same if they are reanalyzed for
revisions. 
\end{comment}
Finally, recall an important characteristic of Markov chains, namely,
that the chain has to have converged (reached its stationary
distribution) in order to regard samples as coming from the posterior distribution. In
practice, that means you have to throw out some of the initial samples
– called the burn-in. We will talk about this in more when we talk
about convergence diagnostics. For now, you can use the
\verb#plot(out[,1])# or \verb#plot(out[,2])# command to make a time
series plot of the samples of each parameter and visually assess how
many of the initial samples you should discard. Fig. \ref{plotsofPD.fig} c and d shows
plots for the estimates of $\mu$ and $\sigma$ from our simulated data set;
you see that in this simple example the Markov chain apparently
reaches its stationary distribution very quickly – the chains look
'grassy' seemingly from the start. It is hard to discern a burn-in
phase visually (but we will see examples further on where the burn-in
is clearer) and you may just discard the first 500 draws to be sure
you only use samples from the posterior distribution. The mean of the
remaining samples are your estimates of mu and sig:
\begin{verbatim}
> summary(mod[501:10000,])
       mu                      sig
 Min.   : 4.936      Min.   : 0.4569
 1st Qu.: 4.984      1st Qu.: 0.4889
 Median : 4.994      Median : 0.4961
 Mean   : 4.994      Mean   : 0.4964
 3rd Qu.: 5.005      3rd Qu.: 0.5037
 Max.   : 5.062      Max.   : 0.5356
\end{verbatim}

\begin{figure}
\begin{center}
\includegraphics[height=3in]{Ch7/figs/postdist}
\end{center}
\caption{Joint posterior distribution of mu and sig from a Normal Model}
\label{postdist.fig}
\end{figure}

\begin{figure}
\begin{center}
\includegraphics[width=2.5in]{Ch7/figs/plotsofPD}
\end{center}
\caption{
Plots of the posterior distributions of $\mu$ (panel a) and
  $\sigma$ (b)
  from a Normal model and time series plots of $\mu$ (c) and $\sigma$ (d).}
\label{plotsofPD.fig}
\end{figure}
\begin{comment} 
Rahel: there is no (a) (b) etc.. in the figure.
might want to remake figure and label (a) (b) or else say ``upper
left'' and so on
\end{comment}

\subsection{ Metropolis-Hastings sampling   }

Although it is applicable to a wide range of problems, the limitations
of Gibbs sampling are immediately obvious – what if we do not want to
use conjugate priors (or what if we cannot recognize the full
conditional distribution as a parametric distribution, or simply do
not want to worry about these issues)? The most general solution is to
use the Metropolis-Hastings (MH) algorithm, which also goes back to
the work by \citet{metropolis_ulam:1953}. You saw the basics of this
algorithm in Chapt. \ref{chapt.glms}. In a nutshell, because we do not recognize the
posterior $p(\theta|y)$ as a parametric distribution, the MH algorithm
generates samples from a known proposal distribution, say $h(\theta)$,
that depends on $\theta$ at $t-1$. The $t^{th}$ sample is accepted with probability. 

\[
r = \frac{ f(\theta^{(t-1)}) h(\theta^{(t)}|\theta^{(t-1)})}
    {f(\theta^{(t)}) h(\theta^{(t-1)}|\theta^{(t)}) }
\]

Proposal distributions can be absolutely
anything!  You can generate candidate values from a $normal(0,1)$
distribution, from a uniform(-3455,3455) distribution, or anything of
proper support.  Note, however, that good choices of $h()$ are those
that approximate the posterior distribution. Obviously if $h() =
f(\theta|y)$ (i.e., the posterior) then you always accept the draw,
and it stands to reason that proposals that are more similar to
$f(\theta|y)$ will lead to higher acceptance probabilities. 

The original Metropolis algorithm
required $h(\theta)$ to be symmetric so that
$h(\theta^{(t)}|\theta^{(t-1)}) = h(\theta^{(t-1)}|\theta^{(t)})$. 
In that case these two terms just cancel
out from the MH acceptance probability and $r$ is then just the ratio
of the target density evaluated at the candidate value to that
evaluated at the current value. A later
development of the algorithm by \citet{hastings:1970} lifted this
condition. 
Since using a symmetric proposal distribution makes life a little
easier, we are going to focus on this specific case. A type of symmetric proposal useful in many situations is the
so-called {\it random-walk} proposal distribution where candidate values
are drawn from a normal distribution with mean equal to the current
value and some standard deviation, say $\delta$, which is prescribed by
the user (see below for further explanation). 

{\bf Parameters with bounded support}: Many models contain parameters that
have  bounded support. E.g., variance parameters live on $[0,\infty]$,
parameters that represent probabilities live on $[0,1]$, etc..
For such cases, it is sometimes convenient to use a random
walk proposal distribution that can generate any real number (e.g., a
normal random walk proposal). Then, 
we can just reject parameters that are
outside of the parameter space (XXXX REF FOR THIS MAYBE ROBERT AND
CASELLA BOOK ???? XXXX).

It is worth
knowing that there are alternatives to the random walk MH algorithm. For
example, in the independent M-H, $\theta^{(t)}$ does not depend on
$\theta^{(t-1)}$, while the Langevin algorithm \citep{roberts_etal:1998}
aims at avoiding the random walk by favoring moves towards regions of
higher posterior probability density. The interested reader should
look up these algorithms in \citet{robert_casella:2004} or
\citet{robert_casella:2010}.

Building a MH sampler can be broken down into several steps. We are going to demonstrate these steps using a different but still simple and common model – the logit-normal or logistic regression model. For simplicity, assume that
\[
y \sim \mbox{Bern}(\exp(\theta)/(1+ exp(\theta)))
\]
and
\[
\theta \sim \mbox{Normal}(\mu, \sigma)
\]
The following steps are required to set up a random walk MH algorithm:

{\flushleft {\bf Step 0}: Choose initial values, $\theta^{(0)}$.}

{\flushleft {\bf Step 1}: Generate a proposed value of $\theta$ at $t$ from $h(\theta^{(t)}|\theta^{(t-1)})$. }
We often use a Normal proposal distribution, so we draw $\theta^{(1)}$ from $\mbox{Normal}(\theta^{(0)}, \delta)$, where $\delta$ is the variance of the Normal proposal distribution, the tuning parameter that we have to set.

{\flushleft {\bf Step 2}: Calculate the ratio of posterior densities for the proposed and the original value for $\theta$: }
\[
r = \frac{p(\theta^{(t)}|y)}  {p(\theta^{(t-1)}|y)}
\]
In our example,
\[
r = \frac{\mbox{Bern}(y|\theta^{(t)}) * \mbox{Normal}(\theta^{(t)}|\mu, \sigma)} {\mbox{Bern}(y|\theta^{(t-1)}) * \mbox{Normal}(\theta^{(t-1)}|\mu, \sigma)}
\]
{\bf Step 3}: Set
\begin{eqnarray*}
\theta^{(t)}  &= &   \theta^{(t)} \mbox{ with probability min(r,1)}\\
	 & = & 	\theta^{(t-1)} \mbox{ otherwise }
\end{eqnarray*}

%should work now


We can do that by drawing a random number $u$ from a
$\mbox{Unif}(0,1)$ and accept $\theta^{(t)}$ if
$u<r$.
Repeat for $t = 1,2,\ldots$ a large number of samples.
The {\bf R} code for this MH sampler is provided in Panel 2 XXXX.

\begin{panel}[htp]
\centering
\rule[0.15in]{\textwidth}{.03in}
%\begin{minipage}{2.5in}
{\small
\begin{verbatim}
Logreg.MH<-function(y=y, mu0=mu0, sig0=sig0, niter=niter) {

out<-c()

theta<-runif(1, -3,3) #initial value

for (iter in 1:niter){
theta.cand<-rnorm(1, theta, 0.2)

loglike<-sum(dbinom(y, 1, exp(theta)/(1+exp(theta)), log=TRUE))
logprior <- dnorm(theta,mu0 ,sig0, log=TRUE)
loglike.cand<-sum(dbinom(y, 1, exp(theta.cand)/(1+exp(theta.cand)), log=TRUE))
logprior.cand <- dnorm(theta.cand, mu0, sig0, log=TRUE)

if (runif(1)<exp((loglike.cand+logprior.cand)-(loglike+logprior))){
theta<-theta.cand
}
out[iter]<-theta
}

return(out)
}
\end{verbatim}
}
%\end{minipage}
\rule[-0.15in]{\textwidth}{.03in}
\caption{
{\bf R} code to run a Metropolis sampler on a simple Logit-Normal model.
}
\label{mcmc.panel.logitnormal}
\end{panel}



The reason we sum the logs of the likelihood and the prior, rather than multiplying the original values, is simply computational. The product of small probabilities can be numbers very close to 0, which computers do not handle well. Thus we add the logarithms, sum, and exponentiate to achieve the desired result. Similarly, in case you have forgotten some elementary math, $x/y = exp(log(x)-log(y))$, with the latter being favored for computational reasons.

Comparing MH sampling to Gibbs sampling, where all draws from the
conditional distribution are used, in the MH algorithm we discard a
portion of the candidate values, which inherently makes in less
efficient than Gibbs sampling – the price you pay for its increased
generality.  In Step 1 of the MH sampler we had to choose a variance,
$\delta$, for the Normal proposal distribution. Choice of the
parameters that define our candidate distribution is also referred to
as 'tuning', and it is important since adequate tuning will make your
algorithm more efficient.  $\delta$ should be chosen (a) large enough
so that each step of drawing a new proposal value for $\theta$ can
cover a reasonable distance in the parameter space, as otherwise,
mixing of the Markov chain is inefficient and chains will tend to have
strong autocorrelation; and (b) small enough so that proposal values
are not rejected too often, as otherwise the random walk will 'get
stuck' at specific values for too long.  As a rule of thumb, your
candidate value should be accepted in about 40\% of all
cases. Acceptance rates of 20 -- 80\% are probably ok, but anything
below or above may well render your algorithm inefficient (this does
not mean that it will give you wrong results – only that you will need
more iterations to converge to the posterior distribution). In
practice, tuning will require some 'trial-and-error', some common
sense and, with enough experience, some intuition. Or, one can use an adaptive phase, where the tuning parameter
is automatically adjusted until it reaches a user-defined acceptance
rate, at which point the adaptive phase ends and the actual Markov
chain begins. This is computationally a little more
advanced. \citet{link_barker:2009} discuss this in more detail. It is
important the samples drawn during the adaptive phase are discarded.
To illustrate the effects of tuning, we ran the
Metropolis-within-Gibbs algorithm in Panel \ref{mcmc.panel.logitnormal} with $\delta=0.01$,
$\delta=0.2$ and $\delta=1$. The first 150 iterations for $\theta$ are
shown in Fig. \ref{mcmc.fig.tuning}. We see that for a very small
$\delta$ (the dashed line) the burn-in is extremely slow - after 150
iterations the chain isn't even half way there, while for the other
two values of $\delta$ (solid and dotted)the burn-in phase seems to be
over after only about 10 iterations. While $\delta=0.2$ leads to
reasonably good mixing, the chain clearly gets stuck on certain values
with $\delta=1$.
%'tuning' is a new figure I made... don't know about the size specifications, just copied those from another picture. Do you set them at 
%the actual size of the figure or the size you want it to be??
 \begin{figure}
\begin{center}
\includegraphics[height=3in,width=4in]{Ch7/figs/tuning}
\end{center}
\caption{Time series plots of $\theta$ from a MH algorithm with tuning parameter  $\delta = 0.01$ (dashed line), 0.2 (solid line) and  1 (dotted line).}
\label{mcmc.fig.tuning}
\end{figure}

Other than graphically, you can easily check acceptance rates for the parameters you monitor (that are part of your output) using the rejectionRate() function of the package coda (we will talk more about this package a little later on). Do not let the term 'rejection rate' confuse you; it is simply 1 -- acceptance rate. There may be parameters – for example, individual values of a random effect or latent variables – that you do not want to save, though, and in our next example we will show you a way to monitor their acceptance rates with a few extra lines of code.



\subsection{ Metropolis-within-Gibbs }

One weakness of the MH sampler is that formulating the joint posterior when evaluating whether to accept or reject the candidate values for $\theta$ becomes increasingly complex or inefficient as the number of parameters in a model increases. As you already saw in Chapter 2, in these cases you can simply combine MH sampling and Gibbs sampling. You can use Gibbs sampling to break down your high-dimensional parameter space into easy-to-handle one-dimensional conditional distributions and use MH sampling for these conditional distributions. Better yet – if you have some conjugacy in your model, you can use the more efficient Gibbs sampling for these parameters and one-dimensional MH for all the others. You have already seen the basics of how to build both types of algorithms, so we can jump straight into an example here and build a Metropolis-within-Gibbs algorithm.

\section{ GLMMs – Poisson regression with a random effect }

Let's assume a model that gets us closer to the problem we ultimately
want to deal with - a GLMM. Here, we assume we have Poisson counts,
$y_{ij}$, from $i=1,2,\ldots,n$ plots \begin{comment} Rahel: I put
  ``n'' here because you had ``n'' used above. But I wonder if we
  should have a convention for ``number of sites'' and use that
  throughout the book?  I think we use j=1,..,J for traps .... I can't
  remember what I used in Ch. 2. but maybe i'm overthinking
  this. lets talk later. \end{comment} in $j$ different study sites, and we believe that the counts are influenced by some plot-specific covariate, $x$, but that there is also a random site effect. So our model is:
\[
y_{ij} \sim \mbox{Poisson}(\lambda_{ij})
\]
\[
\lambda_{ij} = \exp (a_j + bx_i)
\]
XXX INDEXING OF x should be ij here??? XXXXXX
Let's use Normal priors on $a$ and $b$,  \[
a_j \sim \mbox{Normal} (\mu_a, \sigma_a)
\]
and
\[
b \sim \mbox{Normal} (\mu_b, \sigma_b)
\].

Since we want to estimate the random effect in this model, we do not
specify $\mu_a$ and $\sigma_a$, but instead, estimate them as well, so we have
to specify hyperpriors for these parameters:
\begin{eqnarray*}
\mu_a  &\sim &  \mbox{Norm}(\mu_0, \sigma_0)  \\
\sigma_{a} & \sim & \mbox{InvGamma}(a_0, b_0)
\end{eqnarray*}
% In this entire section below I am unsure of the indexing of a, y and
% x. I know what I'm trying to say but I'm not sure I'm saying it
% correctly... could someone please check? So what I am trying to say,
% for example in the expression for a1 (a at j=1), is that you need
% all the yi's at j=1, and all the x at j=1.
With the model fully specified, we can compile the full conditionals,
breaking the multi-dimensional parameter space into one-dimensional
components:
%%this works but it doesn't look paritcularly pretty
\begin{comment} 
Rahel: check these eqns out , I set it using eqnarray 
\end{comment}
\begin{eqnarray*}
p(a_1|a_2,a_3,\ldots,a_j,b,{\bf y}) & \propto &   p({\bf y}_{1}|a_1,b) * p(a_1) \\
	 & \propto  &   \mbox{Poisson}({\bf y}_{1}| \exp(a_1 + bx_i)) * \mbox{Norm}(a_1|\mu_a, \sigma_a)
\end{eqnarray*}
where ${\bf y}_{1} = (y_{11},y_{21}, \ldots, y_{n1})$ is the vector of
observed counts for site $j=1$ and, in general, ${\bf y}_{j}$ is the
vector of all counts for site $j$. The other full conditionals for
each $a_{j}$ are constructed similarly:
\begin{comment} 
Rahel: I think the indexing is out of order. you
  should have ``groups'' be the first index, i.e., ``i'' in this case,
  and ``replicates'' being the 2nd index, what you have as ``j''
\end{comment}
\begin{eqnarray*}
p(a_2|a_1,a_3,\ldots,a_j,b,{\bf y}) & \propto&  p({\bf y}_{2}|a_2,b) * p(a_2) \\
	 & \propto  & \mbox{Poisson}({\bf y}_{2}| \exp(a_2 + bx_i)) * \mbox{Norm}(a_2|\mu_a, \sigma_a)
\end{eqnarray*}
and so on for all elements of ${\bf a}$. The full-conditional for $b$ is:
\begin{eqnarray*}
p(b|a,y) &\propto & p({\bf y}|a,b) * p(b) \\
	 &\propto& \mbox{Poisson}({\bf y}|exp(a + b{\bf x})) *\mbox{Norm}(b|\mu_b, \sigma_b)
\end{eqnarray*}

Finally, we need to update the hyperparameters for the random effects
vector ${\bf a}$:
\[
p(\mu_a|{\bf a}) \propto p({\bf a}|\mu_a, \sigma_a) *p(\mu_a)
\]
\[
p(\sigma_a|a) \propto p(a|\mu_a, \sigma_a) *p(\sigma_a)
\]
Since we assumed $a$ to come from a Normal distribution, the choice of priors for $\mu_a$ (Normal) and $\sigma_a$ (Inverse-Gamma) leads to the same conjugacy we observed in our initial Normal model, so that both hyperparameters can be updated using Gibbs sampling.

Now let's build the updating steps for these full conditionals. Again, for the MH steps that update $a$ and $b$ we use Normal proposal distributions with standard deviations $\delta_{a}$ and $\delta_{b}$.

First, we set the initial values $a^{(0)}$ and $b^{(0)}$. Then, starting with $a_1$, we draw $a_1^{(1)}$ from $\mbox{Norm}(a_1^{(0)}, \delta_{a})$, calculate the conditional posterior density of $a_1^{(0)}$ and $a_1^{(1)}$  and compare their ratios,
\[
r = \frac{\mbox{Poisson}({\bf y}_{1}|exp(a_1^{(1)} + b x_i)) *
  \mbox{Norm}(a_1^{(1)}|\mu_a, \sigma_a)} {\mbox{Poisson}({\bf y}_{1}|exp(a_1^{(0)} + bx_i)) * \mbox{Norm}(a_1^{(0)}|\mu_a, \sigma_a)}
\]
and accept $a_1^{(1)}$ with probability $min(r,1)$. We repeat this for all a's.

For $b$, we draw $b_1^{(1)}$ from $\mbox{Norm} (b^{(0)}, \delta_{b})$, compare the posterior densities of $b^{(0)}$ and $b^{(1)}$,
\[
r = \frac{\mbox{Poisson}({\bf y}|exp(a + b_1^{(1)}{\bf x}))
  *\mbox{Norm}(b_1^{(1)}|\mu_b, \sigma_b)} { \mbox{Poisson}({\bf
    y}|exp(a + b_1^{(0)}{\bf x})) *\mbox{Norm}(b_1^{(0)}|\mu_b, \sigma_b)},
\]
and accept $b_1^{(1)}$  with probability $min(r,1)$.

For $\mu_a$ and $\sigma_a$, we sample directly from the full conditional distributions (Eq. \ref{XX}  and Eq. \ref{XX}):
\[
\mu_a^{(1)} \sim \mbox{Norm} (\mu_n, \sigma_n)
\]
where 
\[\mu_n =  \frac{\sigma_a^{(0)}}  {\sigma_a^{(0)}   +n_a  *  \sigma_0} *  \mu_0 +  \frac{n_a * \sigma_0} {\sigma_a^{(0)}   +n_a* \sigma_0} *\bar{a}^{(1)}
\]
and 
\[
\sigma_n= \frac{\sigma_a^{(0)}  * \sigma_0 } {\sigma_a^{(0)}  + n* \sigma_0}
\]
Here, $\bar{a}$ is the current mean of the vector $\bf{a}$, which we
updated before, and $n_a$ is the length of $\bf{a}$. 
For $\sigma_a$ we use $\sigma_a^{(1)}\sim InvGamma (a_n, b_n)$,
where  $a_n = n_a/2   + a_0$, and $b_n = 0.5 ( \displaystyle\sum\limits_{j=1}^{n_a} a_j^{(1)}-\mu_a^{(1)})^2+ b_0$.
\begin{comment} Rahel: in preceeding setntence you are using ``a'' for
  parameters of the prior distribution of ``sigma'' (and also ``b'' is
  a prior parameter) -- but ``a'' and ``b'' have been previously
  defined as regression parameter.
Recommend changing all ``a'' and ``b'' in the original model to
$\alpha$ and $\beta$ (I realize a pain in the ass
\end{comment}



We repeat these steps over $T$ iterations of the MCMC algorithm.
In this example we may not want to save each individual $a$, but are only interested in their mean and standard deviation. Since these two parameters will change as soon as the value for one element in $\bf{a}$ changes, their acceptance rates will always be close to 1 and are not representative of how well your algorithm performs. To monitor the acceptance rates of parameters you do not want to save, you simply need to add a few lines of code into your updater to see how often the individual parameters are accepted. The full code for the MCMC algorithm of our Poisson GLMM in Panel 3 (XXX) shows one way how to monitor acceptance of individual $a$'s.

\begin{comment}
 Rahel: The panel below might be too big for a panel. Can you just
 have it in the text in verbatim?
\end{comment}
{\small
\begin{verbatim}
Panel 3: R code for the Metropolis-within-Gibbs sampler for
a Poisson regression with random intercepts.

Pois.reg<-function(y=y,site=site,mu0=mu0,sig0=sig0,a0=a0,b0=b0,
          mub=mub, sigb=sigb, niter=niter){

lev<-length(unique(site))     #number of sites
a<-runif(lev,-5,5)		#initial values a
b<-runif(1,0,5)			#initial value b
mua<-mean(a)
siga<-sd(a)

out<-matrix(nrow=niter, ncol=3)
colnames(out)<-c('mua','siga','b')

for (iter in 1:niter) {

#update a
aUps<-0			  #initiate counter for acceptance rate of a
for (j in 1:lev) { 	  #loop over sites
a.cand<-rnorm(1, a[j], 0.1)	#update intercepts a one at a time
loglike<- sum(dpois (y[site==j], exp(a[j] + b*x[site==j]), log=TRUE))
logprior<- dnorm(a[j], mua,siga, log=TRUE)
loglike.cand<- sum(dpois (y[site==j], exp(a.cand + b *x[site==j]), log=TRUE))
logprior.cand<- dnorm(a.cand,  mua,siga, log=TRUE)
if (runif(1)< exp((loglike.cand+logprior.cand) –(loglike+logprior))) {
a[j]<-a.cand
aUps<-aUps+1
}
}

if(iter %% 100 == 0) {  #this lets you check the acceptance rate of a at every 100th iteration
            cat("   Acceptance rates\n")
            cat("     a =", aUps/lev, "\n")
}

#update b
b.cand<-rnorm(1, b, 0.1)
avec<-rep(a, times=c(rep(10,10)))
loglike<- sum(dpois (y, exp(avec + b*x), log=TRUE))
logprior<- dnorm(b, mub,sigb, log=TRUE)
loglike.cand<- sum(dpois (y, exp(avec + b.cand *x), log=TRUE))
logprior.cand<- dunif(b.cand, mub,sigb, log=TRUE)
if (runif(1)< exp((loglike.cand+logprior.cand) – (loglike+logprior) )) {
b<-b.cand
}

#update mua using Gibbs sampling
abar<-mean(a)
mun<- (siga/(siga+lev*sig0))*mu0 + (lev*sig0/(siga+lev* sig0))*abar
sign <- (siga*sig0)/ (siga+lev*sig0)
mua<-rnorm(1,mun, sqrt(sign))

#update siga using Gibbs sampling
a0n<-lev/2 + a0
b0n<- 0.5 * (sum((a-mua)^2)) +b0
siga<-1/rgamma(1,shape=a0n, rate=b0n)

out[iter,]<-c(mua, sqrt(siga), b)

}

return(out)
}
\end{verbatim}
}

\subsection{Rejection sampling and slice sampling }

While MH and Gibbs sampling are probably the most widely applied
algorithms for posterior approximation, there are other options that
work under certain circumstances and may be more efficient when
applicable. {\bf WinBUGS} applies these algorithms and we want you to be
aware that there is more out there to approximate posterior
distributions than Gibbs and MH.  One alternative algorithm is
rejection sampling. Rejection sampling is not an MCMC method, since
each draw is independent of the others. The method can be used when
the posterior $p(\theta|y)$ is not a known parametric distribution but
can be expressed in closed form. Then, we can use a so-called envelope
function, say, $g(\theta)$, that we can easily sample from, with the
restriction that $p(\theta|y) < M * g(\theta)$. We then sample a
candidate value for $\theta$ from $g(\theta)$, calculate $r =
p(\theta|y)/M*g(\theta)$ and keep the sample with the probability
$r$. $M$ is a constant that has to be picked so that $r$ lies between
0 and 1, for example by evaluating both $p(\theta|y)$ and $g(\theta)$
at $n$ points and looking at their ratios. Rejection sampling only
works well if $g(\theta)$ is similar to $p(\theta|y)$, and packages
like {\bf WinBUGS} use adaptive rejection sampling \citep{gilks_wild:1992},
where a complex algorithm is used to fit an adequate and efficient
$g(\theta)$ based on the first few draws. 
Though efficient in some
situations, rejection sampling does not work well with
high-dimensional problems, since it becomes increasingly hard to
define a reasonable envelope function. For an example of rejection
sampling in the context of SCR models, see
Chapt. \ref{chapt.state-space}, where we use it to simulation
non-stationary point processes.  Another alternative is slice sampling
\citep{neal:2003}. In slice sampling, we sample uniformly from the
area under the plot of $p(\theta|y)$. Considering a single univariate
theta. Let's define an auxiliary variable, $U \sim \mbox{Unif}(0,
p(\theta|y))$. Then, $\theta$ can be sampled from the vertical slice
of $p(\theta|y)$ at $U$ (Fig. \ref{mcmc.fig.slicesample}):
\[
\theta|U \sim \mbox{Unif}(B),
\]
where $B = \{\theta: p(\theta|y) \geq U\}$
%do these symbols mean  'B is element of the interval of all theta for which p(theta) is larger than or equal to U?' 

\begin{figure}
\begin{center}
\includegraphics[height=2in]{Ch7/figs/slicesampling}
\end{center}
\caption{Slice sampling. For...}
\label{mcmc.fig.slicesample}
\end{figure}

\footnote{there are supposed to be equations in the caption of figure
4 but it kept causing errors. Rahel: Let me see the equations you want
in there....}

Slice sampling can be applied in many situations; however,
implementing an efficient slice sampling procedure can be
complicated. We refer the interested reader to 
\citet[][Chapt. 7]{robert_casella:2010} for a simple example.  Both rejection
sampling and slice sampling can be applied on one-dimensional
conditional distributions within a Gibbs sampling setup.

\section{MCMC for closed capture-recapture Model Mh}
\begin{comment}
\subsection{Building your own MCMC algorithm}
% subsection needed here? maybe not?
\end{comment}

By now you have seen MCMC samplers for some simple GL(M)M's. Now, to
ease you into more complex models, we construct our own MCMC algorithm
using a Metropolis-within-Gibbs sampler for the non-spatial Model with
individual heterogeneity in capture probability $M_{h}$, developed in
Chapt. \ref{chapt.closed}.

To recapitulate: Under the non-spatial model, each of the $n$ observed
individuals is either detected (1) or not (0) during each of $K$
sampling occasions. We estimate $N$ using data augmentation and have a
Bernoulli model for the zero-inflation variables $z_{i}$. The binomial
observation model is expressed conditional on the latent variables
$z_{i}$. Further, we prescribe a distribution for the capture
probability $p_{i}$. Here we assume
\[
\mathrm{logit}(p_{i}) \sim \mbox{Normal}(\mu,\sigma^2)
\]

As usual, we have to go through two general steps before we write the MCMC algorithm:
\begin{itemize}
\item[  (1)] Identify your model with all its components (including
    priors)
\item[  (2)] Recognize and express the full conditional distributions for
    all parameters
\end{itemize}
Our model components are as follows: $[y_{i}| p_{i},z_{i}]$,
$[p_{i}|\mu_{p},\sigma_{p}]$, and $[z_{i}|\psi]$
for {\it each} $i=1,2,\ldots,M$ and then prior distributions
$[\mu_{p}]$, $[\sigma_{p}]$ and $[\psi]$.
The joint posterior distribution of all unknown quantities in the model
is proportional to the joint distribution of all elements
$y_{i},p_{i},z_{i}$ and also the prior distributions of the prior parameters:
\[
\left\{ \prod_{i=1}^{M} [y_{i}|p_{i},z_{i}][p_{i}|\mu_{p},\sigma_{p}]
[z_{i}|\psi] \right\} [\mu_{p},\sigma_{p},\psi]
\]
For prior distributions, we assume that $\mu_{p},\sigma_{p}, \psi$ are
mutually independent and for $\mu_{p}$ and $\sigma_{p}$ we use
improper uniform priors, and $\psi \sim \mbox{Unif}(0,1)$.  Note that
the likelihood contribution for each individual, when conditioned on
$p_{i}$ and $z_{i}$, does not depend on $\psi$, $\mu_{p}$, or
$\sigma_{p}$.  As such, the full-conditionals for the structural
parameters $\psi$ only depends on the collection of data augmentation
variables $z_{i}$, and that for $\mu_{p}$ and $\sigma_{p}$ will only
depends on the collection of latent variables $p_{i}; i=1,2,\ldots,M$.
The full conditionals for all the unknowns are as follows:

{\bf (1)} For $p_{i}$:
\begin{eqnarray*}
[p_{i}|y_{i}, \mu_p, \sigma_{p},z_{i}=1] &\propto  &
[y_{i}|p_{i}][p_{i}|\mu_p,\sigma_{p}^{2}] \mbox{ if $z_{i}=1$ }  \\
                 &  &  [p_{i}|\mu_p,\sigma_{p}] \mbox{ if $z_{i}=0$ }
\end{eqnarray*}

{\bf (2)} for $z_{i}$:
\[
z_{i} | \cdot \propto [y_{i}|z_{i}*p_{i}] \mbox{Bern}(z_{i}|\psi)
\]

{\bf (3)} For $\mu_{p}$:
\[
[\mu_{p} | \cdot ] \sim \left\{ \prod_{i} [p_{i}| \cdot] \right\} *\mbox{const}
\]


{\bf (4)} For $\sigma_{p}$:
\[
[ \sigma_{p}|\cdot ] \sim \left\{ \prod_{i}[p_{i}| \cdot ] \right\} *\mbox{const}
\]

{\bf (5)} For $\psi$:
\[
\psi|\cdot\sim \mbox{Beta}(1 + \sum z_{i}, 1 + M - \sum z_{i})
\]


What we've done here is identify each of the full conditional
distributions in sufficient detail to toss them into our
Metropolis-Hastings algorithm. With the exception of $\psi$ which has
a convenient analytic solution -- it is a beta distribution which we
can easily sample directly. In truth, we could also sample $\mu_{p}$
and $\sigma_{p}^{2}$ directly with certain choices of prior
distributions. For example, if $\mu_{p} \sim \mbox{Norm}(0, 1000)$
then the full conditional for $\mu_{p}$ is also normal (see
sec. \ref{mcmc.sec.gibbs}), etc..
We implement an MCMC algorithm for this model in the following block
of {\bf R} code.  
\begin{comment}
Rahel: Yes you can have the script and modify to be consistent with
your stuff. I will send shortly.
%Andy, I think we should edit code so it's comparable to the rest in this chapter (eg loglik instead of lik.curr etc. 
%Should I do this in here or do you have an R script file for this code that's goint in the scrbook package? Then, I'd probably better
% edit the .R file and paste it in here so they match.
\end{comment}
\begin{verbatim}

## obtain the bear data by executing the previous data grabbing
## function

temp<-getdata()
M<-temp$M
K<-temp$K
ytot<-temp$ytot


###
### MCMC algorithm for Model Mh

out<-matrix(NA,nrow=100000,ncol=4)
dimnames(out)<-list(NULL,c("mu","sigma","psi","N"))
lp<- rnorm(M,-1,1)
p<-expit(lp)
mu<- -1
p0<-exp(mu)/(1+exp(mu))
sigma<- 1
psi<- .5
z<-rbinom(M,1,psi)
z[ytot>0]<-1

for(i in 1:100000){

### update the logit(p) parameters
lp.cand<- rnorm(M,lp,1)  # 0.5 is a tuning parameter
p.cand<-expit(lp.cand)
lik.curr<-log(dbinom(ytot,K,z*p)*dnorm(lp,mu,sigma))
lik.cand<-log(dbinom(ytot,K,z*pc)*dnorm(lpc,mu,sigma))
kp<- runif(M) < exp(lik.cand-lik.curr)
p[kp]<-pc[kp]
lp[kp]<-lpc[kp]

p0c<- rnorm(1,p0,.05)
if(p0c>0 & p0c<1){
muc<-log(p0c/(1-p0c))
lik.curr<-sum(dnorm(lp,mu,sigma,log=TRUE))
lik.cand<-sum(dnorm(lp,muc,sigma,log=TRUE))
if(runif(1)<exp(lik.cand-lik.curr)) {
 mu<-muc
 p0<-p0c
}
}

sigmac<-rnorm(1,sigma,.5)
if(sigmac>0){
lik.curr<-sum(dnorm(lp,mu,sigma,log=TRUE))
lik.cand<-sum(dnorm(lp,mu,sigmac,log=TRUE))
if(runif(1)<exp(lik.cand-lik.curr))
 sigma<-sigmac
}

### update the z[i] variables
zc<-  ifelse(z==1,0,1)  # candidate is 0 if current = 1, etc..
lik.curr<- dbinom(ytot,K,z*p)*dbinom(z,1,psi)
lik.cand<- dbinom(ytot,K,zc*p)*dbinom(zc,1,psi)
kp<- runif(M) <  (lik.cand/lik.curr)
z[kp]<- zc[kp]

psi<-rbeta(1, sum(z) + 1, M-sum(z) + 1)

out[i,]<- c(mu,sigma,psi,sum(z))
}
\end{verbatim}



{\bf Remarks}: (1) for parameters with bounded support, i.e.,
$\sigma_{p}$ and $p_{0}$, we are using a random walk candidate
generator but rejecting draws outside of the parameter space.  (2) We
mostly use Metropolis-Hastings except for the data augmentation
parameter $\psi$ which we sample directly from its full-conditional
distribution which is a beta distribution.  (3) Even the latent data
augmentation variables $z_{i}$ are updated using Metropolis-Hastings
although they too can be updated directly from their full-conditional.

\section{MCMC algorithm for the basic spatial capture-recapture model}

Conceptually, but also in terms of MCMC coding, it is only a small step from the non-spatial model Mh to a fully spatial capture-recapture model. Next, we'll walk you through the steps of building your own MCMC sampler for the basic SCR model (i.e. without any individual, site or time specific covariates) with both a Poisson and a binomial encounter process.
As usual, we will have to go through two general steps before we write the MCMC algorithm:
\begin{itemize}
\item[  (1)] Identify your model with all its components (including
    priors)
\item[  (2)] Recognize and express the full conditional distributions for
    all parameters
\end{itemize}
It is worthwhile to go through all of step 1 for an SCR model, but you
have probably seen enough of step 2 in our previous examples to get
the essence of how to express a full conditional
distribution. Therefore, we will exemplify step 2 for some parameters
and tie these examples directly to the respective R code.

{\bf Step 1 -- Identify your model}

Recall the components of the basic SCR model with a Poisson encounter process from Chapt. \ref{chapt.poisson-mn}:
We assume that individuals $i$, or rather, their activity centers
${\bf s}_i$, are uniformly distributed across the state space ${\cal S}$,
\[
{\bf s}_i  \sim \mbox{Unif}({\cal S})
\]
and that the number of times individual $i$ encounters trap $j$, $y_{ij}$, is a random Poisson variable with mean $\lambda_{ij}$,
\[
y_{ij} \sim \mbox{Poisson}(\lambda_{ij})
\]
The link between individual location, movement and trap encounter
rates is made by the assumption that $\lambda_{ij}$, is a decreasing
function of the distance between ${\bf s}_i$ and the location of $j$,
${\bf x}_{j}$, say $D_{ij} = ||{\bf s}_{i} - {\bf x}_{j}||$, of the half-normal form
\[
\lambda_{ij} =  \lambda_0  \exp(-D_{ij}^2/2\sigma^2),
\]
where $\lambda_0$ is the baseline trap encounter rate at $D_{ij}=0$ and $\sigma$ controls the shape of the half-normal function.

In order to estimate the number of ${\bf s}_i$ in ${\cal S}$, or any
subset of ${\cal S}$, $N$, we use data augmentation (sec. \ref{closed.sec.da}) and create $M-n$ all-0 encounter histories, where $n$ is the number of individuals we observed and $M$ is an arbitrary number that is larger than $N$. We estimate $N$ by summing over the auxiliary data augmentation variables, $z_i$, which is 1 if the individual is part of the population and 0 if not, and assume that $z_i$ is a random Bernoulli variable,
\[
z_{i} \sim \mbox{Bern}(\psi)
\]

To link the two model components, we modify our trap encounter model to
\[
\lambda_{ij} = \lambda_0 * exp(-D_{ij}^2/2\sigma^2) * z_{i}.
\]
The model has the following structural parameters, for which we need to specify priors:
\begin{itemize}
\item[ $\psi$:] the $\mbox{Unif}(0,1)$ is required as part of the data augmentation procedure and in general is a natural choice of an uninformative prior for a probability; note that this is equivalent to a $\mbox{Beta}(1,1)$ prior, which will come in handy later.
\item[ ${\bf s}_{i}$:] since ${\bf s}_{i}$ is a pair of coordinates it is two-dimensional and we use a uniform prior limited by the extent of our state-space over both dimensions.
\item[ $\sigma$:] we can conceive several priors for $\sigma$ but let's assume an improper prior, one that is Uniform over $(-\infty, \infty)$. We will see why this is convenient when we construct the full conditionals for $\sigma$.
\item[ $\lambda_{0}$:] analogous, we will use a $\mbox{Unif}(-\infty, \infty)$ improper prior for $\lambda_{0}$.
\end{itemize}
The parameter that is the objective of our modeling, $N$, is a derived parameter that we can simply obtain by summing all $z_i$:
\[
N = \sum_{i=1}^{M} z_{i}
\]

{\bf Step 2 -- Construct the full conditionals:}
Having completed step 1, let's look at the full conditional distributions for some of these parameters.
We find that with improper priors, full conditionals are proportional only to the likelihood of the observations; for example, take the movement parameter $\sigma$:
\[
[\sigma|{\bf s}, \lambda_{0}, {\bf z}, {\bf y}] \propto \left\{ \prod_{i} [y_{i}| {\bf
    s}_{i}, \lambda_{0}, z_{i}, \sigma] \right\} * [\sigma]
\]
Since the improper prior implies that $[\sigma] \propto 1$, we can reduce this further to
\[
[\sigma|{\bf s}, \lambda_{0}, {\bf z}, {\bf y}] \propto \left\{
  \prod_{i} [y_{i}| {\bf s}_{i}, \lambda_{0}, z_{i}, \sigma] \right\}
\]
The {\bf R} code to update $\sigma$ is shown in Panel
\ref{mcmc.panel.updatesigma}.
 Notice that we automatically reject negative candidate values, since $\sigma$ cannot be $<0$.  

\begin{panel}[htp]
\centering
\rule[0.15in]{\textwidth}{.03in}
%\begin{minipage}{2.5in}
{\small
\begin{verbatim}
sig.cand <- rnorm(1, sigma, 0.1)	#draw candidate value
 if(sig.cand>0){   #automatically reject sig.cand that are <0
     lam.cand <- lam0*exp(-(D*D)/(2*sig.cand*sig.cand))
     ll<- sum(dpois(y, lam*z, log=TRUE))
     llcand <- sum(dpois(y, lam.cand*z, log=TRUE))
     if(runif(1) < exp( llcand  - ll) ){
         ll<-llcand
         lam<-lam.cand
         sigma<-sig.cand
      }
  }
\end{verbatim}
}
%\end{minipage}
\rule[-0.15in]{\textwidth}{.03in}
\caption{
{\bf R} code to update sigma within an MCMC algorithm for
an SCR model when using an improper prior
}
\label{mcmc.panel.updatesigma}
\end{panel}


These steps are analogous for  $\lambda_{0}$ and ${\bf s}_i$ and we will 
use MH steps for
all of these parameters. Similar to the random intercepts in our
Poisson GLMM, we update each ${\bf s}_i$ individually. Note that to be fully
correct, the full conditional for ${\bf s}_i$ contains both the likelihood and
prior component, since we did not specify an improper, but a Uniform
prior on ${\bf s}_i$. However, with a Uniform distribution the probability
density of any value is 1/(upper limit - lower limit) =
constant. Thus, the prior components are identical for both the
current and the candidate value and can be ignored (formally, when you
calculate the ratio of posterior densities, $r$, the identical prior
component appears both in the numerator and denominator, so that they
cancel each other out).

We still have to update $z_i$. The full conditional for $z_i$ is
\[
[z_i|y_{i}, \sigma, \lambda_0, s] \propto [y_{i}|z_{i},\sigma, \lambda_0, 
{\bf s}_{i}] * [z_i]
\]
and since $z_i \sim Bernoulli(\psi)$,
the term has to be taken into account when updating $z_i$. The 
{\bf R} code for updating $z_i$ is shown in Panel \ref{mcmc.panel.updatez}.


\begin{panel}[htp]
\centering
\rule[0.15in]{\textwidth}{.03in}
%\begin{minipage}{2.5in}
{\small
\begin{verbatim}
        zUps <- 0		#set counter to monitor acceptance rate
        for(i in 1:M) {
            if(seen[i])	#no need to update seen individuals, since their z =1
                next
            zcand <- ifelse(z[i]==0, 1, 0)
            llz <- sum(dpois(y[i,],lam[i,]*z[i], log=TRUE))
            llcand <- sum(dpois(y[i,], lam[i,]*zcand, log=TRUE))

            prior <- dbinom(z[i], 1, psi, log=TRUE)
            prior.cand <- dbinom(zcand, 1, psi, log=TRUE)
            if(runif(1) < exp( (llcand+prior.cand) - (llz+prior) )) {
                z[i] <- zcand
                zUps <- zUps+1
            }
        }
\end{verbatim}
}
%\end{minipage}
\rule[-0.15in]{\textwidth}{.03in}
\caption{
{\bf R} code to update z................
}
\label{mcmc.panel.updatez}
\end{panel}


$\psi$
 is a hyperparameter of the model, with an uninformative prior 
 distribution of $\mbox{Unif}(0,1)$ or $\mbox{Beta}(1,1)$, so that
\[
\psi|{\bf z} \propto \left\{ \prod_{i} [z_{i}|\psi] \right\} \mbox{Beta}(1,1)
\]
The beta distribution is the conjugate prior to the binomial and 
Bernoulli distributions (remember that $z \sim \mbox{Bern}(\psi))$. 
The general form of a full conditional of a Beta-Binomial model 
with $y_{i} \sim \mbox{Bern} (p) $ and $p \sim \mbox{Beta}(a,b)$ is
\[
p(p|y) \propto \mbox{Beta}(a + \sum y_i, b + n-\sum y_i)
\]
\begin{comment} Rahel: above is why i don't like p for distribution! \end{comment}
In our case, this means we update $\psi$ as follows:
\begin{verbatim}
si<-rbeta(1, 1+sum(z), 1 + M-sum(z))
\end{verbatim}
\begin{comment} Rahel what is M here  ? \end{comment}
These are all the building blocks you need to write the MCMC algorithm
for the spatial null model with a Poisson encounter process.  You can
find the full {\bf R} code (\mbox{\tt SCR0pois.R}) in the {\bf R} package 
\mbox{\tt scrbook}.

\subsection{SCR model with binomial encounter process}
The equivalent SCR model with a binomial encounter process is very similar. Here, each individual $i$ can only be detected once at any given trap $j$ during a sampling occasion $k$.
Thus
\[
y_{ij} \sim \mbox{Bin} (p_{ij}, K)
\]
Where $p_{ij}$ is some function of distance between ${\bf s}_{i}$ and trap location ${\bf x}_{j}$. Here we use:
\[
p_{ij}=1-exp(-\lambda_{ij})
\]
Recall from Chapter 2 that this is the complementary log-log (cloglog) link function, which constrains $p_{ij}$ 
to fall between 0 and 1.
For our MCMC algorithm that means that, instead of using a Poisson 
likelihood, $\mbox{Poisson}(y|\sigma,\lambda_0,{\bf s},z)$, we use a 
Binomial likelihood, $\mbox{Bin}(y| \sigma,\lambda_0,{\bf s},z; K)$, 
in all the conditional distributions. As an example, Panel
\ref{mcmc.panel.updatelam0} shows 
the updating step for $\lambda_0$ under a binomial encounter model. 
The full MCMC code for the binomial SCR (\mbox{\tt SCR0binom.R}) 
can be found in the {\bf R} package \mbox{\tt scrbook}.


\begin{panel}[htp]
\centering
\rule[0.15in]{\textwidth}{.03in}
%\begin{minipage}{2.5in}
{\small
\begin{verbatim}

        lam0.cand <- rnorm(1, lam0, 0.1)
        if(lam0.cand >0){   #automatically reject lam0.cand that are <0
            lam.cand <- lam0.cand*exp(-(D*D)/(2*sigma*sigma))
            p.cand <- 1-exp(-lam.cand)
            ll<- sum(dbinom(y, K, pmat *z, log=TRUE))
            llcand <- sum(dbinom(y, K, p.cand *z, log=TRUE))
            if(runif(1) < exp( llcand  - ll) ){
                ll<-llcand
                pmat<-p.cand
                lam0<- lam0.cand
            }
        }
\end{verbatim}
}
%\end{minipage}
\rule[-0.15in]{\textwidth}{.03in}
\caption{
MCMC updater for lam0 in a SCR model with Binomial encounter
process and cloglog link function on detection. Here, pmat =
1-exp(-lam).
}
\label{mcmc.panel.updatelam0}
\end{panel}


Another possibility is to model variation in the individual and site 
specific detection probability,  $p_{ij}$, directly, without any 
transformation, such that
\begin{comment} Rahel the $\leftarrow$ isn't right here but I couldn't
make this work \end{comment}
\[
p_{ij} \leftarrow p_0 * exp(-D_{ij}^2/(2\sigma^2))
\]
and $p_0 \in [0,1]$.
This formulation is analogous to how detection probability is modeled 
in distance sampling under a half-normal detection function; however, 
in distance sampling $p_0$ -- detection of an individual on the transect 
line -- is assumed to be 1 \citep{buckland_etal:2001}. Under this 
formulation the updater for $\lambda_0$ (equivalent to $p_0$ in Eq XX) 
becomes:
%I think I should rename it p0 in the code; this is a little confusing
\begin{verbatim}
  lam0.cand <- rnorm(1, lam0, 0.1)
  if(lam0.cand >0 & lam0.cand < 1 ){   
      #automatically rejects lam0.cand that are not {0,1}
       lam.cand <- lam0.cand*exp(-(D*D)/(2*sigma*sigma))
       ll<- sum(dbinom(y, K, lam *z, log=TRUE)) #no transformation needed
       llcand <- sum(dbinom(y, K, lam.cand *z, log=TRUE))
       if(runif(1) < exp( llcand  - ll) ){
          ll<-llcand
            lam<-lam.cand
            lam0<- lam0.cand
         }
     }
\end{verbatim}


\subsection{Looking at model output}
Now that you have an MCMC algorithm to analyze spatial capture-recapture 
data with, let's run an actual analysis so we can look at the output. As 
an example, we will use the Fort Drum 
bear data set we already analyzed in Chapt. \ref{chapt.closed} with 
traditional non-spatial models (and that you will see again in Chapt. 
\ref{XX}). You can load the Fort Drum data
(\mbox{\tt data(''beardata'') }), extract the 
trap locations (\mbox{\tt trapmat}) and 
detection data and build the augmented $M \times J$ array of individual 
encounter histories.
 In addition to these data, we need to specify 
the outermost coordinates of the state-space. Since bears are wide 
ranging animals we add a 20--km buffer to the maximum and minimum 
coordinates of the trap array:

\begin{verbatim}
xl<- min(trapmat[,2])- 20  
yl<- min(trapmat[,3])- 20 
xu<- max(trapmat[,2])+ 20
yu<- max(trapmat[,3])+ 20
\end{verbatim}

Finally, source the MCMC code for the binomial encounter model algorithm 
with the cloglog link and run 5000 iterations. This should take 
approximately 25 minutes.
\begin{comment} Rahel: this is nice. you might make Scr.0 an R func! 
\end{comment}
\begin{verbatim}
 source('SCR0binom.txt')
 mod0<-SCR.0(y=Xaug, X=trapmat[,2:3], M=M, xl=xl, xu=xu, yl=yl, 
                   yu=yu, K=8, niter=5000)
\end{verbatim}

Before, we used simple {\bf R} commands to look at model results. 
However, there is a specific {\bf R} package to summarize MCMC 
simulation output and perform some convergence diagnostics -- package 
coda \citep{plummer_etal:2006}. Download and install coda, then 
convert your model output to an mcmc object
\begin{verbatim}
  chain<-mcmc(mod0)
\end{verbatim} 
which can be used by coda to produce MCMC specific output.

\subsubsection{Markov chain time series plots}

Start by looking at time series plots of your Markov chains using 
\verb#plot(chain)#. This command produces a time series plot and
 marginal posterior density plots for each monitored parameter, 
 similar to what we did before using the \verb#hist()# and \verb#plot()# 
 commands (Fig. \ref{mcmc.fig.timeseries}). Time series plots will tell 
 you several things:
First, recall from Sect. XXXXXX that the way the chains move 
through the parameter space gives you an idea of whether your MH 
steps are well tuned. If chains were constant over many iterations 
you would need to decrease the tuning parameter of the (Normal) 
proposal distribution. If a chain moves along some gradient to a 
stationary state very slowly, you may want to increase the tuning 
parameter so that the parameter space is explored more efficiently.


\begin{figure}
\begin{center}
\includegraphics[height=2.5in]{Ch7/figs/timeseries}
\end{center}
\caption{Time series and posterior density plots for $\sigma$ and $\lambda_0$.}
\label{mcmc.fig.timeseries}
\end{figure}


Second, you will be able to see if your chains converged and how many initial simulations you have to discard as burn-in. In the case of the chains shown in Fig. \ref{timeseries.fig}, we would probably consider the first 750 - 1000 iterations as burn-in, as afterwards the chains seem to be fairly stationary.

\subsection{Posterior density plots}
The \verb#plot()# command also produces posterior density plots and it is worthwhile to look at those carefully. For parameters with priors that have bounds (e.g. Uniform over some interval), you will be able to see if your choice of the prior is truncating the posterior distribution. In the context of SCR models, this will mostly involve our choice of $M$, the size of the augmented data set. If the posterior of $N$ has a lot of mass concentrated close to $M$ (or equivalently the posterior of $\psi$ has a lot of mass concentrated close to 1), as in the example in Fig. \ref{timeseries2.fig}, we have to re-run the analysis with a larger $M$.  A diffuse
posterior plot suggests
that the parameter may not be well-identified. 
There may not be enough information in your data to estimate model parameters and you may have to consider a simpler model. Finally, posterior density plots will show you if the posterior distribution is symmetrical or skewed -- if the distribution has a heavy tail, using the mean as a point estimate of your parameter of interest may be biased and you may want to opt for the median or mode instead.

\begin{figure}
\begin{center}
\includegraphics[height=2.5in]{Ch7/figs/timeseries2}
\end{center}
\caption{Time series and posterior density plots of $\psi$ and $N$ for the bear data set truncated by the upper limit of $M$ (500).}
\label{timeseries2.fig}
\end{figure}

\subsection{Serial autocorrelation and effective sample size}

Checking the degree of autocorrelation in your Markov chains and 
estimating the effective sample size your chain has generated should 
be part of evaluating your model output. If you use {\bf WinBUGS}
 through the \mbox{\tt R2WinBUGS} package, the \verb#print()# command 
 will automatically return the effective sample size for all monitored 
 parameters. In the coda package there are several functions you can use 
 to do so. \verb#effectiveSize()# will directly give you an estimate 
 of the effective sample size for you parameters:
\begin{verbatim}
> effectiveSize(chain)
    sigma      lam0       psi         N
 3.930303 78.259159 30.436348 32.047392
\end{verbatim}

Alternatively, you can use the \verb#autocorr.diag()# function, which will show you the degree of autocorrelation for different lag values (which you can specify within the function call, we use the defaults below):
\begin{verbatim}
> autocorr.diag(mcmc(mod))
           sigma      lam0       psi         N
Lag 0  1.0000000 1.0000000 1.0000000 1.0000000
Lag 1  0.9979948 0.9494134 0.9847503 0.9774201
Lag 5  0.9915567 0.8038168 0.9111951 0.9113525
Lag 10 0.9836016 0.6714021 0.8462108 0.8509803
Lag 50 0.8985337 0.1983780 0.6138516 0.6233994
\end{verbatim}
In the present case we see that autocorrelation is especially high for the 
parameter $\sigma$ and our effective sample size for this parameter is 
4! \footnote{Anyone have any idea how the autocorrelation in sigma could 
be reduced? XXXXXXXXXX YES: Mess with the MH tuning parameter......XXXXXXXX} 
This means we would have to run the model for much longer to 
obtain a reasonable effective sample size. Unfortunately, with many SCR models we observe high degrees of serial autocorrelation. For now, let's continue using this small set of samples to continue looking at the output.


\subsection{Summary results}
Now that we checked that our chains apparently have converged and pretending 
that we have generated enough samples from the posterior distribution, we 
can look at the actual parameter estimates. The \verb#summary()# function 
will return two sets of results: the mean parameter estimates, with their standard deviation, the naïve standard error -- i.e. your regular standard error calculated for $T$ (= number of iterations) 
samples without 
accounting for serial autocorrelation -- and the 
Time-series SE (in {\bf WinBUGS} 
and earlier in this book referred to as MC error), which accounts for 
autocorrelation. Remember our rule of thumb that this error 
decreases with increasing chain length and should be 1\% or less of the 
parameter estimate. In {\bf WinBUGS} the MC error is only given in the log 
output within {\bf BUGS} itself.
You should adjust the \verb#summary()# call by removing the burn-in from
calculating parameter summary statistics. To do so, use the \verb#window()#
command, which lets you specify at which iteration to start
'counting'. In contrast to {\bf WinBUGS}, which requires you to set the
burn-in length before you run the model, this command gives us full
flexibility to make decisions about the burn-in after we have seen the
trajectories of our Markov chains. For our example,
\verb#summary(window(chain, start=1001))# returns the following output:


\begin{verbatim}
Iterations = 1001:5000
Thinning interval = 1
Number of chains = 1
Sample size per chain = 4000

1. Empirical mean and standard deviation for each variable,
   plus standard error of the mean:

          Mean       SD  Naive SE Time-series SE
sigma   1.9986  0.13805 0.0021827       0.016091
lam0    0.1096  0.01523 0.0002407       0.001401
psi     0.6113  0.09148 0.0014465       0.010734
N     489.8535 71.79695 1.1352094       8.431119

2. Quantiles for each variable:

           2.5%       25%      50%      75%    97.5%
sigma   1.75780   1.89847   1.9900   2.0944   2.2772
lam0    0.08357   0.09824   0.1087   0.1192   0.1427
psi     0.45110   0.54838   0.6052   0.6639   0.8192
N     366.00000 440.00000 485.0000 530.0000 654.0000
\end{verbatim}

Looking at the MC errors (column labeled \mbox{\tt Time-series SE}), 
we see that in spite of the high autocorrelation, the MC error for 
$\sigma$ is below the 1\% threshold, whereas for all other parameters, 
MC errors are still above, another indication that for a thorough 
analysis we should run a longer chain.
Our algorithm gives us a posterior distribution of $N$, but we are usually 
interested in the density, $D$. Density itself is not a parameter of our 
model, but we can derive a posterior distribution for $D$ by dividing 
each value of $N$ ($N$ at each iteration) by the area of the state-space
 (here 3032.719 km$^2$) and we can use summary statistics of the 
 resulting distribution to characterize $D$:
\begin{verbatim}
> summary(window(chain[,4]/ 3032.719, start=1001))
Iterations = 1001:5000
Thinning interval = 1
Number of chains = 1
Sample size per chain = 4000

1. Empirical mean and standard deviation for each variable,
   plus standard error of the mean:

          Mean             SD       Naive SE Time-series SE
     0.1615229      0.0236741      0.0003743      0.0027801

2. Quantiles for each variable:

  2.5%    25%    50%    75%  97.5%
0.1207 0.1451 0.1599 0.1748 0.2156
\end{verbatim}
We see that our mean density of $0.16/km^2$ is very similar to the estimate of $0.18/km^2$ obtained under the non-spatial model M0 in Chapt. \ref{chapt.closed}.


\subsection{Other useful commands }
While inspecting the time series plot gives you a first idea of how well you tuned your MH algorithm, use \verb#rejectionRate()# to obtain the rejection rates (1 -- acceptance rates) of the parameters that are written to your output:
\begin{verbatim}
> rejectionRate(chain)
     sigma       lam0        psi          N
0.44108822 0.77675535 0.00000000 0.01940388
\end{verbatim}
 Recall (section XXXXXX?) that rejection rates should lie between 0.2 and 0.8, so our tuning seems to have been appropriate here. $\psi$ is never rejected since we update it with Gibbs sampling, where all candidate values are kept. And since $N$ is the sum of all $z_i$, all it takes for $N$ to change from one iteration to the next are small changes in the z-vector, so the rejection rate of $N$ is always low.
If you have run several parallel chains, you can combine them into a single mcmc object using the \verb#mcmc.list()# command on the individual chains (note that each chain has to be converted to an mcmc object before combining them with \verb#mcmc.list()#). You can then easily obtain the Gelman-Rubin diagnostic \citep{gelman_etal:2004}, in {\bf WinBUGS} called R-hat, using \verb#gelman.diag()#, which 
will indicate if all chains have converged to the same stationary distribution.
For details on these and other functions, see the \mbox{\tt coda} manual, 
which can be found (together with the package) on the CRAN mirror.

\section{Manipulating the state-space}

So far, we have constrained the location of the activity centers to fall
within the outermost coordinates of our rectangular state space by posing 
upper and lower bounds for $x$ and $y$. But what if ${\cal S}$ 
has an irregular 
shape -- maybe there is a large water body we would like to remove from 
${\cal S}$, because we know our terrestrial study species does not occur there.
Or the study takes place in a clearly defined area such as an island. 
As mentioned before, this situation is difficult to handle in {\bf WinBUGS}.
In some simple cases we can adjust the state space by setting one of the
coordinates of ${\bf s}_{i}$ to be some function of the other. 
In this manner, we can cut off corners of the rectangle to approximate 
the actual state space. In {\bf R}, we are much more flexible, as we can 
use the actual state-space polygon to constrain out ${\bf s}_i$. 
\footnote{ Have to check if we can use panther stuff for the book; 
otherwise, use raccoon example.} To illustrate that, let's look at a camera 
trapping study of Florida panthers (\emph{Puma concolor coryi}) conducted 
in the Picayune Strand Restoration Project (PSRP) area, southwest Florida 
(Fig. \ref{pantercamera.fig}), by XXX, and financed by XXX. In the 1960ies 
the PSRP area was slated for housing development, but then bought back by 
the State of Florida and is currently being restored to its original 
hydrology and vegetation. In an effort to estimate the density of the 
local Florida panther population, 98 camera traps were operated in the area 
for 21 months between 2005 and 2007. Florida panthers are wide-ranging 
animals and in order to account for their wide movements, the state-space 
was defined as the trapping grid buffered by 15 km around its outermost 
coordinates. However, the resulting rectangle contained some ocean in its 
southwestern corner (Fig. \ref{pantercamera.fig}).
In order to precisely describe the state-space, the ocean has to be 
removed. You can create a precise state-space polygon in {\bf ArcGIS} and 
read it into {\bf R}, or create the polygon directly within {\bf R}. In 
the present case we intersected two shape files -- one of the state of 
Florida and one of the rectangle defined by a strip of 15 km around the
 camera-trapping grid.
While you will most likely have to obtain the shapefile describing the 
landscape of and around your trapping grid (coastlines, water bodies etc.) 
from some external source, a polygon shapefile buffering your outermost 
trapping grid coordinates can easily be written in {\bf R}.

If \mbox{\tt xmin}, \mbox{\tt xmax}, \mbox{\tt ymin} and 
\mbox{\tt ymax}, mark the most extreme
$x$ and $y$ coordinates of your 
trapping grid and $b$ is the distance you want to buffer with, load the 
package \mbox{\tt shapefiles} \citep{stabler:2006} and issue the following
{\bf R} commands:
\begin{verbatim}
xl= xmin-b
xu= xmax+b
yl= ymin-b
yu= ymax+b

            #create data frame with coordinate pairs
dd <- data.frame(Id=c(1,1,1,1,1),X=c(xl,xu,xu,xl,xl), Y=c(yl,yl,yu,yu,yl)) 
ddTable <- data.frame(Id=c(1),Name=c("Item1"))
            #convert to shapefile, type polygon
ddShapefile <- convert.to.shapefile(dd, ddTable, "Id", 5) 
            # save to location of choice
write.shapefile(ddShapefile, 'c:/…’, arcgis=T) 
\end{verbatim}


\begin{figure}
\begin{center}
\includegraphics[height=2.5in]{Ch7/figs/panthercamera}
\end{center}
\caption{Rectangular state-space for a Florida panther camera trapping
study in the PSRP area (grey outline, red block inset map of Florida)
contain some ocean (white) that needs to be removed from the state-space.}
\label{mcmc.fig.pantercamera}
\end{figure}

You can read shapefiles into {\bf R} loading the package \mbox{\tt 
maptools}
\citep{lewin-koh_etal:2011} and using the function
\verb#readShapeSpatial()#. Make sure you read in shapefiles in UTM format, so
that units of the trap array, the movement parameter sigma and the
state-space are all identical.  Intersection of polygons can be done
in {\bf R} also, using the package \mbox{\tt rgeos} 
\citep{bivand_rundel:2011} and the
function \verb#gIntersect()#. The area of your (single) polygon can be
extracted directly from the state-space object \mbox{\tt SSp}:

\begin{verbatim}
 area <- SSp@polygons[[1]]@Polygons[[1]]@area /1000000
\end{verbatim}

 Note that dividing by 1000000 will return the area in km$^2$ if your coordinates describing the polygon are in UTM. If your state-space consists of several disjunct polygons, you will have to sum the areas of all polygons to obtain the size of the state-space.
To include this polygon into our MCMC sampler we need one last spatial 
{\bf R} package, \mbox{\tt sp} \citep{pebesma_bivand:2011}, which has a 
function, \verb#over()#, which allows us to check if a pair of coordinates 
falls within a polygon or not. All we have to do is embed this new check 
into the updating steps for ${\bf s}$:
\begin{verbatim}
    #draw candidate value
Scand <- as.matrix(cbind(rnorm(M, S[,1], 2), rnorm(M, S[,2], 2)))
     #convert to spatial points on UTM (m) scale
Scoord<-SpatialPoints(Scand*1000)   
     # check if scand is within the polygon
SinPoly<-over(Scoord,SSp)		

for(i in 1:M) {
    #if scand falls within polygon, continue update
   if(is.na(SinPoly[i])==FALSE) {		
… [rest of the updating step remains the same]
\end{verbatim}
Note that it is much more time-efficient to draw all $M$ candidate values 
for $s$ and check once if they fall within the state-space, rather than 
running the \verb#over()# command for every individual pair of 
coordinates. To make sure that our initial values for {\bf s} also fall 
within the polygon of ${\cal S}$, we use the function \verb#runifpoint()# 
from the package \mbox{\tt spatstat} \citep{baddeley_turner:2005}, 
which generates random uniform points within a specified polygon. You'll 
find this modified MCMC algorithm (\mbox{\tt SCR0poisSSp}) in the {\bf R} 
package \mbox{\tt scrbook}.
Finally, observe that we are converting candidate coordinates of ${\cal S}$ 
back to meters to match the UTM polygon. In all previous examples, 
for both the trap locations and the activity centers we have used UTM 
coordinates divided by 1000 to estimate $\sigma$ on a km scale. This is 
adequate for wide ranging individuals like bears. In other cases you 
may center all coordinates on 0. No matter what kind of transformation you 
use on your coordinates , make sure to always convert candidate values for 
${\cal S}$ back to the original scale (UTM) before running the 
\verb#over()# command.

\section{MCMC software packages}

Throughout most of this book we will use {\bf WinBUGS} and, occasionally, {\bf JAGS} to run MCMC analyses. 
Here, we will briefly discuss the main pros and cons of these two programs 
as well as {\bf WinBUGS} successor {\bf OpenBUGS}. 

\subsection{WinBUGS}

In a nutshell, {\bf WinBUGS} (and the other programs) do everything that we 
just went through in this chapter (and quite a bit more). Looking through 
your model, {\bf WinBUGS} determines which parameters it can use standard 
Gibbs sampling for (i.e. for conjugate full conditional distributions). 
Then, it determines, in the following hierarchy, whether to use adaptive 
rejection sampling, slice sampling or -- in the 'worst' case -- 
Metropolis-Hastings sampling for the other full conditionals 
\citep{spiegelhalter_etal:2003}. If it uses MH sampling, it will 
automatically tune the updater so that it works efficiently.
While {\bf WinBUGS} is a convenient piece of software that is still 
widely used, its major drawback is that it is no longer being developed, 
i.e. no new functions or distributions are added and no bugs are fixed.

\subsection{OpenBUGS}
{\bf OpenBUGS} is essentially the successor of {\bf WinBUGS}. While the 
latter is
no longer actively developed, {\bf OpenBUGS} continues to be 
developed. The
name '{\bf OpenBUGS}' refers to the software being open source, so users 
do
not need to download a license key, like they have to for {\bf WinBUGS}
(although the license key for {\bf WinBUGS} is free and valid for life).

Compared to {\bf WinBUGS}, {\bf OpenBUGS} 
has  more built-in functions. The
method of how to determine the right updater for each model parameter
has changed and the user can manually control the MCMC algorithm used
to update model parameters.  Several other changes have been
implemented in {\bf OpenBUGS} and a detailed list of differences between the
two {\bf BUGS} versions, can be found at
\url{http://www.openbugs.info/w/OpenVsWin}.

While {\bf OpenBUGS} is a useful program for a lot of MCMC sampling
applications, for reasons we do not understand, simple SCR models do
not converge sometimes in {\bf OpenBUGS}. It is therefore advisable that 
you check any
{\bf OpenBUGS} SCR model results against result from {\bf WinBUGS}. Also,
currently, the {\bf R} package \mbox{\tt BRugs} \citep{thomas_etal:2006}, 
necessary
for running {\bf OpenBUGS} through {\bf R}, has problems with 64-bit 
machines, so
you may have to use the 32-bit version of {\bf R} and {\bf OpenBUGS} 
in order to
make it work. The {\bf BUGS} project site at 
\url{http://www.openbugs.info}
provides a lot of information on and download links for {\bf OpenBUGS}.

There is an extensive help archive for both {\bf WinBUGS} and {\bf OpenBUGS}
 and you can subscribe to a mailing list, where people pose and answer 
 questions of how to use these programs at 
 \url{http://www.mrc-bsu.cam.ac.uk/bugs/overview/list.shtml}

\subsection{JAGS -- Just Another Gibbs Sampler}

{\bf JAGS}, currently at Version 3.1.0, is another free program for analysis 
of Bayesian hierarchical models using MCMC simulation. Originally, {\bf JAGS}
 was the only program using the {\bf BUGS} language that would run on 
 operating systems other than the 32 bit Windows platforms. By now, there 
 are {\bf OpenBUGS} versions for Linux or Macintosh machines.
{\bf JAGS} 'only' generates samples from the posterior distribution; 
analysis of the output is done in {\bf R}, either by running {\bf JAGS} 
through {\bf R} using either the packages \mbox{\tt rjags} 
\citep{plummer:2011} or \mbox{\tt R2jags} \citep{su_yajima:2011}, or by 
using coda on your {\bf JAGS} output. The program, manuals and \mbox{\tt rjags} 
can be downloaded at \url{http://sourceforge.net/projects/mcmc-jags/files/}
When run from within {\bf R} using the package \mbox{\tt rjags] or \mbox{R2jags}, 
writing a \mbox{\bf JAGS} model is virtually identical to writing a {\bf WinBUGS}
 model. However, some functions may have slightly different names and you 
 can look up available functions and their use in the {\bf JAGS} 
 manual. One potential downside is that {\bf JAGS} can be very particular 
 when it comes to initial values. These may have to be set as close to 
 truth as possible for the model to start. Although {\bf JAGS} lets 
 you run several parallel Markov chains, this characteristic interferes 
 with the idea of using overdispersed initial values for the different 
 chains. Also, we have occasionally experienced {\bf JAGS} to crash and 
 take the {\bf R} GUI with it. Only re-installing both {\bf JAGS} and 
 {\tt rjags} seemed to solve this problem.
On the plus side, {\bf JAGS} usually runs a little faster than {\bf WinBUGS},
 sometimes considerably faster (see Sect. \ref {4.XYZ}), is constantly 
 being developed and improved and it has a variety of functions that are 
 not available in {\bf WinBUGS}. For example, {\bf JAGS} allows you to 
 supply observed data for some deterministic functions of unobserved 
 variables. In {\bf BUGS} we cannot supply data to logical nodes. 
 Another useful feature is that the adaptive phase of the model 
 (the burn-in) is run separately from the sampling from the stationary 
 Markov chains. This allows you to easily add more iterations to the 
 adaptive phase if necessary without the need to start from 0. There 
 are other, more subtle differences and there is an entire manual section 
 on differences between {\bf JAGS} and {\bf OpenBUGS}.
For questions and problems there is a {\bf JAGS} forum online at 
\url{http://sourceforge.net/projects/mcmc-jags/forums/forum/610037}.
\footnote{As we make progress on the book, lets be sure  to add 
linkages to places where we use JAGS in examples.}

\section{Summary and Outlook}

Although there are a number of flexible and extremely useful software 
packages to perform MCMC simulations, it sometimes is more efficient to 
develop your own MCMC algorithm. Building an MCMC code follows three basic 
steps: Identify your model including priors and express full conditional 
distributions for each model parameter. If full conditionals are parametric 
distributions, use Gibbs sampling to draw candidate parameter values from 
those distributions; otherwise use Metropolis-Hastings sampling to draw 
candidate values from a proposal distribution and accept or reject them 
based on their posterior probability densities.
These custom-made MCMC algorithms give you more modeling flexibility than 
existing software packages, especially when it comes to handling the
 state-space: In {\bf BUGS} (and {\bf JAGS} for that matter) we define
  a continuous rectangular state-space using the corner coordinates to 
  constrain the Uniform priors on the activity centers ${\bf s}$.
   But what if a continuous rectangle isn't an adequate description of 
   the state-space? In this chapter we saw that in {\bf R} it only takes 
   a few lines of code to use any arbitrary polygon shapefile as the 
   state-space, which is especially useful when you are dealing with 
   coastlines or large bodies of water that need removing from the 
   state-space. Another example is the SCR {\bf R} package \mbox{\tt SPACECAP}
    \citep{gopalaswamy_etal:2011} that was developed because implementation
     of an SCR model with a discrete state-space was inefficient in {\bf WinBUGS}.
Another situations in which using {\bf BUGS}/{\bf JAGS} becomes
increasingly
complicated or inefficient is when using point processes other 
than the 
 Binomial point process (''uniformity'') which underlies the basic 
 SCR model (see Chapt. \ref {Chapter X}). In the Chapt. 
 \ref {Chapter X} and XX you will see examples of different point processes,
  implemented using custom-made MCMC algorithms.
   \footnote{Richard, Beth expand on that?}
Finally, the Chapt. \ref {Chapter X} and XX deal with unmarked or 
partially marked populations using hand-made MCMC algorithms to 
handle the (partially) latent individual encounter histories. 
While some of these models can be written in {\bf BUGS}/{\bf JAGS}, 
\footnote{the Poisson one for partially marked we wrote in BUGS and it 
should work with a known number of marked; the Bernoulli in JAGS with the 
dsum() function should work for the fully unknown; maybe some others?
 I don’t remember. We may have to try writing the others before saying 
 that they don’t work in {\bf BUGS}/{\bf JAGS}; they are certainly much faster in {\bf R}, 
 though.}, they are painstakingly slow; others cannot be implemented in 
 {\bf BUGS}/{\bf JAGS} at all (e.g., the classes of models
 considered in Chapts. \ref{chapt.ecoldist} and  \ref{chapt.state-space}).
In conclusion, while you can certainly get by using {\bf BUGS}/{\bf JAGS} 
for standard SCR models, knowing how to write your own MCMC sampler 
allows you to tailor these models to your specific needs.



\chapter{Goodness of Fit and stuff}
\label{chapt.gof}

\chapter{Modeling Encounter Probability}
\label{chapt.covariates}

\chapter{
%Modeling Animal space-usage with
%Detection Models based on Ecological Distance
%Ecological Distance Models in Spatial Capture-Recapture
Modeling Space Usage: Ecological Distance in Spatial Capture-Recapture Models
}
\markboth{Chapter XXX}{}
\label{chapt.ecoldist}


\vspace{.3in}

\begin{comment} % RBC commented this out as suggested by Rahel

%% this material is a general introduction for a manuscript
%Spatial capture-recapture models are a relatively new class of models
%for estimating animal density from capture-recapture data with
%auxiliary information about individual capture locations
%\citep{efford:2004,borchers_efford:2008, royle_young:2008, efford_etal:2009ecol,
%  royle_etal:2009ecol}.
Spatial capture-recapture models
express encounter probability
as a function of the distance between an individual's activity center,
say ${\bf s}_{i}$, and trap location, say ${\bf x}_{j}$.
In these models ${\bf s}_{i}$ is regarded as a latent variable and
conventional methods of statistical inference either based on marginal
likelihood \citep{borchers_efford:2008} or Bayesian analysis by MCMC
\citep{royle_young:2008}.

While the models are a relatively recent innovation, their use is
already becoming widespread \citep{efford_etal:2009ecol,
  gardner_etal:2010jwm, gardner_etal:2010ecol,kery_etal:2010,
  borchers:2011,gopalaswamy_etal:2012, foster_harmsen:2012} because they resolve
critical problems with using ordinary non-spatial capture-recapture
methods such as ill-defined area sampled, and heterogeneity in
encounter probability due to the juxtaposition of individuals with
traps, and they provide a framework for modeling of trap-specific
covariates.  Furthermore, essentially all real capture-recapture
studies produce auxiliary spatial information and therefore SCR models
are directly relevant to standard data collected from such studies.
% Indeed, the use of ordinary
%capture-recapture models essentially admits a model misspecification
%(i.e. homogeneous encounter probability) by ignoring the explicit
%spatial information.

XXX MAYBE YOU COULD START THE CHAPTER AT THIS POINT; THE OTHER STUFF HAS BEEN COVERED BY THE PREVIOUS BOOK CHAPTERS XXXXX
\end{comment}

Every spatial capture-recapture model that we have considered so far
has expressed encounter probability as function of the Euclidean
distance between individual activity
centers $\bf s$ and trap locations $\bf x$. While these simple encounter
probability models will often
be sufficient for practical
purposes, especially in small data sets, sometimes developing more
complex models of the detection process as it relates to space usage
of individuals will be useful.  Animals may not judge distance in
terms of Euclidean distance but, rather, according to quality of local
habitat, landscape connectivity, perceived mortality risk, and other
considerations affecting movement behavior.
\begin{comment}
As an example of the potential problem of parameterizing SCR models
using Euclidean distance, imagine a study area bisected by a large
semi-permeable barrier. In standard SCR models, the probability of
capturing an animal in a trap located on the opposite side of the
barrier would simply be a function of distance, whereas in reality it
should be a function of both distance and the permeability of the
barrier.
Such situations are extremely common in capture-recapture
studies where multiple habitats occur in the study area or when
animals use linear features such as trails, corridors, or rivers.
\end{comment}
Moreover, because encounter probability and the distance
metric upon which it is based represent outcomes of individual
movements about their home range, ecologists might have explicit
hypotheses about how environmental variables affect the distance
metric, and it is therefore desirable to incorporate these hypotheses
directly into SCR models so that they may be formally evaluated
statistically.

In this chapter we develop a framework for modeling animal space usage
in SCR models, by parameterizing models for encounter probability
based on ``ecological distance''.  In particular, following
\citet{royle_etal:2012ecol}, we adopt a cost-weighted distance metric
(the least-cost path) used widely in landscape ecology for modeling
connectivity, movement and gene flow
\citep{adriaensen_etal:2003,manel_etal:2003,mcrae_etal:2008}. In the
context of SCR models we can use this as the basis for computing the
distance between traps and individuals activity centers. In this way
we can explicitly accommodate landscape structure and account for how
animals use space in SCR models. We develop a likelihood-based
inference framework for SCR model parameters using this new distance
metric when the ecological distance function is known.  We show that
the MLEs are approximately unbiased in moderate sample sizes, as
expected, but also that the misspecified model based on Euclidean
distance can produce substantial bias in estimates of $N$ and hence
density.  Further, we extend the model to allow for likelihood
estimation of parameters of the cost function.

Using this methodological extension of SCR models, it is possible to
make formal statistical inferences
about movement and connectivity from
capture-recapture studies that generate sparse individual encounter
history data without subjective prescription of resistance
or cost surfaces.


\section{Distance Models}


In the standard SCR model we model encounter probability as a function
of Euclidean distance. For example, using the binomial observation model
as an example (Chapt. \ref{chapt.scr0}), let
$y_{ij}$ be individual- and trap specific binomial counts
with sample size $K$ and probabilities
$p_{ij}$. The Gaussian or ``half-normal'' model is \footnote{Note the
  parameter labeling is not consistent with the rest of the book}
\[
log(p_{ij})= \theta_{0} + \theta_{1} dist({\bf x}_{j} - {\bf s}_{i})^{2}
\]
or, equivalently,
\[
p_{ij} = \lambda_{0} exp(-  dist({\bf x}_{j} - {\bf s}_{i})^{2}
/(2\sigma^{2}) )
\]
where $\theta_{0} = log(\lambda_{0})$ and $\theta_{1} =
-1/(2\sigma^2)$.

%In all previous applications of SCR models in this book, as well as in
%the literature,
The main problem with the normal Euclidean distance metric, i.e., 
$dist({\bf x}_{j} - {\bf s}_{i}) = ||{\bf x}_{j} - {\bf s}_{i}||$,
%and
%the parameters $\theta_0$ and $\theta_1$ have been estimated using
%standard methods (likelihood or Bayesian).  The main problem with this
%approach
is that it is unaffected by
habitat or landscape structure, and it implies that the space used by
individuals is stationary, and symmetric which may be unreasonable
assumptions for some species. By stationary here we mean in the formal
sense of
invariance to translation. That is, the properties of an individual
home range centered at some point ${\bf s}$ are exactly the same as
any other point say ${\bf s}'$.

As an example, if the common detection model based on a bivariate
normal probability distribution function is used, then the implied
space usage by {\it all} individuals, no matter their location in
space or local habitat conditions, is symmetric with circular contours
of usage intensity (density contours of the pdf).

\citet{royle_etal:2012ecol} extended this class of SCR models to
accommodate alternative distance metrics that explicitly incorporate
information about the landscape so that a unit of distance is variable
depending on identified covariates.  Thus, ``where'' an individual
lives on the landscape, and the state of the surrounding landscape,
will determine the character of its usage of space. In particular, they
suggest distance metrics that imply irregular, asymmetric and
non-stationary home ranges of individuals. As an example,
Fig. \ref{fig.distort} shows a typical symmetric home range, and an
comressed home range resulting from the effect of an environmental
variable on an animal's movement behavior.

\begin{figure}[h]
\centering
\includegraphics[width=5in,height=1.3in]{Ch10/figs/distort}
\caption{A symmetric home range (left), a habitat variable (center),
  and a non-symmetric home range (right) resulting from the cost imposed on
  movement by the habitat variable.}
\label{fig.distort}
\end{figure}


\section{Cost-Weighted Distance}

We adopt the use of a cost-weighted distance metric here which defines
the distance between points by accumulating pixel-specific costs
determined under a cost function defined by the user.  The idea of
cost-weighted distance to characterize animal use of landscapes is
widely used in landscape ecology for modeling connectivity, movement
and gene flow \citep{beier_etal:2008}. As is customary for reasons of
computational tractability we consider a discrete landscape
defined by
a raster of some prescribed resolution. The distance between any two
points ${\bf x}$ and ${\bf x}'$ can be represented by a sequence of
line segments connecting neighboring pixels say ${\bf l}_{1},{\bf
  l}_{2},\ldots,{\bf l}_{m}$. Then the cost-weighted distance between
${\bf x}$ and ${\bf x}'$ is

\begin{equation}
 d({\bf x},{\bf x}')
  =  \sum_{i=1}^{m-1} cost({\bf l}_{i},{\bf l}_{i+1})||{\bf l}_{i} - {\bf l}_{i+1}||
\label{eq.costweighted}
\end{equation}

{\flushleft
where } $cost({\bf l}_{i},{\bf l}_{i+1})$ is the user-defined cost function
to move
from pixel ${\bf l}_{i}$ to neighboring pixel ${\bf l}_{i}$ in the sequence.
Given the ``cost'' of each pixel, it is a simple matter to compute the
cost-weighted distance between any two pixels, along {\it any} path,
simply by accumulating the incremental  costs weighted by
distances.
In the context of
spatial capture-recapture models (and, more generally, landscape
connectivity) we are concerned with the {\it minimum} cost-weighted
distance, or the {\it least-cost path}, between any two points which
we will denote by $d_{lcp}$, which is
the
sequence ${\bf l}_{1},{\bf l}_{2},\ldots,{\bf l}_{m}$ that minimizes
the objective function defined by Eq. \ref{eq.costweighted}. That is,

\begin{equation}
 d_{lcp}({\bf x},{\bf x}')
  =  min_{{\bf l}_{1},\ldots,{\bf l}_{m}}  \sum_{i=1}^{m-1} cost({\bf l}_{i},{\bf l}_{i+1})||{\bf l}_{i} - {\bf l}_{i+1}||
\label{eq.lcp}
\end{equation}

{\flushleft
 Least-cost} path distance can be calculated in
 many geographic information systems and other software packages,
including the {\bf R} package \mbox{\tt
  gdistance} \citep{vanetten:2011}.

The key ecological aspect of least-cost path modeling is the
development
of models for pixel-specific cost.
In this paper we model cost as a function of one or more covariates
defined on every pixel of the according raster. For example, using a
single covariate $z(x)$ we define the cost of moving from some pixel
${\bf x}$ to neighboring pixel ${\bf x}'$ as
\begin{equation}
 log(cost({\bf x},{\bf x}'))=  \theta_{2} \frac{z({\bf x})+z({\bf x}')}{2}
\label{ecoldist.eq.cost}
\end{equation}
Thus, if $\theta_{2} = 0$ then substituting $cost({\bf x},{\bf x}')
=exp(0) = 1$ into
Eq. \ref{eq.lcp} will produce the ordinary Euclidean distance
between points. Here we assume the covariate $z$ is positive-valued
and constrain $\theta_{2}\ge 0$ so as to avoid
negative costs. While not necessarily problematic from a mathematical
standpoint, negative costs are unrealistic biologically. %unless there's a people mover....

In practical applications, variables that influence the cost of moving
across the landscape include things like highways
\citep[e.g.,][]{epps_etal:2005}, elevation \citep{cushman_etal:2006},
ruggedness \citep{epps_etal:2007}, snow cover
\citep{schwartz_etal:2009}, distance to escape terrain
\citep{shirk_etal:2010}, range limitations \citep{mcrae_beier:2007},
or distance from urban areas, highways, human disturbance or other
factors that animals might avoid.  Together multiple environmental
variables create a resistance surface, which forms the linchpin of all
connectivity planning \citep{spear_etal:2010}.  Often $\theta_{2}$ is
fixed by the investigator. Although $\theta_{2}$ is rarely known,
conservation biologists design linkages that require this resistance
value as input \citep[see][and articles cited
therein]{beier_etal:2008}.  Typically planners pick a value based on
expert opinion \citep{beier_etal:2008}, although recently researchers
have begun to define costs based on resource selection functions,
animal movement \citep{tracy:2006, fortin_etal:2005}, or genetic
distance data (e.g., \citet{gerlach_musolf:2000};
\citet{epps_etal:2007}; \citet{schwartz_etal:2009}.

To formalize the use of cost-weighted distance in SCR models, we
substitute Eq. \ref{eq.lcp} in the expression for encounter
probability (Eq. \ref{eq.encounter}) and maximize the resulting
likelihood which we address below. This allows us to formally model
these factors that influence space usage, and test explicit hypotheses
about these things using only individual level encounter history data
from capture-recapture studies.

\subsection{Example of Computing Cost-weighted distance}

As an example of the cost-weighted distance calculation consider the
following landscape comprised of 16 pixels with unit spacing
identified as follows, along with the pixel-specific cost:
\begin{center}
\begin{verbatim}
  pixel ID                 Cost
  1  5  9  13          100   1   1  1
  2  6 10  14          100 100   1  1
  3  7 11  15          100 100 100  1
  4  8 12  16          100 100   1  1
\end{verbatim}
\end{center}
This simple cost
raster is shown in Fig. \ref{ecoldist.fig.raster}. We assume the scale
is such that the distance between neighboring pixels in any cardinal
direction is 1 unit, and the distance between neighbors on a diagonal
is $\sqrt{2}$ units.
We assigned low cost of 1 to ``good habitat'' pixels (or pixels
we think of as ``highly connected'' by virtue of being in good
habitat) and, conversely, we assign high cost (100) to ``bad
habitat''. So the shortest cost-weighted distance between pixels 5 and
9 in this example is just 1 unit, the shortest cost-distance between
pixels 5 and 10 is $\sqrt{2}(1+1)/2 = 1.414214$ units, the shortest
distance between pixels 4 and 8 is 100 units, while the shortest
cost-distance between 4 and 12 is 150.5. A tough one is: what is the
shortest distance between 7 and 16? An individual at pixel 7 can move
diagonal (which has distance $\sqrt{2}$) and pay $sqrt(2)*(100+1)/2 + 1 =72.41778$.

\begin{figure}
\begin{center}
\includegraphics[height=3.25in,width=3.25in]{Ch10/figs/raster_2values}
\end{center}
\caption{A $4 \times 4$ raster with cost = 1 (white) or 100 (shaded) to represent ease of movement across a pixel.}
\label{ecoldist.fig.raster}
\end{figure}


Once the cost raster is created, the least-cost path distances are
computed with just a couple {\bf R} commands, and those can be
inserted directly into the likelihood construction for an ordinary
spatial capture-recapture model The {\bf R} package
\mbox{\tt gdistance} uses the implementation of Dijkstra's algorithm
\citep{dijkstra:1959} found in the \mbox{\tt igraph} package
\citep{csardi:2010}.  Using \mbox{\tt gdistance}, we 
define the incremental cost of moving from one pixel to another as the
distance-weighted {\it average} of the 2 pixel costs. We demonstrate
how to do this subsequently.

The {\bf R} commands for computing the least-cost distance between all pairs of pixels
are as follows:
\begin{verbatim}
r<-raster(nrows=4,ncols=4)
projection(r)<- "+proj=utm +zone=12 +datum=WGS84"
extent(r)<-c(.5,4.5,.5,4.5)
costs1<- c(100,100,100,100,1,100,100,100,1,1,100,1,1,1,1,1)
values(r)<-matrix(costs1,4,4,byrow=FALSE)
par(mfrow=c(1,1))
plot(r)
\end{verbatim}
Then we use the functions \mbox{\tt transition}, \mbox{\tt
  geoCorrection} (which is only necessary if the data are not
projected or if cells are considered to have more than 4 neighbors)
 and \mbox{\tt costDistance} to compute the distance
matrix. The transition function computes the cost of making a
transition between
any two pixels, and it operates on the inverse-scale (''conductance'')
and so the
\mbox{\tt transitionFunction} argument is given as $1/mean(x)$.
To compute the cost distance we prescribe a set of points, or  we
can compute it  between
two sets of points (which is handy when one of the sets is of trap
locations, and the other is of individual activity centers).
To compute the distances for pixels in a raster,
we use the center points of each raster.  The {\bf R}
 commands altogether are as follows:
{\small
\begin{verbatim}
tr1<-transition(r,transitionFunction=function(x) 1/mean(x),directions=8)
tr1CorrC<-geoCorrection(tr1,type="c",multpl=FALSE,scl=FALSE)
pts<-cbind( sort(rep(1:4,4)),rep(4:1,4))
costs1<-costDistance(tr1CorrC,pts)
outD<-as.matrix(costs1)
\end{verbatim}
}
Now we can look at the result and see if it makes sense to us. Here we
print the first 5 columns of this distance matrix to illustrate a
couple of examples of calculating the minimum cost-weighted distance
between points:
\begin{center}
{\small
\begin{verbatim}
> outD[1:5,1:5]
         1         2        3        4         5
1   0.0000 100.00000 200.0000 205.2426  50.50000
2 100.0000   0.00000 100.0000 200.0000  71.41778
3 200.0000 100.00000   0.0000 100.0000 171.41778
4 205.2426 200.00000 100.0000   0.0000 154.74264
5  50.5000  71.41778 171.4178 154.7426   0.00000
\end{verbatim}
}
\end{center}
An interesting case is that between point 1 and 4. Note that simply
taking the shortest Euclidean distance, weighted by cost, produces a
cost-weighted distance of $100 \times 1$ to move from pixel 1 to pixel
2, and similarly from 2 to 3 and 3 to 4, producing a total
cost-weighted distance of $300$. However, the actual {\it least-cost
  path} has cost-weighted distance $205.2426$.
The shortest path has an individual moving from pixel 1 to 5, then 5
to 10, 10 to 15, 15 to 12, 12 to 8 and 8 to 4 which should add up to
$205.2426$.

\section{Fitting Models of Space Usage by MLE}
\label{ecoldist.sec.mle}

Throughout much of this book we rely on Bayesian analysis by MCMC
mostly using
{\bf BUGS}, but sometimes (as in Chapt. \ref{chapt.mcmc}) developing
our own
implementations. However, occasionally we prefer to use likelihood
estimation, such as when
we can compare a set of models directly by likelihood either to do a
direct hypothesis test of a parameter, or to tabulate a bunch of AIC
values. It turns out, for this class
of models for space usage based on ecological distance, we actually
prefer likelihood methods
not because they have any conceptual or methodological benefit, but
simply because
they are more computationally efficient to implement
\citep{royle_etal:2012ecol}.
So here we adopt our formulation of maximum likelihood estimation
\citep{borchers_efford:2008}
from Chapt. \ref{chapt.mle}
for the class of models based on ecological distance. This is really
just a straightforward
adaption of that.

We continue to work here with the binomial model:
\[
	y_{ij}| {\bf s}_{i} \sim \mbox{Bin}(K, p_{\theta}(d_{lcp}({\bf x}_{j},{\bf s}_{i};\theta_{2}); \theta_{0}, \theta_{1})
\]
where we have indicated the dependence of $p_{ij}$ on the parameters
${\bm \theta}$, and also $d_{lcp}$ which
itself depends on $\theta_{2}$, and the latent variable ${\bf s}$.
%The parameters
%${\bm \theta}$ include whatever parameters are involved in the
%cost-weighted distance function, i.e., at least $\theta_{2}$ from
%Eq. \ref{eq.cost}.
For the random effect we have ${\bf s}_{i} \sim  \mbox{Unif}({\cal
  S})$. Recall that the state-space $\cal S$ is defined by the raster
data in this context.
The joint distribution of the data for individual $i$ is the product
of $J$ binomial terms (i.e., contributions from each of $J$ traps):

\[
  [{\bf y}_{i} | {\bf s}_{i} , \theta] =
  \prod_{j=1}^{J} \mbox{Bin}(K, p_{\theta}({\bf x}_{j},{\bf s}_{i}) )
\]

{\flushleft This} assumes that encounter of individual $i$ in each
trap is independent of encounter in every other trap. Conditional on
${\bf s}_{i}$ this is reasonable in most applications in our view.
 The so-called marginal likelihood is computed by removing
${\bf s}_{i}$, by integration,  from the conditional-on-${\bf s}$
likelihood and regarding the {\it marginal} distribution of the data
as the likelihood. That
is, we compute:

\[
  [y|{\bm \theta}] =
\int_{{\cal S}}  [ {\bf y}_{i} |{\bf s}_{i},{\bm \theta}] g({\bf s}_{i}) d{\bf s}_{i}
\]

{\flushleft where}, under the uniformity assumption, we have
$g({\bf s}) = 1/||{\cal S}||$.
The joint likelihood for all $N$ individuals, assuming independence of
encounters among individuals, is the product of $N$ such terms:

\[
{\cal L}({\bm \theta} | {\bf y}_{1},{\bf y}_{2},\ldots, {\bf y}_{N}) = \prod_{i=1}^{N}
[{\bf y}_{i}|{\bm \theta}]
\]

The key operation for computing the likelihood is solving the
2-dimensional integration problem to remove ${\bf s}$, which we
resolve as we did previously in Chapt. \ref{chapt.mle}, using the
rectangular rule for integration, and averaging the integrand over a
fine mesh of points.
Therefore,
the marginal pmf of ${\bf y}_{i}$, is
approximated by
\begin{equation}
         [{\bf y}_{i}|\theta] = \frac{1}{nG} \sum_{u=1}^{nG}  [ {\bf
            y}_{i} |{\bf s}_u, \theta]
\label{mle.eq.intlik}
\end{equation}
To deal with the fact that $N$ is unknown, there are two key issues
that need to be addressed.  First is that we don't observe the
``all-zero'' encounter histories (i.e., $y_{ij} = 0$ for all $j$)
corresponding to uncaptured individuals, so we have to make sure we
compute the probability for that all zero encounter history which we
do operationally by tacking a row of zeros onto the encounter history
matrix. We include the number of such all-zero encounter histories as
an unknown parameter of the model, which we label $n_{0}$.  In
addition, we have to be sure to include a combinatorial term to
account for the fact that of the $n$ observed individuals there are
${N \choose n}$ ways to realize a sample of size $n$. The
combinatorial term involves the unknown $n_{0}$ and thus it must be
included in the likelihood.

We wrote an {\bf R} function to evaluate the likelihood which we optimize
using the {\bf R} function \mbox{\tt nlm}.
The likelihood is given in the {\tt scrbook} package as the function
\mbox{\tt intlik3ed}. The help file
provides an example of its usage and for simulating data.

To use this function the cost covariate $z(x)$ has to be of class
\mbox{\tt RasterLayer} which requires packages \mbox{\tt sp} and
\mbox{\tt raster} to manipulate.
The following is a stylized and more concise verstion of the actual
function, and we apply this in the following section.

{\small
\begin{verbatim}
intlik3ed<-function(start=NULL,y=y,K=NULL,X=traplocs,
distmet="ecol",covariate,theta2=NA){

nc<-covariate@ncols
nr<-covariate@nrows
Xl<-covariate@extent@xmin
Xu<-covariate@extent@xmax
Yl<-covariate@extent@ymin
Yu<-covariate@extent@ymax
### ASSUMES SQUARE RASTER -- NEED TO GENERALIZE THIS
delta<- (Xu-Xl)/nc
xg<-seq(Xl+delta/2,Xu-delta/2,delta)
yg<-seq(Yl+delta/2,Yu-delta/2,delta)
npix.x<-length(xg)
npix.y<-length(yg)
area<- (Xu-Xl)*(Yu-Yl)/((npix.x)*(npix.y))
G<-cbind(rep(xg,npix.y),sort(rep(yg,npix.x)))
nG<-nrow(G)

if(distmet=="euclid")
D<- e2dist(X,G)
if(distmet=="ecol"){
if(is.na(theta2))
theta2<-exp(start[4])
cost<- exp(theta2*covariate)
tr1<-transition(cost,transitionFunction=function(x) 1/mean(x),directions=8)
tr1CorrC<-geoCorrection(tr1,type="c",multpl=FALSE,scl=FALSE)
D<-costDistance(tr1CorrC,X,G)
}

theta0<-start[1]; theta1<-start[2]; n0<-exp(start[3])

probcap<- (exp(theta0)/(1+exp(theta0)))*exp(-theta1*D*D)
Pm<-matrix(NA,nrow=nrow(probcap),ncol=ncol(probcap))
ymat<-y ; ymat<-rbind(y,rep(0,ncol(y)))
lik.marg<-rep(NA,nrow(ymat))
for(i in 1:nrow(ymat)){
Pm[1:length(Pm)]<- (dbinom(rep(ymat[i,],nG),rep(K,nG),probcap[1:length(Pm)],log=TRUE))
lik.cond<- exp(colSums(Pm))
lik.marg[i]<- sum( lik.cond*(1/nG) )
}
nv<-c(rep(1,length(lik.marg)-1),n0)
part1<- lgamma(nrow(y)+n0+1) - lgamma(n0+1)
part2<- sum(nv*log(lik.marg))
out<-  -1*(part1+ part2)
out
}
\end{verbatim}
}

\subsection{Bayesian Analysis}

While implementation of these ecological distance SCR models is reasonably straightforward, it is difficult to fit them using the {\bf BUGS} engines
because it is not possible, to the best of our knowledge, to compute
the least-cost path distance.  It would be possible to fit the models
in {\bf BUGS} if the parameter $\theta_{2}$ was fixed. In that case,
one could compute the distance matrix ahead of time and reference the
required elements for a given ${\bf s}$.
Alternatively, it would be possible to write a custom MCMC routine
using the methods we present in Chapt. \ref{chapt.mcmc}, although we
have not yet developed our own implementation.




\section{Example: SCR model based on ecological distance}

In this section we provide examples that we think are typical of how
cost-weighted distance models can be used in real capture-recapture
problems.  We define a $20 \times 20$ pixel covariate raster with
extent = $[0.5, 4.5] \times [0.5, 4.5]$.  We regard this, for the
purposes of our example, as a coarse landscape covariate, with pixels
having some arbitrary scaling say, a $2 \times 2$ km resolution. Thus,
the raster defines a landscape of $40 \times 40$ km and we suppose
that 16 camera traps are established at the integer coordinates
$(1,1), (1,2), \ldots, (4,4)$. We could think of this as a landscape
within which we're studying a population of ocelots, lynx or some
other cat.

For our analyses, cost is characterized by a single covariate raster
and we consider two specific cases. First is an increasing trend from
the NW to the SE (''systematic raster''), where $z(x)$ is defined as
$z(x) = r(x) + c(x)$ and $r(x)$ and $c(x)$ are just the row and
column, respectively, of the raster.  This might define something
related to distance from an urban area or a gradient in habitat
quality due to land use, or environmental conditions such as
temperature or precipitation gradients.  In the second case we make up
a covariate by generating a field of spatially correlated noise to
emulate a typical patchy habitat covariate (''patchy raster'') such as
tree or understory density. The two covariates are shown in
Fig. \ref{ecoldist.fig.raster100}, along with a sample realization of
$N=100$ individuals (left panel only).  For both covariates we use a
cost function in which transitions from pixel ${\bf x}$ to ${\bf x}'$
is given by:

\[
 log(cost({\bf x},{\bf x}'))=  \theta_2 \frac{z({\bf x}) + z({\bf x}')}{2}
\]

{\flushleft where} $\theta_2 = 1$ for simulating the observed data.
 Remember that with $\theta_2=0$ the
model reduces to one in which the cost of moving across each pixel is
constant, and therefore Euclidean distance is operative.

\begin{figure}
\begin{tabular}{cc}
\includegraphics[height=3.25in,width=3.25in]{Ch10/figs/raster_withN100}
\includegraphics[height=3.25in,width=3.25in]{Ch10/figs/raster_krige} &
\end{tabular}
\caption{Two covariate rasters used for simulations. A hypothetical
  realization of $N=100$ activity centers is superimposed on the left,
along with 16 trap locations. }
\label{ecoldist.fig.raster100}
\end{figure}

\subsection{Non-stationarity of home range structure}

When distance is defined by the cost-weighted distance metric given
by Eq. \ref{eq.lcp} then individual space-usage varies
spatially in response to the landscape covariate(s) used in the
distance metric.  As a consequence, home ranges contours are no longer
circular, as in SCR models based on Euclidean distance.
 For example, using one of the covariates we use in
our simulation study below (Fig. \ref{ecoldist.fig.raster100}, right
panel) with a Gaussian pdf detection function but having distance
metric defined by Eq. \ref{eq.lcp}, produces home ranges such
as those shown in Fig. \ref{fig.homeranges}. Later we simulate data
under the model that produces these home ranges and fit spatial
capture-recapture models to evaluate the efficacy of likelihood
estimation under this model.

\begin{figure}
\begin{center}
\includegraphics[height=6in,width=3.75in]{Ch10/figs/home_ranges}
\end{center}
\caption{
Typical home ranges for 6 individuals based on the cost surface shown in
  Fig. \ref{ecoldist.fig.raster100} with $\theta_{2}=1$. The black dot indicates the home
  range center and the pixels around each home range center are shaded
according to the probability of encounter, if a trap were located in
that pixel.
}
\label{fig.homeranges}
\end{figure}


\subsection{Simulation and Analysis}

We begin by simulating some data... we have to load the \mbox{\tt
scrbook} library, use the function \mbox{\tt make.EDcovariates} to generate
our raster covariates, process that into a least-cost path distance
matrix, and then simulate observed encounter data using standard methods
which we have used many times previously in this book. The complete set
of {\bf R} commands is:

{\small
\begin{verbatim}
library("scrbook")
out<-make.EDcovariates()
covariate<-out$covariate.patchy
set.seed(2013)

N<-200
theta0<- -2
sigma<- .5
K<- 5

theta1<- 1/(2*sigma*sigma)
r<-raster(nrows=20,ncols=20)
projection(r)<- "+proj=utm +zone=12 +datum=WGS84"
extent(r)<-c(.5,4.5,.5,4.5)
theta2<-1
cost<- exp(theta2*covariate)
tr1<-transition(cost,transitionFunction=function(x) 1/mean(x),directions=8)
tr1CorrC<-geoCorrection(tr1,type="c",multpl=FALSE,scl=FALSE)

# make up some trap locations
xg<-seq(1,4,1); yg<-4:1
pts<-cbind( sort(rep(xg,4)),rep(yg,4))

traplocs<-pts
points(traplocs,pch=20,col="red")
ntraps<-nrow(traplocs)

S<-cbind(runif(N,.5,4.5),runif(N,.5,4.5))
D<-costDistance(tr1CorrC,S,traplocs)
probcap<-plogis(theta0)*exp(-theta1*D*D)
# now generate the encounters of every individual in every trap
Y<-matrix(NA,nrow=N,ncol=ntraps)
for(i in 1:nrow(Y)){
 Y[i,]<-rbinom(ntraps,K,probcap[i,])
}
Y<-Y[apply(Y,1,sum)>0,]

\end{verbatim}
}


Now we use the {\bf R} function \mbox{\tt nlm} along with
our \mbox{\tt intlik3ed} function to evaluate the likelihood so that we can obtain the MLEs of the
model parameters. We'll do that for both the standard Euclidean distance
and then for the ecological distance based on the ``patchy'' covariate:
{\small
 \begin{verbatim}
frog1<-nlm(intlik3ed,c(theta0,theta1,3)),hessian=TRUE,y=Y,K=K,X=traplocs,
               distmet="euclid",covariate=covariate,theta2=1)

frog2<-nlm(intlik3ed,c(theta0,theta1,3,-.3),hessian=TRUE,y=Y,K=K,X=traplocs,
               distmet="ecol",covariate=covariate,theta2=NA)
\end{verbatim}
}

Show nlm() output for each and comment .......................XXXX

\subsection{Simulation study}

\citet{royle_etal:2012ecol}
carried-out a limited simulation study to evaluate the
general statistical performance of the density estimator under
this new model, the effect of mis-specifying the model with a
normal Euclidean distance metric and whether the parameter of the
cost function could be effectively estimated.
We recapitulate their results here.
For population sizes of 100 and 200 individuals with activity
centers randomly distributed on the $20 \times 20$ landscape, they
subjected individuals
to encounter by 16 traps arranged in a $4\times 4$ grid
using a Gaussian
encounter model with least-cost path distance metric:
\[
log(p_{ij})= \theta_{0} + \theta_{1} d_{lcp}({\bf x}_{j},{\bf
  s}_{i}; \theta_{2})^{2}
\]
where  $\theta_{0} = -2$ and $\theta_{1} = 2$, the latter value
corresponding to $\sigma = 0.5$ of a stationary bivariate normal home
range model.  Different numbers of replicate samples were considered,
$K=3,5,10$
(e.g., nights in a camera trapping study), in order
to produce varying sample
sizes.

Three different models were fitted
to each simulated data set: the
misspecified euclidean distance model; (ii) the true data-generating
model with the relative cost raster {\it known} and (iii) the true
data-generating model but estimating the relative cost parameter by
maximum likelihood.  We used both the ``systematic'' and ``patchy''
covariates defined previously.

\subsection{Simulation Results}

For both landscapes and all simulation conditions (levels of $K$ and
$N$) the average sample sizes of individuals captured are given in
Tab. \ref{tab.samplesize}.  The simulation results for estimating $N$
for the prescribed state-space are presented in Table
\ref{tab.results1}.  For the ``patchy'' landscape we see extreme
bias in estimates of $N$ when the Euclidean distance is used. There is
moderate small sample bias of 3-5\% in the MLE of $N$ using the
least-cost distance which becomes negligible as $K$ increases. For
$N=200$ the bias is on the order of 2\% for the lowest sample size
case ($K=3$) but negligible otherwise.  Interestingly, for the
landscape exhibiting systematic structure, there is a persistent bias
in the MLE of $N$ of 1-3\% even for the highest level of $K$.
We were
initially surprised by this but, in fact, it is due to the fact that
the state-space is small relative to the extent of the trap grid and
sensitivity to a state-space that is too small is expected because the
support of the integrand is truncated. In the particular case of the
systematic landscape, we find that, in the NW corner of the raster
where cost of movement is low, individuals use large areas of space,
and the fitted model is under-stating the apparent
heterogeneity in encounter probability for the prescribed raster.  We
found that the issue is resolved when the traps are moved away from
the boundary (Tab. \ref{tab.results3}).

The performance of estimating the cost parameter $\theta_{2}$ mirrors
the results for estimating $N$ for the prescribed state space. In the
patchy landscape where we don't expect a systematic gradient in space
usage around the edge of the state-space, we see
(Table \ref{tab.results2}) that $\theta_{2}$ is estimated with
diminishing bias as the sample size increases, but with persistent
bias due to truncation of the likelihood under the systematic
landscape which, as with the MLE of $N$, is resolved by moving the
traps away from the edge of the raster. Equivalently, in practice,
this could be resolved by expanding the raster away from the trap
locations so that all regions used by animals exposed to capture are
included in the state-space.



\begin{table}[htp]
\centering
\caption{
Expected sample sizes of captured individuals under each configuration of
$N$ (population size for the prescribed state-space) and $K$ (number of replicate samples).
}
\begin{tabular}{l|rrrr}
 & \multicolumn{2}{c}{Systematic} & \multicolumn{2}{c}{Patchy}  \\
    & N=100 &  N=200  &   N=100 &  N=200  \\ \hline
K=3 &  38.69 &   78.17  &   37.30 &   74.93  \\
K=5 &  51.10 &  103.18  &   51.89 &  103.71 \\
K=10&  65.81 &  132.39  &   69.44 &  138.76 \\
\end{tabular}
\label{tab.samplesize}
\end{table}




\begin{table}[htp]
\label{tab.results1}
{\tiny
\caption{Simulation results for estimating population size $N$ for a prescribed state-space with
$N=100$ or $N=200$ and various levels of replication ($K$) chosen to affect the observed sample
size of individuals (Tab. \ref{tab.samplesize}). For each simulated data set, the SCR model was fitted with
standard Euclidean distance (``euclid''), least-cost path assuming the
cost parameter $\theta_2$ is known (``lcp/known''), or allowing it to
be estimated by maximum likelihood (``lcp/est'').
The summary statistics of the
sampling distribution reported are the mean, standard deviation
(``SD'') and quantiles (0.025, 0.50, 0.975).
}
{\bf Systematic trend raster:} \\
\begin{tabular}{l|rrrrr|rrrrr}
         & \multicolumn{5}{c}{N=100   } & \multicolumn{5}{c}{N=200  }  \\
         &   mean &  SD  & 0.025 & 0.50 & 0.975  & mean  & SD   & 0.025 & 0.50  & 0.975 \\ \hline
K=3      &        &      &       &      &        &       &      &       &       &       \\
euclid   &   63.65& 12.62& 44.77 & 61.17&  90.98 & 126.68& 17.05&  98.93& 124.49& 168.26 \\
lcp/known&   99.28& 20.80& 68.83 & 97.55& 152.59 & 196.47& 27.39& 152.03& 192.96& 259.78\\
lcp/est  &  101.93& 21.68& 67.95 &101.56& 156.21 & 201.58& 28.14& 154.96& 200.15& 263.20\\
K=5      &        &      &       &      &        &       &      &       &       &        \\
euclid   &  64.60 & 7.11 & 51.52 & 63.86&  77.33 & 130.02& 10.25& 113.48& 128.96& 151.32\\
lcp/known&  95.96 &11.64 & 74.21 & 96.16& 117.65 & 193.04& 17.13& 166.84& 191.88& 226.16\\
lcp/est  &  98.94 &12.97 & 74.68 & 99.00& 123.88 & 198.80& 19.60& 166.87& 197.97& 239.46\\
K=10     &        &      &       &      &        &       &      &       &       &       \\
euclid   &  69.24 & 4.83 & 59.37 & 69.47&  79.18 & 139.83&  7.62& 125.65& 139.65& 154.82\\
lcp/known&  94.46 & 7.04 & 81.45 & 94.04& 108.83 & 190.47& 11.55& 170.49& 189.74& 213.19\\
lcp/est  &  97.53 & 8.18 & 82.02 & 97.62& 113.16 & 195.19& 13.28& 171.63& 194.58& 217.96\\ \hline
\end{tabular}
\\
{\bf Patchy "random" raster: } \\
\begin{tabular}{l|rrrrrrrrrr}
         & \multicolumn{5}{c}{N=100  } & \multicolumn{5}{c}{N=200   }  \\
         &   mean &  SD  & 0.025 & 0.50  & 0.975  & mean  & SD   & 0.025 & 0.50  & 0.975 \\ \hline
K=3      &        &      &       &       &        &       &      &       &       &       \\
euclid   &  78.68 & 18.12& 49.40 & 76.34 & 125.47 & 154.34& 33.74& 107.00& 146.34& 221.43\\
lcp/known& 109.09 & 27.52& 69.50 &104.86 & 183.72 & 207.18& 46.53& 143.31& 198.42& 315.89\\
lcp/est  & 110.96 & 28.65& 69.55 &106.98 & 181.84 & 208.77& 49.29& 141.68& 197.89& 325.77\\
K=5      &        &      &       &       &        &       &      &       &       &        \\
euclid   &  77.85 & 11.55& 59.17 & 77.44 & 101.14 & 153.39& 15.57& 129.31& 149.54& 185.38\\
lcp/known& 103.57 & 15.83& 78.15 &100.58 & 137.48 & 201.57& 21.25& 165.94& 199.95& 243.26\\
lcp/est  & 104.44 & 15.79& 78.38 &101.47 & 139.55 & 200.91& 20.78& 164.42& 200.47& 246.46\\
K=10     &        &      &       &       &        &       &      &       &       &       \\
euclid   &  78.01 & 5.26 & 68.00 & 77.96 & 87.81  & 156.27&  8.51& 142.17& 156.05& 174.55\\
lcp/known&  99.84 & 7.09 & 86.86 & 99.84 & 114.11 & 198.64& 11.04& 181.43& 197.62& 220.45\\
lcp/est  & 100.42 & 7.56 & 86.72 &100.34 & 115.47 & 198.45& 11.44& 180.06& 198.04& 219.52\\ \hline
\end{tabular}
}
\end{table}





\begin{table}[htp]
\centering
\caption{
Mean of sampling distribution of the cost function parameter
$\theta_{2}$ for the different simulation
conditions.
}
\begin{tabular}{l|rrrr}
 & \multicolumn{2}{c}{Patchy} & \multicolumn{2}{c}{Systematic} \\
    & N=100 &  N=200  &   N=100 &  N=200  \\ \hline
K=3 &   1.05&    1.03 &     1.17 & 1.14 \\
K=5 &   1.02&    1.01 &     1.12 &1.12 \\
K=10&   1.01&    1.00 &     1.10 &1.08 \\
\end{tabular}
\label{tab.results2}
\end{table}




\begin{table}[htp]
{\tiny
\caption{Simulation results for estimating population size $N$ for a prescribed state-space with
$N=100$ or $N=200$ and various levels of replication ($K$) chosen to affect the observed sample
size of individuals. These results correspond to those of the
systematic landscape in Table 2 except with the traps
moved 0.5 units in from the boundary of the raster.
Each grouping of 3 rows (for a given value of $K$) summarizes the
performance of $\hat{N}$ under 3 distance models: (1) A model in which
Euclidean distance was used (``euclid''); (2) A model in which the
least-cost path distance was used, with the coefficient of the cost
function fixed (``lcp/known''); and (3) A model in which the
coefficient was estimated (``lcp/est''). The summary statistics of the
sampling distribution reported are the mean, standard deviation
(``SD'') and quantiles (0.025, 0.50, 0.975).
}
{\bf Systematic trend raster:} \\
\begin{tabular}{l|rrrrr|rrrrr}
         & \multicolumn{5}{c}{N=100   } & \multicolumn{5}{c}{N=200  }  \\
         &   mean &  SD  & 0.025 & 0.50 & 0.975  & mean  & SD   & 0.025 & 0.50  & 0.975 \\ \hline
K=3      &        &      &       &      &        &       &      &       &       &       \\
euclid   &   84.48& 20.42& 51.16 & 81.51& 140.62 &163.70 &24.55 &126.64 &157.67 &223.63 \\
lcp/known&  104.14& 25.49& 65.67 &101.50& 173.19 &200.16 &29.27 &158.65 &191.04 &268.78\\
lcp/est  &  105.90& 26.19& 65.95 &103.40& 182.30 &201.34 &29.54 &161.88 &192.36 &268.98\\
K=5      &        &      &       &      &        &       &      &       &       &       \\
euclid   & 81.21  &11.33 &61.35  &79.20 & 98.86  &163.27 &13.06 &140.21 &162.97 &185.94\\
lcp/known& 99.93  &12.86 &76.97  &99.75 &117.76  &199.80 &16.60 &170.25 &198.23 &227.66\\
lcp/est  & 100.84 &13.15 &79.96  &99.51 &119.08  &200.25 &16.53 &168.88 &199.29 &227.39\\
K=10     &        &      &       &      &        &       &      &       &       &       \\
euclid   &  80.10 & 7.81 &66.45  &79.14 &93.33   &158.40 & 9.25 &142.74 &157.86 &173.18\\
lcp/known& 100.07 & 9.50 &82.99  &100.33&114.81  &197.62 &12.58 &171.95 &199.21 &217.19\\
lcp/est  & 100.10 & 9.88 &82.31  &100.91&116.27  &197.52 &13.03 &169.49 &200.68 &217.82\\ \hline
\end{tabular}
}
\label{tab.results3}
\end{table}





\section{Illustration: Example Good vs. Bad habitat}

We provide another illustration of how to employ ecological distance
calculations in SCR models. This example shows more GIS-like analysis
for a situation where we have something like a hard habitat boundary
created to mimic a habitat corridor or park unit or some other block
of relatively homogeneous good-quality habitat for some species. This
particular system (shown in Fig. \ref{ecoldist.fig.corridor}), could
be habitat surrounded by a suburban wasteland of McDonalds and
Wal-Marts, much less hospital habitat for most species.  For our
purposes, we suppose that individuals live within the buffered ``f''
shaped region, although we could also imagine the negative of the
situation in which individuals live outside of the region, so that the
polygon represents a barrier (a lake) or bad habitat (an urban area)
or similar.  We describe the steps for creating this landscape
shortly, so that the reader can use a similar process to generate more
relevant landscapes for their own problems.

In this case we're not going to estimate any parameters of the cost
function (though we could) but instead we're going to use ecological
distance ideas only to constrain movement within (or to avoid)
landscape features.  However, the reader is encouraged to adapt the
likelihood function given in the previous section for this specific
case, so that a parameter of the cost function can be estimated.

\subsection{Basic Geographic Analysis in R}

In practical applications our landscape will contain one more more
polygons which delineate good or bad habitat or other important
characterisetics of the landscape.  These might exist as GIS
shapefiles or merely as a text file with coordinates defining polygon
boundaries. To work with polygons in the context of SCR models we need
to create a raster, overly the polygon and assign values to each pixel
depending on whether pixels are in the polygon or not, or how far they
are from polygon boundaries. These operations are relatively easy to
do within a GIS system but we need to be able to do them in ${\bf R}$
and we develop methods for this here.  See also
secs. \ref{mle.sec.shapefile} and \ref{mcmc.sec.state-space} 
for examples of reading in the shapefile and using them to affect
calculations in SCR models.

The first thing we do here is create a set of polygons by
buffering and joining some line segments.
In the {\bf R} library \mbox{\tt scrbook}, we provide
 a function \mbox{\tt make.seg} which allows the user to make such
 lines segments given a
specific trap region.  To involve \mbox{\tt make.seg} we first
create a plot region and then call \mbox{\tt make.seg} which has a
single argument being the number of points used to define the line
segment. In the following set of commands we generate two line
segments, \mbox{\tt l1} consisting of 9 points and \mbox{\tt l2}
consisting of 5 points, and these reside in a geographic region
enclosedd by $[0,10] \times [0,10]$:
{\small
\begin{verbatim}
library("scrbook")
library("sp")
plot(NULL,xlim=c(0,10),ylim=c(0,10))
l1<-make.seg(9)
plot(l1)
l2<-make.seg(5)
plot(l1)
lines(l2)
\end{verbatim}
}

We used this function to create a couple of line segments of class
\mbox{\tt SpatialLines} from the {\bf R} package \mbox{\tt sp}, which
can be loaded from \mbox{\tt scrbook} as  follows
\begin{verbatim}
data("fakecorridor")
\end{verbatim}
This has 2 line files in it (\mbox{\tt l1} and \mbox{\tt l2}) and a
trap locations file (\mbox{\tt traps}).
We use some functions from the {\bf R} packages \mbox{\tt sp} and
\mbox{\tt rgeos} to join and
buffer (by 0.5 units) the two segments. The commands are as follows
and the result is shown in Fig. \ref{ecoldist.fig.corridor}.

{\small
\begin{verbatim}
data("fakecorridor")
library("sp")
library("rgeos")

buffer<- 0.5
par(mfrow=c(1,1))
aa<-gUnion(l1,l2)
plot(gBuffer(aa,width=buffer),xlim=c(0,10),ylim=c(0,10))
pg<-gBuffer(aa,width=buffer)
pg.coords<- pg@polygons[[1]]@Polygons[[1]]@coords

xg<-seq(0,10,,40)
yg<-seq(10,0,,40)

delta<-mean(diff(xg))
pts<- cbind(sort(rep(xg,40)),rep(yg,40))
points(pts,pch=20)

in.pts<-point.in.polygon(pts[,1],pts[,2],pg.coords[,1],pg.coords[,2])
points(pts[in.pts==1,],pch=20,col="red")
\end{verbatim}
}

\begin{figure}
\begin{center}
\includegraphics[height=3.25in,width=3.25in]{Ch10/figs/corridor}
\end{center}
\caption{A made-up corridor or reserve.}
\label{ecoldist.fig.corridor}
\end{figure}


We focus on devising a SCR model for this corridor system and we
imagine that animals will tend to severely avoid leaving the buffered
habitat zone. Therefore, we assign $\mbox{\tt cost}=1$ if a pixel
is within the buffer,
and $\mbox{\tt cost} = 10000$ if a pixel is outside of a
buffer. Therefore the cost to move to a neighboring pixel outside of
the buffered area is $5000.5$ compared to the cost of 1 to move to a
neighboring pixel inside the buffer.

In this example, we're not going to estimate parameters of the cost
function. Therefore, in that case, we can compute the ecological
distance matrix one time and modify our likelihood code to accept the
distance matrix as input. We give that likelihood in the library
\mbox{\tt scrbook} as the function \mbox{\tt intlik3edv2}.
We note also that it provides a vector of 0's and 1's that
define any potential state-space restrictions. i.e., 1 if the pixel is
an element of the state-space and 0 if it is not.
In the analysis of this
simulated data set, we define the state-space to be the buffered
corridor system. The help file for \mbox{\tt intlik3edv2} contains the
script that follows.

Here we simulate $N=200$ guys in the corridor system and so we
restrict out state-space accordingly for purposes of fitting the
model. However we encourage the reader to refit the model without the
state-space restriction (for fitting the model only) and then
contemplate the result.  The code for doing all of this is as follows

{\small
\begin{verbatim}
cost<-rep(NA,nrow(pts))
cost[in.pts==1]<-1      # low cost to move among pixels but not 0
cost[in.pts!=1]<-10000  # high cost

library("raster")
r<-raster(nrows=40,ncols=40)
projection(r)<- "+proj=utm +zone=12 +datum=WGS84"
extent(r)<-c(0-delta/2,10+delta/2,0-delta/2,10+delta/2)
values(r)<-matrix(cost,40,40,byrow=FALSE)
par(mfrow=c(1,1))
plot(r)
points(pts,pch=20,cex=.4)

library("gdistance")
tr1<-transition(r,transitionFunction=function(x) 1/mean(x),directions=8)
tr1CorrC<-geoCorrection(tr1,type="c",multpl=FALSE,scl=FALSE)
costs1<-costDistance(tr1CorrC,pts)
outD<-as.matrix(costs1)
plot(pts,pch=".")
points(pts[in.pts==1,],pch=20,col="red")

library(``scrbook'')
traplocs<-traps$loc
trap.id<-traps$locid
ntraps<-nrow(traplocs)

set.seed(2013)
N<-200
S.possible<- (1:nrow(pts))[in.pts==1]
S.id<-sample(S.possible,N,replace=TRUE)
S<- pts[S.id,]

D<- outD[S.id,trap.id]
eD<- e2dist(S,traplocs)
Dtraps<-outD[trap.id,]

alpha0<- -1.5
sigma<- 1.5
beta<- 1/(2*sigma*sigma)
K<-10

probcap<-plogis(alpha0)*exp(-beta*D*D)
Y<-matrix(NA,nrow=N,ncol=ntraps)
for(i in 1:nrow(Y)){
 Y[i,]<-rbinom(ntraps,K,probcap[i,])
}
Y<-Y[apply(Y,1,sum)>0,]

frog1<-nlm(intlik3edv2,c(-2.5,2,log(4)),hessian=TRUE,y=Y,K=K,X=traplocs,
            S=pts,D=Dtraps,inpoly=in.pts)
frog2<-nlm(intlik3edv2,c(-2.5,2,log(4)),hessian=TRUE,y=Y,K=K,X=traplocs,
            S=pts,D=Deuclid,inpoly=in.pts)
\end{verbatim}
}

In the example that we ran above we compared the result for using
distance-within-the-corridor to normal Euclidean distance and the
results do not differ too much in this single instance. One reason is
that the distance between individuals and traps that they are likely
to be captured in is well-approximated by normal Euclidean distance.


\section{A stream network}

Later we might add a 3rd prototype situation involving a stream network.

We could use ``distance from stream'' to model effects of habitat
and corridors or whatever


\section{Summary and Outlook}


All published applications of SCR models to date have been based on models for the
encounter probability that are functions of the standard Euclidean
distance between individuals and traps. The obvious limitations are
that it is unaffected by landscape or habitat structure and implies
stationary, isotropic and symmetrical home ranges. These are standard
criticisms of the basic SCR model as universally applied in
practice. However, it is not a relevant criticism of the basic
conceptual formulation of SCR models, because, as we have
demonstrated, one can modify the Euclidean distance metric to
accommodate more realistic space usage considerations.  Following
\citet{royle_etal:2012ecol},
we demonstrated how to use
minimum cost-weighted distance (i.e., ``least-cost
path'') between points, and where ``cost'' is characterized by one or
more spatially explicit covariates that are believed to influence
movement or space-usage of individuals.

How animals use space and therefore how distance to a trap is
perceived by individuals is not something that can ever be known. We
can only ever conjure up models to describe this phenomenon and fit
those models to limited data on a sample of individuals during a
limited amount of time.  Here we have shown that there is hope to
estimate parameters, from capture-recapture data, that describe how
animals use space and thereby allow for irregular home range geometry
that is influenced by landscape structure.

Not surprisingly, our simulation study demonstrated
(Table 2) that the MLE of model parameters is
approximately unbiased in moderate sample sizes. Moreover, the effect
of ignoring ecological distance and using normal Euclidean distance in
the model for encounter probability, has the logical effect of causing
negative bias in estimates of $N$.  We expect this because the effect
is similar to failing to model heterogeneity. i.e., if we mis-specify
``model $M_h$'' \citep{otis_etal:1978} with ``model $M_0$''
\citep{otis_etal:1978} then we will expect to under-estimate $N$. So
the effect of mis-specifying the ecological distance metric with a
standard homogeneous Euclidean distance has the same effect. As a
practical matter, it stands to reason that many previous applications
of SCR models based on homogeneous distance metrics have under-stated
density of the focal population.

In our view, this bias is not really the most important reason to
consider models of ecological distance. Rather, inference about the
structure of ecological distance is fundamental to many problems in
applied and theoretical ecology related to modeling landscape
connectivity, corridor and reserve design, population viability
analysis, gene flow, and other phenomena.  Our new model allows
investigators to evaluate landscape factors that influence movement of
individuals over the landscape from non-invasively collected
capture-recapture data.  Therefore SCR models based on ecological
distance metrics might aid in understanding
aspects of space usage and movement in animal populations and, ultimately, in addressing conservation-related problems such as corridor design.

We considered inference for ecological distance models based on
marginal likelihood \citep{borchers_efford:2008}
(see Chapt. \ref{chapt.mle}).
In principle,
Bayesian analysis does not pose any unique challenges for this new
class of models, except that computing the cost-weighted distance is
computationally intensive.  So, having to do this at each iteration of
an MCMC algorithm may be impractical using existing algorithms.  A
related issue is that the size of the raster slows things down. For
very large rasters, even likelihood analysis can be computationally
challenging and methods for efficient calculation of the ecological
distance given the raster covariate(s) and parameters might be needed.






























%\chapter{Ecological Distance Models in Spatial Capture-Recapture}
%\label{chapt.ecoldist}




\chapter{%State-space Covariates
%Modeling Spatial Variation in Density Using State-Space Covariates
Modeling Spatial Variation in Density
}
\markboth{Spatial Variation in Density}{}
\label{chapt.state-space}

\vspace{0.3cm}

\begin{comment} ok this is a minor tech mpoint for now: but this is introduced as a ``point process''
but what is being decribed here is a REALiZATION of a point process. Lets clarify this in the final
draft
\end{comment}
Underlying all spatial capture-recapture models is a point process
model that describes the distribution of individual activity
centers (${\bf s}$) within the state space ($\cal{S}$).
%, which is
%typically a two-dimensional polygon defining the study area.
Point process models are charcterized by $\mathcal{S}$ and by an
intensity parameter defined at each point in $\mathcal{S}$. If this
intensity is constant, the point process is said to be homogeneous,
and thus far we have focused our
attention on the homogeneous binomial point process whose realized
values are:
${\bf s}_i \sim \mbox{Unif}({\cal S}), i=1,2,\dots,N$, where $N$ is the
size of the population. This is a model of
``spatial-randomness''\footnote{The phrase ``complete
  spatial-randomness'' is reserved for the homogeneous Poisson point
  process}
because the intensity of the
activity centers is constant across the study area.
% and the activity
%centers are distributed independently of each other.

The spatial-randomness assumption is often viewed as restrictive
because ecological processes such as
territoriality and habitat selection can result in non-uniform
distributions of organisms. We have argued, however, that this
assumption is less restrictive than may be recognized because the
homogeneous point process actually allows for infinite
possible configurations of activity centers. Furthermore, given enough data,
the uniform prior will have very little influence on the estimated
locations of activity centers. Nonetheless, the homogeneous point
process model does not allow one to model population density using
covariates, which is a central objective of much ecological research.
For example, a homogeneous point process model
may result in a density surface map indicating that individuals were
more abundant in one habitat than another, but it does not do so
explicitly and so cannot be used to make predictions about
habitat-specific abundance in other regions. A more direct approach would be to replace
the homogeneous model with an inhomogeneous model in which the point process
intensity varies spatially.
%density using covariates as is done in generalized linear models (GLMs)
%\citep{mccullagh_nelder:1989}. % where a
%link function is used to connect the intensity parameter to the linear
%predictor.

In this chapter we present a method
for fitting inhomogeneous binomial point process models by modeling
the intensity parameter as a function of
covariates in much the same way as is done with generalized linear
models. The covariates we consider differ
from those covered in previous chapters, which were typically
attributes of the animal ({\it e.g.} sex or age) or the trap ({\it
  e.g.} baited or not) and were used to model movement or encounter
rate. In contrast, here we wish to
model covariates that are defined for all points in
$\cal{S}$, which we will refer to as
state-space covariates or density covariates. These may
include continuous covariates such as elevation, or discrete
covariates such as habitat type.

Inhomogeneous Poisson point process models were discussed in the original
formulation of SCR models \citep{efford:2004} and were described in
detail by \citet{borchers_efford:2008}. Our approach is
similar to that of \citet{borchers_efford:2008}, except that we use a binomial
rather than a Poisson model because the binomial model is
easily integrated into our MCMC algorithm.  %data augmentation scheme
%and is consistent
%with the objective of determining how a {\it fixed} number of activity
%centers are distributed with respect to covariates.
The method we use to accommodate inhomogeneous binomial point process
models %within our MCMC algorithm
is simple---we
replace the uniform prior with a prior describing the
distribution of the $N$ activity centers conditional on the
covariates. Development of this prior, which does not have a
standard form, is a central component of this chapter. First we
begin with a review of homogeneous point process models.


\section{Homogeneous point process revisited}

The homogeneous Poisson point process is \textit{the} model of ``complete
spatial randomness'' and is often used in ecology as a null model
to test for departures from randomness
\citep{diggle:2003,illian_etal:2008}. Given its central role in the
analysis of point processes, it is helpful to compare it with
the binomial model that we use in our SCR models. The sole parameter
of the homogeneous Poisson point process model is the
intensity parameter $\mu$ which describes the expected number
% start with \mu(s)???? Or, wait for inhomogeneous case?
of points in an infinitesimally small area. %Note that this intensity
%parameter is a single value, i.e. it does not vary spatially.
The intensity parameter can also be used to compute the expected number of points
in any region $B$ of the state-space $\cal{S}$. Specifically,
$\mathbb{E}[n(B)] = A(B)\mu$ where $A(B)$ is the area of region $B$.
This just says that
the expected number of points is the area of $B$
multiplied by the intensity parameter.
%This is one
%of the distinctions between the Poisson model and the binomial model,
%for which the counts $\{n(B_k)\}$ are not i.i.d., as we will explain
%shortly.

An important distinction between the Poisson point process and the
binomial point process is that $N$ is a random variable in the former
model but not in
the latter. In other words, the binomial point process conditions on $N$.
Here is some simple \R~code to illustrate this point:
\begin{verbatim}
mu <- 4                            # intensity
Np <- rpois(1, mu)                 # Np is random
PPP <- cbind(runif(Np), runif(Np)) # Poisson point process

Nb <- 4                            # Nb is fixed
BPP <- cbind(runif(Nb), runif(Nb)) # Binomial point process
\end{verbatim}
which generates realizations from Poisson and binomial point
processes in the unit square ($\mathcal{S} = [0,1]\times[0,1]$).
For both models, the $N$ points are
%independent of one another and
distributed uniformly
in $\mathcal{S}$, and they have the same intensity parameter,
$\mu=4$. However, in the binomial case the intensity parameter is
defined different, being a function of $N$ and the area of the
state-space, $\mu = N/A(\mathcal{S})$.

Another distinction between the two models is that if we divide the
state-space into $K$ disjunct regions, the number of points in each
region $\{ n(B_k): k=1,\dots,K \}$ are
independent and identically distributed (i.i.d.) under the Poisson model,
but some dependence exists under the binomial model.
%In the Poisson case
%we have $\n(B_k) \sim \text{Pois}(A(B_k)\mu)$, and if the points were independent
%$n(B_k) \sim Bin(N, p(B_k))$ for the binomial,
%where $p(B_k)$ is simply the proportion of the state-space in region
%$B_k$.
Fig.~\ref{state-space.fig.homo} illustrates this point.
The depicted state-space is the unit square, and thus the probability of a
point falling in each of the 25 disjunct regions is $p(B_k) = 1/25$ and
the expected counts are $\mathbb{E}(n(B_k)) = Np_k$.
%In
%the figure $N=50$, and consequently we would expect 2 points per pixel, which
%happens to be the empirical mean in this instance.
However, these counts are not
independent realizations from a binomial distribution since $\sum_{k=1}^K
n(B_k) = N$. Instead, the model for the entire vector
is ${n(B_1), n(B_2), \dots, n(B_k)} \sim \mbox{Multin}(N, \{p(B_1), p(B_2), \dots,
p(B_K) \})$ \citep{illian_etal:2008}.
%\begin{verbatim}
%n.Bk <- rmultinom(1, size=50, prob=rep(1/25, 25))
%matrix(n.Bk, 5, 5)
%\end{verbatim}
The dependence among counts has virtually
no practical consequence when the number of pixels is large. For
example, if there are 100 pixels, the number of points in one pixels
carries very little information about the expected number of points in another
pixel. However, if there are only 2 pixels, then clearly the number of
points in one pixel allows one to determine how many points will occur in the
remaining pixel.
%To gain familiarity with the multinomial distribution
%and the discrete representation of space, use the \verb+rmultinom+
%function in \R~to simulate counts similar to those shown in
%Fig.~\ref{state-space.fig.homo}, for example using commands
%such as:
%\begin{verbatim}
%n.Bk <- rmultinom(1, size=50, prob=rep(1/25, 25))
%matrix(n.Bk, 5, 5)
%\end{verbatim}


\begin{figure}[ht!]
\centering
\includegraphics[width=5in,height=2.5in]{Ch11/figs/homoPlots}
\label{state-space.fig.homo}
\caption{Homogeneous binomial point process with $N$=50 points
  represented in continuous and discrete space.}
\end{figure}


The discrete space representation of the binomial point process is of
practical importance when fitting SCR models because spatial covariates
are almost always represented in a discrete-space format called
``rasters'' in GIS-speak. In such cases, we often need to change our
definition of the prior for an activity center from ${\bf s}_i \sim
\mbox{Unif}(\cal{S})$ to ${\bf s}_i \sim \mbox{Multin}(1, \mathbf{\pi})$. In the
latter case, the activity center is simply defined as an integer
representing pixel ``id''.
%Note also that the multinomial distribution
%with an index of 1 (\emph{i.e.} \verb+size=1+ in \verb+rmultinom+)
%is referred to as the categorical distribution,
%which we will frequently use in the \verb+BUGS+ language.



\section{Inhomogeneous binomial point process}

\hl{Check for x instead of s throughout}

As with the homogeneous model, the inhomogeneous binomial point process
model is developed conditional on $N$. The primary distinction is that
the uniform distribution is replaced with another distribution
allowing for the intensity parameter to vary spatially. To arrive at
this new distribution, replace the scalar intensity parameter $\mu$
with the function $\mu(s, {\bm \beta})$, where $\bm \beta$ is a
vector of coefficients describing the effects of
spatially-referenced covariates on the point process intensity. In
what follows, we will often abbreviate the intensity function as $\mu(s)$,
dropping the vector of coefficients for readability. Since an intensity must be strictly
positive, and because the logarithm is the canonical link function of the Poisson
generalized linear model, it is natural to model $\mu(s, \beta)$ as
\[
\log(\mu(s, \beta)) = \beta_0 + \sum_{j=1}^J \beta_j v_j(s), \quad  x \in \cal{S}
\]
where $\beta_j$ is the regression coefficient for covariate
$v_j(s)$. To be clear, $v(s)$ is the value of any covariate, such as
habitat type or elevation, at location $x$ and it assumed to be
defined at all locations in the state-space.
This equation should look
familiar because it is the standard linear predictor used in log-linear
GLMs. One caveat is that the intercept $\beta_0$ is not a
unique parameter to be estimated.
%Note, however, that we have not included
%an intercept. The reason for this is that it would be confounded with
%$N$ (see Chapt. \ref{chapt.hscr}).
The reason for this is that %This should be intuitive since
$\beta_0$ represents population density at the location $x$ when
%, the expected value of $N$ in some infinitesimally
%small area when
the other $\beta$'s equal 0. However, we already
have a parameter in the model for expected abundance, namely $\mathbb{E}[D] =
N/A(\mathcal{S}) =  \psi M / A(\mathcal{S})$\footnote{Remember, $M$ is the size of the augmented population, and
$\psi$ is the probability that a member of $M$ is an actual
constituent of the population (Chapt. ~\ref{chapt.scr0}).}. Thus, in
practice, we can either remove $\beta0$ and model a value proportional
to the intensity, or we can define $\beta_0=\log(N/A(\mathcal{S}))$ and model the
intensity directly.
%an intercept would be
%redundant, and without it we are still able to achieve our goal of
%describing the distribution of $N$ activity centers as a function of
%spatial covariates. One caveat is that if we wish to make predictions
%to unsampled regions, it is useful to include the intercept $\beta_0 =
%\log(N)$.

Now that we have a model of the intensity parameter $\mu(s)$,
we need to develop the associated probability density function
$[\bf s]$ to use
in place of the uniform prior. Remembering that
the integral of a pdf must be unity, we can create the pdf
$[\bf s]$ by dividing
$\mu(s)$ by a normalizing constant, which in this case is the integral
of $\mu(s)$ evaluated over the entire state-space.
The probability density function is therefore
\begin{equation}
[\mathbf{s}] = \frac{\mu(s, \beta)}{\int_{x \in \mathcal{S}} \mu(s, \beta)\, \mathrm{d}s}
\label{eq.pdf.ipp}
\end{equation}
Substituting this distribution for the
uniform prior allows us to fit inhomogeneous binomial point process
models to spatial capture-recapture data. We can also use this
distribution to obtain the expected number of individuals in any given
region. Specifically, the proportion of $N$ expected to occur in any
region $B$ %when heterogeneity in density is present
is $p(B) = \int_B
f(s, \beta)\, \mathrm{d}x$. These are
also the multinomial cell probabilities if the regions are
disjoint and compose the entire state-space. We provide an example in
the next section, and in Fig.\ref{state-space.fig.hetero}.

As a practical matter, note that the integral in the
denominator of $f(s, \beta)$ is evaluated over space, and since we always regard
space as two-dimensional, this is a two-dimensional integral that can
be approximated using the methods discussed in
Chapter~\ref{chapt.poisson-mn}, which include
Monte Carlo integration and Gaussian quadrature. Alternatively, if
our state-space covariates are in raster format, \emph{i.e} they are
in discrete space, the integral can be replaced with a sum over
all pixels,
\begin{equation}
f(s, \beta) = \frac{\mu(s, \beta)}{\sum_{x \in \mathcal{S}} \mu(s, \beta)\, \mathrm{d}x}
\label{eq.pdf.dipp.d}
\end{equation}
which is much more efficient computationally.

Although the discrete space approach is standard practice, it is
technically unjustified because covariate values must be known for all
points in space. This same problem is present anytime that we have a
sample of the spatial covariates, rather than a function defining
their value for all points in space. In such cases, it may be necessary to
interpolate the values of the covariates for points in space where
they were not measured. One option would be to use a Kriging
interpolator, as demonstrated by \citet{rathbun:1996}. Another option
is to sample the spatial covariates using probabalistic sampling
methods, which allow for design-based estimators of their values for
the entire study area \citep{rathbun_etal:2007}. Either option could
be implemented as part of the MCMC algorithm, but even though such
approaches are technically necessary, we do not demonstrate them here
because it seems likely that they will be inconsequential in most
cases where the raster data are of high resolution, such that the loss
of information is negligible when going from continuous space to
discrete space.

We now have all the tools needed to fit inhomogeneous point process
(IPP) models. If we refer to the distribution for the
inhomogeneous point process as ``IPP'', we can write a
hierarchical description of a SCR model with a Poisson encounter process and
a half-normal detection function as
\begin{gather*}
w_i \sim \mbox{Bern}(\psi) \\
{\bf s_i} \sim \mbox{IPP}(\mu(s,\beta)) \\
\lambda_{ij} = \lambda_0 \exp(-\|{\bf s_i} - {\bf x_{j}}\|^2/(2\sigma^2)) \\
y_{ij} \sim \mbox{Poisson}(\lambda_{ij} w_i)
\end{gather*}
The use of $\mbox{IPP}(\mu(s, \beta))$ instead of
$\mbox{Unif}(\cal{S})$ is the only difference between a homogeneous
point process model and an inhomogeneous point process model, and the
two are equivalent when $\beta=0$.

\begin{comment}
The IPP for the activity centers
results in another IPP for the observation process, $\lambda(s)$, the
expected number of captures for a trap
at point. As was true for the homogeneous model, this
intensity function is a product of the point process intensity
and the encounter rate function, $\lambda(s) = \mu(s, {\bm \beta})
\lambda_{ij}$.
\end{comment}

In the next sections we walk through a few examples, building up from
the simplest case where we actually observe the activity centers as
though they were data. In the second example, we fit our new model to simulated
data in which density is a function of a single continuous
covariate. \hl{To build upon the developments in the previous chapter, we
further consider the plausible case where a state-space covariate is also a
covariate of ecological distance.} A small simulation study indicates
that both effects can be estimated. A fourth example shows an analysis in discrete space using
both \secr~\citep{efford:2011} and \jags~\citep{plummer:2003}. In the
fifth and final example, we model the intensity of
activity centers for a real dataset collected on jaguars
(\emph{Panthera onca}) in Argentina.

\section{Observed Point Processes}

In SCR models, the point process is not directly observed, but in
other contexts it is. Examples include the locations of disease
outbreaks, the locations of trees in a forest, or the locations of
radio-tracked animals. Indeed Eq.~\ref{eq.pdf.ipp} has been used
extensively in the radio-telemetry literature to model so-called
``resource selection functions'' \citep{manly_etal:2002,lele_keim:2006}.
When the point locations are directly observed,
estimating the parameters $\bf \beta$ is straight-forward as
demonstrated in the following example. This example also illustrates
the fundamental process that we will later embed in our MCMC algorithm
used to fit SCR models with IPP.

Suppose we knew the locations of 100 animals' activity
centers, perhaps as the result of an extensive telemetry study. To
estimate the intensity surface $\mu(s, \beta)$ underlying these
points, we need to derive the likelihood for our data under this
model. Given the probability density function $f(s, \beta)$
(Eq.~\ref{eq.pdf.ipp}) and assuming that the points are
mutually independent of one another,
the likelihood is given by the product
of $R$ such terms, where $R=100$ is the sample size in our
hypothetical example,
\emph{i.e.} the observed number of activity centers.
\[
\mathcal{L}({\bf \beta} | {\bf x}_i, \beta) = \prod_{i=1}^R f(s_i)
\]
Having defined the likelihood we could choose a prior distribution for
$\beta$ and obtain the posterior distribution of
$\bf \beta$ using Bayesian methods, or we can find the maximum likelihood
estimates (MLEs) using standard numerical methods as is demonstrated
below.

First, we simulate some data. Simulating data under an inhomogeneous point process model is often
accomplished using indirect methods such as rejection
sampling. Rejection sampling proceeds by
simulating data from a standard distribution and then accepting or
rejecting each sample using probabilities defined by the distribution
of interest. For more information, readers should consult an
accessible text such as \citet{robert_casella:2010}. In our example, we
simulate from a uniform distribution and then accept or reject using
the (scaled) probability density function $f(s, \beta)$. Note that we first define a
spatial covariate (elevation) that is a simple function of the spatial
coordinates increasing from the southwest to the northeast of our
state-space.\footnote{Such functional forms of
covariates are rarely available. Instead,  continuous spatial
covariates are more often measured on a discrete grid.}

The following \R~commands demonstrate the use of rejection sampling to
simulate an inhomogeneous point process for the covariate depicted in
Fig.~\ref{state-space.fig.hetero}. The code uses the \verb+cuhre+ function in
the {\tt R2Cuba} package to integrate the intensity function over
space \citep{hahn_etal:2011}. An alternative would be to evaluate the
integral on a fine grid of points as we have done in previous
chapters, but it is useful to gain familiarity with more efficient
integration functions in \R.

\begin{small}
\begin{verbatim}
# spatial covariate (with mean 0)
elev.fn <- function(s) x[1]+x[2]-1
# intensity function
mu <- function(s, beta) exp(beta*elev.fn(s=x))

# Simulate IPP using rejection sampling
set.seed(300225)
N <- 100
count <- 1
s <- matrix(NA, N, 2)
beta <- 2 # parameter of interest
elev.fn <- function(s) x[1]+x[2]-1
# Intensity function, mu(s,beta)
mu <- function(s, beta) exp(beta*elev.fn(x=x))
# 2-dimensional integration over space
int.mu <- R2Cuba:::cuhre(2, 1, mu, beta=beta)$value
elev.min <- elev.fn(c(0,0)) #elev.fn(cbind(0,0))
elev.max <- elev.fn(c(1,1)) #elev.fn(cbind(1,1))
Q <- max(c(exp(beta*elev.min) / int.mu,   #2d(beta),
           exp(beta*elev.max) / int.mu))   #2d(beta)))
while(count <= 100) {
  x.c <- runif(1, 0, 1); y.c <- runif(1, 0, 1)
  s.cand <- c(x.c,y.c)
  pr <- exp(beta*elev.fn(s.cand)) / int.mu #2d(beta)
  if(runif(1) < pr/Q) {
    s[count,] <- s.cand
    count <- count+1
    }
  }
\end{verbatim}
\end{small}


\begin{figure}[ht]
\centering
\includegraphics[width=5in,height=2.5in]{Ch11/figs/heteroPlots}
\label{state-space.fig.hetero}
\caption{An example of a spatial covariate, say elevation, and a
  realization of a inhomogeneous binomial point process with $N$=100
  and $\mu(s) = exp(\beta \mbox{elev}(s))$ where $\beta=2$.}
\end{figure}

The simulated data are shown in Fig~\ref{state-space.fig.hetero}. High elevations
are represented by light green and low elevations by dark green. The
activity centers of 100 animals are shown as
points, and it is clear that these simulated animals prefer the high
elevations.  %Perhaps they are mountain goats.
The underlying model describing this preference is
$\log(\mu(s)) = \exp(\beta \times elev(s))$
where $\beta=2$ is the parameter to be estimated.

Given these points, we will now estimate $\beta$ by minimizing the
negative-log-likelihood using \verb+R+'s \verb+optim+ function.

\begin{small}
\begin{verbatim}
# Negative log-likelihood
nll <- function(beta) {
    int.mu <- R2Cuba:::cuhre(2, 1, mu, beta=beta)$value
    -sum(beta*elev.fn(s) - log(int.mu))
}
starting.value <- 0
fm <- optim(starting.value, nll, method="Brent",
            lower=-5, upper=5, hessian=TRUE)
c(Est=fm$par, SE=sqrt(1/fm$hessian)) # estimates and SEs
\end{verbatim}
\end{small}


Maximizing the likelihood took a small fraction of a second, and we
obtained an estimate of $\hat{\beta}=1.99$. We could plug
this estimate into our linear model at each point in the state-space to
obtain the MLE for the intensity surface.

This example demonstrates
that if we had the data we wish we had, {\it i.e.} if we knew the
coordinates of the activity centers $\bf s$, we could easily estimate the
parameters governing the underlying point process. Unfortunately, in
SCR models, the activity centers cannot be directly observed, but
spatial re-captures provide us with the information needed to
estimate these latent parameters.

\section{Fitting inhomogeneous point process SCR models}

\subsection{Continuous space}

One of the nice things about hierarchical models is that they allow us
to break a problem up into a series of simple conditional
sub-models. Thus,
we can simply add the methods described above into our existing MCMC
algorithm to simulate the posterior distributions of $\beta$ conditional on the
simulated values of $\mathbf{s}$. To demonstrate, we will continue with
the previous example. Specifically, we will overlay a grid of
traps on the map shown in Fig.~\ref{state-space.fig.hetero}. We will then
simulate capture histories conditional upon the activity
centers. Then, we will attempt to estimate the activity center
locations as though we did not know where they were, as is the case in
real applications.

The following \R~code simulates encounter histories under a
Poisson observation model (see Chapt. \ref{chapt.poisson-mn}), which could be appropriate in camera
trapping studies or when using other methods in which animals could
be detected multiple times at a trap during a single occasion.

\begin{small}
\begin{verbatim}
# Create trap locations
xsp <- seq(-0.8, 0.8, by=0.2)
len <- length(xsp)
X <- cbind(rep(xsp, each=len), rep(xsp, times=len))

# Simulate capture histories, and augment the data
ntraps <- nrow(X)
T <- 5
y <- array(NA, c(N, ntraps, T))

nz <- 50 # augmentation
M <- nz+nrow(y)
yz <- array(0, c(M, ntraps, T))

sigma <- 0.1  # half-normal scale parameter
lam0 <- 0.5   # basal encounter rate
lam <- matrix(NA, N, ntraps)

set.seed(5588)
for(i in 1:N) {
    for(j in 1:ntraps) {
        distSq <- (s[i,1]-X[j,1])^2 + (s[i,2] - X[j,2])^2
        lam[i,j] <- exp(-distSq/(2*sigma^2)) * lam0
        y[i,j,] <- rpois(T, lam[i,j])
    }
}
yz[1:nrow(y),,] <- y # Fill
\end{verbatim}
\end{small}

Now that we have a simulated capture-recapture dataset $y$, and we have
augmented it to create the new data object $yz$, we are ready to
begin sampling from the posteriors. A commented Gibbs sampler written
in \R~is available in the accompanying \R~package \scrbook~(see
?scrIPP).
\begin{comment} see Ch 7 MCMC for SCR and cite some section of that \end{comment}
% There are two small parts of the
% \R~code that distinguish it from previous code we have shown to
% fit homogeneous point processes. First, we need to update the parameter
% ${\bf \beta}$ conditional on all other parameters in the model. The code to
% do so is: %\begin{comment} need cite to Ch 2 or 7 on MCMC for this \end{comment}
% \begin{small}
% \begin{verbatim}
% # Denominator of f(x, beta). Integral of mu(x, beta) over space
% D1 <- cuhre(2, 1, mu, lower=c(xlims[1], ylims[1]),
%             upper=c(xlims[2], ylims[2]), beta=beta1)$value
% # Compute the denominator again using a proposed beta1
% beta1.cand <- rnorm(1, beta1, tune[3])
% D1.cand <- cuhre(2, 1, mu, lower=c(xlims[1], ylims[1]),
%                  upper=c(xlims[2], ylims[2]), beta=beta1.cand)$value
% # Compute log(f(x))
% ll.beta1 <- sum(  beta1*elev.fn.v(S) - log(D1) )
% ll.beta1.cand <- sum( beta1.cand*elev.fn.v(S) - log(D1.cand) )
% if(runif(1) < exp(ll.beta1.cand - ll.beta1) )  {
%      beta1<-beta1.cand
% }
% \end{verbatim}
% \end{small}
% Next, we need to put the new prior on the activity centers:
% \begin{small}
% \begin{verbatim}
% # Compute the prior for s_i and a candidate. denominator is constant
% prior.S <- beta1*elev(S[i,1], S[i,2]) # - log(D1)
% prior.S.cand <- beta1*elev(Scand[1] + Scand[2]) # - log(D1)
% if(runif(1)< exp((ll.S.cand+prior.S.cand) - (ll.S+prior.S))) {
%     S[i,] <- Scand
%     lam <- lam.cand
%     D[i,] <- dtmp
%     }
% \end{verbatim}
% \end{small}
We can apply this modified sampler to our data using the
following \R~commands:
\begin{small}
\begin{verbatim}
set.seed(3434)
fm1 <- scrIPP(yz, X, M, 6000, xlims=c(0,1), ylims=c(0,1),
            tune=c(0.003, 0.08, 0.3, 0.07) )
plot(mcmc(fm1$out))
rejectionRate(mcmc(fm1$out))
\end{verbatim}
\end{small}
We obtain posterior distributions that are summarized in
Table~\ref{ch9.tab.simIPP}.
%Mixing is good, and as usual,
%life is very nice when we are working with simulated data.

\begin{table}[b]
\centering
\caption{Posterior summaries from inhomogeneous point process model
  fitted to simulated data. Space was treated as continuous.}
\begin{tabular}{lrrrrr}
\hline
& Mean & SD & 2.5\% & 50\% & 97.5\% \\
\hline
 $\sigma =0.10$ &   0.1026 &   0.0048 &   0.0935 &   0.1025 &   0.1123 \\
 $\lambda_0=0.50$ &   0.4419 &   0.0493 &   0.3496 &   0.4400 &   0.5390 \\
 $\psi =0.66$ &   0.6826 &   0.0554 &   0.5762 &   0.6820 &   0.7923 \\
 $\beta =2.00$ &   2.1601 &   0.3390 &   1.5193 &   2.1583 &   2.8043 \\
 $N =100$ & 102.7696 &   6.2689 &  92.0000 & 102.0000 & 117.0000 \\
\hline
\end{tabular}
\label{ch9.tab.simIPP}
\end{table}


Fitting continuous space IPP models is somewhat
difficult in \bugs~because our prior ``IPP'' is not one of the
available distributions that come with the software. It is
possible to add new distributions in \bugs, but it is somewhat
cumbersome.  \secr~allows
users to fit continuous space IPPs using polynomials of the x- and y-
coordinates, but it does not accept truly continuous covariates that
are functions of space. However, these
are not really important limitations because discrete
space versions of the IPP model are straight-forward, and virtually all spatial
covariates are, or can be, defined as such.


\subsection{Discrete space}

To fit IPPs using covariates in discrete space, \emph{i.e.} in raster
format, we follow the same steps
as outlined in Chapter~\ref{chapt.poisson-mn}---we define ${\bf s}_i$ as
pixel ID, and we use the categorical distribution as a prior. A good
example is found in \citep{mollet_etal:2012}. Here we present
an analysis of the simulated data shown in the %right panel of
Fig.~\ref{state-space.fig.hetero}. The spatial covariate, let's call it
elevation again, was simulated
using using the code shown on the help page
\verb+ch9simData+ in \scrbook. The points are the number of
activity centers in each pixel, generated from a single realization of
the inhomogeneous point process model with intensity
$\mu(x) = 2 \times \mbox{elev}(s)$.
\begin{figure}[ht]
\centering
\includegraphics[width=3in,height=3in]{Ch11/figs/discrete}
\label{ch9.fig.discrete}
\caption{Simulated activity centers in discrete space. The spatial
  covariate, elevation, is highest in the lighter areas. Density of
  activity centers (circles) increases with elevation. A single
  activity center is shown as a small circle, and larger circles
  represent two activity centers in a pixel. Trap locations
  are shown as crosses.}
\end{figure}

The \bugs~code to fit an IPP model to these data is shown in
panel~\ref{ch9.panel1}.The vector \verb+probs[]+ is the prior
probability defined
by~\ref{eq.pdf.ipp.d}, which is the probability that an individual's
activity center is located at pixel $x$. \verb+Sgrid+ is the
matrix of coordinates for each pixel.

%\begin{panel}[h!]
%\centering
%\rule[0.15in]{\textwidth}{.03in}
\begin{small}
\begin{verbatim}
model{
sigma ~ dunif(0, 1)
lam0 ~ dunif(0, 5)
beta ~ dnorm(0,0.1)
psi ~ dbeta(1,1)
for(x in 1:nPix) {
  theta[x] <- exp(beta*elevation[j])
  probs[x] <- theta[j]/sum(theta[])
}
for(i in 1:M) {
  w[i] ~ dbern(psi)
  s[i] ~ dcat(probs[])
  x0g[i] <- Sgrid[s[i],1]
  y0g[i] <- Sgrid[s[i],2]
  for(j in 1:ntraps) {
    dist[i,j] <- sqrt(pow(x0g[i]-grid[j,1],2) +
                      pow(y0g[i]-grid[j,2],2))
    lambda[i,j] <- lam0*exp(-dist[i,j]*dist[i,j] /
                            (2*sigma*sigma)) * w[i]
    y[i,j] ~ dpois(lambda[i,j])
    }
  }
N <- sum(w[])
Density <- N/1 # unit square
}
\end{verbatim}
\end{small}
%\rule[0.15in]{\textwidth}{.03in}
%\caption{\bugs~code for fitting inhomogeneous point process model in
%  discrete space.}
%\label{ch9.panel1}
%\end{panel}

This model can also be fit in \secr, which refers
to the raster data as a ``habitat mask''. \R~code to
fit the models using \secr~and \jags~is available in \scrbook---see
\verb#help(ch9secrYjags)#. Results of the
comparison are shown in Table \ref{ch9:tab:secrYjags} and are
very similar as expected.
\begin{comment}
\hl{ANDY, is there any point in discussing
  the slight differences?}
  YES: If we can explain it. Could it be MC error alone?
Otherwise I guess attributing it to differences between MLE and BAyes is ok. That seems like
a reasonable thing.
\end{comment}
\begin{table}[h!]
\centering
\caption{Comparison of \secr~and \jags~results. Point estimates from
  the Bayesian analysis are posterior means. Intervals are lower and
  upper 95\% CIs.}
\begin{tabular}{llrrrr}
\hline
Software & Parameter & Estimate & SD & lower & upper \\
\hline
 secr & $N=50$ & 49.2803 & 5.7535 & 41.0087 & 64.3879 \\
      & $\beta=2$ &  2.1772 & 0.5628 &  1.0741 &  3.2804 \\
      & $\lambda_0=0.8$ &  0.9203 & 0.0764 &  0.7824 &  1.0825 \\
      & $\sigma=0.1$ &  0.0990 & 0.0038 &  0.0918 &  0.1068 \\
\hline
 JAGS & $N=50$ & 48.2072 & 5.4053 & 39.0000 & 60.0000 \\
      & $\beta=2$ &  2.1026 & 0.5323 &  1.0889 &  3.1506 \\
      & $\lambda_0=0.8$ &  0.9328 & 0.0766 &  0.7898 &  1.0921 \\
      & $\sigma=0.1$ &  0.1004 & 0.0041 &  0.0929 &  0.1089 \\
\hline
\end{tabular}
\label{ch9:tab:secrYjags}
\end{table}


\section{Ecological distance and state-space covariates}

Habitat characteristics that affect population
density could also affect home range size and movement behavior. For
example, a
species that occurs in high density in a forest may be reluctant to
venture from a forest patch into an adjacent field. Thus, even if a
trap placed in a field is located very close to an animal's activity
center, the probability of capture may be very low. In this case
forest cover is a covariate of both density and encounter probability,
and we could model it as such by combining the methods described in
this chapter and in Chapter~\ref{chapt.ecoldist}. To demonstrate, we
continue with our analysis of the data shown in
Fig~\ref{state-space.fig.hetero}. Once again, we suppose that density
increases with elevation, but this time, we also make the
assumption that home range size decreases as density increases. This
commonly-observed phenomenon can be explained by numerous factors such
as intra-specific competition \citep{sillett_etal:2004} or optimal
foraging behavior \citep{tufto_etal:1996,said_servanty:2005}. To model
this effect, we
introduce the parameter $\theta$, which determines the ``cost'' of
moving between pixels. If $\theta=0$, then the animal perceives
distance as Euclidean. If $\theta>0$, then least-cost distance (LCD)
is greater than than Euclidean distance (ED). In most cases, we would
not expect,
or should not even consider the possibility of $\theta<0$ because this
implies that LCD$<$ED, which would mean that an animal could view
1000km as 1m. In addition to the fact that this is not biologically
justifiable, it also suggests that the area of the state-space could
be infinitely large. Thus, one may want to enforce the constraint that
$\theta$ is strictly $\geq 0$. See Chapter~\ref{chapt.ecoldist} for
more details.

One may wonder if it is possible to estimate both $\beta$
and $\theta$ using standard SCR data. Currently, it is not possible to
model least-cost distance using \jags~or \secr, so we wrote our own
function, \verb+scrDED+, to fit the model using maximum likelihood. An
example analysis is provided on the help page for the function in our
\R~package \scrbook. We briefly note here that the function requires
the capture history data, the trap locations, and the raster data
formatted using the {\tt raster} package
\citep{hijmans_vanetten:2012}. The linear model for the
intensity parameter $\mu(s, \beta)$ and the least-cost distance
function $lcd(\theta)$ are specified using \R's formula interface. A
simple function call is
\begin{verbatim}
fm <- scrDED(y, traplocs=X, den.formula=~elev, dist.formula=~elev,
             rasters=elev.raster)
\end{verbatim}
To assess the possibility of estimating both $\beta$ and $\theta$, we
conducted a small simulation study, generating 500 datasets from the
model with both parameters set to 1, which corresponds to the
conditions described above. Rather incredibly, we see that it is
possible to estimate both parameters with high accuracy
(Fig~\ref{ch9.fig.sim}).

\begin{figure}[ht]
\centering
\includegraphics[width=4in,height=2in]{Ch11/figs/scrDEDsim}
\caption{Histograms of parameter estimates from 500 simulations under
  the model in which both density and ecological distance are affected
by the same covariate, elevation. The vertical lines indicate the
data-generating value.}
\label{ch9.fig.sim}
\end{figure}



\section{The jaguar data}

Estimating density of large felines has been a priority for many
conservation organizations, but no robust methodologies existed before
the advent of SCR. Distance sampling is not feasible for such rare and
cryptic species, and traditional capture-recapture methods yield
estimates that are highly sensitive to the subjective choice of the
effective survey area. In this example, we
demonstrate how readily density can be estimated for a
globally imperiled species using SCR. Furthermore, we show how
inhomogeneous point process models can be used to test important
hypotheses regarding the ecological factors affecting density.

In this example, we make use of a single year of data from an 8-year
camera-trapping study of jaguars in Argentina,
along the borders with Brazil and Paraguay. The data come from 46
camera stations, each consisting of a pair of cameras placed along
roads or trails. Forty-five detections of 16 jaguars (8 males and 8
females) were made over a 95-day sampling period. The mean number of
sampling days at each camera station was 48.2.

Estimating density is a central objective of this study because
ultimately, an estimate of the total population size for the entire
study area is needed, which can only be obtained by extrapolation of
density estimates. A second, and related, objective was to assess
the influence of poaching on jaguar density. Although jaguars
themselves are occasionally killed by poachers, the larger concern is
the influence of poaching on prey species. To protect jaguars and
related species, protected areas have
been established and three levels of protection are
recognized in the study region as depicted in Fig.~\ref{ch9.fig.jaguarCts}.

\begin{figure}[ht]
\centering
\includegraphics[width=3in,height=3in]{Ch11/figs/jaguarCountMap}
\label{ch9.fig.jaguarCts}
\caption{Jaguar detections at 46 camera trap stations. The three levels of
  protection status are no protection (beige), some protection (light
  green), and national park (dark green). Non-habitat is shown in gray
  and represents large soybean monocultures. }
\end{figure}

To assess the influence of poaching on jaguar density, we treated
protection status as an ordinal variable with 3 levels: no protection,
some protection, and high protection (national parks). Clearly these
are ordered, and our
hypothesis is that poaching pressure should decrease and jaguar
density should increase with the level of
protection. Thus, $\beta$ in this example is a ``slope''
parameter describing the degree to which protection status affects
jaguar density. We also hypothesized that males and females could have
different home range sizes and that the sex ratio may not be
1:1. Furthermore, we restricted the state-space to exclude the large
soybean monocultures surrounding the study area, and we only
considered
area south of the Iguazu River, which runs along the northern border
of the park shown in dark green in
Fig.~\ref{ch9.fig.jaguarCts}. Rather than restricting the
state-space, we could have modeled the permeability of the river using
the methods described in the previous section and in
Chapter~\ref{chapt.ecoldist}; however, no sampling was conducted on
the northern side of the river, and ancillary data indicates that
jaguars very rarely forge the waterway. \R~code to fit the model is
available in \scrbook  on the help page \verb+jaguarDataCh9+. Parameter
estimates are shown in Table\ref{ch9.tab.jagposts}.
\begin{table}
\centering
\caption{Summaries of posterior distributions from the model of jaguar
  density. $\sigma_f$ and $\sigma_m$ are the scale parameters of
  the half-normal detection function for females and males
  respectively. $\rho$ is the
  sex-ratio. $\lambda_0$ is base-line encounter rate. $\beta$ is the
  effect of protection on jaguar density. D is the overall density
  estimate. D1, D2, and D3 are the density estimates
  (jaguars/100km$^2$) for the three levels of protection. }
\begin{tabular}{lrrrrr}
\hline
& Mean & SD & 2.5\% & 50\% & 97.5\% \\
\hline
 $\sigma_f$ &  7361.731 &  1907.566 &  4899.740 &  7002.770 & 12083.110 \\
 $\sigma_m$ &  8177.068 &  1545.717 &  5916.151 &  7955.788 & 11842.486 \\
 $\rho$ &     0.516 &     0.118 &     0.286 &     0.516 &     0.741 \\
 $\lambda_0$ &     0.007 &     0.002 &     0.003 &     0.007 &     0.012 \\
 $\beta$ &     4.405 &     1.443 &     2.553 &     4.143 &     7.775 \\
 D &     0.533 &     0.708 &     0.000 &     0.000 &     0.072 \\
 D1 &     0.132 &     0.010 &     0.095 &     0.095 &     0.616 \\
 D2 &     1.415 &     0.050 &     0.214 &     0.531 &     1.503 \\
 D3 &     3.516 &     0.000 &     0.292 &     3.105 &     4.220 \\
\hline
\end{tabular}
\label{ch9.tab.jagposts}
\end{table}

Our results
indicate that efforts to protect jaguars by reducing poaching are
working. Density was $>$26 times higher in the national park than in the
unprotected area. Fig.~\ref{ch9:fig:Dsurface} shows the estimated
density surface.

\begin{figure}[ht]
\centering
\includegraphics[width=3in,height=3in]{Ch11/figs/Dsurface34}
\label{ch9:fig:Dsurface}
\caption{Estimated density surface for the jaguar dataset}
\end{figure}


We note that there is room for improvement in our analysis. The
political boundaries used to demarcate protected areas are not as
concrete as we might like. In reality poaching pressure is likely to
be higher near remote park boundaries than in well-guarded park
interiors. One option
for addressing this would be to use a continuous measure of poaching
pressure such as distance from the nearest town, or some other
accessibility metric. It would also be interesting to model density
separately for each sex. Many of the detections outside of the park
were of males, and thus it is possible that the sexes use habitat
differently. Developing models for these two hypotheses could be
readily accomplished using slight modifications of the code found in
the \R~package \scrbook.



\section{Summary}

When state-space covariates are available,
density can be modeled by replacing the uniform prior on the activity
centers with a
prior based on a normalized log-linear function of covariates. This
distribution has been widely used in ecology to model point processes
as well as resource selection probability functions
\citep{manly_etal:2002,lele_keim:2006}. In the SCR
context, use of this new prior results in
a model for the inhomogeneous point process describing the
location of activity centers, which can be used to test hypotheses
about spatial variation in density. In
rare cases, these covariates are truly continuous in the sense that
they are defined as a function of space. More often, covariates are
represented as rasters, which simplifies the analysis. Fitting these
models can be accomplished using \bugs, \secr, or the custom \R~code
presented in this chapter and found in the package \scrbook.
%However,
%at the time this book was written, \scrbook is only software available
%for fitting models with covariates of both density and ecological
%distance.

All the examples in this section included a single state-space
covariate, but this was for simplicity only. Including multiple
covariates poses no additional challenges. Similarly, additional model
structure such sex-specific encounter rate parameters or behavioral
responses can be accommodated. Even more remarkable is the ability to
consider covariates that affect both density and ecological
distance. The ramifications of this are enormous for applied
ecological research and conservation efforts because, for instance,
researchers can use capture-recapture data to identify areas where
density is high, and to model important quantities such as landscape
connectivity \citep{royle_etal:2012ecol}. Addressing such questions
is simply not possible using standard, non-spatial capture-recapture
methods. Accomplishing these goals will of course require more data
than is needed to estimate the parameters of a basic SCR model.



%\chapter{Inhomogeneous Point Process}
%\label{chapt.ipp}

\chapter{Open models}
\label{chapt.open}




\chapter{Spatial Capture-Recapture for Unmarked Populations}
\markboth{Chapter 14 }{}
\label{chapt.scr-unmarked}

\vspace{0.3cm}


Traditional capture-recapture models share the fundamental
assumption that each individual in a population can be uniquely
identified when captured. This can often be accomplished
by marking individuals with color bands, ear tags, or some other
artifical mark that can be read in the field. For other species, such as
tigers or marbled salamanders, individuals can be easily identified
using only their natural markings. In a great number of cases, however,
species do not possess sufficient natural markings and are too
difficult to capture to make it practical to apply artifical marks. So
we must throw up our hands and not study these species. End of
chapter.

When capture-recapture methods are not a viable option, researchers
often collect simple count data or even detection/non-detection data
to estimate population parameters. These data are often analyzed using
Poisson regression or logistic regression, perhaps with random
effects; but when detection is imperfect, as it almost always is,
these methods cannot be used to obtain unbiased estimates of
population size or occurrence probability. Even when these data are
used an index of abundance or occurrence, standard models may yield
unreliable results when covariates affect both the state variable and
detection probability. A classic example is the finding by
\citet{bibby_buckland:1987} who reported that the probability of detecting
songbirds in restocked confier plantations decreased with vegetation
height; whereas population density was positively related to
vegetation height. This intuitive and common phenomenon has led to the
development a vast number of models to estimate population size and
detection probability when individuals are unmarked. A review of these
models is beyond the scope of this 
chapter, but we mention a few deficiencies of existing methods
that warrant the exploration of alternatives for robust inference when
standard capture-recapture methods do not apply.

Distance sampling, which we briefly introduced in chapter XXXX,
is perhaps the most widely used method for
estimating population density when individuals are unmarked and
detection probability is less than one. This class of methods is known
to work impecibly when estimating the number of stakes in a field or
the number of duck nests in a wetland. It can also work very well in
more interesting situations; however, %In many other situations,
common issues such as animal movement and measurement error may result in
substantial bias. In addition, traditional distance sampling methods
assume that individuals are randomly located with respect to the
observer and are available for detection (but see
\citet{johnson_etal:2010,chandler_etal:2011}). % Add ISSJ paper too
Most other
methods, such as double-observer sampling and repeated counts, can be
used to estimate population size, but as with traditional CR methods,
it may be difficult to covert abundance estimates to
density estimates because the effective area sampled is unknown. We
mention these issues not to suggest that existing models do not have
value---
indeed we believe that they can be used to obtain reliable density
estimates in many situations---rather our aim is to highlight the need for
alternative methods when the assumptions of existing methods cannot be
met. Additionally, the model we develop in this chapter serves as the
foundation for a broad class of SCR models in which all or some of the
individuals cannot be uniquely identified.

In this chapter we highlight the work of \citet{chandler_royle:2012}
who demonstrated that the ``individual recognition'' assumption of
CR models is not a requirement of spatial capture-recapture
models. They showed that spatial correlation alone is sufficient for
making inference about animal distribution and density. That is, if
we simply have spatially-correlated
count data at a collection of survey points, we can estimate density
even if all individuals are unmarked, assuming that the underlying SCR
model is valid. The details of how this is
accomplished is the subject of this chapter. 

The ability to fit
SCR models to data from unmarked populations has important
consequences in several respects. For one, it means that SCR models can
be applied to data collected using methods like points counts in which
observers record simple counts of animals at an array of survey
points. This development also has important implications for
traditional SCR studies because many resulting datasets include some
individuals that cannnot be identified due to poor photo quality or
the indistiguishable natural markings.


\section{Spatial correlation as Information}


Imagine a 10 $\times$ 10 grid of camera traps and a single individual
exposed to capture whose home range center lies in the center of the
trapping grid. If the individual has a small home range size relative
to the extent of the trapping grid, we can imagine what the
spatial correlation structure of the encounters might look
like. If the animal's movement is symmetric around the activity center
then the number of times the individual is detected at each
trap (the trap counts) is a function of the distance between the home
range center and the trap, and so traps with the same distance from the
activity center will be counts that are more highly correlated with
one another than traps located at different distances from the
activity center. Thus, the correlation in counts tells us something
about the location of the activity center. That is, correlation
carries information about distribution. What about density?

Imagine now that there are two activity centers located in our traping
grid. Using trap counts alone, can our model tell us both where the
activity centers are and how many exist in the population exposed to
capture? The answer is yes, at least under certain circumstances. 
EXPLAIN.

This heuristic is useful for understanding the model proposed by
\citet{chandler_royle:2012}. We will now formalize these concepts and
describe practical issues that arise when applying SCR
models to data from unmarked populations.



\section{Data Requirements and Survey Designs}





\section{Encounter Histories as Latent Variables}

Just when you thought we ran out of things to treat as latent
variables, we are now going to regard even the data itself as latent.


State model is the same as other SCR models.


It is natural to regard the encounter rate of an individual
as a function of the Euclidean distance between the individual's
activity center and the trap location, $d_{ir} = \| {\bf x_r} - {\bf
  s_i} \|$.
To be precise about this, we let $z_{irt}$ be the encounter frequency
of
individual $i$ in trap $r$ during occasion $t$. While we will adopt the view
that  the variables $z_{irt}$ are latent variables (see below), it will
be convenient to formulate the model in terms of these variables.

Therefore, we assume that the expected encounter frequency of an
individual in some trap is related to $d_{ir}$ as follows:
\[
E[z_{irt}] = \lambda_{ir} = \lambda_0 k_{ir}
\]
where $\lambda_0$ is the expected encounter rate at $d=0$ and $k_{ir}$
is some positive-valued
function of distance $d_{ir}$. We assume
\[
k_{ir} = exp(-d_{ir}^2 / 2\sigma^2)
\]
where $\sigma$ is a scale parameter related to home
range size. $\sigma$ also determines the degree of correlation among
counts since animals with large home ranges are more likely to be
detected at multiple traps relative animals with small home ranges.
The phenomenon is analogous to correlation induced by averaging
spatial noise, in which case there is a unique correlation between the
smoothing kernel and the induced covariance function
\citep{higdon:2002}.

We emphasize that our focus is on
situations in which individuals are {\it not}
uniquely identifiable, and therefore the encounter frequencies
for each individual
cannot be observed, and so they are latent variables. We assume that
these latent variables are realizations from a Poisson distribution
with mean $\lambda_{ir}$:
\begin{equation}
 z_{irt} \sim \mbox{Poisson}(\lambda_{ir}).
\label{eq.latentPoisson}
\end{equation}
In traditional SCR models, $z_{irt}$ are the observed data, {\it
  i.e.}, the frequency of encounters of individual $i$ at trap $r$ on
replicate survey $t$. However, when individual identity is not known,
the observed data are the sample- and trap-specific totals,
aggregated over all individuals:
\[
n_{rt} = \sum_{i=1}^{N} z_{irt}.
\]
Thus the data required by our model are a reduced-information
summary of the latent encounter histories.


Under the Poisson encounter model we have that
\begin{equation}
n_{rt} \sim \mbox{Poisson}( \Lambda_{r} )
\label{eq:nagg}
\end{equation}
where
\[
 \Lambda_{r} = \lambda_{0} \sum_{i} k_{ir}.
\]
Further, because $\Lambda_{r}$ does not depend on $t$, we can
aggregate the replicated counts, defining
$n_{r.} = \sum_{t} n_{rt}$ and then
\[
 n_{r.} \sim \mbox{Poisson}( T \Lambda_{r} )
\]
As such, $T$ and $\lambda_{0}$ serve equivalent roles as affecting
baseline encounter rate.
This formulation of the model in terms of the aggregate count
simplifies computations as the latent variables
$z_{irt}$ do not need to be updated in the MCMC estimation
scheme (see below). However, retaining $z_{irt}$
in the formulation of the model
is important if some individuals are uniquely marked, in which case
modifying
the MCMC algorithm (see below) to include both types of data is
trivial. This is because uniquely identifiable individuals produce
observations of some of the $z_{irt}$ variables.

We imagine that other observation models
might be possible (see Discussion) although we focus here on the
Poisson encounter model because it has considerable relevance to
animal surveys, and has additional methodological context related to
point process models which we address in the Discussion.





\section{Estimation by MCMC}
\label{s:mcmc}

We adopt a Bayesian framework for inference allowing estimation of $N$
while retaining the formulation of the model that is conditional on
the latent activity centers $\bf s_i$.
Specifically, we employ Markov chain Monte Carlo
(MCMC) to simulate posterior distributions of the parameters. However,
the fact that $N$ is unknown presents a
technical challenge because the size of the parameter space can change
with each MC iteration. To resolve this, we
adopt the formulation of data augmentation in \citet{royle_etal:2007} who
used a specific prior construction for $N$ in terms of individual level
Bernoulli trials. In particular, we assume $N \sim \mbox{Unif}(0,M)$
for some large integer $M$. We construct this prior by assuming
$N|M,\phi \sim \mbox{Bin}(M,\phi)$ and $\phi \sim \mbox{DUnif}(0,1)$
which implies, marginally, that $N$ has the requisite
$\mbox{DUnif}(0,M)$ distribution. However
the hierarchical formulation of the prior suggests an implementation
in which we introduce a set of latent indicator variables $w_{i} \sim
\mbox{Bern}(\phi)$ and, furthermore, the model implies
that $z_{irt}$ are obtained
from the specified distribution (Eq. \ref{eq.latentPoisson})
if $w_{i} = 1$, or if
$w_{i}=0$, %the model implies that
$z_{irt} =0$ with probability 1. In
effect, extending the model in this way induces a reparameterization
for the latent counts %$z_{irt}$
that is a zero-inflated version
of the original conditional-on-$N$ model. Specifically, the model
under
data augmentation becomes
\begin{eqnarray*}
 z_{irt}|w_{i} &\sim & \mbox{Poisson}(\lambda_{ir} w_i) \\
 w_{i} & \sim & \mbox{Bern}(\phi)
\end{eqnarray*}
Under this formulation $N = \sum_{i=1}^{M} w_i$, and population
density is simply $D = N/A({\cal S})$ where $A({\cal S})$ is the area of the
point process state-space ${\cal S}$.

We developed two distinct MCMC implementations for this model (\ref{suppA}). In the
first, we devised an algorithm for the model conditional on the latent
variables $z_{irt}$. This formulation is useful for problems in which
one or more individual identities are available, in which case the
$z_{irt}$ are observable for those individuals. The unobserved
$z_{irt}$ are easily updated using their full-conditional
distribution which is multinomial with sample size $n_{rt}$. The
remaining parameters are updated using Metropolis-Hastings steps (see
\ref{suppA}).  In the second formulation of the algorithm we applied
the Metropolis-Hastings algorithm to the model {\it unconditional} on
the $z_{irt}$ variables. In that case, the marginal distribution for
$n_{rt}$ is precisely Eq.~\ref{eq:nagg}.  This algorithm is slightly more
convenient because
it avoids having to update the $z_{irt}$ variables of which there are many.




\section{Northern Parula Example}



To apply our model to data collected in the field, we designed a point
count study of the northern parula ({\it Parula americana}), a
Neotropical-Nearctic migratory passerine. This species defends
well-defined territories during the breeding season
\citep{moldenhaer_regelski_1996}, and thus our modeling effort was focused
on estimating the number and location of territory centers. Points
were located on a 50-m grid to ensure spatial
correlation. This small grid spacing contrasts with the conventional
practice of spacing points by $>$ 200 m to obtain \emph{i.i.d.}
counts. Figure~\ref{fig:nopaDat} depicts the spatially-correlated
counts ($n_{r.}$) from the 105 point count locations
surveyed three times each during June 2006
at the Patuxent Wildlife Research Center in Laurel Maryland, USA.
A total of 226 detections were made with a maximum count of 4 during a
single survey. At 38 points, no warblers were detected. All but one of
the detections were of singing males, and this one observation was
not included in the analysis.


%\begin{comment}

\begin{figure}
  \centering
  \includegraphics[width=3in,height=2.25in]{Ch14/figs/nopa}
  \caption{Spatially-correlated counts of northern parula on a 50-m
    grid. The size of the circle represents the total number of
    detections at each point.}
  \label{fig:nopaDat}
\end{figure}

%\end{comment}


In our analysis of the parula data, we defined the point process
state-space by buffering the grid of point
count locations by 250 m and used $M=300$. We simulated posterior
distributions using three Markov chains,
each consisting of 300000 iterations after discarding the initial 10000
draws. Convergence was satisfactory, as indicated by an $\hat{R}$
statistic of $<$ 1.02 \citep{gelman_rubin:1992}.

One benefit of a Bayesian analysis is that it can accommodate prior
information on the home range size and encounter rate parameters,
which are readily available for many
species. To illustrate, we analyzed the parula data using two sets of
priors. In the first set, all priors were
improper, customary non-informative priors (see Table \ref{t:nopaPosts}).
Uniform priors were also used in the second set, with the exception of
an informative prior for the scale parameter $\sigma \sim
\mbox{Gamma}(13,10)$. We arrived at this prior using the methods
described by \citet{royle_etal:2011mee} and published
information on the warbler's home range size and detection probability
\citep{moldenhaer_regelski_1996,simons_etal:2009}. More details on this
derivation are found in \ref{suppA}. We briefly note here that this prior
includes the biologically-plausible range of values from $\sigma$
suggested by the published literature.

The posterior distribution for
$N$ was highly skewed with a long right tail resulting in a wide 95\%
credible interval (Table \ref{t:nopaPosts}). Nonetheless, the interval
for density, $D$, includes estimates reported from more intensive field
studies \citep[][]{moldenhaer_regelski_1996}. This was true when
considering
both sets of priors, although posterior precision was higher under the
informative set of priors. Specifically, the use of prior information
reduced posterior density at high, biologically implausible,
values of $\sigma$, and hence decreased the posterior mass for
low values of $N$ (Fig.~\ref{fig:prior}).

In addition to estimating density, our model can be used to produce
density surface maps, which are often used in applied ecological
research to direct management efforts and develop hypotheses regarding
the factors influencing abundance.
Density surface maps can be produced by discretized the
state-space and tallying the number of activity centers occurring in
each pixel during each MCMC iteration. Parula density was
highest near the northeastern corner of the study plot, which may
correspond to important habitat features such as suitable nest site
locations (Fig.~\ref{fig:nopaDen}). We anticipate future model
extensions to directly model the
point process intensity using habitat covariates.


\begin{table}%[t]
  \caption{Posterior summary statistics for spatial Poisson-count
    model applied to the northern parula data. Two sets of priors were
    considered. $M=300$ was used in both cases. Parulas/ha, $D$, is a
    derived parameter.}
  \scriptsize
  \begin{tabular}{l l rrrrrr}
    \hline
    Par        & Prior                  & Mean  & SD    & Mode   & q0.025  & q0.50  & q0.975  \\
    \hline
    $\sigma$   & $U(0, \infty)$   & 2.154   & 1.222  & 1.230   & 0.896   & 1.665   & 5.170    \\
    $\lambda_0$ & $U(0, \infty)$  & 0.284   & 0.149 & 0.212    & 0.084  & 0.256  & 0.665   \\
    $N$        & $U(0, M)$             & 40.953   & 38.072  & 4.000  & 3.000       & 31.000     & 143.000     \\
    $D$        &  --                   & 0.427    & 0.397 & 0.0417   & 0.0313  & 0.323  & 1.490    \\
    \hline
    $\sigma$    & $G(13, 10)$          & 1.301    & 0.258 & 1.230    & 0.889   & 1.266   & 1.908    \\
    $\lambda_0$ & $U(0, \infty)$ & 0.298    & 0.132 & 0.240    & 0.098   & 0.279  & 0.603   \\
    $N$         & $U(0, M)$            & 59.321   & 36.489  & 36.000 & 18.000      & 50.000     & 157.000     \\
    $D$         &  --                  & 0.618    & 0.380 & 0.375   & 0.188   & 0.521  & 1.635    \\
    \hline
  \end{tabular}
  \label{t:nopaPosts}
\vspace{0.5cm}
\end{table}


%\begin{comment}

\begin{figure}
  \centering
  \includegraphics[width=1.5in,height=3in]{Ch14/figs/prior} % was 3,7
  \caption{Effects of $\sigma \sim \mbox{Gamma}(13,10)$
    prior on the posterior distributions from the northern parula
    model. Posteriors from model with uniform priors are
    shown in black, and posteriors from the informative prior model
    are shown in gray. The prior itself is shown as dotted line in the
    upper panel.}
  \label{fig:prior}
\end{figure}




\begin{figure}
  \centering
  \includegraphics[width=3in,height=2.25in]{Ch14/figs/nopaDen}
  \caption{Estimated density surface of northern parula activity
    centers. The grid of point count locations with count totals is
    superimposed. See Fig. 1 for additional details.  }
  \label{fig:nopaDen}
\end{figure}

%\end{comment}



\section{On (Im)precision}





\section{How Much Correlation Is Enough?}



\section{Mutants}

\subsection{Other observation models}

\subsection{Linear designs}




\section{Summary}








In this paper, we confronted one of the most difficult challenges
faced in wildlife sampling ---
estimation of density in the absence of data to distinguish among
individuals. To do so, we developed a novel class of
spatially-explicit models that
applies to spatially organized counts, where the count locations or
devices are located sufficiently close together so that individuals
are exposed to encounter at multiple devices. This design yields
correlation in the observed counts, and this correlation proves to be
informative about encounter probability parameters and hence density.
We note that sample locations in count-based studies are typically
{\it not} organized close
together in space because conventional wisdom and standard practice
dictate that independence of sample units is necessary
\citep{hurlbert:1984}. Our model
suggests that in some cases it might be advantageous to deviate from
the conventional wisdom if one is interested in direct inference about
density. Of course, this is also known in the application of standard spatial
capture-recapture  models \citep{borchers_efford:2008}
where individual
identity is preserved across trap encounters, but it is seldom, if
ever, considered in the design of more traditional count surveys.

Our model has broad relevance to an incredible number of animal
sampling problems. Our motivating problem involved bird point counts
where individual
identity is typically not available. The model also applies
to other standard methods used to sample unmarked
populations,  such as camera traps
or even methods that yield sign ({\it e.g.} scat, track) counts
indexed by space. However, results of our simulation study reveal some
important limitations of the basic
estimator applied to situations in which none of the individuals can
be uniquely identified. In particular, posterior
distributions are highly skewed in typical small to moderate sample
size situations and posterior precision is low.

Several modifications of the model can lead to improved
performance of the estimator.
Our simulation results demonstrate that marking a subset of
individuals can yield substantial increases in posterior
precision. Marking a subset of individuals is
commonplace is animal studies such as when a small number of individuals are
radio-collared in conjunction with a count-based survey
\citep{bartmann_etal:1987}. In many other situations a subset of
individuals can be identified by natural marks alone, and thus our
model could be applied to data from camera-trapping studies of
species such as mountain lions, deer, coyotes for which traditional
SCR methods are not effective \citep{kelly_etal:2008}.
Thus, the ability to study partially-marked populations
adds flexibility to existing SCR methods, and also
creates new opportunities for designing efficient SCR studies
since the costs of marking all individuals in a population can be
prohibitive.

We note the existence of traditional approaches to combining data on
marked and unmarked animals based on either the Lincoln-Peterson
estimator or so-called ``mark-resight'' methods.
\citep{bartmann_etal:1987, mintaMangel:89, mcclintockHoeting:09}. In their
simplest form, mark-resight methods involve fitting standard
closed-population mark-recapture models to the data on marked
individuals, and the resultant estimate of detection probability
($\hat{p}$) is used to estimate population size as $\hat{N} = m +
u/\hat{p}$ where $m$ and $u$ are the number of
marked and unmarked individual, respectively. In this case,
the unmarked individuals provide no information about the
encounter rate parameters, and thus mark-resight methods cannot be
used unless a large sample of marked individuals is available. This
contrasts with our approach which can be used even when all
individuals are unmarked.

In some cases, such as in point counts of birds, it may not be
practical to mark individuals. An alternative to increasing posterior
precision is to utilize prior information on
home range size. Indeed, extensive information on home range size has
been compiled for many species in diverse habitats %\emph{e.g.}
\citep[\emph{e.g.},][]{degraaf_yamasaki:2001}. It is
easy to embody this information in a prior distribution as we
demonstrated for the parula data.

An additional design extension that could increase precision is to use
multiple sampling methods, in which one method generates encounter
frequencies and the other method generates individuality.
For example, camera traps are now commonly used with surveys for
sign (scat or tracks), or hair snares for sampling bear populations.
These distinct methods would have different basal detection
rates but share an underlying spatial model describing the
organization of individuals in space.
Our models show promise for using
these disparate data types efficiently
for estimating density.




\subsection{Alternative Observation Models}
\label{ss:ext}

Several aspects of our ``spatial $N$-mixture model'' can be modified
to accommodate
alternative sampling designs or parametric distributions.
We considered situations where an individual can be detected more than
once at a trap during a single occasion, but under some designs this
is not possible. When collecting DNA samples, for instance, an
individual can often be detected at most once during an
occasion, because multiple samples of biological material cannot be
attributed
to distinct episodes. Therefore, rather than $z_{irt} \sim Poisson(\lambda_{ir})$
we have $z_{irt} \sim Bernoulli(p_{ir})$ where, for example,  $p_{ir} = p_0
exp(-d_{ir}^2/(2\sigma^2))$, and $p_0$ is the probability of
detecting an individual whose home range is centered on trap $r$. This
Bernoulli model is a focus of ongoing investigations.

Both the Poisson and the Bernoulli models
produce count observations when aggregated over individuals to form
trap-specific totals; however, ecologists often collect so-called
``detection/non-detection'' data because it can be easier to determine
if ``at least one'' individual was present rather than enumerating all
individuals in a location. In this case, the underlying $z_{irt}$
array is the same as the above cases, but we observe $y_{rt} =
I(\sum_{i=1}^{N} z_{irt} > 0)$ where $I$ is the indicator
function. This ``Poisson-binary model'' is
a spatially explicit extension of the model of
\citet{royle_nichols:2003} in which the underlying abundance state
is inferred from binary data. We have investigated this model to a
limited extent but do not report on those results here.


\subsection{Spatial point process models}
\label{ss:similar}

Our model has some direct linkages to existing point process
models. We note that the observation intensity function (i.e.,
corresponding to the observation
locations) is a compound Gaussian kernel similar to
that of the Thomas process
\citep[pp. 61-62]{thomas:1949, moller_waagepetersen:2003}.
Also, the Poisson-Gamma Convolution models
\citep{wolpert_ickstadt:1998} are structurally similar (see also \cite{higdon:1998}
and \cite{best_etal:2000}).
 In particular, our model is such a model but
with a {\it constant} basal encounter rate $\lambda_{0}$
and {\it unknown} number and location of ``support points'', which in
our case are the animal activity centers, $\bf{s_i}$.
We can thus regard our model as a model for
{\it estimating} the location and local density of support points in
such models, which we believe could be useful in the application of
convolution models.  \citet{best_etal:2000} devise an MCMC algorithm for the
Poisson-Gamma model based on data augmentation, which is
similar to the component of our algorithm for
updating the $z$ variables in
the conditional-on-$z$ formulation of the model.  We emphasize that
our model is distinct from these Poisson-Gamma models
in that the number {\it and} location of such
support points are estimated.


If individuals were perfectly observable then the resulting point
process of locations is clearly a standard Poisson or Binomial (fixed
$N$) cluster process or Neyman-Scott process.
If detection is uniform over space but
imperfect, then the basic process is unaffected by this random thinning.
Our model can therefore be viewed formally as a Poisson (or Binomial)
cluster process model but one in which the thinning is
non-uniform, governed by the encounter model which dictates that
thinning rate increases with distance from the observation points. In
addition, our inference objective is, essentially, to estimate the
number of parents in the underlying Poisson cluster
process,
where the observations are biased by an incomplete sampling apparatus
(points in space).


As a model of a thinned point process, our model has much in common
with classical distance sampling models \citep{buckland_etal:2001}.
The main distinction is that our data structure does {\it not} include
observed distances, although the underlying observation model is
fundamentally the same as in distance sampling if there is only a
single replicate sample and $\bf{s}_i$ is defined as an individual's
location at an instant in time. For replicate samples, our model preserves
(latent) individuality across samples and traps which is not a feature
of distance sampling. We note that error in measurement of distance is
not a relevant consideration in our model, and we explicitly do not
require the standard distance sampling assumption that the probability
of detection is 1 if an individual occurs at the survey point. More
importantly, distance sampling models cannot be applied to data from
many of the sampling designs for which our model is relevant. For
example, many rare and endangered species can only be
effectively surveyed using methods such as hair snares and camera
traps that do not produce distance data \citep{oconnell_etal:2010}.


\section{Conclusion}

Concerns about ``statistical independence'' have prompted
ecologists to design count-based studies such that observed
random variables can be regarded as {\it i.i.d.} outcomes
\citep{hurlbert:1984}. Interestingly, this
often proves impossible in practice, and elaborate
methods have been devised to model spatial dependence as a nuisance
parameter. Our paper presents a modeling framework that directly
confronts this view by demonstrating that spatial
correlation carries information about the locations of individuals,
which can be used to estimate density even when individuals
are unmarked and distance-related heterogeneity exists in encounter
probability.




\chapter{
Spatial capture-recapture models for partially identifiable
populations: Spatial mark-resight models
}
\markboth{Spatial mark-resight models}{}
\label{chapt.partialID}

\vspace{.3in}


So far, this book has dealt with the situation where all detected
individuals are identifiable, and in Chapt. \ref{chapt.scr-unmarked}
we introduced and developed an SCR model for non-identifiable
populations, a spatial {\it non}-capture-recapture model, if you will. These
two extremes are common in the study of animal populations with
non-invasive sampling methods. However, there is also an intermediate
situation, where a part of the population is tagged or otherwise
marked and can thus be identified upon recapture, while the untagged
portion remains unidentified. In this situation so-called mark-resight
models \citep{bartmann_etal:1987, arnason_etal:1991, neal_etal:1993}
can be used to estimate population size and density combining data
from both the marked and unmarked individuals.

Traditionally, capture-recapture studies involved physical capture of
individuals throughout the study; new individuals are marked on every
re-capture occasion. This methodology is still widely applied to small
mammals, but can be very costly, logistically challenging and risky
when dealing with larger species. In contrast, in mark-resight studies
a sample of individuals is captured and tagged (or otherwise marked)
during a single marking event. Marking is followed by resighting
surveys, upon which both the detection of marked and recognizable
individuals and unmarked animals is recorded. Resighting surveys are
usually non-invasive (hence the name �resighting�), so that they
don't involve handling of animals. As such, mark-resight models have a
major advantage over traditional capture-recapture models in that they
only require individuals to be captured and handled once, during the
initial marking. This reduces field costs and risks for the animals
(and potentially the researchers).

Mark-resight models have a set of underlying assumptions, most of
which are analogous to those for mark-recapture models,
e.g. demographic population closure (violation of geographic
population closure can be accommodated by some models) and no loss or
misidentification of marks. Just like regular capture-recapture
models, there are means to incorporate heterogeneity in capture
probability. However, a new and essential assumption of mark-resight
models is that the tagged (or otherwise identifiable) individuals are
a representative sample of the study population, so that inference
about individual detection can be made for the whole population from
the tagged sample. This issue is usually addressed by using a
different method for marking than for resighting, and by marking a
random sample of the population.

Owing to the advantages of mark-resight over capture-recapture,
especially when dealing with hard-to-trap species, mark-resight is a
popular tool in wildlife population studies. The method has been
applied for decades and to a suite of species and survey techniques,
ranging from banding and resighting Canada geese
\citep{hestbeck_malecki:1989} to ear-tagging and camera-trapping
grizzly bears \citep{mace_etal:1994} to paintball marking and areal
resightings of large ungulates \citep{skalski_etal:2005}.

\subsection{Types of partial ID data}

Before we start exploring mark-resight approaches in more detail, we
need a clear understanding of what types of mark-resight data we can
have, in order to appreciate and understand the different flavors of
mark-resigh models.  In general, we have (at least) two sets of data:
encounter histories for identifiable individuals $i$ at trap $j$ and
occasion $k$, $y_{ijk}$, and counts of unidentified records for each
$j$ and $k$, $n_{jk}$. Depending on the sampling technique, we can
conceive of three slightly different types of partial ID data.



If you implement your resighting survey shortly after the marking
session, you may be confident that none of the marked individuals has
died or lost its mark. Under these circumstances you know that the
number of marked individuals available for resighting, $m$, is equal
to the number of individuals you tagged. Alternatively, tags might be
radio-transmitters, allowing you to confirm the presence or absence of
marked individuals in the resighting survey area using radio-telemetry
\citep{white_shenk:2001}. In both cases, you know the number of marked
individuals in the population you survey.

In this situation, even though you may fail to resight some of the
tagged individuals, since you know how many there are, you can simply
assign those you never resighted all-zero capture histories - in other
words, contrary to regular capture-recapture models, in mark-resight
models with a known number of tagged individuals, we can observe
all-zero encounter histories. Under these circumstances, estimating
$N$ reduces to estimating the number of unmarked individuals, $U$.

If we suspect that some of the marks may have been lost between
tagging and conducting the resighting samples, we obtain a slightly
different type of mark-resight data. Here, we do not know the accurate
number of marked individuals available for resighting. As a
consequence, individuals have to be resighted at least once for us to
know they are still tagged and alive and thus available for
resighting. So, contrary to the situation where we know $m$ and
analogous to regular capture-recapture models, we cannot observe
all-zero encounter histories of marked individual. Here, estimating
$N$ involves both estimating $m$ and $U$.

A special case of this kind of data can arise from camera
trapping. Even when dealing with a species that has no spots or
stripes, some individuals in the study population can have natural
marks that make them identifiable on pictures, such as scars or some
distinct coloration. Again, in this scenario an individual has to be
photographed at least once to be known. Here, the fact that both the
`marking� method and the subsequent resighting method are the same
(although marking in this case does not involve any actual physical
marking) can be cause for concern: our sample of `marked�
individuals may not be a random sample of the population but consist
of individuals that for some reason are more likely to be
photographed. In that case, a basic assumption of the mark-resight
model is violated.

Finally, consider a scat or hair snare survey, where only a part of
the samples are analyzed genetically (or DNA can only be extracted
from a subset of samples due to sample quality). In this scenario,
your $n_{jk}$ can contain both completely unknown individuals that are
not represented at all in {\bf $Y$}, but it can also contain samples
from individuals that we previously identified. The difference is that
in the first two scenarios, part of the population of individuals is
identifiable, while in the second scenario, part of the
samples is identifiable. This type of data
actually violates one of the basic assumptions of mark-resight models,
namely, that tagged individuals are always correctly identified as
such. To our knowledge there are currently no mark-resight models
available that account for possible misidentification of the marking status of individuals (although there some literature is available on misidentification of individuals in capture-recapture study, e.g., \citealp{yoshizaki_etal:2009, lukacs_burnham:2005, link_etal:2010}). In this chapter we will ignore this kind of data and focus instead
on the two types of typical mark-resight data:

\begin{itemize}
\item[(1)] Known number of tagged individuals 
\item[(2)] Unknown number of tagged individuals, 
\end{itemize}

For both types of data a slightly different situation arises when in some instances we can only tell that an individual is tagged, but not who it is. You may be able to see that an individual is tagged but the identifying feature of the tag (a number or coloration) may have become unreadable, or may be hidden from view. In this case, in addition to your $y_{ijk}$ and your $n_{jk}$ you also have a number of sightings of tagged but unidentified individuals, say $r_{jk}$. 

\subsection{A short history of mark-resight models}

Initially, mark-resight methods focused on radio-tagged individuals to
estimate population size \citep{white_shenk:2001}. Radio-collars
provide a means of determining which of the animals were in the study
area and available for sampling, i.e. determining the number of marked
individuals in the population. Knowing this number was a prerequisite
for most earlier mark-resight approaches \citep{white:1996}. The
oldest mark-resight model is the good old Lincoln-Petersen estimator,
 where individuals are marked and a single resight/recapture occasion is carried out \citep{krebs:1999}. We need not identify individuals, but only tell apart marked from unmarked individuals. Let $m$ be the number of marked individuals in the population, $m_{(R)}$ the number of marked individuals seen on the resighting occasion, and $n_{(R)}$ the total number of marked and unmarked individuals observed during resighting. Abundance $N$ is then estimated as 
\[
N = m \times n_{(R)}/m_{(R)}
\]

A suite of more elaborate models using individual capture histories
over several resighting occasions were developed in the 1980ies and
90ies and compiled into the program NOREMARK \citep{white:1996}. Apart
from the basic model with known number of marked individuals and no
individual variation in resighting probabilities (joint hypergeometric
maximum likelihood estimator) \citep{bartmann_etal:1987,
  white_garrot:1990, neal:1990, neal_etal:1993}, NOREMARK contains
models that account for lack of geographical population closure
\citep{neal_etal:1993}, individual heterogenenity in resighting rates
and sampling with replacement (i.e. individuals can be seen more than
once on any occasion, \citep{minta_mangel:1989, bowden:1993}). A first
mark-resight model allowing for an unknown number of marked
individuals was developed by \citet{arnason_etal:1991}.

While many of these models perform well under certain situations, they
are somewhat limited: they do not allow for combining data across
several surveys \citep{mcclintock_etal:2006} and not all of them are
likelihood-based or allow for different parameterization, so that
selection of the most appropriate model cannot be based on standard
approaches such as AIC, but is largely left up to educated guesswork
\citep{mcclintock_etal:2006}. Recently, more flexible and generalized
likelihood-based mark-resight models have been developed. These models
can account for individual heterogeneity in detection, unknown number
of marked individuals and lack of geographical closure, as well as a
less than 100\% individual identification rate of tagged individuals;
they can be applied to sampling with and without replacement and can
combine data across several primary sampling occasions in a robust
design type of analysis
\citep{mcclintock_etal:2009biometrics,mcclintock_etal:2009mdp}. Since
they are all likelihood-based, model selection among different
parameterizations and model averaging based on AIC is an option. Most
of these models have also been incorporated into the program MARK
\citep{mcclintock_white:2010}.

For a detailed treatment of these different non-spatial mark-resight
models, we refer you to the original papers cited in the preceeding
paragraph. In short, these models are based on the joint likelihood of
two major model components: one describing the resighting process of
marked individuals (either using a Poisson or a Bernoulli observation
model, depending on whether sampling is with or without replacement),
where resighting probabilities can have both fixed effects to model
individual and environmental covariates, and a random-effect component
to accommodate variation in detection due to individual heterogeneity;
and one describing the total observations of unmarked individuals,
$n_t$ which are approximated as a normal distribution
\citep{mcclintock_etal:2006}, or a normal distribution left-truncated
at 0 \citep{mcclintock_etal:2009biometrics}:
\[
n_t \sim Normal (E(n_t), V(n_t))
\]
Although this is a simplification of the actual sampling process, \citet{mcclintock_etal:2006} found this Normal distribution to be a satisfactory approximation, which allows $N$ to enter the model likelihood via $E(n_t)$ and $V(n_t)$.

In the simplest model case without any variation in detection, the
expected number of resightings of unmarked individuals, $E(n_t)$, can
be written as the number of unmarked individuals times the expected number of detections of a single individual, which is the mean or expected value of the underlying observation model:
\begin{equation}
E(n_t) = (N-m) * \theta 
\end{equation}
\label{partialID.eq.E_n}
where $\theta = k \times p$ for a Binomial observation model with $k$
replicates and individual detection probability $p$, or $\theta$ =
expected/average individual encounter rate $\lambda$ for a Poisson
observation model. Similarly, $V(n_t)$ depends on the underlying
observation model and is based on the parameters
that determine the individual detection probability/encounter
rate. Combining these two components, $N$ is directly incorporated
into the joint likelihood of the model.

XXXXX Opinions: More details? Full model description? XXXXX
XXX I'm ok with this right now  -- andy XXXX

While these mark-resight models are very flexible, they
share the shortcomings of �regular� capture-recapture models
when it comes to estimating population density (e.g., Chapts. \ref{chapt.intro, chapt.closed}). 
In the following sections we will consider mark-resight sampling in the framework of spatial capture-recapture. We will look at models for both known and unknown numbers of marked individuals, and for imperfect individual identification of marks. In the spatial framework, most of the information on model parameters comes from the marked individuals. But in Sect. \ref{partialID.sec.info} we will see that, analogous to the models we developed in the previous Chapt. \ref{chapt.scr-unmarked}, the spatial correlation in counts of unmarked individuals also contributes information about detection and movement. 

\section{Known number of marked individuals}

Let's begin with the easiest data situation: a known number of
individuals constituting a random, representative sample from the
population are marked and a series of resight samples are conducted
following marking. No marks (or marked animals) are lost between
marking and resighting, all individuals are correctly identified as
marked or unmarked, and marked individuals are 100 \% correctly
identified to individual level.

Recall that without individual identity, the observed counts at trap
$j$ and occasion $k$, $n_{jk}$, represent the sum of all latent
individual detections at $j$ and $k$,
$\displaystyle\sum\limits_{i=1}^{N} y_{ijk}$, where $y_{ijk}$ are the
latent individual encounter histories. We can model these counts as
\[
n_{jk} \sim \mbox{Poisson}( \Lambda_{j} )
\]
Under this formulation we do not need to update the individual
$y_{ijk}$ in our model, which  is more efficient in terms of
computing. However, we can also formulate the model as conditional on
the latent $y_{ijk}$. This is useful because if we have $m$
individually known animals in our study population, than those $m$
$y_{ijk}$ are no longer latent, but fully observed and can easily be
included in the analysis. 

The formulation conditional on $y_{ijk}$ basically brings us back to the original SCR model, where individual site and occasion specific counts, $y_{ijk}$, are modeled as
\[
y_{ijk} \sim \mbox{Poisson}(\lambda _{ij})
\] 
and
\[
\lambda _{ij} = \lambda_0 * exp(-D_{ij}^2/(2 \sigma^2))
\]
XXXX You use $D_{ij}$ and maybe Richard does too. I haven't
been..... maybe we should standardize on $D_{ij}$? XXXXX


Unobserved $y_{ijk}$ are essentially missing data and have to be
updated as part of the MCMC procedure. We can do that by using their
full conditional distribution, which is multinomial with sample size
$n_{jk}$:
\[
y_{ujk} \sim Multinomial (n_{jk}, \lambda_{uj})
\]
where \textbf{\emph{u}} is an index vector of the $M-m$ hypothetical unmarked individuals.

  
While in the non-spatial mark-resight analysis known individuals
provide direct information about individual detection probability (or
rate), in the spatial setting they also inform the movement parameter
$\sigma$. Including known individuals into the analysis helps estimate
model parameters more accurately and precisely. We will address the
relationship between the number of marked individuals and accuracy of
the estimated parameters in section \ref{partialID.sec.info}.

\subsection{MCMC for a spatial mark-resight model}


Just as for the model without individual identification, for the
partial ID model, knowing how to write your own MCMC algorithm comes
in extremely handy. You will find that we only have to make relatively
simple modifications to the MCMC code for the model without any
individual identification presented in
Chapt. \ref{chapt.scr-unmarked}, which, in turn, has much in common
with the algorithms we developed for regular SCR models in
Chapt. \ref{chapt.mcmc}.
Essentially, since we observe individual detections for the marked part of the population, we have to update only the unobserved part of ${\bf Y}$, and
modify the updating steps for $z_i$ and $\psi$ to reflect some
contribution to our
knowledge of these parameters from the $m$ tagged individuals.

First, we set up an array to hold ${\bf Y}$, fill the first $m$ rows
of the array with the $m$ observed individual encounter histories,
then update ${\bf Y}$ for the unknown individuals only (note that the
code is set up so that $n_{jk}$ contains both pictures of marked {\bf
  and} unmarked individuals at $j$ and $k$):

{\small
\begin{verbatim}
# set up placeholders and create vectors for marked and unmarked    
 Y <- array(NA, c(M, J, K))
    nMarked <- nrow(y)
    marked <- rep(FALSE, M)
        marked[1:nMarked] <- TRUE
        Y[1:nMarked, , ] <- y
    z[marked] <- 1
    Ydata <- !is.na(Y)
    for (j in 1:J) {
        for (k in 1:K) {
            if (y[j, k] == 0) {
                Y[, j, k] <- 0
                next
            }
            unmarked <- !Ydata[, j, k]
            nUnknown <- n[j, k] - sum(Y[!unmarked, j,k])
            if (nUnknown < 0) 
                browser()
            probs <- lam[, j] * z
            probs <- probs[unmarked]
            probs <- probs/sum(probs)
            Y[unmarked, j, k] <- rmultinom(1, nUnknown, probs)
        }
    }
\end{verbatim}
}

XXX andy stopped here XXXX

When we know the number of marked individuals in the population estimating $N$ is reduced to etimating $u$. Thus, we only need to estimate the $z_i$ for $M-m$ unknown individuals. Thus, the updater for $z_i$ becomes:

\begin{verbatim}
zUps <- 0
seen <- apply(Y > 0, 1, any)
   for (i in 1:M) {
       if (seen[i] | marked[i]) 
                next
       zcand <- ifelse(z[i] == 0, 1, 0)
       ll <- sum(dpois(Y[i, , ], lam[i, ] * z[i], log = TRUE))
       llcand <- sum(dpois(Y[i, , ], lam[i, ] * zcand, 
                  log = TRUE))
       prior <- dbinom(z[i], 1, psi, log = TRUE)
       prior.cand <- dbinom(zcand, 1, psi, log = TRUE)
          if (runif(1) < exp((llcand + prior.cand) - (ll + 
                prior))) {
          z[i] <- zcand
          zUps <- zUps + 1
            }
        }
\end{verbatim}

Observe that while we skip the update of $z_i$ for the �seen� individuals, seen is defined based on ${\bf Y}$ and ${\bf Y}$ is updated at each iteration, so the $z_i$ for the �seen� but unmarked individuals are effectively still updated.

Finally, our update for $\psi$ needs to reflect that we are effectively only estimating $U$. In the full conditional beta distribution we have to replace $M$ with $M-m$ and $\sum z$ with $\sum z -m$:

\begin{verbatim}
  psi<-rbeta(1,1+sum(w[!marked]),1+sum(!marked)-sum(w[!marked]))   
\end{verbatim}

The remainder of the code is essentially identical to the MCMC code for regular SCR models we developed in Chapt. \ref{chapt.mcmc}.
You can find the full MCMC code (including the modeling options we'll discuss in the upcoming sections) in the accompanying {\bf R} package {\tt scrbook} by invoking {\tt scrPID()}. 

\subsection{Binomial encounter model}
So far, we have only worked with Poisson encounter models for partially identifiable or unmarked populations. When we use a Bernoulli model instead, we have to make some changes to how we update the latent $y_{ijk}$, to ensure that a hypothetical individual receives at most a single observation at a given trap and occasion from the pool of $n_{jk}$ pictures. Effectively, we move from a multinomial situation where the same individual could be drawn repeatedly, to a sampling without replacement situation (an individual drawn once at $j$ and $k$ cannot be drawn again); here is how we implement this in our MCMC algorithm:

\begin{verbatim}
 Y <- array(NA, c(M, J, K))
#[...]
    for (j in 1:J) {
        for (k in 1:K) {
            if (y[j, k] == 0) {
                Y[, j, k] <- 0
                next
            }
            unmarked <- !Ydata[, j, k]
            nUnknown <- n[j, k] - sum(Y[!unmarked, j,k])
            if (nUnknown < 0) 
                browser()
            probs <- lam[, j] * z
            probs <- probs[unmarked]
            probs <- probs/sum(probs)
            Y[unmarked, j, k] <- 0
            guys <- sample(which(unmarked), nUnknown, 
			prob = probs)
            Y[guys, j, k] <- 1
        }
    }
\end{verbatim}


{\flushleft \bf Example: Canada geese in North Carolina } 
We applied the spatial mark-resight model with a Bernoulli encounter process to a dataset of of Canada goose resightings \citep{rutledge:2012} XXXget citation with LizXXX. During the molt of 2008, 751 individual geese were captured and tagged with neck and leg bands in Greensboro, North Carolina. Geese were resighted at 87 different locations on 81 resighting events over a period of 18 months. In addition to the banded geese, the number of unmarked geese was recorded during each resighting event. Here, we only looked at a subset of the data, from mid July to the end of October 2008, which corresponds to the first part of the post-molt season, before migratory Canada geese arrive in North Carolina.  
During this time frame, 746 of the 751 marked geese were known to be alive. Of those, 654 were resighted 3994 times at 40 different sites. In addition, 7944 sightings of unmarked geese were recorded at 48 sites. 

In this model, we also allowed $\sigma$ to vary between males and females. We augmented the data set with 4500 - $m$ all-zero encounter histories, ran 50000 MCMC iterations and removed a burn-in of 1000 iterations. We provide all the data (data(canadageese)) and functions for you to repeat this analysis but be aware that given the large data set it will take days to do so. The model results, including the derived parameter density ($D$) in individuals per $km^2$ are shown below. 
XXXX HAVE TO ASK FOR PERMISSION TO INCLUDE DATA - IN THE PROCESS XXXX

\begin{verbatim}
Iterations = 1001:50000
Thinning interval = 1 
Number of chains = 1 
Sample size per chain = 49000 

1. Empirical mean and standard deviation for each variable,
   plus standard error of the mean:

            Mean        SD  Naive SE Time-series SE
sigmaF    1.0594 1.902e-02 8.593e-05      0.0011674
sigmaM    1.1347 2.375e-02 1.073e-04      0.0014294
lam0      0.3245 8.037e-03 3.631e-05      0.0003103
psi       0.7924 3.284e-02 1.483e-04      0.0017778
phi       0.4337 1.857e-02 8.387e-05      0.0003754
N      3720.8128 1.210e+02 5.466e-01      6.6264003
D         6.6832 2.173e-01 9.817e-04      0.0119021

2. Quantiles for each variable:

            2.5%       25%       50%       75%     97.5%
sigmaF    1.0218    1.0470    1.0594    1.0717    1.0971
sigmaM    1.0909    1.1182    1.1341    1.1502    1.1834
lam0      0.3088    0.3191    0.3244    0.3298    0.3407
psi       0.7298    0.7698    0.7917    0.8143    0.8573
phi       0.3976    0.4210    0.4336    0.4462    0.4702
N      3492.0000 3637.0000 3717.0000 3802.0000 3961.0000
D         6.2722    6.5326    6.6763    6.8290    7.1146
\end{verbatim}

We see that credible intervals of estimates are pretty tight. Take, for example, $\sigma$ for males and females: Although they differ only by 0.08, there is barely any overlap between the respective credible intervals, surely an effect of the large data set. Phi in this model is the probability of being a male, or the sex ratio of the populations in terms of males:females, which is close to 1:1.      


\section {Unknown number of marked individuals}

Now let us consider the case where we do not know the exact number of tagged individuals available for resighting so that we have to capture an individual at least once to be sure that it is available. Unless we have a direct means of confirming the number of marked animals available for resighting, treating this number as unmarked is probably more realistic in most circumstances. As a consequence of not knowing the exact number of marked individuals, we cannot observe all-zero encounter histories. When using maximum likelihood inference, this situation requires a model where detection rates of known individuals are modeled using a zero-truncated distribution \citep{mcclintock_etal:2009biometrics}. If we did not account for the fact that 0�s are unobservable, our estimates of detection rates would be artificially inflated and estimates of population size would be negatively biased. 

Working with zero-truncated distributions in a spatial mark-resight setting is less straight-forward than for non-spatial mark-resight. A marked individual only has to show up once, anywhere on the sampling grid, for us to know that it is there. When resightings are pooled across the entire sampling grid,then the total individual counts $\sum_j y_{ij}$ have to be $>$ 0 for all resighted individuals and a zero-truncated distribution can be used to model these counts. However, we are concerned with trap-specific encounters, $y_{ij}$, which can easily be 0 for a resighted individual, as long as a single $y_{ij}$ is $>$ 0. Thus, the zero-truncation does not apply to the individual and trap specific counts we observe, but only to the sum of these counts over all traps. 

As an alternative to a zero-truncated distribution, in a Bayesian framework, we can make use of data augmentation to estimate the number of marked individuals\footnote{For the interested reader, \citet{mcclintock_hoeting:2010} implement a non-spatial mark-resight model with a binomial observation model in a Bayesian framework using data augmentation}. 
In the previous example, where we knew the number of marked individuals, we essentially removed those individuals from the augmented population by fixing their $z_i$ at 1 and letting $\psi$ refer only to the unmarked population, $M-m$. All we have to do in the spatial mark-resight model with unknown number of marked individuals is to let our marked individuals be part of the augmented population again, analogous to the situation in regular SCR models:
\begin{verbatim}
        psi <- rbeta(1, 1 + sum(z), 1 + M - sum(z))
\end{verbatim}
Whether you have a known or an unknown number of marked individuals is included as an option in {\tt scrPID}.
 
XXXX Other example data set? XXXX

\section  {Individual identification rate of tagged animals $<$ 100 \%}
Often during resighting, it may be possible to see that an individual is tagged but impossible to determine the individual identity of the tag. In such a situation in addition to the $y_{ijk}$ and $n_{jk}$, we also have site and occasion specific counts of marked but unidentified individuals, $r_{jk}$. Here, the individual encounter histories of marked animals are essentially incomplete, and if we used these incomplete data to inform the detection parameter of the model, we would be likely to underestimate detection/trap encounter rate and overestimate abundance. Some non-spatial mark-resigh models do not require that marked animals be identified individually, as long as the marking status can be observed unambiguously, but ignoring individual level information means that we cannot accomodate heterogeneity in detection \citet{mcclintock_white:2010}. In a spatial framework we could ignore marked and unmarked status completely and apply the model by \citet{chandler_royle:2012} we discussed in Chapt. \ref{chapt.scr-unmarked}. But again, that would mean losing important information on individual detection and movement. Therefore, being able to retain the individual identity of records that can be identified while at the same time accounting for an identification rate $<$ 100 \% is extremely useful. 
\citet{mcclintock_etal:2009biometrics,mcclintock_etal:2009mdp} suggest an intuitive means of correcting for this bias in a non-spatial model framework when dealing with a Poisson encounter model (or sampling with replacement). When marked but unknown resightings are part of the data, the expected number of records of unmarked individuals at time $t$, $n_t$, changes from Eq. \ref{partialID.eq.E_n} to:

\[
E(n) = (N-m) { \lambda  + \eta/m}
\]
Here, $\lambda$ is the individual encounter rate estimated from the known resighted individuals and $\eta$ is the number of records of marked but unidentified individuals. So essentially, because observed $\lambda$ is known to be too low, the average number of unidentified pictures per known individual is added as a correction factor. This procedure assumes that the inability to identify a marked individual occurs at random throughout the population, which seems to be a reasonable assumption under most circumstances.

We can relatively easily translate this concept to our spatial mark-resight models. In the spatial model framework we are interested in the individual and trap specific encounter rate, $\lambda_{ij}$. Further, we do not look at the sum of all records of unmarked individuals, but formulate the model conditional on the latent individual encounter histories. Thus, instead of using $\eta/m$ as a correction factor, we need something that applies at the individual and trap level. If we take the sum of all correctly identified records of marked individuals, $\sum y_c$ and divide it by the total number of records of marked individuals, $\sum y_m$, we get the average rate of correct individual identification for marked individuals, say, $c$:
\[
c = \sum y_c/\sum y_m
\]

For the marked individuals we can then multiply $\lambda_0$ with $c$ to account for the fact that we observe incomplete individual encounter histories. For example, if on average we are able to assign 80\% of the records of marked animals to the correct individual, we would multiply $\lambda_0 \times 0.8$ for the marked individuals. Since we don't have this identification issue for unmarked individuals, their baseline trap encounter rate remains as before simply $\lambda_0$ (or in other words, their $c$ equals 1). Observe that now, in addition to assuming that failure to identify tagged individuals occurs at random throughout the population, we also assume that it occurs at random throughout space, i.e. our success of identifying a tagged individual does not depend on the trap we encounter it in. 
It is straightforward to  include this correction factor in our MCMC algorithm, by simply specifying a vector of length $M$, where $c = 1$ for all unmarked (hypothetical) individuals and $c = \sum y_c/\sum y_m$ for all marked individuals. Incomplete individual identification of marked individuals is included as an option in the {\tt scrPID} function.

XXXX maybe we can include the ISSJ as an example, but estimates of N will be huuuge XXX

As long as individuals are identified based on the same type of tags the assumption that failure to identify marked individuals occurs at random throughout the population should be valid. The assumption that failure to identify marked individuals occurs at random in space could be violated, for example when spatially varying habitat conditions influence the ability to recognize individual tags. Also, observer effects could influence individual identification rates. While we haven't ourselves experimented with it, we belive that the above described approach could readily be extended to account for these differences. For example, identification rates could be calculated separately for different observers, or be modeled as functions of habitat covariates. 


\section{How much information do marked and unmarked individuals contribute?}
\label{partialID.sec.info}
It is intuitive that having marked individuals in the study population should lead to more accurate and precise parameter estimates than when no individuals are identifiable. To evaluate how strongly adding marked individuals to a population improves parameter esimtates, \citet{chandler_royle:2012} performed a simulation study. They used a 15 � 15 trap grid and
simulated detection data of $N = 75$ individuals in a 20 x 20 units state-space over $k = 5$ occasions with
$\sigma = 0.5$ and $\lambda_0 = 0.5$. They generated 100 datasets each for
$m$ = (0, 5, 15, 25, 35) where $m$ is the number of marked individuals randomly sampled from the population and is assumed to be known.

Without any marked individuals in the population, the posterior distribution of $N$ turned out to be fairly skewed, but its mode was still an approximately unbiased point estimator of $N$. As anticipated, posterior precision increased substantially with the proportion of marked individuals (Tab. \ref{partialID.tab.sim} and Fig. \ref{partialID.fig.nposts}). The posterior mode was approximately unbiased as a point estimator, and the relative root-mean squared error decreased from 0.246 when no individuals were marked to 0.085 when 35 individuals were marked (Tab. \ref{partialID.tab.sim}). Coverage was nominal for all values of $m$ and posterior skew greatly diminished with increasing $m$(Tab. \ref{partialID.tab.sim}).

\begin{figure}%[ht]
  \centering
  \includegraphics[width=4in,height=4in]{Ch15/figs/Nposts2.png}
  \caption{Overlaid posterior distributions of $N$ from 100 simulations
    for four levels of marked individuals.}
  \label{partialID.fig.nposts}
\end{figure}

\begin{table}%[hb]
\caption{Posterior mean, mode, and associated relative RMSE for simulations in
  which $m$ of $N$=75 individuals were marked. One hundred simulations of each case were conducted. }
\begin{tabular}{llrrrrr}
     &	Parameter    &	Mean   &	rRMSE  & Mode   & rRMSE &	BCI    \\
     \hline
 m=0 &	$N$          &	85.866 &    0.259 & 77.720 &    0.242 & 0.950  \\
     &	$\lambda_0$  &	0.506  &	0.180 &	0.488  &	0.182 &	0.960  \\
     &	$\sigma$     &	0.495  &	0.115 &	0.486  &	0.113 &	0.960  \\
     \hline
 m=5 &	$N$          &	80.898 &    0.184 & 76.360 &    0.182 & 0.970  \\
     &	$\lambda_0$  &	0.510  &    0.178 & 0.494  &    0.180 & 0.950  \\
     &	$\sigma$     &	0.496  &    0.089 & 0.488  &    0.086 & 0.970  \\
     \hline
 m=15&	$N$          &	79.028 &    0.148 & 76.250 &    0.147 & 0.950  \\
     &	$\lambda_0$  &	0.508  &    0.163 & 0.494  &    0.164 & 0.950  \\
     &	$\sigma$     &	0.496  &    0.073 & 0.492  &    0.071 & 0.970  \\
     \hline
 m=25&	$N$          &	77.765 &    0.114 & 75.810 &    0.113 & 0.950  \\
     &	$\lambda_0$  &	0.511  &    0.153 & 0.498  &    0.157 & 0.950  \\
     &	$\sigma$     &	0.496  &    0.067 & 0.493  &    0.065 & 0.940  \\
     \hline
 m=35&	$N$          &	76.446 &    0.085 & 74.900 &    0.085 & 1.000  \\
     &	$\lambda_0$  &	0.513  &    0.142 & 0.501  &    0.144 & 0.950  \\
     &	$\sigma$     &	0.497  &    0.056 & 0.493  &    0.057 & 0.940  \\
 \hline
\end{tabular}
\label{partialID.tab.sim}
\end{table}

As we saw in the previous chapter, the spatial correlation in unmarked counts can be sufficient to obtain estimates of movement and detection parameters. However, only marked and thus identifiable individuals provide us with direct information about these parameters and may well dominate estimates. 
To single out the contribution of marked and unmarked individuals to parameter estimates, we re-ran the same simulations but let $\sigma$ and $\lambda_0$ be updated based solely on the data of marked individuals. Results are summarized in Tab. \ref{partialID.tab.sim2}.
We see that if we update $\lambda_0$ and $\sigma$ based on marked individuals only, estimates of $\lambda_0$ and $\sigma$ are more biased and less precise. For estimates of $N$, especially for $m$= 5 and $m$ = 15, we observe a stronger positive bias, lower accuracy and considerably lower BCI coverage as compared to when both marked and unmarked individuals contribute to parameter estimates (Tab. \ref{partialID.tab.sim2}). Thus, unmarked individuals do actually contribute noticeably to estimating model parameters. 

\begin{table}%[hb]
\caption{Posterior mean, mode, and associated relative RMSE for simulations in
  which $m$ of $N$=75 individuals were marked and unmarked individuals 
  did not contribute to estimating $\lambda_0$ and $\sigma$. 
  One hundred simulations of each case were conducted. }
\begin{tabular}{llrrrrr}
     &	Parameter    &	Mean   &	RMSE  &	Mode   &	RMSE &	BCI    \\
     \hline
 m=5 &	$N$          &	88.621 &	0.369 &	83.139 &	0.421 &	0.810  \\
     &	$\lambda_0$  &	1.255  &	1.247 &	0.606  &	1.148 &	0.950  \\
     &	$\sigma$     &	0.472  &	0.252 &	0.426  &	0.333 &	0.910  \\
     \hline
 m=15&	$N$          &	81.031 &	0.192 &	78.361 &	0.175 &	0.820  \\
     &	$\lambda_0$  &	0.535  &	0.281 &	0.476  &	0.284 &	0.970  \\
     &	$\sigma$     &	0.503  &	0.109 &	0.490  &	0.107 &	0.940  \\
     \hline
 m=25&	$N$          &	78.206 &	0.129 &	76.594 &	0.123 &	0.920  \\
     &	$\lambda_0$  &	0.531  &	0.204 &	0.496  &	0.202 &	0.960  \\
     &	$\sigma$     &	0.497  &	0.081 &	0.489  &	0.084 &	0.950  \\
     \hline
 m=35&	$N$          &	76.833 &	0.099 &	75.422 &	0.096 &	0.940  \\
     &	$\lambda_0$  &	0.528  &	0.192 &	0.505  &	0.186 &	0.940  \\
     &	$\sigma$     &	0.499  &	0.069 &	0.493  &	0.070 &	0.960  \\
 \hline
\end{tabular}
\label{partialID.tab.sim2}
\end{table}


\section{Integrating telemetry data}
As we expected, parameter estimates of spatial mark-resight models get better the more marked individuals we have in our study population. While this is great advice in theory, it may not be very helpful in practise, especially when dealing with animals that are hard or somewhat dangerous to capture, such as large carnivores. Oftentimes, studies involving the physical capture of such animals will employ telemetry tags in order to learn about the study species' spatial ecology and behavior. In the context of spatial mark-resight models, the actual locational data collected by telemetry tags can provide detailed information on individual location and movement, and being able to incorporate this information directly into the SMR model should improve estimates of these parameters, especially when resighting information is sparse.

So how could we combine resighting data and telemetry data in a unified mark-resight model? Recall that the basic SCR model underlying all the SMR models we discuss here uses a half-normal detection function.
By using this function, we can relate the parameters $\sigma$ and ${\bf s}_{i}$ directly to those from a bivariate normal movement model, with mean = ${\bf s}_{i}$, and variance-covariance matrix $\Sigma$, where the variance in both dimensions is $\sigma^2$ and the covariance is 0. Ordinarily, these parameters are estimated directly from the spatial distribution of individual recaptures/resightings. Telemetry data, however, provide more detailed information on individual location and movement, since the resolution and extent of the data are not limited by the trapping grid and potentially more locations can be accumulated through telemetry than resighting (depending on the monitoring frequency and resighting rates of individuals).  

By assuming that the locations of individual $i$, ${\bf l}_{i}$ (consisting of a pari of x and y coordinates, $l_{ix}$ and $l_{iy}$), are a bivariate normal random variable:
\[
{\bf l}_i\sim Normal_2 ({\bf s}_i,\Sigma)
\]
we can estimate $\sigma$ as well as ${\bf s}_{i}$ for the collared individuals directly from telemetry locations, using their full conditional distributions:
\[
[\sigma|{\bf l},{\bf s}] \propto \left\{\prod_{i=1}^m \frac{1}{2 \pi \sigma^2} exp\left(-1/2 \left[ \frac {l_{ix}-s_{ix}} {\sigma^2} + \frac{l_{iy}-s_{iy}}{\sigma^2} \right]\right)\right\}*[\sigma] 
\] 
and
\[
[{\bf s}_{i}|{\bf l}_{i},\sigma] \propto \left\{\frac{1}{2 \pi \sigma^2} exp\left(-1/2 \left[ \frac {l_{ix}-s_{ix}} {\sigma^2} + \frac{l_{iy}-s_{iy}}{\sigma^2} \right]\right)\right\}*[{\bf s}_{i}] 
\]
XXXXX ANYONE - I am pretty sure that's what the BVN pdf reduces to when covariance = 0, but would someone mind cross-checking? XXXXXX 
Under the standard mark-resight assumption that marked individuals are a representative sample of the population, the estimate of $\sigma$ can be applied for the entire population. For the unmarked individuals ${\bf s}_{i}$ are estimated as described before conditional on their latent encounter histories. 

{\bf R} makes it easy to implement the update of $\sigma$ and ${\bf s}_i$ based on telemetry data and the above described full conditionals within our existing MCMC algorithm. We simply replace the current updating step for $\sigma$ with: 

\begin{verbatim}
#ntot = number of telemetry-tagged individuals
#locs = list of length ntot; each element is a matrix 
#with telemetry locations
#telID = vector with identifier for telemetry-tagged
#individuals

sigma.cand <- rnorm(1, sigma, tune[1])
if (sigma.cand > 0) {

llsig<-llsig.cand<-rep(NA, ntot) 

for (x in 1:ntot) {
lls[x]<-sum(dmvnorm(x=locs[[x]],mean=c(S[telID[x],1],S[telID[x],2]), 
			sigma=cbind(c(sigma^2,0), c(0,sigma^2)), log=T))   
lls.cand[x]<-sum(dmvnorm(x=locs[[x]],mean=c(S[telID[x],1],S[telID[x],2]), 
	sigma=cbind(c(sigma.cand^2,0), c(0,sigma.cand^2)), log=T))   
	}
   if(runif(1) < exp( sum(lls.cand)  - sum(lls) ) ){
    sigma<-sigma.cand
    lam <- lam0*exp(-(D*D)/(2*sigma.cand*sigma.cand))
					}
			}
\end{verbatim} 
For ${\bf s}$ we use an analogous updater for the telemetry-tagged individuals and the regular updater for individuals without associated telemetry location information. A full example code can be found in the {\bf R} package {\tt scrbook}, by calling {\tt scrPID.telemetry}. Note that not all marked individuals need to be telemetry-tagged. This approach of incorporating telemetry data into a spatial mark-resight model can easily be extended to update $\sigma$ and ${\bf s}$ conditional on both resighting and telemetry data and applies equally to regular SCR models where all individuals are identifiable. 

{\flushleft \bf Example: Raccoons on the Outer Banks of North Carolina } 
XXXX Will have to ask for permission to include this; alternatively use the panthers; both examples are written up XXXXXXX

\section{Summary}
In this chapter we extended the spatial model for unmarked populations to a mark-resight situation, where part of the population is individually identifiable, usually through artificial tags. The basic model with known number of marked individuals and 100 \% individual identification of marked is easily modified for situations where the number of marked individuals is unknown, or where marked animals can sometimes not be identified to individual level. As expected, having marked individuals in the study population improved accuracy and precision of parameter estimates when compared to fully unmarked populations, but we also saw that the spatial counts of unmarked individuals still contribute information to parameter estimates. Finally, we present an approach of how to incorporate telemetry location data into the spatial mark-resight model to inform estimates of $\sigma$ and activity centers. Especially for difficult-to-study, cryptic species where often only a small sample of the population can be tagged this enables researchers to make optimal use of all existing data and obtain robust density estimates without the need for additional invasive methods. 
Beyond the models presented in this chapter, just as SCR, the spatial mark-resight model framework is flexible to account for a variety of factors that may influence individual movement and detection, as well as survey-related parameters. As such, the approach is applicable to a wide range of population estimation problems when dealing with animals that cannot be identified based on natural marks. 











 





\chapter{Spatial Capture-Recapture with Distance Sampling Data}
\markboth{Chapter 17}{}
\label{chapt.scrds}

\vspace{0.3cm}



In SCR models, the locations of animal activity
centers are unknown and must be inferred from the trap locations where
individuals are captured. Intuitively, the more
we know about the locations of activity centers, the more precise will
be our density estimates, and thus we strive to increase the
number of spatial recaptures. That is, we want to catch each
individual at multiple points in space so that we can pinpoint
its activity center. However, obtaining a large
number of spatial recaptures can be difficult due to the associated costs
of traps and the labor required to set and check them. This is true
even of ``cheap'' methods like camera traps which can easily run $>$
200\$ a pop.

Distance-samplers, of course, know that a much easier way to record an
animal's location in space is to go traverse a transect or stand at some
point and directly record the coordinates of the animal at some
instant in time\footnote{Generally only the distance between the
  observer and the animal is used in the analysis, but the exact
  coordinates are often recorded.}. \hl{Hmm, this make you wonder
  about telemetry data too}. This is cheap and easy data, which
partially explains the popularity of distance sampling.


Given the ease
with which one can record distance data, it is natural to wonder how
it could be included in a SCR analysis. Before doing so, a better
question is why bother with SCR when a distance sampling analysis would
be straight-forward. Good point. In some cases, there probably is no
need to use SCR if the sole quantity of interest is density, and the
assumptions of distance sampling can be met. However, SCR let's us
study more than just density. Think of space use. Think up a ship,
think up a long trip, think up the Vipper, the Vipper of Biff. Oh the
thinks you can think up if only you try (Seuss 1960s).

Some regard movement as a nuisance that is
best left untouched \citep{borchers:2010}. This is
convenient, but movement is actually a part of the problem, not to
mention a central focus of an enormous
branch of ecology.

Use of data on animal locations at some distance from the observer
forces us to consider movement a little more explictily than we had
previously. Currently, only two papers that we know of have attempted
to estimate explicity movement parameters
\citep{royle_young:2008,royle_etal:2009jae}. The underlying approach
is simple. First, define $\bf u$ to be an individual's location in
space at some instant in time. We now need a movement model that links
the activity center $\bf s$ to $\bf u$ and finally a detection model that
is a function of $\bf \| u - x\|$ instead of $\bf \|s - x\|$,
\emph{i.e.}, a function of distance between the observer
at point $\bf x$ and the animal at point $\bf u$---just as in distance
sampling. A natural movement model is the bivariate normal, but we
will consider other options in this chapter. In addition, we will
consider pragmatic issues such as what to do if not all individuals
are marked.







\section{Everybody is marked}



\begin{verbatim}

model {
sigHome ~ dunif(0, 5)
sigObs ~ dunif(0, 5)
tauHome <- 1/pow(sigHome,2)
tauObs <- 1/pow(sigObs,2)
psi ~ dunif(0, 1)
for(i in 1:M) {
  w[i] ~ dbern(psi)
  sx[i] ~ dunif(0, 15)
  sy[i] ~ dunif(0, 15)
  for(r in 1:R) {
    ux[i,r] ~ dnorm(sx[i], tauHome)
    uu[i,r] ~ dnorm(sy[i], tauHome)
    d2[i,r] <- pow(X[r,1]-ux[i,r], 2) + pow(X[r,2]-uy[i,r], 2)
    p[i,r] <- exp(-d2[i,r]/(2*sigObs*sigObs)) * w[i]
    y[i,r] ~ dbern(p[i,r])
    }
  }
N <- sum(w[])
}

\end{verbatim}







\section{Nobody is marked}





\section{Partially-marked populations}





\section{An Implicit SCRDS Model Without Distance Data}

The idea here is to convolve two Gaussian kernels, one for the animal
(movement) and one for the observer (detection | distance). This is
just another two-parameter observation model I guess.




\bibliography{AndyRefs_alphabetized}


\markboth{Index}{Index}

%\printindex

\documentclass{book}

\usepackage{elsst-book}
\usepackage{float}
\usepackage{amsmath}
\usepackage{amsfonts}
\usepackage{graphicx}
\usepackage{lineno}
\usepackage{natbib}
\usepackage{hyperref}
\usepackage{verbatim}
\usepackage{soul}
\usepackage{color}

\bibliographystyle{asa}

\usepackage{makeidx,bm,amsmath,url}
\makeindex

\floatstyle{plain}
\floatname{panel}{Panel}
\newfloat{algorithm}{h}{txt}[chapter]
\newfloat{panel}{h}{txt}[chapter]


\newcommand{\R}{\textbf{R}}
\newcommand{\bugs}{\textbf{BUGS}}
\newcommand{\jags}{\textbf{JAGS}}
\newcommand{\secr}{\mbox{\tt secr}}
\newcommand{\scrbook}{\mbox{\tt scrbook}}


\linenumbers

\begin{document}

\title{ Spatial Capture-Recapture  }
\subtitle{
%Hierarchical modeling of capture-recapture data with auxiliary spatial information
}
\author{The Four Horsemen (and women) }

\affiliation{First Author Short Address\\ Second Author Short Address}
\address{
USGS Patuxent Wildlife Research Center \\
North Carolina State University
}

\maketitle

\newpage

\setcounter{tocdepth}{2}
\tableofcontents

%\chapter{Introduction}
%\label{chapt.intro}

\chapter{
Introduction to Spatial Capture-Recapture
}
\markboth{Introduction}{}
\label{chapt.intro}


\vspace{.3in}

Information about abundance or density of populations, and their vital
rates, is fundamental to applied ecology and conservation biology.  To
that end, a huge variety of statistical methods have been devised, and
among these, the most well-developed are collectively known as
capture-recapture (or capture-mark-recapture) methods. For example,
the volumes by \citet{seber:1982}, \citet{borchers_etal:2002},
\citet{williams_etal:2002}, and \citet{amstrup_etal:2005} are largely
synthetic treatments of such methods, and contributions on modeling
and estimation using capture-recapture are plentiful in the
peer-reviewed ecology literature.  Capture-recapture techniques make
use of individual encounter history data, by which we mean sequences
of 0's and 1's denoting if an individual was encountered at a
particular trap during a certain time period. For example, the
encounter history ``010'' indicates that this individual was
encountered only during the second of three trapping occasions. As we
will see, these data contain
information about encounter probability, abundance, and other
parameters of interest in the study of population dynamics.

A diverse and growing number of methods exist for obtaining encounter
history data. Such methods are, naturally, taxa-specific. They include
classical ``traps'' which capture and retain animals until visited by
a biologist who removes the individual, marks it, or otherwise molests
it in some scientific fashion.  Small-mammal traps and mist nets for
birds are standard examples. Traps that physically capture and
restrain individuals are common, but capture-recapture methods no
longer require ``capture'' or even physical marking of individuals.
Recent technological advances have produced a
large number of passive detection devices that produce individual
encounter history data. These include camera traps
\citep{karanth_nichols:1998, oconnell_etal:2010}, acoustic recording
devices \citep{dawson_efford:2009}, and methods that obtain DNA
samples such as hair snares for bears \citep{gardner_etal:2010jwm}, scent
posts for many carnivores \citep{kery_etal:2010}, and related methods which allow DNA
to be extracted from scat, urine or animal tissue in order to identify
individuals.  This book is concerned with how such data can be used to
carry out inference about animal abundance or density, and other
demographic parameters such as survival, recruitment, and movement
using new classes of capture-recapture models which utilize auxiliary
spatial information related to the encounter process.  We refer to
such methods as spatial capture-recapture (SCR) models\footnote{In
the literature the term spatially explicit capture-recapture (SECR) is
also used}.

As the name implies, the primary feature of SCR models that
distinguishes them from traditional CR methods is that they make use
of the spatial information inherent to capture-recapture studies. That
is, the encounter histories are associated with spatial coordinates,
and these coordinates are informative about home range
characteristics, movement and space usage.
As we will see, this allows us to overcome three critical
deficiencies of non-spatial methods, namely,
traditional CR methods cannot be used to formally estimate density,
include of trap-level covariates of density or capture probability, or
account for heterogeneity in encounter probability that
results from the spatial organization of animals and traps.
Thus, spatial modeling is not just
a fun academic exercise; it provides a solution to basic problems in
the study of animal populations that have been acknowledged for more
than 70 years \citep{dice:1938}.


\section{Scope of this Book}

In this book, we try to achieve a broad methodological scope from
basic closed population models %using a number of distinct observation
%models
for inference about population density on up to open population models
for inference about vital rates such as survival and recruitment. %---spatial versions of
%conventional Jolly-Seber models. %A number of conceptual and
%methodological themes unify the main topical coverage of this book, and
%those are:
Much of the material is a synthesis of recent research but we also expand SCR models in a 
number of useful directions, including to accomodate unmarked individuals
(Chapt. \ref{chapt.xxxx}), use of telemetry information (Chapt. XXXX), and developing 
explicit models of individual space usage (Chapt. \ref{chapt.ecoldist}), and many other
new topics that have yet to appear in the literature. 
Our intent is to
provide a comprehensive resource for ecologists interested in
understanding and applying SCR models to solve common problems
faced in the study of population dynamics. To do so, we make use of
hierarchical models, which allow extrodinary
flexibility in accomodating virtually any type of capture-recapture
data. We present many example analyses, of real and simulated data
using likelihood-based and Bayesian methods---examples that readers
can replicate using the code presented in the text and
the resources made available on-line and in our accompanying {\bf R} package
{\tt scrbook}.

Although we aim to reach a
broad audience, at times we go into details that may only be of
interest to advanced practitioners who need to extend these models to
unique situations.  We hope that these advanced topics will not
discourage those new to these methods, but instead we believe this
material will allow readers to advance their understanding and become
less reliant on restrictive tools and software. Before disucssing the
specifics of SCR models, we begin with an overview of the methods
used to collect capture-recapture data, and a brief summary of
traditional non-spatial capture-recapture models.

\section{Lions and Tigers and Bears, oh my:  Genesis of
Spatial capture-recapture data}

A diverse number of methods and devices exist for producing individual
encounter history data with auxiliary spatial information about
individual locations. Historically, physical ``traps'' have been widely
used to sample animal populations. These include live traps, leg-hold
traps, mist nets, pitfall traps and many other types of
devices. Although these are still widely used, huge advances have been
made in developing new methodologies for obtaining encounter history
data non-invasively. We briefly review some of these here, which we
will consider more explicitly in later chapters of this book.

\subsection{Camera trapping}

Considerable recent work has gone into the development of
camera-trapping methodologies. For a historical overview of this
method see \citet{kays_etal:2008, kucera_barrett:2011}.  Several
recent synthetic works have been published including
\citet{nichols_karanth:2002}, and an edited volume by
\citet{oconnell_etal:2010} devoted solely to camera trapping concepts
and methods. As a method for estimating abundance some of the earliest
work that relates to the use of camera trapping data in
capture-recapture models originates from Karanth and colleagues
\citep{karanth:1995, karanth_nichols:1998, karanth_nichols:2000}. In
studies that use camera trapping, cameras are situated along trails or
at baited stations and individual animals are photographed and
subsequently identified either manually by a person sitting behind a
computer,  or sometimes now using computational
methods. Camera trapping methods are widely used for species that have
unique stripe or spotting patterns such as tigers \citep{karanth:1995,
  karanth_nichols:1998}, ocelots
\citep{trolle_kery:2003,trolle_kery:2005}, leopards
\citep{balme_etal:2010}, and many other cat species.
% Scientific names
Camera traps are
also used for other species such as wolverines
\citep{magoun_etal:2011}, and even species that are less easy to
identify uniquely such as mountain lions and coyotes
(e.g. \citet{kelly_etal:2008}.  We note that even for species that are
not readily identified by pelage patterns, it might be efficient to use 
camera traps in conjunction with spatial capture-recapture models to
estimate density (see Chapt.~\ref{chapt.scr-unmarked}).
%, if an initial sample of individuals can be collared
%or tagged in some way so that subsequent encounter by camera-traps can
%yield individual information. In this way, the probability of
%encounter can be estimated from the camera traps based on the
%pre-marked individuals, and this is applied to the frequencies of
%unmarked individuals to estimate density.


\begin{figure}
\begin{center}
\includegraphics[width=5in]{Ch1/figs/wolverinetiger}
\end{center}
\caption{Wolverine in camera trap from A. Magoun (left). Picture of Tiger in
  camera trap from U. Karanth (right)}
\label{fig.wolverinetiger}
\end{figure}

\subsection{DNA Sampling}

Recent technological advances in the extraction and analysis of
genetic information have made a huge positive impact on the study of
animal populations. DNA obtained from hair, blood or scat is now
routinely used to obtain individual identity and encounter history
information about individuals \citep{taberlet_bouvent:1992,
  woods_etal:1999, mills_etal:2000, schwartz_monfort:2008}.  A common
method is based on the use of ``hair snares'' (Fig. \ref{fig.bearcat})
which are widely used to study bear populations
\citep{woods_etal:1999, gardner_etal:2010jwm, garshelis_etal:2006,
  kendall_etal:2009}.  A sample of hair is obtained as individuals
pass under or around barbed-wire (or other physical mechanism) to take
bait. Hair snares have also been used to sample felid populations
\citep{garciaalaniz_etal:2010} and other species. DNA information can
also be extracted from urine and as a result DNA can be used to study
feline populations which are attracted to scent-sticks and deposit
urine which is subsequently analyzed in the lab
\citep{valiere_taberlet:2000, kery_etal:2010}.


\begin{figure}
\begin{center}
\includegraphics[width=5in]{Ch1/figs/bearcat}
\end{center}
\caption{Picture of hair snare. Bear (left). European wildcat
  (right). Pictures from??}
\label{fig.bearcat}
\end{figure}

\begin{figure}
\begin{center}
\includegraphics[width=5in]{Ch1/figs/beardog}
\end{center}
\caption{Guy holding fisher (left). Scat dog team working the ground
  (right). Pictures from Craig Thompson.}
\label{fig.fisherscatdog}
\end{figure}


\subsection{Acoustic surveys}

Many studies of birds \citep{dawson_efford:2009}, bats, and whales \citep{marques_etal:2009}  now collect data using
devices that record vocalizations. When vocalizations can be identified by individual from multiple
recording devices, then spatial encounter histories are produced that are amenable to 
the application of SCR models \citep{dawson_efford:2009, efford_etal:2009ecol}.

\subsection{Search-Encounter Methods}

There are other methods which don't fall into a nice clean taxonomy of
``devices''. Spatial encounter histories\footnote{defined? probably
  not! need to do that} are commomnly obtained by conducting manual
searches of geographic sample units such as quadrats, transects or
road or trail networks.
For example,
DNA-based encounter histories can be obtained from scat
samples located along roads or trails or by specially trained dogs
\citep{mackay_etal:2008} searching space
(Fig. \ref{fig.fisherscatdog}). This method has been used in studies
of martens, fishers \citep{thompson_etal:inpress}, lynx, coyotes,
birds \citet{kery_etal:2010}, and many other species. We might search
space on foot and pick up individuals and physically mark them
somehow. This is pretty common in surveys that involve reptiles and
amphibians, e.g., we might walk transects through a forest and pick-up
box turtles \citep{hall_etal:1999} or search space for lizards
\citep{royle_young:2008} and also surveys designed to obtain animal
scat. These methods don't seem like normal capture-recapture in the
sense that the encounter of individuals is not associated with
specific trap location, but SCR models are equally relevant for
analysis of such data (see Chapt. \ref{chapt.searchencounter}).


\section{ Historical Context: A Brief Synopsis of the Literature}

Spatial capture-recapture is a relatively new methodological
development, at least with regard to formal estimation and
inference. However, the basic problems that motivate the need for
formal spatially-explicit models have been recognized for decades and
quite a large number of ideas have been proposed to deal with these
problems. We review some of these ideas here.


\subsection{Buffering}

 The standard approach to estimating density even now is to estimate $N$ using
conventional closed population models \citep{otis_etal:1978} and then
try to associate with this estimate some specific sampled area, say $A$,
the area which is contributing individuals to the population for which
$N$ is being estimated. The strategy is to define $A$ by placing a buffer
of say $W$ around the trap array or some polygon which encloses the trap
array. The historical context is succintly put by \citep{obrien:2011}
from which we draw this description:

\begin{quote}
  ``At its most simplistic, $A$ may be described by a concave polygon
  defined by connecting the outermost trap locations ($A_{tp}$; \citet{mohr:1947}).
 This assumes that animals do not move from outside the
  bounded area to inside the area or vice versa. Unless the study is
  conducted on a small island or a physical barrier is erected in the
  study area to limit movement of animals, this assumption is unlikely
  to be true. More often, a boundary area of width $W$ ($A_{w}$) is added to
  the area defined by the polygon $A_{tp}$ to reflect the area beyond the
  limit of the traps that potentially is contributing animals to the
  abundance estimate \citep{otis_etal:1978}. The sampled area, also known
  as the effective area, is then $A(W) = A_{tp} + A_{w}$. Calculation of the
  buffer strip width ($W$) is critical to the estimation of density and
  is problematic because there is no agreed upon method of estimating
  $W$. Solutions to this problem all involve ad hoc methods that date
  back to early attempts to estimate abundance and home ranges based
  on trapping grids
  \citep[see][]{hayne:1949}. \citet{dice:1938} first drew attention
  to this problem in small mammal studies and recommended using
  one-half the diameter of an average home range. Other solutions have
  included use of inter-trap distances \citep{blair:1940,burt:1943}, mean
  movements among traps, maximum movements among traps \citep{holdenried:1940, hayne:1949},
 nested grids \citep{otis_etal:1978}, and assessment
  lines \citep{smith_etal:1971}.''
\end{quote}

The idea of using 1/2 mean maximum distance moved
\citep{wilson_anderson:1985a} seems to be the standard approach even
today, presumably justified by Dice''s suggestion to use 1/2 the home
range diameter. Alternatively, some studies have used the full
MMDM\footnote{Do they really say that?}
(e.g. \citet{parmenter_etal:2003}). And, sometimes home range size is
estimated by telemetry \citep{karanth:1995}\footnote{Is this correct
  cite for this?}. This is usually combined
with an AIC-based selection from among the closed-population models in
\citet{otis_etal:1978} which most often suggests heterogeneity (Model
Mh).  Almost all of these early methods were motivated by studies of
small mammals using classical ``trapping grids'' but, more recently,
their popularity has increased with the advent of new technologies and
especially related to non-invasive sampling methods such as camera
trapping. In particular, the series of papers by Karanth and Nichols
\citep{karanth:1995, karanth_nichols:1998, karanth_nichols:2002}
has led to fairly widespread adoption of these ideas.

Some of the heuristic ideas based on buffer strips do have some
technical justification in the sense of estimating parameters of an
underlying movement model from observed movements. For example, if we
let $x$ be a random variable indicating movement outcomes of an
individual about its  home range center, and suppose that $x$ has pdf
$g(x)$ then we can understand properties of MMDM by studying the
properties of the sample order statistics, as the maximum distance
moved is the sample range based on a sample of observations of
individual locations. 



%As an illustration, imagine a 1-dimensional
%system where individuals have a home range that amounts to a line
%segment. Then suppose that individual movements are $\mbox{uniform}(0,A)$. It
%can be shown that the sampling distribution of the sample range, R,
%scaled by $A$, say $R/A$ has a beta distribution, $\mbox{beta}(n-1,2)$
%\citep[][p. 235]{casella_berger:2002}
%and thus the diameter of the home range, i.e. $A$, is
%estimated (biasedly) by$ R/( (n-1)/(n+1) )$. For large $n$ we could then
%say that the sample range, i.e., ''maximum distance moved'' seems like a good estimator of home range diameter and, therefore, $R/2$ is an estimator of home-range radius.

%There are a number of technical issues that arise in attempting to use
%such heuristics to justify the application in practice. For one, the
%moments of the sample order statistics are strongly affected by sample
%size, which is typically quite small (per individual encountered) and
%thus, in general, are biased and estimated with variable precision
%depending on sample size. For example, the expected value of MMDM is
%$k(n)*A$ , i.e., the true home range diameter is related to observed
%MMDM by some function of sample size, $k(n)$, that increases to 1. In
%the case where the underlying movement model is uniform, $k(n) =
%(n-1)/(n+1)$ (from above) which motivates a formula for ``adjusting''
%observed MMDM for small sample size. We suspect that many such
%formulae are obtainable depending on the assumed movement distribution
%\citep[e.g., formula 6.16 in][]{obrien:2011}. We might also think about taking
%the {\it maximum} (over individuals) of the maximum distance moved
%because under the specific model considered here (iid uniform) then
%all individuals have the same home range radius. This increases our
%sample size ($n$) and thus the observed sample range should be more
%accurate.


%%Another issue of somewhat more importance (and less easy to
%rectify) is that the {\it observation} of movement outcomes is biased
%by the locations of traps. We cannot observe movements ``off the
%trapping grid'' (or between traps) and thus our observed movements
%will generally be smaller than expected under any particular model
%(the uniform in this case). Moreover, the trap spacing also induces a
%discreteness to the movements that causes a further level of
%approximation based on hypothetical movement
%distributions. Nevertheless, formal analysis of `` buffering''
%strategies based on sample order statistics under specific models for
%movement does at least provide some heuristic support for specific
%choices.  The interested reader should ponder the distribution of the
%sample minimum, maximum and range under other distributions such as a
%normal (and bivariate normal), exponential distribution and perhaps
%others. In addition, contemplate the effect of censoring of movements
%to some arbitrary limit ($B<A$) to mimic bias in observed movement
%outcomes due to a finite trap grid.

\subsection{Trapping webs}

The use of buffer strips is conventional and widespread due to the
heuristic appeal of that idea and its easy implementation, but other
conceptual approaches exist to address specific problems motivated by
the spatial context of capture-recapture data. D.R. Anderson came up
with the idea of the ``trapping web'' \citep{anderson_etal:1983} which
does not seem to have been widely adopted in practice.
% although there
%is a clear mathematical formalization to the trapping web design
%\citep{link_barker:1994}.
One reason for this is
the design is somewhat restrictive in the sense that it requires
a large number of traps be organized in close proximity to one
another.

\subsection{Temporary Emigration}

Another intuitively appealing idea is that by \citet{white_shenk:2000}
who discuss ``correcting bias of grid trapping estimates'' by
recognizing that the basic problem is like random temporary emigration
\citep{kendall_etal:1997} where individuals flip a coin with
probability $\phi$ to determine if they are ``available'' to be
sampled or not.  White and Shenk's idea was to estimate $\phi$ from
radio telemetry, as the proportion of time an individual spends in the
study area. They obtain the estimated super-population size by using
standard closed population models and then obtain density by $\hat{D}
= \hat{N}\hat{\phi}/A$ where $A$ is the nominal area of the trapping
array (e.g., minimum convex hull).  A problem with this approach is
that individuals that were radio collared represent a biased sample
i.e., you fundamentally have to sample individuals randomly from the
population {\it in proportion to their exposure to sampling} and that
seems practically impossible to accomplish. In other words, ``in the study area'' has no
precise meaning itself and is impossible to characterize in almost all capture-recapture studies.
Deciding what is ``in the study area'' is effectively the same as choosing an arbitrary buffer which defines
who is in the study area who who isn't.
That said, the temporary
emigration analogy is a good heuristic for understanding SCR models
and has a precise technical relevance to certain models.

Another very interesting idea is that of using some summary of
``average location'' as an individual covariate in standard
capture-recapture models. \citet{boulanger_mclellan:2001} use
distance-to-edge (DTE) as a covariate in the Huggins-Alho type of
model. \citet{ivan:2012} uses this approach in conjunction with an
adjustment to the estimated $N$ obtained by estimating the proportion of
time individuals are ``on the area formally covered by the grid''
using radio telemetry.  We do not dwell too much on these different
variations but we do note that the use of DTE as an individual
covariate amounts to some kind of intermediate model between simple
closed population models and fully spatial capture-recapture models,
which we address directly in Chapt. \ref{chapt.closed}.
%We note that no adjustment
%based on telemetry information is necessary if one were simply to
%place a prior distribution on the individual covariate (which is not
%to say that telemetry data isn't useful, just that the same objective
%can be achieved without telemetry data).

While these procedures are all heuristically appealing, they are also
essentially ad hoc in the sense that the underlying model remains
unspecified or at least imprecisely characterized and so there is
little or no basis for modifying, extending or generalizing the
methods. These methods are distinctly {\it not} model-based procedures
even though they might well be heuristically appealing under specific
movement models. Despite this, there seems to be an enormous amount of
literature developing, evaluating and ``validating'' these literally
dozens of heuristic ideas that solve specific problems, as well as
various related tweeks and tunings of them and really it hasn't led to
any substantive breakthroughs that are sufficiently general or
theoretically rigorous.



%A classical argument in favor of the HA model is
%that it ``doesn't require assumptions about the covariate'' but the
%assumption is explicit in capture-recapture models and thus it is
%natural to attack inference based on the ``joint likelihood''
%\citep{borchers_etal:2002}. This has proven necessary in certain other
%classes of individual covariate models in which natural models arise
%for the individual covariate, such as time-varying individual
%covariates \citep{bonner_schwarz:2006}, or covariates with measurement
%error (e.g., distance sampling; see
%\citet[][ch. 7]{royle_dorazio:2008}).
%The model-based formulation is easily adapted to standard
%individual covariate models as well \citep{royle:2008}. Throughout
%this book we rely heavily on Bayesian inference of the joint
%likelihood, using the formulation based on data-augmentation
%\citep{royle_etal:2007, royle_young:2008, royle:2009} though we also
%discuss the development of likelihood-based inference in chapter 5 and
%apply those methods in some cases.


\section{The Failure of Classical Capture-Recapture}

We briefly introduced and reviewed a number of classical techniques for applying non-spatial capture-recapture
models to studies of animal populations. These techniques, such as buffering, are based on many heuristically appealing
ideas. 
But these are just heuristics and do not resolve the essential, basic problem with conventional
(''non-spatial'') capture-recapture models which is that there is no linkage {\it in the model} between 
the quantity being informed by the data (i.e., $N$) and any stated or prescribed ``area'', $A$.
For capture-recapture models to provide a coherent framework for inference about population density,
$N$ has to scale, as part of the model, with $A$ so that the model imposes biological context
on $A$ (i.e., as the area over which the $N$ individuals reside). SCR models achieve this.

Put another way, 
ordinary capture-recapture methods are 
distinctly non-spatial. They don't admit spatial indexing of sampling
(observation) or
of individuals (process). This leads immediately to 3 main deficiencies:
 (1) there is no coherent basis for estimating density
 (2) non-spatial models {\it induce} a form of heterogeneity that can
 only at best be approximate by classical models of latent heterogeneity
 (3) ordinary models do not accommodate trap-level covariates which
 exist in a preponderance of studies.


We confront some of the issues that motivate the need for spatial
capture-recapture models by considering analysis of data from a study
design to estimate black bear abundance on the Fort Drum Military
Installation in upstate New York (see Ch. 3 for more details). The
specific data used here are encounter histories on 47 individuals
obtained from an array of 38 baited ``hair snares'' during June and
July 2006. The study area and locations of the 38 hair snares are
shown in Fig. \ref{fig.hairsnares}.  Barbed wire traps (see
Fig. \ref{fig.bearcat}) were baited and checked for hair samples each
week for eight weeks.  Analysis of these data appears in
\citet{gardner_etal:2010jwm} and we use the data in a number of analyses
in later chapters.

\begin{figure}
\begin{center}
\includegraphics[height=3in]{Ch1/figs/hairsnares}
\end{center}
\caption{Locations of black bear hair snares on Fort Drum.}
\label{fig.hairsnares}
\end{figure}

We regarded this data set as a standard capture-recapture data set -
an encounter history matrix with 47 rows and 8 columns with entries
$y_{ik}$, where $y_{ik}=1$ if individual $i$ was captured in sample
$k$ and $y_{ik}=0$ otherwise. There is a standard closed population
model, colloquially referred to as ``Model M0'' (see Ch. 3), which
assumes that encounter probability $p$ is constant for all individuals
and sample periods.  We fitted Model M0 to the Fort Drum data using
traditional likelihood methods, yielding the maximum likelihood
estimate (MLE) of $\hat{N} = 49.19$ with an asymptotic standard error
(SE) of $1.9$.

The key issue in using closed population models with such data is how
on earth do we interpret this estimate of $N=49.19$ bears? Does it
represent the entire population of Fort Drum? Certainly not -- we
merely sampled half of the Fort! So to get at the total bear
population size of Fort Drum , we'd have to convert our $\hat{N}$ to
an estimate of density and extrapolate. To get at density, then,
should we
assert that $N$ applies to the southern half of Fort Drum below some
arbitrary line? Surely bears move on and off of Fort Drum without
regard to hypothetical boundaries. Without additional information
there is simply no way of converting this estimate of $N$ to density,
and hence it is really not meaningful biologically. To resolve this
problem, we will adopt the customary approach of converting $N$ to $D$
by buffering the convex hull around the trap array. The convex hull
has area $157.135$ $km^2$. We follow \citet{bales_etal:2005} in
buffering the convex hull of the trap array by the radius of the mean
female home range size\footnote{Did Bales et al. actually do this?}.
The mean female home range radius was
estimated \citep{wegan:2008} for our study region to be $2.19$
km\footnote{Is this number right out of Wegan's disseration?}, and
the area of the convex hull buffered by $2.19$ km is $277.01$
km$^2$. ({\bf R}
commands to compute the convex hull, buffer it, and compute the area
are given in the {\bf R} package \mbox{scrbook} which accompanies the
book).  Hence, the estimated density
here is approximately $0.178$ bears/km$^2$ for an estimated population
size obtained using Model $M_0$.  We could assert that the problem has
been solved, go home, and have a beer.  But then, on the other hand,
maybe we should question this estimated home range radius from
\citep{wegan:2008} -- after all, home ranges can change for many reasons. Instead, we may decide to rely on a buffer width based on
one-half MMDM estimated from the actual hair snare data as is more customary
\citep{dice:1938}. In that case the buffer width is $1.19$ km, and the
resulting estimated density is increased to $0.225$ bears/ha$^2$ about
27 \% larger.  But wait - some studies actually found the full MMDM
\citep{parmenter_etal:2003} to be a more appropriate measure of
movement (e.g \citet{soisalo_cavalcanti:2006}). So maybe we should use
the full MMDM
which is $2.37$ km, pretty close to the telemetry-based estimate
and therefore providing a similar estimate of density ($0.171$
bears/ha$^2$). So in trying to decide how to buffer our trap array we
have already generated 3 density estimates. The crux of the matter is
obvious: Although it is intuitive that $N$ should scale with area --
the number of bears should go up as area increases and go down as area
decreases -- in this ad hoc approach of accounting for animal movement
$N$ remains the same, no matter what area we decide we sampled. The
number of bears and the area they live in are not formally tied
together within the model, because estimating $N$ and estimating the
area $N$ refers to are two completely independent analytical steps.

Unfortunately, our problems don't end here. In thinking about the use of model M0, we might naturally question
some of the basic assumptions that go into that model. The obvious one
to question is that which declares that $p$ is constant. One obvious
source of variation in $p$ is variation {\it among individuals}. We
expect that individuals may have more or less exposure to trapping due
to their location relative to traps.
%Maybe we could add a table of how many traps each bear was caught in
% #traps: 1   2  3  4  5  6  7  8 10
% #bears: 19 15  5  2  2  1  1  1  1

This has led many to consider
capture-recapture models that allow for individual heterogeneity in
$p$. Such models have the colloquial name of ``Model Mh.''
We fitted this model (see ch. 3 for details) to the Fort Drum data
using each of the 3 buffer widths previously described (telemetry, 1/2
MMDM and MMDM), producing the estimates reported in Table
\ref{intro.tab.fdests}. While we can tell by the models' AIC that Mh is
clearly favored by more than 30 units, we might still not be entirely
happy with our results. Clearly there is information in our data that
could tell us something about the exposure of individual bears to the
trap array -- where they were captured, and how many times -- but
since space has no representation in our model, we can't make use of
this information. Model Mh thus merely accounts for what we observe in
our data (some bears were more frequently captured than others) rather
than explicitly accounting for the processes that generated the data.

So what are we left with?  Our density estimates span  a range
from $0.17$ to $0.43$ bears/km$^2$ depending on which estimator of $N$ we use and
what buffer strip we apply. Should we feel strongly about one or the other?
AIC favors model Mh, but did it adequately account for the differences in exposure of individuals to the trap array? If so, which buffer should we
prefer? \footnote{Give AIC of models in table. Andy to finish. }
%Moreover, we could find more variations of
%model Mh to choose among, but see \citep{link:2003}.
And if we choose one type of buffer, how do we compare our density estimates to those from other studies that may opt for a different kind of buffer?
Clearly, there is no compelling solution to deriving density from our
estimate of $N$, and we are left not much wiser about bear density at
Fort Drum than we were before we conducted this analysis.

%%%% We could just finish this part off with a paragraph about these additional open questions - the whipped cream of problems on the capture-recapture sundae - including the trap-level covariates; or we could come up with some trap-level covariate example for the bears (different baits used, blabla).
Some of the open questions at this point:
How do we characterize uncertainty of the buffer ``estimate''?  And,
in what sense is the
buffer even an estimate of something? What is it an estimate of?
The summary here should be that there's not a compelling solution to be derived from
this ``estimate $N$ conjure up a buffer'' approach.
{\bf The main point that N doesn't scale with A is not made
  clearly here.}

\begin{table}[ht]
\centering
\caption{Table on estimates of D for the Fort Drum data
using M0 and Mh and different buffers.}
\begin{tabular}{ll|cc}
\hline
model & buffer &  $\hat{D}$ & SE \\ \hline
M0   & telemetry &  0.178 & 0.178 \\
M0    & MMDM     &  0.171 & 0.171\\
M0   & 1/2 MMDM  &  0.225 & 0.225\\
Mh(ln) & telemetry &0.341 & 0.144\\
Mh(ln) & MMDM    &  0.327 & 0.138\\
Mh(ln) & 1/2 MMDM & 0.432 & 0.183\\
\end{tabular}
\label{intro.tab.fdests}
\end{table}



\section{Extension of Closed Population Models}

The deficiency with classical closed population models is that they
have no spatial context. $N$ is just an integer parameter that applies
equally well to some population in a computer, estimating the number
of unique words in a book, or a bucket full of goldfish.  The question
of {\it where} the $N$ items belong is central both to interpretation
of data and estimates from all capture-recapture studies and, in fact,
to the construction of spatial capture-recapture models considered in
this book.  Surely it must matter whether the $N$ items exist as words
in a book, or goldfish in a bowl, or birds in a forest patch! That
classical closed population models have no spatial context leads to a
number of conceptual and methodological problems or limitations as we
have discussed and even encountered in our analyses so far.

Thus, the essential problem is that classical closed population models
are too simple - they ignore the spatial attribution of traps and
encounter events, movement and variability in exposure of individuals
to trap proximity, and they do not yield estimates of {\it density}.
These are not problems per se but rather just features
of this simple class of models, and they
should be addressed formally by the development of
more general models.



\subsection{The modern age}

%Spatial capture-recapture models are
%statistical and mathematical models that extend non-spatial
%``ordinary'' capture-recapture models to accommodate the spatial
%structure inherent in sampling animal populations - i.e., trap
%locations, individual locations, and individual use of space.

The solution to the various issues that arise in the application of
ordinary capture-recapture models is to extend the closed population
model so that $N$ becomes spatially explicit.  \
%A natural way is to
%define a point process \citep{efford:2004} that describes how
%individuals are organized in space and that, when points are
%aggregated over space, the value $N$ is derived in a meaningful way.
%Thus, in this book, we adopt the view that the locations of the $N$
%individuals in the population are a {\it realization of a spatial
%  point process}.
\citet{efford:2004} was the first to formalize an explicit model for
spatial capture-recapture problems in the context of trapping arrays.
He adopted a Poisson point process model to describe the distribution
of individuals and then what is essentially a distance sampling
formulation of the observation model which describes the probability
of detection as a function of individual location, regarded as a
latent variable governed by the point process model. While earlier
(and contemporary) methods of estimating density from trap arrays have
been ad hoc in the sense of lacking a formal description of the
spatial model, Efford achieved a formalization of the model,
describing explicit mechanisms governing the spatial distribution of
individuals and how they are encountered by traps, but
adopted a more or less ad hoc framework for inference under that
spatial model using a simulation based method known as inverse
prediction \citep{gopalaswamy:2012}.

Recently, there has been a flurry of effort devoted to formalizing
inference under this model-based framework for the analysis of spatial
capture-recapture data \citep{gopalaswamy:2012}. There are two distinct lines of work which
adopt the model-based formulation in terms of the underlying point
process but differ primarily by the manner in which inference is
achieved. One approach \citep{borchers_efford:2008} is a classical inference approach based on
likelihood, and the other \citep{royle_young:2008} adopts a
Bayesian framework for inference.

To motivate the origins and relevance of these approaches, we note
that, fundamentally, spatial capture-recapture models are related to
classical ``individual covariate'' models (colloquially referred to as
Huggins-Alho models) in capture-recapture \citep{huggins:1989,
  alho:1990}.  In particular, the individual covariate\footnote{have
  we mentioned what the individual covariate is, yet?} is observed in
these classical individual covariate models whereas it is not directly
observed in SCR models.  To accommodate that, a prior distribution for
the individual covariate is required. In essence then, SCR models are
similar to a fully model-based formulation of classical Huggins-Alho
models (see \citet{royle:2009}). Likelihood analysis
\citep{borchers_efford:2008} proceeds by removing the random effect
from the likelihood by integration whereas Bayesian analysis
\citep{royle_young:2008} proceeds by analyzing the conditional model
directly, usually by methods of Markov chain Monte Carlo (MCMC).




\subsection{Abundance as the Aggregation of a Point Process}

Spatial point process models represent a major methodological theme in
spatial statistics \citep[][ch. xyz]{cressie:1992} and they are
widely applied as models for many ecological phenomena
\citep{stoyan_penttinen:2000,illian_etal:2008}. Point process models apply to
situations in which the random variable in question represents the
locations of events or objects: trees in a forest, weeds in a field,
bird nests, etc.  As such, it seems natural to describe the
organization of individuals in space using point process models.

One
of the key features of SCR models is that the point locations are
latent, or unobserved, and we only obtain imperfect information about
the point locations by observing individuals at trap or observation
locations.  Thus, the realized locations of individuals represent a
type of ``thinned'' point process, where the thinning mechanism is not
random but, rather, biased by the observation mechanism.  It is
natural to think about the observed point process as some kind of a
compound or aggregate point process with a set of ``parent'' nodes
being the locations of individual home ranges or their centroids,
and the observed locations as
``offspring'' - i.e., a Poisson cluster process (PCP). In that
context, density estimation in SCR models is analogous to estimating the number of
parents in the PCP \citep{chandler_royle:2012}.
% Other types of point
% process models for the realized locations have direct relevance to SCR
% models (See \citet{chandler_royle:2012}, discussed in chapter XYZ).

In the context of SCR models, we suppose there is a point on the
landscape that we'll think of as a home range center or, if this is
unappealing, we can think of it as the centroid of an individual's
activities during the time of sampling. In general, this point is
unknown for any individual but if we could track an individual over
time and take many observations then we could perhaps get a good idea
of where that point is.  We'll think of the collection of these points
as defining the spatial distribution of individuals in the
population. Most of the recent developments in modeling and inference
from spatial encounter history data, including most methods discussed
in this book, are predicated on the view that individuals are
organized in space according to a relatively simple point process
model. More specifically, we assume that the collection of individual
activity centers are ``$iid$'' random variables distributed uniformly
over some region. This is consistent with the assumption that the
activity centers represent the realization of a Poisson point process
or, if the total number of activity centers if fixed, then this is
usually referred to as a binomial point process.

%%I think we could shorten the home range paragraph; I like the definition
%%'the centroid of an individual's
%%% activities during the time of sampling'. I think the definition of
%% home range is something like the colleciton of points/sites/areas
%% an animal uses over the course of its lifetime so it's vague anyway
%% and what that definition means for the different forms of home
%%ranges - territory, migratory species etc - is pretty much left open.
We use the terms home range or activity center interchangeably. The
term ``home range center'' suggests that models are only relevant to
animals that exhibit such behavior of establishing home ranges or
territories and since not all species do that, perhaps the
construction of SCR models based on this idea is flawed. However,
 the notion of a home range center is just a conceptual
device and we don't view this concept as being strictly consistent
with classical notions of animal territories. Rather our view is
that a home range or territory is inherently dynamic, temporally, and thus it is a
transient quantity - where the animal lived during the period of
study.  Therefore, whether or not individuals of a species establish home ranges
is irrelevant because, once a precise time period is defined, this defines a distinct region of space
that an individual must have occupied. In other
words, the definition of ``home range center'' is predicated, in a
sense, on the specification of a time period over which individuals
are studied. A term that might be less offensive than ``home range
center'' is ``centroid of space usage (CSU)'' which should not
conflict directly with preconceived understandings and interpretations
of home range\footnote{Utilization distribution is the same thing
  I guess?}.


\subsection{The state-space}

If we let ${\bf s}_{i}; i=1,2,\ldots,N$ be the locations of individual
activity centers, then the question ``what are the possible values of
${\bf s}$?'' needs to be addressed because the individual ${\bf
  s}_{i}$ are {\it unknown}. As a technical matter, we will regard
them as random effects and in order to apply standard methods of
statistical inference we need to provide a distribution for these
random effects.  In the context of the point process model, the
possible values of the point locations referred to as the
``state-space'' of the point process and this is some region or set of
points which we will denote by ${\cal S}$.
${\cal S}$ is a region within which points are located - essentially a
prior distribution for ${\bf s}_{i}$ (or, equivalently, the random effects
distribution).
%%Don't think prior has come up yet; maybe not that important here?
In animal studies as a description of
where individuals that could be captured are located it encloses our
study area -- the region within which we might have located traps or
detection devices.  The state-space of the point process should
accommodate all individuals that could have been captured in the study
area.

In the practical application of SCR models, in most cases estimates of
density will be relatively insensitive to choice of state-space (see
Section XYZ)
%%% I also think the rest of this paragraph could be postponed to a later chapter
unless there are meaningful features to the state-space
which should be accommodated. For example, if the region within which
traps are located contains a coastline or a huge body of water then
clipping that out of the state-space will typically have a large
effect on density. This should be expected because, insofar as the
state-space serves as a prior distribution on the latent variables
${\bf s}_{i}$ then, {\it the state-space is very much a
  component of the model. } We discuss choosing the state-space in
Chapter 4.

When the underlying point process is well-defined, including a precise
definition of the state-space, this in turn induces a precise
definition of the parameter $N$ ``population size'' as the number of
individual activity centers located within the prescribed state-space.
A deficiency with some classical methods of ``adjustment'' is they
attempted to prescribe something like a state-space - a ``sampled
area'' - except absent any precise linkage of individuals with the
state-space. SCR models formalize the linkage between individuals and
space and, in doing so, provide an explicit definition of $N$
associated with
a well-defined spatial region, and hence
density. In a sense, the whole idea of SCR models is that by defining
this point process and its state-space ${\cal S}$, this gives context and
meaning to $N$ which can be estimated directly for that specific
state-space. Thus, it is fixing ${\cal S}$ that resolves the problem of
``unknown area'' that was addressed previously (Section XXXX).
%% I find the next two sentences a little confusing
But the
existence of an explicit state-space ${\cal S}$ is kind of beside the
point -- ${\cal S}$ is really not always terribly important
itself. Instead, as soon as you give the latent variables ${\bf s}$ a
place to live, and this is recognized explicitly in the model upon which inference is based,
 then you achieve spatial explicitness of the model.










\subsection{Other elements of SCR models}

Broadly speaking we differentiate
between two situations: Sampling based on fixed arrays or sampling
based on ``search encounter'' methods. The former includes things like
camera traps, hair snares, mist nets and conventional traps. Fixed
arrays limit the observation location to pre-defined points, where
traps are located. Using such methods the model is a little simpler
because the ``movement process'' of individuals is confounded with the
``observation process''.
The 2nd type of model -- search encounter models -- typically
will allow locations in continuous space, possibly only restricted by
polygon boundaries \citep{royle_young:2008}.
Search-encounter data
usually allow for the separate modeling and estimation of movement
model parameters from encounter model parameters but not always,
depending on whether replication of the sampling is done.  The
classical distance sampling model with no replication (i.e., $t=1$) is a basic model
which confounds the two processes.


Depending on the type of device being considered, certain restrictions
on the observable variable are induced which suggest specific
probability models for the observable random variable, suggesting
either binomial, Poisson or multinomial (and possibly other)
observation models.
One type of a
device is what we think of as the classical ``camera trap'' and which
\citet{efford:2011} refers to as a ``proximity detector''. We can take
pictures of or detect any number of individuals and an individual can
be caught in any number of traps, and an arbitrary number of
times. Iid Bernoulli model is convenient but if you think the
re-encounters are valuable then you can have a frequency model.  Bear
hair snares are slightly different because you cannot differentiate
re-encounters.
The standard observation model that applies for ``single-catch''
\citep{efford_etal:2004} traps posits that individuals are encountered
in at most one trap per sample occasion and traps only hold one
individual.  Unfortunately we're really screwed in the single-catch
situation.
A ``multi-catch'' is like a mist-net or other things - individual is
captured and restrained but traps hold > 1 individual. In this case,
the observation model is a multinomial. There are
many variations on all of these models and new models.







\begin{comment}

\subsection{Why is density so important? }

Knowledge of population size is a fundamental piece of information in
conservation. Since the risk of a species/population going extinct is
a function of how many individuals of that species there are, much of
conservation-related research revolves around abundance. Consider, for
example, the concept of minimum viable population size � to assess
whether a population has a good chance of persistence over some time
frame we need to know how big it is to begin with. The idea of a
minimum viable population is reflected in many applied conservation
efforts. For example, in a range-wide assessment of the jaguar�s
population status, researchers were asked to delineate Jaguar
Conservation Units (JCU�s), of which one criterion was ``holding at
least 50 jaguars'' � a number considered a substantial population
\citep{sanderson_etal:2002}.

While the importance of abundance is indisputable, there are some
major issues associated with this measure. First, you cannot compare
mere values of abundance unless they refer to a specific area. If you
look at the IUCN Red List of Endangered Species entry for the
population status of the tiger, it will tell you that there are an
estimated 1700 tigers in India but only about 20 in Cambodia
\citep{chundawat_etal:2011}. Now, this will not automatically make you
lament the state of tiger conservation in Cambodia as compared to
India (although seeing these numbers you might well lament the state
of the tiger in general), because you know these numbers refer to
countries that are extremely different in size. Rather, if you wanted
to know something about where tigers are currently doing better,
you�d probably divide the number of tigers by the countries�
areas and compare tiger densities (turns out India�s tigers are
still doing better, not by a factor of 85, as mere abundances suggest,
but by a factor of 5). Although abundance and density are obviously
directly related to each other, they are different in their
applicability. Particularly, density as a scaled measure lets us
compare results across sites (as we just demonstrated for the tiger
example). In addition, some concepts incorporated in conservation
biology explicitly deal with density. For example, population growth
rate, home ranges or the probability of epidemics/disease spread are
density-dependent; the Allee effect links individual reproductive
success to population density in low-density populations.

Second, going back to the tiger example once more, we may wonder how
researchers even came up with these numbers for total population
size. Tiger abundance can be estimated using camera-traps, because
individuals have distinct stripe patterns so that photographic data
can be analyzed with capture-recapture models. But surely, no-one ever
camera-trapped the whole of India. This is a typical situation, even
on a much smaller scale. Ecologists generally sample only a small
fraction of the area used by a species or population, but want to
estimate total population size, i.e. the number of individuals
occurring in sampled {\it and unsampled} areas. If we can use the data
from sampled area to obtain a density estimate, explicit predictions
of abundance can be made to regions of any size (assuming that density
is constant across the region we are inferring to and equal to density
in the sampled area)\footnote{Note that the way total tiger abundance
  estimates are derived for India is much more complex than just
  looking at tiger density somewhere in India and then extrapolating
  it to the entire country (for details, see \citep{jhala_etal:2011});
  we merely use these numbers here to illustrate the general
  problem.}.

To summarize, density not only influences several ecological
processes, but also allows us to compare population status among
different sites; even where total abundance is of primary interest,
density can help us arrive at a total population estimate even when
we�re unable to survey the total population. Capture-recapture
models were designed to estimate abundance, but they generally cannot
be used to formally estimate density. This limitation of non-spatial
CR models has long been recognized (REF) and several ad hoc approaches
to overcome this problem have been devised. We will discuss those and
their shortcomings in XXX. The great advantage of SCR models over
non-spatial capture-recapture models is that they formally link
abundance and area so that they actually estimate density.


\end{comment}









\section{Summary: The Promise of Spatial Capture-Recapture}

Spatial capture-recapture models are an extension of ordinary
capture-recapture models to accommodate the spatial organization of
both individuals in a population and the observation mechanism (e.g.,
locations of traps).  They resolve problems which have been recognized
historically and for which various ad hoc solutions have been
suggested: heterogeneity in encounter probability due to the spatial
organization of individuals relative to traps, the ability to model
trap-level effects on encounter, and that a
well-defined sample area does not exist in most studies, and thus
estimates of $N$ using ordinary capture-recapture models cannot be
related directly to density.

However, SCR models are not merely an extension of technique but
rather they represent an extention in a much more
profound way in that they make ecological processes explicit in the
model -- processes of spatial organization of individuals, movement
and space-usage of individuals. While capture-recapture models have
existed for decades this is a completely new element of
closed capture-recapture models.
This is so profoundly important because
ecological scientists study elements of ecological theory using
observational data that exhibits various biases relating to the
observation mechanisms employed. In the context of capture-recapture,
we observe individual encounter history data from which we can use SCR
models to infer where individual live, how they organize themselves in
space and move around in space and how they interact with other
individuals.  Moreover, SCR models show great promise in their ability
to integrate explicit ecological theories directly into the models so
that we can directly test hypotheses about either space usage (e.g.,
Chapter XYZ) or movement (Chapter. XYZ) or the distribution of
individuals in space (Chapter XYZ). We imagine that in the near future
SCR models will include point process models that allow for
interactions among individuals such as inhibition or  clustering.

Thus, SCR models are capture-recapture models that enable ecologists
to explicitly integrate biological context and theory with encounter
history data, which is something that has always been the focus of
``open population'' models but never, until very recently, has been
considered formally in closed population models. We therefore believe
that SCR models will enable ecologists to test theories of space usage
and environmental effects, social behavior and other important
theories.


In chapter 2 we provide the basic analysis tools to understand and
analyze SCR models - namely GLMs with random effects, and their
analysis in R and WinBUGS.  Because SCR models represent extensions of
basic closed population models, we cover ordinary closed population
models in chapter 3 wherein, along with chapter 4, we will see that
SCR models are a type of individual covariate model, which are
conceptual and technical intermediates between Model Mh and classical
individual covariate models.  In subsequent chapters we will cover a
bunch of different types of SCR models related to the type of
encounter process - e.g., type of trap - and also different
embellishments of the basic model structure as alluded to in section
XYZ above.  We will consider many different extensions of SCR models
to accommodate covariates on encounter probability, and density. We
also consider important practical extensions such as SCR for open
populations (Chapter xyz), combining SCR data with auxiliary
information from telemetry (chapter XYZ) and multiple encounter
methods (chapter XYZ).


\chapter{
Bayesian Analysis of GL(M)Ms Using R/WinBUGS
}
\markboth{Bayesian Analysis of GLMMS}{}
\label{chapt.glms}

\vspace{.3in}

%%%% STUFF TO DO
%%% 1. Prior lack of invariance to transformation stuff: Reference and Figure
%%% 2. Full conditional example from ch. 6 copy notation
%%% 3. Check out algorithm environment
%%% 4. reference for sampling from f() with bounded support
%%% 5. need refs on choosing prior disributions
%%% 6. Check Bayesian p-value definition
%%% 7. FIX parameter notation! I have beta0 beta1 , alpha beta, and a,
%%%     b in the same chapter!   Use alpha beta probably?
%%% 8. spell check this document

A major theme of this book is that spatial capture-recapture models
are, for the most part, just generalized linear models (GLMs) wherein
the covariate, distance between trap and home range center, is
partially or fully unobserved  -- and therefore regarded as
a random effect. Such models
are usually referred to as Generalized Linear Mixed Models (GLMMs)
and, therefore, SCR models can be thought of as a specialized type of
GLMM. Naturally then, we should consider analysis of these slightly
simpler models in order to gain some experience and, hopefully,
develop a better understanding of spatial capture-recapture models.

In this chapter, we consider classes of GLM models - Poisson and
binomial (i.e., logistic regression) GLMs - that will prove to be
enormously useful in the analysis of capture-recapture models of all
kinds. Many readers are probably familiar with these models because
they represent probably
the most generally useful models in all of Ecology and, as
such, have received considerable attention in many introductory and
advanced texts. We focus on them here in order to introduce the
readers to the analysis of such models in {\bf R} and {\bf WinBUGS},
which we will
translate directly to the analysis of SCR models in subsequent
chapters.

Bayesian analysis is convenient for analyzing GLMMs because it allows
us to work directly with the conditional model -- i.e., the model that
is conditional on the random effects, using computational methods
known as Markov chain Monte Carlo (MCMC). Learning how to do Bayesian
analysis of GLMs and GLMMs in {\bf WinBUGS} is, in part, the purpose
of this chapter.  While we use {\bf WinBUGS} to do the Bayesian
computations, we organize and summarize our data and execute {\bf
  WinBUGS} from within {\bf R} using the useful package \mbox{\tt
  R2WinBUGS} \citep{sturtz_etal:2005}.  \citet{kery:2010}, and
\citet{kery_schaub:2011} provide excellent introductions to the basics
of Bayesian analysis and GLMs at an accessible level. We don't want to
be too redundant with those books and so we avoid a detailed
treatement of Bayesian methodology - instead just providing a cursory
overview so that we can move on and attack the problems we're most
interested in related to spatial capture-recapture.  In addition,
there are a number of texts that provide general introductions to
Bayesian analysis, MCMC, and their applications in Ecology including
\citet{mccarthy:2007}, \citet{kery:2010}, \citet{link_barker:2009},and
\citet{king_etal:2009}.


While this chapter is about Bayesian analysis of GLMMs, such models
are routinely analyzed using likelihood methods too, as discussed by
\citet{royle_dorazio:2008}, and \citet{kery:2010}. Indeed, likelihood
analysis of such models is the primary focus of many applied
statistics texts, a good one being \citet{zuur_etal:2009}. Later in
this book, we will use likelihood methods to analyze SCR models but,
for now, we concentrate on providing a basic introduction to Bayesian
analysis because that is the approach we will use in a majority of
cases in later chapters.


\section{ Notation}

We will sometimes use conventional ``bracket notation'' \index{bracket
  notation} to refer to
probability distributions. If $y$ is a random variable the $[y]$
indicates its distribution or its probability density/mass function
(pdf, pmf) depending on context. If $x$ is another random variable
then $[y|x]$ is the conditional distribution of $y$ given $x$, and
$[y,x]$ is the joint distribution of $y$ and $x$. To differentiate
specific distributions in some contexts we might label them $g(y)$,
$g(y|\theta)$, $f(x)$, or similar. We will also write $y \sim
\mbox{Normal}(\mu,\sigma^{2})$ to indicate that $y$ ``is distributed as'' a normal
random variable with parameters $\mu$ and $\sigma^{2}$. The expected value
or mean of a random variable is $E[y] = \mu$ ,and $Var[y] = \sigma^{2}$ is
the variance of $y$.  To indicate specific observations we'll use an
index such as ``$i$''. So, $y_{i}$ for $i=1,2,\ldots,n$ indicates
observations for $n$ individuals. Finally, we write $\Pr(y)$ to indicate specific probabilities, i.e., of events ``$y$'' or similar.


To illustrate these concepts and notation, suppose $z$ is a binary
outcome (e.g., species occurrence) and we might assume the model: $z
\sim \mbox{Bern}(p)$ for observations.  Under this model $\Pr(z=1) =
\psi$, which is also the expected value $E[z] = \psi$. The variance is
$Var[z] = \psi*(1-\psi)$ and the probability mass function (pmf) is $[z]
= \psi^{z} (1-\psi)^{1-z}$. Sometimes we write $[z|\psi]$ when it is
important to emphasize the conditional dependence of $z$ on $\psi$. As
another example, suppose $y$ is a random variable denoting whether or
not a species is detected if an occupied site is surveyed. In this
case it might be natural to express the pmf of the observations $y$
{\it conditional} on $z$. That is, $[y|z]$. In this case, $[y|z=1]$ is
the conditional pmf of $y$ given that a site is occupied, and it is
natural to assume that $[y|z=1] = \mbox{Bern}(p)$ where $p$ is the
``detection probability'' - the probability that we detect the
species, given that it is present. The model for the observations $y$
is completely specified once we describe the other conditional pmf
$[y|z=0]$. For this conditional distribution it is sometimes
reasonable to assume $\Pr(y=1|z=0) = 0$ (\citet{mackenzie_etal:2002};
see also \citet{royle_link:2006}). That is, if the species is absent,
the probability of detection is 0. This implies that
$\Pr(y=0|z=0)=1$. To allow for situations in which the true state $z$
is unobserved, we  assume that $[z]$ is Bernoulli with parameter
$\psi$.  In this case, the marginal distribution of $y$ is
\[
 [y] = [y|z=1]Pr(z=1) + [y|z=0]Pr(z=0)
\]
because $[y|z=0]$ is a point mass at $y=0$, by assumption, then
\[
\Pr(y=1) = p \psi
\]
And
\[
\Pr(y=0) = (1-p)*\psi + (1-\psi)
\]


\section{
GLMs and GLMMs}
We have asserted already that SCR models work out most of the time to
be variations of GLMs and GLMMs. Some of you might therefore ask: What
are GLMs and GLMMs, anyhow?   These models are covered extensively in
many very good applied statistics books and we refer the reader
elsewhere for a detailed introduction. We think \citet{kery:2010},
\citet{kery_schaub:2011}, and \citet{zuur_etal:2009} are all
accessible treatments of considerable merit. Here, we'll give the 1
minute
treatment of GLMMs, not trying to be complete but rather only
to preserve a coherent organization to the book.


The generalized linear model (GLM) is an extension of standard linear
models by allowing the response
variable to have some distribution from the exponential family of
distributions (i.e., not just normal). This includes the normal
distribution but also dozens of others such as the Poisson, binomial,
gamma, exponential, and many more. In addition, GLMS allow the
response variable to be related to the predictor variables (i.e.,
covariates) using a
link function, which is usually nonlinear.  Finally, GLMs typically
accommodate a relationship between the mean and variance. The
classical reference for GLMs is \citet{nelder_wedderburn:1972} and
also \citet{mccullagh_nelder:1989}.
The GLM consists of three components:
\begin{itemize}
\item[1.] A probability distribution for the dependent variable $y$,
from a class of probability distributions known as the exponential family.
\item[2.] A ``linear predictor'' $\eta = {\bf X}{\bm \beta}$  .
\item[3.] A link function $g$ that relates $E[y]$ to the linear predictor, $E[y] = \mu = g^{-1}(\eta)$. Therefore $g(E[y]) = \eta$.
\end{itemize}

The dependent variable $y$ is assumed to be an outcome from a
distribution of the exponential family which includes many common
distributions including the normal, gamma, Poisson, binomial, and many
others. The mean of the distribution of $y$ is assumed to depend on predictor variables $x$ according to
\[
 g(E[y]) = {\bf x}'{\bm \beta}
\]
where $E[y]$ is the expected value of $y$, and ${\bf x}'{\bm \beta}$
is termed the {\it linear predictor}, i.e., a linear function of the
predictor variables with unknown parameters ${\bm \beta}$ to be
estimated.  The function $g$ is the link function. In standard GLMs,
the variance of $y$ is a function $V$ of the mean of $y$: $Var(y) =
V(\mu)$ (see below for examples).

A Poisson GLM posits that $y \sim \mbox{Poisson}(\lambda)$ with $E[y]
=\lambda$ and usually the model for the mean is specified using the
{\it log link function} by
\[
log(\lambda_{i}) = \beta_0 + \beta_{1}*x_{i}
\]
The variance function is $\mbox{V}(y_{i}) = \lambda_{i}$.  The
binomial GLM posits that $y_{i} \sim \mbox{Binomial}(K,p)$ where $K$
is the fixed sample size parameter and $E[y_{i}] = K*p_{i}$. Usually
the model for the mean is specified using the {\it logit link
  function} according to
\[
 logit(p_{i}) = \beta_{0} + \beta_{1}*x_{i}
\]
Where $logit(u) = log(u/(1-u))$.  The inverse-logit function, $g^{-1}$ ,
is a function we will refer to as ``expit'', so that $expit(u) =
exp(u)/(1+exp(u))$.

A GLMM is the extension of GLMs to accommodate ``random
effects''. Often this involves adding a normal random effect to the
linear predictor, and so a simple example is:
\[
 \log(\lambda_{i}) = \alpha_{i} + \beta_{1}*x_{i}
\]
where
\[
 \alpha_{i} \sim \mbox{Normal}(\mu,\sigma^{2})
\]
%Many other probability distributions and formulations of the linear
%predictor might be considered.  It is not widely appreicated that
%the link function and
%distribution of the random effect interact directly to affect the
%implied probability distribution of the linear predictor. For the
%Poisson case just considered, $\lambda_{i}$ has a log-normal
%distribution. However, if we set $\lambda_{i} = \alpha_{i}exp(\beta*x_{i})$
%where $\alpha_{i}$ has a Gamma distribution, then $\lambda_{i}$ has
%similarly a gamma distribution with modified scale parameter.  These
%different model assumptions are seldom evaluated formally in practice
%although in many practical situations (in ecology), they imply
%specific things about the ecological process being studied
%(e.g., see \citet{royle_dorazio:2008} section XYZ on occupancy
%logit/cloglog etc..).



\section{Bayesian Analysis}

Bayesian analysis is unfamiliar to many ecological researchers because
older cohorts of ecologists were largely educated in the classical
statistical paradigm of frequentist inference. But advances in
technology and increasing exposure to benefits of Bayesian analysis
are fast making Bayesians out of people or at least making Bayesian
analysis an acceptable, general, alternative to classical, frequentist
inference.

Conceptually, the main thing about Bayesian inference is that it uses
probability directly to characterize uncertainty about things we don't
know.  ``Things'', in this case, are parameters of models and, just as
it is natural to characterize uncertain outcomes of stochastic
processes using probability, it seems natural also to characterize
information about unknown ``parameters'' using probability. At least
this seems natural to us and, we think, most ecologists either
explicitly adopt that view or tend to fall into that point of view
naturally.  Conversely, frequentists use probability in many different
ways, but never to characterize uncertainty about
parameters\footnote{To hear this will be shocking to some readers
  perhaps.} Instead, frequentists use probability to characterize the
behavior of {\it procedures} such as estimators or confidence
intervals (see below), which can lead to some inelegant or unnatural
interpretations of things.  It is paradoxical that people readily
adopt a philosophy of statistical inference in which the things you
don't know (i.e., parameters) should {\it not} be regarded as random
variables, so that, as a consequence, one cannot use probability to
characterize ones state of knowledge about them.


\subsection{Bayes Rule}

As its name suggests, Bayesian analysis makes use of Bayes' rule in
order to make direct probability statements about model
parameters. Given two random variables $z$ and $y$, Bayes rule relates
the two conditional probability distributions $[z|y]$ and $[y|z]$ by
the relationship:
\[
[z|y] = [y|z][z]/[y]
\]
Bayes' rule itself is a mathematical fact and there is no debate in
the statistical community as to its validity and relevance to many
problems. Generally speaking, these distributions are characterized as
follows: $[y|z]$ is the conditional probability distribution of $y$
{\it given} $z$, $[z]$ is the marginal distribution of $z$ and $[y]$
is the marginal distribution of $y$. In the context of Bayesian
inference we usually associate specific meanings in which $[y|z]$ is
thought of as ``the likelihood'', $[z]$ as the ``prior'' and so on. We
leave this for later because here the focus is on this expression of
Bayes rule as a basic fact of probability.

As an example of a simple application of Bayes rule,
consider the problem of determining species presence at a sample
location based on imperfect survey information. Let $z$ be a binary
random variable that denotes species presence $(z=1)$ or absence
$(z=0)$, let $\Pr(z=1) = \psi$ where $\psi$ is usually called
occurrence probability, ``occupancy'' \citep{mackenzie_etal:2002} or ``prevalence''.
Let $y$ be the {\it observed} presence
($y=1$) or absence ($y=0$), and let $p$ be the probability that a
species is detected in a single survey at a site given that it is
present. Thus, $\Pr(y=1|z=1)=p$. The interpretation of this is that,
if the species is present, we will only observe presence with
probability $p$. In addition, we assume here that $\Pr(y=1|z=0) =
0$. That is, the species cannot be detected if it is not present which
is a conventional view adopted in most biological sampling problems (but
see \citet{royle_link:2006}).
If we survey a site $T$ times but never detect the species,
then this clearly does not imply that the species is not present
($z=0$) at this site. Rather, our degree of belief in $z=0$ should be
made with a probabilistic statement
$\Pr(z=1|y_1=0,\ldots,y_{T}=0)$. If the $T$ surveys are independent so
that we might regard $y_{t}$ as $iid$ Bernoulli trials, then the total
number of detections, say $y$, is Binomial with probability $p$ then
we can use Bayes rule to compute the probability that it is present
given that it is not detected in $T$ samples. In words, the expression
we seek is:
\[
\Pr(\mbox{present} | \mbox{not detected}) = \frac{\Pr(\mbox{not detected} |
  \mbox{present})\Pr(\mbox{present})}{\Pr(\mbox{detected})}
\]
Mathematically, this is
\begin{eqnarray*}
\Pr(z=1|y=0) &= &\Pr(y=0|z=1)\Pr(z=1)/\Pr(y=0)  \\
             &= & [(1-p)^{T} \psi]/[ (1-p)^T \psi + (1-\psi) ].
\end{eqnarray*}
To apply this,
suppose that $T=2$ surveys are done at a wetland for a species of
frog, and the species is not detected there. Suppose further that $\psi
= .8$ and $p = .5$ are obtained from a prior study.  Then the
probability that the species is present at this site is
$.25*.8/(.25*.8 + .2) = 0.50$. That is, there seems to be about a
50/50 chance that the site is occupied despite the fact that the
species wasn't observed there.

In summary, Bayes' rule provides a simple linkage between the
conditional probabilities $[y|z]$ and $[z|y]$ which is useful whenever
one needs to deduce one from the other.
Bayes' rule as a basic fact of probability is not disputed.


\subsection{Bayesian Inference}


What is controversial to some is the scope and manner in which Bayes
rule is applied by Bayesian analysts. Bayesian analysts assert that
Bayes rule is relevant, in general, to all statistical problems by
regarding all unknown quantities of a model as realizations of random
variables - this includes ``data'', latent variables, and also
``parameters''. Classical (non-Bayesian) analysts sometimes object to
regarding ``parameters'' as outcomes of random variables. Classically,
parameters are thought of as ``fixed but unknown'' (using the
terminology of classical statistics). Of course, in Bayesian analysis
they are also unknown and, in fact, there is a single data-generating
value and so they are also fixed. The difference is that this fixed
but unknown value is regarded as having been generated from some
probability distribution. Specification of that probability
distribution is necessary to carryout Bayesian analysis, but it is not
required in classical frequentist inference.


To see the general relevance of Bayes rule in the context of
statistical inference, let $y$ denote observations - i.e., ``data'' -
and let $[y|\theta]$ be the observation model (often colloquially
referred to as the ``likelihood'').  Suppose theta is a parameter of
interest having (prior) probability distribution $[\theta]$. These are
combined to obtain the posterior distribution using Bayes' rule, which
is:
\[
 [\theta|y]= [y|\theta][\theta]/[y]
\]
Asserting the general relevance of Bayes rule to all statistical
problems, we can conclude that the two main features of Bayesian
inference are that: (1) ``parameters'' $\theta$ are regarded as realizations of
a random variable and, as a result, (2) inference is based on the
probability distribution of the parameters given the data,
$[\theta|y]$,
which is
called the posterior distribution. This is the result of using Bayes
rule to combine ``the likelihood'' and the prior distribution.  The
key concept is regarding parameters as realizations of a random
variable because, once you admit this conceptual view, this leads
directly to the posterior distribution, a very natural quantity upon
which to base inference about things we don't know -  including
parameters of statistical models.  In particular, $[\theta|y]$ is a
probability distribution for $\theta$ and therefore we can make direct
probability statements to characterize uncertainty about
$\theta$.

The denominator of our invocation of Bayes rule, $[y]$,
is the marginal distribution of the data $y$.  We note without further
remark right now that, in many practical problems, this can be an
enormous pain to compute. The main reason that the Bayesian paradigm
has become so popular in the last 20 years or so is because methods
exist for characterizing the posterior distribution that do not
require that we possess a mathematical understanding of $[y]$, i.e.,
we never have to compute it or know what it looks like, or know
anything specific about it.

A common misunderstanding on the distinction between Bayesian and
frequentist inference goes something like this ``in frequentist
inference parameters are fixed but unknown but in a Bayesian analysis
parameters are random.'' At best this is a sad caricature of the
distinction and at worst it is downright wrong. What is true is that,
to a Bayesian, parameters are random variables. However, a Bayesian
assumes, just like a frequentist, that there was a single
data-generating value of that parameter - a fixed, and unknown value
that produced the given data set.
The distinction between Bayesian and frequentist approaches is that
Bayesians regard the parameter as a random variable, and its value as
the outcome of a random value, on par with the observations. This
allows Bayesians to use probability to make direct probability
statements about parameters. Frequentist inference procedures do not
permit direct probability statements to be made about parameter
values -- because parameters are not random variables!

While we can understand the conceptual basis of Bayesian inference
merely by understanding Bayes rule -- that's really all there is to it
-- it is not so easy to understand the basis of classical
``frequentist'' inference which is mostly
like\footnote{Characterization from Sims REF XYZ} a ``basket of
methods'' with little coherent organization. What is mostly coherent
in frequentist inference is the manner in which items in this basket
of methods are evaluated -- the performance of a given procedure is
evaluated by ``averaging over'' hypothetical realizations of $y$,
regarding the {\it estimator} as a random variable. For example, if
$\hat{\theta}$ is an estimator of $\theta$ then the frequentist is
interested in $E_{y}[\hat{\theta}|y]$ which is used to characterize
bias. If the expected value of $\hat{\theta}$, when averaged over
realizations of $y$, is equal to $\theta$, then $\hat{\theta}$ is
unbiased.

The view of parameters as fixed constants and estimators as random variables
leads to interpretations that are not so straightforward. For
example confidence intervals having the interpretation ``95\%
probability that the interval contains the true value" and p-values
being "the probability of observing an outcome as extreme or more than
the one observed.'' These are far from intuitive interpretations to
most people.  Moreover, this is conceptually probblematic to some
because the hypothetical realizations that characterize the
performance of our procedure we will never get to observe.

While we do tend to favor Bayesian inference for the conceptual
simplicity (parameters are random, posterior inference), we mostly
advocate for a pragamatic non-partisian approach to inference because,
frankly, some of these ``bucket of methods'' are actually very
convenient in certain situations as we will see in later chapters.


\subsection{Prior distributions}


The prior distribution $[\theta]$ is an important feature of Bayesian
inference. As a conceptual matter,
the prior distribution characterizes ``prior beliefs'' or ``prior
information'' about a parameter. Indeed,
an oft-touted benefit of Bayesian analysis is the ease with which
prior information can be included in an analysis.
However, more commonly, the prior is chosen to
express a lack of prior information, even if previous studies have
been done and even if the investigator does in fact know quite a bit
about a parameter.
This is because
the manner in which prior information is embodied in a prior (and the
amount of information) is
usually very subjective and thus the result can wind up being very
contentious, e.g., different investigators might report different
results based on subjective assessments of things. Thus it is usually
better to ``let the data speak'' and use priors that reflect absence
of information beyond the data set being analyzed.

But still the need occasionally arises to embody prior information or
beliefs about a parameter formally into the estimation scheme.
 In SCR models we often have a parameter that is closely linked
to ``home range radius'' and thus auxiliary information on the home
range size of a species can be used as prior information (e.g., see
\citet{chandler_royle:2012} ; also chapter XYZ).

XXXXXXXX
you gonna add something about priors and their potential to truncate posteriors here?
XXXXXXXX

XXXXXXXX

noninformative prior on one scale is informative on another scale.
e.g., flat prior on logit(p) is very different from uniform(0,1) on
p...
show graphic......

reference to non-invariance of prior distributions to transformation......

XXXXXXXX

\subsection{Posterior Inference}

In Bayesian inference, we are not focusing on estimating a single
point or interval but rather on characterizing a whole distribution --
the posterior distribution -- from which one can report any summary of
interest. A point estimate might be the posterior mean, median, mode,
etc..  In many applications in this book, we will compute 95\%
Bayesian intervals using the 2.5\% and 97.5\% quantiles of the
posterior distribution. For such intervals, it is correct to say
$\Pr(L < \theta < U) = 0.95$. That is, "the probability that $\theta$
is between $L$ and $U$ is $0.95$". 

As an
example, suppose we conducted a Bayesian analysis to estimate
detection probability of some species at a study site (p), and we
obtained a posterior distribution of beta(20,10) for the parameter
p. The following R commands demonstrate how we make inferences based
upon summaries of the posterior distribution. Fig. \ref{densityvsdetection.fig} shows the
posterior along with the summary statistics.

\begin{verbatim}
> (post.median <- qbeta(0.5, 20, 10))
[1] 0.6704151
> (post.95ci <- qbeta(c(0.025, 0.975), 20, 10))
[1] 0.4916766 0.8206164
\end{verbatim}

Thus, we can state that there is a 95\% probability that $\theta$ lies
between 0.49 and 0.82.

\begin{figure}
\begin{center}
\includegraphics[height=2.5in]{Ch2/figs/densityvsdetection}
%get figure file from Ch7 folder
\end{center}
\caption{Probability density plot of a hypothetical posterior distribution of beta(20,10); dashed lines indicate mean and upper and lower 95\% interval}
\label{densityvsdetection.fig}
\end{figure}

It is not a subtle thing that this
cannot be said using frequentist methods - although people tend to say
it anyway and not really understand why it is wrong or even that it is
wrong. This is actually a failing of frequentist ideas and the
inability of frequentists to get people to overcome their natural
tendency to use probability - which is something that, as a
frequentist, you simply cannot do in the manner that you would like
to.



Posterior inference is the main practical element of Bayesian
analysis. We get to make an inference conditional on the data that we
actually observed - i.e., what we actually know.  To us, this seems
logical - to condition on what we know. Conversely, frequentist
inference is based on considering average performance over
hypothetical unobserved data sets (i.e., the ``relative frequency''
interpretation of probability).  Frequentists know that their
procedures work well when averaged over all hypothetical, unobserved,
data sets but no one ever really knows how well they work for the
specific data set analyzed. That seems like a relevant question to
biologists who oftentimes only have their one, extremely valuable,
data set.  This distinction comes into play a lot in exposing
philosophical biases in the peer review of statistical analyses in
ecology in the sense that, despite these opposing conceptual views to
inference (i.e. conditional on the data you have, or averaged over
hypothetical realizations), those who conduct a Bayesian analysis are
often (in ecology, almost always) required to provide a frequentist
evaluation of their Bayesian procedure.

\subsection{Small sample inference}

Using Bayesian inference, we obtain an estimate of the posterior
distribution which is an exhaustive summary of the state-of-knowledge
about an unknown quantity. It is the posterior distribution - not an
estimate of that thing. It is also not, usually, an approximation
except to within Monte Carlo error (in cases where we use simulation
to calculate it).  One of the great virtues of Bayesian analysis which
is not really appreciated is that it is completely valid for any
particular sample size. i.e., it is $[\theta|y]$, as precise as we
claim it to be based on our ability to do calculations, for the
particular sample size and observations that we have even if we have
only a single datum $y$.  The same cannot be said for almost all
frequentist procedures in which estimates or variances are very often
(almost always in practice) based on ``asymptotic approximations'' to
the procedure which is actually being employed.

There seems to be a prevailing view in statistical ecology that
classical likelihood-based procedures are virtuous because of the
availability of simple formulas and procedures for carrying out
inference, such as calculating standard errors, doing model selection
by AIC, and assessing goodness-of-fit.  In large samples, this may be
an important practical benefit, but the theoretical validity of these
procedures cannot be asserted in most situations involving small
samples.  This is not a minor issue because it is typical in many
wildlife sampling problems - especially in surveys of carnivores or
rare/endangered species - to wind up with a small, sometimes extremely
small, data set. For example, a recent paper on the fossa
(Cryptoprocta ferox), an endangered carnivore in Madagascar, estimated
an adult density of 0.18 adults / km sq based on 20 animals captured
over 3 years \citep{hawkins_racey:2005}. A similar paper on the
endangered southern river otter (Lontra provocax) estimated a density
of 0.25 animals per river km based on 12 individuals captured over 3
years \citep{sepulveda_etal:2007}. \citet{gardner_etal:2010} analyzed
data from a study of the Pampas cat, a species for which very little
is known, wherein only 22 individual cats were captured .during the
two year period.  \citet{trolle_kery:2005} reported only 9 individual
ocelots captured and \citet{jackson_etal:2006} captured 6 individual
snow leopards using camera trapping. Thus, studies of rare and/or
secretive carnivores necessarily and flagrantly violate one of Le
Cam's Basic Principles, that of ``If you need to use asymptotic
arguments, do not forget to let your number of observations tend to
infinity.''\citep{lecam:1990}.

The biologist thus faces a dilemma with such data. On one hand, these
datasets, and the resulting inference, are often criticized as being
poor and unreliable. Or, even worse\footnote{Actual quote from a
  referee}, ``the data set is so small, this is a poor analysis.''  On
the other hand, such data may be all that is available for species
that are extraordinarily important for conservation and management.
The Bayesian framework for inference provides a valid, rigorous, and
flexible framework that is theoretically justifiable in arbitrary
sample sizes. This is not to say that one will obtain precise
estimates of density or other parameters, just that your inference is
coherent and justifiable from a conceptual and technical statistical
point of view. That is, we report the posterior probability
$\Pr(D|data)$ which is easily interpretable and just what it is
advertised to be and we don't need to do a simulation study to
evaluate how well some approximate $\Pr(D|data)$ deviates from the
actual $\Pr(D|data)$ because they are precisely the same quantity.



\section{Characterizing posterior distributions by MCMC simulation}

In practice, it is not really feasible to ever compute the marginal
probability distribution $\Pr(y)$, the denominator resulting from
application of Bayes' rule. For decades this impeded the adoption of
Bayesian methods by practitioners. Or, the few Bayesian analyses done
were based on asymptotic normal approximations to the posterior
distribution. While this was useful stuff from a theoretical and
technical standpoint and, practically, it allowed people to make the
probability statements that they naturally would like to make, it was
kind of a bad joke around the Bayesian water-cooler to, on one hand,
criticize classical statistics for being, essentially, completely ad
hoc in their approach to things but then, on the other hand, have to
devise various approximations to what they were trying to
characterize. The advent of Markov chain Monte Carlo (MCMC) methods
has made it easier to calculate posterior distributions for just about
any problem to arbitrary levels of precision.

Broadly speaking, MCMC is a class of methods for drawing random
numbers (sampling or simulating) from the target posterior
distribution.  Thus, even though we might not recognize the posterior
as a named distribution or be able to analyze its features
analytically, e.g., devise mathematical expressions for the mean and
variance, we can use these MCMC methods to obtain a large sample from
the posterior and then use that sample to characterize features of the
posterior. What we do with the sample depends on our intentions --
typically we obtain the mean or median for use as a point estimate,
and take a confidence interval based on Monte Carlo estimates of the
quantiles.  These are estimates, but not like frequentist
estimates. Rather, they are Monte Carlo estimates with an associated
Monte Carlo error which is largely determined arbitrarily by the
analyst. They are not estimates qualified by a sampling distribution
as in classical statistics. If we run our MCMC long enough then our
reported value of $E[\theta|y]$ or any feature of the posterior
distribution is precisely what we say it is. There is no ``sampling
variation'' in the frequentist sense of the word.  In summary, the
MCMC samples provide a Monte Carlo characterization of {\it the}
posterior distribution.


\section{What Goes on Under the MCMC Hood}

We will develop and apply MCMC methods in some detail for spatial
capture-recapture models in chapter \ref{chapt.mcmc}. Here we provide
a simple illustration of some basic ideas related to the practice of MCMC.

A type of MCMC method relevant to most problems is Gibbs sampling (REF
XYZ XYZ),
which is based on the idea of iterative simulation from the ``full
conditional'' distributions (also called conditional posterior
distributions). The full conditional distribution for an unknown
quantity is the conditional distribution of that quantity given every
other random variable in the model - the data and all other
parameters. For example, for a normal regression model with $y \sim
\mbox{Normal}(\alpha + \beta x , 1)$ then the two full conditionals are, in
symbolic terms,
\[
[\alpha|y,\beta]
\]
 and
\[
[\beta|y,\alpha].
\]
We might use our knowledge of probability to identify these
mathematically. In particular, by Bayes' Rule, $[\alpha|y,\beta] =
[y|\alpha,\beta][\alpha|\beta]/[y|\beta]$ and similarly for
$[\beta|y,\alpha]$. For example, if we have priors for $[\alpha]$ and $[\beta]$
which are also normal distributions, some algebra reveals that
XXXX COPY NOTATION FFROM CH. 6 XXXXX
\[
[\alpha|y,\beta] = Normal(ybar,...weighted variance here...).
\]
Similarly,
\[
 [\beta|y,\alpha] is normal(........)
\]

The MCMC algorithm for this model has us simulate in succession,
repeatedly, from those two distributions. See \citet{gelman_etal:2004}
for more examples of Gibbs sampling for the normal model. A
conceptual representation of the MCMC algorithm for this simple model
is therefore:
XXXX Check out ALGORITHM environment XXXXX
\begin{verbatim}
 Algorithm

       0. Initialize $\alpha$ and $\beta$

       Repeat{
           1. Draw a new value of $\alpha$ from Eq. \ref{xyz}

           2. Draw a new value of $\beta$ from Eq. \ref{xyz}
       }
\end{verbatim}

As we just saw for this simple ``normal-normal'' model it is sometimes
possible to specify the full conditional distributions
analytically. In general, when certain so-called conjugate prior
distributions are chosen, the form of full conditional distributions
is similar to that of the observation model. In this normal-normal
case, the normal distribution for the mean parameters is the conjugate
prior under the normal model, and thus the full-conditional
distributions are also normal. This is convenient because, in such
cases, we can simulate directly from them using standard methods (or
{\bf R}
functions).  But, in practice, we don't really ever need to know such
things because most of the time we can get by using a simple
algorithm, called the Metropolis-Hastings (henceforth ``MH'')
algorithm, to obtain samples from these full conditional distributions
without having to recognize them as specific, named, distributions.
This gives us enormous freedom in developing models
and analyzing them without having to resolve them mathematically
because to implement the MH algorithm we need only identify the full
conditional distribution up to a constant of proportionality, that
being the marginal distribution in the denominator (e.g., $[y|\beta]$
above).

We will talk about the Metropolis-Hastings algorithm shortly, and we
will use it extensively in the analysis of SCR models (e.g., chapter
\ref{chapt.mcmc}).

\subsection{Rules for constructing full conditional distributions}
\label{glms.sec.rules}

The basic strategy for constructing full-conditional distributions for
devising MCMC algorithms can be reduced conceptually to a couple of
basic steps summarized as follows:
\begin{itemize}
\item[(step 1)] Collect all stochastic components of the model;
\item[(step 2)] Recognize and express the full conditional in question
  as proportional to the product of all components;
\item[(step 3)] Remove the ones that don't have the focal parameter in them.
\item[(step 4)] Do some algebra on the result in order to identify the resulting pdf or pmf.
\end{itemize}
Of the 4 steps, the last of those is the main step that requires quite
a bit of statistical experience and intuition because various
algebraic tricks can be used to reshape the mess into something
noticeable - i.e., a standard, named distribution. But step 4 is not
necessary if we decide instead to use the Metrpolis-Hastings algorithm
as described below.

To illustrate for computing $[\alpha|y,\beta]$ we first apply step 1
and identify the model components as: $[y|\alpha, \beta]$, $[\alpha]$
and $[\beta]$. Step 2 has us write $[\alpha|y,\beta] \propto
[y|\alpha,\beta][\alpha][\beta]$.  Step 3: We note that $[\beta]$ is not a
function of alpha and therefore we remove it to obtain $[\alpha|y,\beta]
\propto [y|\alpha,\beta][\alpha]$. Similarly we obtain $[\beta|y,\alpha]
\propto [y|\alpha,\beta][\beta]$. We apply step 4 and manipulate
these algebraically to arrive at the result or, alternatively, we can
sample them indirectly using the Metropolis-Hastings algorithm (see
below).


\subsection{Metropolis-Hastings algorithm}

The Metropolis-Hastings (MH) algorithm is a completely generic method for
sampling from any distribution, say $f(\theta)$. In our applications,
$f(\theta)$ will typically be the full conditional distribution of
$\theta$.
While we sometimes use Gibbs sampling, we seldom
use ``pure'' Gibbs sampling because we might use MH to sample from one
or more of the full conditional distributions.
When the MH algorithm is used to sample from  full
conditional distributions of a Gibbs sampler the resulting hybrid algorithm is
called
 {\it Metrpolized Gibbs sampling} or
more commonly {\it Metropolis-within-Gibbs}.
Shortly we will
actually construct such an algorithm for a simple class of models.

The MH algorithm generates candidates from some
proposal or candidate-generating distribution, that may be conditional
on the current value of the parameter, denoted by
$h(\theta^{*}|\theta^{t})$. Here, $\theta^{*}$ is the {\it candidate}
or proposed
value and $\theta^{t}$ is the current value, i.e., at iteration $t$ of
the MCMC algorithm.  The proposed value
is accepted with probability

\[
r = \frac{ f(\theta^{*}) h(\theta^{t}|\theta^{*})}
    {f(\theta^{t}) h(\theta^{*}|\theta^{t}) }
\]
which we call the MH acceptance probability.
This ratio can sometimes be $>1$ in which case we set it equal to
1. It is useful to note that $h()$ can be anything at all. No matter
the choice of $h()$, we can evaluate this ratio numerically because
the marginal $f(y)$ cancels from both the numerator and
denominator, which is the magic of the MH algorithm.


\section{Practical Bayesian Analysis and MCMC}

There are a number of really important practical issues to be
considered in any Bayesian analysis and we cover some of these briefly
here.

\subsection{Choice of prior distributions}

{\bf XXX integrate this material with previous section on prior
distributions XXXXXX}

Bayesian analysis requires that we choose prior
distributions for all of the structural parameters of the model (we
use the term structural parameter to mean all parameters that aren't
customary thought of as latent variables). We will strive to use
priors that are meant to express little or no prior information -
default or customary ``non-informative'' or diffuse priors. This will
be $\mbox{Unif}(a,b)$ priors for parameters that have a natural
bounded support and, for parameters that live on the real line we use
either (1) diffuse normal priors; (2) ``improper'' uniform priors or
(3) sometimes even a bounded $\mbox{Unif}(a,b)$ prior if that greatly
improves the performance of {\bf WinBUGS} or other software doing the MCMC
for us.  In {\bf WinBUGS} a prior with low ``precision'', $\tau$, where
$\tau = 1/\sigma^2$, such as $\mbox{Norm}(0,.01)$ will typically be
used. Of course $\tau = 0.01$ ($\sigma^{2} = 100$) might be very
informative for a regression parameter that has a high
variance. Therefore, we recommend that predictor variables {\it
  always} be standardized. Clearly there are a lot of choices for
ostensibly non-informative priors, and the degree of
non-informativeness depends on the parameterization. For example, a
natural non-informative prior for the intercept of a logistic
regression
\[
\mbox{logit}(p_{i}) = \alpha + \beta x_{i}
\]
Would be $[\alpha] = \mbox{const}$ which is the same as saying $a \sim
Unif(\infty,infty)$, the customary improper uniform prior.
However, we might also use a prior on the parameter $p0
= logit^{-1}(a)$, which is $Pr(y=1)$ for the value $x=0$. Since $p0$ is a
probability a natural choice is $p0 \sim Unif(0,1)$. These two priors can
affect results (see Chapter 3.XYZ), yet they are both sensible
non-informative priors. Choice of priors and parameterization is
very much problem-specific and often largely subjective. Moreover, it
also affects the behavior of MCMC algorithms and therefore the analyst
needs to pay some attention to this issue and possibly try different
things out.
XXX REFS on prior distributions XXXXXX

\subsection{Convergence and so-forth}

Once we have carried-out an analysis by MCMC, there are many other
practical issues that we have to confront.  One of the most important
is ``have the chains converged?'' Since we do not know what the stationary posterior distribution of our Markov chain should look like (this is the whole point of doing an MCMC approximation), we effectively have no means to assess whether it has truly converged to this desired distribution or not. Most MCMC algorithms only guarantee
that, eventually, the samples being generated will be from the target
posterior distribution, but no-one can tell us how long this will take. Also, you only now the part of your posterior distribution that the Markov chain has explored so far– for all you know the chain could be stuck in a local maximum, while other maxima remain completely undiscovered.  Acknowledging that there is truly nothing we can do to ever proof convergence of our MCMC chains, there are several things we can do to increase the degree of confidence we have about the convergence of our chains. Some problems are easily detected using simple plots.  Typically a period of transience is
observed in the early part of the MCMC algorithm, and this is usually
discarded as the ``burn-in'' period. The quick diagnostic to whether convergence has been achieved is that
your Markov chains look ``grassy'' -- see Fig.  \ref{glms.fig.grassy}
below.  
Another way to check
convergence is to update the parameters some more and see if the
posterior changes. Yet another option, and one generally implemented in WinBUGS, is to run several Markov chains and to start them off at different initial values that are overdispersed relative to the posterior distribution. Such initial values help to explore different areas of the parameter space simultaneously; if after a while all chains oscillate around the same average value, chances are good that they indeed converged to the posterior distribution. \footnote{Running several parallel chains is computationally expensive. But extra computational demands are not the only and by no means the major concern some people voice when it comes to running several parallel MCMC chains to assess convergence. Again, consider the fact that we do not know anything about the true form of the posterior distribution we are trying to approximate. How do we, then, know how to pick overdispersed initial values? We don’t – all we can do is pick overdispersed values relative to our expectations of what the posterior should look like. To use a quote from the home page of Charlie Geyer, a Bayesian statistician from the University of Minnesota, ``If you don't know any good starting points [...], then restarting the sampler at many bad starting points is [...] part of the problem, not part of the solution.'' (http://users.stat.umn.edu/~charlie/mcmc/diag.html). His suggestion is that your only chance to discover a potential problem with your MCMC sampler is to run it for a very long time. But again, there is no way of knowing how long is long enough.
It is up to you to decide, which school of thoughts appeals more to you – one long versus several parallel Markov chains. Irrespectively, part of developing an MCMC sampler should be to make sure (within reasonable limits) that you are not missing regions of high posterior density because of the way you specify your starting values. Once you have explored the behavior of your chain under a – reasonable – range of starting values, you may feel comfortable enough to run only one long chain.} Gelman and Rubin came up with the so-called``R-hat'' statistic ($\hat{R}$) or Brooks-Gelman-Rubin statistic
 that essentially compares within-chain and between-chain variance to check for convergence of multiple chains (\citep{gelman_etal:1996}). $\hat{R}$ should be close to 1 if the Markov
chains have converged and sufficient posterior samples have been
obtained. In practice, $\hat{R} = 1.2$ is probably good enough for
some problems.  For some models you can't actually realize a low
$\hat{R}$. E.g., if the posterior is a discrete mixture of distributions
then you can be misled into thinking that your Markov chains
have not converged when in fact the chains are just jumping back and
forth in the posterior state-space. 
So, for example, model
selection methods (section XYZ) sometimes suggests non-convergence.
Another situation is when one of the parameters is on the boundary of
the parameter space which might appear to be very poor mixing, but all
within some extreme region of the parameter space.\footnote{it would
  be nice if we could compile examples of this later in the book and
  reference back to this point}.
This
kind of stuff is normally ok and you need to think really hard about
the context of the model and the problem before you conclude that your
MCMC algorithm is ill-behaved.

Some models exhibit ``poor mixing'' of the Markov chains or what
people might also say ``have not coverged'' (or ``slow convergence'')
which is a term we would disagree with because the samples might well
be from the posterior (i.e., the Markov chains have converged to the
proper stationary distribution) but simply mix around the posterior
rather slowly. Anyway, poor mixing can happen for a huge number of
reasons -- when parameters are highly correlated (even confounded), or
barely identified from the data, or the algorithms are very terrible
and probably many other reasons.  Slow mixing equates to high
autocorrelation in the Markov chain - the successive draws are highly
correlated, and thus we need to run the MCMC algorithm much longer to
get an effective sample size that is sufficient for estimation - or to
reduce the MC error to a tolerable level.  A strategy often used to
reduce autocorrelation is ``thinning'' - i.e., keep every $m^{th}$
value of the Markov chain output. However, thinning is necessarily
inefficient from the stand point of inference - you can always get
more precise posterior estimates by using all of the MCMC output
regardless of the level of autocorrelation
\citep{maceachern_berliner:1994}. Practical considerations might
necessitate thinning, even though it is statistically inefficient. For
example, in models with many parameters or other unknowns being
tabulated, the output files might be enormous and unwieldy to work
with. In such cases, thinning is perfectly reasonable. In many cases,
how well the Markov chains mix is strongly influenced by
parameterization, standardization of covariates, and the prior
distributions being used. Some things work better than others, and the
investigator should experiment with different settings and
remain calm when things don't work out perfectly. MCMC is an
art, and a science.


{\bf Is the posterior sample large enough?}  The subsequent samples generated from a Markov chain are not iid samples from the posterior distribution, due to the correlation amongst samples introduced by the Markov process and the sample size has to be adjusted to account for the autocorrelation in subsequent samples (see Chapter 8 in \citet{robert_casella:2010} for more details). This adjusted sample size is referred to as the effective sample size. Checking the degree of autocorrelation in your Markov chains and estimating the effective sample size your chain has generated should be part of evaluating your model output. WinBUGS will automatically return the effective sample size for all monitored parameters. If you find that your supposedly long Markov chain has only generated a very short effective sample, you should consider a longer run. What exactly constitutes a reasonable effective sample size is hard to say. A more palpable measure of whether you've run your chain for enough iterations is the time-series or Monte Carlo error – the 'noise' introduced into your samples by the stochastic MCMC process. The MC error is printed by default in
summaries of BUGS output. You want that to be smallish relative to the
magnitude of the parameter and this might depend on the purpose of the
analysis. For a preliminary analysis you might settle for a few
percent whereas for a final analysis then certainly less than 1\% is
called for, but you can run
your MCMC
algorithm as long as it takes. A consequence of the MC error is that even for the exact same model, results will always be different. Thus, as a good rule
of thumb you should never report
MCMC results to more than 2 decimal places.
Note that MC error in summaries of the
posterior is not the same as having an ``approximate'' solution in a
standard likelihood analysis or similar.  The approximate SE in
likelihood inference is actually wrong in its actual value.... XYZ.


\subsection{Bayesian confidence intervals}

The 95\% Bayesian interval based on percentiles of the posterior
is not a unique interval - there are many of them - and the so-called
``highest posterior density'' (HPD) interval is the narrowest
interval. We might compute that frequently because it is easy to do
with an integer parameter which $N$ is (See the next chapter). The
95 \% HPD is not often exactly 95\% but usually slightly more
conservative than nominal because it is the narrowest interval that
contains at least 95\%  of the posterior mass.

\subsection{Estimating functions of parameters}

A benefit of analysis by MCMC is that we can seamlessly estimate
functions of parameters by simply tabulating the desired function of
the simulated posterior draws. For example, if $\theta$ is the
parameter of interest and let $\theta^{(i)}$ for $i=1,2,\ldots,M$ be
the posterior samples of $\theta$. Let $\eta = exp(\theta)$, then a
posterior sample of $\eta$ can be obtained simply by computing
$exp(\theta^{(i)})$ for $i=1,2,\ldots,M$. We give another example in
section
\ref{glms.sec.xopt}
below and throughout this book.
Almost all SCR models in this book involve at least 1 derived
parameter. For example, density $D$ is a derived parameter, being a
function of population size $N$ and the area $A$ of the underlying
state-space of the point process (see chapter \ref{chapt.scr0}).

\section{Bayesian Analysis using WinBUGS}

We won't be too concerned with devising our own MCMC algorithms for
every analysis
although we will do that a few times for fun.  More often, we
will rely on the freely available software package {\bf WinBUGS} or
{\bf JAGS}
for doing this.  We will always execute these {\bf BUGS} engines from
within {\bf R} using the \mbox{\tt R2WinBUGS} (REF XYZ XYZ) or
\mbox{\tt rjags} pacages. {\bf WinBUGS} and {\bf JAGS} are  MCMC black boxes
that takes a pseudo-code description (i.e., written in the {\bf BUGS}
language) of all of the relevant stochastic
and deterministic elements of a model and generates an MCMC algorithm
for that model. But you never get to see the algorithm. Instead,
{\bf WinBUGS}/{\bf JAGS} will run the algorithm and just return the Markov chain output
- the posterior samples of model parameters.

The great thing about using the {\bf BUGS} language is that it forces
you to become intimate with your statistical model - you have to write
each element of the model down, admit (explicitly) all of the various
assumptions, understand what the actual probability assumptions are
and how data relate to latent variables and data and latent variables
relate to parameters, and how parameters relate to one another.

While we normally use
{\bf WinBUGS} or {\bf JAGS} in this book, we note that {\bf
 OpenBUGS} is the current active development tree of the {\bf BUGS}
language. See \citet[][ch.xyz]{kery:2010} and
\citet[][appendix xyz]{kery_schaub:2011} for more on practial analysis
in {\bf WinBUGS}.
That book should also be consulted
for a more comprehensive introduction to using {\bf WinBUGS}. In this
example, we're going to accelerate pretty fast.

\subsection{Linear Regression in WinBUGS}

We provide a brief introductory example of a normal regression model
using a small simulated data set. The following commands are executed
from within your R workspace, the command line being indicated by
\mbox{\tt ``>''}. First, simulate a covariate $x$ and observations $y$ having
prescribed intercept, slope and variance:
\begin{verbatim}
> x<-rnorm(10)
> mu<- -3.2+ 1.5*x
> y<-rnorm(10,mu,sd=4)
\end{verbatim}
The {\bf BUGS} model specification for a normal regression model is
written within {\bf R} as a character string input to the command
\mbox{\tt cat()} and
then dumped to a text file named \mbox{\tt normal.txt}:
\begin{verbatim}
> cat("
model {
   for (i in 1:10){
      y[i]~dnorm(mu[i],tau)        # the "likelihood"
      mu[i]<- beta0 + beta1*x[i]   # the linear predictor
     }
   beta0~dnorm(0,.01)              # prior distributions
   beta1~dnorm(0,.01)
   sigma~dunif(0,100)
   tau<-1/(sigma*sigma)            # tau is a derived parameter
}
",file="normal.txt")
\end{verbatim}
Alternatively, you
can write the model specifications directly within a text file and
save it in your current working directory, but we do not usually take
that approach in this book.

{\bf Remarks:} {\bf 1.} {\bf WinBUGS} parameterizes the normal in
terms of the mean and inverse-variance, called the precision. Thus,
\mbox{\tt dnorm(0,.01)} implies a variance of 100;
{\bf 2.} We typically use diffuse normal priors for mean parameters, $\beta_0$ and $\beta_1$ in this case, but sometimes we might use uniform priors with suitable bounds -B and +B.
{\bf 3.} We typically use a $\mbox{Unif}(0,B)$ prior on standard
deviation parameters
(Gelman XXX 2006 XXXX). But sometimes we might use a gamma prior on the precision parameter $\tau$.
{\bf 4.} In a {\bf WinBUGS} model file, every variable referenced in
the model description has to be
either data, which will be input (see below), a random variable which
must have a probability distribution associated with it using the
``\verb#~#'', or it has to be a derived parameter connected to variables and
data using ``\mbox{\tt <-}''.


To fit the model, we need to describe various data objects to {\bf
  WinBUGS}. In particular,
we create an {\bf R} list object called \mbox{\tt data} which
are the data objects identified in the BUGS model file.
 In the example, the
data consist of two objects which exist as $y$ and $x$ in the {\bf R}
workspace and also in the {\bf WinBUGS} model definition.
 We also have to create an {\bf R} function
that produces a list of starting values \mbox{\tt inits} that get sent to
{\bf WinBUGS}.
 Finally, we identify
the names of the parameters (labeled correspondingly in the {\bf WinBUGS}
model specification) that we want {\bf WinBUGS} to save the MCMC output
for. In this example, we will ``monitor'' the parameters
$\beta_0$, $\beta_1$, $\sigma$ and $\tau$.
{\bf WinBUGS} is executed using the {\bf R} command
\mbox{\tt bugs()}.
We set the option \mbox{\tt debug=TRUE} if we want the {\bf WinBUGS}
GUI to stay open (useful for analyzing MCMC output and looking at the
{\bf WinBUGS} error log). Also, we set \mbox{\tt working.dir=getwd()}
so that {\bf WinBUGS} output files and the log file are saved in the
current {\bf R} working directory.
  All of these activities look like this:
{\small
\begin{verbatim}
 library("R2WinBUGS")    # "attach" the R2WinBUGS library
 data <- list ( "y","x")
 inits <- function()
  list ( beta1=rnorm(1),beta0=rnorm(1),sigma=runif(1,0,2) )
 parameters <- c("beta0","beta1","sigma","tau")
 out<-bugs (data, inits, parameters, "normal.txt", n.thin=2, n.chains=2,
             n.burnin=2000, n.iter=6000, debug=TRUE,working.dir=getwd())
\end{verbatim}
}

{\bf Remarks:} A common question is ``how should my data be
formatted?'' That depends on how you describe the model in the {\bf
  BUGS} language, how your data are input into {\bf R} and
subsequently formatted.  There is no unique way to describe any
particular model and so you have some flexibility. We talk about data
format further in the context of capture-recapture models and SCR
models in chapter \ref{chapt.scr0} and elsewhere.  In general,
starting values are optional but we recommend to always provide
reasonable starting values for structural parameters, but are not
always necessary for random effects.  Note that the previously created
objects defining data, initial values and parameters to monitor are
passed to the function \mbox{\tt bugs()}.  In addition, various other
things are declared: The number of Markov chains (\mbox{\tt
  n.chains}), the thinning rate (\mbox{\tt n.thin}),
the number of burn-in iterations (\mbox{\tt n.burnin}) and the total
number of iterations
(\mbox{\tt n.iter}).
To develop a detailed understanding of the various parameters and
settings used for MCMC, consult a basic reference such as
\citet{kery:2010}.



You should execute all of the commands given above and then look at
the resulting output. Kill the {\bf WinBUGS} GUI and the data will be
read back into {\bf R} (or specify \mbox{\tt debug=FALSE}).  We don't
want to give instructions on how to navigate and use the GUI - see XYZ
REF (XYZ) for that.
The object \mbox{\tt out} prints important
summaries by default (this is slightly edited):

{\small
\begin{verbatim}
> print(out,digits=2)
Inference for Bugs model at "normal.txt", fit using WinBUGS,
 2 chains, each with 6000 iterations (first 2000 discarded), n.thin = 2
 n.sims = 4000 iterations saved
          mean   sd  2.5%   25%   50%   75% 97.5% Rhat n.eff
beta0    -2.43 1.84 -6.21 -3.50 -2.42 -1.34  1.27    1  4000
beta1     2.62 1.54 -0.42  1.68  2.62  3.57  5.67    1  4000
sigma     5.29 1.66  3.11  4.14  4.95  6.05  9.39    1  4000
tau       0.05 0.02  0.01  0.03  0.04  0.06  0.10    1  4000
deviance 59.85 3.24 56.18 57.47 59.00 61.37 68.32    1   840

For each parameter, n.eff is a crude measure of effective sample size,
and Rhat is the potential scale reduction factor (at convergence, Rhat=1).

DIC info (using the rule, pD = Dbar-Dhat)
pD = 2.6 and DIC = 62.4
\end{verbatim}
}

{\bf Remarks:} (1) convergence is assessed using the $\hat{R}$
statistic -- which we might sometimes write ``$Rhat$''. A value of $Rhat$ near 1
indicates convergence; (2) DIC is the
``deviance information criterion'' \citep{spiegelhalter_etal:2002}
(see section \ref{glms.sec.modsel})
 which
some people use in a manner similar to AIC although it is recognized
to have some problems in hierarchical models \citep{millar:2009}. We
evaluate this in the context of SCR models in chapter XYZ XYZ.

\subsection{Inference about functions of model parameters}
\label{glms.sec.xopt}

Using the MCMC draws for a given model we can easily obtain the
posterior distribution of any function of model parameters.  We showed
this in the above example by providing the posterior of $\tau$ when
the model was parameterized in terms of standar deviation $\sigma$.
 As another example, suppose that the
normal regression model above had a quadratic response function of the
form
\[
	E(y_i) = \beta_0 + \beta_1 x_i + \beta_2 x_{i}^{2}
\]
Then the optimum value of $x$, i.e., that corresponding to the optimal
expected response, can be found by setting the derivative of
this function to 0 and solving for $x$. We find that
\[
df/dx = \beta_1 +
2*\beta_2 x = 0
\]
yields that $x_{opt} = -\beta_1/(2*\beta_2)$.  We can just
take our posterior draws for $beta_1$ and $beta_2$ and obtain a
posterior sample of $x_{opt}$ by this simple calculation. As an exercise, take
the normal model above and simulate a quadratic response and then
describe the posterior distribution of $x_{opt}$.


\section{Model Checking and Selection}
\label{glms.sec.modsel}

In general terms model checking - or assessing the adequacy of the
model - and model selection are quite thorny issues and, despite
contrary and, sometimes, strongly held belief among practitioners, there are not
really definitive, general solutions to either problem. We're against
dogma on these issues and think people need to be open-minded about
such things and recognize that models can be useful whether or not
they pass certain statistical tests. Some models are intrinsically
better than others because they make more biological sense or foster
understanding or achieve some objective that some  bootstrap
or other goodness-of-fit test can't decide for you. That said, it
gives you some confidence if your model seems adequate and we try to
provide some fit assessment in most real applications of SCR models
We provide a very brief overview of concepts here, but provide more
detailed coverage in chapter \ref{chapt.gof}.
See also
\citet[][ch. xyz]{kery:2010} and
\citet[][ch. xyz]{link_barker:2009}
for specific context related to Bayesian
model checking and selection.

\subsection{Goodness-of-fit}

Goodness-of-fit testing is an important element of any analysis
because  our model represents a general set of hypotheses
about the ecological and observation processes that generated our
data. Thus, if our model ``fits'' in some statistical or scientific
sense, then we believe it to be consistent with the hypotheses that
went into the model. More formally, we would conclude that the data
are {\it not inconsistent} with the hypotheses, or that the model
appears adequate. If we have enough
data, then of course we will reject any set of statistical hypotheses.
Conversely, we can always come up with a model that fits by making the
model extremely complex. Despite this paradox, it seems to us that
simple models that you can understand should usually be preferred even
if they don't fit, for example if they embody essential mechanisms
central to our understanding of things, or
if we think that some contributing factors to lack-of-fit are minor or
irrelevant to the scientific context and intended use of the model.
In other words, models can be useful irrespective of whether they fit
according to some formal statistical test of fit.  Yet
the tension is there to obtain fitting models, and this comes naturally at
the expense of models that can be easily interpreted and studied and
effectively used.
Moreover, conducting goodness-of-fit tests is
not always so easy to do. Moreover, it is never really easy (or
especially convenient) to decide if your goodness-of-fit test is worth
anything. It might have 0 power!
Despite this,
we recommend attempting to assess model fit in real applications,
as a general rule, and we provide some basic guidance here and some more
specific to SCR models in
chapter \ref{chapt.gof}.

To evaluate goodness-of-fit in Bayesian analyses, we will most often
use the Bayesian p-value \citep{gelman_etal:1996}.  The basic idea is to define
a fit statistic or ``discrepancy measure'' and compare the posterior distribution of that
statistic to the posterior predictive distribution of that statistic
for hypothetical perfect data sets for which the model is known to be correct. For
example, with count frequency data, a standard measure of fit is the
sum of squares of the ``Pearson residuals'',
\[
D(y_i,\theta) = \frac{(y_i - E(y_i))^{2}}{Var( y_{i} )}
\]
The fit statistic based on the squared residuals is
\[
FIT = \sum_{i} D(y_{i},\theta)^{2}
\]
which can be computed at each iteration of a MCMC algorithm given the
current values of parameters that determine the
 response distribution.  At the same time (i.e., at each MCMC
 iteration),
the equivalent statistic is computed for a
``new'' data set, simulated using the current parameter values. The
Bayesian p-value is simply the posterior probability $\Pr(\mbox{Fit} >
\mbox{Fit}_{new})$\footnote{Check this definition!}
 which should be close to $0.50$ for a good model -- one that
 ``fits'' in the sense that the observed data set is
 consistent with realizations simulated under the model being fitted
 to the observed data. In practice
we judge ``close to 0.50'' as being ``not too close to 0 or 1'' and,
as always, closeness is somewhat subjective. We're happy with anything
$>.1$ and $<.9$ but might settle for $>.05$ and $<0.95$. In summary,
the Bayesian p-value seems like a bootstrap idea, is easy to compute,
and widely used as a result.

Another useful fit statistic is the Freeman-Tukey
statistic\footnote{Ref for this?}, in which
\[
D({\bf y},\theta) = \sum_{i} ( \sqrt(y_{i}) - \sqrt(e_{i}) )^2
\]
\citep{brooks_etal:2000}, where $y_{i}$ is the observed value of
observation $i$ and $e_{i}$ its expected value. In contrast to a
chi-square discrepancy, the Freeman-Tukey statistic removes the need
to pool cells with small expected values.


\subsection{Model Selection }

For model selection we typically use three different methods: First
is, let's say, common sense. If a parameter has posterior mass
concentrated away from 0 then it seems like it should be regarded as
important - that is, it is ``significant.''  This approach seems to
have fallen out of favor with all of the interest over the last 10 or
15 years on model selection in ecology. It seems reasonable to us.


For regression problems we sometimes use the factor weighting idea
which is to introduce a set of binary variables $w_{k}$ for variable
$k$, and express the model as, e.g., for a single covariate model:
 \[
 E(y_i) = \alpha + w \beta x_{i}
\]
where $w$ is given a Bernoulli prior distribution with some prescribed
probability. E.g., $w \sim Bern(0.50)$ to provide a prior probability
of 0.50 that variable $x$ should be an element of the linear
predictor. The posterior probability of the event $w=1$ is a gauge of
the importance of the variable $x$. i.e., high values of $\Pr(w=1)$
indicate stronger evidence to support that ``$x$ is in the model''
whereas values of $\Pr(w=1)$ close to 0 suggest that $x$ is less
important.



This idea seems to be due to \citet{kuo_mallick:1998}\footnote{ Is
  this also what people call Zellner's G-priors?} and see
\citet[][ch. XXXX]{royle_dorazio:2008} for an example in the context
of logistic regression. This approach seems to even work sometimes
with fairly complex hierarchical models of a certain form. E.g.,
\citet{royle:2008} applied it to a random effects model to evaluate
the importance of the random effect component of the model.  The main
problem with this approach is that its effectiveness and results will
typically be highly sensitive to the prior distribution on the
structural parameters (e.g., see \citet[][table xyz]{royle_dorazio:2008}).
The reason for this is obvious: If $w = 0$ for the current
iteration of the MCMC algorithm, so that $\beta$ is sampled from the
prior distribution, and the prior distribution is very diffuse, then
extreme values of $\beta$ are likely. Consequently, when the current value of
$\beta$ is
far away from the mass of the posterior when $w=1$, then the Markov
chain may only jump from $w=0$ to $w=1$ infrequently.  One seemingly
reasonable solution to this problem (Aitken XYZ FIND THIS
XXXXX\footnote{see Royle 2008 paper for reference}) is to fit the full
model to obtain posterior distributions for all parameters, and then
use those as prior distributions in a ``model selection'' run of the
MCMC algorithm.  This seems preferable to more-or-less arbitrary restriction of
the prior support to improve the performance of the MCMC algorithm.

A third method that that we advocate is subject-matter
context. It seems that there are some situations -- some models -- where one should not
have to do model selection because it is necessitated by the specific
context of the problem, thus rendering a formal hypotesis test pointless
\citep{johnson:1999}.
SCR models are such an example. In SCR models, we will see that
``spatial location'' of individuals is an element of the model. The
simpler, reduced, model is an ordinary capture-recapture model which
is not spatially explicit (i.e., chapter \ref{chapt.closed}),
but it seems silly and pointless to think about actually using the
reduced model even if we could concoct some statistical test to refute
the more complex model.  The simpler model is manifestly wrong but,
more importantly, not even a plausible data-generating model!
Other examples are when effort, area or
sample rate is used as a covariate. One might prefer to have such things in
models regardless of whether or not they pass some statistical litmus
test (although one can always find referees to argue for pedantic procedure
over thinking).


Many problems can be approached using one of these methods but there
are also broad classes of problems that can't and, for those, you're
on your own. In later chapters we will address model selection in
specific contexts and we hope those will prove useful for a majority
of the situations you encounter.


\section{Poisson GLMs}
\label{glms.sec.poisson}

The Poisson GLM (also known as ``Poisson regression'') is probably the
most relevant and important class of models in all of ecology. The
basic model assumes observations $y_{i}; i=1,2,...,n$ follow a Poisson
distribution with mean $\lambda$ which we write
\[
 	y_{i} \sim \mbox{Poisson}(\lambda)
\]
Commonly $y_{i}$ is a count of animals or plants at some point in
space and lambda might depend on i. For example, $i$ might index point
count locations in a forest, BBS route centers, or sample quadrats, or
similar.  If covariates are available it is typical to model them as
linear effects on the log mean. If $x(i)$ is some measured covariate
associated with observation $i$. Then,
\[
 	log(x(i)) = \alpha  + \beta*x(i)
\]

While we only specify the mean of the Poisson model directly, the
Poisson model (and all GLMs) has a ``built-in'' variance which is
directly related to the mean. In this case, $\mbox{Var}(y) = \mbox{E}(y) =
\lambda$. Thus the model accommodates a linear increase in variance
with the mean.

\subsection{Important properties of the Poisson distribution}
\label{glms.sec.properties}

There are two properties of the Poisson distribution
that make it extremely useful in ecology. First
is the property of {\it compound additivity}. If $y_1$ and $y_2$ are
Poisson random variables with means $\lambda_1$ and $\lambda_2$,
then their sum $N=y_1+y_2$ is Poisson with mean $\lambda_1+\lambda_2$. Thus,
if the observations can be viewed as an aggregate of counts over some
finer unit of measurement, then the mean aggregates in a corresponding
manner.
Secondly, the Poisson distribution has a direct relationship to the multinomial.
If $y_1$ and $y_2$ are $iid$ Poisson then,
conditional on their sum $N = y_1 + y_2$, their joint distribution is multinomial
 with sample size $N$ and cell probabilities
$\lambda_1/(\lambda_1+\lambda_2)$ and
$\lambda_2/(\lambda_1+\lambda_2)$.  As a result of this, most
multinomial models can be analyzed as a Poisson GLM and {\it vice versa}.

\subsection{Example: Breeding Bird Survey Data}

As an example we consider a classical situation in ecology where
counts of an organism are made at a collection of spatial
locations. In this particular example, we have mourning dove counts
made along North American Breeding Bird Survey (BBS) routes in
Pennsylvania, USA. A route consists of 50 stops separated by 0.5
mile. For the purposes here we are defining $y_i$ = route total count
and he sample location will be marked by the center point of the BBS
route.  The survey is run annually and the data set we have is
1966-1998. BBS data can be obtained online at \mbox{\tt http:\//\//www.pwrc.usgs.gov\//bbs\//}.
We will make use of the whole data set shortly but for now we're going
to focus on a specific year of counts -- 1990 -- for the sake of
building a simple model.
 For 1990 there were 77 active routes. We have the data stored
in a \mbox{\tt .csv} file\footnote{check this data format} where rows index the unique route, column 1 is the
route ID, columns 2-3 are the route coordinates (longitude/latitude),
column 4 is a habitat covariate ``forest cover'' (standardized, see
below) and the remaining columns are the yearly counts. Years for
which a route was not run are coded as ``\mbox{\tt NA}'' in the data matrix. We
imagine that this will be a typical format for many ecological
studies, perhaps with more columns representing covariates.  To read
in the data and display the first few elements of this matrix, do
this:
{\small
\begin{verbatim}
> a<-read.csv("pa-bbsdovedata-all.csv")
> data[1:2,1:6]
      X     lon    lat    habitat X66 X67
1 72002 -80.445 41.501 -0.3871372  NA  24
2 72003 -80.347 41.214 -1.0171629  NA  NA
\end{verbatim}
}

It is useful to display the spatial pattern in the observed counts. For that we use a
spatial dot plot - where we plot the coordinates of the observations
and mark the color of the plotting symbol based on the magnitude of
the count.  We have a special plotting function for that which is
called \mbox{\tt spatial.plot()} and it is available with the
supplemental {\bf R} package.
Actually, what we want to do here is plot the
log-count (+1 of course) which (Fig. \ref{glms.fig.padovecounts}) displays a notable pattern that could
be related to something. The {\bf R} commands for obtaining this figure are:
{\small
\begin{verbatim}
data<-read.csv("pa-bbsdovedata-all.csv")
y<-data[,29]  # pick out 1990
notna<-!is.na(y)
y<-y[notna]
spatial.plot(data[notna,2:3],y)
\end{verbatim}
}
 We can ponder the potential effects that
might lead to dove counts being high....corn fields, telephone wires,
barn roofs along with misidentification of pigeons, these could all
correlated reasonably well with the observed count of mourning doves.
Unfortunately we don't have any of that information.

\begin{figure}
\begin{center}
\includegraphics[height=2.75in]{Ch2/figs/PA1}
\end{center}
\caption{Needs a caption}
\label{glms.fig.padovecounts}
\end{figure}

We do have a measure of forest cover in the vicinity of each point
which is contained in the data set (variable ``habitat''). This was derived
from a larger GIS coverage of the state (provided in the data file
``\mbox{\tt pahabdata.csv}'') which can be plotted using the \mbox{\tt spatial.plot} function
using the following commands
{\small
\begin{verbatim}
> map('state',regions="penn",lwd=2)
> spatial.plot(pahabdata[,2:3],pahabdata[,"dfor"],cx=2)
> map('state',regions="penn",lwd=2,add=TRUE)
\end{verbatim}
}
where the result appears in Fig. \ref{glms.fig.paforest}.
We see a prominent pattern that indicates high forest coverage in the
central part of the state and low forest cover in the SE.  Inspecting
the previous figure of log-counts suggests a relationship between
counts and forest cover which is perhaps not surprising.
\begin{figure}
\begin{center}
\includegraphics[height=2.75in]{Ch2/figs/PA2}
\end{center}
\caption{Needs a caption}
\label{glms.fig.paforest}
\end{figure}

\subsection{Doing it in WinBUGS}

Here we demonstrate how to fit a Poisson GLM in {\bf WinBUGS} using the
covariate $x_{i} =$ forest cover. It is advisable that $x_i$ be
standardized in most cases as this will improve mixing of the Markov
chains. Recall that the data we have stored include a standardized
covariate (forest cover) and so we don't have to worry about that
here.  To read the BBS data into {\bf R} and get things set up for
{\bf WinBUGS}
we issue the following commands:
{\small
\begin{verbatim}
data<-read.csv("pa-bbsdovedata-all.csv")
y<-data[,29]                        # pick out 1990
notna<-!is.na(y)
y<-y[notna]                         # discard missing
habitat<-data[notna,4]              # get habitat data
library("R2WinBUGS")                # load R2WinBUGS
data <- list ( "y","M","habitat")   # bundle data for WinBUGS
\end{verbatim}
}
Now we write out the Poisson model specification in {\bf WinBUGS}
pseudo-code, provide initial values, identify parameters to be
monitored and then execute {\bf WinBUGS}:
{\small
\begin{verbatim}
cat("
model {
    for (i in 1:M){
      y[i]~dpois(lam[i])
      log(lam[i])<- beta0+beta1*habitat[i]
     }
 beta0~dunif(-5,5)
 beta1~dunif(-5,5)
}
",file="PoissonGLM.txt")

inits <- function()  list ( beta0=rnorm(1),beta1=rnorm(1))
parameters <- c("beta0","beta1")
out<-bugs (data, inits, parameters, "PoissonGLM.txt", n.thin=2,n.chains=2,
                n.burnin=2000,n.iter=6000,debug=TRUE,working.dir=getwd())
\end{verbatim}
}
{\bf Remarks:} (1) Note the close correspondence in how the model is
specified here compared with the normal regression model
previously. As an exercise you should discuss the specific differences
between the {\bf BUGS} model specifications for the normal and Poisson
models.
{\small
\begin{verbatim}
> print(out,digits=3)
Inference for Bugs model at
``PoissonGLM.txt'', fit using WinBUGS,
 2 chains, each with 4000 iterations (first 1000 discarded), n.thin = 2
 n.sims = 3000 iterations saved
             mean     sd     2.5%      25%      50%      75%    97.5%  Rhat n.eff
beta0       3.151  0.025    3.102    3.135    3.151    3.168    3.199 1.001  2300
beta1      -0.498  0.021   -0.539   -0.512   -0.498   -0.484   -0.457 1.001  3000
fit       869.930 19.856  835.500  855.700  868.600  881.900  913.602 1.002  1600
fitnew     76.709 12.519   54.098   68.107   76.215   84.510  102.602 1.001  3000
deviance 1116.605  2.014 1115.000 1115.000 1116.000 1117.000 1122.000 1.001  3000
\end{verbatim}
}

We might wonder whether this model provides an adequate fit to our
data.  To evaluate that, we used a Bayesian p-value analysis with fit
statistic based on the Freeman-Tukey residual by replacing the model
specification above with this:
{\small
\begin{verbatim}
cat("
model {
    for (i in 1:M){
      y[i]~dpois(lam[i])
      log(lam[i])<- beta0+beta1*habitat[i]
      d[i]<-  pow(pow(y[i],0.5)-pow(lam[i],0.5),2)   #

      ynew[i]~dpois(lam[i])
      dnew[i]<-pow( pow(ynew[i],0.5)-pow(lam[i],0.5),2)

     }
 fit<-sum(d[])
 fitnew<-sum(dnew[])
 beta0~dunif(-5,5)
 beta1~dunif(-5,5)
}
",file="PoissonGLM.txt")
\end{verbatim}
}
The Bayesian p-value is the proportion of times $fitnew > fit$ which,
for this data set, is 0, which was 1.0 in this case (calculation
omitted). This suggests that the basic Poisson model does not fit
well.

\subsection{ Constructing your own MCMC algorithm}

At this point it might be helpful to suffer through an example
building a custom MCMC algorithm. Here, we develop an MCMC algorithm
for
the Poisson regression model, using a Metropolis-within-Gibbs sampling framework. Building MCMC algorithms is covered in more detail in Chapt. \ref{chapt.mcmc} where you can also find step-by-step instructions for Metropolis-within-Gibbs samplers, should the following section move through all this stuff too quickly.  

We will assume that the two parameters have diffuse
normal priors, say $[\alpha] = \mbox{Norm}(0,100)$ and
$[\beta]=\mbox{Norm}(0,100)$ where each has {\it standard deviation}
100 (recall that {\bf WinBUGS} parameterizes the normal in terms of $1/\sigma^{2}$).
We need to assemble the relevant elements of the model which are these
two prior distributions and the
likelihood $[{\bf y}|\alpha,\beta] = \prod_{i} [y_i|\alpha \beta] $ which is,
mathematically, the product of the Poisson pmf evaluated at each $y_i$,
given particular values of $\alpha$ and $\beta$.
Next, we need to identify the full conditionals
$[\alpha|\beta, {\bf y}]$ and $[\beta|\alpha,{\bf y}]$.  We use the all-purpose
rule for constructing full conditionals
(section \ref{glms.sec.rules})
 to discover that:
\[
 [\alpha|\beta,{\bf y}] \propto \left\{ \prod_{i} [y_{i}|\alpha,\beta]\right\}[\alpha]
\]
and
\[
 [\beta|\alpha,{\bf y}] \propto \left\{ \prod_{i}
   [y_{i}|\alpha,\beta]\right\} [\beta]
\]
Remember, we could replace the ``$\propto$'' with ``$=$'' if we
put $[y|\beta]$ or $[y|\alpha]$ in the denominator. But, in general,
$[y|\alpha]$ or $[y|\beta]$ will be quite a pain to compute and, more
importantly, it is a constant as far as the operative parameters
($\alpha$ or $\beta$,
respectively) are concerned. Therefore,
the MH acceptance probability will be the ratio of the
ful-conditional evaluated at a candidate draw to that evaluated at the
current draw, and so the denominator required to change $\propto$ to $=$
winds up canceling from the MH acceptance probability.
Here we will
use the so-called random walk candidate generator, which is a Normal proposal distribution, so that, for example,
 $\alpha^{*} \sim \mbox{Normal}(\alpha^{t},\delta)$ where $\delta$ is
 the standard-deviation of the proposal distribution, which is just a
 tuning parameter that is set by the user and adjusted to achieve efficient mixing of chains (see Section XX in Chapt. \ref{chapt.mcmc}) \footnote{
It would help
lots of people out to see a non-symmetric proposal distribution, and
the extra step needed to account for it. RS: We can include this in the MCMC chapter
}.
We remark also that calculations are often done on the log-scale to
preserve numerical integrity of things when quantities evaluate to
small or large numbers, so keep in mind, for example,
$a*b = exp(log(a) + log(b))$.
 The ``Metropolis within
Gibbs'' algorithm for a Poisson regression turns out to be  remarkably simple:
{\small
\begin{verbatim}
set.seed(2013)

out<-matrix(NA,nrow=1000,ncol=2)   # matrix to store the output
alpha<- -1                         # starting values
beta <- -.8

# begin the MCMC loop ; do 1000 iterations
for(i in 1:1000){

# update the alpha parameter
lambda<- exp(alpha+beta*habitat)
lik.curr<- sum(log(dpois(y,lambda)))
prior.curr<- log(dnorm(alpha,0,100))
alpha.cand<-rnorm(1,alpha,.05)         # generate candidate
lambda.cand<- exp(alpha.cand + beta*habitat)
lik.cand<- sum(log(dpois(y,lambda.cand)))
prior.cand<- log(dnorm(alpha.cand,0,100))
mhratio<- exp(lik.cand +prior.cand - lik.curr-prior.curr)
if(runif(1)< mhratio)
     alpha<-alpha.cand

# update the beta parameter
lik.curr<- sum(log(dpois(y,exp(alpha+beta*habitat))))
prior.curr<- log(dnorm(beta,0,100))
beta.cand<-rnorm(1,beta,.25)
lambda.cand<- exp(alpha+beta.cand*habitat)
lik.cand<- sum(log(dpois(y,lambda.cand)))
prior.cand<- log(dnorm(beta.cand,0,100))
mhratio<- exp(lik.cand + prior.cand - lik.curr - prior.curr)
if(runif(1)< mhratio)
     beta<-beta.cand

out[i,]<-c(alpha,beta)             # save the current values
}


plot(out[,1],ylim=c(-1.5,3.3),type="l",lwd=2,ylab="parameter value",
     xlab="MCMC iteration")
lines(out[,2],lwd=2,col="red")
\end{verbatim}
}
{\bf XXX Andy I removed the bad tuning example and the respective exercise here and added it in Ch7 XXXX}
The first 300 iterations of the MCMC history of each parameter
are shown in Fig. \ref{glms.fig.poissonmcmc2}. These chains are
not very appealing but a couple of things are evident: 
We see
that the burn-in takes about 250 iterations and that after that chains seem to mix reasonably well, although this is not so clear given the scale of the y-axis.
We generated 10,000 posterior samples,
discarding the first 500 as burn-in, and the result is shown in
Fig. \ref{glms.fig.grassy}, this time seperate panels for each
parameter.
The ``grassy''
look of the MCMC history is diagnostic of Markov chains that are
well-mixing and we would generally be very satisfied with results that
look like this.

\begin{figure}
\begin{center}
\includegraphics[height=3in,width=4in]{Ch2/figs/poissonmcmc2}
\end{center}
\caption{Same as previous fig but with $\delta = 0.05$.}
\label{glms.fig.poissonmcmc2}
\end{figure}

\begin{figure}
\begin{center}
\includegraphics[height=4in,width=5in]{Ch2/figs/poissonmcmc3}
\end{center}
\caption{nice grassy mcmc output, longer run of previous with $\delta
  = 0.05$.}
\label{glms.fig.grassy}
\end{figure}

{\bf Remarks:} (1) We used a specific set of starting values for these
simulations. It should be clear that starting values closer to the
mass of the posterior distribution might cause burn-in to occur
faster. As an exercise, evaluate that.  
(2) For the flat normal prior distributions
here we could leave the prior contribution out of the full conditional
evaluation since it is locally constant, i.e., constant in the vicinity of
the posterior mass, and thus has no practical effect. Removing the
prior contribution from the MH acceptance probability is equivalent to
saying that the parameters have an improper uniform prior, i.e.,
$\alpha \sim \mbox{const}$, which is commonly used for mean parameters
in practice.
Note also that we have
used a different prior than in our {\bf WinBUGS} model specification
given previously. As an
exercise, evaluate whether this seems to affect the result.

\section{Poisson GLM with Random Effects}

What we will be doing in most of this book is dealing with random
effects in GLM-like models - similar to what
are usually referred to as generalized
linear mixed models (GLMMs). We provide a brief introduction by way of
example, extending our Poisson regression model to include a random effect.

ANDY STOPPED HERE

{\bf The Log-Normal mixture:} The classical situation involves a GLM
with a normally distributed random effect that is additive on the
linear predictor. For the Poisson case, we have:
\[
 	log(\lambda_{i}) = \alpha  + \beta x_{i} + \eta_{i}
\]
where $\eta_{i} \sim \mbox{Normal}(0,\sigma^{2})$.  A natural
alternative is to have multiplicative gamma-distributed noise,
$exp(\eta_{i}) \sim  \mbox{Gamma}(a,b)$ which would correspond to a
negative binomial kind of over-dispersion, implying a different
mean/variance relationship to the log-normal mixture (the interested
reader should work that out).   Choosing between such possibilities is
not a topic we will get into here because it doesn't seem possible to
provide general guidance on it.
For this model we carried-out a goodness-of-fit evaluation using the
Bayesian p-value based on a Pearson residual statistic. See also
\citep[][ch. 18]{kery:2010}
for an example involving a binomial mixed model\footnote{Kery has
  noticed that such tests probably have 0 power. Should use the
  marginal frequency of the data}.
 Anyhow, it is really amazingly simple
to express this model in {\bf WinBUGS} and have {\bf WinBUGS}  draw samples from the posterior distribution using the following code for the BBS dove counts:
{\small
\begin{verbatim}
data<-read.csv("pa-bbsdovedata-all.csv")
locs<-data[,2:3]
habitat<-data[,4]
y<-data[,29]     # grab year 1990
M<-length(y)

set.seed(2013)

cat("
model {
  for (i in 1:M){
     y[i]~dpois(lam[i])
     log(lam[i])<- alpha+ beta*habitat[i] + eta[i]
     frog[i]<-beta*habitat[i] + eta[i]
     eta[i] ~ dnorm(0,tau)
     d[i]<-  pow(pow(y[i],0.5)-pow(lam[i],0.5),2)

     ynew[i]~dpois(lam[i])
     dnew[i]<- pow(pow(ynew[i],0.5)-pow(lam[i],0.5),2)
   }
 fit<-sum(d[])
 fitnew<-sum(dnew[])

 alpha~dunif(-5,5)
 beta~dunif(-5,5)
 sigma~dunif(0,10)
 tau<-1/(sigma*sigma)
}

",file="model.txt")
data <- list ( "y","M","habitat")
inits <- function()
  list ( alpha=rnorm(1),beta=rnorm(1),sigma=runif(1,0,4))
parameters <- c("alpha","beta","sigma","tau","fit","fitnew")
library("R2WinBUGS")

out<-bugs (data, inits, parameters, "model.txt", n.thin=2,n.chains=2,
 n.burnin=1000,n.iter=5000,debug=TRUE)
\end{verbatim}
}
This produces the following posterior summary statistics:
{\small
\begin{verbatim}
> print(out,digits=2)
Inference for Bugs model at "model.txt", fit using WinBUGS,
 2 chains, each with 5000 iterations (first 1000 discarded), n.thin = 2
 n.sims = 4000 iterations saved
           mean    sd   2.5%    25%    50%    75%  97.5% Rhat n.eff
alpha      2.98  0.08   2.82   2.93   2.98   3.03   3.12 1.00  1400
beta      -0.53  0.07  -0.68  -0.58  -0.53  -0.49  -0.38 1.01   350
sigma      0.60  0.06   0.49   0.56   0.59   0.64   0.73 1.00  2000
tau        2.88  0.57   1.88   2.47   2.86   3.24   4.12 1.00  2000
fit       26.58  3.72  19.87  23.96  26.37  29.01  34.46 1.00  4000
fitnew    26.83  3.90  19.60  24.12  26.68  29.36  35.04 1.00  4000
deviance 445.94 12.18 424.00 437.40 445.20 453.90 471.50 1.00  4000

[... some output deleted ...]
\end{verbatim}
}
The Bayesian p-value for this model is
\begin{verbatim}
> mean(out$sims.list$fit>out$sims.list$fitnew)
[1] 0.4815
\end{verbatim}
indicating a pretty good fit. Given the site-level random effect, it
would be surprising for this model to not fit! One thing we notice is
that the posterior standard deviations of the regression parameters
are much higher, a result of the excess variation. Wwe would also
notice much less precise predictions of hypothetical new
observations.


ANDY STOPPED HERE.




\section{Binomial GLMs}

Another extremely important class of models in ecology are
binomial models. We use binomial models for count data whenever the
observations are counts or frequencies and it is natural to condition
on a ``sample size'', say $K$, the maximum frequency possible in a sample.
 The random variable, $y \le K$, is then the
frequency of occurrences out of $K$ ``trials''. The parameter of the binomial
models is $p$, often called ``success probability'' which is related
to the expected value of $y$ by $E(y) = pK$. Usually we are interested
in modeling covariates that affect the parameter $p$, and such models
are called binomial GLMs , binomial
regression models or logistic regression, although logistic regression
 really only applies when the logistic link is used to model
the relationship between $p$ and covariates (see below).

One of the most typical binomial GLMs occurs when the sample size
equals 1 and the outcome, $y$, is ``presence'' ($y=1$) or ``absence''
($y=0$) of a species. This is a classical ``species distribution''
modeling situation. A special situation occurs when presence/absence
is observed with error \citep{mackenzie_etal:2002,tyre_etal:2003}.
In that case, $K>1$ samples
are usually needed for effective estimation of model parameters.

 In standard binomial regression problems the sample size
is fixed by design but interesting models also arise when the sample
size is itself a random variable. These are the $N$-mixture models
\citep{royle:2004, kery_etal:2005, royle_dorazio:2008, kery:2010}
and related models (in this case, $N$ being the sample size,
which we labeled $K$ above)\footnote{Some of the jargon is actually a little
bit confusing here
because the binomial index is customarily referred to as ``sample size''
but in the context of $N$-mixture models $N$ is actually the
``population size''}.
Another
situation in which the binomial sample size is ``fixed'' is closed
population capture-recapture models in which a population of
individuals is sampled $K$ times.  The number of times each individual
is encountered is a binomial outcome with parameter - encounter
probability -- $p$, based on a sample of size $K$.  In addition, the
total number of unique individuals observed, $n$, is also a binomial
random variable based onpopulation size $N$.  We consider such
models in the chapter \ref{chapt.closed}.


\subsection{Binomial regression}

In binomial models, covariates are modeled on a suitable
transformation (the link function) of the binomial success
probability, $p$.  Let $x_{i}$ denote some measured covariate for
sample unit $i$ and let $p_{i}$ be the success probability for unit $i$.
The standard choice is the ``logit'' link function which is:
\[
log(p_i/(1-p_i)) = \alpha + \beta*x_{i}.
\]
The inverse-logit (or ``expit'') is
\[
p_{i} = \mbox{expit}(\alpha + \beta*x_{i}) =
 \frac{ \exp(\alpha + \beta*x_{i})}
      {1 + \exp(\alpha + \beta*x_{i} ) }
\]
There are many other possible link functions. However, ecologists seem
to adopt the logit link function without question in most
applications\footnote{a notable exception is distance sampling, which
  is all about choosing among link functions}.  We sometimes use the
``complementary log-log'' (= ``cloglog'') link function in ecological
applications because it arises naturally in many situations
\citep[][p. 150]{royle_dorazio:2008}. For example, consider the
``probability of observing a count greater than 0'' under a Poisson
model: $\Pr(y>0) = 1-exp(- \lambda)$. In that case,
\[
cloglog(p) =log(- log(1-p)) = log(\lambda)
\]
So that if you have covariates in your linear predictor for $E(y)$
under a Poisson model then they are linear on the complementary
log-log link of $p$.
In models of species occurrence it seems natural to view occupancy as
being derived from local abundance $N$
\citep{royle_nichols:2003,royle_dorazio:2006,dorazio:2007}.
Therefore,
models of local abundance in which $N \sim \mbox{Poisson}(A \lambda)$
for a habitat patch of area $A$ implies a model for occupancy $\psi$
of the form
\[
 cloglog(\psi) = log(A) + log(\lambda).
\]
We will use the cloglog link in some analyses of
SCR models in chapter \ref{chapt.scr0} and elsewhere.


\subsection{ Example: Waterfowl Banding Data}

It would be easy to consider a standard ``distribution modeling''
application where $K=1$ and the outcome is occurrence ($y=1$) or not
($y=0$) of some species. Such examples abound in books (e.g.,
\citet[][ch. 3]{royle_dorazio:2008}; \citet[][ch. 21]{kery:2010};
\citet[][ch. 13]{kery_schaub:2011}) and in the literature.
Instead, we will
consider an example involving band returns of waterfowl which were
analyzed by \citet{royle_dubovsky:2001}\footnote{I hate this example.
  Anyone got a better one thats not distribution modeling?}.

For these data, $y_i$ is the number of waterfowl bands recovered out
of $B_i$ birds banded at some location ${\bf s}_{i}$. In this case $B_{i}$ is
fixed. Thinking about recovery rate as being proportional to harvest
rate, we use these data to explore geographic gradients in recovery rate
resulting from variability in harvest pressure experienced by
populations depending on their migration ecology. As such, we fit a
basic binomial GLM with a linear response to geographic coordinates
(including an interaction term). The data are provided with the {\bf
  R} package \mbox{\tt scrbook}. Here we
 provide the part of the script for creating the model and fitting the
 model in
{\bf WinBUGS} using the \mbox{\tt bugs} function.
There are few structural differences between this model and the
Poisson GLM fitted previously. The main things are due to the data
structure (we have a matrix here instead of a vector) and otherwise we
change the main distributional assumption to binomial (specified with
\mbox{\tt dbin}) and then use the \mbox{\tt logit} function to relate
the parameter $p_{it}$ to the covariates.  Here is the script:

{\small
\begin{verbatim}
load("mallarddata")  # not sure how this will look

sink("model.txt")
cat("
model {
 for(t in 1:5){
    for (i in 1:nobs){
       y[i,t] ~ dbin(p[i,t], B[i,t])
       logit(p[i,t]) <- alpha0[t] + alpha1*X[i,1] + alpha2*X[i,2] + alpha3*X[i,1]*X[i,2]
     }
}
	alpha1~dnorm(0,.001)
	alpha2~dnorm(0,.001)
	alpha3~dnorm(0,.001)
	for(t in 1:5){
 	alpha0[t] ~ dnorm(0,.001)
 }
}
",fill=TRUE)
sink()

data  <- list(B=mallard.bandings, y=mallard.recoveries,
             nobs=nrow(banding.locs),X=banding.locs)
inits <- function(){
      list(alpha0=rnorm(5),alpha1=0,alpha2=0,alpha3=0) }
parms <- list('alpha0','alpha1','alpha2','alpha3')
out   <- bugs(data,inits, parms,"model.txt",n.chains=3,
 	n.iter=2000,n.burnin=1000, n.thin=2,debug=TRUE)
\end{verbatim}
}


Posterior summaries of model parameters are as follows:
{\small
\begin{verbatim}
> print(out,digits=3)
Inference for Bugs model at "model.txt", fit using WinBUGS,
 3 chains, each with 2000 iterations (first 1000 discarded), n.thin = 2
 n.sims = 1500 iterations saved
              mean    sd     2.5%      25%      50%      75%    97.5%  Rhat n.eff
alpha0[1]   -2.346 0.036   -2.417   -2.370   -2.346   -2.323   -2.277 1.001  1500
alpha0[2]   -2.356 0.032   -2.420   -2.379   -2.356   -2.335   -2.292 1.001  1500
alpha0[3]   -2.220 0.035   -2.291   -2.244   -2.219   -2.197   -2.153 1.001  1500
alpha0[4]   -2.144 0.039   -2.225   -2.169   -2.143   -2.116   -2.068 1.000  1500
alpha0[5]   -1.925 0.034   -1.990   -1.949   -1.924   -1.901   -1.856 1.004   570
alpha1      -0.023 0.003   -0.028   -0.025   -0.023   -0.022   -0.018 1.001  1500
alpha2       0.020 0.006    0.009    0.016    0.020    0.024    0.031 1.001  1500
alpha3       0.000 0.001   -0.002   -0.001    0.000    0.000    0.002 1.001  1500
deviance  1716.001 4.091 1710.000 1713.000 1715.000 1718.000 1726.000 1.001  1500

[... some output deleted ...]
\end{verbatim}
}

The basic result suggests a negative east-west gradient and a positive
south to north gradient but no interaction. A map of the response
surface is shown in Fig. \ref{glms.fig.bandrecovery}.
 We did an additional MCMC run where we saved the binomial
parameter $p$ and computed the Bayesian p-value (double use of ``p''
here is confusing, but I guess that happens sometimes!)
using a fit statistic based on the Freeman-Tukey
statistic (see Section XXX above). The result indicates that the
linear response surface model does not provide an adequate fit of the
data. The reader should contemplate whether this invalidates the basic
interpretation of the result.


\begin{figure}
\begin{center}
\includegraphics[height=2.75in]{Ch2/figs/responsesurface}
\end{center}
\caption{Predicted recovery rate of bands.}
\label{glms.fig.bandrecovery}
\end{figure}

\section{ Summary and Outlook}

GLMs and GLMMs are the most useful statistical methods in all of
ecology. The principles and procedures underlying these methods are
relevant to nearly all modeling and analysis problems in every branch
of ecology. Moreover, understanding how to analyze these models is
crucial in a huge number of diverse problems. If you understand and
can conduct classical likelihood and Bayesian analysis of Poisson and
binomial GLM(M)s, then you will be successful analyzing and
understanding more complex classes of models that arise. We will see
shortly that spatial capture-recapture models are a type of GLMM
and thus having a basic
understanding of the conceptual origins and formulation of GLM(M)s and
their analysis is extremely useful.

We note that GLM(M)s are routinely
analyzed by likelihood methods but we have focused on Bayesian
analysis here in order to develop the tools that are less familiar to
most ecologists.  In particular, Bayesian analysis of models with random
effects is relatively straightforward because the models
are easy to analyze conditional on the random effect, using methods of
MCMC.  Thus, we will often analyze SCR models in later chapters by
MCMC, explicitly adopting a Bayesian inference framework.
In that regard, the various {\bf BUGS} engines ({\bf WinBUGS}, {\bf
  OpenBUGS}, {\bf JAGS}) are enormously useful because they
provide an accessible platform for
carrying out  analyses by MCMC by just
describing the model, and not having to worry about how to actually
build MCMC algorithms.  That said, the {\bf BUGS} language is more important
than just to the extent that it enables one to do MCMC - it is useful
as a modeling tool because it fosters understanding, in the sense that
it forces you to become intimate with your model. You have to write
down all of the probability assumptions, the relationships between
observations and latent variables and parameters. This is really a
great learning paradigm that you can grow with.

While we have emphasized Bayesian analysis in this chapter, and make
primary use of it through the book, we
we will provide an introduction to likelihood analysis in chapter
\ref{chapt.mle} and use those  methods also from time to time.
 Before getting to that, however, it will be useful to
talk about more basic, conventional closed population
capture-recapture models and these are the topic of the next chapter.


\chapter{
 Closed Population Models
}
\markboth{Chapter 3}{}
\label{chapt.closed}

\vspace{.3in}
%%Andy, I really like connecting a new chapter to the previous ones with a few words, so I added this half sentence
Having covered the basics of hierarchical models and their implementation, in this chapter we will consider ordinary capture-recapture (CR)
models for estimating population size in closed populations. We will
see that such models are closely related to binomial (or logistic)
regression type models. In fact, when $N$ is known, they are precisely
such models.  We consider some important extensions of ordinary closed
population models that accommodate various types of ``individual
effects'' --- either in the form of explicit covariates (sex, age,
body mass) or unstructured ``heterogeneity'' in the form of an
individual random effect. In general, these models are variations of
generalized linear or generalized linear mixed models (GLMMs).
Because of the paramount importance of this concept, we focus mainly
on fairly simple models in which the observations are individual
encounter frequencies, $y_{i}$ = the number of encounters of
individual $i$ out of $K$ replicate samples of the population which,
for the models we consider here, is the outcome of a binomial random
variable.  Along the way, we consider the spatial context of
capture-recapture data and models and demonstrate that density cannot
be formally estimated when spatial information is ignored. We also
review some of the informal methods of estimating density using CR
methods, and consider some of their limitations.  We will be exposed
to our first primitive spatial capture-recapture models which arise as
relatively minor variations of so-called ``individual covariate
models'' (of the \citet{huggins:1989} and \citet{alho:1990}
variety). In a sense, the point of this chapter is to establish that
linkage XX between non-spatial and spatial capture-recapture models XXX in a direct and concise manner beginning with the basic
``Model $M_0$'' and extensions of that model to include individual
heterogeneity and also individual covariates. A special type of
individual covariate models is distance sampling, which could be
thought of as the most primitive spatial capture-recapture model.  In
later chapters we further develop and extend ideas introduced in this
chapter.

We emphasize Bayesian analysis of capture-recapture models and we
accomplish this using a method related to classical ``data
augmentation'' from the statistics literature XXX SOMETHING WRONG WITH BRACKETS IN REF XXX
\citet[e.g.,][]{tanner_wong:1987}).  This is a general concept in
statistics but, in the context of capture-recapture models where $N$
is unknown, it has a consistent implementation across classes of
capture-recapture models and one that is really convenient from the
standpoint of doing MCMC \citep{royle_etal:2007}. We use data
augmentation throughout this book and thus emphasize its conceptual
and technical origins and demonstrate applications to closed
population models.  We refer the reader to
\citet[][ch. 6]{kery_schaub:2011} for an accessible and complimentary
development of ordinary closed population models.


\section{The Simplest Closed Population Model: Model $M_0$}

To start looking at the simples capture-recapture model, let's suppose that there exists a population of $N$ individuals which we
subject to repeated sampling, say over $K$ nights, where individuals
are captured, marked, and subsequently recaptured.  We suppose that
individual encounter histories are obtained, and these are of the form
of a sequence of 0's and 1's indicating capture $(y=1)$ or not $(y=0)$
during any sampling occasion (``sample'').  As an example, suppose
$K=5$ sampling occasions, then an individual captured during sample 2
and 3 but not otherwise would have an encounter history of the form
${\bf y}=(0,1,1,0,0)$. Thus, the observation ${\bf y}_{i}$ for each
individual $(i)$ is a vector having elements denoted by $y_{ik}$ for
$k=1,2,..,K$. Usually this is organized as a row of a matrix with
elements $y_{ik}$, see Table \ref{tab.3.1}.  Except where noted
explicitly, we suppose that observations are independent within
individuals and among individuals.  Formally, this allows us to say
that $y_{ik}$ are $iid$ Bernoulli random variables and we may write $y_{ik}
\sim \mbox{Bern}(p)$.  Consequently, for this very simple model in
which $p$ is in fact constant, then we can declare that the individual
encounter frequencies (total captures), $y_{i} = \sum_{k} y_{ik}$,
have a binomial distribution based on a sample of size $K$. That is
\[
y_{i}  = \sum_{k} y_{ik} \sim \mbox{Bin}(p,K)
\]
for every individual in the population. This is a remarkably simple
model that forms the cornerstone of almost all of classical
capture-recapture models, including most spatial capture-recapture
models discussed throughout this book.  

Evidently, the basic
capture-recapture model structure is precisely a simplistic version of
a logistic-regression model with only an intercept term
($\mbox{logit}(p) = \mbox{constant}$).  To say that all
capture-recapture models are just logistic regressions is only
slightly inaccurate. In fact, we are proceeding here ``conditional on
$N$'', i.e., as if we knew $N$. In practice we don't, of course, and
that is kind of the point of capture-recapture models as estimating
$N$ is the central objective. But, by proceeding conditional on $N$,
we can specify a simple model and then deal with the fact that $N$ is
unknown using standard methods that you are already familiar with
(i.e., GLMs - see chapter 2).
\begin{table}
\centering
\caption{a capture-recapture data set with $n=6$ observed individuals
and $K=5$ samples.}
\begin{tabular}{r|ccccc|c}
&  \multicolumn{5}{c}{Sample occasion} &  \\ \hline
 indiv $i$ &  1 & 2 & 3 & 4 & 5 & $y_{i}$ \\ \hline
  1 &     1 & 0 & 0 & 1 & 0  & 2   \\
  2 &     0 & 1 & 0 & 0 & 1  & 2   \\
  3 &     1 & 0 & 0 & 1 & 0  & 2   \\
  4 &     1 & 0 & 1 & 0 & 1  & 3   \\
  5 &     0 & 1 & 0 & 0 & 0  & 1   \\
  $n=6$ & 1 & 0 & 0 & 0 & 0  & 1   \\ \hline
\end{tabular}
\label{tab.3.1}
\end{table}

Assuming individuals of the population are observed independently, the
joint probability distribution of the observations is the product of
$N$ binomials
\begin{eqnarray*}
  \Pr(y_1, \ldots, y_N | p) &=& \prod_{i=1}^N  \mathrm{Bin}(y_i | K, p) \\
   &=& \prod_{k=0}^K  \pi(k)^{n_k}
\end{eqnarray*}
where $\pi(k) = \mathrm{Bin}(k | K,p)$ and where $n_k = \sum_{i=1}^N
I(y_i = k)$ denotes the number of individuals captured $k$ times in
$K$ surveys. We emphasize that this is conditional on $N$, in which
case we get to observe the $y=0$ observations and the resulting data
are just $iid$ binomial counts. Because this is a binomial regression
model of the variety described in Chapt. \ref{glms}, fitting this model using
a {\bf BUGS} engine poses no difficulty.

The essential problem in capture-recapture, however, is that $N$ is
not known because the number of uncaptured/missing individuals (i.e.,
those in the zero cell that occur with probability $\pi(0)$) is
unknown.  Consequently, the observed capture frequencies $n_k$ are no
longer independent. Instead, their joint distribution is multinomial
(e.g., see \citet[][p. xyz]{illian_etal:2008}):
\begin{equation}
    n_1, n_2, \ldots, n_K \sim \mathrm{Multin}(N, \pi(1), \pi(2), \ldots, \pi(K))
\label{closed.eq.multinomial4m0}
\end{equation}
Note that in our notation the number of uncaptured/missing individuals is
denoted by $n_0 = N - n$, where $n = \sum_{k=1}^K n_k$ denotes the total
number of distinct individuals seen in the $K$ samples.
XXX ANDY; MAYBE IT MIGHT BE WORTH MENTIONING WHY THE n0 DOESNT SHOW UP IN THE MULTINOMIAL XXXXX

To fit the model in which $N$ is {\it unknown}, we can regard $N$ as a
parameter and maximize the multinomial likelihood directly.  While
direct likelihood analysis of the multinomial model is
straightforward, that does not prove to be too useful in practice
because we seldom are concerned with models for the aggregated
encounter history frequencies, XXX which entail that capture probabilities are the same for all individuals XXX. In many instances, including for
spatial capture-recapture (SCR) models, we require a formulation of
the model that can accommodate individual level covariates XXX to account for differences in detection among individuals XXX which we
address subsequently in this chapter.


\subsection{The Spatial Context of Capture-Recapture}

XXX I WOULD CHANGE THE SECTION HEADING TO SOMETHING LIKE 'POPULATION CLOSURE AND THE SPATIAL CONTEXT OF CAPTURE-RECAPTURE XXX
A common assumption made is that of population ``closure'' which is
really just a colloquial way of saying (in part) the Bernoulli
assumptions stated explicitly above. In the biological context,
closure means, strictly, no additions or subtractions from the
population during study. This is manifest by the statement that the
encounters are independent and identically distributed (iid) Bernoulli
trials.  In practice, closure is usually interpreted by the manner in
which potential violations of that assumption arise. In particular,
two important elements of the closure assumption are ``demographic''
and ``geographic'' closure. If an individual dies then subsequent
values of $y_{ik}$ are clearly no longer Bernoulli trials with the
same parameter $p$; XXX since the probability of capturing that individual becomes 0 XXX. If there is no mortality or recruitment in the
population, then we say that demographic closure is
satisfied. Similarly, animals may emigrate or immigrate. If they do
not, then geographic closure is satisfied. Sometimes a distinction is
made between temporary and permanent emigration or immigration. That
is a relevant distinction in spatial capture-recapture models, because
SCR models explicitly accommodate ``temporary emigration'' of a
certain type, due to individuals moving about their home range. XXX In contrast, ordinary capture-recapture models cannot explicitly deal with the fact that, unless we're sampling a fenced enclosure or an island, individuals are bound to move off the trapping grid. XXX The
demographic closure assumption can also be relaxed using SCR models,
but we will save that discussion for Chapt. \ref{chapt.scr0}.
XXXX I FEEL LIKE THIS SECTION STILL NEEDS A SENTENCE THAT MAKES THE POINT - SPATIAL CONTEXT; POP CLOSURE AND SCR; BUT I AM HAVING TROUBLE PUTTING THAT INTO A FEW WORDS RIGHT NOW XXXX

\subsection{Conditional likelihood}

We saw that a basic closed population model is a simple logistic
regression model if $N$ is known and, when $N$ is unknown, the model
is multinomial with index or sample size parameter $N$. This
multinomial model, being conditional on $N$, is sometimes referred to
as the ``joint likelihood'' the ``full likelihood'' or the
``unconditional likelihood'' (or model in place of likelihood). This
formulation differs from the so-called ``conditional likelihood''
approach in which the likelihood of the observed encounter histories
is devised conditional on the event that an individual is captured at
least once.  To construct this likelihood, we have to recognize that
individuals appear or not in the sample based on the value of the
random variable $y_{i}$, that is, we capture them if and only if
$y_{i}>0$.  The observation model is therefore based on $\Pr(y|y>0)$.
For the simple case of Model $M_0$, the resulting conditional
distribution is a ``zero truncated'' binomial distribution which
accounts for the fact that we cannot observe the value $y=0$ in the
data set \citep[see][sec. 5.1]{royle_dorazio:2008}.  Both the
conditional and unconditional models are legitimate modes of analysis
in all capture-recapture types of studies, and they provide equally
valid descriptions of the data and for many practical purposes provide
equivalent inferences, at least in large sample sizes
\citep{sanathanan:1972}.

In this book we emphasize Bayesian analysis of capture-recapture
models using data augmentation (discussed subsequently), which
produces yet a third distinct formulation of capture recapture-models
based on the zero-{\it inflated} binomial distribution that we
describe in the next section.  Thus, there are 3 distinct formulations
of the model -- or modes of analysis -- for analyzing all
capture-recapture models based on the (1) binomial model for the joint
or unconditional specification; (2) zero-truncated binomial that
arises ``conditional on $n$''; and (3) the zero-inflated binomial that
arises under data augmentation.  Each formulation has a distinct
complement of model parameters (shown in Table \ref{tab.3.modes} for
Model $M_0$).


\begin{table}
\centering
\caption{Modes of analysis of capture-recapture models. Closed
  population models can be analyzed using the joint or ``full
  likelihood'' which contains $N$ as an explicit parameter, the
  conditional likeilhood which does not involve $N$, or by data
  augmentation which replaces $N$ with $\psi$. Each approach yields a
  distinct likelihood.}
\begin{tabular}{ccc}
Mode of analysis & parameters in model & statistical model \\ \hline
Joint likelihood                &	$p$, $N$	&	multinomial with index $N$\\
Conditional likelihood 		&	$p$	&	zero-truncated binomial \\
Data augmentation		&	$p$, $\psi$	&	zero-inflated binomial\\
\end{tabular}
\label{tab.3.modes}
\end{table}



\section{ Data Augmentation }
\label{closed.sec.da}

We consider a method of analyzing closed population models using data
augmentation (DA) which is useful for Bayesian analysis and, in
particular, analysis of models using the various BUGS engines and
other software.  Data augmentation is a general statistical concept
that is widely used in statistics in many different settings. The
classical reference is \citet{tanner_wong:1987} but see also
\citet{liu_wu:1999}.  Data augmentation can be adapted to provide a
very generic framework for Bayesian analysis of capture-recapture
models with unknown $N$. This idea was introduced for closed
populations by \citet{royle_etal:2007}, and has subsequently been
applied to a number of different contexts including individual
covariate models \citep{royle:2009}, open population models
\citep{royle_dorazio:2008,royle_dorazio:2010, gardner_etal:2010ecol},
spatial capture-recapture models \citep{royle_young:2008,
  royle_etal:2010, gardner_etal:2009}, and many
others. \citet[][Chapt. 6]{kery_schaub:2011} provides a good introduction to data
augmentation in the context of closed population models. 


Conceptually, data augmentation is a reparameterization of the
``complete data'' model -- that which is conditional on $N$. The
reparameterization is achieved by embedding this data set into a
larger data set having $M> N$ ``rows'' (individuals) and reexpressing
the model conditional on $M$ instead of $N$. XXX The great thing about data augmentation is that we do not need to know $N$ for this reparameterization. XXX Although this has a whiff of
arbitrariness or even ad hockery to it in the choice of $M$, 
it is always possible, in practice, to choose $M$ pretty easily for
a given problem and context and results will be insensitive to choice
of $M$\footnote{Unless the data set is sufficiently small that parameters are
weakly
identified}.
Then, under data augmentation, analysis
 is focused on the ``augmented data set.'' That is, we analyze the bigger
 data set - the one having $M$ rows - with an appropriate model that
 accounts for the augmentation. Inference is focused directly on
 estimating the proportion $\psi = E[N]/M$, instead of directly on $N$,
 where $\psi$ is the ``data augmentation parameter.''


\subsection{DA links occupancy models and closed population models}

%We provide a heuristic description of data augmentation based on the
XXX There is a XXX close correspondence between so-called ``occupancy'' models and closed
population models following \citet[][sec. 5.6]{royle_dorazio:2008}.

In occupancy models \citep{mackenzie_etal:2002, tyre_etal:2003} the
sampling situation is that $M$ sites, or patches, are sampled multiple
times to assess whether a species occurs at each site.  This yields
encounter data such as that illustrated in the left panel of Table
\ref{closed.tab.occ}. The important problem is that a species may occur at
a site, but go undetected, yielding the ``all-zero'' encounter
histories which are observed. However, some of the all-zeros may well
correspond to sites where the species in fact {\it does}
occur. Thus, while the zeros are observed, there are too many of them
and, in a sense, the inference problem is to allocate the zeros into
``structural'' (fixed) and ``sampling'' (or stochastic) zeros. More
formally, inference is focused on the parameter $\psi$, the
probability that a site is occupied.  In contrast, in classical closed
population studies, we observe a data set as in the middle panel of
Table \ref{closed.tab.occ} where {\it no} zeros are observed. The inference
problem is, essentially, to estimate how many sampling zeros there are
- or should be - in a ``complete'' data set. This objective
(how many sampling zeros?) is precisely the same for both types of
problems if an upper limit $M$ is specified for the closed population
model. The only distinction being that, in occupancy models, $M$ is
set by design (i.e., the number of sites to visit) whereas a natural
choice of $M$ for capture-recapture models may not be
obvious. However, by assuming a uniform prior for $N$ on the integers
$[0,M]$, this upper bound is induced \citep{royle_etal:2007}. Then,
one can analyze capture-recapture models by adding $M-n$ all-zero
encounter histories to the data set and regarding the augmented data
set, essentially, as a site-occupancy data set.

Thus, the heuristic motivation of data augmentation is to fix the size
of the data set by adding {\it too many} all-zero encounter histories
to create the data set shown in the right panel of Table
\ref{closed.tab.occ} - and then analyze the augmented data set using an
occupancy type model which includes both ``unoccupied sites'' as well
as ``occupied sites'' at which detections did not occur. We call these
$M-n$ all-zero histories ``potential individuals'' because they exist
to be recruited (in a non-biological sense) into the population, for
example during an analysis by MCMC.

To analyze the augmented data set, we recognize that it is a
zero-inflated version of the known-$N$ data set. That is, some of the
augmented all-zeros are sampling zeros (corresponding to actual
individuals that were missed) and some are ``structural'' zeros, which
do not correspond to individuals in the population. For a basic
closed-population model, the resulting likelihood under data
augmentation - that is, for the data set of size $M$ -- is a simple
zero-inflated binomial likelihood.  The zero-inflated binomial model
can be described ``hierarchically'', by introducing a set of binary
latent variables, $z_{1},z_{2},\ldots, z_{M}$, to indicate whether
each individual $i$ is ($z_i=1$) or is not ($z_i=0$) a member of the
population of $N$ individuals exposed to sampling. We assume that
$z_{i} \sim \mbox{Bern}(\psi)$ where $\psi$ is the probability that an
individual in the data set of size $M$ is a member of the sampled
population - in the sense that $1-\psi$ is the probability of
realizing a ``structural zero'' in the augmented data set.  The
zero-inflated binomial model which arises under data augmentation can
be formally expressed by the following set of assumptions:

\begin{eqnarray*}
 y_{i}|{z_{i}=1} & \sim  &\mbox{Bin}(K, p) \\
 y_{i}|{z_{i}=0} & \sim &  \delta(0)  \\
 z_{i} & \stackrel{iid}{\sim} & \mbox{Bern}(\psi) \\
 \psi & \sim & \mathrm{Unif}(0,1) \\
 p & \sim & \mathrm{Unif}(0,1)
\end{eqnarray*}
for $i=1, \ldots, M$, where $\delta(0)$ is a point mass at $y=0$.

Note that, under data augmentation, 
$N$ is no longer an explicit parameter of this
model. Instead, we estimate $\psi$ and functions of the latent
variables. In particular, under the assumptions of the zero-inflated
model, $z_{i} \stackrel{iid}{\sim} \mbox{Bern}(\psi)$; therefore, $N$
is a function of these latent variables:
 \[
 N = \sum_{i=1}^{M} z_{i}.
\]
Further, we note that the latent $z_i$ parameters can be removed from
the model by integration, in which case the joint probability of the
data is
\begin{equation}
  \Pr(y_1, \ldots, y_M | p, \psi) = \prod_{i=1}^M  \psi \mathrm{Bin}(y_i | K, p) +  I(y_i=0) (1-\psi)
\end{equation}
Which can be maximized directly to obtain the MLEs of the structural
parameters $\psi$ and $p$ or those of other more complex models
\citep[e.g., see][]{royle:2006}. We could estimate these parameters
and then use them to obtain an estimator of $N$ using the so-called
``Best unbiased predictor'' \citep[see][]{royle_dorazio:2011}.

\begin{table}
\centering
\caption{Hypothetical occupancy data set (left), capture-recapture data
 in standard form (center), and capture-recapture data augmented with
 all-zero capture histories (right). }
\begin{tabular}{cccc|cccc|cccc}
\hline
\multicolumn{4}{c}{Occupancy data}    &
\multicolumn{4}{c}{Capture-recapture} &
\multicolumn{4}{c}{Augmented C-R}     \\ \hline
site    & k=1 & k=2 & k=3 & ind & k=1 &k=2  & k=3 & ind & k=1 & k=2 & k=3           \\ \hline
1  & 0   & 1   & 0   & 1   & 0   & 1  & 0   & 1   & 0   & 1   & 0                   \\
2  & 1   & 0   & 1   & 2   & 1   & 0 & 1    & 2 & 1 & 0 & 1 \\
3  & 0   & 1   & 0   & .   & 0   & 1 & 0    & 3 & 1 & 0 & 1 \\
4  & 1   & 0   & 1   & .   & 1   & 0 & 1    & 4 & 1 & 0 & 1 \\
5  & 0   & 1   & 1   & .   & 0   & 1 & 1    & 5 & 1 & 0 & 1 \\
.  & 0   & 1   & 1   & .   & 0   & 1 & 1    & . & 0 & 1 & 1 \\
.  & 1   & 1   & 1   & .   & 1   & 1 & 1    & . & 0 & 1 & 1 \\
.  & 1   & 1   & 1   & .   & 1   & 1 & 1    & . & 1 & 1 & 1 \\
   & 1   & 1   & 1   & .   & 1   & 1 & 1    & . & 1 & 1 & 1 \\
n  & 1   & 1   & 1   & n   & 1   & 1 & 1    & n & 1 & 1 & 1 \\
.  & 0   & 0   & 0   &     &     &   &      & . & 0 & 0 & 0 \\
.  & 0   & 0   & 0   &     &     &   &      & . & 0 & 0 & 0 \\
   & 0   & 0   & 0   &     &     &   &      &   & 0 & 0 & 0 \\
   & 0   & 0   & 0   &     &     &   &      &   & 0 & 0 & 0 \\
   & 0   & 0   & 0   &     &     &   &      &   & 0 & 0 & 0 \\
   & 0   & 0   & 0   &     &     &   &      & N & 0 & 0 & 0 \\
.  & 0   & 0   & 0   &     &     &   &      & . & 0 & 0 & 0 \\
.  & 0   & 0   & 0   &     &     &   &      &   & 0 & 0 & 0 \\
M  & 0   & 0   & 0   &     &     &   &      & . & 0 & 0 & 0 \\
   &     &     &     &     &     &   &      & . & . & . & . \\
   &     &     &     &     &     &   &      & . & . & . & . \\
   &     &     &     &     &     &   &      & . & . & . & . \\
   &     &     &     &     &     &   &      & M & 0 & 0 & 0 \\
\end{tabular}
\label{closed.tab.occ}
\end{table}


\subsection{Model $M_0$ in BUGS}

For model $M_0$ in which we can aggregate the encounter data to
individual-specific encounter frequencies, the augmented data are
given by the vector of frequencies $(y_{1}, \ldots, y_{n}, 0, 0,
\ldots, 0)$. The zero-inflated model of the augmented data combines
the model of the latent variables, $z_{i} \sim \mbox{Bern}(\psi)$ with
the conditional-on-$z$ binomial model:
\begin{eqnarray*}
y_{i}|z_{i} = 1   &\sim& \mbox{Bin}(K,p) \\
y_{i} | z_{i} = 0 &\sim& \delta(0) 
\end{eqnarray*}
It is convenient to express the conditional-on-$z$ observation model concisely as:
\[
 y_{i}|z_{i} \sim \mbox{Bin}(K, p z_{i})
\]
Thus, if $z_{i}=0$ then the success probability of the binomial
distribution is identically 0 whereas, if $z_{i}=1$, then the success
probability is $p$. This is useful in describing the model in the {\bf
  BUGS}
language, as shown in Panel \ref{closed.panel.da4m0}.
 Note the last line of the model
specification  provides the expression for computing $N$ from the
data augmentation variables $z_{i}$.

\begin{panel}[htp]
\centering
\rule[0.15in]{\textwidth}{.03in}
%\begin{minipage}{5in}
{\small
\begin{verbatim}
model{
p  ~ dunif(0,1)
psi~dunif(0,1)

# nind = number of individuals captured at least once
#   nz = number of uncaptured individuals added for PX-DA
for(i in 1:(nind+nz)) {
    z[i]~dbern(psi)
   mu[i]<-z[i]*p
    y[i]~dbin(mu[i],K)
 }

N<-sum(z[1:(nind+nz)])
}
\end{verbatim}
}
%\end{minipage}
\rule[-0.15in]{\textwidth}{.03in}
\caption{Model $M_{0}$ under data augmentation.}
\label{closed.panel.da4m0}
\end{panel}




Specification of a more general model in terms of the individual
encounter observations $y_{ik}$ is not much more difficult than for
the individual encounter frequencies.  We define the
observation model by a double loop and change the indexing of things
accordingly, i.e.,
\begin{verbatim}
for(i in 1:(nind+nz)) {
    z[i]~dbern(psi)
  for(k in 1:K){
      mu[i,k]<-z[i]*p
      y[i,k]~dbin(mu[i,k],1)
  }
}
\end{verbatim}
In this manner, it is straightforward to incorporate covariates on $p$ XXX for both individuals and sampling occasions XXX
(see discussion of this below and also Chapt. \ref{chapt.covariates} 
and consider other extensions.

\subsection{Formal development of data augmentation}

Use of DA for solving inference problems with unknown $N$ can be
justified as originating from the choice of uniform prior on $N$.  The
$\mathrm{Unif}(0,M)$ prior for $N$ is innocuous in the sense that the
posterior associated with this prior is equal to the likelihood for
sufficiently large $M$.  One way of inducing the $\mathrm{Unif}(0,M)$
prior on $N$ is by assuming the following hierarchical prior:
\begin{eqnarray}
\label{closed.eq.NgivenM}
  N &\sim& \mathrm{Bin}(M, \psi) \\ \nonumber
  \psi &\sim& \mathrm{Unif}(0,1)
\end{eqnarray}
which includes a new model parameter $\psi$ XXX (note that we have seen $\psi$ in the previous section as the proportion $E[N]/M$).XXX This parameter denotes
the probability that an individual in the super-population of size $M$
is a member of the population of $N$ individuals exposed to sampling.
The model assumptions, specifically the multinomial model 
(Eq. \ref{closed.eq.multinomial4m0})
and Eq. \ref{closed.eq.NgivenM}, may be combined to yield a
reparameterization of the conventional model that is appropriate for
the augmented data set of known size $M$:
\begin{equation}
\label{closed.eq.multinomial4DA}
    (n_1, n_2, \ldots, n_K) \sim \mathrm{Multin}(M, \psi  \pi(1), \psi \pi(2), \ldots, \psi \pi(K))
\end{equation}
This arises by removing $N$ from Eq. \ref{closed.eq.multinomial4m0} by 
integrating
over the binomial prior distribution for $N$. Thus, the models we
analyze under data augmentation arise formally by removing the
parameter $N$ from the ordinary model - the model conditional on $N$ -
by integrating over a binomial prior distribution for $N$.

Note that the $M-n$ unobserved individuals in the augmented data set
have probability $\psi \pi(0) + (1-\psi)$, indicating that these
unobserved individuals are a mixture of individuals that are sampling
zeros ($\psi \pi_0$, and belong to the population of size $N$) and
others that are ``structural zeros'' (occurring in the augmented data
set with probability $1 - \psi$). In Eq.~\ref{closed.eq.multinomial4DA} $N$
has been eliminated as a formal parameter of the model by
marginalization (integration) and replaced with the new parameter
$\psi$, the data augmentation parameter.
However, the full likelihood containing both $N$ and $\psi$ can also be
analyzed \citep[see][]{royle_etal:2007}.


\subsection{Remarks on Data Augmentation}

Data augmentation may seem like a strange and mysterious black-box,
and likely it is unfamiliar to most people, even those with substantial
experience with capture-recapture models. However, it really is a
formal reparameterization of capture-recapture models in which $N$ is
removed from the ordinary (conditional-on-$N$) model by integration.
In the case of Model $M_0$, data augmentation produces the zero-inflated
binomial which is distinct from the original observation model, but
only in the sense that it embodies, explicitly, the $\mbox{Unif}(0,M)$
prior for $N$.  Choice of $M$ might be cause for some concern related
to potential sensitivity to choice of $M$. The guiding principle is
that it should be chosen large enough so that the posterior for $N$ is
not truncated, but no larger because large values entail more
computational burden. It seems likely that the properties of the
Markov chains should be affected by $M$ and so some optimality might
exist \citep{gopalaswamy_etal:2012}, as in occupancy models
\citep{mackenzie_royle:2005}. Formal analysis of this is needed.


We emphasize the motivation for data augmentation being that it
produces a data set of fixed size, so that the parameter dimension in
any capture-recapture model is also fixed.  As a result, MCMC is a
relatively simple proposition using standard Gibbs Sampling.  Consider
the simplest context - analyzing Model $M_0$ using the occupancy type
model. In this case, DA converts Model $M_0$ to a basic occupancy model
and the parameters $p$ and $\psi$ have known full-conditional
distributions (in fact, beta distributions) that can be sampled from
directly.  Furthermore, the data augmentation variables - i.e., the 
data augmentation variables $z$, can be sampled from Bernoulli full
conditionals. MCMC is not too much more difficult for complicated
models - sometimes the hyperparameters need to be sampled using a
Metropolis-Hastings step, but nothing more sophisticated than that is
required.

There are other approaches to analyzing models with unknown $N$, using
reversible jump MCMC (RJMCMC) or other so-called ``trans-dimensional''
(TD) algorithms
 \citep{durban_elston:2005, king_brooks:2001, king_etal:2008,
schofield_barker:2008, wright_etal:2009}. What distinguishes DA from RJMCMC and
related TD methods is that DA is used to create a distinctly new model
that is unconditional on $N$ and we (usually) analyze the
unconditional model. The various TD/RJMCMC approaches seek to analyze
the conditional-on-$N$ model in which the dimensional of the parameter
space is a variable function of $N$. TD/RJMCMC approaches might appear
to have the advantage that one can model $N$ explicitly or consider
alternative priors for $N$. However, despite that $N$ is removed as an
explicit parameter in DA, it is possible to develop hierarchical
models that involve structure on $N$ \citep{converse_royle:2010,
  royle_etal:2011ms} which we consider in Chapt. \ref{chapt.hscr}.

\subsection{Example: Black Bear Study on Fort Drum}

To illustrate the analysis of Model $M_0$ using data augmentation, we use
a data set collected at Fort Drum Military Installation in upstate New
York by the Department of Defense, Cornell University and
colleagues. These data have been analyzed in various forms by
\citet{wegan:2008,gardner_etal:2009} and \citet{gardner_etal:2010jwm}.
The specific data used here are encounter histories on 47 individuals
obtained from an array of 38 baited ``hair snares''
(Fig. \ref{fig.3.bears1}) during June and July 2006.  Barbed wire
traps were baited and checked for hair samples each week for eight
weeks, thus we have $K=8$ sample intervals. The data are provided 
in the {\bf R} package \mbox{\tt scrbook} 
and the analysis can be set up and run as
follows. Here, the data were augmented with $M-n = 128$ ($M=175$)
all-zero encounter histories.

\begin{figure}
\centering
\includegraphics[height=2.5in,width=1.9in]{Ch3/figs/hairsnares.png}
\caption{Fort Drum study area and hair snare locations.}
\label{fig.3.bears1}
\end{figure}

{\small
\begin{verbatim}
library("scrbook")
data("beardata")
trapmat<-beardata$trapmat
nind<-dim(beardata$bearArray)[1]
K<-dim(beardata$bearArray)[3]
ntraps<-dim(beardata$bearArray)[2]

M=175
nz<-M-nind
Yaug <- array(0, dim=c(M,ntraps,K))

Yaug[1:nind,,]<-beardata$bearArray
y<- apply(Yaug,c(1,3),sum) # summarize by ind x rep
y[y>1]<- 1             # toss out duplicate obs
ytot<-apply(y,1,sum)   # total encounters out of K
\end{verbatim}
}

The raw data object, \mbox{\tt beardata\$bearArray} is a 3-dimensional
array $\mbox{\tt nind} \times \mbox{\tt ntraps} \times K$ of
individual encounter events (i.e., $y_{ijk} = 1$ if individual $i$ was
encountered in trap $j$ during occasion $k$, and 0 otherwise).  For
fitting model $M_{0}$ or $M_{h}$ (see below), it is sufficient to
reduce the data to individual encounter frequencies which we have
labeled \mbox{\tt ytot} above.  The {\bf BUGS} model file along with
commands to fit the model are as follows:

{\small
\begin{verbatim}
set.seed(2013)               # to obtain the same results each time
library("R2WinBUGS")
data0<-list(y=y,M=M,K=K)
params0<-list('psi','p','N')
zst=c(rep(1,nind),rbinom(M-nind, 1, .5))
inits =  function() {list(z=zst, psi=runif(1), p=runif(1)) }

cat("
model {

psi~dunif(0, 1)
p~dunif(0,1)

for (i in 1:M){
   z[i]~dbern(psi)
   for(k in 1:K){
      tmp[i,k]<-p*z[i]
      y[i,k]~dbin(tmp[i,k],1)
       }
       }
N<-sum(z[1:M])
}
",file="modelM0.txt")

fit0 = bugs(data0, inits, params0, model.file="modelM0.txt",
       n.chains=3, n.iter=2000, n.burnin=1000, n.thin=1,
       debug=TRUE,working.directory=getwd())
\end{verbatim}
}
This produces the follow posterior
 summary statistics:
{\small
\begin{verbatim}
> print(fit0,digits=2)
Inference for Bugs model at "modelM0.txt", fit using WinBUGS,
 3 chains, each with 2000 iterations (first 1000 discarded)
 n.sims = 3000 iterations saved
           mean    sd   2.5%    25%    50%    75%  97.5% Rhat n.eff
psi        0.29  0.04   0.22   0.26   0.29   0.31   0.36    1  3000
p          0.30  0.03   0.25   0.28   0.30   0.32   0.35    1  3000
N         49.94  1.99  47.00  48.00  50.00  51.00  54.00    1  3000
deviance 489.05 11.28 471.00 480.45 488.80 495.40 513.70    1  3000

[.. some output deleted ...]
\end{verbatim}
}
{\bf WinBUGS} did well in choosing an MCMC algorithm for this model --
we have $\hat{R} = 1$ for each parameter, and an effective sample size
of 3000, equal to the total number of posterior samples.
We see that the posterior mean of $N$ under this
model is $49.94$ and a 95\% posterior interval is $(48,54)$.  We
revisit these data later in the context of more complex models.



In order to obtain an estimate of density, $D$, we need an area to
associate with the estimate of $N$, XXXX and in Chapt. \ref{chapt.intro} we already went through a number of commonly used procedures to
conjure up such an area, including buffering the trap array by the home
range radius, often estimated by the mean maximum distance moved
(MMDM) \citep{parmenter_etal:2003},
$1/2$ MMDM \citep{dice:1938} or
directly from telemetry data (REF XXX NEED REF HERE, WALLACE ET AL 2003 DO THIS; I HAVE SEEN 2 PAPERS CITING OTIS ET AL 1978 IN THIS CONTEXT BUT I ONLY FOUND THE SECITON WHERE THEY SUGGEST USING INFORMATION ON ANIMAL HOME RANGE AS OBTAIN FROM TRAPPING DATA; I GUESS THIS DICE GUY SAID TO USE THE HOME RANGE RADIUS AND PEOPLE JUST TRY TO GET AT THIS WHICHEVER WAY THEY CAN; BE IT RECAPTURES OR OTHER HOME RANGE INFORMATION XXXXX).
Typically, the trap
array is defined by the convex hull around the trap locations, and
this is what we applied a buffer to. We computed the buffer by using
an estimate of the mean female home range radius (2.19 km) estimated from
telemetry studies \citep{bales_etal:2005} instead of using an estimate
based on our relatively more sparse recapture data.
 For the Fort Drum study, the convex hull has area
$157.135$ $km^2$, and the buffered convex hull has area $277.011$
$km^2$.
To create this we used functions contained in the {\bf R} package
\mbox{\tt rgeos} and created a utility function \mbox{\tt bcharea}
which is in our {\bf R} package \mbox{\tt scrbook}. The commands are
as follows:
\begin{verbatim}
library("rgeos")

bcharea<-function(buff,traplocs){
p1<-Polygon(rbind(traplocs,traplocs[1,]))
p2<-Polygons(list(p1=p1),ID=1)
p3<-SpatialPolygons(list(p2=p2))
p1ch<-gConvexHull(p3)
 bp1<-(gBuffer(p1ch, width=buff))
 plot(bp1, col='gray')
 plot(p1ch, border='black', lwd=2, add=TRUE)
 gArea(bp1)
}

bcharea(2.19,traplocs=trapmat)
\end{verbatim}
The resulting buffered convex hull is shown in Fig. \ref{closed.fig.bch}.
\begin{figure}
\begin{center}
\includegraphics[height=3in,width=3in]{Ch3/figs/bufferedCH}
\end{center}
\caption{buffered convex hull of the bear hair snare array}
\label{closed.fig.bch}
\end{figure}

To conjure up a
density estimate under model $M_0$, we compute the appropriate
posterior summary of $N$ and the prescribed area ($277.011$ $km^2$):
\begin{verbatim}
> summary(fit0$sims.list$N/277.011)
   Min. 1st Qu.  Median    Mean 3rd Qu.    Max.
 0.1697  0.1733  0.1805  0.1803  0.1841  0.2130

> quantile(fit0$sims.list$N/277.011,c(0.025,0.975))
     2.5%     97.5%
0.1696684 0.1949381
\end{verbatim}
which yields a density estimate of about $0.18$ ind/km$^2$, and a $95\%$ Bayesian
confidence interval of $(0.170, 0.195)$.

In summary, we have an estimate of density if we have faith in our
stated value of the ``sample area''. Clearly though this is largely
subjective, and not something we can formally evaluate from the data.
How certain are we of this area? Can
we quantify our uncertainty about this quantity? 
 More important, what exactly is
the meaning of this area and, in this context, how do we gauge bias
and/or variance of ``estimators'' of it? (i.e., what is it
estimating?).  
XXX I DON'T KNOW IF IT'S WORTH MENTIONING THE DELTA APPROXIMATION KARANTH AND NICHOLS (1998) USE XXX
There is no theory to guide us in trying to answer these important questions.


\section{Temporally varying and behavioral effects}

The purpose of this chapter is mainly to emphasize the central
importance of the binomial model in capture-recapture and so we have
considered models for individual encounter frequencies - the number of
times individuals are captured out of $K$ samples.  Sometimes it is
not acceptable to aggregate the encounter data for each individual --
such as when encounter probability varies over time among samples. 
Time-varying responses that are relevant in many
capture-recapture studies are ``effort'' such as amount of search time,
number of observers, or trap nights, or when encounter probability
varies over time or as a function of date or season due to species behavior
\citep{kery_etal:2010}.
  A common situation in a large number of carnivore studies is that in
which there exists a ``behavioral response'' to trapping (even if the
animal is not physically trapped).
XXXX IS THERE ANY PARTICULAR REASON WHY YOU ONLY REFER TO CARNIVORES HERE? XXXX

Behavioral response is an important concept in carnivore studies
because individuals might learn to come to baited traps or avoid traps
due to trauma related to being encountered.  There are a number of
ways to parameterize a behavioral response to encounter. The
distinction between persistent and ephemeral was made by
\citet{yang_chao:2005} who considered a general behavioral response
model of the form:
\[
\mbox{logit}(p_{ik}) = \alpha_{0} + \alpha_{1}*y_{i,k-1} + \alpha_{2} x_{ik}
\]
where $x_{ik}$ is a covariate indicator variable of previous capture
(i.e., $x_{ik} = 1$ if captured in any previous period). Therefore,
encounter probability changes depending on whether an individual was
captured in the immediate previous period (ephemeral behavioral
response XXX described by the term $\alpha_{1}*y_{i,k-1}$) or in any previous period (persistent behavioral
response). The former probably models a behavioral response due to
individuals moving around their territory relatively slowly over time
and the latter probably accommodates trap happiness due to baiting or
shyness due to trauma.   XXX Spatial capture-recapture models allow us to include trap-specific covariates, XXX and in such models it makes
sense to consider a local behavioral response that is trap-specific
\citep{royle_etal:2011jwm} - that is, the encounter probability is
modified for an individual trap depending on previous capture in
that trap.

Models with temporal effects are easy to describe in the {\bf BUGS} language
and analyze and we provide a number of examples in
Chapt. \ref{chapt.covariates} and elsewhere. 


\section{ Models with individual heterogeneity}
\label{closed.sec.modelmh}

Here we consider models with individual-specific encounter probability
parameters, say $p_{i}$, which we model according to some probability
distribution, $g(\theta)$. We denote this basic model assumption as
$p_{i} \sim g(\theta)$. This type of model is similar in concept to
extending a GLM to a GLMM but in the capture-recapture context $N$ is
unknown.  The basic class of models is often referred to as ``Model
$M_h$'' but really this is a broad class of models, each being
distinguished by the specific distribution assumed for $p_{i}$.  There
are many different varieties of Model $M_{h}$ including parametric and
various putatively non-parametric approaches
\citep{burnham_overton:1978, norris_pollock:1996, pledger:2000}. One
important practical matter is that estimates of $N$ can be extremely
sensitive to the choice of heterogeneity model
\citep{fienberg_etal:1999, dorazio_royle:2003, link:2003}. Indeed,
\citet{link:2003} showed that in some cases it's possible to find
models that yield precisely the same expected data, yet produce wildly
different estimates of $N$. In that sense, $N$ for most practical
purposes is not identifiable across classes of mixture models, and
this should be understood before fitting any such model. One solution
to this problem is to seek to model explicit factors that contribute
to heterogeneity, e.g., using individual covariate models (See
\ref{closed.sec.indcov} below). Indeed, spatial capture-recapture
models seek to do just that, by modeling heterogeneity due to the
spatial organization of individuals in relation to traps or other
encounter mechanism.  For additional background and applications of
Model $M_{h}$ see \citet[][chapt. 6]{royle_dorazio:2008} and
\citet[][chapt. 6]{kery_schaub:2011}.

Model $M_{h}$ has important historical relevance to spatial
capture-recapture situations \citep{karanth:1995} because
investigators recognized that the juxtaposition of individuals with
the array of trap locations should yield heterogeneity in encounter
probability, and thus it became common to use some version of Model $M_h$
in spatial trapping arrays to estimate $N$.  While this doesn't
resolve the problem of not knowing the area relevant to $N$, it does
yield an estimator that accommodates the heterogeneity in $p$ induced
by the spatial aspect of capture-recapture studies.

To see how this juxtaposition induces heterogeneity, we have to
understand the relevance of movement in capture-recapture models.
Imagine a quadrat that can be uniformly searched by a crew of
biologists for some species of reptile (see
\citet{royle_young:2008}).  Figure \ref{closed.fig.quadrat} shows a
sample quadrat searched repeatedly over a period of time. Further,
suppose that species exhibits some sense of spatial fidelity in the
form of a home range or territory, and individuals move about their
home range (home range centroids are given by the blue dots) in some
kind of random fashion.  
%It is natural to think about it in terms of a
%movement process and sometimes that movement process can be modeled
%explicitly using hierarchical models \citep{royle_young:2008,
%  royle_etal:2011mee}.  
Heuristically, we imagine that each individual in
the vicinity of the study area is liable to experience variable
exposure to encounter due to the overlap of its home range with the
sampled area - essentially the long-run proportion of times the
individual is within the sample plot boundaries, say $\phi$. We
might model the exposure of an individual to capture by supposing that
$z_{i} = 1$ if individual $i$ is available to be captured (i.e.,
within the survey plot) during any sample, and $0$ otherwise. Then,
$\Pr(z_{i}=1) = \phi$.  In the context of spatial studies, it is
natural that $\phi$ should depend on {\it where} an individual lives,
i.e., it should be individual-specific $\phi_{i}$
\citep{chandler_etal:2011}. This system describes, precisely, that of
``random temporary emigration'' \citep{kendall_etal:1997} where $\phi_{i}$
is the individual-specific probability of being ``available'' for
capture.

Conceptually, SCR models aim to deal with
this problem of variable exposure to sampling due to movement in the
proximity of the trapping array explicitly and formally with auxiliary
spatial information.  If individuals are detected with probability
$p_{0}$, {\it conditional} on $z_{i} = 1$, then the marginal
probability of detecting  individual $i$ is
\[
 p_{i} = p_{0}\phi_{i}
\]
so we see clearly that individual heterogeneity in encounter
probability is induced as a result of the juxtaposition of individuals
(i.e., their home ranges) with the sample apparatus and the movement
of individuals about their home range.

\begin{figure}
\begin{center}
\includegraphics[height=3in]{Ch3/figs/quadrat}
\end{center}
\caption{A quadrat searched for lizards and the locations of each
  lizard over some period of time.}
\label{closed.fig.quadrat}
\end{figure}

We will work with a specific type of Model $M_{h}$ here, that in which
we extend the basic binomial observation model of Model $M_{0}$ so
that
\[
\mbox{logit}(p_{i}) = \mu + \eta_{i}
\]
where
\[
\eta_{i} \sim \mbox{Normal}(0, \sigma_{p}^2)
\]
We could as well write
\[
\mbox{logit}(p_{i}) \sim \mbox{Normal}(\mu,\sigma_{p}^2)
\]
This ``logit-normal mixture'' was analyzed by
\citet{coull_agresti:1999} and elsewhere. It is a natural extension of
the basic model with constant $p$, as a mixed GLMM, and similar models
occur throughout statistics. It is also natural to consider a beta
prior distribution for $p_{i}$ \citep{dorazio_royle:2003} and
so-called ``finite-mixture'' models XXX (models in which individuals are assumed to belong to a finite number of latent classes, each of which has its own capture probability) XXX are also popular
\citep{norris_pollock:1996, pledger:2000}.

\subsection{Analysis of Model $M_h$}

If $N$ is known, it is worth taking note of the essential simplicity
of model $M_h$ as a binomial GLMM.  This is a type of model that is
widely applied in just about every scientific discipline and using
standard methods of inference based either on integrated likelihood
\citep{laird_ware:1982, berger_etal:1999} which we discuss in
Chapt. \ref{chapt.mle} or standard Bayesian
methods. However, because $N$ is not known, inference is somewhat more
challenging. We address that here using Bayesian analysis based on
data augmentation (DA). Although we use data augmentation in the context of
Bayesian methods here, we note that
heterogeneity models formulated under DA are easily analyzed by
conventional likelihood methods as zero-inflated binomial mixtures
\citep{royle:2006} and more traditional analysis of model $M_h$ based on
integrated likelihood, without using data augmentation, has been
considered by \citet{coull_agresti:1999}, \citet{dorazio_royle:2003},
and others.

As with model $M_{0}$, we have the Bernoulli model for the
zero-inflation variables: $z_{i} \sim \mbox{Bern}(\psi)$ and the model
of the observations expressed conditional on the latent variables
$z_{i}$. For $z_{i}=1$, we have a binomial model with
individual-specific $p_{i}$:
\[
y_{i}|{z_{i} \! = \! 1} \sim \mbox{Bin}(K,p_{i})
\]
and otherwise $y_{i} |{ z_{i} \! = \! 0} \sim \delta(0)$. Further, we
prescribe a distribution for $p_{i}$. Here we assume
\[
\mathrm{logit}(p_{i}) \sim \mbox{Normal}(\mu,\sigma^2)
\]
The basic {\bf BUGS} description for this model, assuming a
$\mbox{Unif}(0,1)$ prior for $p_{0} = \mbox{logit}^{-1}(\mu)$, is given
as follows:
{\small
\begin{verbatim}
model{

p0 ~ dunif(0,1)       # prior distributions
mup<- log(p0/(1-p0))
taup~dgamma(.1,.1)
psi~dunif(0,1)

for(i in 1:(nind+nz)){
  z[i]~dbern(psi)     # zero inflation variables
  lp[i] ~ dnorm(mup,taup) # individual effect
  logit(p[i])<-lp[i]
  mu[i]<-z[i]*p[i]
  y[i]~dbin(mu[i],J)  #  observation model
 }

N<-sum(z[1:(nind+nz)])  # N is a derived parameter
}
\end{verbatim}
}


\subsection{Analysis of the Fort Drum data}

The logit-normal heterogeneity model was fitted to the bear data from
the Fort Drum study, and we used data augmentation to produce a data
set of $M=500$ individuals.  We ran the model using {\bf JAGS} with
the instructions given as follows:
{\small
\begin{verbatim}
[... get data as before ....]

set.seed(2013)

cat("
model{
p0 ~ dunif(0,1)       # prior distributions
mup<- log(p0/(1-p0))
sigmap ~ dunif(0,10)
taup<- 1/(sigmap*sigmap)
psi~dunif(0,1)

for(i in 1:(nind+nz)){
  z[i]~dbern(psi)     # zero inflation variables
  lp[i] ~ dnorm(mup,taup) # individual effect
  logit(p[i])<-lp[i]
  mu[i]<-z[i]*p[i]
  y[i]~dbin(mu[i],K)  #  observation model
 }

N<-sum(z[1:(nind+nz)])
}
",file="modelMh.txt")

data1<-list(y=ytot, nz=nz, nind=nind,K=K) 
params1= c('p0','sigmap','psi','N')
inits =  function() {list(z=as.numeric(ytot>=1), psi=.6, p0=runif(1),
          sigmap=runif(1,.7,1.2),lp=rnorm(M,-2)) }

library("rjags")
jm<- jags.model("modelMh.txt", data=data1, inits=inits, n.chains=4,
                 n.adapt=1000)
jout<- coda.samples(jm, params1, n.iter=200000, thin=1)
\end{verbatim}
}
This produces the posterior distribution for $N$ shown
in Fig. \ref{closed.fig.bearMh}. Posterior summaries of parameters are
given as follows:
{\small
\begin{verbatim}
> summary(jout)

Iterations = 2001:202000
Thinning interval = 1 
Number of chains = 4 
Sample size per chain = 2e+05 

1. Empirical mean and standard deviation for each variable,
   plus standard error of the mean:

           Mean       SD  Naive SE Time-series SE
N      117.7740 56.31633 6.296e-02       1.960115
p0       0.0728  0.05522 6.174e-05       0.001655
psi      0.2366  0.11362 1.270e-04       0.003909
sigmap   2.0795  0.53096 5.936e-04       0.016789

2. Quantiles for each variable:

            2.5%      25%       50%      75%    97.5%
N      62.000000 82.00000 102.00000 134.0000 277.0000
p0      0.003143  0.02842   0.06077   0.1066   0.2036
psi     0.117269  0.16377   0.20522   0.2712   0.5560
sigmap  1.211900  1.69434   2.02113   2.4028   3.2694
\end{verbatim}
}


We used $M=500$ for this analysis and we
note that  while the posterior mass of $N$ is concentrated away from this
upper bound (Fig. \ref{closed.fig.bearMh}), the posterior has an
extremely long right tail, with some posterior values at the upper
bound $N=500$. Maybe or
maybe not sufficient data augmentation.\footnote{
{\bf to do: } insert final results. longer run. more data
augmentation. compare with winbugs.
}
The model runs effectively in {\bf WinBUGS} but sometimes with apparently
inefficient mixing for reasons that may be related to bad starting
values. In some cases this was resolved if we supplied starting values
for the $logit(p_{i})$ parameters and $\tau$.


Because of the skewed posterior we see that the posterior mean ($N=117$)
is
considerably higher than the posterior mode ($N=102$). Moreover, 
posterior summaries are estimated with a relatively high error
(``Time-series SE'' of around 2.0)\footnote{need to define this somewhere XXX THIS COMES UP IN CH2 XXX}.
Further, it may be surprising that the posterior mode does not compare
well with the MLE. To compute the posterior mode we could easily find
the posterior value of $N$ with the highest mass because $N$ is
discrete. But we want to smooth out some of the Monte Carlo error a
bit so we used a smoothing spline to the posterior frequencies of $N$
as follows:
\begin{verbatim}
  tt<-table(jout[[1]][,"N"])[1:80]
  xg<-as.numeric(names(tt))
  plot(xg,tt)
  sp<- smooth.spline(xg,tt,df=9)
  sp$x[sp$y==max(sp$y)]
[1] 80
\end{verbatim}
The \mbox{\tt df} argument controls the degree of smoothing and we
find in this case that the modal value (i.e., 80) is not too sensitive
to the smoothing parameter but this should be checked in any specific
instance\footnote{we need to give examples of using \mbox{\tt
    density()} to obtain modes}.

To compute the MLE, we used 
the {\bf R} code contained in Panel 6.1 of \citet{royle_dorazio:2008}.  The
MLE of $log(n_{0})$, the logarithm of the number of uncaptured
individuals, is $\widehat{log(n0)} = 3.86$ and therefore $\hat{N} =
exp(3.86)+47 = 94.47$ which is not at all consistent with the apparent
mode in 
Fig. \ref{closed.fig.bearMh}.
\footnote{We note that the result is inconsistent with Gardner et
  al. (2009) who reported an MLE of 104.1 ($density = 0.437
  inds/km^2$) although we do not know the reason for this at the
  present time.}  
%To convert this to density we use the buffered area
%as computed above (255.3 $km^2$)\footnote{WRONG \#} and perform the
%required summary analysis on the posterior samples of $N$, which
%results in about $0.37$ individuals/$km^2$. The reader should carry
%out this analysis to confirm the estimates, and also obtain the $95\%$
%confidence interval.

{\bf Remarks:} First of all the posterior for this model and data set is
very sensitive to prior distributions. While MLEs are invariant to
transformation of the parameters, the posterior distribution
definitely is {\it not} invariant. In the present case, the use of a
$\mbox{Unif}(0,1)$ prior for $p_{0} = \mbox{expit}(\mu)$ is somewhat
informative -- in particular, it is not at all ``flat'' on the scale
of $\mu$ -- and this affects the posterior.  We generally always
recommend use of a $\mbox{Unif}(0,1)$ prior for $\mbox{expit}(\mu)$ in such
models. That said, we were surprised at this result, and we
experimented with other prior configurations including putting a flat
prior on $\mu$ directly. That specific prior suggests the possibility
that the posterior distribution may be improper for that prior
specification. This kind of small sample instability has been widely
noted in model $M_h$ \citep{fienberg_etal:1999, dorazio_royle:2003} and
is not unrelated to sensitivity to
model XXX WORD MISSING? XXX which has also been identified as an important issue in model
$M_{h}$ \citep{dorazio_royle:2003,link:2003}.
Conclusion: The mode is well-defined but the data set is sparse and
hence inferences are poor and sensitive to model choices. Get over it.


\begin{figure}
\centering
\includegraphics[height=4.5in,width=4.5in]{Ch3/figs/bear-modelMh-post}
\caption{Posterior of $N$ for Fort Drum bear study data under the
logit-normal version of model $M_h$. 
}
\label{closed.fig.bearMh}
\end{figure}


\subsection{Building your own MCMC algorithm}

For fun, we construct our own MCMC algorithm using a Metropolized
Gibbs sampler for model $M_{h}$ in Chapt. \ref{chapt.mcmc}, where we
also develop the MCMC 
algorithms for spatial capture-recapture models.
XXX MAYBE PUT THIS IN A FOOTNOTE? XXX

\begin{comment}

To begin, we first collect all of our model components
which are as follows: $[y_{i}| p_{i},z_{i}]$,
$[p_{i}|\mu_{p},\sigma_{p}]$, and $[z_{i}|\psi]$
for {\it each} $i=1,2,\ldots,M$ and then prior distributions
$[\mu_{p}]$, $[\sigma_{p}]$ and $[\psi]$.
The joint posterior distribution of all unknown quantities in the model
is proportional to the joint distribution of all elements
$y_{i},p_{i},z_{i}$ and also the prior distributions of the prior parameters:
\[
\left\{ \prod_{i=1}^{M} [y_{i}|p_{i},z_{i}][p_{i}|\mu_{p},\sigma_{p}]
[z_{i}|\psi] \right\} [\mu_{p},\sigma_{p},\psi]
\]
For prior distributions, we assume that $\mu_{p},\sigma_{p}, \psi$ are
mutually independent and for $\mu_{p}$ and $\sigma_{p}$ we use
improper uniform priors, and $\psi \sim \mbox{Unif}(0,1)$.  Note that
the likelihood contribution for each individual, when conditioned on
$p_{i}$ and $z_{i}$, does not depend on $\psi$, $\mu_{p}$, or
$\sigma_{p}$.  As such, the full-conditionals for the structural
parameters $\psi$ only depends on the collection of data augmentation
variables $z_{i}$, and that for $\mu_{p}$ and $\sigma_{p}$ will only
depends on the collection of latent variables $p_{i}; i=1,2,\ldots,M$.
The full conditionals for all the unknowns are as follows:

{\bf (1)} For $p_{i}$:
\begin{eqnarray*}
[p_{i}|y_{i}, \mu_p, \sigma_{p},z_{i}=1] &\propto  &
[y_{i}|p_{i}][p_{i}|\mu_p,\sigma_{p}^{2}] \mbox{ if $z_{i}=1$ }  \\
                 &  &  [p_{i}|\mu_p,\sigma_{p}] \mbox{if $z_{i}=0$ }
\end{eqnarray*}

{\bf (2)} for $z_{i}$:
\[
z_{i} | \cdot \propto [y_{i}|z_{i}*p_{i}] \mbox{Bern}(z_{i}|\psi)
\]

{\bf (3)} For $\mu_{p}$:
\[
[\mu_{p} | \cdot ] \sim \prod_{i} [p_{i}| \cdot] *\mbox{const}
\]


{\bf (4)} For $\sigma_{p}$:
\[
[ \sigma_{p}|\cdot ] \sim\prod_{i}[p_{i}| \cdot ]*\mbox{const}
\]

{\bf (5)} For $\psi$:
\[
\psi|\cdot\sim \mbox{Beta}(1 + \sum z_{i}, 1 + M - \sum z_{i})
\]


We've  identified each of the full conditional
distributions in sufficient detail to apply the
Metropolis-Hastings algorithm. With the exception of $\psi$ which has
a convenient analytic solution -- it is a beta distribution which we
can easily sample directly. In truth, we could also sample $\mu_{p}$
and $\sigma_{p}^{2}$ directly with certain choices of prior
distributions. For example, if $\mu_{p} \sim \mbox{Normal}(0, 1000)$
then the full conditional for $\mu_{p}$ is also normal, etc..
We implement an MCMC algorithm for this model in the following block
of {\bf R} code.  The basic structure is: initialize the parameters
and create any required output or intermediate data holders, and then
begin the main MCMC loop which, in this case, generates 100000
samples.\footnote{This data grabbing function is not implemented yet}

{\small
\begin{verbatim}
## obtain the bear data by executing the previous data grabbing
## function

temp<-getdata()
M<-temp$M
K<-temp$K
ytot<-temp$ytot

###
### MCMC algorithm for Model Mh

out<-matrix(NA,nrow=100000,ncol=4)
dimnames(out)<-list(NULL,c("mu","sigma","psi","N"))
lp<- rnorm(M,-1,1)
p<-expit(lp)
mu<- -1
p0<-exp(mu)/(1+exp(mu))
sigma<- 1
psi<- .5
z<-rbinom(M,1,psi)
z[ytot>0]<-1

for(i in 1:100000){

### update the logit(p) parameters
lpc<- rnorm(M,lp,1)  # 0.5 is a tuning parameter
pc<-expit(lpc)
lik.curr<-log(dbinom(ytot,K,z*p)*dnorm(lp,mu,sigma))
lik.cand<-log(dbinom(ytot,K,z*pc)*dnorm(lpc,mu,sigma))
kp<- runif(M) < exp(lik.cand-lik.curr)
p[kp]<-pc[kp]
lp[kp]<-lpc[kp]

p0c<- rnorm(1,p0,.05)
if(p0c>0 & p0c<1){
muc<-log(p0c/(1-p0c))
lik.curr<-sum(dnorm(lp,mu,sigma,log=TRUE))
lik.cand<-sum(dnorm(lp,muc,sigma,log=TRUE))
if(runif(1)<exp(lik.cand-lik.curr)) {
 mu<-muc
 p0<-p0c
}
}

sigmac<-rnorm(1,sigma,.5)
if(sigmac>0){
lik.curr<-sum(dnorm(lp,mu,sigma,log=TRUE))
lik.cand<-sum(dnorm(lp,mu,sigmac,log=TRUE))
if(runif(1)<exp(lik.cand-lik.curr))
 sigma<-sigmac
}

### update the z[i] variables
zc<-  ifelse(z==1,0,1)  # candidate is 0 if current = 1, etc..
lik.curr<- dbinom(ytot,K,z*p)*dbinom(z,1,psi)
lik.cand<- dbinom(ytot,K,zc*p)*dbinom(zc,1,psi)
kp<- runif(M) <  (lik.cand/lik.curr)
z[kp]<- zc[kp]

psi<-rbeta(1, sum(z) + 1, M-sum(z) + 1)

out[i,]<- c(mu,sigma,psi,sum(z))
}
\end{verbatim}
}


{\bf Remarks}: (1) for parameters with bounded support, i.e.,
$\sigma_{p}$ and $p_{0}$, we are using a random walk candidate
generator but rejecting draws outside of the parameter space.  (2) We
mostly use Metropolis-Hastings except for the data augmentation
parameter $\psi$ which we sample directly from its full-conditional
distribution which is a beta distribution.  (3) Even the latent data
augmentation variables $z_{i}$ are updated using Metropolis-Hastings
although they too can be updated directly from their full-conditional.
\end{comment}


\begin{comment}

\subsection{Exercises related to model Mh}

\begin{itemize}
\item[(1)] Enclose the MCMC algorithm in an R function and provide
  arguments for some of the parameters of the function that a user
  might wish to modify.
\item[(2)] Execute the function and compare the results to those
  generated from WinBUGS in the previous section
\item[(3)] Note that the prior distribution for the ``mean'' parameter
  is given on $p_0=exp(\mu)/(1+exp(\mu))$.  Reformulate the algorithm
  with a flat prior on $\mu$ and see what happens. Contemplate this.
\item[(4)] Using Bayes rule, figure out the full conditional for
  $z_{i}$ so that you don't have to use MH for that one. It might be
  more efficient. Is it?
\item[(5)] Modify the MCMC algorithm so that the prior for $\mu_{p}$
  is an improper flat prior. i.e., $[\mu_{p}] \propto 1$. Describe the
  posterior distribution of $N$. 
\end{itemize}

\end{comment}



\section{Individual Covariate Models: Toward Spatial Capture-Recapture}
\label{closed.sec.indcov}


A standard situation in capture-recapture models is when an individual
covariate is measured, and this covariate is thought to influence
encounter probability.  As with other closed population models, we
begin with the basic binomial observation model:
\[
y_{i} \sim \mbox{Bin}(K, p_{i})
\]
and we assume also  a model for encounter probability according to:
\begin{equation}
 \mbox{logit}(p_{i}) = \alpha + \beta x_{i}
\label{closed.eq.ha}
\end{equation}
Classical examples of covariates influencing detection probability are
type of animal (juvenile/adult or male/female), a continuous covariate
such as body mass \citep[][ch. 6]{royle_dorazio:2008}, or a
discrete covariate such as group or cluster size. For example, in
models of aerial survey data, it is natural to model detection
probabilities as a function of the observation-level individual
covariate, ``group size'' \citep{royle:2008, royle:2009,
  langtimm_etal:2011}.

Such ``individual covariate models'' are similar in structure to Model
$M_{h}$, except that the individual effects are {\it observed} for the
$n$ individuals that appear in the sample. These models are important
here because spatial capture-recapture models are precisely a form of
individual covariate model, an idea that we will develop here and
elsewhere. Specifically, they are such models, but where the
individual covariate is a partially observed latent variable for 
captured individuals. As such, it is a type of measurement error.
That is, unlike model $M_h$, we do have some direct information about the
latent variable, which comes from the spatial locations/distribution
of individual recaptures.

Traditionally, estimation of $N$ in individual covariate models is
achieved using methods based on ideas of unequal probability sampling
(i.e., Horwitz-Thompson estimation; see \citet{huggins:1989} and
\citet{alho:1990}). An estimator of $N$ is
\[
\hat{N} = \sum_{i}^{n} \frac{1}{\tilde{p}_{i}}
\]
where $\tilde{p}_{i}$ is the probability that individual $i$ appeared
in the sample.  That is, $\tilde{p}_{i} = \Pr(y_{i}>0)$
where, in closed population capture-recapture models, 
\[
\Pr(y_{i}>0) = (1- (1-p_{i})^K)
\]
where $p_{i}$ is a function of parameters $\alpha$ and $\beta$
according to
Eq. \ref{closed.eq.ha}.
In practice, parameters are estimated 
from the conditional-likelihood of the observed encounter histories
which is, for observation $y_{i}$, 
\[
{\cal L}_{c}(\alpha, \beta | y_{i}) = \frac{ \mbox{Bin}(y_{i}|\alpha,\beta) } { \tilde{p}_{i}}.
\]

Here we take a formal model-based approach to Bayesian analysis of
such models based on the joint likelihood
using data augmentation \citep{royle:2009}. Classical
likelihood analysis of the so-called ``full likelihood'' is covered 
 by \citet{borchers_etal:2002}.  For Bayesian analysis of
individual covariate models, because the individual covariate is
unobserved for the $N-n$ uncaptured individuals, we require a model to
describe variation among individuals, essentially allowing the sample
to be extrapolated to the population\footnote{weak argument}.  For our present purposes, we
consider a continuous covariate and we assume that it has a normal
distribution:
\[
x_{i} \sim \mbox{Normal}(\mu,\sigma^{2})
\]

Data augmentation can be applied directly to this class of models. In
particular, reformulation of the model under DA yields a basic
zero-inflated binomial model of the form:
\begin{eqnarray*}
z_{i} &\sim& \mbox{Bern}(\psi) \; \; \; i=1,2,\ldots,M\\
y_{i}|{z_{i}\! =\! 1} &\sim& \mbox{Bin}(K,p_{i}(x_{i})) \\
y_{i} |{ z_{i}\! =\! 0} &\sim& \delta(0)  \\
x_{i} & \sim & \mbox{Normal}(\mu,\sigma^{2})
\end{eqnarray*}
Fully spatial capture-recapture models use this
formulation with a latent covariate that is directly related to the
individual detection probability (see next section). As with the
previous models, implementation is trivial in the {\bf BUGS} language. The
{\bf BUGS} specification is very similar to that for model $M_h$, but we
require the distribution of the covariate to be specified, along with
priors for the parameters of that distribution.


\subsection{Example: Location of capture as a covariate.}

If we had a regular grid of traps over some closed geographic system
then we imagine that the average location of capture would be a decent
estimate (heuristically) of an individual's home range center.
Intuitively some measure of typical distance from home range center to
traps for an individual should be a decent covariate to explain
heterogeneity in encounter probability, i.e., individuals with more
exposure to traps should have higher encounter probabilities and vice
versa.  A version of this idea was put forth by
\citet{boulanger_mclellan:2001} (see also \citet{ivan:2012}), but
using the Huggins-Alho estimator and with covariate ``distance to
edge'' of the trapping array. A limitation of this  approach is
that it does not provide a solution to the problem that the trap area
is fundamentally ill-defined, nor does it readily accommodate the
inherent and heterogeneous variation in this measured covariate.

Here, we provide an example of this type of heuristically motivated
approach using the fully model-based individual covariate model
described above analyzed by data augmentation. We take a slightly
different approach than that adopted by
\citet{boulanger_mclellan:2001}. By analyzing the full likelihood and
placing a prior distribution on the individual covariate, we resolve
the problem of having an ill-defined area over which the population
size is distributed. After you read later chapters of this book, it
will be apparent that SCR models represent a formalization of this
heuristic procedure.

For our purposes here, we define $x_{i} = ||{\bf s}_{i} - {\bf
  x}_{0}||$ where ${\bf s}_{i}$
is the average encounter location of individual $i$ and ${\bf x}_{0}$ is the
centroid of the trap array.  Conceptually, individuals in the middle
of the array should have higher probability of encounter and, as
$x_{i}$ increases, $p_{i}$ should therefore decrease. We note that we
have defined ${\bf s}_{i}$ in terms of a sample quantity - the observed mean
- which is ad hoc but consistent with existing applications in the literature.
For an expansive, dense trapping grid then we might expect the sample mean
encounter location to be a good estimate of home range center but,
clearly this is biased for individuals that live around the edge (or
off) the trapping array. Regardless, it should be good enough for our
present purposes of demonstrating this heuristically appealing
application of an individual covariate model. A key point is that
${\bf s}_{i}$ is missing for each individual that is not encountered and
thus so is $x_{i}$. Thus,
it is a latent variable, or random effect, and we need therefore to
specify a probability distribution for it.
As a measurement of distance we know it must be
positive-valued. Thinking about this like a distance sampling problem
lets first try to make $x_{i}$ uniform from $0$ to some large number,
say $D_{max}$, beyond which it would be difficult to imagine an
individual being captured. For example, $D_{max}$ should be at a home
range diameter past the furthest trap from the center.
As such, we use this distribution for the individual covariate
``distance from center of the trap array''
\[
 x_{i} \sim \mbox{Unif}(0,D_{max})
\]
where $D_{max}$ is a specified constant, which we may choose to be
arbitrarily large.  In practice, people have
used distance from edge of the trap array but that is less easy to
make sense of.


\subsubsection{Fort Drum Bear Study}


\begin{figure}
\centering
\includegraphics[height=3.5in,width=3.5in]{Ch3/figs/bear_spiderplot.png}
\caption{Spider plot of the Fort Drum study data.}
\label{closed.fig.spiderplot}
\end{figure}


We have to do a little bit of data processing to fit this individual
covariate model to the Fort Drum data. 
We need to compute the individual covariate ${\bf x}_{i}$ (distance from the centroid of the trapping
array) using the {\bf R} function
\mbox{\tt spiderplot}
provided in \mbox{\tt scrbook}. This function also produces a keen plot shown in
Fig. \ref{closed.fig.spiderplot} which we call a ``spider plot''.
The {\bf R} commands for obtaining the individual covariate ``distance from trap centroid''
are as follows:
\begin{verbatim}
library("scrbook")
data("beardata")
toad<- spiderplot(beardata$bearArray,beardata$trapmat)
xcent<-toad$xcent
\end{verbatim}
We picked $D_{max} = 11.5$ $km^2$ which is about the distance from the
array center to the furthest trap. 
Once we specific $D_{max}$ then the implication is that the population
size parameter applies to the area 
within 11.5 units of the trap centroid\footnote{To be convincing
  this might  need a little bit of hand-holding}. The {\bf BUGS} model
specification and {\bf R} commands to package the data and fit the model are
as follows:

{\small
\begin{verbatim}
cat("
model{
p0 ~ dunif(0,1)       # prior distributions
mup<- log(p0/(1-p0))
psi~dunif(0,1)
beta~dnorm(0,.01)

for(i in 1:(nind+nz)){
  xcent[i]~dunif(0,maxD)
  z[i]~dbern(psi)     # DA variables
  lp[i] <- mup + beta*xcent[i] # individual effect
  logit(p[i])<-lp[i]
  mu[i]<-z[i]*p[i]
  y[i]~dbin(mu[i],K)  #  observation model
 }
N<-sum(z[1:(nind+nz)])
}
",file="modelMcov.txt")

data2<-list(y=ytot,nz=nz,nind=nind,K=K,xcent=xcent,Dmax=maxD)
params2<-list('p0','psi','N','beta')
inits =  function() {list(z=zst, psi=psi, p0=runif(1),beta=rnorm(1) ) }
fit2 = bugs(data2, inits, params2, model.file="modelMcov.txt",working.directory=getwd(),    
       debug=T, n.chains=3, n.iter=11000, n.burnin=1000, n.thin=1)
\end{verbatim}
}

This produces the following posterior summaries:
{\small
\begin{verbatim}
Inference for Bugs model at "modelMcov.txt", fit using WinBUGS,
 3 chains, each with 11000 iterations (first 1000 discarded)
 n.sims = 30000 iterations saved
           mean    sd   2.5%    25%    50%    75%  97.5% Rhat n.eff
p0         0.54  0.07   0.40   0.50   0.54   0.59   0.67    1  1100
psi        0.34  0.05   0.25   0.31   0.34   0.37   0.44    1  3500
N         58.92  5.49  50.00  55.00  58.00  62.00  71.00    1  1900
beta      -0.25  0.06  -0.36  -0.29  -0.25  -0.21  -0.12    1   780
deviance 459.51 13.21 435.80 450.20 458.80 467.90 487.40    1  2600
\end{verbatim}
}


It might be 
perplexing that the estimated $N$ is much lower than obtained by model
$M_h$ but there is a good explanation for this, discussed
subsequently. That issue notwithstanding, it is worth pondering how
this model could be an improvement (conceptually or technically) over
some other model/estimator including $M_0$ and $M_h$ considered
previously. Well, for one, we have accounted formally for
heterogeneity due to spatial location of individuals relative to
exposure to the trap array, characterized by the centroid of the
array. Moreover, we have done so using a model that is based on an
explicit mechanism, as opposed to a phenomenological one such as Model
$M_h$. Moreover, importantly, using our new model, {\it the estimated N
  applies to an explicit area which is defined by our prescribed value
  of maxD}. That is, this area is a fixed component of the model and
the parameter $N$ therefore has explicit spatial context, as the number
of individuals with home range centers less than $D_{max}$ from the
centroid of the trap array. As such, the implied ``effective trap
area''\footnote{This is a bad use of this term. We have never defined
  ETA or ESA. What is it, exactly? XXX IT IS SOMEWHAT DEFINED IN CH1; IN THE QUOTE FROM OBRIEN; ALTHOUGH HE NAMES IT EFFECTIVE AREA XXX} for a given $D_{max}$ is that of a circle
with radius $D_{max}$.



\begin{figure}
\begin{center}
\includegraphics[width=3.5in]{Ch3/figs/Nchains}
\end{center}
\caption{Needs a caption}
\label{closed.fig.ha}
\end{figure}

\subsection{Extension of the Model}

This model is actually not a very good model for one important reason:
Imposing a uniform prior distribution on $x$
implies that density is {\it not constant} over space. In
particular, this model implies that it {\it decreases} as we move away
from the centroid of the trap array. 
That is, $x_{i} \sim \mbox{Unif}(0,D_{max})$ implies constant $N$ in
each distance band from the centroid but obviously the {\it area} of
each distance band is increasing.  
This is one reason we have a
lower estimate of density than that obtained previously from model $M_0$ and also why,
if we were to increase $D_{max}$, we would see density continue to
decrease.

Fortunately, the use of an individual covariate model is {\it not} restricted to
use of this specific distribution for the individual
covariate. Clearly, it is a bad choice and, therefore, we should think
about whether we can choose a better distribution for $D_{max}$ - one that
doesn't imply a decreasing density as distance from the centroid
increases.  Conceptually, what we want to do is impose a prior on
distance from the centroid, $x$, such that density is proportional to
the amount of area in each successive distance band as you move
farther away from the centroid.  In fact, there is theory that exists
which tells us what the correct distribution of $x$ is
$2x/D_{max}^2$. This can be derived by noting that $F(x) = \Pr(X<x) =
\pi*x*x/\pi*D_{max}^{2}$ . Then, $f(x) = dF/dx =
2*x/(D_{max}^{2})$. This is a sort of triangular distribution in
density
induced because the incremental area in each additional distance band
increases linearly with radius (i.e., distance from centroid). It is
sometimes comforting to verify things empirically:
{\small
\begin{verbatim}
 u<-runif(10000,-1,1)
 v<-runif(10000,-1,1)
 d<- sqrt(u*u+v*v)
 hist(d[d<1])
 hist(d[d<1],100)
 hist(d[d<1],100,probability=TRUE)
 abline(0,2)
\end{verbatim}
}

It would be useful if we could describe this distribution in {\bf BUGS} but
there is not a built-in way to do this that we are aware of.  One possibility is to use a
discrete version of the pdf. We might also be able to use what is
referred to in {\bf WinBUGS} jargon as the ``zeros trick'' (see {\it Advanced
BUGS tricks} in the manual) although we haven't pursued this approach. Instead, we
consider using a discrete version and break $D_{max}$ into $L$ distance
classes of width $\delta$, with probabilities proportional to
$2*x$. In particular, if we denote the cut-points by $xg_{1}=0,xg_{2}, \ldots,
xg_{L+1}=D_{max}$ and the interval midpoints are $xm_{i} = 
xg_{i+1}-\delta$ then the interval probabilities are $p_{i} = 
2*xm_{i}*\delta/(D_{max}^{2})$, which we can compute once and then pass
them to {\bf WinBUGS} as data.

The {\bf R} commands for doing all of this (noting that we have already loaded and processed
the Fort Drum bear data) are given  as follows. In the model description the
variable $x$ (observed distance from centroid of the trap array) has been rounded so that the
discrete version of the $f(x)$ can be used as described
previously. The new variable labeled \mbox{\tt xround} is actually
then the integer category label in units of $\delta$ from 0. Thus, to
convert back to distance in the expression for $lp[i]$, \mbox{\tt
  xround[i]} has to be multiplied by $\delta$. Here is the {\bf BUGS} model 
  specification:
{\small
\begin{verbatim}
delta<-.2
xround<-xcent%/%delta  + 1
Dgrid<- seq(delta,maxD,delta)
xprobs<- delta*(2*Dgrid/(maxD*maxD))
xprobs<-xprobs/sum(xprobs)

cat("
model{
p0 ~ dunif(0,1)       # prior distributions
mup<- log(p0/(1-p0))
psi~dunif(0,1)
beta~dnorm(0,.01)

for(i in 1:(nind+nz)){
  xround[i]~dcat(xprobs[])
  z[i]~dbern(psi)                     # zero inflation variables
  lp[i] <- mup + beta*xround[i]*delta # individual effect
  logit(p[i])<-lp[i]
  mu[i]<-z[i]*p[i]
  y[i]~dbin(mu[i],K)  #  observation model
 }

N<-sum(z[1:(nind+nz)])
}
",file="modelMcov.txt")
\end{verbatim}
}

To fit the model we do this - keeping in mind that the data objects
required below have been defined in previous analyses of this chapter:
{\small
\begin{verbatim}
data2<-list(y=ytot,nz=nz,nind=nind,K=K,xround=xround,xprobs=xprobs,delta=delta)
params2<-list('p0','psi','N','beta')
inits =  function() {list(z=z, psi=psi, p0=runif(1),beta=rnorm(1) ) }
fit = bugs(data2, inits, params2, model.file="modelMcov.txt",
          working.directory=getwd(), debug=FALSE, n.chains=3, n.iter=11000, 
          n.burnin=1000, n.thin=2)
\end{verbatim}
}

This is a useful model because it induces a clear definition of area
in which the population of $N$ individuals reside. Under this model,
that area is defined by specification of $D_{max}$. We can apply the model
for different values of $D_{max}$ and observe that the estimated $N$ varies
with $D_{max}$. Fortunately, we see empirically, that while $N$ seems
highly sensitive to the prescribed value of $D_{max}$, density seems to
be invariant to $D_{max}$ as long as it is chosen to be sufficiently
large. We fit the model for a random of values of $D_{max}$ from $D_{max}=12$ (restricting
values of $x$ to be in close proximity to
the trap array) on up to 20. The results are given in Table
\ref{closed.tab.Dmax}.


\begin{table}[htp]
\centering
\caption{Analysis of Fort Drum bear hair snare data using the individual covariate model, for different values of Dmax, the upper limit of the uniform distribution of `distance from centroid of the trap array' }
\begin{tabular}{ccc}
\hline \hline
 Dmax & mean & SD \\ \hline
  12& 0.230 & 0.038 \\
  15& 0.244 &0.041 \\
  17& 0.249 &0.044 \\
  18& 0.249 &0.043\\
  19& 0.250 &0.043\\
  20& 0.250 &0.044
\end{tabular}
\label{closed.tab.Dmax}
\end{table}


We see that the posterior mean and SD of density (individuals per
square km) appear insensitive to choice of $D_{max}$ once we get a 
ways away from the maximum observed value of about 11.5. The estimated
density of 0.25 per km$^2$ is actually quite a bit lower than we 
reported using model $M_h$ 
for which  no relevant ``area'' quantity is explicit in the model.
Using MLEs of $N$ in conjunction with buffer strips
(see Table \ref{intro.tab.fdtests}) our estimates were in the range of $0.32-0.43$ and
the Bayesian estimates were XXXX (posterior mode of N = 102) or XXX (posterior mean of N = 117)
(see sec.
\ref{closed.sec.modelmh} above). 
On the other hand our estimate of $\hat{D} = 0.25$ here (based on the posterior mean) is 
higher than that reported from model $M_0$ using the buffered area
(0.18). There is no basis really for comparing or contrasting these
various estimates and it would be a useful philosophical exercise for
the reader to discuss this matter. In particular, application of models
$M_0$ and $M_h$ are distinctly {\it not} spatially explicit models -- the
area within which the population\footnote{We need to look back at
  Chapter 1 and make sure we quit calling this ``sample area'' - it
  really isn't that at al, but rather the area within which $N$
  resides.} resides is not defined under either model. There is
therefore no reason at all to think that the estimates produced under
either either closed population model, based on a buffered ``trap area'', 
are justifiable by any
theory. In fact, we would get exactly the same estimate of $N$ no
matter what we declare the area to be. On the other hand, the
individual covariate model explicitly describes a distribution for
``distance from centroid'' that is a reasonable and standard null
model - it posits, in the absence of direct information, that
individual home range centers are randomly distributed in space and
that probability of detection depends on the distance between home
range center and the centroid of the trap array. Under this definition
of the system, we see that density is invariant to the choice of
sample area which seems like a desirable feature. 

The individual
covariate model is not ideal, however, because it does not make full
use of the spatial information in the data set, i.e., the trap
locations and the locations of each individual encounter, and there is hope
to extend this model in order to resolve remaining deficiencies.


\subsection{Invariance of density to $D_{max}$}

Under the model above, and also under models that we consider in later
chapters, a general property of the estimators is that while $N$
increases with the prescribed trap area (equivalent to $D_{max}$ in this
case), we expect that density estimators should be invariant to this
area. In the model used above, we note that $Area(D_{max}) = 
\pi*D_{max}^{2}$ and $E[N(D_{max})] = \lambda*Area(D_{max})$ and thus
$E[Density(D_{max})] = \lambda$, i.e., constant. This should be 
interpreted as the {\it prior} density. Absent data, then realizations
under the model will have density $\lambda$ regardless of what $D_{max}$
is prescribed to be.  As we verified empirically above, the posterior
density is also invariant Of $D_{max}$ as long as the implied area
is large enough so that the data no longer provide
information about density (i.e., ``far away'').

\subsection{Toward Fully Spatial Capture-recapture Models}

We developed this model for the average observed location and equated
it to home range center ${\bf s}_{i}$. Intuitively, taking the average
encounter location as an estimate of home range center makes sense but
more so when the trapping grid is dense and expansive relative to
typical home range sizes.  However, our approach also ignored the
variable precision with which each ${\bf s}_{i}$ is estimated and also, as
noted previously, estimates of ${\bf s}_{i}$ around the ``edge'' (however we
define that) are biased because the observations are truncated (we can
only observe locations within the trap array).  In the next chapter we
provide a further extension of this individual covariate model that
definitively resolves the ad hoc nature of the individual covariate
approach we took here. In that chapter we build a model in which ${\bf s}_{i}$
are regarded as latent variables and the observation locations (i.e.,
trap specific encounters) are linked to those latent variables with an
explicit model. We note that the model fitted previously could be
adapted easily to deal with ${\bf s}_{i}$ as a latent variable, simply by
adding a prior distribution for ${\bf s}_{i}$. The reader should contemplate
how to do this in {\bf BUGS}.


\section{DISTANCE SAMPLING: A primative Spatial Capture-Recapture Model}

Distance sampling is one of the most popular methods for estimating
animal abundance. One of the great benefits of distance sampling is
that it provides explicit estimates of {\it density}. The distance
sampling model is a special case of a closed population model with a
covariate. The covariate in this case, $x_{i}$, is the distance
between an individual's location ``$u$'' and the observation location
or transect. In fact, the model underlying distance sampling is
precisely the same model as that which applies to the
individual-covariate models, except that observations are made at only
$K=1$ sampling occasion. In a sense, distance sampling is a spatial
capture-recapture model, but without the ``recapture.''  This first
and most basic spatial capture-recapture model has been used routinely
for decades and, formally, it is a spatially-explicit model in the
sense that it describes, explicitly, the spatial organization of
individual locations (although this is not always stated explicitly)
and, as a result, somewhat general models of how individuals are
distributed in space can be specified \citep{royle_etal:2004,
  johnson_etal:2010, sillett_etal:2011}.

As before, the distance sampling model, under data augmentation,
includes a set of $M$ zero-inflation variables $z_{i}$ and the
binomial model expressed conditional on $z$ (binomial for $z=1$, and
fixed zeros for $z=0$).  In distance sampling we pay for having only a
single sample (i.e., $K=1$) by requiring constraints on the model of
detection probability. A standard model is
\[
\log(p_{i}) = \beta x_{i}^{2}
\]
for $\beta < 0$, where $x_i$ denotes the distance at which the $i$th
individual is detected relative to some reference location where
perfect detectability ($p=1$) is assumed. This function corresponds to
the ``half-normal'' detection function (i.e., with $\beta =
1/\sigma^{2}$).  If $K>1$ then an intercept in this model is
identifiable and
such models are usually called ``capture-recapture distance
sampling''\citep{alpizar_pollock:1996,borchers_etal:1998}.

As with previous examples, we require a distribution for the individual covariate $x_{i}$. The customary choice is
\[
x_{i} \sim \mbox{Unif}(0,B)
\]
wherein $B>0$ is a known constant, being the upper limit of data
recording by the observer (i.e., the point count radius, or transect
half-width). In practice, this is sometimes asserted to be infinity,
but in such cases the distance data are usually truncated.
Specification of this distance sampling model in the {\bf BUGS} language is
shown in Panel \ref{closed.panel.distance} from \citet{royle_dorazio:2008}.


\begin{panel}[htp]
\centering
\rule[0.15in]{\textwidth}{.03in}
\begin{minipage}{5in}
\begin{verbatim}
beta~dunif(0,10)
psi~dunif(0,1)

for(i in 1:(nind+nz)){
   z[i]~dbern(psi)    # DA Variables
   x[i]~dunif(0,B)    # B=strip width
   p[i]<-exp(logp[i])   # DETECTION MODEL
   logp[i]<-   - beta*(x[i]*x[i])
   mu[i]<-z[i]*p[i]
   y[i]~dbern(mu[i])  # OBSERVATION MODEL
 }
N<-sum(z[1:(nind+nz)])
D<- N/striparea  # area of transects
\end{verbatim}
\end{minipage}
\rule[-0.15in]{\textwidth}{.03in}
\caption{Distance sampling model in {\bf BUGS}, using a half-normal
detection function.}
\label{closed.panel.distance}
\end{panel}

As with the individual covariate model in the previous section, the
distance sampling model can be equivalently specified by putting a
prior distribution on individual {\it location} instead of distance
between individual and observation point (or transect).  Thus we can
write the general distance sampling model as
\[
p_{i} = f(\beta,||{\bf u}_{i} - {\bf x}_0||)
\]
along with
\[
 {\bf u}_{i} \sim \mbox{Unif}({\cal S})
\]
where ${\bf x}_{0}$ is a fixed point (or line) and ${\bf u}_{i}$ is
the individual's location which is observable for $n$ individuals. In
practice it is easier to record distance instead of location.  Basic
math can be used to argue that if individuals have a uniform
distribution in space, then the distribution of Euclidean distance is
also uniform. In particular, if a transect of length $L$ is used and $x$
is distance to the transect then $F(x) = \Pr(X\le x) = L*x/L*B = x/B$ and
$f(x) = dF/dx = (1/B)$. For measurements of radial distance, see the
previous section.

In the context of our general characterization of SCR models 
(Chapt. \ref{modeling.sec.characterization}),
we suggested that every SCR model can be described,
conceptually, by a hierarchical model of the form:
\[
 [y|u][u|s][s].
\]
Distance sampling ignores the part of the model pertaining to ${\bf
  s}$, and deals only with the model components for the observed
data  ${\bf u}$\footnote{Equivalently, we could also say that $[u]$ in
  the distance sampling model is $[u] = \int [u|{\bf s}][{\bf s}]
  d{\bf s}$}. Thus, we are left with a hierarchical model of the form
\[
[y|{\bf u}][{\bf u}].
\]
In contrast, as we will see in the next chapters, basic SCR models
(Chapt. \ref{chapt.scr0}) ignore ${\bf u}$ and condition on ${\bf s}$,
which is not observed:
\[
[y|{\bf s}][{\bf s}]
\]
Since $[{\bf u}]$ and $[{\bf s}]$ are both assumed to be uniformly
distributed, these are structurally equivalent models! The main
differences have to do with interpretation of model components and
whether or not the latent variables are observable (in distance
sampling they are).

So why bother with SCR models when distance sampling yields density
estimates and accounts for spatial heterogeneity in detection? For
one, imagine trying to collect distance sampling data on tigers!
Clearly, distance sampling requires that one can collect large
quantities of distance data, which is not always possible. For tigers,
it is much easier, efficient, and safer to employ camera traps or
tracking plates and then apply SCR models. Furthermore, as we will see
in Chapts.
\ref{chapt.searchencounter} and \ref{chapt.scrds}, SCR models can use distance data to estimate all the
parameters of our enchilada, allowing us to study distribution,
movement, and density. Thus, SCR models are much more general and
versatile than distance sampling models (which clearly are a special
case), and can accommodate data from virtually all animal survey
designs.


\subsection{Example: Muntjac deer survey from Nagarahole, India }

Here we fit distance sampling models to distance sampling data on the
muntjac deer (Muntiakus muntjak) collected in the year 2004 from
Nagarahole National Park in southern India
(Kumar et al. unpublished data). The muntjac is
a solitary species and distance measurements were made on 57 groups
that were largely singletons with 4 pairs of individuals.  Commands
for reading in and organizing the data for {\bf WinBUGS}, followed by
writing the model to a text file, are given below. Note that the total sampled area of
the transects is fed in as ``striparea'' which is $708$ (km of transect walked)
multiplied by the strip width ($B=120 = 0.12$ km) multiplied by 2.
{\small 
\begin{verbatim}
library("R2WinBUGS")
data<- read.csv("Muntjac.csv")
hist(data[,3],30)
nind<-nrow(data)
y<-rep(1,nind)
nz<-400
y<-c(y,rep(0,nz))
x<-data[,3]
x<-c(x,rep(NA,nz))
z<-y

cat("
model{
beta~dunif(0,10)
psi~dunif(0,1)

for(i in 1:(nind+nz)){
   z[i]~dbern(psi)    # DA Variables
   x[i]~dunif(0,B)    # B=strip width
   p[i]<-exp(logp[i])   # DETECTION MODEL
   logp[i]<-   -beta*(x[i]*x[i])
   mu[i]<-z[i]*p[i]
   y[i]~dbern(mu[i])  # OBSERVATION MODEL
 }
N<-sum(z[1:(nind+nz)])
D<- N/striparea  # area of transects
}
",file="dsamp.txt")
\end{verbatim}
}
Next, we provide inits, indicate which parameters to monitor, and then
pass those things to {\bf WinBUGS}:
{\small
\begin{verbatim}
data<-list(y=y,x=x,nz=nz,nind=nind,B=120,striparea=(708*2*.120))
params<-list('beta','N','D','psi')
inits =  function() {list(z=z, psi=runif(1), beta=runif(1,0,.02) )}
fit = bugs(data, inits, params, model.file="dsamp.txt",working.directory=getwd(),    
       debug=T, n.chains=3, n.iter=11000, n.burnin=1000, n.thin=2)
\end{verbatim}
}
Posterior summaries are provided in the following table. Estimated
density is pretty low, 1.1 individuals per sq. km.\footnote{ This is much
  lower than Samba's estimate produced from WinBUGS accounting for group
  size. Reason unknown. }
{\small
\begin{verbatim}
Inference for Bugs model at "dsamp.txt", fit using WinBUGS,
 3 chains, each with 11000 iterations (first 1000 discarded), n.thin = 2
 n.sims = 15000 iterations saved
           mean    sd   2.5%    25%    50%    75%  97.5% Rhat n.eff
beta       0.00  0.00   0.00   0.00   0.00   0.00   0.00    1  1100
N        185.73 26.53 138.00 167.00 184.00 203.00 242.00    1   570
D          1.09  0.16   0.81   0.98   1.08   1.20   1.42    1   570
psi        0.41  0.06   0.30   0.36   0.40   0.45   0.54    1   670
deviance 655.74 16.26 626.00 644.50 655.10 666.40 689.80    1  1300

[.... some output deleted .... ]
\end{verbatim}
}

\section{Summary and Outlook}

Traditional closed population capture-recapture models are closely
related to binomial generalized linear models.  Indeed, the only real
distinction is that in capture-recapture models, the population size
parameter $N$ (corresponding also to the size of a hypothetical
``complete'' data set) is unknown.  This requires special
consideration in the analysis of capture-recapture models. The
classical approach to inference recognizes that the observations don't
have a standard binomial distribution but, rather, a truncated
binomial (from which which the so-called ``conditional likelihood''
derives) since we only have encounter frequency data on observed
individuals. If instead we analyze the models using data augmentation,
the observations can be modeled using a zero-inflated binomial
distribution. In short, when we deal with the unknown-$N$ problem using
data augmentation then we are left with zero-inflated GLM and GLMMs
instead of ordinary GLM or GLMMs. The analysis of such zero-inflated
models is practically convenient, especially using the various
Bayesian analysis packages that use the {\bf BUGS} language.

Spatial capture-recapture models that we will consider in the rest of
the chapters of this book are closely related to what have been called
individual covariate models. Heuristically, spatial capture-recapture
models arise by defining individual covariates based on observed
locations of individuals -- we can think of using some function of
mean encounter location as an individual covariate. We did this in a
novel way, by using distance to the centroid of the trapping array as
a covariate. We analyzed the ``full likelihood'' using data
augmentation, and placed a prior distribution on the individual
covariate which was derived from an assumption that individual
locations are, a priori, uniformly distributed in space. This
assumption provides for invariance of the density estimator to the
choice of population size area (induced by maximum distance from the
centroid of the trap array). The model addressed some important problems in the
use of closed population models: it allows for heterogeneity in
encounter probability due to the spatial context of the problem and it
also provides a direct estimate of density because area is a feature
of the model (via the prior on the individual covariate). The model is
still not completely general because it does not make use of
the fully spatial encounter histories, which provide direct
information about the locations and density of individuals.  A
specific individual covariate model that is in widespread use is
classical ``distance sampling.'' The model underlying distance
sampling is precisely a special kind of SCR model - but one without
replicate samples. Understanding distance sampling and individual
covariate models more broadly provides a solid basis for understanding
and analyzing spatial capture-recapture models.



%%% TO DO  as of 12/29/11

 %%% Spell check document

 %%% Change "beta" to "theta"

 %%% Fix up R scripts and consolidate for R package
 %%% R commands to process wolverine data need included in that section

 %%% Run Wolverine 2k 4k and 8k grids in JAGS compare to WinBUGS
 %%%     insert those results in text

 %%%  For discrete state-space stuff, convert BUGS output to JAGS and
 %%%  figure out MC errors
 %%% Finish Table that has those results in it

 %% pick up all hard references to chapters and make float


\chapter{Fully Spatial Capture-Recapture Models}
\markboth{Chapter 4 }{}
\label{chapt.scr0}

\vspace{.3in}

In previous sections we discussed some classes of models that could be
viewed as primitive spatial capture-recapture models. We looked at a
basic distance sampling model and we also considered a classical
individual covariate modeling approach in which we defined a covariate
to be the distance from (estimated) home range center to the center of
the trap array. These were spatial in the sense that they included
some characterization of where individuals live but, on the other
hand, only a primitive or no characterization of trap location.  That
said, very little distinguishes these two models from spatial
capture-recapture models that we consider in this chapter which fully
recognize the spatial attribution of both individual animals {\it and}
the locations of encounter devices.

Fully spatial capture-recapture models must accommodate the spatial
organization of individuals and the encounter devices because the
encounter process occurs at the level of individual traps.  Failure to
consider the trap-specific collection of data is the key deficiency
with classical ad-hoc approaches which aggregate encounter information
to the resolution of the entire trap array. We have  previously
addressed some problems that this induces including induced
heterogeneity in encounter probability, imprecise notation of ``sample
area'' and not being able to accommodate trap-specific
effects.
In this chapter we resolve these issues by developing 
our first fully spatial capture-recapture
model which turns out to be precisely the model considered in sec. \ref{closed.sec.indcov}
 but instead of defining the individual covariate to be distance
to centroid of the array we define $J$ individual covariates - the
distance to {\it each} trap. And, instead of using estimates of
individual locations ${\bf s}$, we consider a fully hierarchical model in
which we regard ${\bf s}$ as a latent variable and impose a prior
distribution on it.  We can think of having $J$ independent
capture-recapture studies generating one data set for each trap, and
applying the individual covariate model with random activity centers,
and that is all the basic SCR model is.

In the following sections of this chapter we investigate the basic
spatial capture-recapture model, which we refer to as ``model SCR0'',  and address some important
considerations related to its analysis in {\bf WinBUGS}. We also demonstrate
how to summarize posterior output for the purposes of producing
density maps or spatial predictions of density.

\section{Sampling Design and Data Structure}

In our development here, we will assume a standard sampling design in
which an array of $J$ traps is operated for $K$ time periods (say,
nights) producing encounters of $n$ individuals.  Because sampling
occurs by traps and also over time, the most general data structure
yields encounter histories for {\it each individual} that are
temporally {\it and} spatially indexed. Thus a typical data set will
include an encounter history {\it matrix} for each individual.  For
the most basic model, there are no time-varying covariates that
influence encounter, there are no explicit individual-specific
covariates, and there are no covariates that influence density we will
develop models in this chapter for encounter data that are aggregated
over the temporal replicates. For example, suppose we observe 6
individuals in sampling at 4 traps over 3 nights of sampling then a
plausible data set is the $6 \times 4$ matrix of encounters, out of 3,
of the form:
\begin{verbatim}
      trap1 trap2 trap3 trap4
 [1,]     1     0     0     0
 [2,]     0     2     0     0
 [3,]     0     0     0     1
 [4,]     0     1     0     0
 [5,]     0     0     1     1
 [6,]     1     0     1     0
\end{verbatim}

We develop models in this chapter for devices such as ``hair snares''
or other DNA sampling methods \citep{kery_etal:2010,
  gardner_etal:2010jwm} and related types of sampling devices in which
(i) effective ``traps'' may capture any number of individuals (i.e.,
they don't fill up; This is referred to as a ``multi-catch'' type of
sampling \citep{efford_etal:2009ecol}); (ii) an individual may be
captured in any number of traps during each occasion but (iii)
individuals can be encountered at most 1 time in a trap during any
occasion.  The statistical assumptions are that individual encounters
within and among traps are independent, and this allows us to regard
individual- and trap-specific encounters as $iid$ Bernoulli trials
(see next section).  These basic (but admittedly at this point
somewhat imprecise) assumptions define the basic spatial
capture-recapture model, which we will refer to as ``SCR0'' 
so that we may use that model as a point of reference without having
to provide a long-winded enumeration of assumptions and sampling
design each time we do. We will make things more precise as we develop
a formal statistical definition of the model shortly.

While the model is mostly directly relevant
to hair snares and other DNA sampling methods for which multiple
detections of an individual are not distinguishable,
we will also make use of the model for data that arise from
camera-trapping studies. In practice, with camera trapping,
individuals might be photographed several times in a night but we will
typically distill such data into a single binary encounter event for
reasons discussed later in Chapt. \ref{chapt.poisson-mn}.


\section{The binomial observation model }

We assume that the individual and trap-specific encounters, $y_{ij}$,
are mutually independent outcomes of a binomial random variable:
\begin{equation}
	y_{ij} \sim \mbox{Bin}(K, p_{ij})
\label{scr0.eq.bin}
\end{equation}
This is the basic model underlying ``logistic regression'' (Chapt. \ref{chapt.glms})
as well as standard closed population models
(Chapt. \ref{chapt.closed}). The key
element of the model is that the encounter probability $p_{ij}$ is
indexed by (i.e., depends on) both individual and trap. In a sense,
then, we can think of each {\it trap} as producing individual level
encounter history data of the classical variety - an $\mbox{\tt nind}
\times \mbox{\tt nreps}$
matrix of 0's and 1's (this is the ``encountered at most 1 time''
assumption).


As we did in sec. \ref{closed.sec.indcov}, we will make explicit the notion that
$p_{ij}$ is defined conditional on ``where'' individual $i$
lives. Naturally, we think about defining an individual home range and
then relating $p_{ij}$ explicitly to the centroid of the individuals
home range, or its center of activity \citep{efford:2004,
  borchers_efford:2008, royle_young:2008}.  Therefore, define ${\bf
  s}_{i}$, a two-dimensional spatial coordinate, to be the activity
center for individual $i$. Then, the SCR model postulates that
encounter probability, $p_{ij}$, is a decreasing function
of distance between ${\bf s}_{i}$ and the location of trap $j$, ${\bf x}_{j}$.
 Naturally, if we think of modeling binomial counts using
logistic regression, we might specify the model according to:
\begin{equation}
	\mbox{logit}(p_{ij}) = \alpha_{0} + \alpha_1 ||{\bf s}_{i}-{\bf x}_{j} ||
\label{scr0.eq.logit}
\end{equation}
where, here, $||{\bf s}_{i}-{\bf x}_{j}||$ is the distance between
${\bf s}_{i}$ and ${\bf x}_{j}$. We sometimes write $||{\bf
  s}_{i}-{\bf x}_{j}|| = dist({\bf s}_{i},{\bf x}_{j}) =
d_{ij}$. Alternatively, if we think about distance sampling then we
might use the ``half-normal'' model of the form:
\[
p_{ij} = p_{0}*\exp(-\alpha_{1} *||{\bf s}_{i}-{\bf x}_{j}||^2)
\]
Or any of a large number of standard detection models that are
commonly used (we consider more in Chapt. \ref{chapt.covariates}). The half-normal model implies
\begin{equation}
\log(p_{ij})  = \log(p_{0}) - \alpha_{1} *||{\bf s}_{i}-{\bf x}_{j}||^2
\label{scr0.eq.norm}
\end{equation}
%We would always like to be clear that encounter probability depends on individual activity
%centers {\it and} trap locations {\it and} parameter(s) $\theta$, and
%so it would be ideal to write $p({\bf s}_{i},{\bf x}_{j}; \theta)$ or
%something similar. However, this can be extremely unwieldy and
%clutter up what are otherwise extremely simple mathematical
%expressions and formulae. As such, we will usually abbreviate these
%various dependencies by writing $p_{ij}$ or sometimes $p_{\theta,ij}$,
%understanding that $p_{ij}$ is actually a function of the various important
%quantities.
We probably expect that the parameter $\alpha_{1}$ in
Eq. \ref{scr0.eq.logit} or \ref{scr0.eq.norm} should be negative, so
that the probability of encounter decreases with distance between the
trap and individual home range center.  
Whatever model encounter probability we choose, we should always keep
in mind that the model is described conditional on ${\bf s}_{i}$,
which is an unobserved random variable.  Thus, to be precise about
this, we should write the observation model as
\[
y_{ij}|{\bf s}_{i} \sim \mbox{Bin}(K, p({\bf s}_{ij};\alpha_{1}))
\]


The joint likelihood for the
data, conditional on the collection of individual activity centers,
can therefore be expressed as
\[
{\cal L}(\alpha_{1} | \{ {\bf y}_{i},{\bf s}_{i} \}_{i=1}^{N})
 =  \prod_{i} \prod_{j} \mbox{Bin}(y_{ij}|p_{ij}(\alpha_{1}))
\]
Which, if we switch the indices on the product operators, this shows
the SCR likelihood (conditional on ${\bf s}$) to be the product of $J$
{\it independent} capture-recapture likelihoods - one for each trap.
However, the data have a distinct ``repeated measures'' type of structure, with
each of the $j$ likelihood contributions for each individual being
grouped by individual. Thus, we cannot analyze the model
meaningfully by $J$ trap-specific models. In classical repeated measures
types of models, we accommodate the group structure of the data using
random effects (random individual or group level variables). For SCR
models we take the same basic approach, which we develop subsequently.

\subsection{Distance as a latent variable}

If we knew precisely every ${\bf s}_{i}$ in the population (and how
many, $N$), then the model specified by eqs. \ref{scr0.eq.bin} and
\ref{scr0.eq.logit} is just an ordinary logistic
regression type of a model which we learned how to fit using {\bf
  WinBUGS} previously (Chapt. \ref{chapt.glms}), with a covariate $d_{ij}$. However,
the activity centers are unobservable even in the best possible
circumstances. In that case, $d_{ij}$ is an unobserved variable,
analogous to classical random effects models. We need to therefore
extend the model to accommodate these random variables with an
additional model component. A standard, and perhaps not unreasonable,
assumption is the so-called ``uniformity assumption'' which is to say
that the ${\bf s}_{i}$ are uniformly distributed over space (the
obvious next question ``which space?'' is addressed below).  This
uniformity assumption amounts to a uniform prior distribution on ${\bf
  s}_{i}$, i.e., the pdf of ${\bf s}_{i}$ is constant, which we may
express
\begin{equation}
	\Pr({\bf s}_{i}) \propto \mbox{\tt const}
\label{scr0.eq.sprior}
\end{equation}
 As it turns out, this assumption is usually not precise
enough to fit SCR models in practice for reasons we discuss in the
following section.  We will give another way to represent this prior
distribution that is more concrete, but it depends on specifying the
``state-space'' of the random variable ${\bf s}_{i}$. The term
state-space is a technical way of saying ``possible outcomes''.

To summarize the preceeding model developing, a basic SCR model is
defined by 3 essential components:
\begin{itemize}
\item[(1)] Observation model: $y_{ij}|{\bf s}_{i} \sim \mbox{Bin}(K, p_{ij})$
\item[(2)] Encounter probability: $\mbox{logit}(p_{ij}) = \alpha_{0} +
  \alpha_{1}*||{\bf s}_{i}-{\bf x}_{j}||$
\item[(3)] Point process model: $\Pr({\bf s}_{i} ) \propto \mbox{\tt const}$
\end{itemize}
Therefore, the SCR model is little more than an ordinary
capture-recapture model for closed populations. It is such a model,
but augmented with a set of ``individual effects'', ${\bf s}_{i}$,
which relate some sense of individual location to encounter
probability. 

\section{ The Binomial Point-process Model}

The collection of individual activity centers ${\bf s}_{1},\ldots,
{\bf s}_{N}$ represent a realization of a {\it binomial point process}
\citep[][p. xyz]{illian_etal:2008}.  The binomial point process (BPP)
is analogous to a Poisson point process in the sense that it
represents a ``random scatter'' of points in space - except that the
total number of points is {\it fixed}, whereas, in a Poisson point
process it is random (having a Poisson distribution).  As an example,
we show in Fig. \ref{scr0.fig.bpp} locations of 20 individual activity
centers (black dots) in relation to a grid of 25 traps. For a Poisson
point process the number of such points in the prescribed state-space
would be random whereas often we will simulate fixed numbers of
points, e.g., for evaluating the performance of procedures such as how
well does our estimator perform of $N=50$?
\begin{figure}
\begin{center}
\includegraphics[height=2.5in]{Ch4/figs/binomialpoint}
\end{center}
\caption{Realization (small circles) of a binomial point process with $N=20$. The
  large circles represent trap locations.}
\label{scr0.fig.bpp}
\end{figure}

It is natural to consider a binomial point process in the context of
capture-recapture models because it preserves $N$ in the model and thus
preserves the linkage directly with closed population models. In fact,
under the binomial point process model then model $M_0$ and other closed
models are simple limiting cases of SCR models, i.e., as the
coefficient on distance tends to 0.
In addition, use of
the BPP model allows us to use data augmentation for Bayesian analysis
of the models as in Chapt. \ref{chapt.closed}, thus yielding a methodologically
coherent approach to analyzing the different classes of
models. Despite this, making explicit assumptions about $N$, such as
Poisson, is convenient in some cases (see Chapt. \ref{chapt.hscr}).

One consequence of having fixed $N$, in the BPP model, is that the
model is not strictly a model of ``complete spatial randomness''. This
is because if one forms counts $n(A_{1}),\ldots, n(A_{k})$ in any set
of disjoint regions say $A_{1}, \ldots, A_{k}$, then these counts are
{\it not} independent.  In fact, they have a multinomial distribution
\citep[see][p. XYZ]{illian_etal:2008}. Thus, the BPP model introduces
a slight bit of dependence in the distribution of points. However, in
most situations this will have no practical effect on any inference or
analysis and, as a practical matter, we will usually regard the BPP
model as one of spatial independence among individual activity centers
because each activity center is distributed independently of each
other activity center. Despite this implicit independence we see in
Fig. \ref{scr0.fig.bpp} that {\it realizations} of randomly distributed
points will typically exhibit distinct non-uniformity. Thus,
independent, uniformly distributed points will almost never appear
regularly, uniformly or systematically distributed. For this reason,
the basic binomial (or Poisson) point process models are enormously
useful in practical settings.  More relevant for SCR models is that we
actually have a little bit of data for some individuals and thus the
resulting posterior point pattern can deviate strongly from
uniformity, a point we come back to repeatedly in this book.
The uniformity hypothesis is only
a {\it prior} distribution which is directly affected by the quantity
and quality of observations, to produce a posterior distribution which
may appear distinctly non-uniform.


\subsection{Definition of home range center}

Some will be offended by our use of the concept of ``home range
center'' and thus will have difficulty in believing that the resulting
model is really useful for anything.  Indeed, the idea of a home range
or activity center is a vague concept anyway, a purely
phenomenological construct.  Despite this, it doesn't really matter
whether or not a home range makes sense for a particular species -
individuals of any species inhabit {\it some} region of space and we
can define the ``home range center'' to be the center of the space
that individual was occupying (or using) during the period in which
traps were active. Thinking about it in that way, it could even be
observable (almost) as the centroid of a very large number of radio
fixes over the course of a survey period or a season.  Thus, this
practical version of a home range center in terms of space usage is a well-defined construct
regardless of whether one thinks the home range concept is meaningful,
even if individuals are not particularly territorial.  This is why we
usually use the term ``activity center'' or maybe even ``centroid of
space usage'' and we recognize that this construct is a transient
thing which applies only to a well-defined period of study.



\subsection{The state-space of the point process}

Shortly we will focus on Bayesian analysis of this model with $N$
known so that we can directly apply what we learned in
Chapt. \ref{chapt.glms} to 
this situation. To do this, we note that the individual effects ${\bf
  s}_{i},\ldots, {\bf s}_{N}$ are unknown quantities and we will need
to be able to simulate each ${\bf s}_{i}$ in the population from the
posterior distribution.  It should be self-evident that we cannot
simulate the ${\bf s}_{i}$ unless we describe precisely the region
over which they are uniformly distributed. This is
the quantity referred to above as the state-space, denoted henceforth
by ${\cal S}$, which is a region or a set of points comprising the
potential values of ${\bf s}_{i}$. Thus, an equivalent explicit
statement of the ``uniformity assumption'' is
\[
{\bf s}_{i} \sim \mbox{Unif}({\cal S})
\]
where ${\cal S}$ is a precisely defined region. e.g., in Fig. 
\ref{scr0.fig.bpp}, ${\cal S}$ is the square defined by $[-1,7] \times
[-1, 7]$. Thus each of the $N=20$ points were generated by randomly
selecting each coordinate on the line $[-1, 7]$. 


\subsubsection{Prescribing the state-space}

Evidently, we need to define the state-space, ${\cal S}$. How can we
possibly do this objectively? Prescribing any particular ${\cal S}$
seems like the equivalent of specifying a ``buffer'' which we
criticized previously as being ad hoc. How is it, then, is choosing a
state-space is {\it not} ad hoc? As a practical matter, it turns out
that estimates of density are insensitive to choice of the
state-space. As we observed in Chapt. \ref{chapt.closed}, it is true that $N$ increases
with ${\cal S}$, but only at the same rate as the area of ${\cal S}$
increases under the
prior assumption of constant density. As a result, we say that density
is invariant to ${\cal S}$ as long as ${\cal S}$ is sufficiently
large. Thus, while choice of ${\cal S}$ is (or can be) essentially
arbitrary, once ${\cal S}$ is chosen, it defines the population being
exposed to sampling, which scales appropriately with the size of the
state-space.

For our simulated system developed previously in this chapter, we
defined the state space to be a square within which our trap array was
centered. For many practical situations this might be an
acceptable approach to defining the state-space. We provide an example
of this in sec. \ref{scr0.sec.wolverine} below in which the trap array is
irregular and also situated within a realistic landscape that is
distinctly irregular.  In general, it is most practical to define the
state-space as a regular polygon (e.g., rectangle) containing the trap
array without differentiating unsuitable habitat. Although defining
the state-space to be a regular polygon has computational advantages
(e.g., we can implement this more efficiently in {\bf WinBUGS} and
cannot for irregular polygons), a regular polygon induces an apparent
problem of admitting into the state-space regions that are distinctly
non-habitat (e.g., oceans, large lakes, ice fields, etc.).  It is
difficult to describe complex sets in mathematical terms that can be
used in {\bf BUGS}. As an alternative, we can provide a
representation of the state-space as a discrete set of points (sec.
\ref{scr0.sec.discrete}) that will allow specific points to be deleted
or not depending on whether they represent habitat, or we can define
the state-space as an arbitrary  collection of polygons stored as a GIS
shapefile
which can be analyzed easily using MCMC
(see sec. \ref{mcmc.sec.state-space}), but not so easily in the {\bf
  BUGS} variants.  In what follows below we provide an
analysis of the camera data defining the state-space to be a regular
continuous polygon (a rectangle).


\subsection{Invariance and the State-space as a model assumption}
\label{scr0.sec.invariance}

We will assert for all models we consider in this book that density is
invariant to the size and extent of ${\cal S}$, if ${\cal S}$ is
sufficiently large as long 
as our model relating $p_{ij}$ to ${\bf  s}_{i}$ is a decreasing
function of distance.  
We can prove this easily by drawing an analogy with a 1-d case such as
in distance sampling.  Let $y_{j}$ be the number of individuals
captured in some interval $[d_{j-1},d_{j})$, and define $d_{J} = B$
for some large value of $B$.  By choosing $B$ large enough we
guarantee that $E[y_{J+1}] = 0$ and therefore this ``last cell'' 
contributes nothing to
the likelihood
in regular situations in which the detection function decays
monotonically with distance and prior density is constant.  


Sometimes
our estimate of density can be influenced if we make ${\cal S}$ too small but
this might be sensible if ${\cal S}$ is naturally well-defined. As we discussed
in chapter 1, {\bf choice of ${\cal S}$ is part of the model and thus it makes
  sense that estimates of density might be sensitive to its definition
  in problems where it is natural to restrict ${\cal S}$}.
One could imagine
however that in specific cases where you're studying a small
population with well-defined habitat preferences that a problem could
arise because changing the state-space around based on differing
opinions and GIS layers really changes the estimate of total
population size. But this is a real biological problem and a natural
consequence of the spatial formalization of capture-recapture models -
a feature, not a bug or some statistical artifact - and it should be
resolved with better information, research, and thinking.
 For situations where there is not a natural
choice of ${\cal S}$, we should default to choosing ${\cal S}$ to be very large in order
to achieve invariance or otherwise evaluate sensitivity of density
estimates by trying a couple of different values of ${\cal S}$. This is a
standard ``sensitivity to prior'' argument that Bayesians always have
to be conscious of.  We demonstrate this in our analysis of section
\ref{scr0.sec.wolverine}
below. Note that $area({\cal S})$ affects data augmentation. If you
increase $area({\cal S})$ then there are more individuals to account for and
therefore the size of the augmented data set $M$ must increase.

We have been told that one can carry-out non-Bayesian analyses of SCR
models without having to specify the state-space of the point process
or perhaps while only specifying it imprecisely.  This assertion is
incorrect. We assume people are thinking this because {\it they} don't
have to specify it explicitly because someone else has done it for
them in a package that does integrated likelihood. Even to do
integrated likelihood (see Chapt. \ref{chapt.mle}) we have to integrate the
conditional-on-${\bf s}$ likelihood over some 2-dimensional space.  It might
work that the integration can be done from $-\infty$ to $+\infty$ but
that is a mathematical artifact of specific detection functions, and
an implicit definition of a state-space that doesn't make biological
sense, even though it may in fact be innocuous;


\subsection{Connection to Model  $M_h$}  \label{scr0.sec.scrmh}

SCR models are closely related to heterogeneity models. In SCR models,
heterogeneity in encounter probability is induced by both the effect
of distance in the model for detection probability and also from
specification of the state-space. Hence, the state-space  is an
explicit element of the model. 
To understand this, suppose we have a random
effect with some prior distribution:
\[
{\bf s} \sim \mbox{Unif}({\cal S})
\]
And $p({\bf s}) = p(y=1|{\bf s})$ is some function of ${\bf
  s}$. Therefore, for any specific $g(p)$ and ${\cal S}$ we can work
out what the implied heterogeneity model is for example, the mean,
variance or other moments of the population distribution of $p$ can be
evaluated by integrating $p({\bf s})$ over the state-space of ${\bf
  s}$.  We
show an illustration in Fig. \ref{scr0.fig.buffereffect} which
shows a histogram of $p$ for a hypothetical population of 100000
individuals on a state-space enclosing our $5 \times 5$ trap array
above, under the logistic model for distance. {\bf R} code is
provided in the {\bf R} package \mbox{\tt scrbook} to produce this analysis for the
logistic and half-normal models. The histogram shows the encounter
probability under buffers of 0.2, 0.5 and 1.0. We see the mass shifts
to the left as the buffer increases, implying more individuals
 with lower encounter probabilies, as their home range
centers increase in distance from the trap array.


\begin{figure}
\begin{center}
\includegraphics[width=5in]{Ch4/figs/buffereffect}
\end{center}
\caption{Implied population distribution of $p_{i}$ for a population
  of individuals as a function of the size of the state-space buffer
  around a trap array. The trap array is fixed and centered within a
  square state-space.}
\label{scr0.fig.buffereffect}
\end{figure}

Another way to understand this is by representing ${\cal S}$ as a set
of discrete points on a grid. In the coarsest possible case where
${\cal S}$ is a single arbitrary point, then every individual has
exactly the same $p$. As we increase the number of points in ${\cal
  S}$ then more distinct values of $p$ are possible. As such, when
${\cal S}$ is characterized by discrete points then SCR models are
precisely a type of finite-mixture model \citep{norris_pollock:1996,
  pledger:2000}, except, in the case of SCR models, we have some information about which
group an individual belong (i.e., where their activity center is), as
a result of their captures in traps.

This context suggests the problem raised by \citet{link:2003}. He
showed that in most practical situations $N$ may not be identifiable
across classes of mixture distributions which in the context of SCR
models is the pair $(g, {\cal S})$.  The difference, however, is that
we do obtain some direct information about ${\bf s}$ in SCR models and
therefore it may be reasonable to expect that
$N$ is identifiable across models characterized by $(g,{\cal
  S})$.

\subsection{Connection to Distance Sampling}

It is worth emphasizing that the basic SCR model is a binomial
encounter model in which distance is a covariate. As such, it is
striking similarity to a classical distance sampling model. Both have
distance as a covariate but in classical distance sampling problems
the focus is on the distance between the observer and the animal at an
instant in time, not the distance between a trap and an animal's home
range center. As a practical matter, in distance sampling, ``distance'' is {\it
  observed} for those individuals that appear in the
sample. Conversely, in SCR problems, it is only imperfectly observed
(we have partial information in the form of trap observations).
Clearly, it is preferable to observe distance if possible, but 
distance sampling requires field methods that
are often not practical in many situations, e.g. when surveying
tigers. Furthermore, SCR models allow us to relax many of the
assumption made in classical distance sampling, and SCR models allow
for estimates of quantities other than density, such as home range
size, and space usage (see Chapt. \ref{chapt.ecoldist}).


\section{Simulating SCR Data}

It is always useful to simulate data because it allows you to
understand the system that you're modeling and also calibrate your
understanding with the parameter values of the model. That is, you can
simulate data using different parameter values until you obtain data
that ``looks right'' based on your knowledge of the specific situation
that you're interested in. Here we provide a simple script to
illustrate how to simulate spatial encounter history data. In this
exercise we simulate data for 100 individuals and a 25 trap array laid
out in a $5 \times 5$ grid of unit spacing.  The specific encounter model is
the half-normal model given above and we used this code to simulate
data used in subsequent analyses.  The 100 activity centers were
simulated on a state-space defined by a $8 \times 8$ square within which the
trap array was centered (thus the trap array is buffered by 2
units). Therefore, the density of individuals in this system is fixed
at $100/64$.

{\small
\begin{verbatim}
	set.seed(2013)
# create 5 x 5 grid of trap locations with unit spacing
traplocs<- cbind(sort(rep(1:5,5)),rep(1:5,5))
Dmat<-e2dist(traplocs,traplocs) # in cases where speed doesn't matter, it might be
                                # clearer to just show the slow for-loop.
                                # Plus, people will want to copy/paste this stuff
ntraps<-nrow(traplocs)

# define state-space of point process. (i.e., where animals live).
# "delta" just adds a fixed buffer to the outer extent of the traps.
delta<-2
Xl<-min(traplocs[,1] - delta)
Xu<-max(traplocs[,1] + delta)
Yl<-min(traplocs[,2] - delta)
Yu<-max(traplocs[,2] + delta)

N<-100   # population size
K<- 20    # number nights of effort

sx<-runif(N,Xl,Xu)    # simulate activity centers
sy<-runif(N,Yl,Yu)
S<-cbind(sx,sy)
D<- e2dist(S,traplocs)  # distance of each individual from each trap

alpha0<- -2.5      # define parameters of encounter probability
sigma<- 0.5        #
alpha1<- 1/(2*sigma*sigma)
probcap<- expit(-2.5)*exp( - alpha1*D*D)    # probability of encounter
# now generate the encounters of every individual in every trap
Y<-matrix(NA,nrow=N,ncol=ntraps)
for(i in 1:nrow(Y)){
   Y[i,]<-rbinom(ntraps,K,probcap[i,])
}
\end{verbatim}
}


Subsequently we will generate data using this code packaged in an {\bf
  R}
function called \mbox{\tt simSCR0.fn} in the package \mbox{\tt
  scrbook} which takes a number of
arguments including \mbox{\tt discard0} which, if \mbox{\tt TRUE}, will return
only the encounter histories for captured individuals.  A second
argument is \mbox{\tt array3d} which, if \mbox{\tt TRUE}, returns the 3-d
encounter history array instead of the aggregated \mbox{\tt nind}
$\times \mbox{\tt ntraps}$ encounter frequencies (see below). Finally
we provide a random number seed, \mbox{\tt sd = 2013} to ensure
repeatability of the analysis here. We obtain a data set as above using the
following command:
\begin{verbatim}
data<-simSCR0.fn(discard0=TRUE,array3d=FALSE,sd=2013)
\end{verbatim}
The {\bf R} object \mbox{\tt data} is a list, so let's take a look at
what's in the list and then harvest some of its elements for further
analysis below.
\begin{verbatim}
> names(data)
[1] "Y"        "traplocs" "xlim"     "ylim"     "N"        "alpha0"   "beta"
[8] "sigma"    "K"
> Y<-data$Y
> traplocs<-data$traplocs
\end{verbatim}


\subsection{Formatting and manipulating real data sets}
\label{scr0.sec.formats}

Conventional capture-recapture data are easily stored and manipulated
as a 2-dimensional array, an $\mbox{\tt nind} \times \mbox{\tt
  nperiod}$ matrix, which is maximally informative for any
conventional capture-recapture model, but not for spatial
capture-recapture models.  For SCR models we must preserve the spatial
information in the encounter history information. We will routinely
analyze data from 3 standard formats:
\begin{itemize}
\item[(1)] The basic 2-dimensional data format, which is an \mbox{\tt
    nind} $\times$ \mbox{\tt ntraps} encounter frequency matrix such
  as that simulated previously; These are the total encounters in each
  trap, summed over replicate samples.
\item[(2)] The maximally informative 3-dimensional array which we
  establish here the convention that it has dimensions \mbox{\tt nind}
  $\times$ \mbox{\tt nperiods} $\times$ \mbox{\tt ntraps} and
\item[(3)] We use a compact format - the ``SCR flat format'' - which
  we describe below in section \ref{scr0.sec.wolverine}.
\end{itemize}
To simulate data in the most informative format - the ``3-d array'' -
we can use the {\bf R} commands given previously but replace the last
4 lines with the following:
{\small
\begin{verbatim}
Y<-array(NA,dim=c(N,K,ntraps))
for(i in 1:nrow(Y)){
for(j in 1:ntraps){
 Y[i,1:K,j]<-rbinom(K,1,probcap[i,j])
}
}
\end{verbatim}
}
We see that a collection of $K$ binary encounter events are generated
for {\it each} individual and for {\it each} trap.  The probabilities
have those Bernoulli trials are computed based on the distance from
each individuals home range center and the trap (see calculation
above), and those are housed in the matrix \mbox{\tt probcap}. Our data simulator
function \mbox{\tt simSRC0.fn} will return the full 3-d array if
\mbox{\tt array3d=TRUE} is specified in the function call.  To recover
the 2-d matrix from the 3-d array, and subset the 3-d array to
individuals that were captured, we do this:
{\small
\begin{verbatim}
Y2d<- apply(Y,c(1,3),sum) # sum over the ``replicates'' dimension (2nd margin of the array)
ncaps<-apply(Y2d,1,sum)   # compute how many times each individual was captured
Y<-Y[ncaps>0,,]           # keep those individuals that were captured
\end{verbatim}
}

\section{Fitting an SCR Model in BUGS}
\label{scr0.sec.winbugs1}

Clearly if we somehow knew the value of $N$ then we could fit this
model directly because, in that case, it is a special kind of logistic
regression model - one with a random effect, but that enters into the
model in a peculiar fashion - and also with a distribution (uniform)
which we don't usually think of as standard for random effects models.
So our aim here is to analyze the known-$N$ problem, using our
simulated data, as an incremental step in our progress toward fitting
more generally useful models.

To begin, we use our simulator to grab a data set and then harvest the
elements of the resulting object for further analysis.
\begin{verbatim}
data<-simSCR0.fn(discard0=FALSE,sd=2013)
y<-data$Y
traplocs<-data$traplocs
nind<-nrow(y)
X<-data$traplocs
J<-nrow(X)
y<-rbind(y,matrix(0,nrow=(100-nrow(y)),ncol=J ) )
Xl<-data$xlim[1]
Yl<-data$ylim[1]
Xu<-data$xlim[2]
Yu<-data$ylim[2]
\end{verbatim}

Note that we specify \mbox{\tt discard0 = FALSE} so that we have a
"complete" data set, i.e., one with the all-zero encounter histories
corresponding to uncaptured individuals. Now, within an {\bf R} session, we
can create the {\bf BUGS} model file and fit the model using the following
commands. 
{\small
\begin{verbatim}
cat("
model {
alpha0~dnorm(0,.1)
logit(p0)<- alpha0
alpha1~dnorm(0,.1)
for(i in 1:N){
 s[i,1]~dunif(Xl,Xu)
 s[i,2]~dunif(Yl,Yu)
for(j in 1:J){
d[i,j]<- pow(pow(s[i,1]-X[j,1],2) + pow(s[i,2]-X[j,2],2),0.5)
y[i,j] ~ dbin(p[i,j],K)
p[i,j]<- p0*exp(- alpha1*d[i,j]*d[i,j])
}
}

}
",file = "SCR0a.txt")
\end{verbatim}
}
This model describes the half-normal detection model but it
would be trivial to modify that to various others including the
logistic described above. One consequence of using the half-normal is
that we have to constrain the encounter probability to be in $[0,1]$
which we do here by defining \mbox{\tt alpha0} to be the logit of the
intercept parameter \mbox{\tt p0}. Note that the distance covariate is
computed within the {\bf BUGS} model specification given the matrix of trap
locations, \mbox{\tt X}, which is provided to {\bf WinBUGS} as data.

Next we do a number of organizational activities including bundling
the data for {\bf WinBUGS}, defining some initial values, the parameters to
monitor and some basic MCMC settings.  We choose initial values for
the activity centers ${\bf s}$ by generating uniform random numbers in
the state-space but, for the observed individuals, we replace those
values by each individual's mean trap coordinate for all encounters
{\small
\begin{verbatim}
sst<-cbind(runif(nind,Xl,Xu),runif(nind,Yl,Yu))  # starting values for s
for(i in 1:nind){
if(sum(y[i,])==0) next
sst[i,1]<- mean( X[y[i,]>0,1] )
sst[i,2]<- mean( X[y[i,]>0,2] )
}

data <- list (y=y,X=X,K=K,N=nind,J=J,Xl=Xl,Yl=Yl,Xu=Xu,Yu=Yu)
inits <- function(){
  list (alpha0=rnorm(1,-4,.4),alpha1=runif(1,1,2),s=sst)
}

library("R2WinBUGS")
parameters <- c("alpha0","alpha1")
nthin<-1
nc<-3
nb<-1000
ni<-2000
out <- bugs (data, inits, parameters, "SCR0a.txt", n.thin=nthin,
n.chains=nc, n.burnin=nb,n.iter=ni,debug=TRUE,working.dir=getwd())
\end{verbatim}
}
There is little to say about the preceding basic operations other than
to suggest that the interested reader explore the output and
additional analyses by running the script provided in the {\bf R}
package \mbox{\tt scrbook}.
 We ran $1000$ burn-in and $1000$ after burn-in, 3 chains,
to obtain 3000 posterior samples.  Because we know $N$ for this
particular data set we only have 2 parameters of the detection model
to summarize (\mbox{\tt alpha0} and \mbox{\tt alpha1}).  When the
object \mbox{\tt out} is produced we print a summary of the results as
follows:
{\small
\begin{verbatim}
> print(out,digits=3)
Inference for Bugs model at "SCR0a.txt", fit using WinBUGS,
 3 chains, each with 2000 iterations (first 1000 discarded)
 n.sims = 3000 iterations saved
            mean     sd    2.5%     25%    50%     75%   97.5%  Rhat n.eff
alpha0    -2.496  0.224  -2.954  -2.648  -2.48  -2.340  -2.091 1.013   190
alpha1     2.442  0.419   1.638   2.145   2.44   2.721   3.303 1.005   530
deviance 292.803 21.155 255.597 277.500 291.90 306.000 339.302 1.006   380

For each parameter, n.eff is a crude measure of effective sample size,
and Rhat is the potential scale reduction factor (at convergence, Rhat=1).

DIC info (using the rule, pD = Dbar-Dhat)
pD = -138.8 and DIC = 154.0
DIC is an estimate of expected predictive error (lower deviance is better).
\end{verbatim}
}

We know the data were generated with \mbox{\tt alpha0} $= -2.5$ and
\mbox{\tt alpha1 = -2}. The estimates look reasonably close to those
data-generating values and we probably feel pretty good about the
performance of the Bayesian analysis and MCMC algorithm that {\bf WinBUGS}
cooked-up based on our sample size of 1 data set.  It is worth noting
that the Rhat statistics indicate reasonable convergence but, as a
practical matter, we might choose to run the MCMC algorithm for
additional time to bring these closer to 1.0 and to increase the
effective posterior sample size (\mbox{\tt n.eff}). Other summary output includes
``deviance'' and related things including the deviance information
criterion (DIC). We discuss these things in Chapts. \ref{chapt.mcmc}
and \ref{chapt.gof}.



\section{Unknown N}
\label{scr0.sec.unknownN}

In all real applications $N$ is unknown and that fact is kind of an
important feature of the capture-recapture problem!  We handled this
important issue in Chapt. \ref{chapt.closed} using the method of data augmentation
which we apply here to achieve a realistic analysis of model SCR0. As
with the basic closed population models considered previously, we
formulate the problem here by augmenting our observed data set with a
number of ``all zero'' encounter histories - what we referred to in
Chapt. \ref{chapt.closed} as potential individuals. If $n$ is the number of observed
individuals, then let $M-n$ be the number of potential individuals in
the data set. For the basic $y_{ij}$ data structure (individuals x
traps encounter frequencies) we simply add additional rows of ``all
0'' observations to that data set. This is because such
``individuals'' are unobserved, and therefore necessarily have
$y_{ij}=0$ for all $j$.  A data set, say with 4 traps and 6 individuals,
augmented with 4 pseudo-individuals therefore might look like this:
{\small
\begin{verbatim}
      trap1 trap2 trap3 trap4
 [1,]     1     0     0     0
 [2,]     0     2     0     0
 [3,]     0     0     0     1
 [4,]     0     1     0     0
 [5,]     0     0     1     1
 [6,]     1     0     1     0
 [7,]     0     0     0     0
 [8,]     0     0     0     0
 [9,]     0     0     0     0
[10,]     0     0     0     0
\end{verbatim}
}
We typically have more than 4 traps and, if we're fortunate, many more
individuals in our data set.

For the augmented data, we introduce a set of binary latent variables
(the data augmentation variables), $z_{i}$, and the model is extended
to describe $\Pr(z_{i} = 1)$ which is, in the context of this problem,
the probability that an individual in the augmented data set is a
member of the population that was sampled. In other words, if $z_{i}=1$
for one of the ``all zero'' encounter histories, this is implied to be
a sampling zero whereas observations for which $z_{i}=0$ are
``structural zeros'' under the model.

How big does the augmented data set have to be? We discussed this
issue in Chapt. \ref{chapt.closed} where we noted that the size of the data set is
equivalent to the upper limit of a uniform prior distribution on $N$.
Practically speaking, it should be sufficiently large so that the
posterior distribution for $N$ is not truncated. On the other hand, if
it is too large then unnecessary calculations are being done. An
approach to choosing $M$ by trial-and-error is indicated. You can take
a ballpark estimate of the probability that an individual is captured
at all during the study, say $\tilde{p}$, which is related to the
``per sample'' encounter probability, $p$, by $\tilde{p} = 1-(1-p)^{K}$, obtain $N$ as $n/\tilde{p}$, and then set $M =
2*N$, as a first guess. Do a short MCMC run and then consider whether
you need to increase $M$. See Chapt. \ref{chapt.mcmc} for an
example of this. \citet[][ch. 6]{kery_schaub:2011}
 provide an assessment of choosing $M$ in closed population models.

Analysis by data augmentation removes $N$ as an explicit parameter of
the model. Instead, $N$ is a derived parameter, computed by $N=
\sum_{i=1}^{M} z_{i}$. Similarly, {\it density}, $D$, is also a
derived parameter computed as $D=N/area({\cal S})$. For our
simulator, we're using an $8 \times 8$ state-space and thus we will
compute $D$ as $D=N/64$.

\subsection{Analysis using data augmentation in WinBUGS}

As before we begin by obtaining a data set using our \mbox{\tt
  simSCR0.fn} routine and then harvesting the required data objects
from the resulting data list.  Note that we use the \mbox{\tt
  discard0=TRUE} option this time so that we get a ``real'' data set
with no all-zero encounter histories. After harvesting the data we
produce the {\bf WinBUGS} model specification which now includes $M$
encounter histories including the augmented potential individuals, the
data augmentation parameters $z_{i}$, and the data augmentation
parameter $\psi$.
{\small
\begin{verbatim}
data<-simSCR0.fn(discard0=TRUE,sd=2013)
y<-data$Y
traplocs<-data$traplocs
nind<-nrow(y)
X<-data$traplocs
J<-nrow(X)
Xl<-data$xlim[1]
Yl<-data$ylim[1]
Xu<-data$xlim[2]
Yu<-data$ylim[2]

cat("
model {
alpha0~dnorm(0,.1)
logit(p0)<- alpha0
alpha1~dnorm(0,.1)
psi~dunif(0,1)

for(i in 1:M){
 z[i] ~ dbern(psi)
 s[i,1]~dunif(Xl,Xu)
 s[i,2]~dunif(Yl,Yu)
for(j in 1:J){
d[i,j]<- pow(pow(s[i,1]-X[j,1],2) + pow(s[i,2]-X[j,2],2),0.5)
y[i,j] ~ dbin(p[i,j],K)
p[i,j]<- z[i]*p0*exp(- alpha1*d[i,j]*d[i,j])
}
}
N<-sum(z[])
D<-N/64
}
",file = "SCR0a.txt")
\end{verbatim}
}

To prepare our data we have to augment the data matrix \mbox{\tt y}
with $M-n$ all-zero encounter histories, we have to create starting
values for the variables $z_{i}$ and also the activity centers ${\bf
  s}_{i}$ of which, for each, we require $M$ values. Otherwise the
remainder of the code for bundling the data, creating initial values
and executing {\bf WinBUGS} looks much the same as before except with more
or differently named arguments.
{\small
\begin{verbatim}
## Data augmentation stuff
M<-200
y<-rbind(y,matrix(0,nrow=M-nind,ncol=ncol(y)))
z<-c(rep(1,nind),rep(0,M-nind))

sst<-cbind(runif(M,Xl,Xu),runif(M,Yl,Yu))  # starting values for s
for(i in 1:nind){
if(sum(y[i,])==0) next
sst[i,1]<- mean( X[y[i,]>0,1] )
sst[i,2]<- mean( X[y[i,]>0,2] )
}
data <- list (y=y,X=X,K=K,M=M,J=J,Xl=Xl,Yl=Yl,Xu=Xu,Yu=Yu)
inits <- function(){
  list (alpha0=rnorm(1,-4,.4),alpha1=runif(1,1,2),s=sst,z=z)
}

library("R2WinBUGS")
parameters <- c("alpha0","alpha1","N")
nthin<-1
nc<-3
nb<-1000
ni<-2000
out <- bugs (data, inits, parameters, "SCR0a.txt", n.thin=nthin,n.chains=nc,
 n.burnin=nb,n.iter=ni,debug=TRUE,working.dir=getwd())
\end{verbatim}
}

{\bf Remarks}:  (1) Note the differences in this new {\bf WinBUGS} model
with that appearing in the known-$N$ version.  (2) Also the input data
has changed - the augmented data set has more rows of
all-zeros. Previously we knew that $N=100$ but in this analysis we
pretend not to know $N$, but think that $N=200$ is a good upper-bound;
(3) Population size $N({\cal S})$ is a derived parameter, being computed by
summing up all of the data augmentation variables $z_{i}$ (as we've
done previously in Chapt. \ref{chapt.closed}); (4) Density, $D\equiv D({\cal S})$, is also a derived
parameter. Summarizing the output from {\bf WinBUGS} produces:
{\small
\begin{verbatim}
> print(out1,digits=2)
Inference for Bugs model at "SCR0a.txt", fit using WinBUGS,
 3 chains, each with 2000 iterations (first 1000 discarded)
 n.sims = 3000 iterations saved
           mean    sd   2.5%    25%    50%    75%  97.5% Rhat n.eff
alpha0    -2.57  0.23  -3.04  -2.72  -2.56  -2.41  -2.15 1.01   320
alpha1     2.46  0.42   1.63   2.16   2.46   2.73   3.33 1.02   120
N        113.62 15.73  86.00 102.00 113.00 124.00 147.00 1.01   260
D          1.78  0.25   1.34   1.59   1.77   1.94   2.30 1.01   260
deviance 302.60 23.67 261.19 285.47 301.50 317.90 354.91 1.00  1400

For each parameter, n.eff is a crude measure of effective sample size,
and Rhat is the potential scale reduction factor (at convergence, Rhat=1).

DIC info (using the rule, pD = var(deviance)/2)
pD = 279.9 and DIC = 582.5
DIC is an estimate of expected predictive error (lower deviance is better).
\end{verbatim}
}

The column labeled ``MC error'' is the Monte Carlo error - the error
inherent in the attempt to compute these posterior summaries by
MCMC
(see secs.  for discussion of this quantity
\ref{glms.sec.convergence} \ref{mcmc.sec.mcmcsummary}).
It is desirable to run the Markov chain algorithm long enough so
as to reduce the MC error to a tolerable level. What constitutes
tolerable is up to the investigator. Certainly less than 1\% is called
for. As a general rule, Rhat gets closer to 1 and MC error decreases
toward 0 as the number of iterations increases.  We see that the
estimated parameters ($\alpha_0$ and $\alpha1$) are comparable to the
previous results obtained for the known-$N$ case, and also not too
different from the data-generating values. The posterior of $N$
overlaps the data-generating value substantially with a mean of
$113.62$.  To obtain these results we fitted the true data-generating
model, that based on the half-normal detection model, to a single
simulated data set. For fun and excitement we fit the {\it wrong}
model, one with a logistic-linear detection model
(Eq. \ref{scr0.eq.logit}),
to the same  
data set. This is easily achieved by modifying the {\bf WinBUGS} model
specification above, although we provide the {\bf R} script in the
{\bf R} package \mbox{\tt scrbook}.
Those results are given below. We see that the estimate of
$N$, the main parameter of interest, is very similar to that obtained
under the correct model, convergence is worse (as measured by Rhat)
which may not have anything to do with the model being wrong,
and the posterior deviance favors the correct model (it is smaller) while the DIC does not.
We consider 
 the effectiveness of DIC for carrying-out model selection in chapter
\ref{chapt.gof}.
{\small
\begin{verbatim}
> print(out2,digits=2)
Inference for Bugs model at "SCR0a.txt", fit using WinBUGS,
 3 chains, each with 2000 iterations (first 1000 discarded)
 n.sims = 3000 iterations saved
           mean    sd   2.5%    25%    50%    75%  97.5% Rhat n.eff
alpha0    -1.59  0.27  -2.16  -1.77  -1.58  -1.42  -1.07 1.05    60
beta       3.77  0.43   2.92   3.48   3.79   4.05   4.66 1.04    70
N        122.57 18.67  90.00 109.00 122.00 135.00 163.00 1.00  3000
D          1.92  0.29   1.41   1.70   1.91   2.11   2.55 1.00  3000
deviance 312.67 22.43 271.00 297.20 311.50 327.00 359.60 1.02   130

For each parameter, n.eff is a crude measure of effective sample size,
and Rhat is the potential scale reduction factor (at convergence, Rhat=1).

DIC info (using the rule, pD = var(deviance)/2)
pD = 247.5 and DIC = 560.1
DIC is an estimate of expected predictive error (lower deviance is better).
\end{verbatim}
}

\subsection{Use of other BUGS engines: JAGS}

There are two other popular {\bf BUGS} engines in widespread use: {\bf
  OpenBUGS} \citep{thomas_etal:2006} and {\bf JAGS}
\citep{plummer:2003}. Both of these are easily called from {\bf
  R}. {\bf OpenBUGS} can be used instead of {\bf WinBUGS} by changing
the package option in the \mbox{\tt bugs} call to \mbox{\tt
  package=OpenBUGS}.  {\bf JAGS} can be called using the function
\mbox{\tt jags()} in package \mbox{\tt R2JAGS} which has nearly the
same arguments as \mbox{\tt bugs()}.  We prefer to use the {\bf R}
library \mbox{\tt rjags} \citep{plummer:2009} which has a slightly
different implementation that we demonstrate here as we reanalyze the
simulated data set in the previous section (note: the same {\bf R}
commands are used to generate the data and package the data, inits and
parameters to monitor). The function \mbox{\tt jags.model} is used to
initialize the model and run the MCMC algorithm for an adaptive
burn-in period.  Then the Markov chains are updated using \mbox{\tt
  coda.samples()} to obtain posterior samples for analysis, as
follows:
\begin{verbatim}
jm<- jags.model("SCR0a.txt", data=data, inits=inits, n.chains=nc,
                 n.adapt=nb))
jm<- coda.samples(jm, parameters, n.iter=ni-nb, thin=nthin)
\end{verbatim}
We find that {\bf JAGS} seems to be 20-30\% faster for the basic SCR
model which the reader can evaluate using the function \mbox{\tt
  SCR0bayes} in the {\bf R} package \mbox{\tt scrbook}.



\section{Wolverine Camera Trapping Study}
\label{scr0.sec.wolverine}

We provide an analysis here of A. Magoun's wolverine data
\citep{magoun_etal:2011, royle_etal:2011jwm}. The study took place in SE
Alaska (Fig. \ref{scr0.fig.wolverinelocs}) where 37 cameras were
operational for variable periods of time (min = 5 days, max = 108
days, median = 45 days).  A consequence of this is that the binomial
sample size $K$ (see Eq. \ref{scr0.eq.bin})
 is variable for each camera. Thus, we
must provide a matrix of sample sizes as data to {\bf BUGS} and modify the
model specification in sec. \ref{scr0.sec.unknownN}
accordingly. Our treatment of the
data here is based on the analysis of  \citet{royle_etal:2011jwm}.

\begin{figure}
\begin{center}
\includegraphics[height=3in]{Ch4/figs/wolverinelocs}
\end{center}
\caption{Wolverine camera trap locations from \citet{magoun_etal:2011}.}
\label{scr0.fig.wolverinelocs}
\end{figure}

To carry-out an analysis of these data, we require the matrix of trap
coordinates and the encounter history data.  We store data in an the
``scr flat format'' (see sec.  \ref{scr0.sec.formats} above), an
efficient file format which is easily manipulated and also used as the
input file format in {\bf SPACECAP} \citep{gopalaswamy_etal:2012} and
in the {\bf R} package \mbox{\tt SCRbayes} \citep{russell_etal:2012}.
To illustrate this format, the wolverine data are available in the
package \mbox{\tt scrbook} by typing:
\begin{verbatim}
data(wolverine)
\end{verbatim}
which contains a list having elements \mbox{\tt wcaps} and
\mbox{\tt wtraps}.
The ``encounter data file''
\mbox{\tt wcaps}  has 3 columns and 115 rows, each representing a
unique encounter event including the trap identity, the individual
identity and the sample occasion index (\mbox{\tt sample}).
The first 10 rows of this matrix are as
follows:
{\small
\begin{verbatim}
> wolverine$wcaps[1:10,]
       trapid individual sample
  [1,]      1          2    127
  [2,]      1          2    128
  [3,]      1          2    129
  [4,]      1         18    130
  [5,]      2          3    106
  [6,]      2         18    104
  [7,]      5          5     73
  [8,]      5          5     89
  [9,]      6         18    117
 [10,]      6         18    118
\end{verbatim}
}
Each row is a unique 
individual/trap encounter, and the 3 variables (columns) are: 
\mbox{\tt trapid} -- an
integer that runs from \mbox{\tt 1:ntraps}, \mbox{\tt individual} runs from
\mbox{\tt 1:nind} and
\mbox{\tt sample} 
runs from \mbox{\tt 1:nperiods}. Often (as the case here) \mbox{\tt
  sample} 
will
correspond to daily sample intervals. The variable \mbox{\tt trapid} will have to
correspond to the row of a matrix containing the trap coordinates - in
this case the file \mbox{\tt wtraps} which we describe further below.

Note that the information provided in this encounter data file
\mbox{\tt wcaps}
does not represent a completely informative summary
of the data. For example, if no individuals were captured in a certain
trap or during a certain period, then this compact data format will
have no record. Thus we will need to know ntraps and nperiods when
reformatting this SCR data format into a 2-d encounter frequency
matrix or 3-d array. In addition, the encounter data file does not
provide information about which periods each trap was operated. This
additional information is also necessary as the trap-specific sample
sizes must be passed to {\bf BUGS} as data. We provide this information in a
2nd data file, along with the trap coordinates, in the 
 ``trap deployment'' file which is described
below.

For our purposes we
need to convert the \mbox{\tt wcaps} file
into the $n \times J$ array of
binomial encounter frequencies, although more general models might
require an encounter-history formulation of the model which requires a
full 3-d array.  To obtain our $n \times J$ encounter frequency
matrix, we do this the hard way by first converting the encounter data
file into a 3-d array and then summarize to trap totals. We have a
handy function \mbox{\tt SCR23darray.fn} which takes the compact
encounter data file with optional arguments ntraps and nperiods, and
converts it to a 3-d array, and then we use the {\bf R} function
\mbox{\tt apply} to summarize over the ``sample'' period dimension (by
convention here, this is the 2nd dimension). To apply this to the
wolverine
data in order to compuate the 3-d array we do this:
{\small
\begin{verbatim}
y3d <-SCR23darray.fn(wolverine$wcaps,wolverine$wtraps)
y <- apply(y3d,c(1,3),sum)
\end{verbatim}
}
See the help file for more information on \mbox{\tt SCR23darray.fn}.
The 3-d array is necessary to fit certain types
of models (e.g., behavioral response) and this is why we sometimes
will require this maximally informative 3-d data format but, here, we
analyze the summarized data.



The other important information needed to fit SCR models is the
``trap deployment'' file
which provides the additional information
not contained in the encounter data file. The traps file has \mbox{\tt
  nperiods} $+ 3$ columns. The first column is assumed to be a trap identifier,
columns 2 and 3 are the easting and northing coordinates (assumed to
be in a Euclidean coordinate system), and columns 4 to (\mbox{\tt nperiods} + 3)
are binary indicators of whether each trap was operational in each
time period. The first 10 rows (out of 37) and 10 columns (out of 167)
of the trap deployment file for the wolverine data are:
{\small
\begin{verbatim}
> wolverine$wtraps[1:10,1:10]

   Easting Northing 1 2 3 4 5 6 7 8 
1   632538  6316012 0 0 0 0 0 0 0 0
2   634822  6316568 1 1 1 1 1 1 1 1
3   638455  6309781 0 0 0 0 0 0 0 0
4   634649  6320016 0 0 0 0 0 0 0 0
5   637738  6313994 0 0 0 0 0 0 0 0
6   625278  6318386 0 0 0 0 0 0 0 0
7   631690  6325157 0 0 0 0 0 0 0 0
8   632631  6316609 0 0 0 0 0 0 0 0
9   631374  6331273 0 0 0 0 0 0 0 0
10  634068  6328575 0 0 0 0 0 0 0 0
\end{verbatim}
}
This tells us that trap 2 was operated in periods (days) 1-7 but the other
traps were not operational during those periods. It is extremely
important to recognize that each trap was operated for a variable
period of time and thus the binomial "sample size" is different for
each, and this needs to be accounted for in the {\bf BUGS} model specification.
To compute the vector of sample sizes $K$, and extract the trap
locations,  we do this:
\begin{verbatim}
traps<- wolverine$wtraps
traplocs<- traps[,1:2]
K<- apply(traps[,3:ncol(traps)],1,sum)
\end{verbatim}
This results in a matrix traplocs which contains the coordinates of
each trap and a vector $K$ containing the number of days that each trap
was operational. We now have all the information required to fit a
basic SCR model in {\bf BUGS}.

Summarizing these data files for the wolverine study, we see that 21
unique individuals were captured a total of 115 times. Most
individuals were captured 1-6 times, with 4, 1, 4, 3, 1, and 2
individuals captured 1-6 times, respectively.  In addition, 1
individual was captured each 8 and 14 times and 2 individuals each
were captured 10 and 13 times.  The number of unique traps that
captured a particular individual ranged from 1-6, with 5, 10, 3, 1, 1,
and 1 individual captured in each of 1-6 traps, respectively, for a
total of 50 unique wolverine-trap encounters.  These numbers might be
hard to get your mind around whereas some tabular summary is often
more convenient. For that it seems natural to tabulate individuals by
trap and total encounter frequencies. The spatial information in SCR
data is based on multi-trap captures\footnote{I will add more 
context here on revision about spatial recaptures, lost recaptures,
ordinary recaptures. Function \mbox{\tt SCRsmy} in \mbox{\tt
  scrbook}}, 
and so, it is informative to
understand how many unique traps each individual is captured in. At
the same, it is useful to understand how many total captures we have
of each individual because this is, in an intuitive sense, the
effective sample size.  So, we reproduce Table 1 from
\citet{royle_etal:2011jwm} which shows the trap and total encounter
frequencies:

\begin{table} [htp]
  \caption{Individual frequencies of capture for wolverines captured
    in camera traps in Southeast Alaska in 2008. Rows index unique
    trap frequencies and columns represent total number of captures
    (e.g., we captured 4 individuals 1 time, necessarily in only 1
    trap; we captured 3 individuals 3 times but in 2 different traps)}
\centering
\begin{tabular}{c c c c c c c c c c c}
\hline
 & & & & & & & &  No.&of&captures \\
\hline
No. of traps & 1 & 2 & 3 & 4 & 5 & 6 & 8 & 10 &13 &14 \\
\hline
1 & 4 & 1 & 0 & 0 & 0 & 0 & 0 & 0 & 0 & 0 \\
2 & 0 & 0 & 3 & 3 & 0 & 2 & 1 & 2 & 0 & 0 \\
3 & 0 & 0 & 1 & 1 & 0 & 0 & 0 & 0 & 0 & 1 \\
4 & 0 & 0 & 0 & 0 & 0 & 0 & 0 & 0 & 1 & 0 \\
5 & 0 & 0 & 0 & 0 & 1 & 0 & 0 & 0 & 0 & 0 \\
6 & 0 & 0 & 0 & 0 & 0 & 0 & 0 & 0 & 1 & 0 \\
\hline
\end{tabular}
\end{table}

\subsection{Fitting the model in WinBUGS}

For illustrative purposes here we fit the simplest SCR model with the
half-normal distance function although we revisit these data with more
complex models in later chapters. The model is summarized by the
following 3 components:
\begin{itemize}
\item[(1)] $y_{ij}|{\bf s}_{i} \sim \mbox{Bin}(K, z_{i}\; p_{ij})$
\item[(2)] $p_{ij} = p_{0} \exp(-\alpha1 \; ||{\bf s}_{i}-x_{j}||^2)$
\item[(3)] $ {\bf s}_{i} \sim \mbox{Unif}({\cal S})$
\item[(4)] $ z_{i} \sim \mbox{Bern}(\psi)$
\end{itemize}
We assume customary flat priors on the structural (hyper-) parameters
of the model, $\alpha_{0} = \mbox{logit}(p_{0})$, $\alpha1$ and $\psi$.  It remains to define the
state-space ${\cal S}$. For this, we nested the trap array (Fig.
\ref{scr0.fig.wolverinelocs}) in a
a rectangular state-space extending $20$ km beyond the traps in each cardinal
direction.  We also considered larger state-spaces up to 50 km to
evaluate that choice.  The buffer of the state space should be larger
enough so that individuals beyond the state-space boundary are not
likely to be encountered. Thus some knowledge of typical space usage
patterns of the species is useful.  For the analysis, 
we scaled the coordinate system 
so that a unit distance was equal to $10$ km, producing a rectangular
state-space of dimension $9.88 \times 10.5$ units ($area = 10374$ km$^2$)
within which the trap array was nested. As a general rule, we
recommend scaling the state-space so that it is defined near the
origin $(x,y)=(0,0)$. While the scaling of the coordinate system is
theoretically irrelevant, a poorly scaled coordinate system can
produce Markov chains that mix poorly.  For the scaled coordinate
system we fit models for various choices of a rectangular state-space
based on 
buffers from 1.0 to 5.0 units on the scaled coordinate system (10 km to
50 km). In the {\bf R} package \mbox{\tt scrbook} we provide a
function
\mbox{\tt wolvSCR0.fn} which will fit the basic SCR model. For
example, to fit the model in 
{\bf WinBUGS} using data augmentation with $M=300$ potential individuals,
using 3 Markov chains each of 12000 total iterations, discarding the
first 2000 as burn-in, we execute the following {\bf R} commands:
{\small
\begin{verbatim}
library("scrbook")
data(wolverine)
traps<-wolverine$wtraps
y3d <-SCR23darray.fn(wolverine$wcaps,wolverine$wtraps)
toad<-wolvSCR0.fn(y3d,traps,nb=12000,ni=2000,delta=1,M=300)
\end{verbatim}
}
The argument $\delta$ determines the buffer size of the state-space.
Note that this analysis takes 
between 1-2 hours on many machines so we recommend trying it out with
lower values of $M$ and fewer iterations.
The output
follows (note, we have a parameter ``sigma'' which we discuss
shortly)\footnote{Final as of 1/11/2012. 
output saved in \mbox{\tt wolv-buffer-study.txt}}:

{\small
\begin{verbatim}
All based on 3 chains, 12k iters, 2k burn, 30k total
Buffer = 10 km
           mean    sd   2.5%    25%    50%    75%  97.5% Rhat n.eff
psi        0.13  0.03   0.08   0.11   0.13   0.15   0.20    1 10000
sigma      0.65  0.06   0.55   0.61   0.64   0.68   0.76    1  1800
p0         0.06  0.01   0.04   0.05   0.06   0.06   0.08    1 20000
N         39.63  6.70  29.00  35.00  39.00  44.00  54.00    1  7100
D          5.92  1.00   4.33   5.22   5.82   6.57   8.06    1  7100
beta       1.23  0.21   0.85   1.08   1.22   1.36   1.66    1  1800
deviance 410.05 12.06 388.70 401.50 409.20 417.80 435.60    1 22000

Buffer = 15 km
 n.sims = 30000 iterations saved
           mean    sd   2.5%    25%    50%    75%  97.5% Rhat n.eff
psi        0.16  0.04   0.10   0.14   0.16   0.19   0.25    1  3800
sigma      0.64  0.06   0.54   0.60   0.64   0.67   0.76    1   510
p0         0.06  0.01   0.04   0.05   0.06   0.06   0.08    1 17000
N         48.77  9.19  34.00  42.00  48.00  54.00  69.00    1  3300
D          5.78  1.09   4.03   4.98   5.69   6.40   8.18    1  3300
beta       1.25  0.21   0.86   1.10   1.24   1.39   1.70    1   510
deviance 411.00 12.16 389.50 402.40 410.30 418.70 437.00    1  5400

Buffer = 20 km
           mean    sd   2.5%    25%    50%    75%  97.5% Rhat n.eff
psi        0.20  0.05   0.12   0.17   0.20   0.23   0.30    1 16000
sigma      0.64  0.06   0.54   0.60   0.63   0.67   0.76    1  1200
p0         0.06  0.01   0.04   0.05   0.06   0.06   0.08    1  1900
N         59.84 11.89  40.00  51.00  59.00  67.00  86.00    1 20000
D          5.77  1.15   3.86   4.92   5.69   6.46   8.29    1 20000
beta       1.26  0.21   0.87   1.11   1.25   1.40   1.71    1  1200
deviance 411.01 12.36 389.10 402.30 410.20 418.80 437.50    1  1500

Buffer = 25 km
           mean    sd   2.5%    25%    50%    75%  97.5% Rhat n.eff
psi        0.24  0.05   0.15   0.20   0.24   0.28   0.36    1  3400
sigma      0.64  0.05   0.54   0.60   0.63   0.67   0.75    1  3600
p0         0.06  0.01   0.04   0.05   0.06   0.06   0.08    1  5000
N         72.40 14.72  47.00  62.00  71.00  81.00 105.00    1  2700
D          5.79  1.18   3.76   4.96   5.67   6.47   8.39    1  2700
beta       1.26  0.21   0.88   1.12   1.25   1.40   1.71    1  3600
deviance 411.35 12.23 389.70 402.70 410.55 419.20 437.20    1 30000

Buffer = 30 km
           mean    sd   2.5%    25%    50%    75%  97.5% Rhat n.eff
psi        0.29  0.06   0.18   0.24   0.28   0.33   0.43    1  3100
sigma      0.63  0.05   0.54   0.60   0.63   0.67   0.75    1  5600
p0         0.06  0.01   0.04   0.05   0.06   0.06   0.08    1 11000
N         86.42 17.98  56.00  74.00  85.00  97.00 126.02    1  3900
D          5.82  1.21   3.77   4.98   5.72   6.53   8.49    1  3900
beta       1.27  0.21   0.88   1.12   1.26   1.41   1.71    1  5600
deviance 411.06 12.37 389.20 402.50 410.20 418.90 437.60    1 10000

Buffer = 35 km
           mean    sd   2.5%    25%    50%    75%  97.5% Rhat n.eff
psi        0.34  0.08   0.21   0.29   0.34   0.39   0.50    1 30000
sigma      0.63  0.05   0.54   0.60   0.63   0.67   0.75    1  4500
p0         0.06  0.01   0.04   0.05   0.06   0.06   0.08    1 24000
N        101.79 21.54  65.00  87.00 100.00 115.00 148.00    1 30000
D          5.85  1.24   3.74   5.00   5.75   6.61   8.51    1 30000
beta       1.27  0.21   0.89   1.12   1.25   1.40   1.70    1  4500
deviance 411.10 12.20 389.50 402.40 410.30 418.90 437.20    1 22000

Buffer = 40 km
           mean    sd   2.5%    25%    50%    75%  97.5% Rhat n.eff
psi        0.39  0.09   0.24   0.33   0.39   0.45   0.60 1.01   480
sigma      0.64  0.05   0.54   0.60   0.63   0.67   0.75 1.01   410
p0         0.06  0.01   0.04   0.05   0.06   0.06   0.08 1.00 21000
N        118.05 26.14  75.00 100.00 116.00 133.00 178.00 1.01   450
D          5.87  1.30   3.73   4.97   5.76   6.61   8.84 1.01   450
beta       1.27  0.21   0.89   1.12   1.25   1.40   1.72 1.01   410
deviance 411.37 12.35 389.30 402.60 410.60 419.30 437.50 1.00  9700

Buffer = 45 km
           mean    sd   2.5%    25%    50%    75%  97.5% Rhat n.eff
psi        0.45  0.10   0.28   0.38   0.44   0.51   0.66    1  3600
sigma      0.64  0.05   0.54   0.60   0.63   0.67   0.75    1 10000
p0         0.06  0.01   0.04   0.05   0.06   0.06   0.08    1  8100
N        134.43 28.68  85.00 114.00 132.00 153.00 196.00    1  3300
D          5.83  1.24   3.68   4.94   5.72   6.63   8.50    1  3300
beta       1.26  0.21   0.88   1.11   1.24   1.39   1.69    1 10000
deviance 411.36 12.19 389.60 402.70 410.60 419.10 437.30    1  9400

Buffer = 50 km
           mean    sd   2.5%    25%    50%    75%  97.5% Rhat n.eff
psi        0.51  0.11   0.31   0.43   0.50   0.57   0.74    1  3200
sigma      0.63  0.05   0.54   0.60   0.63   0.67   0.75    1  4700
p0         0.06  0.01   0.04   0.05   0.06   0.06   0.08    1  3300
N        151.61 31.65  96.00 129.00 149.00 172.00 221.00    1  3400
D          5.79  1.21   3.66   4.92   5.69   6.56   8.43    1  3400
beta       1.27  0.21   0.89   1.12   1.25   1.40   1.70    1  4700
deviance 410.81 12.18 389.20 402.30 410.10 418.50 436.70    1 30000

Buffer = 55 km 
           mean    sd   2.5%    25%    50%    75%  97.5% Rhat n.eff
psi        0.56  0.12   0.35   0.48   0.55   0.64   0.82 1.01   260
sigma      0.64  0.05   0.54   0.60   0.63   0.67   0.76 1.00  1600
p0         0.06  0.01   0.04   0.05   0.06   0.06   0.08 1.00 30000
N        169.28 35.81 108.00 143.00 166.00 192.00 247.00 1.01   260
D          5.73  1.21   3.66   4.84   5.62   6.50   8.36 1.01   260
beta       1.25  0.21   0.88   1.11   1.24   1.39   1.69 1.00  1600
deviance 411.28 12.38 389.40 402.60 410.50 419.10 437.50 1.00 26000
\end{verbatim}
}

We see that the estimated density is roughly consistent as we increase
the state-space buffer from $15$ to $50$ $km$. We do note that the data
augmentation parameter $\psi$ (and, correspondingly, $N$) increase with
the size of the state space in accordance with the deterministic
relationship $N= D*A$. However, density is constant more or less as we
increase the size of the state-space beyond a certain point.  For the
10 $km$ state-space buffer, we see a slight effect on the posterior
distribution of $D$. This is not a bug but rather a feature. As we noted
above, the state-space is part of the model.


\subsection{Thoughts on the Wolverine Analysis}

Our point estimate of wolverine density from this study, using the
posterior mean from the state-space based on the 20
$km$ buffer, is 
approximately $5.77$ individuals/1000 $km^2$ with  a 95\% posterior
interval of $[3.86, 8.29]$. Density is estimated imprecisely
which might not be surprising given the low sample size ($n=21$
individuals!). This seems to be a basic feature of carnivore studies
although it should not (in our view) preclude the study of their
populations nor attempts to estimate density or vital rates.

One thing we haven't talked about yet is that we can calibrate the
desired size of the state-space by looking at the estimated home range
radius of the species. For some models it is possible to convert the
parameter $\alpha1$ directly into the home range radius (sec. 
XXX MISSING XYZ). For the half-normal model we interpret the half-normal scale
parameter $\sigma$ which is related to $\alpha1$ by $\alpha1 =
1/(2\sigma^2)$ as the radius of a bivariate normal movement model. 
In this case $\sigma = 1.82$ standardized units = 18.2 $km$ which 
translates into a home range area of XXXX MISSING XXXXX. 

It is worth thinking about this model, and these estimates, computed
under a rectangular state space roughly centered over the trapping
array (Fig. \ref{scr0.fig.wolverinelocs}).
Does it make sense to define the state-space to
include, for example, ocean? What are the possible consequences of
this? What can we do about it?  There's no reason at all that the
state space has to be a regular polygon -- we defined it as such here
strictly for convenience and for ease of implementation in {\bf WinBUGS}
where it enables us to specify the prior for the activity centers as
uniform priors for each coordinate.  While it would be possible to
define a more realistic state-space using some general polygon GIS coverage, it
might take some effort to implement that in the {\bf BUGS} language
but it is not difficult to devise custom MCMC algorithms to do that
(see Chapt. \ref{chapt.mcmc}).
Alternatively, we recommend
using a discrete representation of the state-space -- i.e., approximate
${\cal S}$ by a grid of $G$ points. We discuss this in sec. 
\ref{scr0.sec.discrete}.


\section{Constructing Density Maps}
\label{scr0.sec.mapping}

One of the most useful aspects of SCR models is that they are
parameterized in terms of individual locations - i.e., {\it where}
each individual lives -- and, thus, we can compute many useful or
interesting summaries of the activity centers.  For example, we can
make a spatial density plot by tallying up the number of activity
centers ${\bf s}_{i}$ in boxes of arbitrary size and then producing a
nice multi-color spatial plot of those which, we find, increases the
acceptance probability of your manuscripts by a substantial amount.
We discussed in Chapt. \ref{chapt.glms} the idea of estimating derived
parameters from MCMC output. In SCR models, there are many derived
parameters that are functions of the latent point locations $({\bf
  s}_{1},\ldots, {\bf s}_{N})$. In the present context, the number of
individuals living in any well-defined polygon is a derived
parameter. Specifically, let $B({\bf x})$ indicate a box centered at
${\bf x}$ then
\[
N({\bf x})=\sum_{i} I({\bf s}_{i} \in B({\bf x}))
\]
is the population size of box $B({\bf x})$, and $D({\bf x}) = N({\bf
  x})/||B({\bf x})||$ is the local density. These are just ``derived
parameters'' (see Chapt.  \ref{chapt.glms}) which are estimated from
MCMC output using the appropriate Monte Carlo average. One thing to be
careful about, in the context of models in which $N$ is unknown, is
that, for each MCMC iteration $m$, we only tabulate those activity
centers which correspond to individuals in the sampled
population. i.e., for which the data augmentation variable $z_{i} =
1$.  In this case, we take all of the output for MCMC iterations
$m=1,2,\ldots,\mbox{\tt niter}$ and compute this summary:
\[
   N({\bf x},m) = \sum_{z_{i,m}=1} I(s_{i,m} \in B({\bf x}))
\]
Thus, $N({\bf x},1),N({\bf x},2),\dots,$ is the Markov chain for
parameter $N({\bf x})$.  In what follows we will provide a set of {\bf
  R} commands for doing this calculations and making a basic image
plot from the MCMC output.

{\flushleft \bf Step 1:} Define the center points of each box, $B({\bf
  x})$, or point at which local density will be estimated:
\begin{verbatim}
xg<-seq(Xl,Xu,,50)
yg<-seq(Yl,Yu,,50)
\end{verbatim}

{\flushleft \bf Step 2:} Extract the MCMC histories for the activity
centers and the data augmentation variables.  Note that these are each
$N \times \mbox{\tt niter}$ matrices:
\begin{verbatim}
Sxout<-out$sims.list$s[,,1]
Syout<-out$sims.list$s[,,2]
z<-out$sims.list$z
\end{verbatim}

{\flushleft \bf Step 3:} We associate each coordinate with the proper
box using the {\bf R} command \mbox{\tt cut()}. Note that we keep only
the activity centers for which $z=1$ (i.e., individuals that belong to
the population of size $N$):
\begin{verbatim}
Sxout<-cut(Sxout[z==1],breaks=xg,include.lowest=TRUE)
Syout<-cut(Syout[z==1],breaks=yg,include.lowest=TRUE)
\end{verbatim}

{\flushleft \bf Step 4:} Use the \mbox{\tt table()} command to tally
up how many activity centers are in each $B(x)$:
\begin{verbatim}
Dn<-table(Sxout,Syout)
\end{verbatim}

{\flushleft \bf Step 5:} Use the \mbox{\tt image()} command to display
the resulting matrix.
\begin{verbatim}
image(xg,yg,Dn/nrow(z),col=terrain.colors(10))
\end{verbatim}
Praise the Lord! This map is somewhat useful or at least it looks
pretty and will facilitate the publication of your papers.

It is worth emphasizing here that density maps will not usually appear
uniform despite that we have assumed that activity centers are
uniformly distributed. This is because the observed encounters of
individuals provide direct information about the location of the
$i=1,2,\ldots,n$ activity centers and thus their ``estimated''
locations will be affected by the observations. In a limiting sense,
were we to sample space intensely enough, every individual would be
captured a number of times and we would have considerable information
about all $N$ point locations. Consequently, the uniform prior would
have almost no influence at all on the estimated density surface in
this limiting situation. Thus, in practice, the influence of the
uniformity assumption increases as the fraction of the population
encountered decreases.

{\bf On the non-intuitiveness of \mbox{\tt image()} } -- the {\bf R}
function \mbox{\tt image()} might
not be very intuitive to some -- it plots $M[1,1]$ in the lower left
corner. If you want $M[]$ to be plotted ``as
you look at it'' then $M[1,1]$ should be in the upper left corner.  We
have a function \mbox{\tt rot()} which does that. If you do \mbox{\tt image(rot(M))} then it
puts it on the monitor as if it was a map you were looking at.  You
can always specify the $x$ and $y-$ labels explicitly as we did above.

{\bf Spatial dot plots } -- Now here is a cruder version based on the
``spatial dot map'' function \mbox{\tt spatial.plot}, which uses
the function \mbox{\tt image.scale()}.
The \mbox{\tt spatial.plot} function requires arguments of point
locations and the resulting value to be displayed:
\begin{verbatim}
spatial.plot<- function(x,y){
 nc<-as.numeric(cut(y,20))
 plot(x,pch=" ")
 points(x,pch=20,col=topo.colors(20)[nc],cex=2)
 image.scale(y,col=topo.colors(20))
}
# To execute the function do this:
spatial.plot(cbind(xg,yg), Dn/nrow(z))
\end{verbatim}

\subsection{Example: Wolverine density map. }

The {\bf R} commands for producing density maps from MCMC output of
spatial capture-recapture models is provided in the {\bf R} function
\mbox{\tt SCRdensity} in the package \mbox{\tt scrbook}. 
We used the posterior output from the wolverine model fitted previous
to compute a relatively coarse version of a density map, using a $10 \times
10$ grid (Fig. \ref{scr0.fig.density10x10}) and using a $30 \times 30$
grid (Fig. \ref{scr0.fig.density20x20}). The {\bf R} commands for
producing such a plot (for short MCMC run) are as follows:
{\small
\begin{verbatim}
library("scrbook")
data(wolverine)
traps<-wolverine$wtraps
y3d <-SCR23darray.fn(wolverine$wcaps,wolverine$wtraps)
# this takes 341 seconds on a standard CPU circa 2011
unix.time(bln<-wolvSCR0.fn(y3d,traps,nb=1000,ni=2000,delta=1,M=100))
Sx<-bln$sims.list$s[,,1]
Sy<-bln$sims.list$s[,,2]
w<- bln$sims.list$w
obj<-list(Sx=Sx,Sy=Sy,w=w)
tmp<-SCRdensity(obj,scalein=100,scaleout=100)
\end{verbatim}
In these figures density is
expressed in units of individuals per $100$ $km^2$, while the area of
the pixels is about 103.7 $km^2$ and 11.5 $km^2$, respectively. That
calculation is based on:
\begin{verbatim}
> total.area<- (Yu-Yl)*(Xu-Xl)*100
> total.area/(10*10)
[1] 103.7427
> total.area/(30*30)
[1] 11.52697
\end{verbatim}

A couple of things are worth noting: First is that as we move away
from ``where the data live'' - away from the trap array - we see that
the density approaches the mean density. This is a property of the
estimator as long as the ``detection function'' decreases sufficiently
rapidly as a function of distance.
Relatedly, it is also a property of statistical smoothers
such as splines, kernel smoothers, and regression smoothers -
predictions tend toward the global mean as the influence of data
diminishes. Another way to think of it is that it is a consequence of
the prior - which imposes uniformity, and as you get far away from the
data, the predictions tend to the prior. The other thing to note about
this map is that density is not $0$ over water (although the coastline
is not shown). This might be perplexing
to some who are fairly certain that wolverines do not like
water. However, there is nothing about the model that recognizes water
from non-water and so the model predicts over water {\it as if} it
were habitat similar to that within which the array is nested. But,
all of this is ok as far as estimating density goes and, furthermore,
we can compute valid estimates of $N$ over any well-defined region which
presumably wouldn't include water if we so choose.

\begin{figure}
\begin{center}
\includegraphics[height=3in,width=3.375in]{Ch4/figs/density10x10}
\end{center}
\caption{Needs a caption}
\label{scr0.fig.density10x10}
\end{figure}

\begin{figure}
\begin{center}
\includegraphics[height=3in,width=3.375in]{Ch4/figs/density30x30}
\end{center}
\caption{Needs a caption}
\label{scr0.fig.density20x20}
\end{figure}

\section{Discrete State-Space}
\label{scr0.sec.discrete}

The SCR model developed previously in this chapter assumes that
individual activity centers are distributed uniformly over the
prescribed state-space. Clearly this will not always be a reasonable
assumption. In chapter \ref{chapt.state-space} we talk about developing models
that allow explicitly for non-uniformity of the activity centers by
modeling covariate effects on density. A simpler method of affecting
the distribution of activity centers, which we address here, is to
modify the shape of the state-space explicitly. For example, we might
be able to classify the state-space into distinct blocks of habitat
and non-habitat. In that case we can remove the non-habitat from the
state-space and assume uniformity of the activity centers over the
remaining portions judged to be suitable habitat.  There are two ways
to approach this: We can use a regular grid of points to represent the
state-space, i.e., by the set of coordinates ${\bf s}_1, \ldots, {\bf
  s}_{G}$, and assign a equal probabilities to each possible value, or
we can retain the continuous formulation of the state-space but use
basic polygon operations to induce constraints on the state-space We
focus here on the formulation of our basic SCR model in terms of a
discrete state-space but later on (chapter \ref{chapt.mcmc} and also
Appendix XYZ) we demonstrate the latter approach based on using
polygon operations to define an irregular state-space.

Use of a discrete state-space can be computationally expensive in {\bf
  WinBUGS}. That said, it isn't too difficult to do the MCMC
calculations in {\bf R} which we discuss briefly in chapter
\ref{chapt.mcmc}. The {\bf R} package {\tt SPACECAP}
\citep{gopalaswamy_etal:2011} arose from the {\bf R} implementation
developed for the application in \citet{royle_etal:2009}.  As we will
see in chapter \ref{chapt.mle}, we must prescribe the state-space by a
discrete mesh of points in order to do integrated likelihood and so if
we are using a discrete state-space this can be accommodated directly
in our code for obtaining MLEs.

While clipping out non-habitat seems like a good idea, its not obvious
that we accomplish any biologically reasonable objective by doing
so. We might prefer to do it when non-habitat represents a clear-cut
restriction on the state-space such as a reserve boundary or a lake,
ocean or river. It makes sense in those situations.  Unfortunately,
having the capability to do this also causes people to start defining
``habitat'' vs. ``non-habitat'' based on their understanding of the
system whereas it can't be known whether the animal being studied has
the same understanding. Moreover, differentiating of the landscape by
habitat or habitat quality probably affects the geometry and
morphology of home ranges much more than the plausible locations of
activity centers. That is, a home range centroid could, in actual
fact, occur in a walmart parking lot if there is pretty good habitat
around walmart, so there is probably no sense to cut out the walmart
lot and preclude it as the location for an activity center.  It would
generally be better to include some definition of habitat quality in
the model for the detection probability (see chapter XYZ).


\subsection{Evaluation of Coarseness of Discrete Approximation}

The coarseness of the state-space should not really have much of an
effect on estimates if the grain is sufficiently fine relative to
typical animal home range sizes.  Why is this?  We have two analogies
that can help us understand this. First is the relationship to Model
$M_{h}$.  As noted in section \ref{scr0.sec.scrmh} above, we can think
about SCR models as a type of finite mixture
\citep{norris_pollock:1996, pledger:2000} where we are fortunate to be
able to obtain direct information about which ``group'' individuals
belong to (group being location of activity center).  In the standard
finite mixture models we typically find that only 1 or a very small
number of groups (e.g., 2 or 3 at the most) can explain really high
levels of heterogeneity and are adequate for most data sets of small
to moderate sample sizes. We therefore expect a similar effect in SCR
models when we discretize the state-space.
We can also
think about discretizing the state-space as being related
to numerical integration where we find (see
chapter \ref{chapt.mle}) that we don't need a very fine
grid of support points to evaluate the integral to a reasonable
level of accuracy. We demonstrate this here by reanalyzing simulated
data using a state-space defined by a different numbers of support points.
We provide an R script called \mbox{\tt simSCR0discrete.fn} in the
{\bf R} package \mbox{\tt scrbook}.  We note that for this comparison
we generated the actual activity centers as a continuous random
variable and thus the discrete state-space is, strictly speaking, an
approximation to truth. That said, we regard all state-space
specifications as approximations to truth because they are all,
strictly speaking, models of some unknown truth. Thus the use of any
specific discrete state-space is not intrinsically more ``wrong'' than
any specific continuous representation.


We used {\bf JAGS} from the \mbox{\tt rjags} function to obtain the results
for $6 \times 6$, $9 \times 9$, $12 \times 12$, $15\times 15$,
$20\times 20$, $25 \times 25$ and $30 \times 30$ state-space grids.
We used 2000 burn, 12000 total iters with 3 chains, therefore a total
of 30000 posterior samples.
For {\bf WinBUGS} we used 3 chains of 5k total with 1k burnin means 12k
total posterior samples.
Summary results for these analyses are shown in
Table XYZ\footnote{Andy to finish later. }.

\begin{verbatim}
Table XYZ.
             Mean       SD    NaiveSE  Time-seriesSE  runtime
6    N     109.7717 15.98959 0.0923160    0.377737    1239
9    N     114.4621 16.72025 0.0965344    0.468659    1267
12   N     115.4309 17.12403 0.098866     0.464830    1576
15   N     114.7699 17.0242  0.0982894    0.425238    1638
20   N     116.0370 17.10686 0.0987665    0.486867    1647
25   N     116.3228 16.98323 0.0980527    0.465527    1661
30   N     116.4252 17.4078  0.100504     0.533735    1806
WinBUGS
             Mean       SD    NaiveSE  Time-seriesSE  runtime
6    N     111.67    16.61                             2274
9    N     114.23    17.99                             4300
12   N     115.98    17.38                             7100
15   N     115.38    17.94                            13010

Note: WinBUGS based on fewer samples too!

To get SE and time-series SE do this:
You can use as.mcmc.list() to convert to a coda object. Then use summary.�
\end{verbatim}

The results in terms of the posterior summaries are, as we
expect, very similar using {\bf WinBUGS}. However, it was interesting
to note that {\bf WinBUGS} runtime is much worse (note the number of
iterations is lower for {\bf WinBUGS} yet the runtime is much longer)
and, furthermore, it seems to scale with the size of the
discrete state-space grid. While that was expected, it was unexpected
that the runtime of {\bf JAGS} would seem relatively consistent
as we increase the grid size.
We suspect that {\bf WinBUGS} is evaluating the full-conditional for
each activity center at all $G$ possible values whereas it may be that
{\bf JAGS} is evaluating the full-conditional only at a subset of
values or perhaps using previous calculations more effectively.

While this might suggest that one should always use {\bf JAGS} for
this analysis, we found in our analysis of the wolverine (next
section) that {\bf JAGS} could be extremely sensitive to starting
values, producing MCMC algorithms that sometimes simply did not work.

\subsection{Analysis of the wolverine camera trapping data}

We reanalyzed the wolverine data using discrete state-space grids with points spaced by 2,
4 and 8 km (depicted in Fig. \ref{scr0.fig.wolvgrids}). These were
constructed from
the 40 km buffered state-space, and deleting the points over water \citep[see][]{royle_etal:2011jwm}.
 Our interest in doing this was
to evaluate the relative influence of grid resolution on estimated
density because the coarser grids will be more efficient from a
computational stand-point and so we would prefer to use them, but perhaps not
if there is a strong influence on estimated density.

{\bf Note}: Results from WinBUGS are given below -- these are updated
based on longer MCMC runs and replace prelim results as of Jan 1 2012
or so. 
To be done: density map.



\begin{figure}
\begin{center}
\includegraphics[height=2.5in,width=5in]{Ch4/figs/wolvgrids}
\end{center}
\caption{2 km 4 km and 8km wolverine state-space grids extending about
40 km from the vicinity of the trap array. }
\label{scr0.fig.wolvgrids}
\end{figure}

{\small
\begin{verbatim}
This took about 6 days in WinBUGS. Terrible mixing for the 2km and
8km. Why is this? We may never know!

> print(out.2km,digits=2)
Inference for Bugs model at "modelfile.txt", fit using WinBUGS,
 3 chains, each with 11000 iterations (first 1000 discarded)
 n.sims = 30000 iterations saved
       mean    sd  2.5%   25%   50%   75%  97.5% Rhat n.eff
psi    0.43  0.09  0.27  0.37  0.43  0.49   0.63 1.00   560
sigma  0.62  0.05  0.54  0.59  0.62  0.65   0.73 1.01   160
lam0   0.05  0.01  0.04  0.04  0.05  0.06   0.07 1.01   320
p0     0.05  0.01  0.03  0.04  0.05  0.05   0.06 1.01   320
N     86.56 16.94 57.00 75.00 85.00 97.00 124.00 1.00   510
D      8.78  1.72  5.78  7.60  8.62  9.83  12.57 1.00   510

For each parameter, n.eff is a crude measure of effective sample size,
and Rhat is the potential scale reduction factor (at convergence, Rhat=1).
> print(out.4km,digits=2)
Inference for Bugs model at "modelfile.txt", fit using WinBUGS,
 3 chains, each with 11000 iterations (first 1000 discarded)
 n.sims = 30000 iterations saved
       mean    sd  2.5%   25%   50%    75%  97.5% Rhat n.eff
psi    0.45  0.09  0.28  0.38  0.44   0.50   0.64    1  1300
sigma  0.61  0.04  0.53  0.58  0.61   0.64   0.71    1  1600
lam0   0.05  0.01  0.04  0.05  0.05   0.06   0.07    1  2500
p0     0.05  0.01  0.03  0.04  0.05   0.05   0.07    1  2500
N     89.25 17.44 59.00 77.00 88.00 100.00 127.00    1  1100
D      9.01  1.76  5.96  7.77  8.88  10.10  12.82    1  1100

For each parameter, n.eff is a crude measure of effective sample size,
and Rhat is the potential scale reduction factor (at convergence, Rhat=1).
> print(out.8km,digits=2)
Inference for Bugs model at "modelfile.txt", fit using WinBUGS,
 3 chains, each with 11000 iterations (first 1000 discarded)
 n.sims = 30000 iterations saved
       mean    sd  2.5%   25%   50%   75%  97.5% Rhat n.eff
psi    0.42  0.09  0.26  0.36  0.41  0.47   0.61 1.00   940
sigma  0.68  0.05  0.59  0.64  0.67  0.71   0.77 1.01   220
lam0   0.05  0.01  0.03  0.04  0.05  0.05   0.06 1.00   560
p0     0.05  0.01  0.03  0.04  0.04  0.05   0.06 1.00   560
N     83.18 16.14 56.00 72.00 82.00 93.00 119.00 1.00   700
D      8.28  1.61  5.57  7.17  8.16  9.26  11.84 1.00   700

For each parameter, n.eff is a crude measure of effective sample size,
and Rhat is the potential scale reduction factor (at convergence, Rhat=1).
\end{verbatim}
}

The density is a bit different depending on the grid size. Also the
effectiveness of the MCMC algorithsm is pretty remarkably different. 
We did the analysis in JAGS also. The results are shown below. {\bf Note}: I
am going to run these again but for longer to finalize the results.

{\small
\begin{verbatim}
 ### 01/10/2012 -- need to rerun these JAGS runs but use more
iterations and check results.


2km
Iterations = 7001:13000
Thinning interval = 1
Number of chains = 3
Sample size per chain = 6000

          Mean        SD  Naive SE Time-series SE
N     86.28522 16.950626 1.263e-01      0.4878973
lam0   0.04807  0.007512 5.599e-05      0.0002199
p0     0.04581  0.006820 5.083e-05      0.0001996
psi    0.28904  0.062117 4.630e-04      0.0017481
sigma  0.62769  0.043596 3.249e-04      0.0018724

4km
          Mean        SD  Naive SE Time-series SE
N     85.53139 16.998966 1.267e-01      0.5181297
lam0   0.04636  0.007542 5.621e-05      0.0002382
p0     0.04425  0.006867 5.118e-05      0.0002172
psi    0.28650  0.061922 4.615e-04      0.0018276
sigma  0.64281  0.048321 3.602e-04      0.0022911

8km
          Mean        SD  Naive SE Time-series SE
N     83.97039 16.508146 1.230e-01      0.4548782
lam0   0.04519  0.006919 5.157e-05      0.0001738
p0     0.04319  0.006319 4.710e-05      0.0001589
psi    0.28146  0.060653 4.521e-04      0.0016555
sigma  0.66956  0.040989 3.055e-04      0.0015070
\end{verbatim}
}

\subsection{SCR models as multi-state models}

While we invoke a discrete state-space artificially, by gridding the
underlying continuous state-space, sometimes the state-space is more
naturally discrete. Consider a situation in which discrete patches of
habitat are searched using some method and it might be convenient (or
occur inadvertently) to associate samples to the patch level instead
of recording observation locations. In this case we might use a model
${\bf s}_{i} \sim dcat(probs[])$  where $probs[]$ are the probabilities that
an individual inhabits a particular patch. We consider such a case
study in chapter XXPoissonXXX from \citet{mollet_etal:2012} who
obtained a population size estimate of a large grouse species known as
the capracaillie. Forest patches were searched for scat which was
identified to individual by DNA analysis.
Even when space is {\it not}
naturally discrete, measurements are often made at a fairly coarse
grain (e.g., meters or tens of meters along a stream), or associated
with spatial quadrats for scat searches and therefore the state-space
may be effectively discrete in many situations.

This discrete formulation of SCR models suggests that SCR models are
related to ordinary multi-state models \citep[][ch. 9]{kery_schaub:2011}
which are also parameterized in terms of a discrete state
variable which is often defined as a spatially-indexed state related
either to location of capture or breeding location. While many
multi-state models exist in which the state variable is not related to
space, multi-state models have been extremely useful in development
models of movements among geographic states and indeed this type of
problem motivated their early developments by \citet{arnason:1972,
  arnason:1973} and \citet{hestbeck_etal:1991}.  We pursue this
connection a little bit more in chapter XXX XYZ.




\section{ Summary and Outlook }

A point we tried to emphasize in this chapter is that the basic SCR
model is not much more than an ordinary capture-recapture model for
closed populations -- it is simply that model but augmented with a set
of ``individual effects'', ${\bf s}_{i}$, which relate encounter
probability to some sense of individual location. SCR models are
therefore a type of individual covariate model (as introduced in
chapter \ref{chapt.closed} -- but with imperfect information about the
individual covariate. In other words, they are GLMM type models when
$N$ is known or, when $N$ is unknown, they are zero-inflated GLMMs
(see \citet{royle:2006}).  Another class of capture-recapture models
that SCR models are closely related to is so-called ``Model $M_{h}$.''
The effect of introducing a spatial location for individuals is that
it induces heterogeneity in detection probability, as in Model
$M_{h}$. However, unlike Model $M_{h}$, we obtain some information
about the individual effect which is completely latent in Model
$M_{h}$. If the state-space of the random effect ${\bf s}$ is discrete
then the SCR model resembles more closely the finite-mixture class of
heterogeneity models \citep{norris_pollock:1996} which parameterizes
heterogeneity by assuming that individuals belong to discrete classes
or groups (e.g., high, medium, low). In the context of SCR models we
obtain some information about the ``group membership'' in the
locations where individuals are captured.  Given the direct
relationship of SCR models with so many standard classes of models, we
find that they are really quite easy to analyze using standard MCMC
methods encased in black boxes such as {\bf WinBUGS} or {\bf JAGS} and
possibly other packages. They are also easy to analyze using classical
likelihood methods, which we address in chapter \ref{chapt.mle}.

Formal consideration of the collection of individual locations $({\bf
  s}_{1}, \ldots, {\bf s}_{N})$ in the model is fundamental to all of
the models considered in this book. In statistical terminology, we
think of the collection of points $\{ {\bf s}_{i} \}$ as a realization of a
point process and part of the promise, and ongoing challenge, of SCR
models is to develop models that reflect interesting biological
processes, for example interactions among points or temporal dynamics
in point locations.  Here we considered the simplest possible point
process model - the points are independent and uniformly
(``randomly'') distributed over space. Despite the simplicity of this
assumption, it should suffice in many applications of SCR models
although we do address generalizations of this model in later
chapters. Moreover, even though the {\it prior} distribution on the
point locations is uniform, the realized pattern may deviate markedly
from uniformity as the observed encounter data provide information to
impart deviations from uniformity. Thus, the estimated density map
will typically appear distinctly non-uniform.  As a general rule,
information in the data will govern estimates of individual point
locations so even fairly complex patterns of non-independence or
non-uniformity will appear in the data. That is, we find in
applications of the basic SCR model that this simple {\it a priori}
model can effectively reflect or adapt to complex realizations of the
underlying point process.  For example, if individuals are highly
territorial then the data should indicate this in the form of
individuals not being encountered in the same trap - the resulting
posterior distribution of point locations should therefore reflect
non-independence.  Obviously the complexity of posterior estimates of
the point pattern will depend on the quantity of data, both number of
individuals and captures per individual.  Because the point process is
such an integral component of SCR models, the state-space of the point
process plays an important role in developing SCR models. As we tried
to emphasize in this chapter, the choice of the stat-espace is part of
the model. It can have an influence on parameter estimates and other
inferences such as model selection (see chapter \ref{chapt.gof}). We
emphasize however that this is not an arbitrary decision like
``buffering'' because the model induces an explicit interpretation of
parameters and statistical effect on estimators.

We showed how to conduct inference about the underlying point process
including calculation of density maps from posterior output. We can do
other things we normally do with spatial point processes such as
compute ``K-functions'' and test for ``complete spatial randomness''
(CSR) which we develop in chapter \ref{chapt.gof}.  Modifying and
applying point process methods to SCR problems seems to us to be a
fruitful area of research.

An obvious question that might be floating around in your mind is why
should we ever go through all of this trouble when we could just use
{\bf MARK} or {\bf CAPTURE} to get an estimate of $N$ and apply $1/2$
MMDM methods?  The main reason is that these conventional methods are
predicated on models that represent explicit misspecifications of both
the observation and ecological process - they are wrong!  Not just
wrong, because of course all models are wrong, but they're not even
{\it plausible} models! Thus while we might be able to show adequate
fit or whatever, we think as a conceptual and philosophical model one
should not be using models that are not even plausible data-generating
models -- even if the plausible ones don't fit!  Perhaps more
charitably, these ordinary non-spatial models are models of the wrong
system. They do not account for trap identity. They don't account for
spatial organization or ``clustering'' of individual encounters in
space. And, ``density'' is not a parameter of those models because
density has no meaning absent an explicit representation of space. If
we do define space explicitly, e.g., as a buffered minimum convex
hull, then the normal models ($M_{0}$, $M_{h}$, etc..) assume that
individual capture-probability is not related to space, no matter how
we define the buffer.  Conversely, the SCR model is a model for
trap-specific encounter data - how individuals are organized in space
and interact with traps. SCR models provide a coherent framework for
inference about density or population size and also, because of the
formality of their derivation, can be extended and generalized to a
large variety of different situations, as we demonstrate in subsequent
chapters.

In the next few chapters we continue to work with this basic SCR
design and model but consider some important extensions of the basic
model.  For example, we consider
extensions
to  include covariates that vary by individual, trap, or over time
(chapter \ref{chapt.covariates}), spatial covariates on density
(chapter \ref{chapt.state-space}),
 open populations (chapter \ref{chapt.open}), model assessment and
 selection (chapter \ref{chapt.gof}) and other topics.
We also consider technical details of Bayesian (chapter
\ref{chapt.mcmc}) and  maximum
likelihood (chapter \ref{chapt.mle}) estimation so that the interested
reader can develop or extend their own methods to suit their needs.


\chapter{Other observation models}
\label{chapt.poisson}

%\chapter{Alternative Models for the Encounter Process}
\label{chapt.poisson-mn}

In the previous chapter we considered a very specific although not
terribly limited observation model. The observation model consisted of
two main elements: First a description of the encounter process 
by which individuals are detected in traps. Specifically, we 
assumed individual trap-specific encounters were iid Bernoulli
trials. The consequence of this is that individuals function
independently of one another and can be captured in
any number of traps during a specific interval of trapping
effort. The type of device is typical of bear hair snares, which we
considered as an example in that section. The 2nd element of the
encounter process model was the specific model – functional form –
relating encounter probability to individual activity center
(``detection probability model'').  It is natural to consider
alternative functional forms of this detection probability model which
we do in Chapt. \ref{chapt.covariates} and elsewhere. 

In this chapter we consider alternative observation models which
accommodate Poisson or multinomial observation models. For example, if
sampling devices can detect an individual some arbitrary number of
times during an interval, then it is natural to consider observation
models for encounter frequencies, such as the Poisson model. Another
type of encounter device is the ``multi-catch'' device (REF XYZ) which
is a physical device that can capture and hold an arbitrary number of
individuals. A typical example is a mist-net for birds 
\citep{borchers_efford:2008}.

We talk about how SCR are multi-state kinds of models. 

We talk about single catch traps. 


\section{Poisson Observation Model}

The models we analyze in Chapt. \ref{chapt.scr0} assumed binary
observations -- i.e., standard encounter history data -- so
that individuals are captured at most one time in a trap.  This makes
sense for many types of DNA sampling (e.g., based on hair snares)
because distinct visits to sampled locations or devices cannot be
differentiated. However, many encounter methods or devices make it
possible to encounter an individual some arbitrary number of times
during any particular sampling episode. That is, we might observe
encounter frequencies $y_{ijk}>0$ for individual $i$, trap $j$ and
sampling interval $k$.  As an example, if a camera device is
functioning properly it may be programmed to take photos every few
seconds if triggered.  For a second example, suppose we are searching
a quadrat for scat, we may find multiple samples from the same
individual.

Therefore, we seek observation models that accommodate such encounter
frequency data.  Let $y_{ijk}$ be the frequency of encounter for
individual $i$, in trap $j$, during occasion $k$, then a plausible
model is:
\[
 y_{ijk} \sim \mbox{Poisson}(\lambda_{ij})
\]
where the expected encounter frequency $\lambda_{ij}$ depends on both
individual and trap. As we did in the binary model of chapter 4, we
now seek to model the expected value of the observation (which was
$p_{ij}$ in chapter 4) as a function of the individual activity center
${\bf s}_{i}$.
We propose 
\[
 \lambda_{ij} = \lambda_{0}  g({\bf x}_{j},{\bf s}_{i})
\]
Where $g({\bf x},{\bf s})$ is some positive valued function. 
Then $\lambda_{0}g({\bf x},{\bf s})$ is the encounter rate in trap
${\bf x}$ for an individual having activity center ${\bf s}$.  

What does this mean? This means that the encounter rate looks like a
bivariate normal distribution.  If we might interpret encounters as
resulting from the outcome of a movement model in the following
sense. Suppose that we telemeter an individual and take measurements
of location sufficiently far apart in time that locations are
independent. Let $x_{t}$ be the location at time $t$. Take a large
number of samples, make a grid and count up the number of observations
in each grid cell.
\[
 E[y(x)] = E[y(x)| moves to x]\Pr(moves to x|s) = \lambda_{0} g(x|s)
\]


For the simplest model in which we have covariates that vary across
the replicate samples $k$, we can aggregate the observed data by the
propery of compound additivity of the Poisson distribution (if $x$ and
$y$ are $iid$ Poisson with mean $\lambda$ then $x+y$ is Poisson with
mean $2\lambda$). Therefore,
\[
y_{ij} = (\sum_{k=1}^{K} y_{ijk}) =  \mbox{Poisson}(K  \lambda_{0} 
g({\bf x}_{j},{\bf s}_{i}) )
\]
We see that $K$ and $\lambda_{0}$ serve the same role as affecting the
base encounter rate. Since the observation model is the same,
probabilistically speaking, for all values of $K$, evidently we need
only $K=1$ ``survey'' from which to estimate model parameters. We know
this intuitively as sampling by multiple traps serves as replication
in SCR models.


\subsection{Poisson relationship to the Bernoulli model}

There is a sense in which the Poisson and Bernoulli models can
be viewed as consistent with one another. Note that under the Poisson
model we have:
\begin{equation}
 \Pr(y>0) = 1-exp(-\lambda_{0} g({\bf x},{\bf s}))
\label{eq.cloglog}
\end{equation}
Therefore, 
if we equate the event ``encountered'' with the event that the
individual was captured at least 1 time under the Poisson model, i.e., $y>0$, then it would be
natural to set $p_{ij} = \Pr(y>0)$ according to \ref{eq.cloglog}. 

In fact, as $\lambda_0$ gets small, the Poisson model is a close approximation
to the Bernoulli model in the sense that $y$ in that case is almost
always 0 or 1 and, in fact, $\Pr(y>0) \rightarrow \lambda$.  This is
convenient in some cases because the Poisson model might be more
tractable to fit (or even vice versa). For an example, see the models
described in Chapt. \ref{chapt.scr-unmarked}, and we also consider
another case in sec. \ref{XYZ} below.
A plot of that is in order. This near equivalence is shown in  Figure
XYZ. The left panel shows a plot of $\lambda_{ij}$ vs. distance and
superimposed on that is a plot of $p_{ij}$ vs. distance, for values
$\lambda_{0} = .1$ and $\sigma = 1$. The right panel shows a plot of
$\Pr(y>0)$ vs. $E[y]$ and we see therefore that the models are
practically equivalent. 

\begin{verbatim}
x<-seq(0.001,5,,200)
lam0<- .1
sigma<- 1
lam<- lam0*exp(-x*x/(2*sigma*sigma))

par(mfrow=c(1,2))
p1<- 1-exp(-lam)
plot(x,lam,ylab="E[y] or Pr(y>0)",xlab="distance",type="l",lwd=2)
lines(x,p1,lwd=2,col="red")
plot(lam,p1,xlab="E[y]",ylab="Pr(y>0)",type="l",lwd=2)
abline(0,1,col="red")
\end{verbatim}

So under the Poisson model we have
\[
\Pr(y>0) \approx E[y] = \lambda_{0} g(x,s)
\]
whereas in the binary model from chapter 4 we had precisely
\[
\Pr(y>0) \equiv E[y] = p_{0} g(x,s)
\]
and so the models are exactly the same for the {\it expected values}
and very similar for the probability of observing a positive response,
as long as $\lambda_{0}$ is small.


What all of this suggests it that
if we see very few observations $>1$ then we wont lose much
information by using the Bernoulli model. On the other hand, the
Poisson model is more easy to compute with in some cases. 


\begin{figure}
\centering
\includegraphics[width=5in,height=2.5in]{Ch5/figs/Poisson-Bern.png}
\label{fig:elevMap}
\end{figure}



Even if we're not in the range where the Bernoulli model provides a
good approximation, we might choose to truncate the counts to binary
observations anyhow (``quantize'').
We might do
this intentionally, but sometimes this truncation is a feature of the
sampling. For example, in the case of bear hair snares, the number of
encounters might be well approximated by a Poisson distribution but we
cannot determine unique visits and so only get to observe the binary
event ``$y>0$''. Similarly for scat sampling problems it will not
generally be possible to diagnose distinct ``independent'' scat
samples. Under this model the data are only binary encounters and we
might therefore choose a model of the form:
\[
 cloglog(p_{ij}) = log(\lambda0)  + log(g({\bf x},{\bf s}))
\]
\begin{comment} 
This example shows us that the choice of link function is typically
directly related to a specific encounter frequency model and,
furthermore, the choice of link function is equivalent to choice of
``detection function.''  As another example, what if the latent
encounter frequencies are actually geometric random variables where
the mean is a function of distance? For the case where the support of
y includes 0 – so that $y$ is the number of failures before the 1st
success, then the mean is $\mu = (1-p)/p$.  $Pr(y>0) =$ ??
\[
logit() = ….?
\]
\end{comment}

\subsection{A cautionary note on modeling encounter frequencies}

Other models for counts might be appropriate. For example, ecologists
are especially fond of negative binomial models for count data
\citep{verhoef_boveng:2007,
white_bennetts:1996,kery_etal:2005}
but other models for excess-Poisson variation are possible. For
example, we might add a normally distributed random effect to
the linear predictor.

As a general rule we favor the Bernoulli observation model even if
encounter frequencies are obtained by sampling.  The main reason is
that, with frequency data, we are forced to confront a model choice
problem (i.e., Poisson, negative binomial, log-normal mixture) that is
wholly unrelated to the fundamental space usage process that underlies
the genesis of SCR data. Repeated encounters over short time intervals
are not likely to be the result of independent encounter
processes. E.g., an individual moving back and forth in front of a
camera yields a cluster of observations that is not informative about
the spatial structure of the model. Similarly in scat surveys (e.g.,
Thompson et al. in review), dogs are used to locate scats which are
processed in the lab for individuality.  The process of local scat
deposition is not really the outcome of movement but rather the
outcome of complex behavioral considerations as well as dependence in
detection of scat by dogs. E.g., they find one and then more likely to
find a nearby one, or they get into a den area and find lots of scats.
This additional model assumption required to model variation in
observed frequencies (i.e., conditional on location) provides
relatively little information about density, and we feel that the
model selection issue should therefore be avoided.

To elaborate on this, it seems natural to construct models for
encounter data that is conditional on movement outcomes: We suppose
that an individual visits a particular location with some probability
$p_{ik}$ say $z_{ik}\sim  \mbox{Bern}(p_{ik})$ and then deposits a number of scat,
or visits a camera some number of times with frequency $y_{ik}$ which
is 
an integer $> 0$. Therefore, a sensible model might be
$[y|z][z|\phi({\bf x},{\bf s})$
where the encounter frequency $y$ is independent of ${\bf x}$ and
${\bf s}$ conditional
on the binary event ``$z$'' that the individual visited the vicinity of
the trap.

Moreover, consideration of encounter frequency data could lead to
important identifiability problems along the lines of Link (2003). The
basic Poisson model can be over-dispersed in a number of ways to
produce different models of over-dispersion.  i.e., gamma noise,
normal noise, exponential noise, etc..  Thus we have different models
of heterogeneity analogous to the class of models considered by \citet{link:2003}.


\section{Analysis of a Poisson SCR model in BUGS}

We consider the simplest possible model here in which we have no
covariates that vary over replicate samples $k$ so that we work with
the aggregated individual- and trap-specific encounters:
\[
y_{ij} = (\sum_{k=1}^{K} y_{ijk}) =  \mbox{Poisson}(K  \lambda_{ij})
\]
We consider a bivariate normal form of $g({\bf x}_{j},{\bf s}_{i})$ so
that
\[
g({\bf x}_{j},{\bf s}_{i}) = exp( -||{\bf x}_{j} - {\bf
  s}_{i}||^{2} /(2\sigma^{2}))
\]
In this case, note that 
\[
log( \lambda_{ij})  =\alpha_{0} - \beta ||{\bf x}_{j} - {\bf s}_{i}||^2
\]
where $\alpha_{0} = log(\lambda_{0})$ and $\beta = 1/(2\sigma^2)$.


As usual, we approach Bayesian analysis of these
models using data augmentation (section \ref{closed.sec.da}). 
It is interesting in this case that DA
gives us a sort of zero-inflated Poisson model which is amazingly easy
to analyze by likelihood methods which maybe we will do in Chapter
XYZ.

So the model specified conditional on $z_{i}$ is
\[
y_{ij} \sim  Poisson(z_{i} K  \lambda_{ij})
\]
which evaluates to a point mass at $y=0$ if $z=0$. 


\subsection{Simulating Data}

Simulating a sample SCR data set under the Poisson model requires only
a couple minor modifications to the procedure we used in chapter 4. In
particular, we modify the block of code which defines the model to be
that of $E[y]$ and not $\Pr(y=1)$, and we change the random variable
generator from \mbox{\tt rbinom} to \mbox{\tt rpois}:
\begin{verbatim}
D<- e2dist(S,traplocs)

alpha0<- -2.5
sigma<- 0.5
beta<- 1/(2*sigma*sigma)

muy <-  exp(alpha0)*exp(-beta*D*D)
# now generate the encounters of every individual in every trap
Y<-matrix(NA,nrow=N,ncol=ntraps)
for(i in 1:nrow(Y)){
 Y[i,]<-rpois(ntraps,K*muy[i,])
}
\end{verbatim}

We modified our code from SCR0 in chapter 4 to simulate Poisson
encounter frequencies for each trap and then we analyze an ideal data
set using WinBUGS. The new function, available in the R package, is called
{\tt simPoissonSCR.fn}. 
The simulator can produce 3-d encounter history data too although we
don't do that here. 
Here is an example of simulating a data set and harvesting the
required data objects:

\begin{verbatim}
data<-simPoissonSCR.fn(discard0=TRUE,sd=2013)
y<-data$Y
traplocs<-data$traplocs
nind<-nrow(y)
X<-data$traplocs
K<-data$K
J<-nrow(X)
Xl<-data$xlim[1]
Yl<-data$ylim[1]
Xu<-data$xlim[2]
Yu<-data$ylim[2]

## Data augmentation stuff
M<-200
y<-rbind(y,matrix(0,nrow=M-nind,ncol=ncol(y)))
z<-c(rep(1,nind),rep(0,M-nind))
\end{verbatim}

To execute WinBUGS the process is identical to what we've done
previously..............................................
here..................
.................................

The results are given below. We note about the same answer as before.

{\small
\begin{verbatim}
> print(out1,digits=2)
Inference for Bugs model at "SCR-Poisson.txt", fit using WinBUGS,
 3 chains, each with 2000 iterations (first 1000 discarded)
 n.sims = 3000 iterations saved
           mean    sd   2.5%    25%    50%    75%  97.5% Rhat n.eff
alpha0    -2.57  0.19  -2.95  -2.69  -2.57  -2.44  -2.19 1.00  2600
beta       2.34  0.36   1.69   2.08   2.32   2.57   3.12 1.00  3000
N        114.13 15.25  87.97 103.00 113.00 124.00 147.00 1.01   370
D          1.78  0.24   1.37   1.61   1.77   1.94   2.30 1.01   370
deviance 329.95 21.92 290.00 314.20 329.50 344.40 375.80 1.00  1700

For each parameter, n.eff is a crude measure of effective sample size,
and Rhat is the potential scale reduction factor (at convergence, Rhat=1).

DIC info (using the rule, pD = var(deviance)/2)
pD = 240.2 and DIC = 570.2
DIC is an estimate of expected predictive error (lower deviance is better).
\end{verbatim}


At the end of this chaptter we provide an example of a Poisson SCR model fitted to 
real data. This example has some other features which we encounter before
arriving there. 

\subsection{Exercise}

Use the Bernoulli model simulator from Chapt. \ref{chapt.scr0} (\mbox{\tt
  simSCR0.fn}) to simulate a Bernoulli data set and then fit the
Poisson model. Compare the results of fitting the correct
data-generating model with those of fitting the misspecified Poisson
model. 



\begin{comment}
\section{Likelihood analysis of the Poisson model}

Counts are Poisson with a random effect so this is stupidly easy to
implement. 
We do the normal ``full likelihood'' approach in which we retain $N$
as a real parameter in the model. We adapt \mbox{\tt intlik3} from
chapter 5 here..... behold:

Poisson(lambda(s,x))

data augmentation = ZIP
\begin{verbatim}
Pr(yi) =   ( prod_{j} dpois(y) ) *psi + I(y=0)*(1-psi)

Actually if y(i,j) = Poisson( lambda(i,j) ) then we can just add up
sum_{j} y(i,j) =  Poisson( sum_{j} lambda(i,j)) right?

 int_{s} thatthing

Zero-inflate the result
\end{verbatim}
\end{comment}

\section{No real example}

In chapt. \ref{chapt.searchencounter} we analyze the cap crap data.


\section{Independent Multinomial Observations}

Several types of encounter devices yield multinomial observations in
which an individual can be caught in a single trap during a particular
encounter occasion.  Mist nettting is a major example -- these are
``multi-catch'' traps (Efford XYZ NEED REF HERE XXXX). Also some kinds of
mammal traps hold multiples of animals and can be thought of
similarly. Another one is area-searches of reptiles where we think of
a small polygon as the ``trap'' -- we could get multiple individuals
(turtles, lizards) in the same plot but not, in the same sample
session, at different plots.  The key feature is that capture of an
individual in a trap is {\it not} independent of capture in other
traps, because they can't be captured once they are captured. On the
other hand individuals behave independently of one another, or so it
might be reasonable to assume, so whether a trap captures some other
individual doesn't have bearning on whether it captures another.  This
last assumption is violated in an extreme case in classical ``single
catch'' traps which we address in section \ref{poisson-mn.sec.singlecatch}
below. In general we could imagine non-independence being important in
any multi-catch situation but to the best of our knowledge a general
model that encompasses complete dependence (single catch) and complete
independence (multi-catch) of individuals has not been proposed.  So
we treat the cases individually and, in this section , we address the
multi-catch situation wherein individuals behave independently.


In this case we regard the observation ${\bf y}_{ik}$ for
individual $i$ during sample occasion $k$ as a multinomial observation
which consists of a sequence of 0's and at most a single 1 indiciating
the trap of capture. For example, if we capture an individual in trap
2 during a 6 sample period study then ${\bf y}_{i} = (0,1,0,0,0,0)$.
If we sample for 5 periods in all and the individual is also caught
in trap 4 during sample 3, then the 5 encounter observations for that
individual are as follows:
\begin{verbatim}
sample |---- trap ---------|
       1   2   3   4   5   6
 1     0   1   0   0   0   0
 2     0   0   0   0   0   0
 3     0   0   0   1   0   0
 4     0   0   0   0   0   0
 5     0   0   0   0   0   0
\end{verbatim}
Statistically we regard the {\it rows} of this data matrix as {\it
  independent} multinomial trials.

Analogous to our previous Bernoulli and Poisson models, we seek to
construct the multinomial cell probabilities for each individual, as a
function of {\it where} that individual lives, through its center of
activity ${\bf s}$. Thus we suppose that
\[
 {\bf y}_{ik} \sim \mbox{Multinom}(1, {\bm \pi}({\bf s}_{i}) )
\]
where ${\bm \pi}({\bf s}_{i})$ is a vector of length $J+1$, which, by
convention here, we define $\pi_{i,J+1}$, the last cell, or the ``zero
cell'', to correspond to the event ``not captured''.  Now we have to
construct these cell probabilities in some meaningful way that depends
on each individuals' ${\bf s}$, which we do shortly.

A statistically equivalent distribution is the {\it categorical} distribution.
If ${\bf y}$ is a multinomial trial with probabilities
${\bm \pi}$ than the {\it position} of the non-zero
elemment of ${\bf y}$ is a categorical random variable with probabilities
${\bm \pi}$.
We express this as
\[
{\bf y} \sim \mbox{Categorical}( {\bm \pi} )
\]
In the context of SCR models the categorical version of the
multinomial trial corresponds to the {\it trap of capture}.  Using our
example above with 6 traps then ${\bf y}_{i1} = (0,1,0,0,0,0)$ then we
could as well say $y_{ik}$ is a categorical random variable with
possible outcomes $(1,2,3,4,5,6,7)$ where outcome $y=7$ corresponds to
``not captured'' (obviously how this is organized or labeled is
completely irrelevant, although it is convenient to use the integers
$1:(J+1)$).  Therefore, $y_{i1} = 2$, $y_{i2} = 7$, $y_{i3} = 4$ and
so on.

For simulating and fitting data in the {\bf BUGS} engines we will typically use
the categorical representation of the model because it is somewhat
more convenient.  We have found that fitting multinomial models in
{\bf WinBUGS} can be extremely inefficient whereas {\bf JAGS}
typically performs much better. In the examples here, we use {\bf
  JAGS} exclusively.

\subsection{Multinomial Relationship to Poisson}

The multinomial is related directly to the Poisson encounter rate
model in the following sense. Let $y_{ij}$ be the total number of
encounters for individual $i$ in trap $j$. Then, the trap frequencies
(expluding the last cell now), if we condition on the {\it total}
number of captures, $y_{i} = \sum_{j} y_{ij}$, are multinomial with
probabilities
\[
 \pi_{ij} =  \frac{ \lambda_{ij} } { \sum_{j} \lambda_{ij} } 
\]
for $j=1,2,\ldots,J$.
Or equivalently the {\it trap of
  capture} is categorical with probabilities
\[
 \pi_{ij} =  \frac{ \lambda_{ij} } { \sum_{j} \lambda_{ij} } 
\]
which is precsely, under the half normal model, 
\[
 \pi_{ij} =  \frac{ \exp( - \beta \! dist({\bf x},{\bf s})^2 ) }  {
   \sum_{j} \exp(-\beta \! dist({\bf x},{\bf s})^2)}
\]
This expression looks like a multinomial inverse-logit transform of a model having
quadratic distance term, and also ``maximum entropy'' from MAXENT
species distribution modeling, or resource utilzation distribution
from telemetry studies.
So we can think of this multinomial model as arising naturally 
by having Poiosson encounters and then conditioning on the total. 
It is a sensible model to have anyhow, as it just allocates captures
to traps in proportion to the square of distance.  We could try other
models here too (Note: What do Borchers and Efford 2008 do?).

%People might think this multinomioal model is somehow more general
%than assuming Poisson encounter frequencies since we might cook up the
%multinomioal without having to specify a distribution for
%$y_{i}$. That said, we note that it arises under 
%If we now uncondition on the total..... 
%$y_{ij}$ is Poisson with mean $\sum_{j}$ stuff... we have a product of
%Poissons, i.e., the model we started with. 

The interpretation of this model merits some discussion. That is, 
{\it given that an individual is captured}, the probabilities given by
eq. XYZ determine 
the distribution among traps. To fully specify the model, we have to
model the probability that an individual is captured, say $p$.

We deduced the multinomioal by assuming a Poisson distribution
..... so
where did this $p$ come from?

So lets not worry about the distribution of the total count
but instead estimated this excess parameter p (this is what Royle et al.
and Gardner et al. tried to do).  In this case the multinomial gets
another cell probability , the J+1 cell, 
\[
 \pi\_{ij} =  \frac{ p exp( - beta d^2 ) }  { \sum\_{j} exp(-beta d^2)}
\]
and the last cell
\[
 \pi\_{i,J+1} =  1-p 
\]

What i like about this particular multiomial model is that whether or not
an individual is encounter in trap $j$ is just a Bernoulli trial with
probability
\[ 
(p/stuff)*exp(-beta*d^2)
\]
and if we just label (p/stuff) = p0 then this is precisely our
Bernoulli model with a half-normal detection model.  Thus we ``condition
on $y_{ij}$ and we dont have to fess up to a model for this encounter
rate, which is most of the time just reflecting behavioral stuff of the
species under, study and we wind up with a basic default Bernoulli model
which doesn't require any assumptions about the encounter rate of 
individuals.  So not having to model encounter rate seems like a good
benefit of the Bernoulli model -- which is why we said what we did above.


\subsection{Simulating data and fitting in WinBUGS}

We're going to show the nugget of a simulation function which is
used in the function \mbox{\tt sim.mnSCR} found in the {\bf R} package
\mbox{\tt scrbook}.  The first lines of the following {\bf R} code
make use of some things that should be defined but we omit them here:
{\small
\begin{verbatim}
S<-cbind(runif(N,Xl,Xu),runif(N,Yl,Yu))
# how far is each individual from each trap?
D<- e2dist(S,traplocs)

# paramter values
sigma<- 0.5
alpha0<- -1
theta<- 1/(2*sigma*sigma)

# make an empty data matrix and fill it up
Ycat<-matrix(NA,nrow=N,ncol=K)
for(i in 1:N){
for(k in 1:K){
lp<- alpha0 - theta*D[i,]*D[i,]
cp<- exp(c(lp,0))
cp<- cp/sum(cp)
Ycat[i,k]<- sample(1:(ntraps+1),1,prob=cp)
}
}
\end{verbatim}
}
The resulting data matrix in this case has the maximal dimension $N$
and so, for analysis, to mimic a real situation, we would have to discard the uncaptured
individuals. 
\mbox{\tt sim.mnSCR} will also simulate data that includes a
behavioral response, which will be the typical situation in
small-mammal trapping problems, which we first developed this code to
deal with \citep[see][for details]{converse_royle:2012}.

Here we use our function \mbox{\tt sim.mnSCR} to simulate a data set
with $K=7$ periods, etc.. We'll run the model using {\bf JAGS} which we
have found is much more effective for this class of models.
We get the data set-up for analysis by augmenting the size of the data
set to $M=200$. In addition we choose starting values for ${\bf s}$ and the
data augmentation variables $z$.  For ${\bf s}$ here we cheat a little bit
and use the true values for the obseved individuals and then augment
the matrix ${\bf S}$ with $M-n$ randomly selected activity centers.

{\small 
\begin{verbatim}
set.seed(2013)
parms<-list(N=100,alpha0= -.40, alpha1= 0,sigma=0.5)
data<-sim.mnSCR(parms,K=7,ssbuff=2)
nind<-nrow(data$Ycat)

M<-200
Ycat<-rbind(data$Ycat,matrix(nrow(data$X)+1,nrow=(M-nind),ncol=data$K))
Sst <-rbind(data$S,cbind(runif(M-nind,data$xlim[1],data$xlim[2]),
                         runif(M-nind,data$ylim[1],data$ylim[2])))
zst<-c(rep(1,160),rep(0,40))
\end{verbatim}
}

The model specification is not much more complicated than the binomial
or Poisson models given previously. The main consideration is that we
define the cell probabilities for each trap $j=1,2,\dots,J$ and then
define the last cell probability, $J+1$, for ``not captured'', to be
the complement of the sum of the others. The code is shown in Panel
\ref{poisson-mn.panel.mn}.
In the last lines of code here we
specify $N$ and density, $D$, as  derived parameters.


\begin{panel}[htp]
\centering
\rule[0.15in]{\textwidth}{.03in}
%\begin{minipage}{2.5in}
{\small
\begin{verbatim}
cat("
model {
psi ~ dunif(0,1)
alpha0 ~ dnorm(0,10)
sigma ~dunif(0,10)
theta<- 1/(2*sigma*sigma)

for(i in 1:M){
  z[i] ~ dbern(psi)
  S[i,1] ~ dunif(xlim[1],xlim[2])
  S[i,2] ~ dunif(ylim[1],ylim[2])
  for(j in 1:ntraps){
    #distance from capture to the center of the home range
    d[i,j] <- pow(pow(S[i,1]-X[j,1],2) + pow(S[i,2]-X[j,2],2),1)
  }
  for(k in 1:K){
    for(j in 1:ntraps){
      lp[i,k,j] <- exp(alpha0 - theta*d[i,j])*z[i]            
      cp[i,k,j] <- lp[i,k,j]/(1+sum(lp[i,k,]))
    }
    cp[i,k,ntraps+1] <- 1-sum(cp[i,k,1:ntraps])  # last cell = not captured
    Ycat[i,k] ~ dcat(cp[i,k,])
  }  
}   

N <- sum(z[1:M]) 
A <- ((xlim[2]-xlim[1])*trap.space)*((ylim[2]-ylim[1])*trap.space)
D <- N.tot/A
}
",file="model.txt")

\end{verbatim}
}
%\end{minipage}
\rule[-0.15in]{\textwidth}{.03in}
\caption{
WinBUGS model specification for the multinomial observation model. 
}
\label{poisson-mn.panel.mn}
\end{panel}

Finally we need to package everything up (inits, parameters, data) and send
it off to {\bf JAGS} to build a MCMC simulator for us:

{\small
\begin{verbatim}
library("rjags")

inits <- function(){list (z=zst,sigma=runif(1,.5,1) ,S=Sst) }              
parameters <- c("psi","alpha0","theta","sigma","N","D")
data <- list (X=data$X,K=data$K,trap.space=1,Ycat=Ycat,M=M,
              ntraps=nrow(data$X),ylim=data$ylim,xlim=data$xlim)         

out1 <- jags.model("model.txt", data, inits, n.chains=3, n.adapt=500)
out2 <- coda.samples(out1,parameters,n.iter=1000)
\end{verbatim}
}


Summary of analysis for the simulated data set here.....  



\section{ Mist-netting example}

Here we do an analysis of a real data set using the multinomial model.
the data are for 
adult Arctic Warblers ({\it Phylloscopus borealis}) banded 
 along the Colville River near Umiat, Alaska in 2006. The data are from 
 5 MAPS (REF) stations located in close proximity of one another, as
 well as 
 birds target (netids starting with "UMIA") or passive (netids starting 
 with "PASS") netted in the general area (a couple of nets, 
 netid == 'PASS01' and 'UMIAB3' are pretty far away though...). In total, 
 there are 258 captures of 179 individual birds. This is is really a 
 large number of birds of a single species for MAPS stations. 
 
Each of these MAPS stations has 12-15 nets.

We used XYZ....
 
A few issues:
 data is manipulated into multinomial trials and we have to convert.
 multiple captures somehow.....
 lots of space.
 transient individuals?  affect is N = number of guys ``ever available''
 


\section{SCR Models are Multi-State Models}

\begin{comment}
SCR models are multi-state models where stat-especific encounter
probabilities are a function of distance -- or something like that. 
\end{comment}

This multinomial observation model and also the discrete formulation
of the state-model given in section XYZ both allude to the fact that
SCR models are a variation of 
ordinary multi-state models \citep[][Chapt. 9]{kery_schaub:2011}
but where the state variable is static and represents a
geographic location. Multi-state models are extremely useful for
modeling movements among geographic states and indeed this type of
problem motivated their early developments by
\citet{arnason:1972,arnason:1973} and 
\citet{hestbeck_etal:1991} albeit in the context of a dynamic state
variable.  

Sometimes the state-space is naturally discrete. Consider a situation
in which discrete patches of habitat are searched using some method
and it might be convenient (or occur inadvertently) to associate
samples to the patch level instead of recording observation locations,
as in the capracillie example given in section XYZ above.  In this
case we use the discrete analog of the ``uniformity assumption'' in
which ${\bf s}_{i} \sim dcat(probs[])$ where $probs[]$ are the
probabilities that an individual inhabits a particular patch which
should be proportional to area of each patch.  Even when space is {\it
  not} naturally discrete, measurements are often made at a fairly
coarse grain (e.g., meters or tens of meters along a stream), or
associated with spatial quadrats for scat searches. And, of course, we
could approximate any continuous space with a discrete state-space,
and therefore apply multi-state models directly to any SCR problem.

\subsection{Modeling ‘manders on a stream network}

Here is a cool example: We catch salamander’s or fish along a stream
and only record stream segment instead of actual location – this is
motivated by Evan Grant’s work and also Lowe xyz??

each stream segment is individuals current state and its easy to use
either a Markov model or a home range model. ....

This is also a good open population example

\subsection{SCR as a Dynamic multi-state model}

Having a static state variable is not that interesting in the grand
scheme of multi-state models which most of the time consider a dynamic
state variable. Such models will arise frequently in spatial
capture-recapture settings. Let s denote the individual activity
center and suppose its state-space is discrete.  Now let $u[i,t]$ be
the patch in which individual $i$ was observed during sample $t$. Then
a simplistic movement model is that the successive movement outcomes
are $iid$
\[
u[i,t] \sim  dcat[ psi[s[i],] ]
\]

We can reformulate the basic SCR0 model as a dynamic multi-state model
as follows.  First lets grid up the state-space into “survey strata”
which we might define here has .5 unit squares so that the whole
state-space has 16*16 such squares. [actually do this so they are
centered on traps].We retain our assumption
\[
 s_{i} \sim Uniform(S)
\]
Secondly we define a movement model in which
\[
u[I,t] \sim dcat(pi)
\]
Where
\[
 pi_{k} = exp(-dist(x,s)/sigma2)/sum[that]
\]
This is the MAXENT distribution but also corresponds to Poisson with
mean $lam0*exp(-dist^2/sigma)$.  THIS IS CRUCIAL – THIS IS IMPT!!
 Makes it clear that encounter is the same as movement.

Now define
\[
 p|u[i,t] = p0*if(u[i,t] \in trap grid cell)
\]

Multi-state model with a “random movement” process.


We could easily extend this to a kind of Markovian movement model
where the probabilities depend on the previous state $u_{i,t-1}$ but
the simpler model of ``random'' movement satisfies our immediate needs.
 
So we see that SCR models are exactly a type of multi-state model when
the states are naturally discrete.  Another naturally discrete
state-space is ``nest sites''. Goncalo’s study and Florent’s
study. Schaub’s study on woopoos.


\section{Single-catch traps}
\label{poisson-mn.sec.singlecatch}

The classical animal trapping experiment is based on a physical trap
which captures a single animal and holds that individual until
subsequent molestation by a biologist. 
This type of observation model -- the ``single catch'' trap -- 
was the original situation considered by \citet{efford:2004}.

The single-catch model is basically a multinomial model but one in
which the number of available traps is reduced as each individual is
captured. As such, the constraints on the likelihood for each
individual are latent and shit is complicated beyond belief.
As a result, at the time of this writing, there has not been a formal
development of either likelihood  or Bayesian analysis of this model.

Nevertheless, it is not too difficult to describe the basic model
formally. In particular, there is a nice conditional structure resulting from a ``removal
process'' operating on the traps.  The first guy captured has the 
basic multinomial observation model:
\[
{\bf y}_{i} \sim Multinom({\bm \pi}_{i})
\]
whereas the 2nd guy captured has one cell removed:
\[
{\bf y}_{i} \sim Multinom({\bm \pi}_{i}(1-{\bf y}_{i})    )
\]
and so on.
So the {\bf order of capture} is relevant to the construction of these
multinomial cell probabilities. 
Thus the observations each have a multinomial model, but the
multinomial observations have a unique kind of conditional dependence
structure among them.

\subsection{Approximate Analysis}

To analyze the model here we consider using a misspecified model based
on either the Poisson or independent multinomial


How good of an approximation is the multi-catch model?

What about the Poisson model with a really low lambda?

Can we solve the big kahuna?

Use Sarah's data here.


\section{Trapping Webs}


\section{Acoustic Arrays}


\section{Summary and Outlook}

There are other types of encounter models.......

Efford adapts SCR models to acoustic detection devices.... a few words
on that here.....

There are models for which
only specific summary statistics are observable (Chandler and Royle
2011, etc..) which we cover in chapter XYZ.  We consider other models
for detection probability in some prior or later chapter. 







\chapter{
Likelihood Analysis of Spatial Capture-Recapture Models
}
\markboth{Chapter 5}{}
\label{chapt.mle}

%%%% TO-DO LIST
% 1. comparison of Bayes with MLE for wolverine data (need to rerun WinBUGS)
% 2. Beth clean up a couple things in SECR analysis.
% 3. Need to finish MLE for restricted state-space
%%   requires code from Rahel
% 4. draft up intlik3 wrapper "scr()"

\vspace{.3in}



In this book we mainly focus on Bayesian analysis of spatial
capture-recapture models. And, in the previous chapters we learned how
to fit some basic spatial capture-recapture models using a Bayesian
formulation of the models analyzed in BUGS engines including {\bf
  WinBUGS} and {\bf JAGS}.  Despite our focus on Bayesian analysis, it
is instructive to develop the basic conceptual and methodological
ideas behind classical analysis based on likelihood methods and
frequentist inference.  
This has been the approach taken by
\citet{borchers_efford:2008, dawson_efford:2009} and related papers.
Simple SCR models can be analyzed
fairly easily using such methods and, including even some classes of
models that we have not been able to fit using Bayesian methods. One
such class of models are those
that account for ecological distance in the detection model,
which we cover in  Chapt. \ref{chapt.ecoldist}).


This chapter provides some conceptual and technical footing for
likelihood-based analysis of spatial capture-recapture models. We
recognized earlier (Chapt. 4) that SCR models are versions of
binomial (or other) GLMs, but with random effects – i.e., GLMMs. These
models are 
routinely analyzed by likelihood methods. In particular, likelihood
analysis is based on the integrated likelihood in which the random
effects are removed by integration from the likelihood. In SCR models,
the random effect, ${\bf s}$, i.e., the 2-dimensional coordinate, is a
bivariate random effect. 

In this chapter, we show that it is
straightforward to compute the maximum likelihood estimates (MLE) for
SCR models by integrated likelihood. We develop the MLE framework
using {\bf R}, and we also provide a basic introduction to an {\bf R} package
\mbox{\tt secr} \citep{efford:2011} which is based on the stand-alone
package 
{\bf DENSITY} \citep{efford_etal:2004}.
 To set the context we analyze the SCR model
here when $N$ is known because, in that case, it is precisely a GLMM and
does not pose any difficulty at all. We generalize the model to allow
for unknown $N$ using both conventional ideas based on the ``joint
likelihood'' \citep[e.g.,][]{borchers_etal:2002}
and also using a formulation
based on data augmentation.  We obtain the MLEs for 
the SCR model from the wolverine camera trapping study \citep{magoun_etal:2011}
 analyzed in previous chapters to compare/contrast the
results.

\section{Likelihood analysis }

We noted in chapter 4 that, with $N$ known, the basic SCR model is a
type of binomial regression with a random effect. For such models we
can easily obtain maximum likelihood estimators of model parameters
based on integrated likelihood. The integrated likelihood is based on
the marginal distribution of the data $y$ in which the random effects
are removed by integration. Conceptually, our model is a specification
of the conditional-on-${\bf s}$ model $[y|{\bf s},\alpha]$ and we have
a ``prior distribution'' for ${\bf s}$, say $[{\bf s}]$, and the
marginal distribution of the data $y$ is
\[
[y|\alpha] =  \int_{\bf s} [y|{\bf s},\alpha][{\bf s}] d{\bf s}.
\]
When viewed as a function of $\alpha$ for purposes of estimation, the
marginal distribution $[y|\alpha]$ is often referred to as the {\it
  integrated likelihood}.

It is worth analyzing 
the simplest SCR model with known-$N$ in order to understand the
underlying mechanics and basic concepts. These are directly relevant to
the manner in which many capture-recapture models are classically
analyzed, such as model $M_h$, and individual covariate models (see
Chapt. \ref{chapt.closed} and  \citet[][chapt. 6]{royle_dorazio:2008}). To develop integrated
likelihood for SCR models, we first identify the conditional
likelhiood. 

The observation model for each encounter observation $y_{ij}$,
specified conditional on ${\bf s}_{i}$, is 
\begin{equation}
	y_{ij}| {\bf s}_{i} \sim \mbox{Bin}(K, p_{\alpha}({\bf x}_{j},{\bf s}_{i}))
\label{mle.eq.cond-on-s}
\end{equation}
where we have indicated the dependence of $p_{ij}$ on ${\bf s}$ and
parameters $\alpha$
explicitly.
For the random effect we have ${\bf s}_{i} \sim  \mbox{Unif}({\cal
  S})$.
The joint distribution of the data for individual $i$ is the product
of $J$ such terms (i.e., contributions from each of $J$ traps).
\[
  [{\bf y}_{i} | {\bf s}_{i} , \alpha] = 
  \prod_{j=1}^{J} \mbox{Bin}(K, p_{\alpha}({\bf x}_{j},{\bf s}_{i}) )
\]
We note this assumes that encounter of individual $i$ in each
trap is independent of encounter in every other trap, conditional on
${\bf s}_{i}$, this is the fundamental property of the basic model SCR0.


 The so-called marginal likelihood is computed by removing
${\bf s}_{i}$, by integration (hence also {\it integrated} likelihood), from the conditional-on-${\bf s}$
likelihood and regarding the {\it marginal} distribution of the data
as 
the likelihood. That
is, we compute:
\[
  [y|\alpha] = 
\int_{{\cal S}}  [ {\bf y}_{i} |{\bf s}_{i}, \alpha] g({\bf s}_{i}) d{\bf s}_{i}
\]
In most SCR models, $g({\bf s}) = 1/||{\cal S}||$ (but see Chapt. \ref{chapt.state-space}).

The joint likelihood for all $N$ individuals, assuming independence of
encounters among individuals, is the product of $N$ such terms:
\[
{\cal L}(\alpha | {\bf y}_{1},{\bf y}_{2},\ldots, {\bf y}_{N}) =     \prod_{i=1}^{N}
[{\bf y}_{i}|\alpha]
\]
We emphasize that two independence assumptions are explicit in this
development: independence of trap-specific encounters within
individuals and also independence among individuals. In particular,
this would only be valid when individuals are not physically
restrained or removed upon capture, and when traps do not ``fill up''.

The key operation for computing the likelihood is solving a
2-dimensional integration problem. There are some general purpose {\bf
  R} packages that implement a number of 
 multi-dimensional integration routines
including \mbox{\tt adapt} \citep{genz_etal:2007} and \mbox{\tt R2cuba}
\citep{hahn_etal:2011}.  In practice, we won't rely
on these extraneous {\bf R} packages (except see
chapt. \ref{chapt.state-space} for an application of \mbox{\tt Rcuba})
but instead will use perhaps less
efficient methods in which we replace the integral with a summation
over an equal area mesh of points on the state-space ${\cal S}$ and explicitly
evaluate the integrand at each point. We invoke the rectangular rule
for integration here\footnote{e.g., 
\url{http://en.wikipedia.org/wiki/Rectangle_method}
} in which we
evaluate the
integrand on a regular grid of points of equal area and compute the
average of
the integrand over that grid of points. 
Let $u=1,2,\ldots,nG$ index a grid of
$nG$ points, ${\bf s}_{u}$,  where the area of grid cell $u$ is
constant, say $A$.
In this case, the integrand, i.e., the marginal pmf of 
${\bf y}_{i}$, is approximated by  
\begin{equation}
         [{\bf y}_{i}|\alpha] = \frac{1}{nG} \sum_{u=1}^{nG}  [ {\bf
            y}_{i} |{\bf s}_u, \alpha]
\label{mle.eq.intlik}
\end{equation}

This is a specific case of the general expression that could be used
for approximating the integral for any arbitrary (bivariate or otherwise)
distribution $g({\bf s})$. The general case is
\[
[y]  = \frac{A}{nG} \sum_{u} [y|{\bf s}_{u}] [{\bf s}_{u}]
\]
 In the present context it happens that  $[{\bf s}] = (1/A)$
and thus the grid-cell area cancels in the above
expression to yield eq. \ref{mle.eq.intlik}.
The rectangular rule for integration can be seen as an application of
the Law of Total Probability for a discrete random variable ${\bf
  s}$, having $nG$ 
unique values with equal probabilities $1/nG$.



\subsection{ Implementation (simulated data)}

Here we will illustrate how to carryout this integration and
optimization based on the integrated likelihood using simulated data
 (i.e., following that from Chapter 4). Using \mbox{\tt simSCR0.fn}
 we simulate data for 100 individuals and a 25 trap array
laid out in a $5 \times 5$ grid of unit spacing.  The specific encounter
model is the half-normal model. The 100 activity centers were
simulated on a state-space defined by a $8 \times 8$ square 
within which the
trap array was centered (thus the trap array is buffered by 2
units). Therefore, the density of individuals in this system is fixed
at $100/64$.

In the following set of {\bf R} commands we generate the data and 
then harvest the required data objects:
{\small
\begin{verbatim}
data<-simSCR0.fn(discard0=FALSE,sd=2013)
y<-data$Y
traplocs<-data$traplocs
nind<-nrow(y)
X<-data$traplocs
J<-nrow(X)
K<-data$K
Xl<-data$xlim[1]
Yl<-data$ylim[1]
Xu<-data$xlim[2]
Yu<-data$ylim[2]
\end{verbatim}
}
Now we need to define the integration grid, say ${\bf G}$, which we do with
the following set of {\bf R} commands (here, \mbox{\tt delta} is the grid spacing):
{\small
\begin{verbatim}
delta<- .2
xg<-seq(Xl+delta/2,Xu-delta/2,by=delta) 
yg<-seq(Yl+delta/2,Yu-delta/2,by=delta) 
npix<-length(xg)          # assumes xg and yg same dimension here
area<- (Xu-Xl)*(Yu-Yl)/((npix)*(npix)) # don’t need area for anything
G<-cbind(rep(xg,npix),sort(rep(yg,npix)))
nG<-nrow(G)
\end{verbatim}
}
In this case, the integration grid is set up as a grid with spacing
$\delta = 0.2$ which produces a $40 \times 40$ grid of points for evaluating the
integrand if the state-space buffer is set at 2.

We next create an {\bf R} function that defines the likelihood as a function
of the data objects $y$ and $X$ which were created above but, in general,
you would read these files into {\bf R}, e.g., from a .csv file.
In addition to these data
objects, we need to have defined the  quantities $G$ and $nG$ associated
with the integration grid.
However, instead of worrying about making all of these objects and
keeping track of them we just put that code above into the likelihood
function, say \mbox{\tt intlik1}, and pass $\delta$ as an additional (optional) argument and a
few other things that we need such as the boundary of the state-space
over which the integration (summation) is being done. This function is
available in the package \mbox{\tt scrbook}, and it is reproduced here:

{\small 
\begin{verbatim}
intlik1<-function(parm,y=y,delta=.2,X=traplocs,ssbuffer=2){

Xl<-min(X[,1]) - ssbuffer 
Xu<-max(X[,1]) + ssbuffer
Yu<-max(X[,2]) + ssbuffer
Yl<-min(X[,2]) - ssbuffer

xg<-seq(Xl+delta/2,Xu-delta/2,,length=npix) 
yg<-seq(Yl+delta/2,Yu-delta/2,,length=npix) 
npix<-length(xg)

G<-cbind(rep(xg,npix),sort(rep(yg,npix)))
nG<-nrow(G)
D<- e2dist(X,G)  

alpha0<-parm[1]
alpha1<-parm[2]
probcap<- plogis(alpha0)*exp(-alpha1*D*D)
Pm<-matrix(NA,nrow=nrow(probcap),ncol=ncol(probcap))
                    # all zero encounter histories
n0<-sum(apply(y,1,sum)==0) 
                    # encounter histories with at least 1 detection
ymat<-y[apply(y,1,sum)>0,] 
ymat<-rbind(ymat,rep(0,ncol(ymat)))
lik.marg<-rep(NA,nrow(ymat))
for(i in 1:nrow(ymat)){
Pm[1:length(Pm)]<- (dbinom(rep(ymat[i,],nG),K,probcap[1:length(Pm)],log=TRUE))
lik.cond<- exp(colSums(Pm))
lik.marg[i]<- sum( lik.cond*(1/nG))  
}
nv<-c(rep(1,length(lik.marg)-1),n0)
-1*( sum(nv*log(lik.marg)) )
}
\end{verbatim}
}


The function accepts as
input the encounter history matrix, $y$, the trap locations, $X$, and the
state-space buffer. This allows us to vary the state-space buffer and
easily evaluate the sensitivity of the MLE to the size of the
state-space. 
Note that we have a peculiar handling of the encounter history
matrix $y$. In particular, we remove the all-zero encounter histories
from the matrix and tack-on a single all-zero encounter history as the
last row which then gets weighted by the number of such encounter
histories (\mbox{\tt n0}). This is a bit long-winded and strictly unnecessary
when $N$ is known, but we did it this way because the extension to the
unknown-$N$ case is now transparent (as we demonstrate in the following
section). 
 The matrix \mbox{\tt Pm} holds the log-likelihood contributions of
each encounter frequency for each possible state-space location of the
individual. 
The log contributions are summed up and the result
exponentiated on the next line, producing lik.cond, the
conditional-on-${\bf s}$ likelihood (Eq. \ref{mle.eq.cond-on-s}
above). The marginal
likelihood (\mbox{\tt lik.marg}) sums up the conditional elements weighted by
$\Pr({\bf s})$ (Eq. \ref{mle.eq.intlik} above).
This is a fairly primitive function which doesn't allow much
flexibility in the data structure. For example, it assumes that $K$,
the number 
of replicates, is constant for each trap. Further, it assumes that the
state-space is a square. We generalize this to some extent later in
this chapter. 

Here is the {\bf R} command for maximizing the likelihood and saving the
results into an object called \mbox{\tt frog}.  The output is a list of the
following structure and these specific estimates are produced using
the simulated data set:

{\small 
\begin{verbatim}
# should take 15-30 seconds

starts<-c(-2,2)
frog<-nlm(intlik1,starts,y=y,delta=.1,X=traplocs,ssbuffer=2,hessian=TRUE)
frog

$minimum
[1] 297.1896

$estimate
[1] -2.504824  2.373343

$gradient
[1] -2.069654e-05  1.968754e-05

$hessian
          [,1]      [,2]
[1,]  48.67898 -19.25750
[2,] -19.25750  13.34114

$code
[1] 1

$iterations
[1] 11
\end{verbatim}
} 
Details about this output can be found on the help page for
\mbox{\tt nlm}. We note briefly that \mbox{\tt frog\$minimum} is the
negative log-likelihood value at the MLEs, which are stored in the
\mbox{\tt frog\$estimate} component of the list. The Hessian is the
observed Fisher information matrix, which can be inverted to obtain
the variance-covariance matrix using the commands:
\begin{verbatim}
> solve(frog$hessian)
\end{verbatim}

It is worth drawing attention to the fact that the estimates are
different than the Bayesian estimates reported previously in Chapt. \ref{chapt.scr0}.
How can that be?!  There are several reasons for
this.  First Bayesian inference is based on the posterior distribution
and it is not generally the case that the MLE should correspond to any
particular value of the posterior distribution. If the prior
distributions in a Bayesian analysis are uniform, then the
(multivariate) mode of the
posterior is the MLE, but note that Bayesians almost always report
posterior {\it means} and so there will typically be a discrepancy
there. Secondly, we have implemented an approximation to the integral
here and there might be a slight bit of error induced by that. We will
evaluate that shortly. Third, the Bayesian analysis by MCMC is subject
to some amount of Monte Carlo error which the analyst should always be
aware of in practical situations.  All of these different explanations
are likely responsible for some of the discrepancy. Accounting for
these, we see general consistency between the
two estimates.

To compute the integrated likelihood we used a discrete representation
of the state-space so that the integral could be approximated as a
summation over possible values of ${\bf s}$ with each value being
weighted by its probability of occurring, which is $1/nG$ under the
assumption that ${\bf s}$ is uniform on the state-space ${\cal
  S}$. Recall
in Chapt. \ref{chapt.scr0} we 
used a discrete state-space in developing a Bayesian analysis of the
model in order to be able to modify the state-space in a flexible
manner. In that case, we could use the discretized state-space as the
integration grid and just feed it into our integrated likelihood
routine. 

In summary, we note that, for the basic SCR model, integrated
likelihood is a really easy calculation when $N$ is known. Even for $N$
unknown it is not too difficult, and we will do that shortly.
However, if you can solve the known-$N$ problem then you should be able
to do a real analysis, for example by considering different values of
$N$ and computing the results for each value and then making a plot of
the log-likelihood or AIC and choosing the value of $N$ that produces
the best log-likelihood or AIC. As a homework problem we suggest that
the reader take the code given above and try to estimate $N$ without
modifying the code – by just repeatedly calling that code for
different values of $N$ and trying to deduce the best value.
We will formalize the unknown-$N$ problem shortly.

The
software package {\bf DENSITY} \citep{efford_etal:2004} implements
certain types of SCR models using integrated likelihood methods, and
\mbox{\tt secr} \citep{efford:2011} is an {\bf R} package with similar functionality.
We provide an analysis of some data using \mbox{\tt secr} shortly along
with a discussion of its capabilities, and we use \mbox{\tt secr} in
later chapters for likelihood analysis of other SCR models.


\section{MLE when N is Unknown} 

Here we build on the previous introduction to integrated likelihood
but we consider now the case in which $N$ is unknown. We will see that
adapting the analysis based on the known-$N$ model is really
straightforward for the more general problem. The main distinction is
that we don’t observe the all-zero encounter history so we have to
make sure we compute the probability for that encounter history which
we do by tacking a row of zeros onto the encounter history matrix. In
addition, we include the number of such all-zero encounter histories
as an unknown parameter of the model. Call that unknown quantity
$n_{0}$, and we have to 
be sure to include a combinatorial term to
account for the fact that of the $n$ observed individuals there are
${N \choose n}$
 ways to realize a sample of size $n$. The combinatorial term
involves the unknown $n_{0}$ and thus it must be included in the likelihood.

Operationally then, things proceed much as before: 
We compute the marginal probability of each observed ${\bf y}_{i}$,
i.e., by removing the latent ${\bf s}_{i}$ by integration. In
addition, we 
 compute the marginal probability of the ``all-zero'' encounter
history ${\bf y}_{n+1}$, and make sure to weight it $n_{0}$ times. We
accomplish this by ``padding'' the data set with a single encounter
history having $y_{n+1,j}=0$ for all traps $j=1,2,\ldots,J$. Then we
be sure to include the combinatorial term in the likelihood or
log-likelihood computation. We demonstrate this shortly.

To analyze a specific case, we’ll read in our fake data set (simulated
using the parameters given above). To set some things up in our
workspace we do this:
\begin{verbatim}
data<-simSCR0.fn(discard0=TRUE,sd=2013)
y<-data$Y
nind<-nrow(y)
X<-data$traplocs
J<-nrow(X)
K<-data$K
\end{verbatim}
Recall that these data were generated with $N=100$, on an $8 \times 8$ unit
state-space representing the trap locations (${\bf X}$) buffered by 2 units.

As before, the likelihood is defined in the {\bf R} workspace as an
{\bf R}
function, \mbox{\tt intlik2} (contained in the package \mbox{\tt
  scrbook}),
 which takes an argument being the unknown parameters of the
model and additional arguments as prescribed. In particular, 
 we provide the encounter history matrix ${\bf y}$, the trap locations
\mbox{\tt traplocs}, the spacing of the integration grid (argument
\mbox{\tt delta}) and the
state-space buffer. Here is the new likelihood function:
{\small
\begin{verbatim}
intlik2<-function(parm,y=y,delta=.3,X=traplocs,ssbuffer=2){

Xl<-min(X[,1]) -ssbuffer
Xu<-max(X[,1])+ ssbuffer
Yu<-max(X[,2])+ ssbuffer
Yl<-min(X[,2])- ssbuffer

xg<-seq(Xl+delta/2,Xu-delta/2,delta) 
yg<-seq(Yl+delta/2,Yu-delta/2,delta) 
npix.x<-length(xg)
npix.y<-length(yg)
area<- (Xu-Xl)*(Yu-Yl)/((npix.x)*(npix.y))
G<-cbind(rep(xg,npix.y),sort(rep(yg,npix.x)))
nG<-nrow(G)
D<- e2dist(X,G) 

alpha0<-parm[1]
alpha1<-parm[2]
n0<-exp(parm[3])
probcap<- plogis(alpha0)*exp(-alpha1*D*D)
Pm<-matrix(NA,nrow=nrow(probcap),ncol=ncol(probcap))
ymat<-rbind(y,rep(0,ncol(y)))

lik.marg<-rep(NA,nrow(ymat))
for(i in 1:nrow(ymat)){
Pm[1:length(Pm)]<- (dbinom(rep(ymat[i,],nG),K,probcap[1:length(Pm)],log=TRUE))
lik.cond<- exp(colSums(Pm))
lik.marg[i]<- sum( lik.cond*(1/nG) )  
}                                                 
nv<-c(rep(1,length(lik.marg)-1),n0)
part1<- lgamma(nrow(y)+n0+1) - lgamma(n0+1)
part2<- sum(nv*log(lik.marg))
 -1*(part1+ part2)
}
\end{verbatim}
}
To execute this function for the data that we created with \mbox{\tt simSCR0.fn},
 we execute the following command (saving the result in our
friend \mbox{\tt frog}).
This results in the usual output, including the parameter estimates,
the gradient, and the numerical Hessian which is useful for obtaining
asymptotic standard errors (see below):
\begin{verbatim}
starts<-c(-2.5,2,log(4))
frog<-nlm(intlik2,starts,hessian=TRUE,y=y,X=X,delta=.2,ssbuffer=2)

There were 50 or more warnings (use warnings() to see the first 50)

frog
$minimum
[1] 113.5004

$estimate
[1] -2.538334  2.466515  4.232810

[... Additional output deleted ...]
\end{verbatim}
While this produces some {\bf R} warnings, these happen to be harmless
in this case, and we will see from the \mbox{\tt nlm} output that the
algorithm performed satisfactory in minimizing the objective function.
The estimate of population size for the state-space (using the default 
state-space buffer) is
\begin{verbatim}
nrow(y)+exp(4.2328)
[1] 110.9099
\end{verbatim}
Which differs from the data-generating value ($N=100$) as we might
expect for a single realization. We usually will present an estimate of uncertainty associated
with this MLE which we can obtain by inverting the Hessian. Note that
$Var(\hat{N}) = n + \mbox{Var}(\hat{n}_{0})$.
Since we
have parameterized the model in terms of $log(n_{0})$ we use a delta
approximation to obtain the variance on the scale of $n_{0}$ as
follows:
\begin{verbatim}
(exp(4.2328)^2)*solve(frog$hessian)[3,3]
[1] 260.2033
> sqrt(260)
[1] 16.12452
\end{verbatim}
Therefore, the asymptotic ``Wald-type'' confidence interval for $N$ is
$110.91 \pm 1.96 \times 16.125 = (79.305, 142.515)$. To report this in
terms of density, we scale appropriately by the area of the prescribed
state-space which is $64$ units of area (i.e., an $8 \times 8$ square).


\begin{comment}

\subsection{Exercises}

{\flushleft 
{\bf 1.}	
Run the analysis with different state-space buffers and comment on the result. 
}


{\flushleft 
{\bf 2.} Conduct a brief simulation study using this code by
  simulating 100 data sets and obtain the MLEs for each data set. Do
  things seem to be working as you expect?  }

{\flushleft 
{\bf 3.} 
Further extensions: It should be straightforward to
  generalize the integrated likelihood function to accommodate many
  different situations. For examples, if we want to include more
  covariates in the model we can just add stuff to the object \mbox{\tt probcap},
 and add the relevant parameters to the argument that gets
  passed to the main  function.  For the simulated data, make up a
  covariate by generating a Bernoulli covariate (``trap type'' – perhaps
  baited or not baited) randomly and try to modify the likelihood to
  accommodate that.  }

{\flushleft {\bf 4.}  We would probably be interested in devising the
  integrated likelihood for the full 3-d encounter history array so
  that we could include temporally varying covariates. This is not
  difficult but naturally will slow down the execution
  substantially. The interested reader should try to expand the
  capabilities of this basic {\bf R} function.  }
\end{comment}




\subsection{Integrated Likelihood using the model under data augmentation } 

Note that this likelihood analysis is based on the standard likelihood
in which $N$ (or $n_{0}$) is an explicit parameter. This is usually called
the ``joint likelihood'' or ``unconditional likelihood''.  We could also
express the joint likelihood using data augmentation, replacing the
parameter $N$ with $\psi$ \citep[e.g., see Sec. 7.1.6][for an example]{royle_dorazio:2008}.
We don't go into detail here, but we note that the
likelihood under data augmentation is a zero-inflated binomial
mixture – precisely an occupancy type model \citep{royle:2006}.
Thus, while it is possible to carryout likelihood analysis of
models under data augmentation, we primarily advocate data
augmentation for Bayesian analysis.


\subsection{ Extensions}

We have only considered basic SCR models with no additional
covariates. However, in practice, we are interested in other types of
covariate effects including ``behavioral response'', 
sex-specificity of parameters, and potentially other effects. Some of
these  can be added directly to the likelihood – if the covariate is fixed
and known for all individuals captured or not. An example is a
behavioral response, which amounts to having a covariate $x_{ik}=1$ if
individual $i$ was captured prior to occasion $k$ and $x_{ik}=0$
otherwise. For uncaptured individuals, $x_{ik}=0$ for all $k$.
 \citet{royle_etal:2011jwm} called this a global behavioral
response because the covariate is defined for all traps, no matter the
trap in which an individual was captured. We could also define a {\it
  local} behavioral response which occurs at the level of the trap,
i.e., $x_{ijk}=1$ if individual $i$ was captured in trap $j$ prior to
occasion $k$, etc.. 
Trap-specific covariates such as trap type or status, or
time-specific covariates such as date, are easily accommodated as
well. As an example, \citet{kery_etal:2010} develop a model for the
European wildcat in which traps are either baited or not (a
trap-specific covariate with only 2 values), and also encounter
probability varies over time in the form of a quadratic seasonal response.
We consider models with behavioral response or fixed covariates in
Chapt. \ref{chapt.covariates}.
Although the integrated likelihood routines we provided above can be
modified directly for such cases, which we leave to the interested
reader to investigate. 

Sex-specificity is more difficult to deal with since sex is not known
for uncaptured individuals (and sometimes not even for all captured
individuals).  To analyze such models, we do Bayesian analysis of the
joint likelihood facilitated by the use of data augmentation
\citep{gardner_etal:2010jwm,russell_etal:2012}. For covariates that are
not fixed and known for all individuals, it is somewhat more
challenging to do MLE for these based on the joint likelihood as we
have developed above. Instead it is more conventional to use what is
colloquially referred to as the ``Huggins-Alho'' type model which is
one of the approaches taken in the software package \mbox{\tt secr}
\citep[][see Sec. \ref{mle.sec.secr}]{efford:2011}. This idea is
motivated by thinking about unequal probability sampling methods known
as Horvitz-Thompson sampling \citep[e.g.,
see][]{overton_stehman:1995}.  We don't use that method anywhere in
this book because it represents a paradigm shift in the inference
framework which is done historically only for convenience (i.e., ease
of constructing an estimator) and not for philosophical or theoretical
reasons.






\section{Classical model selection and assessment}

In most analyses, one is interested in choosing from among various
potential models, or ranking models, or something else to do with
assessing the relative merits of a set of models. A good thing about
classical analysis based on likelihood is we can apply AIC methods
\citep{burnham_anderson:2002} without difficulty. There are two
distinct contexts for model-selection that we think are relevant to
SCR models. First is, and AIC selecting among models that represent
distinct biological hypotheses (e.g., covariates affecting encounter
probability or density). AIC is convenient for assessing the relative
merits of these different models although if there are only a few
models it is not objectionable to use hypothesis tests or confidence
intervals to determine importance of effects. The second model
selection context has to do with choosing among various detection
functions although, as a general rule, we don't recommend this
application of model selection.  This is because there is hardly ever
(if at all) a rational subject-matter based reason motivating specific
distance functions. As a result, we believe that doing too much model
selection will invariably lead to over-fitting and thus over-statement
of precision. This is the main reason that we haven't loaded you down
with a basket of models for detection probability so far, although we
discuss many possibilities in Chapt. \ref{chapt.gof} where we also
discuss focus more attention on methods and applications of model selection.


{\bf Goodness-of-fit} -- For many standard capture-recapture models,
it is possible to identify goodness-of-fit statistics based on the
multinomial likelihood and evaluate model adequacy using formal
statistical tests. Similar strategies can be applied to SCR models
using expected cell-frequencies based on the marginal distribution of
the observations. Also, because computing MLEs is somewhat more
efficient in many cases compared to Bayesian analysis, it is also
sometimes easy to use bootstrap methods\footnote{I'm not sure if there
  are references in the context of SCR models for this stuff....???}.

Bayesian goodness-of-fit, which we take up in more detail in
Chapt. \ref{chapt.gof}, is almost always addressed with Bayesian
p-values or some other posterior predictive check
(sec. \ref{glms.sec.gof}, \citet[][sec. 2.6]{kery:2010}).
\citet{royle_etal:2011mee} suggested checking model fit for SCR models
by decomposing fit into two components: (1) That of the encounter
process model, evaluated by the expected encounter frequencies
computed {\it conditional} on ${\bf s}$; and, (2) That of the spatial
point process model (``spatial randomness'').


\section{Likelihood analysis of the wolverine camera trapping data}
\label{mle.sec.wolverine}


Here we compute the MLEs for the wolverine data using an expanded
version of the function we developed in the previous section. To
accommodate that each trap might be operational a variable number of
nights, we provided an additional argument to the likelihood function
(allowing for a vector $K$), which requires also a modification to the
construction of the likelihood.  In addition,
we accommodate  the state-space is a general rectangle, and
we included a line in the code to compute the state-space area which
we apply below for computing density.  The more general function
(\mbox{\tt intlik3}) is given in the {\bf R} package \mbox{\tt scrbook}. It has a general
purpose wrapper named \mbox{\tt scr}\footnote{Not written yet} which has other capabilities too. 
To use this function to obtain the MLEs for the wolverine camera trap
study, we execute the following commands (note: these are in the help
file and will execute if you type \mbox{\tt example(intlik3)}:
{\small
\begin{verbatim}
library("scrbook")
data("wolverine")
 
traps<-wolverine$wtraps
traplocs<-traps[,1:2]/10000
K.wolv<-apply(traps[,3:ncol(traps)],1,sum)
traps<-cbind(1:nrow(traps),traps)  # pad an ID variable
y3d<-SCR23darray.fn(wolverine$wcaps,traps)
y2d<-apply(y3d,c(1,3),sum)

starts<-c(-1.5,1.2,log(4))
frog<-nlm(intlik3,starts,hessian=TRUE,y=y2d,K=K.wolv,X=traplocs,delta=.2,ssbuffer=2)
There were 23 warnings (use warnings() to see them)

frog
$minimum
[1] 220.4313

$estimate
[1] -2.817610  1.254757  3.583690

$gradient
[1]  1.210460e-06 -5.255072e-06 -5.710212e-07

$hessian
           [,1]       [,2]      [,3]
[1,]  37.686164 -11.849561  4.686501
[2,] -11.849561  30.842624 -9.193201
[3,]   4.686501  -9.193201 12.973354

$code
[1] 1

$iterations
[1] 12
\end{verbatim}
}
Of course we're interested in obtaining an estimate of population size
for the prescribed state-space, or density, and associated measures of
uncertainty which we do using the delta method approximation
\citep[][Appendix F4]{williams_etal:2002}
\footnote{
We found a good set of notes on the delta approximation on Dr. David
Patterson's ST549 notes: 
\url{http://www.math.umt.edu/patterson/549/Delta.pdf}
}).
To do all of that we need to manipulate the output of
\mbox{\tt nlm} since we have our  estimate in terms of $\mbox{\tt
  log(n0)}$. We execute the following commands:
{\small 
\begin{verbatim}
area<-attr(intlik3(starts,y=y2d,K=K.wolv,X=traplocs,delta=.2,ssbuffer=2),"SSarea")
Nhat<-nrow(y2d)+exp(frog$estimate[3])
area<-attr(intlik3(starts,y=y2d,K=K.wolv,X=traplocs,delta=.2,ssbuffer=2),"SSarea")
Dhat<- Nhat/area

Dhat
[1] 0.5494956

SE<- (1/area)*exp(frog$estimate[3])*sqrt(solve(frog$hessian)[3,3])

SE
[1] 0.1087101
\end{verbatim}
} 
So our estimate of density is $0.55$ individuals per ``standardized
unit'' which is 100 $km^2$, because we divided UTM coordinates by
10000.  So this is about 5.5 individuals per 1000 $km^2$ (the units
reported by \citep{royle_etal:2011jwm}), with a SE of around 1.09
individuals.  This compares closely with $5.77$
reported\footnote{check this!!!} in
sec. \ref{scr0.sec.wolverine} based on Bayesian
analysis of the model.


To evaluate the effect of the integration grid density, 
we obtained the MLEs for a state-space buffer of 2 (standardized
units) and for integration grid with spacing $\delta = .3, .2, .1,
.05$. The MLEs for these 4 cases including the relative runtime are
given in Table \ref{mle.tab.integration}.
We see the results change only slightly as the fineness of the
integration grid increases. Conversely, the runtime on the platform of
the day for the 4 cases increases rapidly. 
As we have suggested previously these runtimes could be regarded in
relative terms,  across platforms, for gaging the decrease in
speed as the fineness of the integration grid increases. The effect of
this is that we anticipate some numerical error in approximating the
integral on a mesh of points, and that error increases as the
coarseness of the mesh increases. 


\begin{table}[ht]
\centering
\caption{Run time and MLEs for different integration grid resolutions.}
\begin{tabular}{l|rccc}
\hline \hline
$\delta$ &   & \multicolumn{3}{c}{Estimates} \\ \hline
         &  runtime        & $\alpha_0$ & $\alpha1$ & $log(n_0)$ \\ \hline
 0.30   &  9.9  &  -2.819786 & 1.258468 & 3.569731  \\
 0.20   & 32.3  &  -2.817610 & 1.254757 & 3.583690 \\
 0.10  & 115.1  &  -2.817570 & 1.255112 & 3.599040 \\
 0.05 &  407.3 &   -2.817559&  1.255281&  3.607158 \\
\end{tabular}
\label{mle.tab.integration}
\end{table}


We studied the effect of the state-space buffer on the MLEs,
using a fixed $\delta = .2$ for all analyses. We used state-space buffers
of 1 to 4 units stepped by .5. This produced the following results,
given here are the state-space buffer, area of the state-space, the
MLE of $N$ for the prescribed state-space and the corresponding MLE of
density:
{\small
\begin{verbatim}
     ssbuff       Ass      Nhat      Dhat
[1,]    1.0  66.98212  37.73338 0.5633352
[2,]    1.5  84.36242  46.21008 0.5477567
[3,]    2.0 103.74272  57.00617 0.5494956
[4,]    2.5 125.12302  69.03616 0.5517463
[5,]    3.0 148.50332  82.17550 0.5533580
[6,]    3.5 173.88362  96.44018 0.5546249
[7,]    4.0 201.26392 111.83524 0.5556646
\end{verbatim}
}
The estimates of $D$ stabilize rapidly and the incremental difference
is within the numerical error associated with approximating the
integral.  


\subsection{Restricted state-space}
\label{mle.sec.shapefile}

In sec. \ref{scr0.sec.discrete} 
 we used a discrete representation of
the state-space in order to have control over its extent and shape,
for example so that we could clip out ``non-habitat''. Clearly that
formulation of the model is relevant to the use of integrated
likelihood in the sense that such a representation of the state-space
underlies the computation of the integral. Thus, for example, we could
easily compute the MLE of parameters under some model with a
restricted state-space merely by creating the required state-space at
whatever grid resolution is desired, and then feed that state-space
into the likelihood evaluation above. We can easily create an explicit
state-space grid for integration from arbitrary polygons or GIS
shapefiles \index{shapefile} which we 
demonstrate here. Our approach here is to create the integration grid
(or state-space grid) outside of the likelihood evaluation, and then
determine which points of the grid lie in the polygon defined by the
shapefile using 
functions in the {\bf R} packages \mbox{\tt sp} \index{R
  package!sp} and
\mbox{\tt maptools} \index{R package!maptools} \index{maptools}.  Here
are the {\bf R} commands for doing this:  
{\small
\begin{verbatim}
library(maptools}
library(sp)
SSp<-readShapeSpatial('Sim_Polygon.shp')
Pcoord<-SpatialPoints(G)
PinPoly<-over(Pcoord,SSp)
Pin<-as.numeric(!is.na(PinPoly[,1]))
G<-G[Pin==1,]
\end{verbatim}
}
We created  the function \mbox{\tt intlik4} which accepts the integration
grid as an explicit argument, and this function is also available in
the package  \mbox{\tt scrbook}.

We apply this modification to the wolverine camera trapping
study. \citet{royle_etal:2011jwm} created 2, 4 and 8 km state-space
grids so as to remove ``non-habitat'' (mostly ocean, bayes, and large
lakes). We previously analyzed the model using {\bf JAGS} and {\bf WinBUGS} in
Chapt. \ref{chapt.scr0}.  To set up the wolverine data and fit the
model we execute the following commands
{\small 
\begin{verbatim}
library("scrbook")
data("wolverine")

traps<-wolverine$wtraps
traplocs<-traps[,1:2]/10000
K.wolv<-apply(traps[,3:ncol(traps)],1,sum)
traps<-cbind(1:nrow(traps),traps)  # pad with an ID variable
y3d<-SCR23darray.fn(wolverine$wcaps,traps)
y2d<-apply(y3d,c(1,3),sum)
G<-wolverine$grid2/10000

starts<-c(-1.5,1.2,log(4))
frog<-nlm(intlik4,starts,hessian=TRUE,y=y2d,K=K.wolv,X=traplocs,G=G)

frog
$minimum
[1] 225.8355

$estimate
[1] -2.995541  1.265021  4.110476

$gradient
[1]  3.808485e-05 -9.930579e-06  3.906668e-06

$hessian
           [,1]       [,2]      [,3]
[1,]  47.059393 -21.415124  4.406148
[2,] -21.415124  38.255192 -7.386245
[3,]   4.406148  -7.386245 15.406613

$code
[1] 1

$iterations
[1] 14
\end{verbatim}
}

Next we convert the parameter estimates to estimates of total
population size for the prescribed state-space, and then obtain an
estimate of density (per 1000
$km^2$) using the area computed as the number of pixels in the
state-space grid \mbox{\tt G} multiplied by the area per grid cell. In
the present case (the calculation above) we used a state-space grid
with $2 \times 2$ $km$ pixels.  Finally, we compute
a standard errors using the delta approximation: 
\begin{verbatim}
Nhat<- 21+exp(frog$estimate[3])
SE<-  exp(frog$estimate[3])*sqrt(solve(frog$hessian)[3,3])
D<- (Nhat/(nrow(G)*area))*1000
SE.D<- (SE/(nrow(G)*area))*1000
\end{verbatim}
We did this for each the 2 $km$, 4 $km$ and 8 $km$ state-space grids
which produced the estimates summarized in Tab. \ref{mle.tab.wolv}.
These estimates compare with the 8.6 (2 km grid) and 8.2 (8 km grid)
reported in 
\citet{royle_etal:2011jwm} based on a clipped state-space as described
in sec. \ref{scr0.sec.discrete}.

\begin{table}
\centering
\caption{MLEs for the wolverine camera trapping data using 2, 4 and 8 km state-space grids.}
\begin{tabular}{cccccccc}
\hline \hline
grid &  $\alpha_0$  &  $\alpha_1$ &   $log(n_0)$  & $N$   &  SE & D(1000) &  SE \\ \hline
2  &  -2.995541& 1.265021 &4.110476 &81.97574& 16.30904 &8.310598 &1.653391\\
4  &  -2.991268&1.344055  &4.157026 &84.88126& 16.76202 &8.570401& 1.692450\\
8   & -3.051705& 1.080083 &4.058542 &78.88983& 15.31392 &7.851296& 1.524077\\
\end{tabular}
\label{mle.tab.wolv}
\end{table}


\begin{comment}
\subsection{
Exercises
}

{\flushleft
1.	Compute the 95\% confidence interval for wolverine density,
somehow. Comment on the practical implication of this level of precision.
}

{\flushleft
2.	Compute the AIC of this model and modify \mbox{\tt intlik3}
 to consider alternative link functions (at least one additional) and
 compare the  AIC of the different models and the estimates. Comment. 
}
\end{comment}


\section{Program DENSITY and the R package \mbox{\tt secr} }
\label{mle.sec.secr}


{\bf DENSITY} is a software program developed by \citet{efford:2004}
for fitting spatial capture-recapture models based mostly on classical
maximum likelihood estimation and related inference methods.
\citet{efford:2011} has also released an {\bf R} package named
\mbox{\tt secr}, that contains much of the functionality of {\bf
  DENSITY} but also incorporates new models and features.  Here, we
will focus on \mbox{\tt secr} as it will continue to be developed,
contains more functionality and is based in {\bf R}.


 To install
and run models in \mbox{\tt secr}, you must download the package and
load it in
{\bf R}.
\begin{verbatim}
 install.packages(“secr”)
 library(secr)
\end{verbatim}
\mbox{\tt secr} allows the user to simulate data and fit a suite of models with
various detection functions and covariate responses.  \mbox{\tt secr}
uses the
standard {\bf R} model specification framework using tildes. E.g., the model
command is \mbox{\tt secr.fit} and is generally written as
\begin{verbatim}
> secr.fit(capturedata, model = list(D~1, g0~1, sigma~1), buffer = 20000)
\end{verbatim}
where we have \verb#g0~1# indicating the intercept model. 
 Possible predictors for detection probability include both
pre-defined variables (e.g., \mbox{\tt t} and \mbox{\tt b}
corresponding to ``time'' and 
``behavior''), and user-defined covariates of several kinds. 
For example, to include a behavioral response, this would be written
as \verb#g0~b#.
The discussion of covariates is developed more in Chapt. \ref{chapt.covariates}\footnote{Beth:
  does secr fit a local trap-specific response or just a global
  behavioral response?}

Before we can fit the models, the data must first be packaged properly
for 
\mbox{\tt secr}.  Two input files are required: trap layout (location and
identification information for each trap) and capture data (e.g.,
sampling session, animal identification, trap day, and trap location).
\mbox{\tt secr} requires that you specify the trap type, the two most common for
camera trapping/hair snares are ‘proximity’ detectors and ‘count’
detectors.  The `proximity' detector type allows, at most, one
detection of each individual at a particular detector on any occasion
(i.e., it is equivalent to the Bernoulli or binomial encounter process
model).
The ‘count’ detector designation allows repeat encounters of each
individual at a particular detector on any occasion.  There are other
detector types that one can select such as: `polygon' detector type
which allows for a trap to be a sampled polygon, e.g., scat surveys,
and 'signal' detector which allows for traps that have a strength
indicator, e.g., acoustic arrays.  The detector types ‘single’ and
‘multi’ can be confusing as ‘multi’ seems like it would appropriate
for something like a camera trap, but instead these two designations
refer to traps that retain individuals, thus precluding the ability
for animals to be captured in other traps during the sampling
occasion.  The ‘single’ type indicates trap that can only catch one
animal at a time, while ‘multi’ indicates traps that may catch more
than one animal at a time.  For a full review of the detector types,
one should look at the help manual, which can be accessed in {\bf R} after
installing the \mbox{\tt secr} package by using the command:
\begin{verbatim}
 RShowDoc("secr-manual", package = "secr")
\end{verbatim}
As with all of the SCR models, \mbox{\tt secr} fits a detection function relating
the probability of detection to the distance of a detector from an
individual activity center. \mbox{\tt secr} allows the user to specify one of a
variety of detection functions including the commonly used
half-normal, hazard rate, and exponential.  There are 12 different
functions, but some are only available for simulating data, and one
should take caution when using different detection functions as the
interpretation of the parameters, such as $\sigma$, may not be consistent
across formulations.  The different detection functions are defined in
the \mbox{\tt secr} manual and can be found by calling the help function for the
detection function:
\begin{verbatim}
 ?detectfn
\end{verbatim}
It is useful to note that \mbox{\tt secr} requires the buffer distance to be
defined in meters and density will be returned as number of animals
per hectare.  Thus to make comparisons between \mbox{\tt secr} and other models,
we will often have to convert the density to the same units.  Also,
note that $\sigma$ is returned in units of meters.

\footnote{One question: SECR only ever reports “sigma”. What exactly is sigma?  It is a scale parameter of a detection function and all detection functions have a scale parameter. But in what sense is this sigma parameter related to “home range diameter”?  Efford doesn’t explain this, does he?  In some sections in chapter 4 or possibly 6 we get into this issue. 
}

\subsection{ Analysis using the \mbox{\tt secr} package}

To demonstrate the use of the \mbox{\tt secr} package, we will show how to do the
same analysis on the wolverine study as shown in section 4.6.  To use
the \mbox{\tt secr} package, the data need to be formatted in a similar but
slightly different manner than we use in {\bf
  WinBUGS}\footnote{Elaborate on this point -- and how is this
  different than introduced in chapter 4?}.  After installing
the \mbox{\tt secr} package, we first have to read in the trap locations and
other related information, such as if the trap is operational during a
sampling occasion.  The \mbox{\tt secr} package reads in the trap data through a
command called ``\mbox{\tt read.traps}'', which requires the detector type as
input.  The detector type is important because it will determine the
likelihood that \mbox{\tt secr} will use to fit the model.  Here, we have
selected ``proximity'' which corresponds to the Bernoulli encounter
model in which individuals are captured at most once in
each trap during each sampling occasion:
{\small
\begin{verbatim}
library("secr")
library("scrbook")
data("wolverine")
traps<-wolverine$wtraps

traps<-as.matrix(traps[,1:3])
dimnames(traps)<-list(NULL,c("trapID","x","y"))
traps<-as.data.frame(traps)
trapfile<-read.traps(data=traps,detector="proximity")
\end{verbatim}
}
Here we note that trap coordinates are extracted from the wolverine
data but we do {\it not} standardize them here. This is because
\mbox{\tt secr} defaults to coordinate scaling of meters which is the
standard scaling for the wolverine data. 

After reading in the data, we now need to create the encounter matrix
or array.  The \mbox{\tt secr} package does this through the use of the
\mbox{\tt make.capthist} command, where we provide the capture histories in raw
data format (each line contains the session, identification number,
occasion, and trap id for only 1 individual).  This is the format that
was shown in the data input file ``\mbox{\tt wcaps}'', and we only need a line or
two to organize the data into the order that the make.capthist command
wants.  In creating the capture history, we provide also the trapfile
with the trap information, and the format (e.g., here \mbox{\tt fmt= ``trapID''})
so that \mbox{\tt secr} knows how to match the encounters to the trap, and
finally, we provide the number of occasions:
{\small 
\begin{verbatim}
wolv.dat<-wolverine$wcaps[,c(2,3,1)]  # different order than SCR!!!
wolv.dat<- cbind(rep(1,nrow(wolv.dat)),wolv.dat)
dimnames(wolv.dat)<-list(NULL,c("Session","ID","Occasion","trapID"))
wolv.dat<-as.data.frame(wolv.dat)
wolvcapt<-make.capthist(wolv.dat,trapfile,fmt="trapID",noccasions=165)
\end{verbatim}
}
The function  \mbox{\tt secr.fit} will fit the model. We are using the
basic model (SCR0), so we do not need to make any specifications in
the command line except for the providing the buffer size (in $m$).  To
specify different models, you can change the default
\verb#D~1, g0~1, sigma~1#, which the interested reader can do with
very little difficulty.

{\small
\begin{verbatim}
 wolv.secr<-secr.fit(wolvcapt,model=list(D~1, g0~1, sigma~1), buffer=20000)

 wolv.secr

secr.fit( capthist = wolvcapt, model = list(D ~ 1, g0 ~ 1, sigma ~ 1), buffer = 20000 )
secr 2.3.1, 14:20:38 05 Mar 2012

Detector type     proximity 
Detector number   37 
Average spacing   4415.693 m 
x-range           593498 652294 m 
y-range           6296796 6361803 m 
N animals       :  21  
N detections    :  115 
N occasions     :  165 
Mask area       :  1037069 ha 

Model           :  D~1 g0~1 sigma~1 
Fixed (real)    :  none 
Detection fn    :  halfnormal 
Distribution    :  poisson 
N parameters    :  3 
Log likelihood  :  -746.754 
AIC             :  1499.508 
AICc            :  1500.92 

Beta parameters (coefficients) 
           beta    SE.beta        lcl       ucl
D     -9.749576 0.23027860 -10.200913 -9.298238
g0    -4.275735 0.15846099  -4.586313 -3.965158
sigma  8.699202 0.07868942   8.544974  8.853430

Variance-covariance matrix of beta parameters 
                  D            g0        sigma
D      0.0530282320  0.0005468918 -0.005226919
g0     0.0005468918  0.0251098856 -0.005885208
sigma -0.0052269186 -0.0058852077  0.006192025

Fitted (real) parameters evaluated at base levels of covariates 
       link     estimate  SE.estimate          lcl          ucl
D       log 5.831941e-05 1.360973e-05 3.713638e-05 9.158548e-05
g0    logit 1.371121e-02 2.142902e-03 1.008756e-02 1.861207e-02
sigma   log 5.998123e+03 4.727205e+02 5.140849e+03 6.998355e+03
\end{verbatim}
}

Under the fitted (real) parameters, we find $D$, the density, given in
units of individuals/hectare (1 hectare = 10000 $m^2$).  To convert this
into individuals/1000 $km^2$, we multiply by 100000, thus our density
estimate is 5.83 individuals/1000 $km^2$.  $\sigma$ is given in units of
meters, to convert to kilometers, we divide by 1000, which puts sigma
at 5.99 $km$.  Both of these estimates are very similar to those
provided in sec. \ref{scr0.sec.wolverine} for the buffer size equal to
20 $km$\footnote{How come the MLES are different from what I computed
  above?  What is sigma from back there? That section is missing
  I think}.


\begin{comment}
As an
exercise, run this analysis for 30 and 40 km buffers and compare those
found in section 4.6 under {\bf WinBUGS}.  
NOTE: The function \mbox{\tt
  secr.fit} 
will return a
warning when the buffer size appears to be too small.  This is useful
particularly with the different units being used between programs and
packages.
\end{comment}

\subsection{Analysis of Efford's Possum Data}



Do a secr analysis of the possum data set to follow up on chapter 5 material.......

Maybe just insert the analysis from the ?secr page?

Use the mask he provides (show picture)
Use rectangular mask.

Cite above material on state-space grid. Use our likelihood function
with his state-space grid.

Secr + grid
secr + no grid
my likelihood + grid

Note: should not compare AIC across analysis platforms because the
likelihoods can be scaled arbitrarily -- depending on what to leave in
or leave out.



\section{Summary and Outlook}

In this chapter, we showed that classical analysis of SCR models based
on likelihood methods is a relatively simple proposition.  Analysis is
based on the so-called integrated likelihood in which the individual
activity centers (random effects) are removed from the
conditional-on-{\bf s} likelihood by integration. We showed how to construct
the integrated likelihood and fit some simple models in the {\bf R}
programming language.  In addition, likelihood analysis for some broad
classes of SCR models can be accomplished in the software package
{\bf DENSITY} 
or the {\bf R}
library \mbox{\tt secr} which we provided an illustration of here. In later
chapters we provide more detailed analyses of SCR data likelihood
methods and the
\mbox{\tt secr}
package.

To compute the marginal (integrated) likelihood we have to precisely describe the
state-space of the underlying point process. In practice, this leads
to a ``buffer'' around the trap array. We note that this is not really a
``buffer strip'' in the sense of \citet{wilson_anderson:1985a},  
but it is somewhat more general here. In particular,
it establishes the support of the integrand and, 
in SCR models, it is an element of the model that
provides an explicit
linkage between population size $N$ and density $D$.
As a practical 
matter, it will typically be the case that, while estimates of $N$
increase with the area of the state-space (as they should!), estimates of density
stabilize. This is not a feature of the classical methods based on
using model $M_0$ or model $M_h$ and buffering the trap array.

Why or why not use likelihood inference exclusively? For certain
specific models, it is probably more computationally efficient to
produce MLEs (e.g., see Chapt. \ref{chapt.ecoldist}). However, {\bf BUGS} is extremely flexible in terms of
describing models, although it sometimes can be quite slow. We can
devise models in the {\bf BUGS} language easily that we cannot fit in
\mbox{\tt secr}. E.g.,
random individual effects of various types
(Chapt. \ref{chapt.covariates}), we can 
handle missing covariates in complete generality and seamlessly, and
impose arbitrary distributions on random variables. Moreover, models
can easily be adapted to include auxiliary data types. For example, we
might have camera trapping and genetic data and we can describe the
models directly in {\bf BUGS} and fit a joint model. For the MLE we have
to write a custom new piece of code for each model or hope someone has
done it for us.  Later we consider open population models which are
straightforward to develop in {\bf BUGS} but, so far, there is no
available platform for doing MLE although we imagine one could develop
this.  On
the other hand, likelihood analysis makes it easy to do
model-selection by AIC and in some cases compute standard errors or
carry-out goodness-of-fit evaluations. 
\begin{comment}
Another thing that is more conceptual here is non-CSR point
processes (Chapt. \ref{chapt.state-space}) and generating predictions of how many
individuals have home range centers in any particular polygon.  Basic
benefits of Bayesian analysis have been discussed elsewhere (XXXXXXXX Chapter
2? BPA book? Link and Barker?) and we believe these are compelling.
\end{comment}

In summary, basic SCR models are easy to implement by either
likelihood or Bayesian methods but some users will 
realize much more flexibility in model development using existing
platforms for Bayesian analysis. While these tend to be slow
(sometimes excruciatingly slow), this will probably not be an
impediment in most problems, especially at some near point in the
future.  Since we spent a lot of time here talking about specific
technical details on how to implement likelihood analysis of SCR
models, we provided a corresponding treatment in the next chapter on
how to devise MCMC algorithms for SCR models. This is a bit more
tedious and requires more coding, but is not technically challenging
(accept perhaps to develop highly efficient algorithms which we don’t
excel at).





\chapter{
MCMC for Spatial Capture-Recapture
}
\markboth{MCMC}{}
\label{chapt.mcmc}

%%% NOTES
%%% Andy's working through this doing format edits mostly and math
%%% stuff , but not in order
%%% anytime you see a XXX or XYZ that is a marker to change some
%%% hard-wired reference to a float

\vspace{.3in}

\section{Introduction}
In this chapter we will dive a little deeper into Markov chain Monte
Carlo (MCMC) sampling. We will construct custom MCMC samplers in {\bf R},
starting with easy-to-code GLMs and GLMMs and moving on to simple CR and SCR
models. Finally, we will illustrate some alternative
ready-to-use software packages for MCMC sampling. We will NOT provide
exhaustive background information on the theory and justification of
MCMC sampling – there are entire books dedicated to that subject and
we refer you to \citet{robert_casella:2004} and
\citet{robert_casella:2010}. Rather we aim to provide you with enough
background and technical know-how to start building your own MCMC
samplers for SCR models in {\bf R}. You will find that quite a few topics that come up 
in this chapter have already been covered in previous chapters, particularly the introduction
into Byesian analysis in Chapt. \ref{chapt.glms}. To keep you from having to leaf back and forth
we will in some places briefly review aspects of Bayesian analysis, but we try to focus on the more 
technical issues of building MCMC samplers relevant to SCR models. 



\subsection{Why build your own MCMC algorithm?}

The standard programs we have used so far to run MCMC analyses are
{\bf WinBUGS} \citep{gilks_etal:1994} and {\bf JAGS}
\citep{plummer:2003}. The wonderful thing about these {\bf BUGS}
engines
is that they automatically use  appropriate and, most of the time,
efficient forms
of MCMC sampling for the model specified by the user.

The fact that we have such a Swiss Army knife type of MCMC machine
begs the question: Why would anyone want to build their own MCMC
algorithm? For one, there are a limited number of distributions and
functions implemented in {\bf BUGS}. While {\bf OpenBUGS} provides more
options, some more complex models may be impossible to build within
these programs. A very simple example from spatial capture-recapture
that can give you a headache in {\bf WinBUGS} is when your state-space is an
irregular-shaped polygon, rather than an ideal rectangle that can be
characterized by four pairs of coordinates. It is easy to restrict
activity centers to any arbitrary polygon in {\bf R} using an ESRI shapefile
(and we will show you an example in a little bit), but you cannot use
a shape file in a {\bf BUGS} model.  Similarly, models of space usage
that take into account ecological distance
(Chapt. \ref{chapt.ecoldist} cannot be implemented in the {\bf BUGS}
engines.  Moreover, there are classes of 
SCR models that we have not been able to implement effectively using
likelihood methods, and are inefficient to run in the {\bf BUGS}
engines. An example are those models covered in Chapts. 
\ref{chapt.scr-unmarked} and \ref{chapt.partialID}. 

Sometimes implementing an MCMC algorithm in R may be faster than in
{\bf WinBUGS} - especially if you want to run simulation studies where you
have hundreds or more simulated data sets, several years' worth of
data or other large models, this can be a big advantage.

Finally, building your own MCMC algorithm is a great exercise to
understand how MCMC sampling works. So while using the {\bf BUGS} language requires you to understand the structure of your model, building an MCMC algorithm requires you to think about the relationship between your data, priors and posteriors, and how these can be efficiently analyzed and characterized. Not to mention that, if you are an R junkie, it can actually be fun.
However, if you don't think you will ever sit down and write your own
MCMC sampler, consider skipping this chapter - apart from coding it
will not cover anything SCR-related that is not covered by other, more
model-oriented chapters as well.


\section{MCMC and posterior distributions}

MCMC is a class of simulation methods for
drawing (correlated) random numbers from a target distribution, which
in Bayesian inference is the posterior distribution.
As a reminder, the posterior distribution is a probability
distribution for an unknown parameter, say $\theta$, given a set of
observed data and its prior probability distribution (the probability
distribution we assign to a parameter before we observe data).  The
great benefit of computing the posterior distribution of $\theta$ is
that it can be used to make probability statements about $\theta$,
such as the probability that $\theta$ is equal to some value, or the
probability that $\theta$ falls within some range of values. 
The posterior distribution summarizes all we know about a parameter
and thus, is the central object of interest in Bayesian
analysis. Unfortunately, in many if not most practical applications,
it is nearly impossible to directly compute the posterior. Recall
Bayes’ theorem:
\begin{equation}
p(\theta|y) = p(y|\theta) * p(\theta) / p(y),
\label{mcmc.eq.bayes}
\end{equation}
where $\theta$ is the parameter of interest, $y$ is the observed data,
$p(\theta|y)$ is the posterior, $p(y|\theta)$ the likelihood of the
data conditional on $\theta$, $p(\theta)$ the prior probability of
$\theta$, and, finally, $p(y)$ is the marginal probability of the
data, defined as 
\[
p(y) = \int p(y|\theta) * p(\theta) d\theta
\]

This marginal probability is a normalizing constant that ensures that
the posterior integrates to 1. Often, the
integral is  hard or impossible to evaluate, unless you are
dealing with a really simple model.  For example, consider 
a Normal model, with a set of $n$ observations, $y_{i};
i=1,2,\ldots,n$: 
\[
 y_{i} \sim \mbox{Normal}(\mu, \sigma),
\]
where $\sigma$ is known and our objective is to obtain an estimate of
$\mu$ using Bayesian statistics. To fully specify the model in a Bayesian
framework, we first have to define a prior distribution for $\mu$. Recall
from Chapt. \ref{chapt.glms} 
that for certain data models, certain priors lead to
conjugacy – i.e. if you choose a certain prior for your parameter,
your posterior distribution will be of a known parametric form. The
conjugate prior for the mean of a normal model is also a Normal
distribution:
\[
\mu \sim \mbox{Normal}(\mu_0, \sigma_{0}^{2})
\]
If $\mu_{0}$ and $\sigma_{0}^{2}$ are fixed, the posterior for $\mu$
has the following form (for the algebraic proof, see XXX RED BOOK? XXXX):
\begin{equation}
\mu|y \sim \mbox{Normal}(\mu_{n}, \sigma_{n}^{2})
\label{mcmc.eq.mu-posterior}
\end{equation}
where
\[
\mu_{n} = \frac{ \sigma^{2}}  {\sigma^{2}   +n* \sigma_{0}^{2}}*  \mu_0 +      \frac{n * \sigma_{0}^{2}}  {\sigma^{2}   +n* \sigma_{0}^{2}} *\bar{y}
\]
And
\[
 \sigma_{n}^{2} = \frac{\sigma^{2}  * \sigma_{0}^{2}} {\sigma^{2} + n*\sigma_{0}^{2}}
\]
We can directly obtain estimates of interest from this Normal
posterior distribution, such as the mean $\hat{\mu}$ and its variance; we
do not need to apply MCMC, since we can recognize the posterior as a
parametric distribution, including the normalizing constant $p(y)$.
But generally we will be interested in more complex models with
several, say $n$, parameters. In this case, computing $p(y)$ from
Eq. \ref{mcmc.eq.bayes} requires $n$-dimensional integration, which is
can be difficult or impossible. Thus, the posterior distribution in
generally only known up to a constant of proportionality:
\[
p(\theta|y) \propto p(y|\theta) * p(\theta)
\]
The power of MCMC is that it allows us to approximate the posterior
using simulation without evaluating the high dimensional integrals and
to directly sample from the posterior, even when the posterior
distribution is unknown! The price is that MCMC is computationally
expensive. Although MCMC first appeared in the scientific literature
in 1949 \citep{metropolis_etal:1949}, widespread use did not occur
until the 1980s when computational power and speed increased
\citep{gelfand_smith:1990}. It is safe to say that the advent of
practical MCMC methods is the primary reason why Bayesian inference
has become so popular during the past three decades.
In a nutshell, MCMC lets us generate sequential draws of $\theta$ (the
parameter(s) of interest) from distributions approximating the unknown
posterior over $T$ iterations. The distribution of the draw at $t$ depends
on the value drawn at $t$-1; hence, the draws from a Markov
chain\footnote{In case you are not familiar with Markov chains, for
  $T$ random samples $\theta^ {(1)}$, ... $\theta^{(T)}$ from a Markov chain
  the distribution of $\theta^{(t)}$ depends only on the immediately preceding
  value, $\theta^{(t-1)}$.}. As $T$ goes to infinity, the Markov chain
converges to the desired distribution – in our case the posterior
distribution for $\theta|y$. Thus, once the Markov chain has reached
its stationary distribution, the generated samples can be used to
characterize the posterior distribution, $p(\theta|y)$, and point
estimates of $\theta$, its standard error and confidence bounds, can
be obtained directly from this approximation of the posterior. 



\section{Types of MCMC sampling}

There are several MCMC algorithms, the most popular being Gibbs
sampling and Metropolis-Hastings sampling, both of which were briefly introduced in Chapt. \ref{chapt.glms}. We will be dealing with
these two classes in more detail and use them to construct the MCMC
algorithms for SCR models. Also, we will briefly review alternative
techniques that are applicable in some situations.


\subsection{Gibbs sampling}
\label{mcmc.sec.gibbs}

Gibbs sampling was named after the physicist J.W. Gibbs by
\citet{geman_geman:1984}, who applied the algorithm to a Gibbs
distribution \footnote{a distribution from physics we are not going to
  worry about, since it has no immediate connection with Gibbs
  sampling other than giving its name}. The roots of Gibbs sampling
can be traced back to work of \citet{metropolis_ulam:1953}, and it is
actually closely related to Metropolis sampling (see Chapter 11.5 in
\citet{gelman_etal:2004}, for the link between the two samplers). We
will focus on the technical aspects of this algorithm, but if you find
yourself hungry for more background, \citet{casella_george:1992}
provide a more in-depth introduction to the Gibbs sampler.

Let's go back to our
simple example from above to understand the motivation and functioning
of Gibbs sampling. Recall that for a Normal model with known variance
and a Normal prior for $\mu$, the posterior distribution of $\mu|y$ is also
Normal. Conversely, with a fixed (known) $\mu$, but unknown variance, the
conjugate prior for $\sigma^2$ is an Inverse-Gamma distribution with shape and scale parameters $a$ and $b$:
\[
\sigma^2 \sim InvGamma(a,b),
\]
With fixed $a$ and $b$, the posterior $p(\sigma|\mu,y)$ is also an Inverse Gamma distribution, namely:
\begin{equation}
\sigma|\mu,y \sim InvGamma (a_n, b_n),
\label{eq. 3}
\end{equation}
 where  $a_n = n/2   + a$ and $b_n = (1/2) \sum (y-\mu)^2 + b$.
However, what if we know neither $\mu$ nor $\sigma$, which is probably the
more common case? The joint posterior distribution of $\mu$ and $\sigma$
now has the general structure
\[
p(\mu, \sigma|y) = \frac{p(y|\mu) p(\mu) p(\sigma)}{ \int p(y|\mu) p(\mu) p(\sigma) d\mu d\sigma }
\]
or
\[
p(\mu, \sigma|y) \propto p(y|\mu) p(\mu) p(\sigma)
\]
\begin{comment} Rahel : use of p() here might violate some convention of the
book -- I dunno. Lets think about it \end{comment}
This cannot easily be reduced to a distribution we recognize. However,
we can condition $\mu$ on $\sigma$ (i.e., we treat $\sigma$ as fixed) and remove
all terms from the joint posterior distribution that do not involve $\mu$
to construct the full conditional distribution,
\[
p(\mu|\sigma,y)  \propto p(y|\mu) p(\mu)
\]

The full conditional of $\mu$ again takes the form of the Normal
distribution shown in Eq. \ref{mcmc.eq.mu-posterior}; similarly, $p(\sigma|\mu,y)$ takes
the form of the Inverse Gamma distribution shown in
Eq. \ref{eq. 3}  – both distribution we can easily sample
from. And this is precisely what we do when using Gibbs sampling – we
break down high-dimensional problems into convenient one-dimensional
problems by constructing the full conditional distributions for each
model parameter separately; and we sample from these full
conditionals, which, if we choose conjugate priors, are known
parametric distributions.
Let's put the concept of Gibbs sampling into the MCMC framework of
generating successive samples, using our simple Normal model with
unknown $\mu$ and $\sigma$ and conjugate priors as an example. These are the
steps you need to build a Gibbs sampler:

{\flushleft {\bf Step 0:} Begin with some initial values for $\theta$, $\theta^{(0)}$.   }
In our example, we have to specify initial values for $\mu$ and $\sigma$, for
example by drawing a random number from some uniform distribution, or
by setting them close to what we think they might be. (Note: This step
is required in any MCMC sampling – chains have to start from
somewhere. We will get back to these technical details a little
later.)
{\flushleft {\bf Step 1:} Draw $\theta^{(1)}$ from the conditional distribution p($\theta_{1}^{(1)}|\theta_{2}^{(0)}$,\ldots, $\theta_{d}^{(0)}$). }
Here, $\theta_1$ is $\mu$, which we draw from the Normal distribution in Eq. \ref{mcmc.eq.mu-posterior}  using $\sigma^{(0)}$ as value for $\sigma$.
{\flushleft {\bf Step 2:} Draw $\theta_{2}^{(1)}$ from the conditional distribution p($\theta_{2}^{(1)}|\theta_{1}^{(1)}$, $\theta_{3}^{(0)}$,\ldots, $\theta_{d}^{(0)}$). }
Here, $\theta_2$ is $\sigma$, which we draw from the Inverse Gamma
distribution of Eq. \ref{eq. 3}, using $\mu^{(1)}$ as value for $\mu$.

{\flushleft {\bf Step 3:} Draw $\theta_{d}^{(1)}$ from the conditional distribution p($\theta_{d}^{(1)}|\theta_{1}^{(1)}$,\ldots, $\theta_{d-1}^{(1)}$). }

In our example we have no additional parameters, so we only need step 0 through to 2.
Repeat Steps 1 to d for $T$ = a large number of samples.
In terms of {\bf R} coding, this means we have to write Gibbs updaters for
$\mu$ and $\sigma$ and embed them into a loop over $T$ iterations. The final
code in the form of an {\bf R} function is shown 
in Panel \ref{mcmc.panel.gibbs1}.


\begin{panel}[htp]
\centering
\rule[0.15in]{\textwidth}{.03in}
%\begin{minipage}{2.5in}
\begin{verbatim}
Normal.Gibbs<-function(y=y,mu0=mu0, sig0=sig0, a=a,b=b,niter=niter) {

ybar<-mean(y)
n<-length(y)
mu<-runif(1) #mean initial value
sig<-runif(1) #sd initial value
an<-n/2 + a

out<-matrix(nrow=niter, ncol=2)
colnames(out)<-c('mu', 'sig')

for (i in 1:niter) {

#update mu
mun<- (sig/(sig+n*sig0))*mu0 + (n*sig0/(sig+n* sig0))*ybar
sign <- (sig*sig0)/ (sig+n*sig0)
mu<-rnorm(1,mun, sqrt(sign))

#update sig
bn<- 0.5 * (sum((y-mu)^2)) +b
sig<-1/rgamma(1,shape=an, rate=bn)
out[i,]<-c(mu,sqrt(sig))

}
return(out)
}
\end{verbatim}
%\end{minipage}
\rule[-0.15in]{\textwidth}{.03in}
\caption{
R-code for a Gibbs sampler for a Normal model with unknown mu
and sig and conjugate (Normal and Inverse Gamma, respectively) priors
for both parameters.
}
\label{mcmc.panel.m0}
\end{panel}













This is it! You can use the code \mbox{\tt NormalGibbs.R} in the {\bf
  R} package \mbox{\tt scrbook}
to simulate some data, $y \sim \mbox{Normal}(5, 0.5)$ and run your first
Gibbs sampler. Your output will be a table with two columns, one per
parameter, and $T$ rows, one per iteration. For this 2-parameter example
you can visualize the joint posterior by plotting samples of $\mu$
against samples of $\sigma$ (Fig. \ref{postdist.fig}):
\begin{verbatim}
plot(out[,1], out[,2])
\end{verbatim}
The marginal distribution of each parameter is approximated by just
examining the samples of this particular parameter – you can visualize
it by plotting a histogram of the samples (Fig. \ref{plotsofPD.fig} a and b):
\begin{verbatim}
par(mfrow=c(1,2))
hist(out[,1]); hist (out[,2])
\end{verbatim}

\begin{comment}
Rahel: for some analyses you might want to use a specified random
number seed so that the reader can obtain exactly these results (or
that Figures look exactly the same if they are reanalyzed for
revisions. 
\end{comment}
Finally, recall an important characteristic of Markov chains, namely,
that the chain has to have converged (reached its stationary
distribution) in order to regard samples as coming from the posterior distribution. In
practice, that means you have to throw out some of the initial samples
– called the burn-in. We will talk about this in more when we talk
about convergence diagnostics. For now, you can use the
\verb#plot(out[,1])# or \verb#plot(out[,2])# command to make a time
series plot of the samples of each parameter and visually assess how
many of the initial samples you should discard. Fig. \ref{plotsofPD.fig} c and d shows
plots for the estimates of $\mu$ and $\sigma$ from our simulated data set;
you see that in this simple example the Markov chain apparently
reaches its stationary distribution very quickly – the chains look
'grassy' seemingly from the start. It is hard to discern a burn-in
phase visually (but we will see examples further on where the burn-in
is clearer) and you may just discard the first 500 draws to be sure
you only use samples from the posterior distribution. The mean of the
remaining samples are your estimates of mu and sig:
\begin{verbatim}
> summary(mod[501:10000,])
       mu                      sig
 Min.   : 4.936      Min.   : 0.4569
 1st Qu.: 4.984      1st Qu.: 0.4889
 Median : 4.994      Median : 0.4961
 Mean   : 4.994      Mean   : 0.4964
 3rd Qu.: 5.005      3rd Qu.: 0.5037
 Max.   : 5.062      Max.   : 0.5356
\end{verbatim}

\begin{figure}
\begin{center}
\includegraphics[height=3in]{Ch7/figs/postdist}
\end{center}
\caption{Joint posterior distribution of mu and sig from a Normal Model}
\label{postdist.fig}
\end{figure}

\begin{figure}
\begin{center}
\includegraphics[width=2.5in]{Ch7/figs/plotsofPD}
\end{center}
\caption{
Plots of the posterior distributions of $\mu$ (panel a) and
  $\sigma$ (b)
  from a Normal model and time series plots of $\mu$ (c) and $\sigma$ (d).}
\label{plotsofPD.fig}
\end{figure}
\begin{comment} 
Rahel: there is no (a) (b) etc.. in the figure.
might want to remake figure and label (a) (b) or else say ``upper
left'' and so on
\end{comment}

\subsection{ Metropolis-Hastings sampling   }

Although it is applicable to a wide range of problems, the limitations
of Gibbs sampling are immediately obvious – what if we do not want to
use conjugate priors (or what if we cannot recognize the full
conditional distribution as a parametric distribution, or simply do
not want to worry about these issues)? The most general solution is to
use the Metropolis-Hastings (MH) algorithm, which also goes back to
the work by \citet{metropolis_ulam:1953}. You saw the basics of this
algorithm in Chapt. \ref{chapt.glms}. In a nutshell, because we do not recognize the
posterior $p(\theta|y)$ as a parametric distribution, the MH algorithm
generates samples from a known proposal distribution, say $h(\theta)$,
that depends on $\theta$ at $t-1$. The $t^{th}$ sample is accepted with probability. 

\[
r = \frac{ f(\theta^{(t-1)}) h(\theta^{(t)}|\theta^{(t-1)})}
    {f(\theta^{(t)}) h(\theta^{(t-1)}|\theta^{(t)}) }
\]

Proposal distributions can be absolutely
anything!  You can generate candidate values from a $normal(0,1)$
distribution, from a uniform(-3455,3455) distribution, or anything of
proper support.  Note, however, that good choices of $h()$ are those
that approximate the posterior distribution. Obviously if $h() =
f(\theta|y)$ (i.e., the posterior) then you always accept the draw,
and it stands to reason that proposals that are more similar to
$f(\theta|y)$ will lead to higher acceptance probabilities. 

The original Metropolis algorithm
required $h(\theta)$ to be symmetric so that
$h(\theta^{(t)}|\theta^{(t-1)}) = h(\theta^{(t-1)}|\theta^{(t)})$. 
In that case these two terms just cancel
out from the MH acceptance probability and $r$ is then just the ratio
of the target density evaluated at the candidate value to that
evaluated at the current value. A later
development of the algorithm by \citet{hastings:1970} lifted this
condition. 
Since using a symmetric proposal distribution makes life a little
easier, we are going to focus on this specific case. A type of symmetric proposal useful in many situations is the
so-called {\it random-walk} proposal distribution where candidate values
are drawn from a normal distribution with mean equal to the current
value and some standard deviation, say $\delta$, which is prescribed by
the user (see below for further explanation). 

{\bf Parameters with bounded support}: Many models contain parameters that
have  bounded support. E.g., variance parameters live on $[0,\infty]$,
parameters that represent probabilities live on $[0,1]$, etc..
For such cases, it is sometimes convenient to use a random
walk proposal distribution that can generate any real number (e.g., a
normal random walk proposal). Then, 
we can just reject parameters that are
outside of the parameter space (XXXX REF FOR THIS MAYBE ROBERT AND
CASELLA BOOK ???? XXXX).

It is worth
knowing that there are alternatives to the random walk MH algorithm. For
example, in the independent M-H, $\theta^{(t)}$ does not depend on
$\theta^{(t-1)}$, while the Langevin algorithm \citep{roberts_etal:1998}
aims at avoiding the random walk by favoring moves towards regions of
higher posterior probability density. The interested reader should
look up these algorithms in \citet{robert_casella:2004} or
\citet{robert_casella:2010}.

Building a MH sampler can be broken down into several steps. We are going to demonstrate these steps using a different but still simple and common model – the logit-normal or logistic regression model. For simplicity, assume that
\[
y \sim \mbox{Bern}(\exp(\theta)/(1+ exp(\theta)))
\]
and
\[
\theta \sim \mbox{Normal}(\mu, \sigma)
\]
The following steps are required to set up a random walk MH algorithm:

{\flushleft {\bf Step 0}: Choose initial values, $\theta^{(0)}$.}

{\flushleft {\bf Step 1}: Generate a proposed value of $\theta$ at $t$ from $h(\theta^{(t)}|\theta^{(t-1)})$. }
We often use a Normal proposal distribution, so we draw $\theta^{(1)}$ from $\mbox{Normal}(\theta^{(0)}, \delta)$, where $\delta$ is the variance of the Normal proposal distribution, the tuning parameter that we have to set.

{\flushleft {\bf Step 2}: Calculate the ratio of posterior densities for the proposed and the original value for $\theta$: }
\[
r = \frac{p(\theta^{(t)}|y)}  {p(\theta^{(t-1)}|y)}
\]
In our example,
\[
r = \frac{\mbox{Bern}(y|\theta^{(t)}) * \mbox{Normal}(\theta^{(t)}|\mu, \sigma)} {\mbox{Bern}(y|\theta^{(t-1)}) * \mbox{Normal}(\theta^{(t-1)}|\mu, \sigma)}
\]
{\bf Step 3}: Set
\begin{eqnarray*}
\theta^{(t)}  &= &   \theta^{(t)} \mbox{ with probability min(r,1)}\\
	 & = & 	\theta^{(t-1)} \mbox{ otherwise }
\end{eqnarray*}

%should work now


We can do that by drawing a random number $u$ from a
$\mbox{Unif}(0,1)$ and accept $\theta^{(t)}$ if
$u<r$.
Repeat for $t = 1,2,\ldots$ a large number of samples.
The {\bf R} code for this MH sampler is provided in Panel 2 XXXX.

\begin{panel}[htp]
\centering
\rule[0.15in]{\textwidth}{.03in}
%\begin{minipage}{2.5in}
{\small
\begin{verbatim}
Logreg.MH<-function(y=y, mu0=mu0, sig0=sig0, niter=niter) {

out<-c()

theta<-runif(1, -3,3) #initial value

for (iter in 1:niter){
theta.cand<-rnorm(1, theta, 0.2)

loglike<-sum(dbinom(y, 1, exp(theta)/(1+exp(theta)), log=TRUE))
logprior <- dnorm(theta,mu0 ,sig0, log=TRUE)
loglike.cand<-sum(dbinom(y, 1, exp(theta.cand)/(1+exp(theta.cand)), log=TRUE))
logprior.cand <- dnorm(theta.cand, mu0, sig0, log=TRUE)

if (runif(1)<exp((loglike.cand+logprior.cand)-(loglike+logprior))){
theta<-theta.cand
}
out[iter]<-theta
}

return(out)
}
\end{verbatim}
}
%\end{minipage}
\rule[-0.15in]{\textwidth}{.03in}
\caption{
{\bf R} code to run a Metropolis sampler on a simple Logit-Normal model.
}
\label{mcmc.panel.logitnormal}
\end{panel}



The reason we sum the logs of the likelihood and the prior, rather than multiplying the original values, is simply computational. The product of small probabilities can be numbers very close to 0, which computers do not handle well. Thus we add the logarithms, sum, and exponentiate to achieve the desired result. Similarly, in case you have forgotten some elementary math, $x/y = exp(log(x)-log(y))$, with the latter being favored for computational reasons.

Comparing MH sampling to Gibbs sampling, where all draws from the
conditional distribution are used, in the MH algorithm we discard a
portion of the candidate values, which inherently makes in less
efficient than Gibbs sampling – the price you pay for its increased
generality.  In Step 1 of the MH sampler we had to choose a variance,
$\delta$, for the Normal proposal distribution. Choice of the
parameters that define our candidate distribution is also referred to
as 'tuning', and it is important since adequate tuning will make your
algorithm more efficient.  $\delta$ should be chosen (a) large enough
so that each step of drawing a new proposal value for $\theta$ can
cover a reasonable distance in the parameter space, as otherwise,
mixing of the Markov chain is inefficient and chains will tend to have
strong autocorrelation; and (b) small enough so that proposal values
are not rejected too often, as otherwise the random walk will 'get
stuck' at specific values for too long.  As a rule of thumb, your
candidate value should be accepted in about 40\% of all
cases. Acceptance rates of 20 -- 80\% are probably ok, but anything
below or above may well render your algorithm inefficient (this does
not mean that it will give you wrong results – only that you will need
more iterations to converge to the posterior distribution). In
practice, tuning will require some 'trial-and-error', some common
sense and, with enough experience, some intuition. Or, one can use an adaptive phase, where the tuning parameter
is automatically adjusted until it reaches a user-defined acceptance
rate, at which point the adaptive phase ends and the actual Markov
chain begins. This is computationally a little more
advanced. \citet{link_barker:2009} discuss this in more detail. It is
important the samples drawn during the adaptive phase are discarded.
To illustrate the effects of tuning, we ran the
Metropolis-within-Gibbs algorithm in Panel \ref{mcmc.panel.logitnormal} with $\delta=0.01$,
$\delta=0.2$ and $\delta=1$. The first 150 iterations for $\theta$ are
shown in Fig. \ref{mcmc.fig.tuning}. We see that for a very small
$\delta$ (the dashed line) the burn-in is extremely slow - after 150
iterations the chain isn't even half way there, while for the other
two values of $\delta$ (solid and dotted)the burn-in phase seems to be
over after only about 10 iterations. While $\delta=0.2$ leads to
reasonably good mixing, the chain clearly gets stuck on certain values
with $\delta=1$.
%'tuning' is a new figure I made... don't know about the size specifications, just copied those from another picture. Do you set them at 
%the actual size of the figure or the size you want it to be??
 \begin{figure}
\begin{center}
\includegraphics[height=3in,width=4in]{Ch7/figs/tuning}
\end{center}
\caption{Time series plots of $\theta$ from a MH algorithm with tuning parameter  $\delta = 0.01$ (dashed line), 0.2 (solid line) and  1 (dotted line).}
\label{mcmc.fig.tuning}
\end{figure}

Other than graphically, you can easily check acceptance rates for the parameters you monitor (that are part of your output) using the rejectionRate() function of the package coda (we will talk more about this package a little later on). Do not let the term 'rejection rate' confuse you; it is simply 1 -- acceptance rate. There may be parameters – for example, individual values of a random effect or latent variables – that you do not want to save, though, and in our next example we will show you a way to monitor their acceptance rates with a few extra lines of code.



\subsection{ Metropolis-within-Gibbs }

One weakness of the MH sampler is that formulating the joint posterior when evaluating whether to accept or reject the candidate values for $\theta$ becomes increasingly complex or inefficient as the number of parameters in a model increases. As you already saw in Chapter 2, in these cases you can simply combine MH sampling and Gibbs sampling. You can use Gibbs sampling to break down your high-dimensional parameter space into easy-to-handle one-dimensional conditional distributions and use MH sampling for these conditional distributions. Better yet – if you have some conjugacy in your model, you can use the more efficient Gibbs sampling for these parameters and one-dimensional MH for all the others. You have already seen the basics of how to build both types of algorithms, so we can jump straight into an example here and build a Metropolis-within-Gibbs algorithm.

\section{ GLMMs – Poisson regression with a random effect }

Let's assume a model that gets us closer to the problem we ultimately
want to deal with - a GLMM. Here, we assume we have Poisson counts,
$y_{ij}$, from $i=1,2,\ldots,n$ plots \begin{comment} Rahel: I put
  ``n'' here because you had ``n'' used above. But I wonder if we
  should have a convention for ``number of sites'' and use that
  throughout the book?  I think we use j=1,..,J for traps .... I can't
  remember what I used in Ch. 2. but maybe i'm overthinking
  this. lets talk later. \end{comment} in $j$ different study sites, and we believe that the counts are influenced by some plot-specific covariate, $x$, but that there is also a random site effect. So our model is:
\[
y_{ij} \sim \mbox{Poisson}(\lambda_{ij})
\]
\[
\lambda_{ij} = \exp (a_j + bx_i)
\]
XXX INDEXING OF x should be ij here??? XXXXXX
Let's use Normal priors on $a$ and $b$,  \[
a_j \sim \mbox{Normal} (\mu_a, \sigma_a)
\]
and
\[
b \sim \mbox{Normal} (\mu_b, \sigma_b)
\].

Since we want to estimate the random effect in this model, we do not
specify $\mu_a$ and $\sigma_a$, but instead, estimate them as well, so we have
to specify hyperpriors for these parameters:
\begin{eqnarray*}
\mu_a  &\sim &  \mbox{Norm}(\mu_0, \sigma_0)  \\
\sigma_{a} & \sim & \mbox{InvGamma}(a_0, b_0)
\end{eqnarray*}
% In this entire section below I am unsure of the indexing of a, y and
% x. I know what I'm trying to say but I'm not sure I'm saying it
% correctly... could someone please check? So what I am trying to say,
% for example in the expression for a1 (a at j=1), is that you need
% all the yi's at j=1, and all the x at j=1.
With the model fully specified, we can compile the full conditionals,
breaking the multi-dimensional parameter space into one-dimensional
components:
%%this works but it doesn't look paritcularly pretty
\begin{comment} 
Rahel: check these eqns out , I set it using eqnarray 
\end{comment}
\begin{eqnarray*}
p(a_1|a_2,a_3,\ldots,a_j,b,{\bf y}) & \propto &   p({\bf y}_{1}|a_1,b) * p(a_1) \\
	 & \propto  &   \mbox{Poisson}({\bf y}_{1}| \exp(a_1 + bx_i)) * \mbox{Norm}(a_1|\mu_a, \sigma_a)
\end{eqnarray*}
where ${\bf y}_{1} = (y_{11},y_{21}, \ldots, y_{n1})$ is the vector of
observed counts for site $j=1$ and, in general, ${\bf y}_{j}$ is the
vector of all counts for site $j$. The other full conditionals for
each $a_{j}$ are constructed similarly:
\begin{comment} 
Rahel: I think the indexing is out of order. you
  should have ``groups'' be the first index, i.e., ``i'' in this case,
  and ``replicates'' being the 2nd index, what you have as ``j''
\end{comment}
\begin{eqnarray*}
p(a_2|a_1,a_3,\ldots,a_j,b,{\bf y}) & \propto&  p({\bf y}_{2}|a_2,b) * p(a_2) \\
	 & \propto  & \mbox{Poisson}({\bf y}_{2}| \exp(a_2 + bx_i)) * \mbox{Norm}(a_2|\mu_a, \sigma_a)
\end{eqnarray*}
and so on for all elements of ${\bf a}$. The full-conditional for $b$ is:
\begin{eqnarray*}
p(b|a,y) &\propto & p({\bf y}|a,b) * p(b) \\
	 &\propto& \mbox{Poisson}({\bf y}|exp(a + b{\bf x})) *\mbox{Norm}(b|\mu_b, \sigma_b)
\end{eqnarray*}

Finally, we need to update the hyperparameters for the random effects
vector ${\bf a}$:
\[
p(\mu_a|{\bf a}) \propto p({\bf a}|\mu_a, \sigma_a) *p(\mu_a)
\]
\[
p(\sigma_a|a) \propto p(a|\mu_a, \sigma_a) *p(\sigma_a)
\]
Since we assumed $a$ to come from a Normal distribution, the choice of priors for $\mu_a$ (Normal) and $\sigma_a$ (Inverse-Gamma) leads to the same conjugacy we observed in our initial Normal model, so that both hyperparameters can be updated using Gibbs sampling.

Now let's build the updating steps for these full conditionals. Again, for the MH steps that update $a$ and $b$ we use Normal proposal distributions with standard deviations $\delta_{a}$ and $\delta_{b}$.

First, we set the initial values $a^{(0)}$ and $b^{(0)}$. Then, starting with $a_1$, we draw $a_1^{(1)}$ from $\mbox{Norm}(a_1^{(0)}, \delta_{a})$, calculate the conditional posterior density of $a_1^{(0)}$ and $a_1^{(1)}$  and compare their ratios,
\[
r = \frac{\mbox{Poisson}({\bf y}_{1}|exp(a_1^{(1)} + b x_i)) *
  \mbox{Norm}(a_1^{(1)}|\mu_a, \sigma_a)} {\mbox{Poisson}({\bf y}_{1}|exp(a_1^{(0)} + bx_i)) * \mbox{Norm}(a_1^{(0)}|\mu_a, \sigma_a)}
\]
and accept $a_1^{(1)}$ with probability $min(r,1)$. We repeat this for all a's.

For $b$, we draw $b_1^{(1)}$ from $\mbox{Norm} (b^{(0)}, \delta_{b})$, compare the posterior densities of $b^{(0)}$ and $b^{(1)}$,
\[
r = \frac{\mbox{Poisson}({\bf y}|exp(a + b_1^{(1)}{\bf x}))
  *\mbox{Norm}(b_1^{(1)}|\mu_b, \sigma_b)} { \mbox{Poisson}({\bf
    y}|exp(a + b_1^{(0)}{\bf x})) *\mbox{Norm}(b_1^{(0)}|\mu_b, \sigma_b)},
\]
and accept $b_1^{(1)}$  with probability $min(r,1)$.

For $\mu_a$ and $\sigma_a$, we sample directly from the full conditional distributions (Eq. \ref{XX}  and Eq. \ref{XX}):
\[
\mu_a^{(1)} \sim \mbox{Norm} (\mu_n, \sigma_n)
\]
where 
\[\mu_n =  \frac{\sigma_a^{(0)}}  {\sigma_a^{(0)}   +n_a  *  \sigma_0} *  \mu_0 +  \frac{n_a * \sigma_0} {\sigma_a^{(0)}   +n_a* \sigma_0} *\bar{a}^{(1)}
\]
and 
\[
\sigma_n= \frac{\sigma_a^{(0)}  * \sigma_0 } {\sigma_a^{(0)}  + n* \sigma_0}
\]
Here, $\bar{a}$ is the current mean of the vector $\bf{a}$, which we
updated before, and $n_a$ is the length of $\bf{a}$. 
For $\sigma_a$ we use $\sigma_a^{(1)}\sim InvGamma (a_n, b_n)$,
where  $a_n = n_a/2   + a_0$, and $b_n = 0.5 ( \displaystyle\sum\limits_{j=1}^{n_a} a_j^{(1)}-\mu_a^{(1)})^2+ b_0$.
\begin{comment} Rahel: in preceeding setntence you are using ``a'' for
  parameters of the prior distribution of ``sigma'' (and also ``b'' is
  a prior parameter) -- but ``a'' and ``b'' have been previously
  defined as regression parameter.
Recommend changing all ``a'' and ``b'' in the original model to
$\alpha$ and $\beta$ (I realize a pain in the ass
\end{comment}



We repeat these steps over $T$ iterations of the MCMC algorithm.
In this example we may not want to save each individual $a$, but are only interested in their mean and standard deviation. Since these two parameters will change as soon as the value for one element in $\bf{a}$ changes, their acceptance rates will always be close to 1 and are not representative of how well your algorithm performs. To monitor the acceptance rates of parameters you do not want to save, you simply need to add a few lines of code into your updater to see how often the individual parameters are accepted. The full code for the MCMC algorithm of our Poisson GLMM in Panel 3 (XXX) shows one way how to monitor acceptance of individual $a$'s.

\begin{comment}
 Rahel: The panel below might be too big for a panel. Can you just
 have it in the text in verbatim?
\end{comment}
{\small
\begin{verbatim}
Panel 3: R code for the Metropolis-within-Gibbs sampler for
a Poisson regression with random intercepts.

Pois.reg<-function(y=y,site=site,mu0=mu0,sig0=sig0,a0=a0,b0=b0,
          mub=mub, sigb=sigb, niter=niter){

lev<-length(unique(site))     #number of sites
a<-runif(lev,-5,5)		#initial values a
b<-runif(1,0,5)			#initial value b
mua<-mean(a)
siga<-sd(a)

out<-matrix(nrow=niter, ncol=3)
colnames(out)<-c('mua','siga','b')

for (iter in 1:niter) {

#update a
aUps<-0			  #initiate counter for acceptance rate of a
for (j in 1:lev) { 	  #loop over sites
a.cand<-rnorm(1, a[j], 0.1)	#update intercepts a one at a time
loglike<- sum(dpois (y[site==j], exp(a[j] + b*x[site==j]), log=TRUE))
logprior<- dnorm(a[j], mua,siga, log=TRUE)
loglike.cand<- sum(dpois (y[site==j], exp(a.cand + b *x[site==j]), log=TRUE))
logprior.cand<- dnorm(a.cand,  mua,siga, log=TRUE)
if (runif(1)< exp((loglike.cand+logprior.cand) –(loglike+logprior))) {
a[j]<-a.cand
aUps<-aUps+1
}
}

if(iter %% 100 == 0) {  #this lets you check the acceptance rate of a at every 100th iteration
            cat("   Acceptance rates\n")
            cat("     a =", aUps/lev, "\n")
}

#update b
b.cand<-rnorm(1, b, 0.1)
avec<-rep(a, times=c(rep(10,10)))
loglike<- sum(dpois (y, exp(avec + b*x), log=TRUE))
logprior<- dnorm(b, mub,sigb, log=TRUE)
loglike.cand<- sum(dpois (y, exp(avec + b.cand *x), log=TRUE))
logprior.cand<- dunif(b.cand, mub,sigb, log=TRUE)
if (runif(1)< exp((loglike.cand+logprior.cand) – (loglike+logprior) )) {
b<-b.cand
}

#update mua using Gibbs sampling
abar<-mean(a)
mun<- (siga/(siga+lev*sig0))*mu0 + (lev*sig0/(siga+lev* sig0))*abar
sign <- (siga*sig0)/ (siga+lev*sig0)
mua<-rnorm(1,mun, sqrt(sign))

#update siga using Gibbs sampling
a0n<-lev/2 + a0
b0n<- 0.5 * (sum((a-mua)^2)) +b0
siga<-1/rgamma(1,shape=a0n, rate=b0n)

out[iter,]<-c(mua, sqrt(siga), b)

}

return(out)
}
\end{verbatim}
}

\subsection{Rejection sampling and slice sampling }

While MH and Gibbs sampling are probably the most widely applied
algorithms for posterior approximation, there are other options that
work under certain circumstances and may be more efficient when
applicable. {\bf WinBUGS} applies these algorithms and we want you to be
aware that there is more out there to approximate posterior
distributions than Gibbs and MH.  One alternative algorithm is
rejection sampling. Rejection sampling is not an MCMC method, since
each draw is independent of the others. The method can be used when
the posterior $p(\theta|y)$ is not a known parametric distribution but
can be expressed in closed form. Then, we can use a so-called envelope
function, say, $g(\theta)$, that we can easily sample from, with the
restriction that $p(\theta|y) < M * g(\theta)$. We then sample a
candidate value for $\theta$ from $g(\theta)$, calculate $r =
p(\theta|y)/M*g(\theta)$ and keep the sample with the probability
$r$. $M$ is a constant that has to be picked so that $r$ lies between
0 and 1, for example by evaluating both $p(\theta|y)$ and $g(\theta)$
at $n$ points and looking at their ratios. Rejection sampling only
works well if $g(\theta)$ is similar to $p(\theta|y)$, and packages
like {\bf WinBUGS} use adaptive rejection sampling \citep{gilks_wild:1992},
where a complex algorithm is used to fit an adequate and efficient
$g(\theta)$ based on the first few draws. 
Though efficient in some
situations, rejection sampling does not work well with
high-dimensional problems, since it becomes increasingly hard to
define a reasonable envelope function. For an example of rejection
sampling in the context of SCR models, see
Chapt. \ref{chapt.state-space}, where we use it to simulation
non-stationary point processes.  Another alternative is slice sampling
\citep{neal:2003}. In slice sampling, we sample uniformly from the
area under the plot of $p(\theta|y)$. Considering a single univariate
theta. Let's define an auxiliary variable, $U \sim \mbox{Unif}(0,
p(\theta|y))$. Then, $\theta$ can be sampled from the vertical slice
of $p(\theta|y)$ at $U$ (Fig. \ref{mcmc.fig.slicesample}):
\[
\theta|U \sim \mbox{Unif}(B),
\]
where $B = \{\theta: p(\theta|y) \geq U\}$
%do these symbols mean  'B is element of the interval of all theta for which p(theta) is larger than or equal to U?' 

\begin{figure}
\begin{center}
\includegraphics[height=2in]{Ch7/figs/slicesampling}
\end{center}
\caption{Slice sampling. For...}
\label{mcmc.fig.slicesample}
\end{figure}

\footnote{there are supposed to be equations in the caption of figure
4 but it kept causing errors. Rahel: Let me see the equations you want
in there....}

Slice sampling can be applied in many situations; however,
implementing an efficient slice sampling procedure can be
complicated. We refer the interested reader to 
\citet[][Chapt. 7]{robert_casella:2010} for a simple example.  Both rejection
sampling and slice sampling can be applied on one-dimensional
conditional distributions within a Gibbs sampling setup.

\section{MCMC for closed capture-recapture Model Mh}
\begin{comment}
\subsection{Building your own MCMC algorithm}
% subsection needed here? maybe not?
\end{comment}

By now you have seen MCMC samplers for some simple GL(M)M's. Now, to
ease you into more complex models, we construct our own MCMC algorithm
using a Metropolis-within-Gibbs sampler for the non-spatial Model with
individual heterogeneity in capture probability $M_{h}$, developed in
Chapt. \ref{chapt.closed}.

To recapitulate: Under the non-spatial model, each of the $n$ observed
individuals is either detected (1) or not (0) during each of $K$
sampling occasions. We estimate $N$ using data augmentation and have a
Bernoulli model for the zero-inflation variables $z_{i}$. The binomial
observation model is expressed conditional on the latent variables
$z_{i}$. Further, we prescribe a distribution for the capture
probability $p_{i}$. Here we assume
\[
\mathrm{logit}(p_{i}) \sim \mbox{Normal}(\mu,\sigma^2)
\]

As usual, we have to go through two general steps before we write the MCMC algorithm:
\begin{itemize}
\item[  (1)] Identify your model with all its components (including
    priors)
\item[  (2)] Recognize and express the full conditional distributions for
    all parameters
\end{itemize}
Our model components are as follows: $[y_{i}| p_{i},z_{i}]$,
$[p_{i}|\mu_{p},\sigma_{p}]$, and $[z_{i}|\psi]$
for {\it each} $i=1,2,\ldots,M$ and then prior distributions
$[\mu_{p}]$, $[\sigma_{p}]$ and $[\psi]$.
The joint posterior distribution of all unknown quantities in the model
is proportional to the joint distribution of all elements
$y_{i},p_{i},z_{i}$ and also the prior distributions of the prior parameters:
\[
\left\{ \prod_{i=1}^{M} [y_{i}|p_{i},z_{i}][p_{i}|\mu_{p},\sigma_{p}]
[z_{i}|\psi] \right\} [\mu_{p},\sigma_{p},\psi]
\]
For prior distributions, we assume that $\mu_{p},\sigma_{p}, \psi$ are
mutually independent and for $\mu_{p}$ and $\sigma_{p}$ we use
improper uniform priors, and $\psi \sim \mbox{Unif}(0,1)$.  Note that
the likelihood contribution for each individual, when conditioned on
$p_{i}$ and $z_{i}$, does not depend on $\psi$, $\mu_{p}$, or
$\sigma_{p}$.  As such, the full-conditionals for the structural
parameters $\psi$ only depends on the collection of data augmentation
variables $z_{i}$, and that for $\mu_{p}$ and $\sigma_{p}$ will only
depends on the collection of latent variables $p_{i}; i=1,2,\ldots,M$.
The full conditionals for all the unknowns are as follows:

{\bf (1)} For $p_{i}$:
\begin{eqnarray*}
[p_{i}|y_{i}, \mu_p, \sigma_{p},z_{i}=1] &\propto  &
[y_{i}|p_{i}][p_{i}|\mu_p,\sigma_{p}^{2}] \mbox{ if $z_{i}=1$ }  \\
                 &  &  [p_{i}|\mu_p,\sigma_{p}] \mbox{ if $z_{i}=0$ }
\end{eqnarray*}

{\bf (2)} for $z_{i}$:
\[
z_{i} | \cdot \propto [y_{i}|z_{i}*p_{i}] \mbox{Bern}(z_{i}|\psi)
\]

{\bf (3)} For $\mu_{p}$:
\[
[\mu_{p} | \cdot ] \sim \left\{ \prod_{i} [p_{i}| \cdot] \right\} *\mbox{const}
\]


{\bf (4)} For $\sigma_{p}$:
\[
[ \sigma_{p}|\cdot ] \sim \left\{ \prod_{i}[p_{i}| \cdot ] \right\} *\mbox{const}
\]

{\bf (5)} For $\psi$:
\[
\psi|\cdot\sim \mbox{Beta}(1 + \sum z_{i}, 1 + M - \sum z_{i})
\]


What we've done here is identify each of the full conditional
distributions in sufficient detail to toss them into our
Metropolis-Hastings algorithm. With the exception of $\psi$ which has
a convenient analytic solution -- it is a beta distribution which we
can easily sample directly. In truth, we could also sample $\mu_{p}$
and $\sigma_{p}^{2}$ directly with certain choices of prior
distributions. For example, if $\mu_{p} \sim \mbox{Norm}(0, 1000)$
then the full conditional for $\mu_{p}$ is also normal (see
sec. \ref{mcmc.sec.gibbs}), etc..
We implement an MCMC algorithm for this model in the following block
of {\bf R} code.  
\begin{comment}
Rahel: Yes you can have the script and modify to be consistent with
your stuff. I will send shortly.
%Andy, I think we should edit code so it's comparable to the rest in this chapter (eg loglik instead of lik.curr etc. 
%Should I do this in here or do you have an R script file for this code that's goint in the scrbook package? Then, I'd probably better
% edit the .R file and paste it in here so they match.
\end{comment}
\begin{verbatim}

## obtain the bear data by executing the previous data grabbing
## function

temp<-getdata()
M<-temp$M
K<-temp$K
ytot<-temp$ytot


###
### MCMC algorithm for Model Mh

out<-matrix(NA,nrow=100000,ncol=4)
dimnames(out)<-list(NULL,c("mu","sigma","psi","N"))
lp<- rnorm(M,-1,1)
p<-expit(lp)
mu<- -1
p0<-exp(mu)/(1+exp(mu))
sigma<- 1
psi<- .5
z<-rbinom(M,1,psi)
z[ytot>0]<-1

for(i in 1:100000){

### update the logit(p) parameters
lp.cand<- rnorm(M,lp,1)  # 0.5 is a tuning parameter
p.cand<-expit(lp.cand)
lik.curr<-log(dbinom(ytot,K,z*p)*dnorm(lp,mu,sigma))
lik.cand<-log(dbinom(ytot,K,z*pc)*dnorm(lpc,mu,sigma))
kp<- runif(M) < exp(lik.cand-lik.curr)
p[kp]<-pc[kp]
lp[kp]<-lpc[kp]

p0c<- rnorm(1,p0,.05)
if(p0c>0 & p0c<1){
muc<-log(p0c/(1-p0c))
lik.curr<-sum(dnorm(lp,mu,sigma,log=TRUE))
lik.cand<-sum(dnorm(lp,muc,sigma,log=TRUE))
if(runif(1)<exp(lik.cand-lik.curr)) {
 mu<-muc
 p0<-p0c
}
}

sigmac<-rnorm(1,sigma,.5)
if(sigmac>0){
lik.curr<-sum(dnorm(lp,mu,sigma,log=TRUE))
lik.cand<-sum(dnorm(lp,mu,sigmac,log=TRUE))
if(runif(1)<exp(lik.cand-lik.curr))
 sigma<-sigmac
}

### update the z[i] variables
zc<-  ifelse(z==1,0,1)  # candidate is 0 if current = 1, etc..
lik.curr<- dbinom(ytot,K,z*p)*dbinom(z,1,psi)
lik.cand<- dbinom(ytot,K,zc*p)*dbinom(zc,1,psi)
kp<- runif(M) <  (lik.cand/lik.curr)
z[kp]<- zc[kp]

psi<-rbeta(1, sum(z) + 1, M-sum(z) + 1)

out[i,]<- c(mu,sigma,psi,sum(z))
}
\end{verbatim}



{\bf Remarks}: (1) for parameters with bounded support, i.e.,
$\sigma_{p}$ and $p_{0}$, we are using a random walk candidate
generator but rejecting draws outside of the parameter space.  (2) We
mostly use Metropolis-Hastings except for the data augmentation
parameter $\psi$ which we sample directly from its full-conditional
distribution which is a beta distribution.  (3) Even the latent data
augmentation variables $z_{i}$ are updated using Metropolis-Hastings
although they too can be updated directly from their full-conditional.

\section{MCMC algorithm for the basic spatial capture-recapture model}

Conceptually, but also in terms of MCMC coding, it is only a small step from the non-spatial model Mh to a fully spatial capture-recapture model. Next, we'll walk you through the steps of building your own MCMC sampler for the basic SCR model (i.e. without any individual, site or time specific covariates) with both a Poisson and a binomial encounter process.
As usual, we will have to go through two general steps before we write the MCMC algorithm:
\begin{itemize}
\item[  (1)] Identify your model with all its components (including
    priors)
\item[  (2)] Recognize and express the full conditional distributions for
    all parameters
\end{itemize}
It is worthwhile to go through all of step 1 for an SCR model, but you
have probably seen enough of step 2 in our previous examples to get
the essence of how to express a full conditional
distribution. Therefore, we will exemplify step 2 for some parameters
and tie these examples directly to the respective R code.

{\bf Step 1 -- Identify your model}

Recall the components of the basic SCR model with a Poisson encounter process from Chapt. \ref{chapt.poisson-mn}:
We assume that individuals $i$, or rather, their activity centers
${\bf s}_i$, are uniformly distributed across the state space ${\cal S}$,
\[
{\bf s}_i  \sim \mbox{Unif}({\cal S})
\]
and that the number of times individual $i$ encounters trap $j$, $y_{ij}$, is a random Poisson variable with mean $\lambda_{ij}$,
\[
y_{ij} \sim \mbox{Poisson}(\lambda_{ij})
\]
The link between individual location, movement and trap encounter
rates is made by the assumption that $\lambda_{ij}$, is a decreasing
function of the distance between ${\bf s}_i$ and the location of $j$,
${\bf x}_{j}$, say $D_{ij} = ||{\bf s}_{i} - {\bf x}_{j}||$, of the half-normal form
\[
\lambda_{ij} =  \lambda_0  \exp(-D_{ij}^2/2\sigma^2),
\]
where $\lambda_0$ is the baseline trap encounter rate at $D_{ij}=0$ and $\sigma$ controls the shape of the half-normal function.

In order to estimate the number of ${\bf s}_i$ in ${\cal S}$, or any
subset of ${\cal S}$, $N$, we use data augmentation (sec. \ref{closed.sec.da}) and create $M-n$ all-0 encounter histories, where $n$ is the number of individuals we observed and $M$ is an arbitrary number that is larger than $N$. We estimate $N$ by summing over the auxiliary data augmentation variables, $z_i$, which is 1 if the individual is part of the population and 0 if not, and assume that $z_i$ is a random Bernoulli variable,
\[
z_{i} \sim \mbox{Bern}(\psi)
\]

To link the two model components, we modify our trap encounter model to
\[
\lambda_{ij} = \lambda_0 * exp(-D_{ij}^2/2\sigma^2) * z_{i}.
\]
The model has the following structural parameters, for which we need to specify priors:
\begin{itemize}
\item[ $\psi$:] the $\mbox{Unif}(0,1)$ is required as part of the data augmentation procedure and in general is a natural choice of an uninformative prior for a probability; note that this is equivalent to a $\mbox{Beta}(1,1)$ prior, which will come in handy later.
\item[ ${\bf s}_{i}$:] since ${\bf s}_{i}$ is a pair of coordinates it is two-dimensional and we use a uniform prior limited by the extent of our state-space over both dimensions.
\item[ $\sigma$:] we can conceive several priors for $\sigma$ but let's assume an improper prior, one that is Uniform over $(-\infty, \infty)$. We will see why this is convenient when we construct the full conditionals for $\sigma$.
\item[ $\lambda_{0}$:] analogous, we will use a $\mbox{Unif}(-\infty, \infty)$ improper prior for $\lambda_{0}$.
\end{itemize}
The parameter that is the objective of our modeling, $N$, is a derived parameter that we can simply obtain by summing all $z_i$:
\[
N = \sum_{i=1}^{M} z_{i}
\]

{\bf Step 2 -- Construct the full conditionals:}
Having completed step 1, let's look at the full conditional distributions for some of these parameters.
We find that with improper priors, full conditionals are proportional only to the likelihood of the observations; for example, take the movement parameter $\sigma$:
\[
[\sigma|{\bf s}, \lambda_{0}, {\bf z}, {\bf y}] \propto \left\{ \prod_{i} [y_{i}| {\bf
    s}_{i}, \lambda_{0}, z_{i}, \sigma] \right\} * [\sigma]
\]
Since the improper prior implies that $[\sigma] \propto 1$, we can reduce this further to
\[
[\sigma|{\bf s}, \lambda_{0}, {\bf z}, {\bf y}] \propto \left\{
  \prod_{i} [y_{i}| {\bf s}_{i}, \lambda_{0}, z_{i}, \sigma] \right\}
\]
The {\bf R} code to update $\sigma$ is shown in Panel
\ref{mcmc.panel.updatesigma}.
 Notice that we automatically reject negative candidate values, since $\sigma$ cannot be $<0$.  

\begin{panel}[htp]
\centering
\rule[0.15in]{\textwidth}{.03in}
%\begin{minipage}{2.5in}
{\small
\begin{verbatim}
sig.cand <- rnorm(1, sigma, 0.1)	#draw candidate value
 if(sig.cand>0){   #automatically reject sig.cand that are <0
     lam.cand <- lam0*exp(-(D*D)/(2*sig.cand*sig.cand))
     ll<- sum(dpois(y, lam*z, log=TRUE))
     llcand <- sum(dpois(y, lam.cand*z, log=TRUE))
     if(runif(1) < exp( llcand  - ll) ){
         ll<-llcand
         lam<-lam.cand
         sigma<-sig.cand
      }
  }
\end{verbatim}
}
%\end{minipage}
\rule[-0.15in]{\textwidth}{.03in}
\caption{
{\bf R} code to update sigma within an MCMC algorithm for
an SCR model when using an improper prior
}
\label{mcmc.panel.updatesigma}
\end{panel}


These steps are analogous for  $\lambda_{0}$ and ${\bf s}_i$ and we will 
use MH steps for
all of these parameters. Similar to the random intercepts in our
Poisson GLMM, we update each ${\bf s}_i$ individually. Note that to be fully
correct, the full conditional for ${\bf s}_i$ contains both the likelihood and
prior component, since we did not specify an improper, but a Uniform
prior on ${\bf s}_i$. However, with a Uniform distribution the probability
density of any value is 1/(upper limit - lower limit) =
constant. Thus, the prior components are identical for both the
current and the candidate value and can be ignored (formally, when you
calculate the ratio of posterior densities, $r$, the identical prior
component appears both in the numerator and denominator, so that they
cancel each other out).

We still have to update $z_i$. The full conditional for $z_i$ is
\[
[z_i|y_{i}, \sigma, \lambda_0, s] \propto [y_{i}|z_{i},\sigma, \lambda_0, 
{\bf s}_{i}] * [z_i]
\]
and since $z_i \sim Bernoulli(\psi)$,
the term has to be taken into account when updating $z_i$. The 
{\bf R} code for updating $z_i$ is shown in Panel \ref{mcmc.panel.updatez}.


\begin{panel}[htp]
\centering
\rule[0.15in]{\textwidth}{.03in}
%\begin{minipage}{2.5in}
{\small
\begin{verbatim}
        zUps <- 0		#set counter to monitor acceptance rate
        for(i in 1:M) {
            if(seen[i])	#no need to update seen individuals, since their z =1
                next
            zcand <- ifelse(z[i]==0, 1, 0)
            llz <- sum(dpois(y[i,],lam[i,]*z[i], log=TRUE))
            llcand <- sum(dpois(y[i,], lam[i,]*zcand, log=TRUE))

            prior <- dbinom(z[i], 1, psi, log=TRUE)
            prior.cand <- dbinom(zcand, 1, psi, log=TRUE)
            if(runif(1) < exp( (llcand+prior.cand) - (llz+prior) )) {
                z[i] <- zcand
                zUps <- zUps+1
            }
        }
\end{verbatim}
}
%\end{minipage}
\rule[-0.15in]{\textwidth}{.03in}
\caption{
{\bf R} code to update z................
}
\label{mcmc.panel.updatez}
\end{panel}


$\psi$
 is a hyperparameter of the model, with an uninformative prior 
 distribution of $\mbox{Unif}(0,1)$ or $\mbox{Beta}(1,1)$, so that
\[
\psi|{\bf z} \propto \left\{ \prod_{i} [z_{i}|\psi] \right\} \mbox{Beta}(1,1)
\]
The beta distribution is the conjugate prior to the binomial and 
Bernoulli distributions (remember that $z \sim \mbox{Bern}(\psi))$. 
The general form of a full conditional of a Beta-Binomial model 
with $y_{i} \sim \mbox{Bern} (p) $ and $p \sim \mbox{Beta}(a,b)$ is
\[
p(p|y) \propto \mbox{Beta}(a + \sum y_i, b + n-\sum y_i)
\]
\begin{comment} Rahel: above is why i don't like p for distribution! \end{comment}
In our case, this means we update $\psi$ as follows:
\begin{verbatim}
si<-rbeta(1, 1+sum(z), 1 + M-sum(z))
\end{verbatim}
\begin{comment} Rahel what is M here  ? \end{comment}
These are all the building blocks you need to write the MCMC algorithm
for the spatial null model with a Poisson encounter process.  You can
find the full {\bf R} code (\mbox{\tt SCR0pois.R}) in the {\bf R} package 
\mbox{\tt scrbook}.

\subsection{SCR model with binomial encounter process}
The equivalent SCR model with a binomial encounter process is very similar. Here, each individual $i$ can only be detected once at any given trap $j$ during a sampling occasion $k$.
Thus
\[
y_{ij} \sim \mbox{Bin} (p_{ij}, K)
\]
Where $p_{ij}$ is some function of distance between ${\bf s}_{i}$ and trap location ${\bf x}_{j}$. Here we use:
\[
p_{ij}=1-exp(-\lambda_{ij})
\]
Recall from Chapter 2 that this is the complementary log-log (cloglog) link function, which constrains $p_{ij}$ 
to fall between 0 and 1.
For our MCMC algorithm that means that, instead of using a Poisson 
likelihood, $\mbox{Poisson}(y|\sigma,\lambda_0,{\bf s},z)$, we use a 
Binomial likelihood, $\mbox{Bin}(y| \sigma,\lambda_0,{\bf s},z; K)$, 
in all the conditional distributions. As an example, Panel
\ref{mcmc.panel.updatelam0} shows 
the updating step for $\lambda_0$ under a binomial encounter model. 
The full MCMC code for the binomial SCR (\mbox{\tt SCR0binom.R}) 
can be found in the {\bf R} package \mbox{\tt scrbook}.


\begin{panel}[htp]
\centering
\rule[0.15in]{\textwidth}{.03in}
%\begin{minipage}{2.5in}
{\small
\begin{verbatim}

        lam0.cand <- rnorm(1, lam0, 0.1)
        if(lam0.cand >0){   #automatically reject lam0.cand that are <0
            lam.cand <- lam0.cand*exp(-(D*D)/(2*sigma*sigma))
            p.cand <- 1-exp(-lam.cand)
            ll<- sum(dbinom(y, K, pmat *z, log=TRUE))
            llcand <- sum(dbinom(y, K, p.cand *z, log=TRUE))
            if(runif(1) < exp( llcand  - ll) ){
                ll<-llcand
                pmat<-p.cand
                lam0<- lam0.cand
            }
        }
\end{verbatim}
}
%\end{minipage}
\rule[-0.15in]{\textwidth}{.03in}
\caption{
MCMC updater for lam0 in a SCR model with Binomial encounter
process and cloglog link function on detection. Here, pmat =
1-exp(-lam).
}
\label{mcmc.panel.updatelam0}
\end{panel}


Another possibility is to model variation in the individual and site 
specific detection probability,  $p_{ij}$, directly, without any 
transformation, such that
\begin{comment} Rahel the $\leftarrow$ isn't right here but I couldn't
make this work \end{comment}
\[
p_{ij} \leftarrow p_0 * exp(-D_{ij}^2/(2\sigma^2))
\]
and $p_0 \in [0,1]$.
This formulation is analogous to how detection probability is modeled 
in distance sampling under a half-normal detection function; however, 
in distance sampling $p_0$ -- detection of an individual on the transect 
line -- is assumed to be 1 \citep{buckland_etal:2001}. Under this 
formulation the updater for $\lambda_0$ (equivalent to $p_0$ in Eq XX) 
becomes:
%I think I should rename it p0 in the code; this is a little confusing
\begin{verbatim}
  lam0.cand <- rnorm(1, lam0, 0.1)
  if(lam0.cand >0 & lam0.cand < 1 ){   
      #automatically rejects lam0.cand that are not {0,1}
       lam.cand <- lam0.cand*exp(-(D*D)/(2*sigma*sigma))
       ll<- sum(dbinom(y, K, lam *z, log=TRUE)) #no transformation needed
       llcand <- sum(dbinom(y, K, lam.cand *z, log=TRUE))
       if(runif(1) < exp( llcand  - ll) ){
          ll<-llcand
            lam<-lam.cand
            lam0<- lam0.cand
         }
     }
\end{verbatim}


\subsection{Looking at model output}
Now that you have an MCMC algorithm to analyze spatial capture-recapture 
data with, let's run an actual analysis so we can look at the output. As 
an example, we will use the Fort Drum 
bear data set we already analyzed in Chapt. \ref{chapt.closed} with 
traditional non-spatial models (and that you will see again in Chapt. 
\ref{XX}). You can load the Fort Drum data
(\mbox{\tt data(''beardata'') }), extract the 
trap locations (\mbox{\tt trapmat}) and 
detection data and build the augmented $M \times J$ array of individual 
encounter histories.
 In addition to these data, we need to specify 
the outermost coordinates of the state-space. Since bears are wide 
ranging animals we add a 20--km buffer to the maximum and minimum 
coordinates of the trap array:

\begin{verbatim}
xl<- min(trapmat[,2])- 20  
yl<- min(trapmat[,3])- 20 
xu<- max(trapmat[,2])+ 20
yu<- max(trapmat[,3])+ 20
\end{verbatim}

Finally, source the MCMC code for the binomial encounter model algorithm 
with the cloglog link and run 5000 iterations. This should take 
approximately 25 minutes.
\begin{comment} Rahel: this is nice. you might make Scr.0 an R func! 
\end{comment}
\begin{verbatim}
 source('SCR0binom.txt')
 mod0<-SCR.0(y=Xaug, X=trapmat[,2:3], M=M, xl=xl, xu=xu, yl=yl, 
                   yu=yu, K=8, niter=5000)
\end{verbatim}

Before, we used simple {\bf R} commands to look at model results. 
However, there is a specific {\bf R} package to summarize MCMC 
simulation output and perform some convergence diagnostics -- package 
coda \citep{plummer_etal:2006}. Download and install coda, then 
convert your model output to an mcmc object
\begin{verbatim}
  chain<-mcmc(mod0)
\end{verbatim} 
which can be used by coda to produce MCMC specific output.

\subsubsection{Markov chain time series plots}

Start by looking at time series plots of your Markov chains using 
\verb#plot(chain)#. This command produces a time series plot and
 marginal posterior density plots for each monitored parameter, 
 similar to what we did before using the \verb#hist()# and \verb#plot()# 
 commands (Fig. \ref{mcmc.fig.timeseries}). Time series plots will tell 
 you several things:
First, recall from Sect. XXXXXX that the way the chains move 
through the parameter space gives you an idea of whether your MH 
steps are well tuned. If chains were constant over many iterations 
you would need to decrease the tuning parameter of the (Normal) 
proposal distribution. If a chain moves along some gradient to a 
stationary state very slowly, you may want to increase the tuning 
parameter so that the parameter space is explored more efficiently.


\begin{figure}
\begin{center}
\includegraphics[height=2.5in]{Ch7/figs/timeseries}
\end{center}
\caption{Time series and posterior density plots for $\sigma$ and $\lambda_0$.}
\label{mcmc.fig.timeseries}
\end{figure}


Second, you will be able to see if your chains converged and how many initial simulations you have to discard as burn-in. In the case of the chains shown in Fig. \ref{timeseries.fig}, we would probably consider the first 750 - 1000 iterations as burn-in, as afterwards the chains seem to be fairly stationary.

\subsection{Posterior density plots}
The \verb#plot()# command also produces posterior density plots and it is worthwhile to look at those carefully. For parameters with priors that have bounds (e.g. Uniform over some interval), you will be able to see if your choice of the prior is truncating the posterior distribution. In the context of SCR models, this will mostly involve our choice of $M$, the size of the augmented data set. If the posterior of $N$ has a lot of mass concentrated close to $M$ (or equivalently the posterior of $\psi$ has a lot of mass concentrated close to 1), as in the example in Fig. \ref{timeseries2.fig}, we have to re-run the analysis with a larger $M$.  A diffuse
posterior plot suggests
that the parameter may not be well-identified. 
There may not be enough information in your data to estimate model parameters and you may have to consider a simpler model. Finally, posterior density plots will show you if the posterior distribution is symmetrical or skewed -- if the distribution has a heavy tail, using the mean as a point estimate of your parameter of interest may be biased and you may want to opt for the median or mode instead.

\begin{figure}
\begin{center}
\includegraphics[height=2.5in]{Ch7/figs/timeseries2}
\end{center}
\caption{Time series and posterior density plots of $\psi$ and $N$ for the bear data set truncated by the upper limit of $M$ (500).}
\label{timeseries2.fig}
\end{figure}

\subsection{Serial autocorrelation and effective sample size}

Checking the degree of autocorrelation in your Markov chains and 
estimating the effective sample size your chain has generated should 
be part of evaluating your model output. If you use {\bf WinBUGS}
 through the \mbox{\tt R2WinBUGS} package, the \verb#print()# command 
 will automatically return the effective sample size for all monitored 
 parameters. In the coda package there are several functions you can use 
 to do so. \verb#effectiveSize()# will directly give you an estimate 
 of the effective sample size for you parameters:
\begin{verbatim}
> effectiveSize(chain)
    sigma      lam0       psi         N
 3.930303 78.259159 30.436348 32.047392
\end{verbatim}

Alternatively, you can use the \verb#autocorr.diag()# function, which will show you the degree of autocorrelation for different lag values (which you can specify within the function call, we use the defaults below):
\begin{verbatim}
> autocorr.diag(mcmc(mod))
           sigma      lam0       psi         N
Lag 0  1.0000000 1.0000000 1.0000000 1.0000000
Lag 1  0.9979948 0.9494134 0.9847503 0.9774201
Lag 5  0.9915567 0.8038168 0.9111951 0.9113525
Lag 10 0.9836016 0.6714021 0.8462108 0.8509803
Lag 50 0.8985337 0.1983780 0.6138516 0.6233994
\end{verbatim}
In the present case we see that autocorrelation is especially high for the 
parameter $\sigma$ and our effective sample size for this parameter is 
4! \footnote{Anyone have any idea how the autocorrelation in sigma could 
be reduced? XXXXXXXXXX YES: Mess with the MH tuning parameter......XXXXXXXX} 
This means we would have to run the model for much longer to 
obtain a reasonable effective sample size. Unfortunately, with many SCR models we observe high degrees of serial autocorrelation. For now, let's continue using this small set of samples to continue looking at the output.


\subsection{Summary results}
Now that we checked that our chains apparently have converged and pretending 
that we have generated enough samples from the posterior distribution, we 
can look at the actual parameter estimates. The \verb#summary()# function 
will return two sets of results: the mean parameter estimates, with their standard deviation, the naïve standard error -- i.e. your regular standard error calculated for $T$ (= number of iterations) 
samples without 
accounting for serial autocorrelation -- and the 
Time-series SE (in {\bf WinBUGS} 
and earlier in this book referred to as MC error), which accounts for 
autocorrelation. Remember our rule of thumb that this error 
decreases with increasing chain length and should be 1\% or less of the 
parameter estimate. In {\bf WinBUGS} the MC error is only given in the log 
output within {\bf BUGS} itself.
You should adjust the \verb#summary()# call by removing the burn-in from
calculating parameter summary statistics. To do so, use the \verb#window()#
command, which lets you specify at which iteration to start
'counting'. In contrast to {\bf WinBUGS}, which requires you to set the
burn-in length before you run the model, this command gives us full
flexibility to make decisions about the burn-in after we have seen the
trajectories of our Markov chains. For our example,
\verb#summary(window(chain, start=1001))# returns the following output:


\begin{verbatim}
Iterations = 1001:5000
Thinning interval = 1
Number of chains = 1
Sample size per chain = 4000

1. Empirical mean and standard deviation for each variable,
   plus standard error of the mean:

          Mean       SD  Naive SE Time-series SE
sigma   1.9986  0.13805 0.0021827       0.016091
lam0    0.1096  0.01523 0.0002407       0.001401
psi     0.6113  0.09148 0.0014465       0.010734
N     489.8535 71.79695 1.1352094       8.431119

2. Quantiles for each variable:

           2.5%       25%      50%      75%    97.5%
sigma   1.75780   1.89847   1.9900   2.0944   2.2772
lam0    0.08357   0.09824   0.1087   0.1192   0.1427
psi     0.45110   0.54838   0.6052   0.6639   0.8192
N     366.00000 440.00000 485.0000 530.0000 654.0000
\end{verbatim}

Looking at the MC errors (column labeled \mbox{\tt Time-series SE}), 
we see that in spite of the high autocorrelation, the MC error for 
$\sigma$ is below the 1\% threshold, whereas for all other parameters, 
MC errors are still above, another indication that for a thorough 
analysis we should run a longer chain.
Our algorithm gives us a posterior distribution of $N$, but we are usually 
interested in the density, $D$. Density itself is not a parameter of our 
model, but we can derive a posterior distribution for $D$ by dividing 
each value of $N$ ($N$ at each iteration) by the area of the state-space
 (here 3032.719 km$^2$) and we can use summary statistics of the 
 resulting distribution to characterize $D$:
\begin{verbatim}
> summary(window(chain[,4]/ 3032.719, start=1001))
Iterations = 1001:5000
Thinning interval = 1
Number of chains = 1
Sample size per chain = 4000

1. Empirical mean and standard deviation for each variable,
   plus standard error of the mean:

          Mean             SD       Naive SE Time-series SE
     0.1615229      0.0236741      0.0003743      0.0027801

2. Quantiles for each variable:

  2.5%    25%    50%    75%  97.5%
0.1207 0.1451 0.1599 0.1748 0.2156
\end{verbatim}
We see that our mean density of $0.16/km^2$ is very similar to the estimate of $0.18/km^2$ obtained under the non-spatial model M0 in Chapt. \ref{chapt.closed}.


\subsection{Other useful commands }
While inspecting the time series plot gives you a first idea of how well you tuned your MH algorithm, use \verb#rejectionRate()# to obtain the rejection rates (1 -- acceptance rates) of the parameters that are written to your output:
\begin{verbatim}
> rejectionRate(chain)
     sigma       lam0        psi          N
0.44108822 0.77675535 0.00000000 0.01940388
\end{verbatim}
 Recall (section XXXXXX?) that rejection rates should lie between 0.2 and 0.8, so our tuning seems to have been appropriate here. $\psi$ is never rejected since we update it with Gibbs sampling, where all candidate values are kept. And since $N$ is the sum of all $z_i$, all it takes for $N$ to change from one iteration to the next are small changes in the z-vector, so the rejection rate of $N$ is always low.
If you have run several parallel chains, you can combine them into a single mcmc object using the \verb#mcmc.list()# command on the individual chains (note that each chain has to be converted to an mcmc object before combining them with \verb#mcmc.list()#). You can then easily obtain the Gelman-Rubin diagnostic \citep{gelman_etal:2004}, in {\bf WinBUGS} called R-hat, using \verb#gelman.diag()#, which 
will indicate if all chains have converged to the same stationary distribution.
For details on these and other functions, see the \mbox{\tt coda} manual, 
which can be found (together with the package) on the CRAN mirror.

\section{Manipulating the state-space}

So far, we have constrained the location of the activity centers to fall
within the outermost coordinates of our rectangular state space by posing 
upper and lower bounds for $x$ and $y$. But what if ${\cal S}$ 
has an irregular 
shape -- maybe there is a large water body we would like to remove from 
${\cal S}$, because we know our terrestrial study species does not occur there.
Or the study takes place in a clearly defined area such as an island. 
As mentioned before, this situation is difficult to handle in {\bf WinBUGS}.
In some simple cases we can adjust the state space by setting one of the
coordinates of ${\bf s}_{i}$ to be some function of the other. 
In this manner, we can cut off corners of the rectangle to approximate 
the actual state space. In {\bf R}, we are much more flexible, as we can 
use the actual state-space polygon to constrain out ${\bf s}_i$. 
\footnote{ Have to check if we can use panther stuff for the book; 
otherwise, use raccoon example.} To illustrate that, let's look at a camera 
trapping study of Florida panthers (\emph{Puma concolor coryi}) conducted 
in the Picayune Strand Restoration Project (PSRP) area, southwest Florida 
(Fig. \ref{pantercamera.fig}), by XXX, and financed by XXX. In the 1960ies 
the PSRP area was slated for housing development, but then bought back by 
the State of Florida and is currently being restored to its original 
hydrology and vegetation. In an effort to estimate the density of the 
local Florida panther population, 98 camera traps were operated in the area 
for 21 months between 2005 and 2007. Florida panthers are wide-ranging 
animals and in order to account for their wide movements, the state-space 
was defined as the trapping grid buffered by 15 km around its outermost 
coordinates. However, the resulting rectangle contained some ocean in its 
southwestern corner (Fig. \ref{pantercamera.fig}).
In order to precisely describe the state-space, the ocean has to be 
removed. You can create a precise state-space polygon in {\bf ArcGIS} and 
read it into {\bf R}, or create the polygon directly within {\bf R}. In 
the present case we intersected two shape files -- one of the state of 
Florida and one of the rectangle defined by a strip of 15 km around the
 camera-trapping grid.
While you will most likely have to obtain the shapefile describing the 
landscape of and around your trapping grid (coastlines, water bodies etc.) 
from some external source, a polygon shapefile buffering your outermost 
trapping grid coordinates can easily be written in {\bf R}.

If \mbox{\tt xmin}, \mbox{\tt xmax}, \mbox{\tt ymin} and 
\mbox{\tt ymax}, mark the most extreme
$x$ and $y$ coordinates of your 
trapping grid and $b$ is the distance you want to buffer with, load the 
package \mbox{\tt shapefiles} \citep{stabler:2006} and issue the following
{\bf R} commands:
\begin{verbatim}
xl= xmin-b
xu= xmax+b
yl= ymin-b
yu= ymax+b

            #create data frame with coordinate pairs
dd <- data.frame(Id=c(1,1,1,1,1),X=c(xl,xu,xu,xl,xl), Y=c(yl,yl,yu,yu,yl)) 
ddTable <- data.frame(Id=c(1),Name=c("Item1"))
            #convert to shapefile, type polygon
ddShapefile <- convert.to.shapefile(dd, ddTable, "Id", 5) 
            # save to location of choice
write.shapefile(ddShapefile, 'c:/…’, arcgis=T) 
\end{verbatim}


\begin{figure}
\begin{center}
\includegraphics[height=2.5in]{Ch7/figs/panthercamera}
\end{center}
\caption{Rectangular state-space for a Florida panther camera trapping
study in the PSRP area (grey outline, red block inset map of Florida)
contain some ocean (white) that needs to be removed from the state-space.}
\label{mcmc.fig.pantercamera}
\end{figure}

You can read shapefiles into {\bf R} loading the package \mbox{\tt 
maptools}
\citep{lewin-koh_etal:2011} and using the function
\verb#readShapeSpatial()#. Make sure you read in shapefiles in UTM format, so
that units of the trap array, the movement parameter sigma and the
state-space are all identical.  Intersection of polygons can be done
in {\bf R} also, using the package \mbox{\tt rgeos} 
\citep{bivand_rundel:2011} and the
function \verb#gIntersect()#. The area of your (single) polygon can be
extracted directly from the state-space object \mbox{\tt SSp}:

\begin{verbatim}
 area <- SSp@polygons[[1]]@Polygons[[1]]@area /1000000
\end{verbatim}

 Note that dividing by 1000000 will return the area in km$^2$ if your coordinates describing the polygon are in UTM. If your state-space consists of several disjunct polygons, you will have to sum the areas of all polygons to obtain the size of the state-space.
To include this polygon into our MCMC sampler we need one last spatial 
{\bf R} package, \mbox{\tt sp} \citep{pebesma_bivand:2011}, which has a 
function, \verb#over()#, which allows us to check if a pair of coordinates 
falls within a polygon or not. All we have to do is embed this new check 
into the updating steps for ${\bf s}$:
\begin{verbatim}
    #draw candidate value
Scand <- as.matrix(cbind(rnorm(M, S[,1], 2), rnorm(M, S[,2], 2)))
     #convert to spatial points on UTM (m) scale
Scoord<-SpatialPoints(Scand*1000)   
     # check if scand is within the polygon
SinPoly<-over(Scoord,SSp)		

for(i in 1:M) {
    #if scand falls within polygon, continue update
   if(is.na(SinPoly[i])==FALSE) {		
… [rest of the updating step remains the same]
\end{verbatim}
Note that it is much more time-efficient to draw all $M$ candidate values 
for $s$ and check once if they fall within the state-space, rather than 
running the \verb#over()# command for every individual pair of 
coordinates. To make sure that our initial values for {\bf s} also fall 
within the polygon of ${\cal S}$, we use the function \verb#runifpoint()# 
from the package \mbox{\tt spatstat} \citep{baddeley_turner:2005}, 
which generates random uniform points within a specified polygon. You'll 
find this modified MCMC algorithm (\mbox{\tt SCR0poisSSp}) in the {\bf R} 
package \mbox{\tt scrbook}.
Finally, observe that we are converting candidate coordinates of ${\cal S}$ 
back to meters to match the UTM polygon. In all previous examples, 
for both the trap locations and the activity centers we have used UTM 
coordinates divided by 1000 to estimate $\sigma$ on a km scale. This is 
adequate for wide ranging individuals like bears. In other cases you 
may center all coordinates on 0. No matter what kind of transformation you 
use on your coordinates , make sure to always convert candidate values for 
${\cal S}$ back to the original scale (UTM) before running the 
\verb#over()# command.

\section{MCMC software packages}

Throughout most of this book we will use {\bf WinBUGS} and, occasionally, {\bf JAGS} to run MCMC analyses. 
Here, we will briefly discuss the main pros and cons of these two programs 
as well as {\bf WinBUGS} successor {\bf OpenBUGS}. 

\subsection{WinBUGS}

In a nutshell, {\bf WinBUGS} (and the other programs) do everything that we 
just went through in this chapter (and quite a bit more). Looking through 
your model, {\bf WinBUGS} determines which parameters it can use standard 
Gibbs sampling for (i.e. for conjugate full conditional distributions). 
Then, it determines, in the following hierarchy, whether to use adaptive 
rejection sampling, slice sampling or -- in the 'worst' case -- 
Metropolis-Hastings sampling for the other full conditionals 
\citep{spiegelhalter_etal:2003}. If it uses MH sampling, it will 
automatically tune the updater so that it works efficiently.
While {\bf WinBUGS} is a convenient piece of software that is still 
widely used, its major drawback is that it is no longer being developed, 
i.e. no new functions or distributions are added and no bugs are fixed.

\subsection{OpenBUGS}
{\bf OpenBUGS} is essentially the successor of {\bf WinBUGS}. While the 
latter is
no longer actively developed, {\bf OpenBUGS} continues to be 
developed. The
name '{\bf OpenBUGS}' refers to the software being open source, so users 
do
not need to download a license key, like they have to for {\bf WinBUGS}
(although the license key for {\bf WinBUGS} is free and valid for life).

Compared to {\bf WinBUGS}, {\bf OpenBUGS} 
has  more built-in functions. The
method of how to determine the right updater for each model parameter
has changed and the user can manually control the MCMC algorithm used
to update model parameters.  Several other changes have been
implemented in {\bf OpenBUGS} and a detailed list of differences between the
two {\bf BUGS} versions, can be found at
\url{http://www.openbugs.info/w/OpenVsWin}.

While {\bf OpenBUGS} is a useful program for a lot of MCMC sampling
applications, for reasons we do not understand, simple SCR models do
not converge sometimes in {\bf OpenBUGS}. It is therefore advisable that 
you check any
{\bf OpenBUGS} SCR model results against result from {\bf WinBUGS}. Also,
currently, the {\bf R} package \mbox{\tt BRugs} \citep{thomas_etal:2006}, 
necessary
for running {\bf OpenBUGS} through {\bf R}, has problems with 64-bit 
machines, so
you may have to use the 32-bit version of {\bf R} and {\bf OpenBUGS} 
in order to
make it work. The {\bf BUGS} project site at 
\url{http://www.openbugs.info}
provides a lot of information on and download links for {\bf OpenBUGS}.

There is an extensive help archive for both {\bf WinBUGS} and {\bf OpenBUGS}
 and you can subscribe to a mailing list, where people pose and answer 
 questions of how to use these programs at 
 \url{http://www.mrc-bsu.cam.ac.uk/bugs/overview/list.shtml}

\subsection{JAGS -- Just Another Gibbs Sampler}

{\bf JAGS}, currently at Version 3.1.0, is another free program for analysis 
of Bayesian hierarchical models using MCMC simulation. Originally, {\bf JAGS}
 was the only program using the {\bf BUGS} language that would run on 
 operating systems other than the 32 bit Windows platforms. By now, there 
 are {\bf OpenBUGS} versions for Linux or Macintosh machines.
{\bf JAGS} 'only' generates samples from the posterior distribution; 
analysis of the output is done in {\bf R}, either by running {\bf JAGS} 
through {\bf R} using either the packages \mbox{\tt rjags} 
\citep{plummer:2011} or \mbox{\tt R2jags} \citep{su_yajima:2011}, or by 
using coda on your {\bf JAGS} output. The program, manuals and \mbox{\tt rjags} 
can be downloaded at \url{http://sourceforge.net/projects/mcmc-jags/files/}
When run from within {\bf R} using the package \mbox{\tt rjags] or \mbox{R2jags}, 
writing a \mbox{\bf JAGS} model is virtually identical to writing a {\bf WinBUGS}
 model. However, some functions may have slightly different names and you 
 can look up available functions and their use in the {\bf JAGS} 
 manual. One potential downside is that {\bf JAGS} can be very particular 
 when it comes to initial values. These may have to be set as close to 
 truth as possible for the model to start. Although {\bf JAGS} lets 
 you run several parallel Markov chains, this characteristic interferes 
 with the idea of using overdispersed initial values for the different 
 chains. Also, we have occasionally experienced {\bf JAGS} to crash and 
 take the {\bf R} GUI with it. Only re-installing both {\bf JAGS} and 
 {\tt rjags} seemed to solve this problem.
On the plus side, {\bf JAGS} usually runs a little faster than {\bf WinBUGS},
 sometimes considerably faster (see Sect. \ref {4.XYZ}), is constantly 
 being developed and improved and it has a variety of functions that are 
 not available in {\bf WinBUGS}. For example, {\bf JAGS} allows you to 
 supply observed data for some deterministic functions of unobserved 
 variables. In {\bf BUGS} we cannot supply data to logical nodes. 
 Another useful feature is that the adaptive phase of the model 
 (the burn-in) is run separately from the sampling from the stationary 
 Markov chains. This allows you to easily add more iterations to the 
 adaptive phase if necessary without the need to start from 0. There 
 are other, more subtle differences and there is an entire manual section 
 on differences between {\bf JAGS} and {\bf OpenBUGS}.
For questions and problems there is a {\bf JAGS} forum online at 
\url{http://sourceforge.net/projects/mcmc-jags/forums/forum/610037}.
\footnote{As we make progress on the book, lets be sure  to add 
linkages to places where we use JAGS in examples.}

\section{Summary and Outlook}

Although there are a number of flexible and extremely useful software 
packages to perform MCMC simulations, it sometimes is more efficient to 
develop your own MCMC algorithm. Building an MCMC code follows three basic 
steps: Identify your model including priors and express full conditional 
distributions for each model parameter. If full conditionals are parametric 
distributions, use Gibbs sampling to draw candidate parameter values from 
those distributions; otherwise use Metropolis-Hastings sampling to draw 
candidate values from a proposal distribution and accept or reject them 
based on their posterior probability densities.
These custom-made MCMC algorithms give you more modeling flexibility than 
existing software packages, especially when it comes to handling the
 state-space: In {\bf BUGS} (and {\bf JAGS} for that matter) we define
  a continuous rectangular state-space using the corner coordinates to 
  constrain the Uniform priors on the activity centers ${\bf s}$.
   But what if a continuous rectangle isn't an adequate description of 
   the state-space? In this chapter we saw that in {\bf R} it only takes 
   a few lines of code to use any arbitrary polygon shapefile as the 
   state-space, which is especially useful when you are dealing with 
   coastlines or large bodies of water that need removing from the 
   state-space. Another example is the SCR {\bf R} package \mbox{\tt SPACECAP}
    \citep{gopalaswamy_etal:2011} that was developed because implementation
     of an SCR model with a discrete state-space was inefficient in {\bf WinBUGS}.
Another situations in which using {\bf BUGS}/{\bf JAGS} becomes
increasingly
complicated or inefficient is when using point processes other 
than the 
 Binomial point process (''uniformity'') which underlies the basic 
 SCR model (see Chapt. \ref {Chapter X}). In the Chapt. 
 \ref {Chapter X} and XX you will see examples of different point processes,
  implemented using custom-made MCMC algorithms.
   \footnote{Richard, Beth expand on that?}
Finally, the Chapt. \ref {Chapter X} and XX deal with unmarked or 
partially marked populations using hand-made MCMC algorithms to 
handle the (partially) latent individual encounter histories. 
While some of these models can be written in {\bf BUGS}/{\bf JAGS}, 
\footnote{the Poisson one for partially marked we wrote in BUGS and it 
should work with a known number of marked; the Bernoulli in JAGS with the 
dsum() function should work for the fully unknown; maybe some others?
 I don’t remember. We may have to try writing the others before saying 
 that they don’t work in {\bf BUGS}/{\bf JAGS}; they are certainly much faster in {\bf R}, 
 though.}, they are painstakingly slow; others cannot be implemented in 
 {\bf BUGS}/{\bf JAGS} at all (e.g., the classes of models
 considered in Chapts. \ref{chapt.ecoldist} and  \ref{chapt.state-space}).
In conclusion, while you can certainly get by using {\bf BUGS}/{\bf JAGS} 
for standard SCR models, knowing how to write your own MCMC sampler 
allows you to tailor these models to your specific needs.



\chapter{Goodness of Fit and stuff}
\label{chapt.gof}

\chapter{Modeling Encounter Probability}
\label{chapt.covariates}

\chapter{
%Modeling Animal space-usage with
%Detection Models based on Ecological Distance
%Ecological Distance Models in Spatial Capture-Recapture
Modeling Space Usage: Ecological Distance in Spatial Capture-Recapture Models
}
\markboth{Chapter XXX}{}
\label{chapt.ecoldist}


\vspace{.3in}

\begin{comment} % RBC commented this out as suggested by Rahel

%% this material is a general introduction for a manuscript
%Spatial capture-recapture models are a relatively new class of models
%for estimating animal density from capture-recapture data with
%auxiliary information about individual capture locations
%\citep{efford:2004,borchers_efford:2008, royle_young:2008, efford_etal:2009ecol,
%  royle_etal:2009ecol}.
Spatial capture-recapture models
express encounter probability
as a function of the distance between an individual's activity center,
say ${\bf s}_{i}$, and trap location, say ${\bf x}_{j}$.
In these models ${\bf s}_{i}$ is regarded as a latent variable and
conventional methods of statistical inference either based on marginal
likelihood \citep{borchers_efford:2008} or Bayesian analysis by MCMC
\citep{royle_young:2008}.

While the models are a relatively recent innovation, their use is
already becoming widespread \citep{efford_etal:2009ecol,
  gardner_etal:2010jwm, gardner_etal:2010ecol,kery_etal:2010,
  borchers:2011,gopalaswamy_etal:2012, foster_harmsen:2012} because they resolve
critical problems with using ordinary non-spatial capture-recapture
methods such as ill-defined area sampled, and heterogeneity in
encounter probability due to the juxtaposition of individuals with
traps, and they provide a framework for modeling of trap-specific
covariates.  Furthermore, essentially all real capture-recapture
studies produce auxiliary spatial information and therefore SCR models
are directly relevant to standard data collected from such studies.
% Indeed, the use of ordinary
%capture-recapture models essentially admits a model misspecification
%(i.e. homogeneous encounter probability) by ignoring the explicit
%spatial information.

XXX MAYBE YOU COULD START THE CHAPTER AT THIS POINT; THE OTHER STUFF HAS BEEN COVERED BY THE PREVIOUS BOOK CHAPTERS XXXXX
\end{comment}

Every spatial capture-recapture model that we have considered so far
has expressed encounter probability as function of the Euclidean
distance between individual activity
centers $\bf s$ and trap locations $\bf x$. While these simple encounter
probability models will often
be sufficient for practical
purposes, especially in small data sets, sometimes developing more
complex models of the detection process as it relates to space usage
of individuals will be useful.  Animals may not judge distance in
terms of Euclidean distance but, rather, according to quality of local
habitat, landscape connectivity, perceived mortality risk, and other
considerations affecting movement behavior.
\begin{comment}
As an example of the potential problem of parameterizing SCR models
using Euclidean distance, imagine a study area bisected by a large
semi-permeable barrier. In standard SCR models, the probability of
capturing an animal in a trap located on the opposite side of the
barrier would simply be a function of distance, whereas in reality it
should be a function of both distance and the permeability of the
barrier.
Such situations are extremely common in capture-recapture
studies where multiple habitats occur in the study area or when
animals use linear features such as trails, corridors, or rivers.
\end{comment}
Moreover, because encounter probability and the distance
metric upon which it is based represent outcomes of individual
movements about their home range, ecologists might have explicit
hypotheses about how environmental variables affect the distance
metric, and it is therefore desirable to incorporate these hypotheses
directly into SCR models so that they may be formally evaluated
statistically.

In this chapter we develop a framework for modeling animal space usage
in SCR models, by parameterizing models for encounter probability
based on ``ecological distance''.  In particular, following
\citet{royle_etal:2012ecol}, we adopt a cost-weighted distance metric
(the least-cost path) used widely in landscape ecology for modeling
connectivity, movement and gene flow
\citep{adriaensen_etal:2003,manel_etal:2003,mcrae_etal:2008}. In the
context of SCR models we can use this as the basis for computing the
distance between traps and individuals activity centers. In this way
we can explicitly accommodate landscape structure and account for how
animals use space in SCR models. We develop a likelihood-based
inference framework for SCR model parameters using this new distance
metric when the ecological distance function is known.  We show that
the MLEs are approximately unbiased in moderate sample sizes, as
expected, but also that the misspecified model based on Euclidean
distance can produce substantial bias in estimates of $N$ and hence
density.  Further, we extend the model to allow for likelihood
estimation of parameters of the cost function.

Using this methodological extension of SCR models, it is possible to
make formal statistical inferences
about movement and connectivity from
capture-recapture studies that generate sparse individual encounter
history data without subjective prescription of resistance
or cost surfaces.


\section{Distance Models}


In the standard SCR model we model encounter probability as a function
of Euclidean distance. For example, using the binomial observation model
as an example (Chapt. \ref{chapt.scr0}), let
$y_{ij}$ be individual- and trap specific binomial counts
with sample size $K$ and probabilities
$p_{ij}$. The Gaussian or ``half-normal'' model is \footnote{Note the
  parameter labeling is not consistent with the rest of the book}
\[
log(p_{ij})= \theta_{0} + \theta_{1} dist({\bf x}_{j} - {\bf s}_{i})^{2}
\]
or, equivalently,
\[
p_{ij} = \lambda_{0} exp(-  dist({\bf x}_{j} - {\bf s}_{i})^{2}
/(2\sigma^{2}) )
\]
where $\theta_{0} = log(\lambda_{0})$ and $\theta_{1} =
-1/(2\sigma^2)$.

%In all previous applications of SCR models in this book, as well as in
%the literature,
The main problem with the normal Euclidean distance metric, i.e., 
$dist({\bf x}_{j} - {\bf s}_{i}) = ||{\bf x}_{j} - {\bf s}_{i}||$,
%and
%the parameters $\theta_0$ and $\theta_1$ have been estimated using
%standard methods (likelihood or Bayesian).  The main problem with this
%approach
is that it is unaffected by
habitat or landscape structure, and it implies that the space used by
individuals is stationary, and symmetric which may be unreasonable
assumptions for some species. By stationary here we mean in the formal
sense of
invariance to translation. That is, the properties of an individual
home range centered at some point ${\bf s}$ are exactly the same as
any other point say ${\bf s}'$.

As an example, if the common detection model based on a bivariate
normal probability distribution function is used, then the implied
space usage by {\it all} individuals, no matter their location in
space or local habitat conditions, is symmetric with circular contours
of usage intensity (density contours of the pdf).

\citet{royle_etal:2012ecol} extended this class of SCR models to
accommodate alternative distance metrics that explicitly incorporate
information about the landscape so that a unit of distance is variable
depending on identified covariates.  Thus, ``where'' an individual
lives on the landscape, and the state of the surrounding landscape,
will determine the character of its usage of space. In particular, they
suggest distance metrics that imply irregular, asymmetric and
non-stationary home ranges of individuals. As an example,
Fig. \ref{fig.distort} shows a typical symmetric home range, and an
comressed home range resulting from the effect of an environmental
variable on an animal's movement behavior.

\begin{figure}[h]
\centering
\includegraphics[width=5in,height=1.3in]{Ch10/figs/distort}
\caption{A symmetric home range (left), a habitat variable (center),
  and a non-symmetric home range (right) resulting from the cost imposed on
  movement by the habitat variable.}
\label{fig.distort}
\end{figure}


\section{Cost-Weighted Distance}

We adopt the use of a cost-weighted distance metric here which defines
the distance between points by accumulating pixel-specific costs
determined under a cost function defined by the user.  The idea of
cost-weighted distance to characterize animal use of landscapes is
widely used in landscape ecology for modeling connectivity, movement
and gene flow \citep{beier_etal:2008}. As is customary for reasons of
computational tractability we consider a discrete landscape
defined by
a raster of some prescribed resolution. The distance between any two
points ${\bf x}$ and ${\bf x}'$ can be represented by a sequence of
line segments connecting neighboring pixels say ${\bf l}_{1},{\bf
  l}_{2},\ldots,{\bf l}_{m}$. Then the cost-weighted distance between
${\bf x}$ and ${\bf x}'$ is

\begin{equation}
 d({\bf x},{\bf x}')
  =  \sum_{i=1}^{m-1} cost({\bf l}_{i},{\bf l}_{i+1})||{\bf l}_{i} - {\bf l}_{i+1}||
\label{eq.costweighted}
\end{equation}

{\flushleft
where } $cost({\bf l}_{i},{\bf l}_{i+1})$ is the user-defined cost function
to move
from pixel ${\bf l}_{i}$ to neighboring pixel ${\bf l}_{i}$ in the sequence.
Given the ``cost'' of each pixel, it is a simple matter to compute the
cost-weighted distance between any two pixels, along {\it any} path,
simply by accumulating the incremental  costs weighted by
distances.
In the context of
spatial capture-recapture models (and, more generally, landscape
connectivity) we are concerned with the {\it minimum} cost-weighted
distance, or the {\it least-cost path}, between any two points which
we will denote by $d_{lcp}$, which is
the
sequence ${\bf l}_{1},{\bf l}_{2},\ldots,{\bf l}_{m}$ that minimizes
the objective function defined by Eq. \ref{eq.costweighted}. That is,

\begin{equation}
 d_{lcp}({\bf x},{\bf x}')
  =  min_{{\bf l}_{1},\ldots,{\bf l}_{m}}  \sum_{i=1}^{m-1} cost({\bf l}_{i},{\bf l}_{i+1})||{\bf l}_{i} - {\bf l}_{i+1}||
\label{eq.lcp}
\end{equation}

{\flushleft
 Least-cost} path distance can be calculated in
 many geographic information systems and other software packages,
including the {\bf R} package \mbox{\tt
  gdistance} \citep{vanetten:2011}.

The key ecological aspect of least-cost path modeling is the
development
of models for pixel-specific cost.
In this paper we model cost as a function of one or more covariates
defined on every pixel of the according raster. For example, using a
single covariate $z(x)$ we define the cost of moving from some pixel
${\bf x}$ to neighboring pixel ${\bf x}'$ as
\begin{equation}
 log(cost({\bf x},{\bf x}'))=  \theta_{2} \frac{z({\bf x})+z({\bf x}')}{2}
\label{ecoldist.eq.cost}
\end{equation}
Thus, if $\theta_{2} = 0$ then substituting $cost({\bf x},{\bf x}')
=exp(0) = 1$ into
Eq. \ref{eq.lcp} will produce the ordinary Euclidean distance
between points. Here we assume the covariate $z$ is positive-valued
and constrain $\theta_{2}\ge 0$ so as to avoid
negative costs. While not necessarily problematic from a mathematical
standpoint, negative costs are unrealistic biologically. %unless there's a people mover....

In practical applications, variables that influence the cost of moving
across the landscape include things like highways
\citep[e.g.,][]{epps_etal:2005}, elevation \citep{cushman_etal:2006},
ruggedness \citep{epps_etal:2007}, snow cover
\citep{schwartz_etal:2009}, distance to escape terrain
\citep{shirk_etal:2010}, range limitations \citep{mcrae_beier:2007},
or distance from urban areas, highways, human disturbance or other
factors that animals might avoid.  Together multiple environmental
variables create a resistance surface, which forms the linchpin of all
connectivity planning \citep{spear_etal:2010}.  Often $\theta_{2}$ is
fixed by the investigator. Although $\theta_{2}$ is rarely known,
conservation biologists design linkages that require this resistance
value as input \citep[see][and articles cited
therein]{beier_etal:2008}.  Typically planners pick a value based on
expert opinion \citep{beier_etal:2008}, although recently researchers
have begun to define costs based on resource selection functions,
animal movement \citep{tracy:2006, fortin_etal:2005}, or genetic
distance data (e.g., \citet{gerlach_musolf:2000};
\citet{epps_etal:2007}; \citet{schwartz_etal:2009}.

To formalize the use of cost-weighted distance in SCR models, we
substitute Eq. \ref{eq.lcp} in the expression for encounter
probability (Eq. \ref{eq.encounter}) and maximize the resulting
likelihood which we address below. This allows us to formally model
these factors that influence space usage, and test explicit hypotheses
about these things using only individual level encounter history data
from capture-recapture studies.

\subsection{Example of Computing Cost-weighted distance}

As an example of the cost-weighted distance calculation consider the
following landscape comprised of 16 pixels with unit spacing
identified as follows, along with the pixel-specific cost:
\begin{center}
\begin{verbatim}
  pixel ID                 Cost
  1  5  9  13          100   1   1  1
  2  6 10  14          100 100   1  1
  3  7 11  15          100 100 100  1
  4  8 12  16          100 100   1  1
\end{verbatim}
\end{center}
This simple cost
raster is shown in Fig. \ref{ecoldist.fig.raster}. We assume the scale
is such that the distance between neighboring pixels in any cardinal
direction is 1 unit, and the distance between neighbors on a diagonal
is $\sqrt{2}$ units.
We assigned low cost of 1 to ``good habitat'' pixels (or pixels
we think of as ``highly connected'' by virtue of being in good
habitat) and, conversely, we assign high cost (100) to ``bad
habitat''. So the shortest cost-weighted distance between pixels 5 and
9 in this example is just 1 unit, the shortest cost-distance between
pixels 5 and 10 is $\sqrt{2}(1+1)/2 = 1.414214$ units, the shortest
distance between pixels 4 and 8 is 100 units, while the shortest
cost-distance between 4 and 12 is 150.5. A tough one is: what is the
shortest distance between 7 and 16? An individual at pixel 7 can move
diagonal (which has distance $\sqrt{2}$) and pay $sqrt(2)*(100+1)/2 + 1 =72.41778$.

\begin{figure}
\begin{center}
\includegraphics[height=3.25in,width=3.25in]{Ch10/figs/raster_2values}
\end{center}
\caption{A $4 \times 4$ raster with cost = 1 (white) or 100 (shaded) to represent ease of movement across a pixel.}
\label{ecoldist.fig.raster}
\end{figure}


Once the cost raster is created, the least-cost path distances are
computed with just a couple {\bf R} commands, and those can be
inserted directly into the likelihood construction for an ordinary
spatial capture-recapture model The {\bf R} package
\mbox{\tt gdistance} uses the implementation of Dijkstra's algorithm
\citep{dijkstra:1959} found in the \mbox{\tt igraph} package
\citep{csardi:2010}.  Using \mbox{\tt gdistance}, we 
define the incremental cost of moving from one pixel to another as the
distance-weighted {\it average} of the 2 pixel costs. We demonstrate
how to do this subsequently.

The {\bf R} commands for computing the least-cost distance between all pairs of pixels
are as follows:
\begin{verbatim}
r<-raster(nrows=4,ncols=4)
projection(r)<- "+proj=utm +zone=12 +datum=WGS84"
extent(r)<-c(.5,4.5,.5,4.5)
costs1<- c(100,100,100,100,1,100,100,100,1,1,100,1,1,1,1,1)
values(r)<-matrix(costs1,4,4,byrow=FALSE)
par(mfrow=c(1,1))
plot(r)
\end{verbatim}
Then we use the functions \mbox{\tt transition}, \mbox{\tt
  geoCorrection} (which is only necessary if the data are not
projected or if cells are considered to have more than 4 neighbors)
 and \mbox{\tt costDistance} to compute the distance
matrix. The transition function computes the cost of making a
transition between
any two pixels, and it operates on the inverse-scale (''conductance'')
and so the
\mbox{\tt transitionFunction} argument is given as $1/mean(x)$.
To compute the cost distance we prescribe a set of points, or  we
can compute it  between
two sets of points (which is handy when one of the sets is of trap
locations, and the other is of individual activity centers).
To compute the distances for pixels in a raster,
we use the center points of each raster.  The {\bf R}
 commands altogether are as follows:
{\small
\begin{verbatim}
tr1<-transition(r,transitionFunction=function(x) 1/mean(x),directions=8)
tr1CorrC<-geoCorrection(tr1,type="c",multpl=FALSE,scl=FALSE)
pts<-cbind( sort(rep(1:4,4)),rep(4:1,4))
costs1<-costDistance(tr1CorrC,pts)
outD<-as.matrix(costs1)
\end{verbatim}
}
Now we can look at the result and see if it makes sense to us. Here we
print the first 5 columns of this distance matrix to illustrate a
couple of examples of calculating the minimum cost-weighted distance
between points:
\begin{center}
{\small
\begin{verbatim}
> outD[1:5,1:5]
         1         2        3        4         5
1   0.0000 100.00000 200.0000 205.2426  50.50000
2 100.0000   0.00000 100.0000 200.0000  71.41778
3 200.0000 100.00000   0.0000 100.0000 171.41778
4 205.2426 200.00000 100.0000   0.0000 154.74264
5  50.5000  71.41778 171.4178 154.7426   0.00000
\end{verbatim}
}
\end{center}
An interesting case is that between point 1 and 4. Note that simply
taking the shortest Euclidean distance, weighted by cost, produces a
cost-weighted distance of $100 \times 1$ to move from pixel 1 to pixel
2, and similarly from 2 to 3 and 3 to 4, producing a total
cost-weighted distance of $300$. However, the actual {\it least-cost
  path} has cost-weighted distance $205.2426$.
The shortest path has an individual moving from pixel 1 to 5, then 5
to 10, 10 to 15, 15 to 12, 12 to 8 and 8 to 4 which should add up to
$205.2426$.

\section{Fitting Models of Space Usage by MLE}
\label{ecoldist.sec.mle}

Throughout much of this book we rely on Bayesian analysis by MCMC
mostly using
{\bf BUGS}, but sometimes (as in Chapt. \ref{chapt.mcmc}) developing
our own
implementations. However, occasionally we prefer to use likelihood
estimation, such as when
we can compare a set of models directly by likelihood either to do a
direct hypothesis test of a parameter, or to tabulate a bunch of AIC
values. It turns out, for this class
of models for space usage based on ecological distance, we actually
prefer likelihood methods
not because they have any conceptual or methodological benefit, but
simply because
they are more computationally efficient to implement
\citep{royle_etal:2012ecol}.
So here we adopt our formulation of maximum likelihood estimation
\citep{borchers_efford:2008}
from Chapt. \ref{chapt.mle}
for the class of models based on ecological distance. This is really
just a straightforward
adaption of that.

We continue to work here with the binomial model:
\[
	y_{ij}| {\bf s}_{i} \sim \mbox{Bin}(K, p_{\theta}(d_{lcp}({\bf x}_{j},{\bf s}_{i};\theta_{2}); \theta_{0}, \theta_{1})
\]
where we have indicated the dependence of $p_{ij}$ on the parameters
${\bm \theta}$, and also $d_{lcp}$ which
itself depends on $\theta_{2}$, and the latent variable ${\bf s}$.
%The parameters
%${\bm \theta}$ include whatever parameters are involved in the
%cost-weighted distance function, i.e., at least $\theta_{2}$ from
%Eq. \ref{eq.cost}.
For the random effect we have ${\bf s}_{i} \sim  \mbox{Unif}({\cal
  S})$. Recall that the state-space $\cal S$ is defined by the raster
data in this context.
The joint distribution of the data for individual $i$ is the product
of $J$ binomial terms (i.e., contributions from each of $J$ traps):

\[
  [{\bf y}_{i} | {\bf s}_{i} , \theta] =
  \prod_{j=1}^{J} \mbox{Bin}(K, p_{\theta}({\bf x}_{j},{\bf s}_{i}) )
\]

{\flushleft This} assumes that encounter of individual $i$ in each
trap is independent of encounter in every other trap. Conditional on
${\bf s}_{i}$ this is reasonable in most applications in our view.
 The so-called marginal likelihood is computed by removing
${\bf s}_{i}$, by integration,  from the conditional-on-${\bf s}$
likelihood and regarding the {\it marginal} distribution of the data
as the likelihood. That
is, we compute:

\[
  [y|{\bm \theta}] =
\int_{{\cal S}}  [ {\bf y}_{i} |{\bf s}_{i},{\bm \theta}] g({\bf s}_{i}) d{\bf s}_{i}
\]

{\flushleft where}, under the uniformity assumption, we have
$g({\bf s}) = 1/||{\cal S}||$.
The joint likelihood for all $N$ individuals, assuming independence of
encounters among individuals, is the product of $N$ such terms:

\[
{\cal L}({\bm \theta} | {\bf y}_{1},{\bf y}_{2},\ldots, {\bf y}_{N}) = \prod_{i=1}^{N}
[{\bf y}_{i}|{\bm \theta}]
\]

The key operation for computing the likelihood is solving the
2-dimensional integration problem to remove ${\bf s}$, which we
resolve as we did previously in Chapt. \ref{chapt.mle}, using the
rectangular rule for integration, and averaging the integrand over a
fine mesh of points.
Therefore,
the marginal pmf of ${\bf y}_{i}$, is
approximated by
\begin{equation}
         [{\bf y}_{i}|\theta] = \frac{1}{nG} \sum_{u=1}^{nG}  [ {\bf
            y}_{i} |{\bf s}_u, \theta]
\label{mle.eq.intlik}
\end{equation}
To deal with the fact that $N$ is unknown, there are two key issues
that need to be addressed.  First is that we don't observe the
``all-zero'' encounter histories (i.e., $y_{ij} = 0$ for all $j$)
corresponding to uncaptured individuals, so we have to make sure we
compute the probability for that all zero encounter history which we
do operationally by tacking a row of zeros onto the encounter history
matrix. We include the number of such all-zero encounter histories as
an unknown parameter of the model, which we label $n_{0}$.  In
addition, we have to be sure to include a combinatorial term to
account for the fact that of the $n$ observed individuals there are
${N \choose n}$ ways to realize a sample of size $n$. The
combinatorial term involves the unknown $n_{0}$ and thus it must be
included in the likelihood.

We wrote an {\bf R} function to evaluate the likelihood which we optimize
using the {\bf R} function \mbox{\tt nlm}.
The likelihood is given in the {\tt scrbook} package as the function
\mbox{\tt intlik3ed}. The help file
provides an example of its usage and for simulating data.

To use this function the cost covariate $z(x)$ has to be of class
\mbox{\tt RasterLayer} which requires packages \mbox{\tt sp} and
\mbox{\tt raster} to manipulate.
The following is a stylized and more concise verstion of the actual
function, and we apply this in the following section.

{\small
\begin{verbatim}
intlik3ed<-function(start=NULL,y=y,K=NULL,X=traplocs,
distmet="ecol",covariate,theta2=NA){

nc<-covariate@ncols
nr<-covariate@nrows
Xl<-covariate@extent@xmin
Xu<-covariate@extent@xmax
Yl<-covariate@extent@ymin
Yu<-covariate@extent@ymax
### ASSUMES SQUARE RASTER -- NEED TO GENERALIZE THIS
delta<- (Xu-Xl)/nc
xg<-seq(Xl+delta/2,Xu-delta/2,delta)
yg<-seq(Yl+delta/2,Yu-delta/2,delta)
npix.x<-length(xg)
npix.y<-length(yg)
area<- (Xu-Xl)*(Yu-Yl)/((npix.x)*(npix.y))
G<-cbind(rep(xg,npix.y),sort(rep(yg,npix.x)))
nG<-nrow(G)

if(distmet=="euclid")
D<- e2dist(X,G)
if(distmet=="ecol"){
if(is.na(theta2))
theta2<-exp(start[4])
cost<- exp(theta2*covariate)
tr1<-transition(cost,transitionFunction=function(x) 1/mean(x),directions=8)
tr1CorrC<-geoCorrection(tr1,type="c",multpl=FALSE,scl=FALSE)
D<-costDistance(tr1CorrC,X,G)
}

theta0<-start[1]; theta1<-start[2]; n0<-exp(start[3])

probcap<- (exp(theta0)/(1+exp(theta0)))*exp(-theta1*D*D)
Pm<-matrix(NA,nrow=nrow(probcap),ncol=ncol(probcap))
ymat<-y ; ymat<-rbind(y,rep(0,ncol(y)))
lik.marg<-rep(NA,nrow(ymat))
for(i in 1:nrow(ymat)){
Pm[1:length(Pm)]<- (dbinom(rep(ymat[i,],nG),rep(K,nG),probcap[1:length(Pm)],log=TRUE))
lik.cond<- exp(colSums(Pm))
lik.marg[i]<- sum( lik.cond*(1/nG) )
}
nv<-c(rep(1,length(lik.marg)-1),n0)
part1<- lgamma(nrow(y)+n0+1) - lgamma(n0+1)
part2<- sum(nv*log(lik.marg))
out<-  -1*(part1+ part2)
out
}
\end{verbatim}
}

\subsection{Bayesian Analysis}

While implementation of these ecological distance SCR models is reasonably straightforward, it is difficult to fit them using the {\bf BUGS} engines
because it is not possible, to the best of our knowledge, to compute
the least-cost path distance.  It would be possible to fit the models
in {\bf BUGS} if the parameter $\theta_{2}$ was fixed. In that case,
one could compute the distance matrix ahead of time and reference the
required elements for a given ${\bf s}$.
Alternatively, it would be possible to write a custom MCMC routine
using the methods we present in Chapt. \ref{chapt.mcmc}, although we
have not yet developed our own implementation.




\section{Example: SCR model based on ecological distance}

In this section we provide examples that we think are typical of how
cost-weighted distance models can be used in real capture-recapture
problems.  We define a $20 \times 20$ pixel covariate raster with
extent = $[0.5, 4.5] \times [0.5, 4.5]$.  We regard this, for the
purposes of our example, as a coarse landscape covariate, with pixels
having some arbitrary scaling say, a $2 \times 2$ km resolution. Thus,
the raster defines a landscape of $40 \times 40$ km and we suppose
that 16 camera traps are established at the integer coordinates
$(1,1), (1,2), \ldots, (4,4)$. We could think of this as a landscape
within which we're studying a population of ocelots, lynx or some
other cat.

For our analyses, cost is characterized by a single covariate raster
and we consider two specific cases. First is an increasing trend from
the NW to the SE (''systematic raster''), where $z(x)$ is defined as
$z(x) = r(x) + c(x)$ and $r(x)$ and $c(x)$ are just the row and
column, respectively, of the raster.  This might define something
related to distance from an urban area or a gradient in habitat
quality due to land use, or environmental conditions such as
temperature or precipitation gradients.  In the second case we make up
a covariate by generating a field of spatially correlated noise to
emulate a typical patchy habitat covariate (''patchy raster'') such as
tree or understory density. The two covariates are shown in
Fig. \ref{ecoldist.fig.raster100}, along with a sample realization of
$N=100$ individuals (left panel only).  For both covariates we use a
cost function in which transitions from pixel ${\bf x}$ to ${\bf x}'$
is given by:

\[
 log(cost({\bf x},{\bf x}'))=  \theta_2 \frac{z({\bf x}) + z({\bf x}')}{2}
\]

{\flushleft where} $\theta_2 = 1$ for simulating the observed data.
 Remember that with $\theta_2=0$ the
model reduces to one in which the cost of moving across each pixel is
constant, and therefore Euclidean distance is operative.

\begin{figure}
\begin{tabular}{cc}
\includegraphics[height=3.25in,width=3.25in]{Ch10/figs/raster_withN100}
\includegraphics[height=3.25in,width=3.25in]{Ch10/figs/raster_krige} &
\end{tabular}
\caption{Two covariate rasters used for simulations. A hypothetical
  realization of $N=100$ activity centers is superimposed on the left,
along with 16 trap locations. }
\label{ecoldist.fig.raster100}
\end{figure}

\subsection{Non-stationarity of home range structure}

When distance is defined by the cost-weighted distance metric given
by Eq. \ref{eq.lcp} then individual space-usage varies
spatially in response to the landscape covariate(s) used in the
distance metric.  As a consequence, home ranges contours are no longer
circular, as in SCR models based on Euclidean distance.
 For example, using one of the covariates we use in
our simulation study below (Fig. \ref{ecoldist.fig.raster100}, right
panel) with a Gaussian pdf detection function but having distance
metric defined by Eq. \ref{eq.lcp}, produces home ranges such
as those shown in Fig. \ref{fig.homeranges}. Later we simulate data
under the model that produces these home ranges and fit spatial
capture-recapture models to evaluate the efficacy of likelihood
estimation under this model.

\begin{figure}
\begin{center}
\includegraphics[height=6in,width=3.75in]{Ch10/figs/home_ranges}
\end{center}
\caption{
Typical home ranges for 6 individuals based on the cost surface shown in
  Fig. \ref{ecoldist.fig.raster100} with $\theta_{2}=1$. The black dot indicates the home
  range center and the pixels around each home range center are shaded
according to the probability of encounter, if a trap were located in
that pixel.
}
\label{fig.homeranges}
\end{figure}


\subsection{Simulation and Analysis}

We begin by simulating some data... we have to load the \mbox{\tt
scrbook} library, use the function \mbox{\tt make.EDcovariates} to generate
our raster covariates, process that into a least-cost path distance
matrix, and then simulate observed encounter data using standard methods
which we have used many times previously in this book. The complete set
of {\bf R} commands is:

{\small
\begin{verbatim}
library("scrbook")
out<-make.EDcovariates()
covariate<-out$covariate.patchy
set.seed(2013)

N<-200
theta0<- -2
sigma<- .5
K<- 5

theta1<- 1/(2*sigma*sigma)
r<-raster(nrows=20,ncols=20)
projection(r)<- "+proj=utm +zone=12 +datum=WGS84"
extent(r)<-c(.5,4.5,.5,4.5)
theta2<-1
cost<- exp(theta2*covariate)
tr1<-transition(cost,transitionFunction=function(x) 1/mean(x),directions=8)
tr1CorrC<-geoCorrection(tr1,type="c",multpl=FALSE,scl=FALSE)

# make up some trap locations
xg<-seq(1,4,1); yg<-4:1
pts<-cbind( sort(rep(xg,4)),rep(yg,4))

traplocs<-pts
points(traplocs,pch=20,col="red")
ntraps<-nrow(traplocs)

S<-cbind(runif(N,.5,4.5),runif(N,.5,4.5))
D<-costDistance(tr1CorrC,S,traplocs)
probcap<-plogis(theta0)*exp(-theta1*D*D)
# now generate the encounters of every individual in every trap
Y<-matrix(NA,nrow=N,ncol=ntraps)
for(i in 1:nrow(Y)){
 Y[i,]<-rbinom(ntraps,K,probcap[i,])
}
Y<-Y[apply(Y,1,sum)>0,]

\end{verbatim}
}


Now we use the {\bf R} function \mbox{\tt nlm} along with
our \mbox{\tt intlik3ed} function to evaluate the likelihood so that we can obtain the MLEs of the
model parameters. We'll do that for both the standard Euclidean distance
and then for the ecological distance based on the ``patchy'' covariate:
{\small
 \begin{verbatim}
frog1<-nlm(intlik3ed,c(theta0,theta1,3)),hessian=TRUE,y=Y,K=K,X=traplocs,
               distmet="euclid",covariate=covariate,theta2=1)

frog2<-nlm(intlik3ed,c(theta0,theta1,3,-.3),hessian=TRUE,y=Y,K=K,X=traplocs,
               distmet="ecol",covariate=covariate,theta2=NA)
\end{verbatim}
}

Show nlm() output for each and comment .......................XXXX

\subsection{Simulation study}

\citet{royle_etal:2012ecol}
carried-out a limited simulation study to evaluate the
general statistical performance of the density estimator under
this new model, the effect of mis-specifying the model with a
normal Euclidean distance metric and whether the parameter of the
cost function could be effectively estimated.
We recapitulate their results here.
For population sizes of 100 and 200 individuals with activity
centers randomly distributed on the $20 \times 20$ landscape, they
subjected individuals
to encounter by 16 traps arranged in a $4\times 4$ grid
using a Gaussian
encounter model with least-cost path distance metric:
\[
log(p_{ij})= \theta_{0} + \theta_{1} d_{lcp}({\bf x}_{j},{\bf
  s}_{i}; \theta_{2})^{2}
\]
where  $\theta_{0} = -2$ and $\theta_{1} = 2$, the latter value
corresponding to $\sigma = 0.5$ of a stationary bivariate normal home
range model.  Different numbers of replicate samples were considered,
$K=3,5,10$
(e.g., nights in a camera trapping study), in order
to produce varying sample
sizes.

Three different models were fitted
to each simulated data set: the
misspecified euclidean distance model; (ii) the true data-generating
model with the relative cost raster {\it known} and (iii) the true
data-generating model but estimating the relative cost parameter by
maximum likelihood.  We used both the ``systematic'' and ``patchy''
covariates defined previously.

\subsection{Simulation Results}

For both landscapes and all simulation conditions (levels of $K$ and
$N$) the average sample sizes of individuals captured are given in
Tab. \ref{tab.samplesize}.  The simulation results for estimating $N$
for the prescribed state-space are presented in Table
\ref{tab.results1}.  For the ``patchy'' landscape we see extreme
bias in estimates of $N$ when the Euclidean distance is used. There is
moderate small sample bias of 3-5\% in the MLE of $N$ using the
least-cost distance which becomes negligible as $K$ increases. For
$N=200$ the bias is on the order of 2\% for the lowest sample size
case ($K=3$) but negligible otherwise.  Interestingly, for the
landscape exhibiting systematic structure, there is a persistent bias
in the MLE of $N$ of 1-3\% even for the highest level of $K$.
We were
initially surprised by this but, in fact, it is due to the fact that
the state-space is small relative to the extent of the trap grid and
sensitivity to a state-space that is too small is expected because the
support of the integrand is truncated. In the particular case of the
systematic landscape, we find that, in the NW corner of the raster
where cost of movement is low, individuals use large areas of space,
and the fitted model is under-stating the apparent
heterogeneity in encounter probability for the prescribed raster.  We
found that the issue is resolved when the traps are moved away from
the boundary (Tab. \ref{tab.results3}).

The performance of estimating the cost parameter $\theta_{2}$ mirrors
the results for estimating $N$ for the prescribed state space. In the
patchy landscape where we don't expect a systematic gradient in space
usage around the edge of the state-space, we see
(Table \ref{tab.results2}) that $\theta_{2}$ is estimated with
diminishing bias as the sample size increases, but with persistent
bias due to truncation of the likelihood under the systematic
landscape which, as with the MLE of $N$, is resolved by moving the
traps away from the edge of the raster. Equivalently, in practice,
this could be resolved by expanding the raster away from the trap
locations so that all regions used by animals exposed to capture are
included in the state-space.



\begin{table}[htp]
\centering
\caption{
Expected sample sizes of captured individuals under each configuration of
$N$ (population size for the prescribed state-space) and $K$ (number of replicate samples).
}
\begin{tabular}{l|rrrr}
 & \multicolumn{2}{c}{Systematic} & \multicolumn{2}{c}{Patchy}  \\
    & N=100 &  N=200  &   N=100 &  N=200  \\ \hline
K=3 &  38.69 &   78.17  &   37.30 &   74.93  \\
K=5 &  51.10 &  103.18  &   51.89 &  103.71 \\
K=10&  65.81 &  132.39  &   69.44 &  138.76 \\
\end{tabular}
\label{tab.samplesize}
\end{table}




\begin{table}[htp]
\label{tab.results1}
{\tiny
\caption{Simulation results for estimating population size $N$ for a prescribed state-space with
$N=100$ or $N=200$ and various levels of replication ($K$) chosen to affect the observed sample
size of individuals (Tab. \ref{tab.samplesize}). For each simulated data set, the SCR model was fitted with
standard Euclidean distance (``euclid''), least-cost path assuming the
cost parameter $\theta_2$ is known (``lcp/known''), or allowing it to
be estimated by maximum likelihood (``lcp/est'').
The summary statistics of the
sampling distribution reported are the mean, standard deviation
(``SD'') and quantiles (0.025, 0.50, 0.975).
}
{\bf Systematic trend raster:} \\
\begin{tabular}{l|rrrrr|rrrrr}
         & \multicolumn{5}{c}{N=100   } & \multicolumn{5}{c}{N=200  }  \\
         &   mean &  SD  & 0.025 & 0.50 & 0.975  & mean  & SD   & 0.025 & 0.50  & 0.975 \\ \hline
K=3      &        &      &       &      &        &       &      &       &       &       \\
euclid   &   63.65& 12.62& 44.77 & 61.17&  90.98 & 126.68& 17.05&  98.93& 124.49& 168.26 \\
lcp/known&   99.28& 20.80& 68.83 & 97.55& 152.59 & 196.47& 27.39& 152.03& 192.96& 259.78\\
lcp/est  &  101.93& 21.68& 67.95 &101.56& 156.21 & 201.58& 28.14& 154.96& 200.15& 263.20\\
K=5      &        &      &       &      &        &       &      &       &       &        \\
euclid   &  64.60 & 7.11 & 51.52 & 63.86&  77.33 & 130.02& 10.25& 113.48& 128.96& 151.32\\
lcp/known&  95.96 &11.64 & 74.21 & 96.16& 117.65 & 193.04& 17.13& 166.84& 191.88& 226.16\\
lcp/est  &  98.94 &12.97 & 74.68 & 99.00& 123.88 & 198.80& 19.60& 166.87& 197.97& 239.46\\
K=10     &        &      &       &      &        &       &      &       &       &       \\
euclid   &  69.24 & 4.83 & 59.37 & 69.47&  79.18 & 139.83&  7.62& 125.65& 139.65& 154.82\\
lcp/known&  94.46 & 7.04 & 81.45 & 94.04& 108.83 & 190.47& 11.55& 170.49& 189.74& 213.19\\
lcp/est  &  97.53 & 8.18 & 82.02 & 97.62& 113.16 & 195.19& 13.28& 171.63& 194.58& 217.96\\ \hline
\end{tabular}
\\
{\bf Patchy "random" raster: } \\
\begin{tabular}{l|rrrrrrrrrr}
         & \multicolumn{5}{c}{N=100  } & \multicolumn{5}{c}{N=200   }  \\
         &   mean &  SD  & 0.025 & 0.50  & 0.975  & mean  & SD   & 0.025 & 0.50  & 0.975 \\ \hline
K=3      &        &      &       &       &        &       &      &       &       &       \\
euclid   &  78.68 & 18.12& 49.40 & 76.34 & 125.47 & 154.34& 33.74& 107.00& 146.34& 221.43\\
lcp/known& 109.09 & 27.52& 69.50 &104.86 & 183.72 & 207.18& 46.53& 143.31& 198.42& 315.89\\
lcp/est  & 110.96 & 28.65& 69.55 &106.98 & 181.84 & 208.77& 49.29& 141.68& 197.89& 325.77\\
K=5      &        &      &       &       &        &       &      &       &       &        \\
euclid   &  77.85 & 11.55& 59.17 & 77.44 & 101.14 & 153.39& 15.57& 129.31& 149.54& 185.38\\
lcp/known& 103.57 & 15.83& 78.15 &100.58 & 137.48 & 201.57& 21.25& 165.94& 199.95& 243.26\\
lcp/est  & 104.44 & 15.79& 78.38 &101.47 & 139.55 & 200.91& 20.78& 164.42& 200.47& 246.46\\
K=10     &        &      &       &       &        &       &      &       &       &       \\
euclid   &  78.01 & 5.26 & 68.00 & 77.96 & 87.81  & 156.27&  8.51& 142.17& 156.05& 174.55\\
lcp/known&  99.84 & 7.09 & 86.86 & 99.84 & 114.11 & 198.64& 11.04& 181.43& 197.62& 220.45\\
lcp/est  & 100.42 & 7.56 & 86.72 &100.34 & 115.47 & 198.45& 11.44& 180.06& 198.04& 219.52\\ \hline
\end{tabular}
}
\end{table}





\begin{table}[htp]
\centering
\caption{
Mean of sampling distribution of the cost function parameter
$\theta_{2}$ for the different simulation
conditions.
}
\begin{tabular}{l|rrrr}
 & \multicolumn{2}{c}{Patchy} & \multicolumn{2}{c}{Systematic} \\
    & N=100 &  N=200  &   N=100 &  N=200  \\ \hline
K=3 &   1.05&    1.03 &     1.17 & 1.14 \\
K=5 &   1.02&    1.01 &     1.12 &1.12 \\
K=10&   1.01&    1.00 &     1.10 &1.08 \\
\end{tabular}
\label{tab.results2}
\end{table}




\begin{table}[htp]
{\tiny
\caption{Simulation results for estimating population size $N$ for a prescribed state-space with
$N=100$ or $N=200$ and various levels of replication ($K$) chosen to affect the observed sample
size of individuals. These results correspond to those of the
systematic landscape in Table 2 except with the traps
moved 0.5 units in from the boundary of the raster.
Each grouping of 3 rows (for a given value of $K$) summarizes the
performance of $\hat{N}$ under 3 distance models: (1) A model in which
Euclidean distance was used (``euclid''); (2) A model in which the
least-cost path distance was used, with the coefficient of the cost
function fixed (``lcp/known''); and (3) A model in which the
coefficient was estimated (``lcp/est''). The summary statistics of the
sampling distribution reported are the mean, standard deviation
(``SD'') and quantiles (0.025, 0.50, 0.975).
}
{\bf Systematic trend raster:} \\
\begin{tabular}{l|rrrrr|rrrrr}
         & \multicolumn{5}{c}{N=100   } & \multicolumn{5}{c}{N=200  }  \\
         &   mean &  SD  & 0.025 & 0.50 & 0.975  & mean  & SD   & 0.025 & 0.50  & 0.975 \\ \hline
K=3      &        &      &       &      &        &       &      &       &       &       \\
euclid   &   84.48& 20.42& 51.16 & 81.51& 140.62 &163.70 &24.55 &126.64 &157.67 &223.63 \\
lcp/known&  104.14& 25.49& 65.67 &101.50& 173.19 &200.16 &29.27 &158.65 &191.04 &268.78\\
lcp/est  &  105.90& 26.19& 65.95 &103.40& 182.30 &201.34 &29.54 &161.88 &192.36 &268.98\\
K=5      &        &      &       &      &        &       &      &       &       &       \\
euclid   & 81.21  &11.33 &61.35  &79.20 & 98.86  &163.27 &13.06 &140.21 &162.97 &185.94\\
lcp/known& 99.93  &12.86 &76.97  &99.75 &117.76  &199.80 &16.60 &170.25 &198.23 &227.66\\
lcp/est  & 100.84 &13.15 &79.96  &99.51 &119.08  &200.25 &16.53 &168.88 &199.29 &227.39\\
K=10     &        &      &       &      &        &       &      &       &       &       \\
euclid   &  80.10 & 7.81 &66.45  &79.14 &93.33   &158.40 & 9.25 &142.74 &157.86 &173.18\\
lcp/known& 100.07 & 9.50 &82.99  &100.33&114.81  &197.62 &12.58 &171.95 &199.21 &217.19\\
lcp/est  & 100.10 & 9.88 &82.31  &100.91&116.27  &197.52 &13.03 &169.49 &200.68 &217.82\\ \hline
\end{tabular}
}
\label{tab.results3}
\end{table}





\section{Illustration: Example Good vs. Bad habitat}

We provide another illustration of how to employ ecological distance
calculations in SCR models. This example shows more GIS-like analysis
for a situation where we have something like a hard habitat boundary
created to mimic a habitat corridor or park unit or some other block
of relatively homogeneous good-quality habitat for some species. This
particular system (shown in Fig. \ref{ecoldist.fig.corridor}), could
be habitat surrounded by a suburban wasteland of McDonalds and
Wal-Marts, much less hospital habitat for most species.  For our
purposes, we suppose that individuals live within the buffered ``f''
shaped region, although we could also imagine the negative of the
situation in which individuals live outside of the region, so that the
polygon represents a barrier (a lake) or bad habitat (an urban area)
or similar.  We describe the steps for creating this landscape
shortly, so that the reader can use a similar process to generate more
relevant landscapes for their own problems.

In this case we're not going to estimate any parameters of the cost
function (though we could) but instead we're going to use ecological
distance ideas only to constrain movement within (or to avoid)
landscape features.  However, the reader is encouraged to adapt the
likelihood function given in the previous section for this specific
case, so that a parameter of the cost function can be estimated.

\subsection{Basic Geographic Analysis in R}

In practical applications our landscape will contain one more more
polygons which delineate good or bad habitat or other important
characterisetics of the landscape.  These might exist as GIS
shapefiles or merely as a text file with coordinates defining polygon
boundaries. To work with polygons in the context of SCR models we need
to create a raster, overly the polygon and assign values to each pixel
depending on whether pixels are in the polygon or not, or how far they
are from polygon boundaries. These operations are relatively easy to
do within a GIS system but we need to be able to do them in ${\bf R}$
and we develop methods for this here.  See also
secs. \ref{mle.sec.shapefile} and \ref{mcmc.sec.state-space} 
for examples of reading in the shapefile and using them to affect
calculations in SCR models.

The first thing we do here is create a set of polygons by
buffering and joining some line segments.
In the {\bf R} library \mbox{\tt scrbook}, we provide
 a function \mbox{\tt make.seg} which allows the user to make such
 lines segments given a
specific trap region.  To involve \mbox{\tt make.seg} we first
create a plot region and then call \mbox{\tt make.seg} which has a
single argument being the number of points used to define the line
segment. In the following set of commands we generate two line
segments, \mbox{\tt l1} consisting of 9 points and \mbox{\tt l2}
consisting of 5 points, and these reside in a geographic region
enclosedd by $[0,10] \times [0,10]$:
{\small
\begin{verbatim}
library("scrbook")
library("sp")
plot(NULL,xlim=c(0,10),ylim=c(0,10))
l1<-make.seg(9)
plot(l1)
l2<-make.seg(5)
plot(l1)
lines(l2)
\end{verbatim}
}

We used this function to create a couple of line segments of class
\mbox{\tt SpatialLines} from the {\bf R} package \mbox{\tt sp}, which
can be loaded from \mbox{\tt scrbook} as  follows
\begin{verbatim}
data("fakecorridor")
\end{verbatim}
This has 2 line files in it (\mbox{\tt l1} and \mbox{\tt l2}) and a
trap locations file (\mbox{\tt traps}).
We use some functions from the {\bf R} packages \mbox{\tt sp} and
\mbox{\tt rgeos} to join and
buffer (by 0.5 units) the two segments. The commands are as follows
and the result is shown in Fig. \ref{ecoldist.fig.corridor}.

{\small
\begin{verbatim}
data("fakecorridor")
library("sp")
library("rgeos")

buffer<- 0.5
par(mfrow=c(1,1))
aa<-gUnion(l1,l2)
plot(gBuffer(aa,width=buffer),xlim=c(0,10),ylim=c(0,10))
pg<-gBuffer(aa,width=buffer)
pg.coords<- pg@polygons[[1]]@Polygons[[1]]@coords

xg<-seq(0,10,,40)
yg<-seq(10,0,,40)

delta<-mean(diff(xg))
pts<- cbind(sort(rep(xg,40)),rep(yg,40))
points(pts,pch=20)

in.pts<-point.in.polygon(pts[,1],pts[,2],pg.coords[,1],pg.coords[,2])
points(pts[in.pts==1,],pch=20,col="red")
\end{verbatim}
}

\begin{figure}
\begin{center}
\includegraphics[height=3.25in,width=3.25in]{Ch10/figs/corridor}
\end{center}
\caption{A made-up corridor or reserve.}
\label{ecoldist.fig.corridor}
\end{figure}


We focus on devising a SCR model for this corridor system and we
imagine that animals will tend to severely avoid leaving the buffered
habitat zone. Therefore, we assign $\mbox{\tt cost}=1$ if a pixel
is within the buffer,
and $\mbox{\tt cost} = 10000$ if a pixel is outside of a
buffer. Therefore the cost to move to a neighboring pixel outside of
the buffered area is $5000.5$ compared to the cost of 1 to move to a
neighboring pixel inside the buffer.

In this example, we're not going to estimate parameters of the cost
function. Therefore, in that case, we can compute the ecological
distance matrix one time and modify our likelihood code to accept the
distance matrix as input. We give that likelihood in the library
\mbox{\tt scrbook} as the function \mbox{\tt intlik3edv2}.
We note also that it provides a vector of 0's and 1's that
define any potential state-space restrictions. i.e., 1 if the pixel is
an element of the state-space and 0 if it is not.
In the analysis of this
simulated data set, we define the state-space to be the buffered
corridor system. The help file for \mbox{\tt intlik3edv2} contains the
script that follows.

Here we simulate $N=200$ guys in the corridor system and so we
restrict out state-space accordingly for purposes of fitting the
model. However we encourage the reader to refit the model without the
state-space restriction (for fitting the model only) and then
contemplate the result.  The code for doing all of this is as follows

{\small
\begin{verbatim}
cost<-rep(NA,nrow(pts))
cost[in.pts==1]<-1      # low cost to move among pixels but not 0
cost[in.pts!=1]<-10000  # high cost

library("raster")
r<-raster(nrows=40,ncols=40)
projection(r)<- "+proj=utm +zone=12 +datum=WGS84"
extent(r)<-c(0-delta/2,10+delta/2,0-delta/2,10+delta/2)
values(r)<-matrix(cost,40,40,byrow=FALSE)
par(mfrow=c(1,1))
plot(r)
points(pts,pch=20,cex=.4)

library("gdistance")
tr1<-transition(r,transitionFunction=function(x) 1/mean(x),directions=8)
tr1CorrC<-geoCorrection(tr1,type="c",multpl=FALSE,scl=FALSE)
costs1<-costDistance(tr1CorrC,pts)
outD<-as.matrix(costs1)
plot(pts,pch=".")
points(pts[in.pts==1,],pch=20,col="red")

library(``scrbook'')
traplocs<-traps$loc
trap.id<-traps$locid
ntraps<-nrow(traplocs)

set.seed(2013)
N<-200
S.possible<- (1:nrow(pts))[in.pts==1]
S.id<-sample(S.possible,N,replace=TRUE)
S<- pts[S.id,]

D<- outD[S.id,trap.id]
eD<- e2dist(S,traplocs)
Dtraps<-outD[trap.id,]

alpha0<- -1.5
sigma<- 1.5
beta<- 1/(2*sigma*sigma)
K<-10

probcap<-plogis(alpha0)*exp(-beta*D*D)
Y<-matrix(NA,nrow=N,ncol=ntraps)
for(i in 1:nrow(Y)){
 Y[i,]<-rbinom(ntraps,K,probcap[i,])
}
Y<-Y[apply(Y,1,sum)>0,]

frog1<-nlm(intlik3edv2,c(-2.5,2,log(4)),hessian=TRUE,y=Y,K=K,X=traplocs,
            S=pts,D=Dtraps,inpoly=in.pts)
frog2<-nlm(intlik3edv2,c(-2.5,2,log(4)),hessian=TRUE,y=Y,K=K,X=traplocs,
            S=pts,D=Deuclid,inpoly=in.pts)
\end{verbatim}
}

In the example that we ran above we compared the result for using
distance-within-the-corridor to normal Euclidean distance and the
results do not differ too much in this single instance. One reason is
that the distance between individuals and traps that they are likely
to be captured in is well-approximated by normal Euclidean distance.


\section{A stream network}

Later we might add a 3rd prototype situation involving a stream network.

We could use ``distance from stream'' to model effects of habitat
and corridors or whatever


\section{Summary and Outlook}


All published applications of SCR models to date have been based on models for the
encounter probability that are functions of the standard Euclidean
distance between individuals and traps. The obvious limitations are
that it is unaffected by landscape or habitat structure and implies
stationary, isotropic and symmetrical home ranges. These are standard
criticisms of the basic SCR model as universally applied in
practice. However, it is not a relevant criticism of the basic
conceptual formulation of SCR models, because, as we have
demonstrated, one can modify the Euclidean distance metric to
accommodate more realistic space usage considerations.  Following
\citet{royle_etal:2012ecol},
we demonstrated how to use
minimum cost-weighted distance (i.e., ``least-cost
path'') between points, and where ``cost'' is characterized by one or
more spatially explicit covariates that are believed to influence
movement or space-usage of individuals.

How animals use space and therefore how distance to a trap is
perceived by individuals is not something that can ever be known. We
can only ever conjure up models to describe this phenomenon and fit
those models to limited data on a sample of individuals during a
limited amount of time.  Here we have shown that there is hope to
estimate parameters, from capture-recapture data, that describe how
animals use space and thereby allow for irregular home range geometry
that is influenced by landscape structure.

Not surprisingly, our simulation study demonstrated
(Table 2) that the MLE of model parameters is
approximately unbiased in moderate sample sizes. Moreover, the effect
of ignoring ecological distance and using normal Euclidean distance in
the model for encounter probability, has the logical effect of causing
negative bias in estimates of $N$.  We expect this because the effect
is similar to failing to model heterogeneity. i.e., if we mis-specify
``model $M_h$'' \citep{otis_etal:1978} with ``model $M_0$''
\citep{otis_etal:1978} then we will expect to under-estimate $N$. So
the effect of mis-specifying the ecological distance metric with a
standard homogeneous Euclidean distance has the same effect. As a
practical matter, it stands to reason that many previous applications
of SCR models based on homogeneous distance metrics have under-stated
density of the focal population.

In our view, this bias is not really the most important reason to
consider models of ecological distance. Rather, inference about the
structure of ecological distance is fundamental to many problems in
applied and theoretical ecology related to modeling landscape
connectivity, corridor and reserve design, population viability
analysis, gene flow, and other phenomena.  Our new model allows
investigators to evaluate landscape factors that influence movement of
individuals over the landscape from non-invasively collected
capture-recapture data.  Therefore SCR models based on ecological
distance metrics might aid in understanding
aspects of space usage and movement in animal populations and, ultimately, in addressing conservation-related problems such as corridor design.

We considered inference for ecological distance models based on
marginal likelihood \citep{borchers_efford:2008}
(see Chapt. \ref{chapt.mle}).
In principle,
Bayesian analysis does not pose any unique challenges for this new
class of models, except that computing the cost-weighted distance is
computationally intensive.  So, having to do this at each iteration of
an MCMC algorithm may be impractical using existing algorithms.  A
related issue is that the size of the raster slows things down. For
very large rasters, even likelihood analysis can be computationally
challenging and methods for efficient calculation of the ecological
distance given the raster covariate(s) and parameters might be needed.






























%\chapter{Ecological Distance Models in Spatial Capture-Recapture}
%\label{chapt.ecoldist}




\chapter{%State-space Covariates
%Modeling Spatial Variation in Density Using State-Space Covariates
Modeling Spatial Variation in Density
}
\markboth{Spatial Variation in Density}{}
\label{chapt.state-space}

\vspace{0.3cm}

\begin{comment} ok this is a minor tech mpoint for now: but this is introduced as a ``point process''
but what is being decribed here is a REALiZATION of a point process. Lets clarify this in the final
draft
\end{comment}
Underlying all spatial capture-recapture models is a point process
model that describes the distribution of individual activity
centers (${\bf s}$) within the state space ($\cal{S}$).
%, which is
%typically a two-dimensional polygon defining the study area.
Point process models are charcterized by $\mathcal{S}$ and by an
intensity parameter defined at each point in $\mathcal{S}$. If this
intensity is constant, the point process is said to be homogeneous,
and thus far we have focused our
attention on the homogeneous binomial point process whose realized
values are:
${\bf s}_i \sim \mbox{Unif}({\cal S}), i=1,2,\dots,N$, where $N$ is the
size of the population. This is a model of
``spatial-randomness''\footnote{The phrase ``complete
  spatial-randomness'' is reserved for the homogeneous Poisson point
  process}
because the intensity of the
activity centers is constant across the study area.
% and the activity
%centers are distributed independently of each other.

The spatial-randomness assumption is often viewed as restrictive
because ecological processes such as
territoriality and habitat selection can result in non-uniform
distributions of organisms. We have argued, however, that this
assumption is less restrictive than may be recognized because the
homogeneous point process actually allows for infinite
possible configurations of activity centers. Furthermore, given enough data,
the uniform prior will have very little influence on the estimated
locations of activity centers. Nonetheless, the homogeneous point
process model does not allow one to model population density using
covariates, which is a central objective of much ecological research.
For example, a homogeneous point process model
may result in a density surface map indicating that individuals were
more abundant in one habitat than another, but it does not do so
explicitly and so cannot be used to make predictions about
habitat-specific abundance in other regions. A more direct approach would be to replace
the homogeneous model with an inhomogeneous model in which the point process
intensity varies spatially.
%density using covariates as is done in generalized linear models (GLMs)
%\citep{mccullagh_nelder:1989}. % where a
%link function is used to connect the intensity parameter to the linear
%predictor.

In this chapter we present a method
for fitting inhomogeneous binomial point process models by modeling
the intensity parameter as a function of
covariates in much the same way as is done with generalized linear
models. The covariates we consider differ
from those covered in previous chapters, which were typically
attributes of the animal ({\it e.g.} sex or age) or the trap ({\it
  e.g.} baited or not) and were used to model movement or encounter
rate. In contrast, here we wish to
model covariates that are defined for all points in
$\cal{S}$, which we will refer to as
state-space covariates or density covariates. These may
include continuous covariates such as elevation, or discrete
covariates such as habitat type.

Inhomogeneous Poisson point process models were discussed in the original
formulation of SCR models \citep{efford:2004} and were described in
detail by \citet{borchers_efford:2008}. Our approach is
similar to that of \citet{borchers_efford:2008}, except that we use a binomial
rather than a Poisson model because the binomial model is
easily integrated into our MCMC algorithm.  %data augmentation scheme
%and is consistent
%with the objective of determining how a {\it fixed} number of activity
%centers are distributed with respect to covariates.
The method we use to accommodate inhomogeneous binomial point process
models %within our MCMC algorithm
is simple---we
replace the uniform prior with a prior describing the
distribution of the $N$ activity centers conditional on the
covariates. Development of this prior, which does not have a
standard form, is a central component of this chapter. First we
begin with a review of homogeneous point process models.


\section{Homogeneous point process revisited}

The homogeneous Poisson point process is \textit{the} model of ``complete
spatial randomness'' and is often used in ecology as a null model
to test for departures from randomness
\citep{diggle:2003,illian_etal:2008}. Given its central role in the
analysis of point processes, it is helpful to compare it with
the binomial model that we use in our SCR models. The sole parameter
of the homogeneous Poisson point process model is the
intensity parameter $\mu$ which describes the expected number
% start with \mu(s)???? Or, wait for inhomogeneous case?
of points in an infinitesimally small area. %Note that this intensity
%parameter is a single value, i.e. it does not vary spatially.
The intensity parameter can also be used to compute the expected number of points
in any region $B$ of the state-space $\cal{S}$. Specifically,
$\mathbb{E}[n(B)] = A(B)\mu$ where $A(B)$ is the area of region $B$.
This just says that
the expected number of points is the area of $B$
multiplied by the intensity parameter.
%This is one
%of the distinctions between the Poisson model and the binomial model,
%for which the counts $\{n(B_k)\}$ are not i.i.d., as we will explain
%shortly.

An important distinction between the Poisson point process and the
binomial point process is that $N$ is a random variable in the former
model but not in
the latter. In other words, the binomial point process conditions on $N$.
Here is some simple \R~code to illustrate this point:
\begin{verbatim}
mu <- 4                            # intensity
Np <- rpois(1, mu)                 # Np is random
PPP <- cbind(runif(Np), runif(Np)) # Poisson point process

Nb <- 4                            # Nb is fixed
BPP <- cbind(runif(Nb), runif(Nb)) # Binomial point process
\end{verbatim}
which generates realizations from Poisson and binomial point
processes in the unit square ($\mathcal{S} = [0,1]\times[0,1]$).
For both models, the $N$ points are
%independent of one another and
distributed uniformly
in $\mathcal{S}$, and they have the same intensity parameter,
$\mu=4$. However, in the binomial case the intensity parameter is
defined different, being a function of $N$ and the area of the
state-space, $\mu = N/A(\mathcal{S})$.

Another distinction between the two models is that if we divide the
state-space into $K$ disjunct regions, the number of points in each
region $\{ n(B_k): k=1,\dots,K \}$ are
independent and identically distributed (i.i.d.) under the Poisson model,
but some dependence exists under the binomial model.
%In the Poisson case
%we have $\n(B_k) \sim \text{Pois}(A(B_k)\mu)$, and if the points were independent
%$n(B_k) \sim Bin(N, p(B_k))$ for the binomial,
%where $p(B_k)$ is simply the proportion of the state-space in region
%$B_k$.
Fig.~\ref{state-space.fig.homo} illustrates this point.
The depicted state-space is the unit square, and thus the probability of a
point falling in each of the 25 disjunct regions is $p(B_k) = 1/25$ and
the expected counts are $\mathbb{E}(n(B_k)) = Np_k$.
%In
%the figure $N=50$, and consequently we would expect 2 points per pixel, which
%happens to be the empirical mean in this instance.
However, these counts are not
independent realizations from a binomial distribution since $\sum_{k=1}^K
n(B_k) = N$. Instead, the model for the entire vector
is ${n(B_1), n(B_2), \dots, n(B_k)} \sim \mbox{Multin}(N, \{p(B_1), p(B_2), \dots,
p(B_K) \})$ \citep{illian_etal:2008}.
%\begin{verbatim}
%n.Bk <- rmultinom(1, size=50, prob=rep(1/25, 25))
%matrix(n.Bk, 5, 5)
%\end{verbatim}
The dependence among counts has virtually
no practical consequence when the number of pixels is large. For
example, if there are 100 pixels, the number of points in one pixels
carries very little information about the expected number of points in another
pixel. However, if there are only 2 pixels, then clearly the number of
points in one pixel allows one to determine how many points will occur in the
remaining pixel.
%To gain familiarity with the multinomial distribution
%and the discrete representation of space, use the \verb+rmultinom+
%function in \R~to simulate counts similar to those shown in
%Fig.~\ref{state-space.fig.homo}, for example using commands
%such as:
%\begin{verbatim}
%n.Bk <- rmultinom(1, size=50, prob=rep(1/25, 25))
%matrix(n.Bk, 5, 5)
%\end{verbatim}


\begin{figure}[ht!]
\centering
\includegraphics[width=5in,height=2.5in]{Ch11/figs/homoPlots}
\label{state-space.fig.homo}
\caption{Homogeneous binomial point process with $N$=50 points
  represented in continuous and discrete space.}
\end{figure}


The discrete space representation of the binomial point process is of
practical importance when fitting SCR models because spatial covariates
are almost always represented in a discrete-space format called
``rasters'' in GIS-speak. In such cases, we often need to change our
definition of the prior for an activity center from ${\bf s}_i \sim
\mbox{Unif}(\cal{S})$ to ${\bf s}_i \sim \mbox{Multin}(1, \mathbf{\pi})$. In the
latter case, the activity center is simply defined as an integer
representing pixel ``id''.
%Note also that the multinomial distribution
%with an index of 1 (\emph{i.e.} \verb+size=1+ in \verb+rmultinom+)
%is referred to as the categorical distribution,
%which we will frequently use in the \verb+BUGS+ language.



\section{Inhomogeneous binomial point process}

\hl{Check for x instead of s throughout}

As with the homogeneous model, the inhomogeneous binomial point process
model is developed conditional on $N$. The primary distinction is that
the uniform distribution is replaced with another distribution
allowing for the intensity parameter to vary spatially. To arrive at
this new distribution, replace the scalar intensity parameter $\mu$
with the function $\mu(s, {\bm \beta})$, where $\bm \beta$ is a
vector of coefficients describing the effects of
spatially-referenced covariates on the point process intensity. In
what follows, we will often abbreviate the intensity function as $\mu(s)$,
dropping the vector of coefficients for readability. Since an intensity must be strictly
positive, and because the logarithm is the canonical link function of the Poisson
generalized linear model, it is natural to model $\mu(s, \beta)$ as
\[
\log(\mu(s, \beta)) = \beta_0 + \sum_{j=1}^J \beta_j v_j(s), \quad  x \in \cal{S}
\]
where $\beta_j$ is the regression coefficient for covariate
$v_j(s)$. To be clear, $v(s)$ is the value of any covariate, such as
habitat type or elevation, at location $x$ and it assumed to be
defined at all locations in the state-space.
This equation should look
familiar because it is the standard linear predictor used in log-linear
GLMs. One caveat is that the intercept $\beta_0$ is not a
unique parameter to be estimated.
%Note, however, that we have not included
%an intercept. The reason for this is that it would be confounded with
%$N$ (see Chapt. \ref{chapt.hscr}).
The reason for this is that %This should be intuitive since
$\beta_0$ represents population density at the location $x$ when
%, the expected value of $N$ in some infinitesimally
%small area when
the other $\beta$'s equal 0. However, we already
have a parameter in the model for expected abundance, namely $\mathbb{E}[D] =
N/A(\mathcal{S}) =  \psi M / A(\mathcal{S})$\footnote{Remember, $M$ is the size of the augmented population, and
$\psi$ is the probability that a member of $M$ is an actual
constituent of the population (Chapt. ~\ref{chapt.scr0}).}. Thus, in
practice, we can either remove $\beta0$ and model a value proportional
to the intensity, or we can define $\beta_0=\log(N/A(\mathcal{S}))$ and model the
intensity directly.
%an intercept would be
%redundant, and without it we are still able to achieve our goal of
%describing the distribution of $N$ activity centers as a function of
%spatial covariates. One caveat is that if we wish to make predictions
%to unsampled regions, it is useful to include the intercept $\beta_0 =
%\log(N)$.

Now that we have a model of the intensity parameter $\mu(s)$,
we need to develop the associated probability density function
$[\bf s]$ to use
in place of the uniform prior. Remembering that
the integral of a pdf must be unity, we can create the pdf
$[\bf s]$ by dividing
$\mu(s)$ by a normalizing constant, which in this case is the integral
of $\mu(s)$ evaluated over the entire state-space.
The probability density function is therefore
\begin{equation}
[\mathbf{s}] = \frac{\mu(s, \beta)}{\int_{x \in \mathcal{S}} \mu(s, \beta)\, \mathrm{d}s}
\label{eq.pdf.ipp}
\end{equation}
Substituting this distribution for the
uniform prior allows us to fit inhomogeneous binomial point process
models to spatial capture-recapture data. We can also use this
distribution to obtain the expected number of individuals in any given
region. Specifically, the proportion of $N$ expected to occur in any
region $B$ %when heterogeneity in density is present
is $p(B) = \int_B
f(s, \beta)\, \mathrm{d}x$. These are
also the multinomial cell probabilities if the regions are
disjoint and compose the entire state-space. We provide an example in
the next section, and in Fig.\ref{state-space.fig.hetero}.

As a practical matter, note that the integral in the
denominator of $f(s, \beta)$ is evaluated over space, and since we always regard
space as two-dimensional, this is a two-dimensional integral that can
be approximated using the methods discussed in
Chapter~\ref{chapt.poisson-mn}, which include
Monte Carlo integration and Gaussian quadrature. Alternatively, if
our state-space covariates are in raster format, \emph{i.e} they are
in discrete space, the integral can be replaced with a sum over
all pixels,
\begin{equation}
f(s, \beta) = \frac{\mu(s, \beta)}{\sum_{x \in \mathcal{S}} \mu(s, \beta)\, \mathrm{d}x}
\label{eq.pdf.dipp.d}
\end{equation}
which is much more efficient computationally.

Although the discrete space approach is standard practice, it is
technically unjustified because covariate values must be known for all
points in space. This same problem is present anytime that we have a
sample of the spatial covariates, rather than a function defining
their value for all points in space. In such cases, it may be necessary to
interpolate the values of the covariates for points in space where
they were not measured. One option would be to use a Kriging
interpolator, as demonstrated by \citet{rathbun:1996}. Another option
is to sample the spatial covariates using probabalistic sampling
methods, which allow for design-based estimators of their values for
the entire study area \citep{rathbun_etal:2007}. Either option could
be implemented as part of the MCMC algorithm, but even though such
approaches are technically necessary, we do not demonstrate them here
because it seems likely that they will be inconsequential in most
cases where the raster data are of high resolution, such that the loss
of information is negligible when going from continuous space to
discrete space.

We now have all the tools needed to fit inhomogeneous point process
(IPP) models. If we refer to the distribution for the
inhomogeneous point process as ``IPP'', we can write a
hierarchical description of a SCR model with a Poisson encounter process and
a half-normal detection function as
\begin{gather*}
w_i \sim \mbox{Bern}(\psi) \\
{\bf s_i} \sim \mbox{IPP}(\mu(s,\beta)) \\
\lambda_{ij} = \lambda_0 \exp(-\|{\bf s_i} - {\bf x_{j}}\|^2/(2\sigma^2)) \\
y_{ij} \sim \mbox{Poisson}(\lambda_{ij} w_i)
\end{gather*}
The use of $\mbox{IPP}(\mu(s, \beta))$ instead of
$\mbox{Unif}(\cal{S})$ is the only difference between a homogeneous
point process model and an inhomogeneous point process model, and the
two are equivalent when $\beta=0$.

\begin{comment}
The IPP for the activity centers
results in another IPP for the observation process, $\lambda(s)$, the
expected number of captures for a trap
at point. As was true for the homogeneous model, this
intensity function is a product of the point process intensity
and the encounter rate function, $\lambda(s) = \mu(s, {\bm \beta})
\lambda_{ij}$.
\end{comment}

In the next sections we walk through a few examples, building up from
the simplest case where we actually observe the activity centers as
though they were data. In the second example, we fit our new model to simulated
data in which density is a function of a single continuous
covariate. \hl{To build upon the developments in the previous chapter, we
further consider the plausible case where a state-space covariate is also a
covariate of ecological distance.} A small simulation study indicates
that both effects can be estimated. A fourth example shows an analysis in discrete space using
both \secr~\citep{efford:2011} and \jags~\citep{plummer:2003}. In the
fifth and final example, we model the intensity of
activity centers for a real dataset collected on jaguars
(\emph{Panthera onca}) in Argentina.

\section{Observed Point Processes}

In SCR models, the point process is not directly observed, but in
other contexts it is. Examples include the locations of disease
outbreaks, the locations of trees in a forest, or the locations of
radio-tracked animals. Indeed Eq.~\ref{eq.pdf.ipp} has been used
extensively in the radio-telemetry literature to model so-called
``resource selection functions'' \citep{manly_etal:2002,lele_keim:2006}.
When the point locations are directly observed,
estimating the parameters $\bf \beta$ is straight-forward as
demonstrated in the following example. This example also illustrates
the fundamental process that we will later embed in our MCMC algorithm
used to fit SCR models with IPP.

Suppose we knew the locations of 100 animals' activity
centers, perhaps as the result of an extensive telemetry study. To
estimate the intensity surface $\mu(s, \beta)$ underlying these
points, we need to derive the likelihood for our data under this
model. Given the probability density function $f(s, \beta)$
(Eq.~\ref{eq.pdf.ipp}) and assuming that the points are
mutually independent of one another,
the likelihood is given by the product
of $R$ such terms, where $R=100$ is the sample size in our
hypothetical example,
\emph{i.e.} the observed number of activity centers.
\[
\mathcal{L}({\bf \beta} | {\bf x}_i, \beta) = \prod_{i=1}^R f(s_i)
\]
Having defined the likelihood we could choose a prior distribution for
$\beta$ and obtain the posterior distribution of
$\bf \beta$ using Bayesian methods, or we can find the maximum likelihood
estimates (MLEs) using standard numerical methods as is demonstrated
below.

First, we simulate some data. Simulating data under an inhomogeneous point process model is often
accomplished using indirect methods such as rejection
sampling. Rejection sampling proceeds by
simulating data from a standard distribution and then accepting or
rejecting each sample using probabilities defined by the distribution
of interest. For more information, readers should consult an
accessible text such as \citet{robert_casella:2010}. In our example, we
simulate from a uniform distribution and then accept or reject using
the (scaled) probability density function $f(s, \beta)$. Note that we first define a
spatial covariate (elevation) that is a simple function of the spatial
coordinates increasing from the southwest to the northeast of our
state-space.\footnote{Such functional forms of
covariates are rarely available. Instead,  continuous spatial
covariates are more often measured on a discrete grid.}

The following \R~commands demonstrate the use of rejection sampling to
simulate an inhomogeneous point process for the covariate depicted in
Fig.~\ref{state-space.fig.hetero}. The code uses the \verb+cuhre+ function in
the {\tt R2Cuba} package to integrate the intensity function over
space \citep{hahn_etal:2011}. An alternative would be to evaluate the
integral on a fine grid of points as we have done in previous
chapters, but it is useful to gain familiarity with more efficient
integration functions in \R.

\begin{small}
\begin{verbatim}
# spatial covariate (with mean 0)
elev.fn <- function(s) x[1]+x[2]-1
# intensity function
mu <- function(s, beta) exp(beta*elev.fn(s=x))

# Simulate IPP using rejection sampling
set.seed(300225)
N <- 100
count <- 1
s <- matrix(NA, N, 2)
beta <- 2 # parameter of interest
elev.fn <- function(s) x[1]+x[2]-1
# Intensity function, mu(s,beta)
mu <- function(s, beta) exp(beta*elev.fn(x=x))
# 2-dimensional integration over space
int.mu <- R2Cuba:::cuhre(2, 1, mu, beta=beta)$value
elev.min <- elev.fn(c(0,0)) #elev.fn(cbind(0,0))
elev.max <- elev.fn(c(1,1)) #elev.fn(cbind(1,1))
Q <- max(c(exp(beta*elev.min) / int.mu,   #2d(beta),
           exp(beta*elev.max) / int.mu))   #2d(beta)))
while(count <= 100) {
  x.c <- runif(1, 0, 1); y.c <- runif(1, 0, 1)
  s.cand <- c(x.c,y.c)
  pr <- exp(beta*elev.fn(s.cand)) / int.mu #2d(beta)
  if(runif(1) < pr/Q) {
    s[count,] <- s.cand
    count <- count+1
    }
  }
\end{verbatim}
\end{small}


\begin{figure}[ht]
\centering
\includegraphics[width=5in,height=2.5in]{Ch11/figs/heteroPlots}
\label{state-space.fig.hetero}
\caption{An example of a spatial covariate, say elevation, and a
  realization of a inhomogeneous binomial point process with $N$=100
  and $\mu(s) = exp(\beta \mbox{elev}(s))$ where $\beta=2$.}
\end{figure}

The simulated data are shown in Fig~\ref{state-space.fig.hetero}. High elevations
are represented by light green and low elevations by dark green. The
activity centers of 100 animals are shown as
points, and it is clear that these simulated animals prefer the high
elevations.  %Perhaps they are mountain goats.
The underlying model describing this preference is
$\log(\mu(s)) = \exp(\beta \times elev(s))$
where $\beta=2$ is the parameter to be estimated.

Given these points, we will now estimate $\beta$ by minimizing the
negative-log-likelihood using \verb+R+'s \verb+optim+ function.

\begin{small}
\begin{verbatim}
# Negative log-likelihood
nll <- function(beta) {
    int.mu <- R2Cuba:::cuhre(2, 1, mu, beta=beta)$value
    -sum(beta*elev.fn(s) - log(int.mu))
}
starting.value <- 0
fm <- optim(starting.value, nll, method="Brent",
            lower=-5, upper=5, hessian=TRUE)
c(Est=fm$par, SE=sqrt(1/fm$hessian)) # estimates and SEs
\end{verbatim}
\end{small}


Maximizing the likelihood took a small fraction of a second, and we
obtained an estimate of $\hat{\beta}=1.99$. We could plug
this estimate into our linear model at each point in the state-space to
obtain the MLE for the intensity surface.

This example demonstrates
that if we had the data we wish we had, {\it i.e.} if we knew the
coordinates of the activity centers $\bf s$, we could easily estimate the
parameters governing the underlying point process. Unfortunately, in
SCR models, the activity centers cannot be directly observed, but
spatial re-captures provide us with the information needed to
estimate these latent parameters.

\section{Fitting inhomogeneous point process SCR models}

\subsection{Continuous space}

One of the nice things about hierarchical models is that they allow us
to break a problem up into a series of simple conditional
sub-models. Thus,
we can simply add the methods described above into our existing MCMC
algorithm to simulate the posterior distributions of $\beta$ conditional on the
simulated values of $\mathbf{s}$. To demonstrate, we will continue with
the previous example. Specifically, we will overlay a grid of
traps on the map shown in Fig.~\ref{state-space.fig.hetero}. We will then
simulate capture histories conditional upon the activity
centers. Then, we will attempt to estimate the activity center
locations as though we did not know where they were, as is the case in
real applications.

The following \R~code simulates encounter histories under a
Poisson observation model (see Chapt. \ref{chapt.poisson-mn}), which could be appropriate in camera
trapping studies or when using other methods in which animals could
be detected multiple times at a trap during a single occasion.

\begin{small}
\begin{verbatim}
# Create trap locations
xsp <- seq(-0.8, 0.8, by=0.2)
len <- length(xsp)
X <- cbind(rep(xsp, each=len), rep(xsp, times=len))

# Simulate capture histories, and augment the data
ntraps <- nrow(X)
T <- 5
y <- array(NA, c(N, ntraps, T))

nz <- 50 # augmentation
M <- nz+nrow(y)
yz <- array(0, c(M, ntraps, T))

sigma <- 0.1  # half-normal scale parameter
lam0 <- 0.5   # basal encounter rate
lam <- matrix(NA, N, ntraps)

set.seed(5588)
for(i in 1:N) {
    for(j in 1:ntraps) {
        distSq <- (s[i,1]-X[j,1])^2 + (s[i,2] - X[j,2])^2
        lam[i,j] <- exp(-distSq/(2*sigma^2)) * lam0
        y[i,j,] <- rpois(T, lam[i,j])
    }
}
yz[1:nrow(y),,] <- y # Fill
\end{verbatim}
\end{small}

Now that we have a simulated capture-recapture dataset $y$, and we have
augmented it to create the new data object $yz$, we are ready to
begin sampling from the posteriors. A commented Gibbs sampler written
in \R~is available in the accompanying \R~package \scrbook~(see
?scrIPP).
\begin{comment} see Ch 7 MCMC for SCR and cite some section of that \end{comment}
% There are two small parts of the
% \R~code that distinguish it from previous code we have shown to
% fit homogeneous point processes. First, we need to update the parameter
% ${\bf \beta}$ conditional on all other parameters in the model. The code to
% do so is: %\begin{comment} need cite to Ch 2 or 7 on MCMC for this \end{comment}
% \begin{small}
% \begin{verbatim}
% # Denominator of f(x, beta). Integral of mu(x, beta) over space
% D1 <- cuhre(2, 1, mu, lower=c(xlims[1], ylims[1]),
%             upper=c(xlims[2], ylims[2]), beta=beta1)$value
% # Compute the denominator again using a proposed beta1
% beta1.cand <- rnorm(1, beta1, tune[3])
% D1.cand <- cuhre(2, 1, mu, lower=c(xlims[1], ylims[1]),
%                  upper=c(xlims[2], ylims[2]), beta=beta1.cand)$value
% # Compute log(f(x))
% ll.beta1 <- sum(  beta1*elev.fn.v(S) - log(D1) )
% ll.beta1.cand <- sum( beta1.cand*elev.fn.v(S) - log(D1.cand) )
% if(runif(1) < exp(ll.beta1.cand - ll.beta1) )  {
%      beta1<-beta1.cand
% }
% \end{verbatim}
% \end{small}
% Next, we need to put the new prior on the activity centers:
% \begin{small}
% \begin{verbatim}
% # Compute the prior for s_i and a candidate. denominator is constant
% prior.S <- beta1*elev(S[i,1], S[i,2]) # - log(D1)
% prior.S.cand <- beta1*elev(Scand[1] + Scand[2]) # - log(D1)
% if(runif(1)< exp((ll.S.cand+prior.S.cand) - (ll.S+prior.S))) {
%     S[i,] <- Scand
%     lam <- lam.cand
%     D[i,] <- dtmp
%     }
% \end{verbatim}
% \end{small}
We can apply this modified sampler to our data using the
following \R~commands:
\begin{small}
\begin{verbatim}
set.seed(3434)
fm1 <- scrIPP(yz, X, M, 6000, xlims=c(0,1), ylims=c(0,1),
            tune=c(0.003, 0.08, 0.3, 0.07) )
plot(mcmc(fm1$out))
rejectionRate(mcmc(fm1$out))
\end{verbatim}
\end{small}
We obtain posterior distributions that are summarized in
Table~\ref{ch9.tab.simIPP}.
%Mixing is good, and as usual,
%life is very nice when we are working with simulated data.

\begin{table}[b]
\centering
\caption{Posterior summaries from inhomogeneous point process model
  fitted to simulated data. Space was treated as continuous.}
\begin{tabular}{lrrrrr}
\hline
& Mean & SD & 2.5\% & 50\% & 97.5\% \\
\hline
 $\sigma =0.10$ &   0.1026 &   0.0048 &   0.0935 &   0.1025 &   0.1123 \\
 $\lambda_0=0.50$ &   0.4419 &   0.0493 &   0.3496 &   0.4400 &   0.5390 \\
 $\psi =0.66$ &   0.6826 &   0.0554 &   0.5762 &   0.6820 &   0.7923 \\
 $\beta =2.00$ &   2.1601 &   0.3390 &   1.5193 &   2.1583 &   2.8043 \\
 $N =100$ & 102.7696 &   6.2689 &  92.0000 & 102.0000 & 117.0000 \\
\hline
\end{tabular}
\label{ch9.tab.simIPP}
\end{table}


Fitting continuous space IPP models is somewhat
difficult in \bugs~because our prior ``IPP'' is not one of the
available distributions that come with the software. It is
possible to add new distributions in \bugs, but it is somewhat
cumbersome.  \secr~allows
users to fit continuous space IPPs using polynomials of the x- and y-
coordinates, but it does not accept truly continuous covariates that
are functions of space. However, these
are not really important limitations because discrete
space versions of the IPP model are straight-forward, and virtually all spatial
covariates are, or can be, defined as such.


\subsection{Discrete space}

To fit IPPs using covariates in discrete space, \emph{i.e.} in raster
format, we follow the same steps
as outlined in Chapter~\ref{chapt.poisson-mn}---we define ${\bf s}_i$ as
pixel ID, and we use the categorical distribution as a prior. A good
example is found in \citep{mollet_etal:2012}. Here we present
an analysis of the simulated data shown in the %right panel of
Fig.~\ref{state-space.fig.hetero}. The spatial covariate, let's call it
elevation again, was simulated
using using the code shown on the help page
\verb+ch9simData+ in \scrbook. The points are the number of
activity centers in each pixel, generated from a single realization of
the inhomogeneous point process model with intensity
$\mu(x) = 2 \times \mbox{elev}(s)$.
\begin{figure}[ht]
\centering
\includegraphics[width=3in,height=3in]{Ch11/figs/discrete}
\label{ch9.fig.discrete}
\caption{Simulated activity centers in discrete space. The spatial
  covariate, elevation, is highest in the lighter areas. Density of
  activity centers (circles) increases with elevation. A single
  activity center is shown as a small circle, and larger circles
  represent two activity centers in a pixel. Trap locations
  are shown as crosses.}
\end{figure}

The \bugs~code to fit an IPP model to these data is shown in
panel~\ref{ch9.panel1}.The vector \verb+probs[]+ is the prior
probability defined
by~\ref{eq.pdf.ipp.d}, which is the probability that an individual's
activity center is located at pixel $x$. \verb+Sgrid+ is the
matrix of coordinates for each pixel.

%\begin{panel}[h!]
%\centering
%\rule[0.15in]{\textwidth}{.03in}
\begin{small}
\begin{verbatim}
model{
sigma ~ dunif(0, 1)
lam0 ~ dunif(0, 5)
beta ~ dnorm(0,0.1)
psi ~ dbeta(1,1)
for(x in 1:nPix) {
  theta[x] <- exp(beta*elevation[j])
  probs[x] <- theta[j]/sum(theta[])
}
for(i in 1:M) {
  w[i] ~ dbern(psi)
  s[i] ~ dcat(probs[])
  x0g[i] <- Sgrid[s[i],1]
  y0g[i] <- Sgrid[s[i],2]
  for(j in 1:ntraps) {
    dist[i,j] <- sqrt(pow(x0g[i]-grid[j,1],2) +
                      pow(y0g[i]-grid[j,2],2))
    lambda[i,j] <- lam0*exp(-dist[i,j]*dist[i,j] /
                            (2*sigma*sigma)) * w[i]
    y[i,j] ~ dpois(lambda[i,j])
    }
  }
N <- sum(w[])
Density <- N/1 # unit square
}
\end{verbatim}
\end{small}
%\rule[0.15in]{\textwidth}{.03in}
%\caption{\bugs~code for fitting inhomogeneous point process model in
%  discrete space.}
%\label{ch9.panel1}
%\end{panel}

This model can also be fit in \secr, which refers
to the raster data as a ``habitat mask''. \R~code to
fit the models using \secr~and \jags~is available in \scrbook---see
\verb#help(ch9secrYjags)#. Results of the
comparison are shown in Table \ref{ch9:tab:secrYjags} and are
very similar as expected.
\begin{comment}
\hl{ANDY, is there any point in discussing
  the slight differences?}
  YES: If we can explain it. Could it be MC error alone?
Otherwise I guess attributing it to differences between MLE and BAyes is ok. That seems like
a reasonable thing.
\end{comment}
\begin{table}[h!]
\centering
\caption{Comparison of \secr~and \jags~results. Point estimates from
  the Bayesian analysis are posterior means. Intervals are lower and
  upper 95\% CIs.}
\begin{tabular}{llrrrr}
\hline
Software & Parameter & Estimate & SD & lower & upper \\
\hline
 secr & $N=50$ & 49.2803 & 5.7535 & 41.0087 & 64.3879 \\
      & $\beta=2$ &  2.1772 & 0.5628 &  1.0741 &  3.2804 \\
      & $\lambda_0=0.8$ &  0.9203 & 0.0764 &  0.7824 &  1.0825 \\
      & $\sigma=0.1$ &  0.0990 & 0.0038 &  0.0918 &  0.1068 \\
\hline
 JAGS & $N=50$ & 48.2072 & 5.4053 & 39.0000 & 60.0000 \\
      & $\beta=2$ &  2.1026 & 0.5323 &  1.0889 &  3.1506 \\
      & $\lambda_0=0.8$ &  0.9328 & 0.0766 &  0.7898 &  1.0921 \\
      & $\sigma=0.1$ &  0.1004 & 0.0041 &  0.0929 &  0.1089 \\
\hline
\end{tabular}
\label{ch9:tab:secrYjags}
\end{table}


\section{Ecological distance and state-space covariates}

Habitat characteristics that affect population
density could also affect home range size and movement behavior. For
example, a
species that occurs in high density in a forest may be reluctant to
venture from a forest patch into an adjacent field. Thus, even if a
trap placed in a field is located very close to an animal's activity
center, the probability of capture may be very low. In this case
forest cover is a covariate of both density and encounter probability,
and we could model it as such by combining the methods described in
this chapter and in Chapter~\ref{chapt.ecoldist}. To demonstrate, we
continue with our analysis of the data shown in
Fig~\ref{state-space.fig.hetero}. Once again, we suppose that density
increases with elevation, but this time, we also make the
assumption that home range size decreases as density increases. This
commonly-observed phenomenon can be explained by numerous factors such
as intra-specific competition \citep{sillett_etal:2004} or optimal
foraging behavior \citep{tufto_etal:1996,said_servanty:2005}. To model
this effect, we
introduce the parameter $\theta$, which determines the ``cost'' of
moving between pixels. If $\theta=0$, then the animal perceives
distance as Euclidean. If $\theta>0$, then least-cost distance (LCD)
is greater than than Euclidean distance (ED). In most cases, we would
not expect,
or should not even consider the possibility of $\theta<0$ because this
implies that LCD$<$ED, which would mean that an animal could view
1000km as 1m. In addition to the fact that this is not biologically
justifiable, it also suggests that the area of the state-space could
be infinitely large. Thus, one may want to enforce the constraint that
$\theta$ is strictly $\geq 0$. See Chapter~\ref{chapt.ecoldist} for
more details.

One may wonder if it is possible to estimate both $\beta$
and $\theta$ using standard SCR data. Currently, it is not possible to
model least-cost distance using \jags~or \secr, so we wrote our own
function, \verb+scrDED+, to fit the model using maximum likelihood. An
example analysis is provided on the help page for the function in our
\R~package \scrbook. We briefly note here that the function requires
the capture history data, the trap locations, and the raster data
formatted using the {\tt raster} package
\citep{hijmans_vanetten:2012}. The linear model for the
intensity parameter $\mu(s, \beta)$ and the least-cost distance
function $lcd(\theta)$ are specified using \R's formula interface. A
simple function call is
\begin{verbatim}
fm <- scrDED(y, traplocs=X, den.formula=~elev, dist.formula=~elev,
             rasters=elev.raster)
\end{verbatim}
To assess the possibility of estimating both $\beta$ and $\theta$, we
conducted a small simulation study, generating 500 datasets from the
model with both parameters set to 1, which corresponds to the
conditions described above. Rather incredibly, we see that it is
possible to estimate both parameters with high accuracy
(Fig~\ref{ch9.fig.sim}).

\begin{figure}[ht]
\centering
\includegraphics[width=4in,height=2in]{Ch11/figs/scrDEDsim}
\caption{Histograms of parameter estimates from 500 simulations under
  the model in which both density and ecological distance are affected
by the same covariate, elevation. The vertical lines indicate the
data-generating value.}
\label{ch9.fig.sim}
\end{figure}



\section{The jaguar data}

Estimating density of large felines has been a priority for many
conservation organizations, but no robust methodologies existed before
the advent of SCR. Distance sampling is not feasible for such rare and
cryptic species, and traditional capture-recapture methods yield
estimates that are highly sensitive to the subjective choice of the
effective survey area. In this example, we
demonstrate how readily density can be estimated for a
globally imperiled species using SCR. Furthermore, we show how
inhomogeneous point process models can be used to test important
hypotheses regarding the ecological factors affecting density.

In this example, we make use of a single year of data from an 8-year
camera-trapping study of jaguars in Argentina,
along the borders with Brazil and Paraguay. The data come from 46
camera stations, each consisting of a pair of cameras placed along
roads or trails. Forty-five detections of 16 jaguars (8 males and 8
females) were made over a 95-day sampling period. The mean number of
sampling days at each camera station was 48.2.

Estimating density is a central objective of this study because
ultimately, an estimate of the total population size for the entire
study area is needed, which can only be obtained by extrapolation of
density estimates. A second, and related, objective was to assess
the influence of poaching on jaguar density. Although jaguars
themselves are occasionally killed by poachers, the larger concern is
the influence of poaching on prey species. To protect jaguars and
related species, protected areas have
been established and three levels of protection are
recognized in the study region as depicted in Fig.~\ref{ch9.fig.jaguarCts}.

\begin{figure}[ht]
\centering
\includegraphics[width=3in,height=3in]{Ch11/figs/jaguarCountMap}
\label{ch9.fig.jaguarCts}
\caption{Jaguar detections at 46 camera trap stations. The three levels of
  protection status are no protection (beige), some protection (light
  green), and national park (dark green). Non-habitat is shown in gray
  and represents large soybean monocultures. }
\end{figure}

To assess the influence of poaching on jaguar density, we treated
protection status as an ordinal variable with 3 levels: no protection,
some protection, and high protection (national parks). Clearly these
are ordered, and our
hypothesis is that poaching pressure should decrease and jaguar
density should increase with the level of
protection. Thus, $\beta$ in this example is a ``slope''
parameter describing the degree to which protection status affects
jaguar density. We also hypothesized that males and females could have
different home range sizes and that the sex ratio may not be
1:1. Furthermore, we restricted the state-space to exclude the large
soybean monocultures surrounding the study area, and we only
considered
area south of the Iguazu River, which runs along the northern border
of the park shown in dark green in
Fig.~\ref{ch9.fig.jaguarCts}. Rather than restricting the
state-space, we could have modeled the permeability of the river using
the methods described in the previous section and in
Chapter~\ref{chapt.ecoldist}; however, no sampling was conducted on
the northern side of the river, and ancillary data indicates that
jaguars very rarely forge the waterway. \R~code to fit the model is
available in \scrbook  on the help page \verb+jaguarDataCh9+. Parameter
estimates are shown in Table\ref{ch9.tab.jagposts}.
\begin{table}
\centering
\caption{Summaries of posterior distributions from the model of jaguar
  density. $\sigma_f$ and $\sigma_m$ are the scale parameters of
  the half-normal detection function for females and males
  respectively. $\rho$ is the
  sex-ratio. $\lambda_0$ is base-line encounter rate. $\beta$ is the
  effect of protection on jaguar density. D is the overall density
  estimate. D1, D2, and D3 are the density estimates
  (jaguars/100km$^2$) for the three levels of protection. }
\begin{tabular}{lrrrrr}
\hline
& Mean & SD & 2.5\% & 50\% & 97.5\% \\
\hline
 $\sigma_f$ &  7361.731 &  1907.566 &  4899.740 &  7002.770 & 12083.110 \\
 $\sigma_m$ &  8177.068 &  1545.717 &  5916.151 &  7955.788 & 11842.486 \\
 $\rho$ &     0.516 &     0.118 &     0.286 &     0.516 &     0.741 \\
 $\lambda_0$ &     0.007 &     0.002 &     0.003 &     0.007 &     0.012 \\
 $\beta$ &     4.405 &     1.443 &     2.553 &     4.143 &     7.775 \\
 D &     0.533 &     0.708 &     0.000 &     0.000 &     0.072 \\
 D1 &     0.132 &     0.010 &     0.095 &     0.095 &     0.616 \\
 D2 &     1.415 &     0.050 &     0.214 &     0.531 &     1.503 \\
 D3 &     3.516 &     0.000 &     0.292 &     3.105 &     4.220 \\
\hline
\end{tabular}
\label{ch9.tab.jagposts}
\end{table}

Our results
indicate that efforts to protect jaguars by reducing poaching are
working. Density was $>$26 times higher in the national park than in the
unprotected area. Fig.~\ref{ch9:fig:Dsurface} shows the estimated
density surface.

\begin{figure}[ht]
\centering
\includegraphics[width=3in,height=3in]{Ch11/figs/Dsurface34}
\label{ch9:fig:Dsurface}
\caption{Estimated density surface for the jaguar dataset}
\end{figure}


We note that there is room for improvement in our analysis. The
political boundaries used to demarcate protected areas are not as
concrete as we might like. In reality poaching pressure is likely to
be higher near remote park boundaries than in well-guarded park
interiors. One option
for addressing this would be to use a continuous measure of poaching
pressure such as distance from the nearest town, or some other
accessibility metric. It would also be interesting to model density
separately for each sex. Many of the detections outside of the park
were of males, and thus it is possible that the sexes use habitat
differently. Developing models for these two hypotheses could be
readily accomplished using slight modifications of the code found in
the \R~package \scrbook.



\section{Summary}

When state-space covariates are available,
density can be modeled by replacing the uniform prior on the activity
centers with a
prior based on a normalized log-linear function of covariates. This
distribution has been widely used in ecology to model point processes
as well as resource selection probability functions
\citep{manly_etal:2002,lele_keim:2006}. In the SCR
context, use of this new prior results in
a model for the inhomogeneous point process describing the
location of activity centers, which can be used to test hypotheses
about spatial variation in density. In
rare cases, these covariates are truly continuous in the sense that
they are defined as a function of space. More often, covariates are
represented as rasters, which simplifies the analysis. Fitting these
models can be accomplished using \bugs, \secr, or the custom \R~code
presented in this chapter and found in the package \scrbook.
%However,
%at the time this book was written, \scrbook is only software available
%for fitting models with covariates of both density and ecological
%distance.

All the examples in this section included a single state-space
covariate, but this was for simplicity only. Including multiple
covariates poses no additional challenges. Similarly, additional model
structure such sex-specific encounter rate parameters or behavioral
responses can be accommodated. Even more remarkable is the ability to
consider covariates that affect both density and ecological
distance. The ramifications of this are enormous for applied
ecological research and conservation efforts because, for instance,
researchers can use capture-recapture data to identify areas where
density is high, and to model important quantities such as landscape
connectivity \citep{royle_etal:2012ecol}. Addressing such questions
is simply not possible using standard, non-spatial capture-recapture
methods. Accomplishing these goals will of course require more data
than is needed to estimate the parameters of a basic SCR model.



%\chapter{Inhomogeneous Point Process}
%\label{chapt.ipp}

\chapter{Open models}
\label{chapt.open}




\chapter{Spatial Capture-Recapture for Unmarked Populations}
\markboth{Chapter 14 }{}
\label{chapt.scr-unmarked}

\vspace{0.3cm}


Traditional capture-recapture models share the fundamental
assumption that each individual in a population can be uniquely
identified when captured. This can often be accomplished
by marking individuals with color bands, ear tags, or some other
artifical mark that can be read in the field. For other species, such as
tigers or marbled salamanders, individuals can be easily identified
using only their natural markings. In a great number of cases, however,
species do not possess sufficient natural markings and are too
difficult to capture to make it practical to apply artifical marks. So
we must throw up our hands and not study these species. End of
chapter.

When capture-recapture methods are not a viable option, researchers
often collect simple count data or even detection/non-detection data
to estimate population parameters. These data are often analyzed using
Poisson regression or logistic regression, perhaps with random
effects; but when detection is imperfect, as it almost always is,
these methods cannot be used to obtain unbiased estimates of
population size or occurrence probability. Even when these data are
used an index of abundance or occurrence, standard models may yield
unreliable results when covariates affect both the state variable and
detection probability. A classic example is the finding by
\citet{bibby_buckland:1987} who reported that the probability of detecting
songbirds in restocked confier plantations decreased with vegetation
height; whereas population density was positively related to
vegetation height. This intuitive and common phenomenon has led to the
development a vast number of models to estimate population size and
detection probability when individuals are unmarked. A review of these
models is beyond the scope of this 
chapter, but we mention a few deficiencies of existing methods
that warrant the exploration of alternatives for robust inference when
standard capture-recapture methods do not apply.

Distance sampling, which we briefly introduced in chapter XXXX,
is perhaps the most widely used method for
estimating population density when individuals are unmarked and
detection probability is less than one. This class of methods is known
to work impecibly when estimating the number of stakes in a field or
the number of duck nests in a wetland. It can also work very well in
more interesting situations; however, %In many other situations,
common issues such as animal movement and measurement error may result in
substantial bias. In addition, traditional distance sampling methods
assume that individuals are randomly located with respect to the
observer and are available for detection (but see
\citet{johnson_etal:2010,chandler_etal:2011}). % Add ISSJ paper too
Most other
methods, such as double-observer sampling and repeated counts, can be
used to estimate population size, but as with traditional CR methods,
it may be difficult to covert abundance estimates to
density estimates because the effective area sampled is unknown. We
mention these issues not to suggest that existing models do not have
value---
indeed we believe that they can be used to obtain reliable density
estimates in many situations---rather our aim is to highlight the need for
alternative methods when the assumptions of existing methods cannot be
met. Additionally, the model we develop in this chapter serves as the
foundation for a broad class of SCR models in which all or some of the
individuals cannot be uniquely identified.

In this chapter we highlight the work of \citet{chandler_royle:2012}
who demonstrated that the ``individual recognition'' assumption of
CR models is not a requirement of spatial capture-recapture
models. They showed that spatial correlation alone is sufficient for
making inference about animal distribution and density. That is, if
we simply have spatially-correlated
count data at a collection of survey points, we can estimate density
even if all individuals are unmarked, assuming that the underlying SCR
model is valid. The details of how this is
accomplished is the subject of this chapter. 

The ability to fit
SCR models to data from unmarked populations has important
consequences in several respects. For one, it means that SCR models can
be applied to data collected using methods like points counts in which
observers record simple counts of animals at an array of survey
points. This development also has important implications for
traditional SCR studies because many resulting datasets include some
individuals that cannnot be identified due to poor photo quality or
the indistiguishable natural markings.


\section{Spatial correlation as Information}


Imagine a 10 $\times$ 10 grid of camera traps and a single individual
exposed to capture whose home range center lies in the center of the
trapping grid. If the individual has a small home range size relative
to the extent of the trapping grid, we can imagine what the
spatial correlation structure of the encounters might look
like. If the animal's movement is symmetric around the activity center
then the number of times the individual is detected at each
trap (the trap counts) is a function of the distance between the home
range center and the trap, and so traps with the same distance from the
activity center will be counts that are more highly correlated with
one another than traps located at different distances from the
activity center. Thus, the correlation in counts tells us something
about the location of the activity center. That is, correlation
carries information about distribution. What about density?

Imagine now that there are two activity centers located in our traping
grid. Using trap counts alone, can our model tell us both where the
activity centers are and how many exist in the population exposed to
capture? The answer is yes, at least under certain circumstances. 
EXPLAIN.

This heuristic is useful for understanding the model proposed by
\citet{chandler_royle:2012}. We will now formalize these concepts and
describe practical issues that arise when applying SCR
models to data from unmarked populations.



\section{Data Requirements and Survey Designs}





\section{Encounter Histories as Latent Variables}

Just when you thought we ran out of things to treat as latent
variables, we are now going to regard even the data itself as latent.


State model is the same as other SCR models.


It is natural to regard the encounter rate of an individual
as a function of the Euclidean distance between the individual's
activity center and the trap location, $d_{ir} = \| {\bf x_r} - {\bf
  s_i} \|$.
To be precise about this, we let $z_{irt}$ be the encounter frequency
of
individual $i$ in trap $r$ during occasion $t$. While we will adopt the view
that  the variables $z_{irt}$ are latent variables (see below), it will
be convenient to formulate the model in terms of these variables.

Therefore, we assume that the expected encounter frequency of an
individual in some trap is related to $d_{ir}$ as follows:
\[
E[z_{irt}] = \lambda_{ir} = \lambda_0 k_{ir}
\]
where $\lambda_0$ is the expected encounter rate at $d=0$ and $k_{ir}$
is some positive-valued
function of distance $d_{ir}$. We assume
\[
k_{ir} = exp(-d_{ir}^2 / 2\sigma^2)
\]
where $\sigma$ is a scale parameter related to home
range size. $\sigma$ also determines the degree of correlation among
counts since animals with large home ranges are more likely to be
detected at multiple traps relative animals with small home ranges.
The phenomenon is analogous to correlation induced by averaging
spatial noise, in which case there is a unique correlation between the
smoothing kernel and the induced covariance function
\citep{higdon:2002}.

We emphasize that our focus is on
situations in which individuals are {\it not}
uniquely identifiable, and therefore the encounter frequencies
for each individual
cannot be observed, and so they are latent variables. We assume that
these latent variables are realizations from a Poisson distribution
with mean $\lambda_{ir}$:
\begin{equation}
 z_{irt} \sim \mbox{Poisson}(\lambda_{ir}).
\label{eq.latentPoisson}
\end{equation}
In traditional SCR models, $z_{irt}$ are the observed data, {\it
  i.e.}, the frequency of encounters of individual $i$ at trap $r$ on
replicate survey $t$. However, when individual identity is not known,
the observed data are the sample- and trap-specific totals,
aggregated over all individuals:
\[
n_{rt} = \sum_{i=1}^{N} z_{irt}.
\]
Thus the data required by our model are a reduced-information
summary of the latent encounter histories.


Under the Poisson encounter model we have that
\begin{equation}
n_{rt} \sim \mbox{Poisson}( \Lambda_{r} )
\label{eq:nagg}
\end{equation}
where
\[
 \Lambda_{r} = \lambda_{0} \sum_{i} k_{ir}.
\]
Further, because $\Lambda_{r}$ does not depend on $t$, we can
aggregate the replicated counts, defining
$n_{r.} = \sum_{t} n_{rt}$ and then
\[
 n_{r.} \sim \mbox{Poisson}( T \Lambda_{r} )
\]
As such, $T$ and $\lambda_{0}$ serve equivalent roles as affecting
baseline encounter rate.
This formulation of the model in terms of the aggregate count
simplifies computations as the latent variables
$z_{irt}$ do not need to be updated in the MCMC estimation
scheme (see below). However, retaining $z_{irt}$
in the formulation of the model
is important if some individuals are uniquely marked, in which case
modifying
the MCMC algorithm (see below) to include both types of data is
trivial. This is because uniquely identifiable individuals produce
observations of some of the $z_{irt}$ variables.

We imagine that other observation models
might be possible (see Discussion) although we focus here on the
Poisson encounter model because it has considerable relevance to
animal surveys, and has additional methodological context related to
point process models which we address in the Discussion.





\section{Estimation by MCMC}
\label{s:mcmc}

We adopt a Bayesian framework for inference allowing estimation of $N$
while retaining the formulation of the model that is conditional on
the latent activity centers $\bf s_i$.
Specifically, we employ Markov chain Monte Carlo
(MCMC) to simulate posterior distributions of the parameters. However,
the fact that $N$ is unknown presents a
technical challenge because the size of the parameter space can change
with each MC iteration. To resolve this, we
adopt the formulation of data augmentation in \citet{royle_etal:2007} who
used a specific prior construction for $N$ in terms of individual level
Bernoulli trials. In particular, we assume $N \sim \mbox{Unif}(0,M)$
for some large integer $M$. We construct this prior by assuming
$N|M,\phi \sim \mbox{Bin}(M,\phi)$ and $\phi \sim \mbox{DUnif}(0,1)$
which implies, marginally, that $N$ has the requisite
$\mbox{DUnif}(0,M)$ distribution. However
the hierarchical formulation of the prior suggests an implementation
in which we introduce a set of latent indicator variables $w_{i} \sim
\mbox{Bern}(\phi)$ and, furthermore, the model implies
that $z_{irt}$ are obtained
from the specified distribution (Eq. \ref{eq.latentPoisson})
if $w_{i} = 1$, or if
$w_{i}=0$, %the model implies that
$z_{irt} =0$ with probability 1. In
effect, extending the model in this way induces a reparameterization
for the latent counts %$z_{irt}$
that is a zero-inflated version
of the original conditional-on-$N$ model. Specifically, the model
under
data augmentation becomes
\begin{eqnarray*}
 z_{irt}|w_{i} &\sim & \mbox{Poisson}(\lambda_{ir} w_i) \\
 w_{i} & \sim & \mbox{Bern}(\phi)
\end{eqnarray*}
Under this formulation $N = \sum_{i=1}^{M} w_i$, and population
density is simply $D = N/A({\cal S})$ where $A({\cal S})$ is the area of the
point process state-space ${\cal S}$.

We developed two distinct MCMC implementations for this model (\ref{suppA}). In the
first, we devised an algorithm for the model conditional on the latent
variables $z_{irt}$. This formulation is useful for problems in which
one or more individual identities are available, in which case the
$z_{irt}$ are observable for those individuals. The unobserved
$z_{irt}$ are easily updated using their full-conditional
distribution which is multinomial with sample size $n_{rt}$. The
remaining parameters are updated using Metropolis-Hastings steps (see
\ref{suppA}).  In the second formulation of the algorithm we applied
the Metropolis-Hastings algorithm to the model {\it unconditional} on
the $z_{irt}$ variables. In that case, the marginal distribution for
$n_{rt}$ is precisely Eq.~\ref{eq:nagg}.  This algorithm is slightly more
convenient because
it avoids having to update the $z_{irt}$ variables of which there are many.




\section{Northern Parula Example}



To apply our model to data collected in the field, we designed a point
count study of the northern parula ({\it Parula americana}), a
Neotropical-Nearctic migratory passerine. This species defends
well-defined territories during the breeding season
\citep{moldenhaer_regelski_1996}, and thus our modeling effort was focused
on estimating the number and location of territory centers. Points
were located on a 50-m grid to ensure spatial
correlation. This small grid spacing contrasts with the conventional
practice of spacing points by $>$ 200 m to obtain \emph{i.i.d.}
counts. Figure~\ref{fig:nopaDat} depicts the spatially-correlated
counts ($n_{r.}$) from the 105 point count locations
surveyed three times each during June 2006
at the Patuxent Wildlife Research Center in Laurel Maryland, USA.
A total of 226 detections were made with a maximum count of 4 during a
single survey. At 38 points, no warblers were detected. All but one of
the detections were of singing males, and this one observation was
not included in the analysis.


%\begin{comment}

\begin{figure}
  \centering
  \includegraphics[width=3in,height=2.25in]{Ch14/figs/nopa}
  \caption{Spatially-correlated counts of northern parula on a 50-m
    grid. The size of the circle represents the total number of
    detections at each point.}
  \label{fig:nopaDat}
\end{figure}

%\end{comment}


In our analysis of the parula data, we defined the point process
state-space by buffering the grid of point
count locations by 250 m and used $M=300$. We simulated posterior
distributions using three Markov chains,
each consisting of 300000 iterations after discarding the initial 10000
draws. Convergence was satisfactory, as indicated by an $\hat{R}$
statistic of $<$ 1.02 \citep{gelman_rubin:1992}.

One benefit of a Bayesian analysis is that it can accommodate prior
information on the home range size and encounter rate parameters,
which are readily available for many
species. To illustrate, we analyzed the parula data using two sets of
priors. In the first set, all priors were
improper, customary non-informative priors (see Table \ref{t:nopaPosts}).
Uniform priors were also used in the second set, with the exception of
an informative prior for the scale parameter $\sigma \sim
\mbox{Gamma}(13,10)$. We arrived at this prior using the methods
described by \citet{royle_etal:2011mee} and published
information on the warbler's home range size and detection probability
\citep{moldenhaer_regelski_1996,simons_etal:2009}. More details on this
derivation are found in \ref{suppA}. We briefly note here that this prior
includes the biologically-plausible range of values from $\sigma$
suggested by the published literature.

The posterior distribution for
$N$ was highly skewed with a long right tail resulting in a wide 95\%
credible interval (Table \ref{t:nopaPosts}). Nonetheless, the interval
for density, $D$, includes estimates reported from more intensive field
studies \citep[][]{moldenhaer_regelski_1996}. This was true when
considering
both sets of priors, although posterior precision was higher under the
informative set of priors. Specifically, the use of prior information
reduced posterior density at high, biologically implausible,
values of $\sigma$, and hence decreased the posterior mass for
low values of $N$ (Fig.~\ref{fig:prior}).

In addition to estimating density, our model can be used to produce
density surface maps, which are often used in applied ecological
research to direct management efforts and develop hypotheses regarding
the factors influencing abundance.
Density surface maps can be produced by discretized the
state-space and tallying the number of activity centers occurring in
each pixel during each MCMC iteration. Parula density was
highest near the northeastern corner of the study plot, which may
correspond to important habitat features such as suitable nest site
locations (Fig.~\ref{fig:nopaDen}). We anticipate future model
extensions to directly model the
point process intensity using habitat covariates.


\begin{table}%[t]
  \caption{Posterior summary statistics for spatial Poisson-count
    model applied to the northern parula data. Two sets of priors were
    considered. $M=300$ was used in both cases. Parulas/ha, $D$, is a
    derived parameter.}
  \scriptsize
  \begin{tabular}{l l rrrrrr}
    \hline
    Par        & Prior                  & Mean  & SD    & Mode   & q0.025  & q0.50  & q0.975  \\
    \hline
    $\sigma$   & $U(0, \infty)$   & 2.154   & 1.222  & 1.230   & 0.896   & 1.665   & 5.170    \\
    $\lambda_0$ & $U(0, \infty)$  & 0.284   & 0.149 & 0.212    & 0.084  & 0.256  & 0.665   \\
    $N$        & $U(0, M)$             & 40.953   & 38.072  & 4.000  & 3.000       & 31.000     & 143.000     \\
    $D$        &  --                   & 0.427    & 0.397 & 0.0417   & 0.0313  & 0.323  & 1.490    \\
    \hline
    $\sigma$    & $G(13, 10)$          & 1.301    & 0.258 & 1.230    & 0.889   & 1.266   & 1.908    \\
    $\lambda_0$ & $U(0, \infty)$ & 0.298    & 0.132 & 0.240    & 0.098   & 0.279  & 0.603   \\
    $N$         & $U(0, M)$            & 59.321   & 36.489  & 36.000 & 18.000      & 50.000     & 157.000     \\
    $D$         &  --                  & 0.618    & 0.380 & 0.375   & 0.188   & 0.521  & 1.635    \\
    \hline
  \end{tabular}
  \label{t:nopaPosts}
\vspace{0.5cm}
\end{table}


%\begin{comment}

\begin{figure}
  \centering
  \includegraphics[width=1.5in,height=3in]{Ch14/figs/prior} % was 3,7
  \caption{Effects of $\sigma \sim \mbox{Gamma}(13,10)$
    prior on the posterior distributions from the northern parula
    model. Posteriors from model with uniform priors are
    shown in black, and posteriors from the informative prior model
    are shown in gray. The prior itself is shown as dotted line in the
    upper panel.}
  \label{fig:prior}
\end{figure}




\begin{figure}
  \centering
  \includegraphics[width=3in,height=2.25in]{Ch14/figs/nopaDen}
  \caption{Estimated density surface of northern parula activity
    centers. The grid of point count locations with count totals is
    superimposed. See Fig. 1 for additional details.  }
  \label{fig:nopaDen}
\end{figure}

%\end{comment}



\section{On (Im)precision}





\section{How Much Correlation Is Enough?}



\section{Mutants}

\subsection{Other observation models}

\subsection{Linear designs}




\section{Summary}








In this paper, we confronted one of the most difficult challenges
faced in wildlife sampling ---
estimation of density in the absence of data to distinguish among
individuals. To do so, we developed a novel class of
spatially-explicit models that
applies to spatially organized counts, where the count locations or
devices are located sufficiently close together so that individuals
are exposed to encounter at multiple devices. This design yields
correlation in the observed counts, and this correlation proves to be
informative about encounter probability parameters and hence density.
We note that sample locations in count-based studies are typically
{\it not} organized close
together in space because conventional wisdom and standard practice
dictate that independence of sample units is necessary
\citep{hurlbert:1984}. Our model
suggests that in some cases it might be advantageous to deviate from
the conventional wisdom if one is interested in direct inference about
density. Of course, this is also known in the application of standard spatial
capture-recapture  models \citep{borchers_efford:2008}
where individual
identity is preserved across trap encounters, but it is seldom, if
ever, considered in the design of more traditional count surveys.

Our model has broad relevance to an incredible number of animal
sampling problems. Our motivating problem involved bird point counts
where individual
identity is typically not available. The model also applies
to other standard methods used to sample unmarked
populations,  such as camera traps
or even methods that yield sign ({\it e.g.} scat, track) counts
indexed by space. However, results of our simulation study reveal some
important limitations of the basic
estimator applied to situations in which none of the individuals can
be uniquely identified. In particular, posterior
distributions are highly skewed in typical small to moderate sample
size situations and posterior precision is low.

Several modifications of the model can lead to improved
performance of the estimator.
Our simulation results demonstrate that marking a subset of
individuals can yield substantial increases in posterior
precision. Marking a subset of individuals is
commonplace is animal studies such as when a small number of individuals are
radio-collared in conjunction with a count-based survey
\citep{bartmann_etal:1987}. In many other situations a subset of
individuals can be identified by natural marks alone, and thus our
model could be applied to data from camera-trapping studies of
species such as mountain lions, deer, coyotes for which traditional
SCR methods are not effective \citep{kelly_etal:2008}.
Thus, the ability to study partially-marked populations
adds flexibility to existing SCR methods, and also
creates new opportunities for designing efficient SCR studies
since the costs of marking all individuals in a population can be
prohibitive.

We note the existence of traditional approaches to combining data on
marked and unmarked animals based on either the Lincoln-Peterson
estimator or so-called ``mark-resight'' methods.
\citep{bartmann_etal:1987, mintaMangel:89, mcclintockHoeting:09}. In their
simplest form, mark-resight methods involve fitting standard
closed-population mark-recapture models to the data on marked
individuals, and the resultant estimate of detection probability
($\hat{p}$) is used to estimate population size as $\hat{N} = m +
u/\hat{p}$ where $m$ and $u$ are the number of
marked and unmarked individual, respectively. In this case,
the unmarked individuals provide no information about the
encounter rate parameters, and thus mark-resight methods cannot be
used unless a large sample of marked individuals is available. This
contrasts with our approach which can be used even when all
individuals are unmarked.

In some cases, such as in point counts of birds, it may not be
practical to mark individuals. An alternative to increasing posterior
precision is to utilize prior information on
home range size. Indeed, extensive information on home range size has
been compiled for many species in diverse habitats %\emph{e.g.}
\citep[\emph{e.g.},][]{degraaf_yamasaki:2001}. It is
easy to embody this information in a prior distribution as we
demonstrated for the parula data.

An additional design extension that could increase precision is to use
multiple sampling methods, in which one method generates encounter
frequencies and the other method generates individuality.
For example, camera traps are now commonly used with surveys for
sign (scat or tracks), or hair snares for sampling bear populations.
These distinct methods would have different basal detection
rates but share an underlying spatial model describing the
organization of individuals in space.
Our models show promise for using
these disparate data types efficiently
for estimating density.




\subsection{Alternative Observation Models}
\label{ss:ext}

Several aspects of our ``spatial $N$-mixture model'' can be modified
to accommodate
alternative sampling designs or parametric distributions.
We considered situations where an individual can be detected more than
once at a trap during a single occasion, but under some designs this
is not possible. When collecting DNA samples, for instance, an
individual can often be detected at most once during an
occasion, because multiple samples of biological material cannot be
attributed
to distinct episodes. Therefore, rather than $z_{irt} \sim Poisson(\lambda_{ir})$
we have $z_{irt} \sim Bernoulli(p_{ir})$ where, for example,  $p_{ir} = p_0
exp(-d_{ir}^2/(2\sigma^2))$, and $p_0$ is the probability of
detecting an individual whose home range is centered on trap $r$. This
Bernoulli model is a focus of ongoing investigations.

Both the Poisson and the Bernoulli models
produce count observations when aggregated over individuals to form
trap-specific totals; however, ecologists often collect so-called
``detection/non-detection'' data because it can be easier to determine
if ``at least one'' individual was present rather than enumerating all
individuals in a location. In this case, the underlying $z_{irt}$
array is the same as the above cases, but we observe $y_{rt} =
I(\sum_{i=1}^{N} z_{irt} > 0)$ where $I$ is the indicator
function. This ``Poisson-binary model'' is
a spatially explicit extension of the model of
\citet{royle_nichols:2003} in which the underlying abundance state
is inferred from binary data. We have investigated this model to a
limited extent but do not report on those results here.


\subsection{Spatial point process models}
\label{ss:similar}

Our model has some direct linkages to existing point process
models. We note that the observation intensity function (i.e.,
corresponding to the observation
locations) is a compound Gaussian kernel similar to
that of the Thomas process
\citep[pp. 61-62]{thomas:1949, moller_waagepetersen:2003}.
Also, the Poisson-Gamma Convolution models
\citep{wolpert_ickstadt:1998} are structurally similar (see also \cite{higdon:1998}
and \cite{best_etal:2000}).
 In particular, our model is such a model but
with a {\it constant} basal encounter rate $\lambda_{0}$
and {\it unknown} number and location of ``support points'', which in
our case are the animal activity centers, $\bf{s_i}$.
We can thus regard our model as a model for
{\it estimating} the location and local density of support points in
such models, which we believe could be useful in the application of
convolution models.  \citet{best_etal:2000} devise an MCMC algorithm for the
Poisson-Gamma model based on data augmentation, which is
similar to the component of our algorithm for
updating the $z$ variables in
the conditional-on-$z$ formulation of the model.  We emphasize that
our model is distinct from these Poisson-Gamma models
in that the number {\it and} location of such
support points are estimated.


If individuals were perfectly observable then the resulting point
process of locations is clearly a standard Poisson or Binomial (fixed
$N$) cluster process or Neyman-Scott process.
If detection is uniform over space but
imperfect, then the basic process is unaffected by this random thinning.
Our model can therefore be viewed formally as a Poisson (or Binomial)
cluster process model but one in which the thinning is
non-uniform, governed by the encounter model which dictates that
thinning rate increases with distance from the observation points. In
addition, our inference objective is, essentially, to estimate the
number of parents in the underlying Poisson cluster
process,
where the observations are biased by an incomplete sampling apparatus
(points in space).


As a model of a thinned point process, our model has much in common
with classical distance sampling models \citep{buckland_etal:2001}.
The main distinction is that our data structure does {\it not} include
observed distances, although the underlying observation model is
fundamentally the same as in distance sampling if there is only a
single replicate sample and $\bf{s}_i$ is defined as an individual's
location at an instant in time. For replicate samples, our model preserves
(latent) individuality across samples and traps which is not a feature
of distance sampling. We note that error in measurement of distance is
not a relevant consideration in our model, and we explicitly do not
require the standard distance sampling assumption that the probability
of detection is 1 if an individual occurs at the survey point. More
importantly, distance sampling models cannot be applied to data from
many of the sampling designs for which our model is relevant. For
example, many rare and endangered species can only be
effectively surveyed using methods such as hair snares and camera
traps that do not produce distance data \citep{oconnell_etal:2010}.


\section{Conclusion}

Concerns about ``statistical independence'' have prompted
ecologists to design count-based studies such that observed
random variables can be regarded as {\it i.i.d.} outcomes
\citep{hurlbert:1984}. Interestingly, this
often proves impossible in practice, and elaborate
methods have been devised to model spatial dependence as a nuisance
parameter. Our paper presents a modeling framework that directly
confronts this view by demonstrating that spatial
correlation carries information about the locations of individuals,
which can be used to estimate density even when individuals
are unmarked and distance-related heterogeneity exists in encounter
probability.




\chapter{
Spatial capture-recapture models for partially identifiable
populations: Spatial mark-resight models
}
\markboth{Spatial mark-resight models}{}
\label{chapt.partialID}

\vspace{.3in}


So far, this book has dealt with the situation where all detected
individuals are identifiable, and in Chapt. \ref{chapt.scr-unmarked}
we introduced and developed an SCR model for non-identifiable
populations, a spatial {\it non}-capture-recapture model, if you will. These
two extremes are common in the study of animal populations with
non-invasive sampling methods. However, there is also an intermediate
situation, where a part of the population is tagged or otherwise
marked and can thus be identified upon recapture, while the untagged
portion remains unidentified. In this situation so-called mark-resight
models \citep{bartmann_etal:1987, arnason_etal:1991, neal_etal:1993}
can be used to estimate population size and density combining data
from both the marked and unmarked individuals.

Traditionally, capture-recapture studies involved physical capture of
individuals throughout the study; new individuals are marked on every
re-capture occasion. This methodology is still widely applied to small
mammals, but can be very costly, logistically challenging and risky
when dealing with larger species. In contrast, in mark-resight studies
a sample of individuals is captured and tagged (or otherwise marked)
during a single marking event. Marking is followed by resighting
surveys, upon which both the detection of marked and recognizable
individuals and unmarked animals is recorded. Resighting surveys are
usually non-invasive (hence the name �resighting�), so that they
don't involve handling of animals. As such, mark-resight models have a
major advantage over traditional capture-recapture models in that they
only require individuals to be captured and handled once, during the
initial marking. This reduces field costs and risks for the animals
(and potentially the researchers).

Mark-resight models have a set of underlying assumptions, most of
which are analogous to those for mark-recapture models,
e.g. demographic population closure (violation of geographic
population closure can be accommodated by some models) and no loss or
misidentification of marks. Just like regular capture-recapture
models, there are means to incorporate heterogeneity in capture
probability. However, a new and essential assumption of mark-resight
models is that the tagged (or otherwise identifiable) individuals are
a representative sample of the study population, so that inference
about individual detection can be made for the whole population from
the tagged sample. This issue is usually addressed by using a
different method for marking than for resighting, and by marking a
random sample of the population.

Owing to the advantages of mark-resight over capture-recapture,
especially when dealing with hard-to-trap species, mark-resight is a
popular tool in wildlife population studies. The method has been
applied for decades and to a suite of species and survey techniques,
ranging from banding and resighting Canada geese
\citep{hestbeck_malecki:1989} to ear-tagging and camera-trapping
grizzly bears \citep{mace_etal:1994} to paintball marking and areal
resightings of large ungulates \citep{skalski_etal:2005}.

\subsection{Types of partial ID data}

Before we start exploring mark-resight approaches in more detail, we
need a clear understanding of what types of mark-resight data we can
have, in order to appreciate and understand the different flavors of
mark-resigh models.  In general, we have (at least) two sets of data:
encounter histories for identifiable individuals $i$ at trap $j$ and
occasion $k$, $y_{ijk}$, and counts of unidentified records for each
$j$ and $k$, $n_{jk}$. Depending on the sampling technique, we can
conceive of three slightly different types of partial ID data.



If you implement your resighting survey shortly after the marking
session, you may be confident that none of the marked individuals has
died or lost its mark. Under these circumstances you know that the
number of marked individuals available for resighting, $m$, is equal
to the number of individuals you tagged. Alternatively, tags might be
radio-transmitters, allowing you to confirm the presence or absence of
marked individuals in the resighting survey area using radio-telemetry
\citep{white_shenk:2001}. In both cases, you know the number of marked
individuals in the population you survey.

In this situation, even though you may fail to resight some of the
tagged individuals, since you know how many there are, you can simply
assign those you never resighted all-zero capture histories - in other
words, contrary to regular capture-recapture models, in mark-resight
models with a known number of tagged individuals, we can observe
all-zero encounter histories. Under these circumstances, estimating
$N$ reduces to estimating the number of unmarked individuals, $U$.

If we suspect that some of the marks may have been lost between
tagging and conducting the resighting samples, we obtain a slightly
different type of mark-resight data. Here, we do not know the accurate
number of marked individuals available for resighting. As a
consequence, individuals have to be resighted at least once for us to
know they are still tagged and alive and thus available for
resighting. So, contrary to the situation where we know $m$ and
analogous to regular capture-recapture models, we cannot observe
all-zero encounter histories of marked individual. Here, estimating
$N$ involves both estimating $m$ and $U$.

A special case of this kind of data can arise from camera
trapping. Even when dealing with a species that has no spots or
stripes, some individuals in the study population can have natural
marks that make them identifiable on pictures, such as scars or some
distinct coloration. Again, in this scenario an individual has to be
photographed at least once to be known. Here, the fact that both the
`marking� method and the subsequent resighting method are the same
(although marking in this case does not involve any actual physical
marking) can be cause for concern: our sample of `marked�
individuals may not be a random sample of the population but consist
of individuals that for some reason are more likely to be
photographed. In that case, a basic assumption of the mark-resight
model is violated.

Finally, consider a scat or hair snare survey, where only a part of
the samples are analyzed genetically (or DNA can only be extracted
from a subset of samples due to sample quality). In this scenario,
your $n_{jk}$ can contain both completely unknown individuals that are
not represented at all in {\bf $Y$}, but it can also contain samples
from individuals that we previously identified. The difference is that
in the first two scenarios, part of the population of individuals is
identifiable, while in the second scenario, part of the
samples is identifiable. This type of data
actually violates one of the basic assumptions of mark-resight models,
namely, that tagged individuals are always correctly identified as
such. To our knowledge there are currently no mark-resight models
available that account for possible misidentification of the marking status of individuals (although there some literature is available on misidentification of individuals in capture-recapture study, e.g., \citealp{yoshizaki_etal:2009, lukacs_burnham:2005, link_etal:2010}). In this chapter we will ignore this kind of data and focus instead
on the two types of typical mark-resight data:

\begin{itemize}
\item[(1)] Known number of tagged individuals 
\item[(2)] Unknown number of tagged individuals, 
\end{itemize}

For both types of data a slightly different situation arises when in some instances we can only tell that an individual is tagged, but not who it is. You may be able to see that an individual is tagged but the identifying feature of the tag (a number or coloration) may have become unreadable, or may be hidden from view. In this case, in addition to your $y_{ijk}$ and your $n_{jk}$ you also have a number of sightings of tagged but unidentified individuals, say $r_{jk}$. 

\subsection{A short history of mark-resight models}

Initially, mark-resight methods focused on radio-tagged individuals to
estimate population size \citep{white_shenk:2001}. Radio-collars
provide a means of determining which of the animals were in the study
area and available for sampling, i.e. determining the number of marked
individuals in the population. Knowing this number was a prerequisite
for most earlier mark-resight approaches \citep{white:1996}. The
oldest mark-resight model is the good old Lincoln-Petersen estimator,
 where individuals are marked and a single resight/recapture occasion is carried out \citep{krebs:1999}. We need not identify individuals, but only tell apart marked from unmarked individuals. Let $m$ be the number of marked individuals in the population, $m_{(R)}$ the number of marked individuals seen on the resighting occasion, and $n_{(R)}$ the total number of marked and unmarked individuals observed during resighting. Abundance $N$ is then estimated as 
\[
N = m \times n_{(R)}/m_{(R)}
\]

A suite of more elaborate models using individual capture histories
over several resighting occasions were developed in the 1980ies and
90ies and compiled into the program NOREMARK \citep{white:1996}. Apart
from the basic model with known number of marked individuals and no
individual variation in resighting probabilities (joint hypergeometric
maximum likelihood estimator) \citep{bartmann_etal:1987,
  white_garrot:1990, neal:1990, neal_etal:1993}, NOREMARK contains
models that account for lack of geographical population closure
\citep{neal_etal:1993}, individual heterogenenity in resighting rates
and sampling with replacement (i.e. individuals can be seen more than
once on any occasion, \citep{minta_mangel:1989, bowden:1993}). A first
mark-resight model allowing for an unknown number of marked
individuals was developed by \citet{arnason_etal:1991}.

While many of these models perform well under certain situations, they
are somewhat limited: they do not allow for combining data across
several surveys \citep{mcclintock_etal:2006} and not all of them are
likelihood-based or allow for different parameterization, so that
selection of the most appropriate model cannot be based on standard
approaches such as AIC, but is largely left up to educated guesswork
\citep{mcclintock_etal:2006}. Recently, more flexible and generalized
likelihood-based mark-resight models have been developed. These models
can account for individual heterogeneity in detection, unknown number
of marked individuals and lack of geographical closure, as well as a
less than 100\% individual identification rate of tagged individuals;
they can be applied to sampling with and without replacement and can
combine data across several primary sampling occasions in a robust
design type of analysis
\citep{mcclintock_etal:2009biometrics,mcclintock_etal:2009mdp}. Since
they are all likelihood-based, model selection among different
parameterizations and model averaging based on AIC is an option. Most
of these models have also been incorporated into the program MARK
\citep{mcclintock_white:2010}.

For a detailed treatment of these different non-spatial mark-resight
models, we refer you to the original papers cited in the preceeding
paragraph. In short, these models are based on the joint likelihood of
two major model components: one describing the resighting process of
marked individuals (either using a Poisson or a Bernoulli observation
model, depending on whether sampling is with or without replacement),
where resighting probabilities can have both fixed effects to model
individual and environmental covariates, and a random-effect component
to accommodate variation in detection due to individual heterogeneity;
and one describing the total observations of unmarked individuals,
$n_t$ which are approximated as a normal distribution
\citep{mcclintock_etal:2006}, or a normal distribution left-truncated
at 0 \citep{mcclintock_etal:2009biometrics}:
\[
n_t \sim Normal (E(n_t), V(n_t))
\]
Although this is a simplification of the actual sampling process, \citet{mcclintock_etal:2006} found this Normal distribution to be a satisfactory approximation, which allows $N$ to enter the model likelihood via $E(n_t)$ and $V(n_t)$.

In the simplest model case without any variation in detection, the
expected number of resightings of unmarked individuals, $E(n_t)$, can
be written as the number of unmarked individuals times the expected number of detections of a single individual, which is the mean or expected value of the underlying observation model:
\begin{equation}
E(n_t) = (N-m) * \theta 
\end{equation}
\label{partialID.eq.E_n}
where $\theta = k \times p$ for a Binomial observation model with $k$
replicates and individual detection probability $p$, or $\theta$ =
expected/average individual encounter rate $\lambda$ for a Poisson
observation model. Similarly, $V(n_t)$ depends on the underlying
observation model and is based on the parameters
that determine the individual detection probability/encounter
rate. Combining these two components, $N$ is directly incorporated
into the joint likelihood of the model.

XXXXX Opinions: More details? Full model description? XXXXX
XXX I'm ok with this right now  -- andy XXXX

While these mark-resight models are very flexible, they
share the shortcomings of �regular� capture-recapture models
when it comes to estimating population density (e.g., Chapts. \ref{chapt.intro, chapt.closed}). 
In the following sections we will consider mark-resight sampling in the framework of spatial capture-recapture. We will look at models for both known and unknown numbers of marked individuals, and for imperfect individual identification of marks. In the spatial framework, most of the information on model parameters comes from the marked individuals. But in Sect. \ref{partialID.sec.info} we will see that, analogous to the models we developed in the previous Chapt. \ref{chapt.scr-unmarked}, the spatial correlation in counts of unmarked individuals also contributes information about detection and movement. 

\section{Known number of marked individuals}

Let's begin with the easiest data situation: a known number of
individuals constituting a random, representative sample from the
population are marked and a series of resight samples are conducted
following marking. No marks (or marked animals) are lost between
marking and resighting, all individuals are correctly identified as
marked or unmarked, and marked individuals are 100 \% correctly
identified to individual level.

Recall that without individual identity, the observed counts at trap
$j$ and occasion $k$, $n_{jk}$, represent the sum of all latent
individual detections at $j$ and $k$,
$\displaystyle\sum\limits_{i=1}^{N} y_{ijk}$, where $y_{ijk}$ are the
latent individual encounter histories. We can model these counts as
\[
n_{jk} \sim \mbox{Poisson}( \Lambda_{j} )
\]
Under this formulation we do not need to update the individual
$y_{ijk}$ in our model, which  is more efficient in terms of
computing. However, we can also formulate the model as conditional on
the latent $y_{ijk}$. This is useful because if we have $m$
individually known animals in our study population, than those $m$
$y_{ijk}$ are no longer latent, but fully observed and can easily be
included in the analysis. 

The formulation conditional on $y_{ijk}$ basically brings us back to the original SCR model, where individual site and occasion specific counts, $y_{ijk}$, are modeled as
\[
y_{ijk} \sim \mbox{Poisson}(\lambda _{ij})
\] 
and
\[
\lambda _{ij} = \lambda_0 * exp(-D_{ij}^2/(2 \sigma^2))
\]
XXXX You use $D_{ij}$ and maybe Richard does too. I haven't
been..... maybe we should standardize on $D_{ij}$? XXXXX


Unobserved $y_{ijk}$ are essentially missing data and have to be
updated as part of the MCMC procedure. We can do that by using their
full conditional distribution, which is multinomial with sample size
$n_{jk}$:
\[
y_{ujk} \sim Multinomial (n_{jk}, \lambda_{uj})
\]
where \textbf{\emph{u}} is an index vector of the $M-m$ hypothetical unmarked individuals.

  
While in the non-spatial mark-resight analysis known individuals
provide direct information about individual detection probability (or
rate), in the spatial setting they also inform the movement parameter
$\sigma$. Including known individuals into the analysis helps estimate
model parameters more accurately and precisely. We will address the
relationship between the number of marked individuals and accuracy of
the estimated parameters in section \ref{partialID.sec.info}.

\subsection{MCMC for a spatial mark-resight model}


Just as for the model without individual identification, for the
partial ID model, knowing how to write your own MCMC algorithm comes
in extremely handy. You will find that we only have to make relatively
simple modifications to the MCMC code for the model without any
individual identification presented in
Chapt. \ref{chapt.scr-unmarked}, which, in turn, has much in common
with the algorithms we developed for regular SCR models in
Chapt. \ref{chapt.mcmc}.
Essentially, since we observe individual detections for the marked part of the population, we have to update only the unobserved part of ${\bf Y}$, and
modify the updating steps for $z_i$ and $\psi$ to reflect some
contribution to our
knowledge of these parameters from the $m$ tagged individuals.

First, we set up an array to hold ${\bf Y}$, fill the first $m$ rows
of the array with the $m$ observed individual encounter histories,
then update ${\bf Y}$ for the unknown individuals only (note that the
code is set up so that $n_{jk}$ contains both pictures of marked {\bf
  and} unmarked individuals at $j$ and $k$):

{\small
\begin{verbatim}
# set up placeholders and create vectors for marked and unmarked    
 Y <- array(NA, c(M, J, K))
    nMarked <- nrow(y)
    marked <- rep(FALSE, M)
        marked[1:nMarked] <- TRUE
        Y[1:nMarked, , ] <- y
    z[marked] <- 1
    Ydata <- !is.na(Y)
    for (j in 1:J) {
        for (k in 1:K) {
            if (y[j, k] == 0) {
                Y[, j, k] <- 0
                next
            }
            unmarked <- !Ydata[, j, k]
            nUnknown <- n[j, k] - sum(Y[!unmarked, j,k])
            if (nUnknown < 0) 
                browser()
            probs <- lam[, j] * z
            probs <- probs[unmarked]
            probs <- probs/sum(probs)
            Y[unmarked, j, k] <- rmultinom(1, nUnknown, probs)
        }
    }
\end{verbatim}
}

XXX andy stopped here XXXX

When we know the number of marked individuals in the population estimating $N$ is reduced to etimating $u$. Thus, we only need to estimate the $z_i$ for $M-m$ unknown individuals. Thus, the updater for $z_i$ becomes:

\begin{verbatim}
zUps <- 0
seen <- apply(Y > 0, 1, any)
   for (i in 1:M) {
       if (seen[i] | marked[i]) 
                next
       zcand <- ifelse(z[i] == 0, 1, 0)
       ll <- sum(dpois(Y[i, , ], lam[i, ] * z[i], log = TRUE))
       llcand <- sum(dpois(Y[i, , ], lam[i, ] * zcand, 
                  log = TRUE))
       prior <- dbinom(z[i], 1, psi, log = TRUE)
       prior.cand <- dbinom(zcand, 1, psi, log = TRUE)
          if (runif(1) < exp((llcand + prior.cand) - (ll + 
                prior))) {
          z[i] <- zcand
          zUps <- zUps + 1
            }
        }
\end{verbatim}

Observe that while we skip the update of $z_i$ for the �seen� individuals, seen is defined based on ${\bf Y}$ and ${\bf Y}$ is updated at each iteration, so the $z_i$ for the �seen� but unmarked individuals are effectively still updated.

Finally, our update for $\psi$ needs to reflect that we are effectively only estimating $U$. In the full conditional beta distribution we have to replace $M$ with $M-m$ and $\sum z$ with $\sum z -m$:

\begin{verbatim}
  psi<-rbeta(1,1+sum(w[!marked]),1+sum(!marked)-sum(w[!marked]))   
\end{verbatim}

The remainder of the code is essentially identical to the MCMC code for regular SCR models we developed in Chapt. \ref{chapt.mcmc}.
You can find the full MCMC code (including the modeling options we'll discuss in the upcoming sections) in the accompanying {\bf R} package {\tt scrbook} by invoking {\tt scrPID()}. 

\subsection{Binomial encounter model}
So far, we have only worked with Poisson encounter models for partially identifiable or unmarked populations. When we use a Bernoulli model instead, we have to make some changes to how we update the latent $y_{ijk}$, to ensure that a hypothetical individual receives at most a single observation at a given trap and occasion from the pool of $n_{jk}$ pictures. Effectively, we move from a multinomial situation where the same individual could be drawn repeatedly, to a sampling without replacement situation (an individual drawn once at $j$ and $k$ cannot be drawn again); here is how we implement this in our MCMC algorithm:

\begin{verbatim}
 Y <- array(NA, c(M, J, K))
#[...]
    for (j in 1:J) {
        for (k in 1:K) {
            if (y[j, k] == 0) {
                Y[, j, k] <- 0
                next
            }
            unmarked <- !Ydata[, j, k]
            nUnknown <- n[j, k] - sum(Y[!unmarked, j,k])
            if (nUnknown < 0) 
                browser()
            probs <- lam[, j] * z
            probs <- probs[unmarked]
            probs <- probs/sum(probs)
            Y[unmarked, j, k] <- 0
            guys <- sample(which(unmarked), nUnknown, 
			prob = probs)
            Y[guys, j, k] <- 1
        }
    }
\end{verbatim}


{\flushleft \bf Example: Canada geese in North Carolina } 
We applied the spatial mark-resight model with a Bernoulli encounter process to a dataset of of Canada goose resightings \citep{rutledge:2012} XXXget citation with LizXXX. During the molt of 2008, 751 individual geese were captured and tagged with neck and leg bands in Greensboro, North Carolina. Geese were resighted at 87 different locations on 81 resighting events over a period of 18 months. In addition to the banded geese, the number of unmarked geese was recorded during each resighting event. Here, we only looked at a subset of the data, from mid July to the end of October 2008, which corresponds to the first part of the post-molt season, before migratory Canada geese arrive in North Carolina.  
During this time frame, 746 of the 751 marked geese were known to be alive. Of those, 654 were resighted 3994 times at 40 different sites. In addition, 7944 sightings of unmarked geese were recorded at 48 sites. 

In this model, we also allowed $\sigma$ to vary between males and females. We augmented the data set with 4500 - $m$ all-zero encounter histories, ran 50000 MCMC iterations and removed a burn-in of 1000 iterations. We provide all the data (data(canadageese)) and functions for you to repeat this analysis but be aware that given the large data set it will take days to do so. The model results, including the derived parameter density ($D$) in individuals per $km^2$ are shown below. 
XXXX HAVE TO ASK FOR PERMISSION TO INCLUDE DATA - IN THE PROCESS XXXX

\begin{verbatim}
Iterations = 1001:50000
Thinning interval = 1 
Number of chains = 1 
Sample size per chain = 49000 

1. Empirical mean and standard deviation for each variable,
   plus standard error of the mean:

            Mean        SD  Naive SE Time-series SE
sigmaF    1.0594 1.902e-02 8.593e-05      0.0011674
sigmaM    1.1347 2.375e-02 1.073e-04      0.0014294
lam0      0.3245 8.037e-03 3.631e-05      0.0003103
psi       0.7924 3.284e-02 1.483e-04      0.0017778
phi       0.4337 1.857e-02 8.387e-05      0.0003754
N      3720.8128 1.210e+02 5.466e-01      6.6264003
D         6.6832 2.173e-01 9.817e-04      0.0119021

2. Quantiles for each variable:

            2.5%       25%       50%       75%     97.5%
sigmaF    1.0218    1.0470    1.0594    1.0717    1.0971
sigmaM    1.0909    1.1182    1.1341    1.1502    1.1834
lam0      0.3088    0.3191    0.3244    0.3298    0.3407
psi       0.7298    0.7698    0.7917    0.8143    0.8573
phi       0.3976    0.4210    0.4336    0.4462    0.4702
N      3492.0000 3637.0000 3717.0000 3802.0000 3961.0000
D         6.2722    6.5326    6.6763    6.8290    7.1146
\end{verbatim}

We see that credible intervals of estimates are pretty tight. Take, for example, $\sigma$ for males and females: Although they differ only by 0.08, there is barely any overlap between the respective credible intervals, surely an effect of the large data set. Phi in this model is the probability of being a male, or the sex ratio of the populations in terms of males:females, which is close to 1:1.      


\section {Unknown number of marked individuals}

Now let us consider the case where we do not know the exact number of tagged individuals available for resighting so that we have to capture an individual at least once to be sure that it is available. Unless we have a direct means of confirming the number of marked animals available for resighting, treating this number as unmarked is probably more realistic in most circumstances. As a consequence of not knowing the exact number of marked individuals, we cannot observe all-zero encounter histories. When using maximum likelihood inference, this situation requires a model where detection rates of known individuals are modeled using a zero-truncated distribution \citep{mcclintock_etal:2009biometrics}. If we did not account for the fact that 0�s are unobservable, our estimates of detection rates would be artificially inflated and estimates of population size would be negatively biased. 

Working with zero-truncated distributions in a spatial mark-resight setting is less straight-forward than for non-spatial mark-resight. A marked individual only has to show up once, anywhere on the sampling grid, for us to know that it is there. When resightings are pooled across the entire sampling grid,then the total individual counts $\sum_j y_{ij}$ have to be $>$ 0 for all resighted individuals and a zero-truncated distribution can be used to model these counts. However, we are concerned with trap-specific encounters, $y_{ij}$, which can easily be 0 for a resighted individual, as long as a single $y_{ij}$ is $>$ 0. Thus, the zero-truncation does not apply to the individual and trap specific counts we observe, but only to the sum of these counts over all traps. 

As an alternative to a zero-truncated distribution, in a Bayesian framework, we can make use of data augmentation to estimate the number of marked individuals\footnote{For the interested reader, \citet{mcclintock_hoeting:2010} implement a non-spatial mark-resight model with a binomial observation model in a Bayesian framework using data augmentation}. 
In the previous example, where we knew the number of marked individuals, we essentially removed those individuals from the augmented population by fixing their $z_i$ at 1 and letting $\psi$ refer only to the unmarked population, $M-m$. All we have to do in the spatial mark-resight model with unknown number of marked individuals is to let our marked individuals be part of the augmented population again, analogous to the situation in regular SCR models:
\begin{verbatim}
        psi <- rbeta(1, 1 + sum(z), 1 + M - sum(z))
\end{verbatim}
Whether you have a known or an unknown number of marked individuals is included as an option in {\tt scrPID}.
 
XXXX Other example data set? XXXX

\section  {Individual identification rate of tagged animals $<$ 100 \%}
Often during resighting, it may be possible to see that an individual is tagged but impossible to determine the individual identity of the tag. In such a situation in addition to the $y_{ijk}$ and $n_{jk}$, we also have site and occasion specific counts of marked but unidentified individuals, $r_{jk}$. Here, the individual encounter histories of marked animals are essentially incomplete, and if we used these incomplete data to inform the detection parameter of the model, we would be likely to underestimate detection/trap encounter rate and overestimate abundance. Some non-spatial mark-resigh models do not require that marked animals be identified individually, as long as the marking status can be observed unambiguously, but ignoring individual level information means that we cannot accomodate heterogeneity in detection \citet{mcclintock_white:2010}. In a spatial framework we could ignore marked and unmarked status completely and apply the model by \citet{chandler_royle:2012} we discussed in Chapt. \ref{chapt.scr-unmarked}. But again, that would mean losing important information on individual detection and movement. Therefore, being able to retain the individual identity of records that can be identified while at the same time accounting for an identification rate $<$ 100 \% is extremely useful. 
\citet{mcclintock_etal:2009biometrics,mcclintock_etal:2009mdp} suggest an intuitive means of correcting for this bias in a non-spatial model framework when dealing with a Poisson encounter model (or sampling with replacement). When marked but unknown resightings are part of the data, the expected number of records of unmarked individuals at time $t$, $n_t$, changes from Eq. \ref{partialID.eq.E_n} to:

\[
E(n) = (N-m) { \lambda  + \eta/m}
\]
Here, $\lambda$ is the individual encounter rate estimated from the known resighted individuals and $\eta$ is the number of records of marked but unidentified individuals. So essentially, because observed $\lambda$ is known to be too low, the average number of unidentified pictures per known individual is added as a correction factor. This procedure assumes that the inability to identify a marked individual occurs at random throughout the population, which seems to be a reasonable assumption under most circumstances.

We can relatively easily translate this concept to our spatial mark-resight models. In the spatial model framework we are interested in the individual and trap specific encounter rate, $\lambda_{ij}$. Further, we do not look at the sum of all records of unmarked individuals, but formulate the model conditional on the latent individual encounter histories. Thus, instead of using $\eta/m$ as a correction factor, we need something that applies at the individual and trap level. If we take the sum of all correctly identified records of marked individuals, $\sum y_c$ and divide it by the total number of records of marked individuals, $\sum y_m$, we get the average rate of correct individual identification for marked individuals, say, $c$:
\[
c = \sum y_c/\sum y_m
\]

For the marked individuals we can then multiply $\lambda_0$ with $c$ to account for the fact that we observe incomplete individual encounter histories. For example, if on average we are able to assign 80\% of the records of marked animals to the correct individual, we would multiply $\lambda_0 \times 0.8$ for the marked individuals. Since we don't have this identification issue for unmarked individuals, their baseline trap encounter rate remains as before simply $\lambda_0$ (or in other words, their $c$ equals 1). Observe that now, in addition to assuming that failure to identify tagged individuals occurs at random throughout the population, we also assume that it occurs at random throughout space, i.e. our success of identifying a tagged individual does not depend on the trap we encounter it in. 
It is straightforward to  include this correction factor in our MCMC algorithm, by simply specifying a vector of length $M$, where $c = 1$ for all unmarked (hypothetical) individuals and $c = \sum y_c/\sum y_m$ for all marked individuals. Incomplete individual identification of marked individuals is included as an option in the {\tt scrPID} function.

XXXX maybe we can include the ISSJ as an example, but estimates of N will be huuuge XXX

As long as individuals are identified based on the same type of tags the assumption that failure to identify marked individuals occurs at random throughout the population should be valid. The assumption that failure to identify marked individuals occurs at random in space could be violated, for example when spatially varying habitat conditions influence the ability to recognize individual tags. Also, observer effects could influence individual identification rates. While we haven't ourselves experimented with it, we belive that the above described approach could readily be extended to account for these differences. For example, identification rates could be calculated separately for different observers, or be modeled as functions of habitat covariates. 


\section{How much information do marked and unmarked individuals contribute?}
\label{partialID.sec.info}
It is intuitive that having marked individuals in the study population should lead to more accurate and precise parameter estimates than when no individuals are identifiable. To evaluate how strongly adding marked individuals to a population improves parameter esimtates, \citet{chandler_royle:2012} performed a simulation study. They used a 15 � 15 trap grid and
simulated detection data of $N = 75$ individuals in a 20 x 20 units state-space over $k = 5$ occasions with
$\sigma = 0.5$ and $\lambda_0 = 0.5$. They generated 100 datasets each for
$m$ = (0, 5, 15, 25, 35) where $m$ is the number of marked individuals randomly sampled from the population and is assumed to be known.

Without any marked individuals in the population, the posterior distribution of $N$ turned out to be fairly skewed, but its mode was still an approximately unbiased point estimator of $N$. As anticipated, posterior precision increased substantially with the proportion of marked individuals (Tab. \ref{partialID.tab.sim} and Fig. \ref{partialID.fig.nposts}). The posterior mode was approximately unbiased as a point estimator, and the relative root-mean squared error decreased from 0.246 when no individuals were marked to 0.085 when 35 individuals were marked (Tab. \ref{partialID.tab.sim}). Coverage was nominal for all values of $m$ and posterior skew greatly diminished with increasing $m$(Tab. \ref{partialID.tab.sim}).

\begin{figure}%[ht]
  \centering
  \includegraphics[width=4in,height=4in]{Ch15/figs/Nposts2.png}
  \caption{Overlaid posterior distributions of $N$ from 100 simulations
    for four levels of marked individuals.}
  \label{partialID.fig.nposts}
\end{figure}

\begin{table}%[hb]
\caption{Posterior mean, mode, and associated relative RMSE for simulations in
  which $m$ of $N$=75 individuals were marked. One hundred simulations of each case were conducted. }
\begin{tabular}{llrrrrr}
     &	Parameter    &	Mean   &	rRMSE  & Mode   & rRMSE &	BCI    \\
     \hline
 m=0 &	$N$          &	85.866 &    0.259 & 77.720 &    0.242 & 0.950  \\
     &	$\lambda_0$  &	0.506  &	0.180 &	0.488  &	0.182 &	0.960  \\
     &	$\sigma$     &	0.495  &	0.115 &	0.486  &	0.113 &	0.960  \\
     \hline
 m=5 &	$N$          &	80.898 &    0.184 & 76.360 &    0.182 & 0.970  \\
     &	$\lambda_0$  &	0.510  &    0.178 & 0.494  &    0.180 & 0.950  \\
     &	$\sigma$     &	0.496  &    0.089 & 0.488  &    0.086 & 0.970  \\
     \hline
 m=15&	$N$          &	79.028 &    0.148 & 76.250 &    0.147 & 0.950  \\
     &	$\lambda_0$  &	0.508  &    0.163 & 0.494  &    0.164 & 0.950  \\
     &	$\sigma$     &	0.496  &    0.073 & 0.492  &    0.071 & 0.970  \\
     \hline
 m=25&	$N$          &	77.765 &    0.114 & 75.810 &    0.113 & 0.950  \\
     &	$\lambda_0$  &	0.511  &    0.153 & 0.498  &    0.157 & 0.950  \\
     &	$\sigma$     &	0.496  &    0.067 & 0.493  &    0.065 & 0.940  \\
     \hline
 m=35&	$N$          &	76.446 &    0.085 & 74.900 &    0.085 & 1.000  \\
     &	$\lambda_0$  &	0.513  &    0.142 & 0.501  &    0.144 & 0.950  \\
     &	$\sigma$     &	0.497  &    0.056 & 0.493  &    0.057 & 0.940  \\
 \hline
\end{tabular}
\label{partialID.tab.sim}
\end{table}

As we saw in the previous chapter, the spatial correlation in unmarked counts can be sufficient to obtain estimates of movement and detection parameters. However, only marked and thus identifiable individuals provide us with direct information about these parameters and may well dominate estimates. 
To single out the contribution of marked and unmarked individuals to parameter estimates, we re-ran the same simulations but let $\sigma$ and $\lambda_0$ be updated based solely on the data of marked individuals. Results are summarized in Tab. \ref{partialID.tab.sim2}.
We see that if we update $\lambda_0$ and $\sigma$ based on marked individuals only, estimates of $\lambda_0$ and $\sigma$ are more biased and less precise. For estimates of $N$, especially for $m$= 5 and $m$ = 15, we observe a stronger positive bias, lower accuracy and considerably lower BCI coverage as compared to when both marked and unmarked individuals contribute to parameter estimates (Tab. \ref{partialID.tab.sim2}). Thus, unmarked individuals do actually contribute noticeably to estimating model parameters. 

\begin{table}%[hb]
\caption{Posterior mean, mode, and associated relative RMSE for simulations in
  which $m$ of $N$=75 individuals were marked and unmarked individuals 
  did not contribute to estimating $\lambda_0$ and $\sigma$. 
  One hundred simulations of each case were conducted. }
\begin{tabular}{llrrrrr}
     &	Parameter    &	Mean   &	RMSE  &	Mode   &	RMSE &	BCI    \\
     \hline
 m=5 &	$N$          &	88.621 &	0.369 &	83.139 &	0.421 &	0.810  \\
     &	$\lambda_0$  &	1.255  &	1.247 &	0.606  &	1.148 &	0.950  \\
     &	$\sigma$     &	0.472  &	0.252 &	0.426  &	0.333 &	0.910  \\
     \hline
 m=15&	$N$          &	81.031 &	0.192 &	78.361 &	0.175 &	0.820  \\
     &	$\lambda_0$  &	0.535  &	0.281 &	0.476  &	0.284 &	0.970  \\
     &	$\sigma$     &	0.503  &	0.109 &	0.490  &	0.107 &	0.940  \\
     \hline
 m=25&	$N$          &	78.206 &	0.129 &	76.594 &	0.123 &	0.920  \\
     &	$\lambda_0$  &	0.531  &	0.204 &	0.496  &	0.202 &	0.960  \\
     &	$\sigma$     &	0.497  &	0.081 &	0.489  &	0.084 &	0.950  \\
     \hline
 m=35&	$N$          &	76.833 &	0.099 &	75.422 &	0.096 &	0.940  \\
     &	$\lambda_0$  &	0.528  &	0.192 &	0.505  &	0.186 &	0.940  \\
     &	$\sigma$     &	0.499  &	0.069 &	0.493  &	0.070 &	0.960  \\
 \hline
\end{tabular}
\label{partialID.tab.sim2}
\end{table}


\section{Integrating telemetry data}
As we expected, parameter estimates of spatial mark-resight models get better the more marked individuals we have in our study population. While this is great advice in theory, it may not be very helpful in practise, especially when dealing with animals that are hard or somewhat dangerous to capture, such as large carnivores. Oftentimes, studies involving the physical capture of such animals will employ telemetry tags in order to learn about the study species' spatial ecology and behavior. In the context of spatial mark-resight models, the actual locational data collected by telemetry tags can provide detailed information on individual location and movement, and being able to incorporate this information directly into the SMR model should improve estimates of these parameters, especially when resighting information is sparse.

So how could we combine resighting data and telemetry data in a unified mark-resight model? Recall that the basic SCR model underlying all the SMR models we discuss here uses a half-normal detection function.
By using this function, we can relate the parameters $\sigma$ and ${\bf s}_{i}$ directly to those from a bivariate normal movement model, with mean = ${\bf s}_{i}$, and variance-covariance matrix $\Sigma$, where the variance in both dimensions is $\sigma^2$ and the covariance is 0. Ordinarily, these parameters are estimated directly from the spatial distribution of individual recaptures/resightings. Telemetry data, however, provide more detailed information on individual location and movement, since the resolution and extent of the data are not limited by the trapping grid and potentially more locations can be accumulated through telemetry than resighting (depending on the monitoring frequency and resighting rates of individuals).  

By assuming that the locations of individual $i$, ${\bf l}_{i}$ (consisting of a pari of x and y coordinates, $l_{ix}$ and $l_{iy}$), are a bivariate normal random variable:
\[
{\bf l}_i\sim Normal_2 ({\bf s}_i,\Sigma)
\]
we can estimate $\sigma$ as well as ${\bf s}_{i}$ for the collared individuals directly from telemetry locations, using their full conditional distributions:
\[
[\sigma|{\bf l},{\bf s}] \propto \left\{\prod_{i=1}^m \frac{1}{2 \pi \sigma^2} exp\left(-1/2 \left[ \frac {l_{ix}-s_{ix}} {\sigma^2} + \frac{l_{iy}-s_{iy}}{\sigma^2} \right]\right)\right\}*[\sigma] 
\] 
and
\[
[{\bf s}_{i}|{\bf l}_{i},\sigma] \propto \left\{\frac{1}{2 \pi \sigma^2} exp\left(-1/2 \left[ \frac {l_{ix}-s_{ix}} {\sigma^2} + \frac{l_{iy}-s_{iy}}{\sigma^2} \right]\right)\right\}*[{\bf s}_{i}] 
\]
XXXXX ANYONE - I am pretty sure that's what the BVN pdf reduces to when covariance = 0, but would someone mind cross-checking? XXXXXX 
Under the standard mark-resight assumption that marked individuals are a representative sample of the population, the estimate of $\sigma$ can be applied for the entire population. For the unmarked individuals ${\bf s}_{i}$ are estimated as described before conditional on their latent encounter histories. 

{\bf R} makes it easy to implement the update of $\sigma$ and ${\bf s}_i$ based on telemetry data and the above described full conditionals within our existing MCMC algorithm. We simply replace the current updating step for $\sigma$ with: 

\begin{verbatim}
#ntot = number of telemetry-tagged individuals
#locs = list of length ntot; each element is a matrix 
#with telemetry locations
#telID = vector with identifier for telemetry-tagged
#individuals

sigma.cand <- rnorm(1, sigma, tune[1])
if (sigma.cand > 0) {

llsig<-llsig.cand<-rep(NA, ntot) 

for (x in 1:ntot) {
lls[x]<-sum(dmvnorm(x=locs[[x]],mean=c(S[telID[x],1],S[telID[x],2]), 
			sigma=cbind(c(sigma^2,0), c(0,sigma^2)), log=T))   
lls.cand[x]<-sum(dmvnorm(x=locs[[x]],mean=c(S[telID[x],1],S[telID[x],2]), 
	sigma=cbind(c(sigma.cand^2,0), c(0,sigma.cand^2)), log=T))   
	}
   if(runif(1) < exp( sum(lls.cand)  - sum(lls) ) ){
    sigma<-sigma.cand
    lam <- lam0*exp(-(D*D)/(2*sigma.cand*sigma.cand))
					}
			}
\end{verbatim} 
For ${\bf s}$ we use an analogous updater for the telemetry-tagged individuals and the regular updater for individuals without associated telemetry location information. A full example code can be found in the {\bf R} package {\tt scrbook}, by calling {\tt scrPID.telemetry}. Note that not all marked individuals need to be telemetry-tagged. This approach of incorporating telemetry data into a spatial mark-resight model can easily be extended to update $\sigma$ and ${\bf s}$ conditional on both resighting and telemetry data and applies equally to regular SCR models where all individuals are identifiable. 

{\flushleft \bf Example: Raccoons on the Outer Banks of North Carolina } 
XXXX Will have to ask for permission to include this; alternatively use the panthers; both examples are written up XXXXXXX

\section{Summary}
In this chapter we extended the spatial model for unmarked populations to a mark-resight situation, where part of the population is individually identifiable, usually through artificial tags. The basic model with known number of marked individuals and 100 \% individual identification of marked is easily modified for situations where the number of marked individuals is unknown, or where marked animals can sometimes not be identified to individual level. As expected, having marked individuals in the study population improved accuracy and precision of parameter estimates when compared to fully unmarked populations, but we also saw that the spatial counts of unmarked individuals still contribute information to parameter estimates. Finally, we present an approach of how to incorporate telemetry location data into the spatial mark-resight model to inform estimates of $\sigma$ and activity centers. Especially for difficult-to-study, cryptic species where often only a small sample of the population can be tagged this enables researchers to make optimal use of all existing data and obtain robust density estimates without the need for additional invasive methods. 
Beyond the models presented in this chapter, just as SCR, the spatial mark-resight model framework is flexible to account for a variety of factors that may influence individual movement and detection, as well as survey-related parameters. As such, the approach is applicable to a wide range of population estimation problems when dealing with animals that cannot be identified based on natural marks. 











 





\chapter{Spatial Capture-Recapture with Distance Sampling Data}
\markboth{Chapter 17}{}
\label{chapt.scrds}

\vspace{0.3cm}



In SCR models, the locations of animal activity
centers are unknown and must be inferred from the trap locations where
individuals are captured. Intuitively, the more
we know about the locations of activity centers, the more precise will
be our density estimates, and thus we strive to increase the
number of spatial recaptures. That is, we want to catch each
individual at multiple points in space so that we can pinpoint
its activity center. However, obtaining a large
number of spatial recaptures can be difficult due to the associated costs
of traps and the labor required to set and check them. This is true
even of ``cheap'' methods like camera traps which can easily run $>$
200\$ a pop.

Distance-samplers, of course, know that a much easier way to record an
animal's location in space is to go traverse a transect or stand at some
point and directly record the coordinates of the animal at some
instant in time\footnote{Generally only the distance between the
  observer and the animal is used in the analysis, but the exact
  coordinates are often recorded.}. \hl{Hmm, this make you wonder
  about telemetry data too}. This is cheap and easy data, which
partially explains the popularity of distance sampling.


Given the ease
with which one can record distance data, it is natural to wonder how
it could be included in a SCR analysis. Before doing so, a better
question is why bother with SCR when a distance sampling analysis would
be straight-forward. Good point. In some cases, there probably is no
need to use SCR if the sole quantity of interest is density, and the
assumptions of distance sampling can be met. However, SCR let's us
study more than just density. Think of space use. Think up a ship,
think up a long trip, think up the Vipper, the Vipper of Biff. Oh the
thinks you can think up if only you try (Seuss 1960s).

Some regard movement as a nuisance that is
best left untouched \citep{borchers:2010}. This is
convenient, but movement is actually a part of the problem, not to
mention a central focus of an enormous
branch of ecology.

Use of data on animal locations at some distance from the observer
forces us to consider movement a little more explictily than we had
previously. Currently, only two papers that we know of have attempted
to estimate explicity movement parameters
\citep{royle_young:2008,royle_etal:2009jae}. The underlying approach
is simple. First, define $\bf u$ to be an individual's location in
space at some instant in time. We now need a movement model that links
the activity center $\bf s$ to $\bf u$ and finally a detection model that
is a function of $\bf \| u - x\|$ instead of $\bf \|s - x\|$,
\emph{i.e.}, a function of distance between the observer
at point $\bf x$ and the animal at point $\bf u$---just as in distance
sampling. A natural movement model is the bivariate normal, but we
will consider other options in this chapter. In addition, we will
consider pragmatic issues such as what to do if not all individuals
are marked.







\section{Everybody is marked}



\begin{verbatim}

model {
sigHome ~ dunif(0, 5)
sigObs ~ dunif(0, 5)
tauHome <- 1/pow(sigHome,2)
tauObs <- 1/pow(sigObs,2)
psi ~ dunif(0, 1)
for(i in 1:M) {
  w[i] ~ dbern(psi)
  sx[i] ~ dunif(0, 15)
  sy[i] ~ dunif(0, 15)
  for(r in 1:R) {
    ux[i,r] ~ dnorm(sx[i], tauHome)
    uu[i,r] ~ dnorm(sy[i], tauHome)
    d2[i,r] <- pow(X[r,1]-ux[i,r], 2) + pow(X[r,2]-uy[i,r], 2)
    p[i,r] <- exp(-d2[i,r]/(2*sigObs*sigObs)) * w[i]
    y[i,r] ~ dbern(p[i,r])
    }
  }
N <- sum(w[])
}

\end{verbatim}







\section{Nobody is marked}





\section{Partially-marked populations}





\section{An Implicit SCRDS Model Without Distance Data}

The idea here is to convolve two Gaussian kernels, one for the animal
(movement) and one for the observer (detection | distance). This is
just another two-parameter observation model I guess.




\bibliography{AndyRefs_alphabetized}


\markboth{Index}{Index}

%\printindex

\documentclass{book}

\usepackage{elsst-book}
\usepackage{float}
\usepackage{amsmath}
\usepackage{amsfonts}
\usepackage{graphicx}
\usepackage{lineno}
\usepackage{natbib}
\usepackage{hyperref}
\usepackage{verbatim}
\usepackage{soul}
\usepackage{color}

\bibliographystyle{asa}

\usepackage{makeidx,bm,amsmath,url}
\makeindex

\floatstyle{plain}
\floatname{panel}{Panel}
\newfloat{algorithm}{h}{txt}[chapter]
\newfloat{panel}{h}{txt}[chapter]


\newcommand{\R}{\textbf{R}}
\newcommand{\bugs}{\textbf{BUGS}}
\newcommand{\jags}{\textbf{JAGS}}
\newcommand{\secr}{\mbox{\tt secr}}
\newcommand{\scrbook}{\mbox{\tt scrbook}}


\linenumbers

\begin{document}

\title{ Spatial Capture-Recapture  }
\subtitle{
%Hierarchical modeling of capture-recapture data with auxiliary spatial information
}
\author{The Four Horsemen (and women) }

\affiliation{First Author Short Address\\ Second Author Short Address}
\address{
USGS Patuxent Wildlife Research Center \\
North Carolina State University
}

\maketitle

\newpage

\setcounter{tocdepth}{2}
\tableofcontents

%\chapter{Introduction}
%\label{chapt.intro}

\input{Ch1/Ch1.tex}

\input{Ch2/Ch2.tex}

\input{Ch3/Ch3.tex}

\input{Ch4/Chapter4.tex}

\chapter{Other observation models}
\label{chapt.poisson}

%\input{Ch5/Ch5.tex}

\input{Ch6/Ch6.tex}

\input{Ch7/Ch7.tex}

\chapter{Goodness of Fit and stuff}
\label{chapt.gof}

\chapter{Modeling Encounter Probability}
\label{chapt.covariates}

\input{Ch10/Ch10.tex}
%\chapter{Ecological Distance Models in Spatial Capture-Recapture}
%\label{chapt.ecoldist}

\input{Ch11/Ch11.tex}

%\chapter{Inhomogeneous Point Process}
%\label{chapt.ipp}

\chapter{Open models}
\label{chapt.open}

\input{Ch14/Ch14.tex}

\input{Ch15/Ch15.tex}


\input{Ch17/Ch17.tex}



\bibliography{AndyRefs_alphabetized}


\markboth{Index}{Index}

%\printindex

\input{previewBook.ind}

\end{document}

\end{document}

\end{document}

\end{document}